Die Bilder, die der Flughafenleiter sah, erschütterten ihn in allem, was er geglaubt hatte. Nicht nur, weil sie an sich schon schrecklich genug waren, sondern weil er sein Leben lang geglaubt hatte, die Menschen hätten sich weiterentwickelt. Wären rücksichtsvoller und verantwortungsbewusster geworden und hätten sich gegenseitig zu respektieren gelernt, so wie sie waren. Und als er gestern Abend vom Balkon seiner Wohnung auf die Straße hinab gesehen hatte, waren ihm seine Mitmenschen auch genau so erschienen.

\par

Doch nun stand er am Seiteneingang des kleinen Nebenhangars und sah, wie Menschen sich gegenseitig umrannten, ihre Habe zu retten versuchten und einfach über die Verletzten auf dem Boden hinweg stiegen, ohne auch nur auf die Idee zu kommen, ihnen zu helfen. Ihre Angst um ihr Leben trieb sie an. Voller Panik rannten sie die Straßen hinunter und drängten auf die wartende Menge ein, als wäre sie eine Herde Vieh, die man vorwärts treiben musste. Das war nicht die Union, die der Chef des Flughafens seit seinen Kindertagen kannte.

\par

Einige der flüchtenden Menschen erkannten, dass die Nebentür zum Hangar geöffnet worden war. Panisch rannten sie auf den Zaun zu, der das Fluggelände von der Straße trennte. Einige schlugen nur wütend gegen den Maschendraht, während andere versuchten, über die Absperrung zu klettern. Gegen Einbrecher war das Gelände gesichert, deswegen ragte eine Spirale Stacheldraht über den Zaun. Einige Ordnungshüter der hiesigen Polizei kamen herbeigerannt, um die Menschen von der Absperrung fern zu halten.

\par

\WR{Herr Kumalo, hier ist Bauer}, quäkte es aus dem Buch des Flughafenchefs. \WR{Die ersten Transporter sind beladen und bereit zum Start.}

\par

Robért Kumalo seufzte und sah in den Himmel. Über der Stadt kreisten die Angreifer wie Geier. Immer wieder ließen sie Salven auf die Häuser und Straßen niederprasseln. Gerade ging das Rathaus, der einzige fünfstöckige Bau in ganz Graustadt in Flammen auf, als er von den giftgrünen Strahlen einer der Jäger getroffen wurde.

\par

Der Chef des Flughafens nahm sein Buch aus der Innentasche seine Gehrocks und drückte die virtuelle Taste für eine Antwort. \WR{Bauer, sind diese Transporter auch so voll wie sie nur sein können?}

\par

\WR{Auf jeden Fall}, entgegnete ihm der Chefmechaniker des Flughafens. \WR{Wir haben sogar Menschen in den Frachträumen untergebracht.}

\par

Kumalo zögerte einen Moment, als sein Blick auf einen der Angreifer fiel. \WR{Dann los!}

\par

Die Antwort des Mechanikers bekam er nicht mehr mit, denn der Flughafenleiter hatte jemanden auf der Straße erkannt, der ihm ein wenig Hoffnung gab. Nico Curiosa saß hinter dem Steuer seines Hovercraft-Cabriolets und versuchte vergeblich, einen Weg durch die flüchtenden Menschenmassen zu finden. Schließlich gab er es auf und parkte sein Auto achtlos am nächstbesten Straßengraben.

\par

Curiosa war Robért Kumalo schon seit langem bekannt. Er wusste von seiner Vergangenheit, aber auch davon, dass er ein ausgezeichneter Pilot war.
\ortswechsel
\WR{Öffnen Sie die Absperrung}, rief Kumalo einem der Polizisten zu, während Nico Curiosa und Ed Steiner gerade einer Frau und ein paar Kindern aus dem Auto halfen. Der Ordnungshüter sah ihn betreten an, bis der Flughafenleiter seine Anordnung wiederholte. Langsam gab der Polizist eine Zahlenkombination in den Öffnungsmechanismus der Gittertür ein und lies Kumalo hinaus. Die Ordnungshüter mussten einige Flüchtende abwehren, die ihre Chance sahen, ein wenig früher ins Flughafengelände zu kommen. Doch sie wurden von der Polizei zum Teil mit Hilfe von Schlagstöcken zurückgedrängt.

\par

Unterdessen trat Kumalo an Curiosa heran. Dieser schien ihn zuerst nicht zu bemerken. Erst als der Chef des Flughafens rief \WR{Nico, Sie schickt der Himmel}, reagierte dieser. Kumalo fuhr fort: \WR{Nico, uns fehlen Piloten. Wir haben nicht genügend Leute, um alle Transporter zugleich zu bemannen. Können Sie so ein Ding steuern?}

\par

Ein Stein von der Größe des Ayers Rock fiel von Nico Curiosas Herzen. Er und Ed sahen sich gegenseitig an und beide lächelten auf dieselbe, erleichterte Weise. \WR{Natürlich}, antwortete der ehemalige Pilot. \WR{Ich habe Erfahrung in \EN{W zweidutzendacht}. Die Starforce hat die früher auch benutzt. Ed kann sie auch fliegen.}

\par

\WR{Kommen Sie mit!}, forderte der Leiter, ohne noch eine Sekunde zu zögern.

\par

Gemeinsam mit seiner Familie und Ed durchquerte Nico Curiosa die kleine Gittertür zum Flughafen. Einige der Wartenden riefen ihnen die schlimmsten Beleidigungen hinterher. Einer von ihnen schaffte es, an den Polizisten vorbei zu kommen und packte Ed Steiner an der Schulter, bevor dieser es durch die Tür schaffte.

\par

\WR{Warum kommst du als erstes hier raus!}, spie der Mann ihm wütend entgegen. \WR{Ich hab genauso ein Recht gerettet zu werden wie du.}

\par

Ed wurde es dabei übel aber irgendwie überwand er sich dazu, den Mann fortzustoßen und durch die Gittertür zu rennen.

\par

In diesem Moment hoben die ersten beiden Transporter der \EN{Mammut}-Klasse ab. Ihr Start verlief nur langsam. Die beiden massigen Schiffe waren so lang wie ein Fußballfeld und eigentlich für den Transport von Waren gebaut worden. Die großen Frachtlager unter dem Passagierbereich machten \EN{Mammuts} so langsam und behäbig.

\par

Gerade als die beiden Schiffe die Starthöhe in Starthöhe erreicht hatten, in denen sie den Hauptantrieb hätten zünden können, trafen einige Strahlensalven der umher kreisenden Jäger seine Flanke und rissen den Rumpf des Schiffes auf. Während das eine seinen Plasmaantrieb hochfuhr und sich unter lautem Gedröhn schnell entfernte, stürzte das andere Schiff brennend und rauchend dem Boden entgegen.

\par

Nico schloss die Augen, um nicht mit ansehen zu müssen, wie das \EN{Mammut} ins Stadtzentrum stürzte und dieses mit seinem geladenen Capezin in einen riesigen Hochofen verwandelte. Die Menschen schrien umso panischer, als sie erkannten, wie das Feuer hungrig dem Himmel entgegen schoss. Von der Innenstadt waren innerhalb von Sekundenbruchteilen nur noch Trümmer und das Wrack des abgestürzten Raumschiffs übrig. Eine riesige schwarze Rauchsäule schraubte sich in die Höhe und der Gestank von Rauch und verbranntem Asphalt schwelte in die Richtung des Flughafens herüber.

\par

In diesem Moment war Nico sehr dankbar dafür, dass ihn der Flughafenleiter gefunden hatte. Nicht allein, weil dies die Chancen seiner Familie zu überleben erhöhte, sondern weil er es selbst in der Hand haben würde. Sein Platz wäre nicht in den Passagierriegen, von denen aus er nichts tun konnte, um seine Frau und seine Kinder zu beschützen. Er würde im Cockpit sitzen und konnte selbst dafür sorgen, dass sie alle überleben würden.

\par

Erst jetzt fiel ihm auf, wie aufgelöst seine beiden Kinder waren. Seine Tochter, die ältere der beiden, sah völlig regungslos in die Ferne. Sie hatte geweint, denn Spuren von Tränen waren auf ihren Gesicht geblieben. Doch nun war sie so ruhig wie verzweifelt. Sein Sohn, der vor wenigen Wochen seinen fünften Geburtstag gehabt hatte, schrie. Zornig versuchte er, sich von der Hand seiner Mutter loszureißen, die ihn fest umklammerte.

\par

Im Nebenhangar angelangt, sahen sich Nico und Ed drei kleineren Transportern entgegen. Sie erinnerten ein wenig an Passagierflugzeuge aus einem längst vergangenen Jahrhundert. Doch der Rumpf war massiger und breiter und bot nicht nur einigen hundert Passagieren Platz, sondern auch einem Maschinenraum. Das Cockpit war am hinteren Ende des Rumpfes zu finden, direkt über den kurzen Tragflächen. Dahinter folgten zwei voluminöse Plasmatriebwerke.

\par

\WR{Das sind unsere \EN{W zweidutzendacht}}, begann Kumalo. \WR{Suchen Sie sich eine aus. Wir lassen jetzt die Leute an Bord.}

\par

\WR{Nimm du die ganz links und ich den Vogel in der Mitte}, sagte Ed Steiner und versuchte dabei so enthusiastisch zu klingen, wie in den Zeiten, in denen er und Nico gemeinsam Einsätze geflogen waren.

\par

\WR{Genau wie damals}, entgegnete dieser. Beide schlugen ein und wünschten sich viel Glück, bevor sie ihre Maschinen bestiegen.

\par

Da Nico klar war, dass er wohl keine Begleitmannschaft bekommen würde, ließ er seine Familie zu sich ins Cockpit sitzen. Er warf seiner Frau einen Blick zu, von dem er hoffte, dass er ihr Zuversicht schenken würde. Doch Vivien klammerte sich nur angsterfüllt in ihren Sitz und streichelte ihrer Tochter über den Kopf, die kaum noch ansprechbar war.

\par

Die Tore des Hangars wurden geöffnet und sogleich stürmten unzählige, panische Menschen ins Innere. Kumalo und seine Helfer hatten Mühe damit, die Menge zurückzuhalten und sie halbwegs geordnet einsteigen zu lassen.

\par

Quälende fünfzehn Minuten dauerte es, bis sich der Tower meldete. \WR{Flug dutzend eins. Sie haben grünes Licht. Vorsicht beim Starten. Dutzend zwei und drei folgen ihnen direkt, denn diese Arschlöcher haben es jetzt auf den Flughafen abgesehen und wir haben wenig Zeit. Viel Glück!}

\par

Nico wartete keinen weiteren Augenblick mehr. Er ließ sein Schiff langsam aus dem Hangar heraus rollen. Danach folgte direkt die Startbahn. Er musste das Schiff nur noch ein wenig gerader ausrichten und dann war er schon bereit, abzuheben.

\par

Es war eine Weile her, seitdem er das letzte mal in einer \EN{W zweidutzendacht} gesessen hatte. Doch die Kontrollen kannte er noch in und auswendig. Und unter Beschuss startete er auch nicht zum ersten mal.

\par

Als Nico sich sicher war, dass sein Fluggerät parallel zur Startbahn ausgerichtet war, stellte er die Seitensteuerung fest, warf seiner Frau erneut einen kurzen Blick zu und drückte den Beschleunigungsregler bis zum ersten Anschlag voll durch.

\par

Der Plasmaantrieb erwachte zum Leben und Flug dutzend elf rüttelte so stark, dass jeder an Bord mit Ausnahme von Nico glaubte, das Schiff würde demnächst auseinanderbrechen. Doch er wusste, was er seinem Flieger zumuten konnte und kippte einen kleinen Schalter am Beschleunigungsregler nach oben. Capezin wurde in die superheiße Flamme der Düsen eingespritzt und erhöhte kurzzeitig die Beschleunigung.

\par

Dann trafen die ersten Strahlen das Rollfeld. Die Feinde mussten im Sturzflug angreifen, denn der Einschlagwinkel war recht steil. Eine Ladungen trafen auf die Startbahn nur wenige Meter vor Flug dutzend eins. Nico musste sich zwingen, nicht das Steuer zu sich zu ziehen, um die Nase vom Boden zu kriegen. Er wusste, dass er noch nicht genügend Geschwindigkeit aufgenommen hatte, um abheben zu können. Wenn er es jetzt versuchte, würde er nur ein wenig Höhe gewinnen und dann wieder zurückfallen. Und bei diesem Start gab es sicher keinen zweiten Versuch.

\par

Einen Augenblick später ging Flug dutzend drei in Flammen auf. Nico blickte aus dem Steuerbordfenster und musste mitansehen, wie eine kompakte Salve auf das Schiff einprügelte und es schließlich zur Explosion brachte~-- mit mehreren hundert Menschen an Bord.

\par

\WR{Jetzt oder nie}, sagte er, mehr zu sich selbst und zog das Steuer zu sich.
