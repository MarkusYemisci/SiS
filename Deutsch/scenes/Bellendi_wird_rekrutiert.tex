Kinder und andere Lehrer warfen Marco Bellendi gleichermaßen schwer zu deutende Blicke zu, als er die Stufen überquerte, die in das karge Hauptgebäude der Ilonka-Malleum-Schule führten. Das Wetter in Neu Kazan war frischer als es für die hiesige Jahreszeit üblich gewesen wäre. Eine gelungene Abwechslung, wie Bellendi vielleicht als einziger fand. Sein Kopf tat ihm nach wie vor weh. Er hatte gar nicht viel getrunken. Ein Gläschen Merlot, der ihm beim Einschlafen hätte helfen sollen. Vergebens, wie sich nach einer halben, schlaflosen Nacht herausgestellt hatte. So war er nun verkatert, müde und musste sich darüber hinaus eingestehen, dass er die Dinge, die ihm während seiner Jugend so viel Freude bereitet hatten, nicht mehr vertrug.

\par

Das Innere des Schulgebäudes wirkte seltsam leer. Durch das seltsam grelle Licht der aufgehenden Sonne war zu erkennen, dass zwar kaum Schüler anwesend waren, aber dafür umso mehr Abfall herumlag. Das Kollegiengebäude war niemals der sauberste Ort der Welt gewesen und damit ein passender Spiegel für ganz Neu Kazan. Aber der Enthusiasmus der Reinigungsmannschaften war ebenfalls nicht zu unterschätzen.

\par

Marco fragte sich sofort, ob er entweder zu früh, oder viel zu spät war. Doch ein kurzer Blick auf die nächstbeste Zentraluhr zeigte ihm, dass er uncharakteristisch pünktlich war. Durch die hohen Fenster des quaderförmigen Baus, der den barocken Stil auf die schlimmste Art imitierte, sah er, dass auch der Pausenhof wie leergefegt war.

\par

Bellendis Schritte hallten laut auf dem glänzenden Marmorboden wieder, als er sich in Richtung seines Klassenzimmers ging. Der Merlot dröhnte dabei jedoch stärker in seinen Ohren. Er fand es von Tag zu Tag schwerer, die banalsten Dinge wie sich zu Waschen, aufzuräumen oder auf modische Kleidung zu achten, hinter sich zu bringen. Insbesondere, wenn er am Abend zuvor seinen Ärger über die verpasste Bewerbung mit einem mehr oder eher weniger guten Wein versucht hatte, zu verscheuchen.

\par

Darum ärgerte es ihn besonders, dass nun keine seiner Kollegen, die ihn unbedingt fallen sehen wollten, erkennen musste, dass er sich für einen weiteren Tag zu seiner Arbeit geschleppt hatte. Er schien die einzige Lehrkraft weit und breit zu sein.

\par

Knarzend öffnete sich die schwere Holztür zum Biologiesaal, die seltsam deplatziert wirkte, als er kurz davor stand. Überhaupt schien das Kollegiengebäude sich nicht entscheiden zu können, ob es denn nun eine moderner Unionsbau oder eher ein Museum irdischer Architektur sein wollte.

\par

Der Biologiesaal an sich war jedoch recht ansehnlich. Er verlief nach hinten hinaus aufsteigend und bot so den größtenteils desinteressierten Schülern einen guten Blick auf Hologramme und Tierattrappen, die sie jedoch meistens noch weniger spannend fanden, als den eigentlichen Unterricht.

\par

Diesmal saßen jedoch nur zwei Kinder in der vordersten Reihe. Der gesamte Rest des Raumes war menschenleer. Das es gerade diese beiden Schüler waren, verwunderte Marco Bellendi nicht. James Whitmore, von seinen Mitschülern eher weniger liebevoll \Wr{Fleischkäse} genannt, war einer der besten in der ganzen Klasse und verpasste freiwillig nie auch nur eine Stunde. James war ein dankbarer und einfach zu unterrichtender Schüler, der Bellendi allerdings auch keine Herausforderungen bot.

\par

Sein Begleiter, Massimo Grosso, war im Gegensatz zu seinem besten Freund, weder am Schulleben, noch am Unterricht interessiert. Er hatte sich James zum besten Freund auserkoren und wich seither nicht mehr von seiner Seite. Auch wenn das hieß, die für ihn langweiligsten Kurse zu besuchen, die ihm zur Auswahl standen. Selten aufmerksam, spielten sich seine Interessen weitestgehend außerhalb ab. Er spielte gerne Fußball, und verfolgte mit großem Interesse die Weltmeisterschaft.

\par

Vielleicht war es das, dachte sich Bellendi. Am irdischen Vortag hatte das Endspiel stattgefunden. In Neu Kazan war es mitten in der Nacht übertragen worden. Wenn man die Feierlichkeiten bedachte, die meistens völlig gewinnerunabhängig folgten, wunderte es Bellendi nicht, warum doch der ein oder andere lieber ausschlafen würde.

\par

Auf der anderen Seite hatte er nichts von irgendwelchen Festen mitbekommen. Und trotz seiner intensiven Unterhaltung mit einer Flasche Merlot, hätte ihm das eigentlich nicht entgehen dürfen.

\par

\WR{Guten Morgen, Herr Belendi}, erklang es von den beiden einzigen Anwesenden unisono. Der Gleichklang wirkte auf den Begrüßten bereits bei einer halbwegs vollen Klasse seltsam. Wo er nun von nur zweien gesprochen wurde, erschien er ihm noch deutlich deplazierter.

\par

\WR{Morgen}, erwiderte der Lehrer und knallte seine Tasche auf das Pult vor der elektronischen Tafel. Diese stellte klar sein Lieblingsspielzeug dar. Sie funktionierte wie ein großes Buch, allerdings mit einem Holoprojektor ausgestattet und somit in der Lage, auch dreidimensionale Skizzen darzustellen. Eigentlich eher ein Werkzeug zur Veranschaulichung von Vektorgeometrie. Aber das Hologramm konnte auch zu einer schönen Doppelhelix werden.

\par

\WR{Wo sind denn alle?}, fragte Marco Bellendi schließlich, nachdem ihn die Blicke von Massimo und James förmlich durchbohrten. Die beiden sahen erst einander durchdringend an, bevor der letztere dem Lehrer antwortete: \WR{Ich nehme an, die sind zur Sicherheit zu Hause geblieben. Genauso wie die meisten Ihrer Kollegen. Wegen dem Angriff.}

\par

Nun war es an Bellendi, James anzustarren. \WR{Welcher Angriff?}

\par

\WR{Sie wissen davon nichts?}, war James ungläubige Antwort. Nachdem Bellendi eine Weile lang still blieb, fasste er zusammen, was über die Ereignisse um Pollux Primus bekannt geworden war. Während er erzählte, ließ sich Bellendi langsam auf seinen Sessel sinken.

\par

Natürlich war ihm nicht entgangen, dass die Regierung tatsächlich Außerirdische für den Angriff verantwortlich machten. Dass sie dies so deutlich und ohne Einschränkung tat, überraschte ihn. Das war der Traum seines Lebens gewesen. Menschen trafen auf intelligentes Leben, dass seinen Ursprung weit weit weg hatte. Nicht einmal hatte er daran gedacht, dass eines solche Begegnung derart gewalttätig werden konnte.

\par

Irgendwann fand er in die Realität zurück und bemerkte, dass ihn von den beiden Schülern zumindest James noch erwartungsvoll ansah, während Massimo bereits etwas in sein Buch kritzelte.

\par

\WR{Wieso seid ihr beiden denn noch da?}, fragte er schlißelich.

\par

\WR{Es gilt immer noch die Schulpflicht}, war James prompte Antwort. Bellendi kannte die wahre Antwort. James hatte die Chance gesehen, einmal einen Tag in der Schule ohne seine Klassenkameraden zu verbringen, die ihn wegen seiner Leibesfülle permanent hänselten. Und Massimo wäre ohne seinen besten Freund vermutlich schnell langweilig geworden.

\par

Marco hasste es, James enttäuschen zu müssen. Trotzdem sagte er: \WR{Geht nach Hause. Kümmert euch um eure Familien. Wenn alles stimmt, was ihr sagt, dann habt ihr sicher einiges zu bereden.}

\par

Massimo räumte bereits seine Sachen zusammen, kaum, dass Bellendi zu Ende gesprochen hatte. James tat dies etwas widerwilliger, verabschiedete sich dann aber dennoch von seinem Lehrer. Dieser gab den beiden ein halbherziges \WR{bis morgen} mit, während er in Gedanken bereits wieder ganz wo anders war.

\par

Es dauerte eine Weile, bis er sein Buch zückte und die Seite aufschlug, die ihm normalerweise die Nachrichten präsentierte. Natürlich berichtete der größte Artikel vom Angriff auf Pollux und enthielt dabei wenig mehr, als ihm die beiden Kinder bereits gesagt hatten. Unbekannte Angreifer, kaum jemand hatte fliehen können, mehrere Millionen Tote. Die Worten wie solche, die man in hundert anderen Artikeln finden konnte. Aber sie bedeuteten viel schlimmeres. So viele Opfer hatte es seit dem Routenkrieg nicht mehr gegeben. Eine ganze Kolonie war vollkommen ausgelöscht worden innerhalb von Stunden.

\par

Erneut verging einige Zeit, bevor Bellendi sich wieder von seinen Gedanken löste. Diesmal war es das Räuspern einer jungen Dame, dass seine Aufmerksamkeit erregte. Sie trug die Uniform der Navy in der Galaausführung für offizielle Anlässe. Er hingegen musste schrecklich aussehe. Seit Tagen hatte er sich nicht mehr rasiert und seine  Kleidung war auch schon seit einiger Zeit nicht mehr im Reiniger gewesen.

\par

Die junge Frau hingegen war in diesem Moment sein exaktes Gegenteil. Ihre Uniform wirkte so adrett wie ihr Haarschnitt und schmeichelte ihrer schlanken und athletischen Figur. Sie war keine Schönheit wie man sie aus den Krypta-Auftritten der großen Modefirmen kannte. Aber sie wirkte zweifellos nicht unattraktiv. Ihr Pony überdeckte ihre Stirn und ihre langen Haare hatte sie sich sauber zu einem Zopf zusammengebunden.

\par

Die Dienstwaffe, die an ihrem Gürtel befestigt war, ließ Marco Bellendi wissen, dass sie sich nicht nur im Dienst, sondern wahrscheinlich auf einer Übung befand.

\par

\WR{Guten Tag}, grüßte sie förmlich. \WR{Sind Sie Marco Bellendi?}

\par

\WR{Ähm, ja}, antwortete er und bemerkte, dass seine Stimme belegt und heiser klang. Er räusperte sich. \WR{Das stimmt.}

\par

Die Soldatin reichte ihm ein Klemmbrett, dass sie unter ihrem Arm getragen hatte. \WR{Ich bin Lieutenant Hanna Moyér. Das ist eine Einladung vom Kommando. Wir bieten Ihnen eine befristete Anstellung an.}

\par

Marco Bellendi gab ihr das Schreiben zurück, ohne es zu lesen. \WR{Tut mir leid. Ich habe schon eine Arbeit. Ich bin nicht an Ihrer Biowaffenabteilung interessiert.}

\par

Das Militär warb hin und wieder weniger erfolgreiche Wissenschaftler für Waffenprojekte an, die niemand sonst machen wollte. Marco wusste davon, hatte aber keinerlei Ambitionen, seinen Intellekt für die Entwicklung von Kriegsmaterial zur Verfügung zu stellen.

\par

Lieutenant Moyér nahm das Klemmbrett jedoch nicht wieder zurück und antwortete: \WR{Mein Herr, es geht nicht um Waffenentwicklung. Wir brauchen fähige Leute um die Invasion zu analysieren.}

\par

In diesem Augenblick war Bellendi froh, bereits von James und Massimo von dem Angriff der Unbekannten erfahren zu haben. Sonst hätte ihm Moyér nun erst einmal erklären müssen, wovon sie sprach. Andererseits war ihm nicht klar, wofür ihn das Militär brauchte. Für das Kommando und für die Regierung arbeiteten gutbezahlte Spezialisten.

\par

\WR{Ich bin kein Analytiker}, antwortete Marco etwas betreten. \WR{Ich wüsste nicht, wie ich behilflich sein könnte.}

\par

Lieutenant Moyér wurde ungeduldig. \WR{Herr Bellendi, ich wurde hergeschickt, um sie abzuholen. Irgendwer da oben glaubt, sie brauchen zu können. Aber deswegen haben trotzdem nicht ewig Zeit. Entweder sie kommen sofort mit oder wir suchen uns jemand anderen.}

\par

Bellendi sah an sich hinab. Sein abgenutztes Sakko wirkte geradezu schäbig, verglichen mit der hellblauen Uniform der jungen Offizierin. Und sie war gerade mal Lieutenant. Ihre vorgesetzten dürften noch einmal deutlich schicker gekleidet sein. Und falls sie Bart trugen, dann sicher einen gut getrimmten, im Gegensatz zu Marcos Faulheitsbehaarung.

\par

Nachdem er seinen Blick durch das leere Klassenzimmer gehen gelassen hatte, wurde ihm klar, dass er überall hingehen konnte. Es würde kaum schlimmer werden, als es sein Leben gerade war. Also wischte er sich in einem von vornherein zum Scheitern verurteilten Versuch, die Falten zu beseitigen, hastig über seine Klamotten und folgte dann Lieutenant Moyér. Ohne zurückzublicken ließ er die Tür hinter sich zufallen. Er und die Soldatin traten den Schulkorridor hinunter, während er sagte: \WR{Ich hoffe, es gibt eine Dusche, wo wir hingehen.}
