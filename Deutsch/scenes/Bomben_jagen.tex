Morten jagte einem der Feinde hinterher. Er war schnell. Definitv schneller, als sein eigenes Schiff und wenn er die Nachbrenner benutzte, dann bestand das Risiko, an seinem Ziel vorbei zu schießen und selbst zum Gejagten zu werden. Hastig gab er ein paar Schüsse aus seinen Bordkanonen ab, denen der Feind aber gekonnt auswich. Vielleicht würde eine Raketenerfassung gelingen. Doch bevor er diese den Zielcomputer ausführen lassen konnte, drang Anna Farleys Stimme an sein Ohr.

\par

\WR{Rot drei und vier beidrehen! Die Abfangjäger haben Schwierigkeiten mit den Marschflugkörpern. Ein paar werden vielleicht durchkommen.}

\par

Morten bestätigte knapp und ließ von seinem Gegner ab. Auf dem runden Radarbildschirm genau in der Mitte seines Cockpits konnte er Rot eins~-- Annas Jäger~-- erkennen. Zusammen mit schier zahllosen anderen Blickpunkten, die feindliche und eigene Jäger repräsentierten.

\par

Während er seinen Falken herumziehen ließ, um sich mit Anna in eine Formation zu begeben, zog das Schlachtfeld an ihm vorüber. Zu seiner rechten waren die drei feindlichen Kriegsschiffe zu erkennen, die permanent ihre dunkellrot schimmernden Salven auf die Flotille um die \EN{Regenvogel} abgaben. Noch prallten die Geschosse nur wirkungslos auf die Schutzfelder. Doch umso intensiver der Beschuss wurde, umso schwerer würde es der Computer der \EN{Regenvogel} haben, zu berechnen, wo die nächsten Strahlen einschlagen würden, um die Schutzfelder dort aufzubauen.

\par

Der Träger erwiderte den Beschuss nicht. Die Kanonen der \EN{Regenvogel} waren zu klein, um einem Großkampfschiff gefährlich werden zu können und besser geeignet, sich feindliche Jäger auf Distanz zu halten. Solche zogen ihre Bahnen um den Träger und seine Eskorte. Einer der Flieger hatte sich wohl zu nah an die \EN{Regenvogel} herangewagt, konnte nicht mehr ausweichen und wurde prompt von den Raketen aus dem Deck-Geschützturm zerlegt.

\par

Dann zuckten die ersten hellgelben Strahlen aus den Kanonen der Zertörer und hämmerten auf die Schutzfelder des größten feindlichen Schiffes ein. Durch den luftleeren Raum verbreiteten sich keine Geräusche, doch Morten konnte sich gut vorstellen, welches Gedröhn nun im Inneren der \EN{Vulkan} ertönen musste, als die beiden Rohre ihres Hauptgeschützes vom Rückstoß nach hinten gedrückt wurden und zwei gleißende Entladungen ausspieen. Fast wie Rauchschwaden schwelten versprengte Rückstände der abgefeuerten Strahlen um die Kanonen.

\par

Für einen Moment glaubte Morten, die Schlacht könne tatsächlich zu den Shutek getragen werden und er nahm sich einen Moment, um wieder zur Ruhe zu kommen. Die Wut aus seinem Traum hatte sich bereits wieder verzogen und langsam kehrte seine kühle Konzentration zurück, die er sich im Verlauf seiner Ausbildung hart erarbeitet hatte.

\par

Doch dann gellte ein verirrter Todesschrei durch den Äther. Fast zeitgleich verschwanden zwei grüne Blickpunkte von Mortens Radarbildschirm. Das Sensorenlogbuch zeigte bereits das Verschwinden der beiden Jäger Charlie drei und vier ab. Die Staffel schwerbewaffneter Abfangjäger war nach der Schlacht von Pollux unter das Kommando von Kenji und Kringel gestellt worden.

\par

Morten atmete auf, als er ersteren sagen hörte: \WR{Leute, wir haben Tyrell und Hutch verloren. Mehrere Marschflugkörper sind durchgekommen! Sie erreichen jeden Augenblick den inneren Zirkel!}

\par

Eine Bezeichnung, die, wie Morten fand, nicht zutraf, da es sich bei dem Bereich unmittelbar um ein größeres, zu schützendes Objekt, um eine gedachte Kugel und nicht um einen Kreis handelte.

\par

\WR{Verstanden, Charlie eins}, begann Farleys knappe Antwort, \WR{wir sind bereits auf dem Weg.}

\par

Morten erkannte Kevins Jäger, dessen Flügel dieser in letzter Minute noch mit einem roten Adlerkopf bemalt hatte. Sein Freund hatte ein gewisses Talent mit Spraydosen. Doch die kurze Zeit bis zum Einsatz in Kreuzpunkt hatte dazu geführt, dass das Gemälde weniger nach einem Raubvogel und mehr nach einem Hahn aussah.

\par

Beide nahmen Flankenposition zu Anna Farleys Jäger ein. Wie die meisten anderen auch, trug ihre Maschine lediglich einige rote Streifen, die üblicherweise vor einem Einsatz aufgeklebt wurden, um die Staffelzughörigkeit anzuzeigen.

\par

\WR{Haut eure Nachbrenner rein, Jungs}, befahl Anna was sich weder Morten noch Kevin noch Nico zweimal sagen ließen. Gemeinsam schossen die vier Jäger nun auf einen gedachten Punkt zu, auf dem sie Marschflugkörper laut Navigationscomputer den Kurs der Raketen kreuzen sollten.

\par

\WR{Scheiße, das sind ja fünf!}, meldete Kevin per Funk.

\par

Morten warf einen kurzen Blick auf seinen Radarbildschirm. Tatsächlich zogen fünf gelbe Blickpunkte mit dem Symbol eines Marschflugkörpers auf die \EN{Regenvogel} zu. Sie waren schnell, variierten aber ständig ihre Flugbahn, um eventuellem Abwehfeuer auszuweichen.

\par

\WR{Rote Staffel}, beschwörte Farley, \WR{wir müssen diese Dinger \textit{in einem  Anflug erledigen}. Sie sind schon viel zu nah am Träger. Wir bekommen keine zweite Chance!}

\par

\WR{Führe Raketenerfassung des vordersten Geschosses aus}, meldete Nico und rief damit sofort Mortens Misstrauen hervor. \WR{Nummer zwei hol ich dann mit den Bordkanonen runter.}

\par

\WR{Morten, du nimmst dir M drei vor, Kevin, du M vier und ich das Schlusslicht.}

\par

\WR{Bestätigt, Madam}, sagten die beiden angesprochenen fast unisono.

\par

Und schon wurden die Geschosse auch für das bloße Auge sichtbar. Der Zielcomputer hob die Flugkörper durch das Zielmonokel farblich hervor. Doch ihre grünen Antriebsschweife waren auch so gut zu sehen.

\par

Mortens Jäger ruckelte hin und her, als er versuchte, sein zugewiesenes Ziel ins Fadenkreuz zu bekommen. Er würde etwas vorhalten müssen, um die schnell fliegende Rakete  zu treffen. Immer wieder wackelte jedoch seine Zielerfassung hin und her. Aber um abzubremsen und einen Schuss mit Kampfgeschwindigkeit abzugeben, blieb ebentuell keine Zeit mehr.

\par

Als ließ er einfach seine Kanonen aufheulen. Die eng am Bug sitzenden Rohre feuerten kurze Entladungen ab, so hellgelb, dass sie bereits fast weiß erschienen. Während sich der Kondensator des Waffensystems entlud, korrigierte Morten seine Flugrichtung. Der erste Teil der Salve ging ins leere, doch der zweite Traf voll auf sein Ziel. Der Marschflugkörper verging in einem grellen Blitz, Sekunden, bevor Mortens Jäger durch die Trümmer brauste.

\par

\WR{Rot zwei, B eins!}, meldete Nico und feuerte eine seiner eigenen Raketen ab. Dann eröffnete er zeitgleich mit Anna und Kevin mit den Bordkanonen das Feuer. Kurz darauf waren drei weitere Marschflugkörper zerstört. Doch der verbleibende zog ein wenig nach oben und wich so der Rakete aus, die Nico auf ihn abgefeuert hatte. Diese drehte zwar um, doch ihr Wendekreis war derart weitläufig, dass sie ihr Ziel niemals treffen konnte, bevor diese auf der \EN{Regenvogel} aufschlagen würde.

\par

Morten beobachtete die Szenerie hilflos. Das Geschoss, ein massiger, länglicher Zylinder, war deutlich größer, als die Waffe, die in Pollux den halben Antrieb der \EN{Regenvogel} ausgeschaltet hatte. Im schlimmsten Fall würde sie den Träger in zwei Teile sprengen.

\par

Doch dieser hatte noch eine letzte Schutzfunktion. Das Raketenschutz auf dem Deck kurz vor der Brücke spie eine Salve Torpedonadeln aus. Kleine Raketen, kaum länger als ein Arm, die ohne eigenen Sprengkopf auf ihr Ziel zuschossen.

\par

Einige der Geschosse verfehlten ihr Ziel, dass sich ständig ein wenig nach links oder rechts bewegte, doch dann bohrte sich eine der Nadeln frontal in den Sprengkopf des Marschflugkörpers und brachte ihn so zur Explosion.

\par

Ein lautes Aufatmen war durch den Funk zu hören, bevor Anna sagte: \WR{An alle: feindliche Geschosse sind hervorragend programmiert. Sie reagieren auf Abwehrfeuer. Zielt genau!}

\par

Morten hatte nicht genau hingehört. Er schenkte seinem Radar gerade mehr Beachtung. Eine Gruppe feindlicher Jäger war wohl auf die rote Staffel aufmerksam geworden. Die Blickpunkte rückten schnell näher und einer von ihnen hatte sich an Mortens Heck geheftet.

\par

Dieser riss sein Steuer herum und entging so bereits der ersten Salve. Die rotglühenden Strahlen verloren sich schnell im Nichts. Doch das Ausweichmanöver brachte ihn auf Kollisionskurs mit der \EN{Vaillant}, einem der Zerstörer. Um diesem wiederum auszuweichen war es zu spät und Mortens Jäger zu schnell. Darum steuerte er unter der Brücke des Schiffes hindurch und schickte ein Stoßgebet gen Himmel, dass die Schutzfelder der \EN{Vaillant} ihn nicht als feindliches Geschoss erkennen würden.

\par

Als er knapp unter der Brücke hindurchschoss und dabei nicht auf eine unsichtbare Mauer traf, atmete er erleichtert auf. Doch sein Schatten folgte ihm nach wie vor. Mortens Computer identifizierte ihn als Halphas drei und zeigte eine vorläufige Abtastung, sowie alle verfügbaren Daten über den Schiffstypus an. Der Jäger war von Platten bedeckt, die ihn wie einen Käfer wirken ließen. Trotzdem sah er genauso flink und wendig aus, wie er auch war.

\par

Morten schaltete auf seine Heckkamera und sah, wie der Feind schnell näher kam. Seine Geschwindigkeit überragte die des Falken um einiges. Bald wäre er in Feuerreichtweite. Wie aus dem Nichts wurde er jedoch von den Strahlen aus Kevins Bordkanonen getroffen, fing an zu brennen und zerplatzte schließlich.

\par

\WR{Vier, Morten}, sagte Kevin, \WR{vier!}
