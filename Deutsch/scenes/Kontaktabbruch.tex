\WR{Weiß jemand, wie es ihm geht?}, fragte Präsidentin Akintola leise. Im Kommandozentrum der \EN{Athena} war es bei weitem zu laut, als das jemand die Präsidentin gut verstanden hätte. Lediglich Richard Bellegardé, der sich noch am ehesten denken konnte, was Akintola wissen wollte, antwortete: \WR{Nein. Er hat wenig Freunde und bekannte und auch kaum Kontakt zu seiner Familie. Aber ich kenne ihn jetzt schon sehr lange. Er wird zurecht kommen, Frau Präsidentin.}

\par

Akintola zeige keine Reaktion. Ihr Blick war fest auf die taktische Ansicht des Raums um Kreuzpunkt Primus gerichtet.

\par

Sie und ihr Stab standen den Offizieren der \EN{Athena} im Weg. Eine Tatsache, welche die Präsidentin kommen hatte sehen, sie jedoch nicht davon abgehalten hatte, das Kommandozentrum und Herz des Konglomerats zu besuchen. Und es verwunderte sie, dass sie sich keine Sehnenscheidenentzündung zugezogen hatte, so viele Sondergenehmigungen hatte sie für die meisten ihrer Berater unterschreiben müssen.

\par

Um das mehrere Meter messende, taktische Hologramm in der Mitte des Kommandozentrums waren kreisförmig mehrere dutzend Arbeitsstationen angeordnet, die zu jenem Zeitpunkt voll besetzt waren. Grandadmiral Burns tigerte unruhig immer wieder um die Monitore herum. Akintola kannte ihn als sehr beherrschten und verschlossenen Mann, was ihr den Ernst der Lage nur noch mehr vor Augen führte.

\par

\WR{Nach wie vor nichts, Lieutenant?}, fragte der Kommandant an einen der Kommunikationsoffiziere gewandt.

\par

Dieser nahm kurz seine Kopfhörer ab, um leichter mit dem Grandadmiral sprechen zu können. \WR{Tut mir leid, Sir. Wir versuchen schon seit Stunden, irgendwen zu erreichen. Aber nicht nur die militärischen Kanäle sind tot. Auch alle zivilen NZTs scheinen einfach weg zu sein.}

\par

Der Ärger darüber, so gut wie nichts über die Situation in Kreuzpunkt zu wissen, bahnte sich schließlich seinen Weg nach außen und Burns schlug mit seiner Faust auf das nächstgelegene Pult. Nicht nur der Offizier, mit dem er gerade gesprochen hatte, fuhr zusammen.

\par

\WR{Verdammt}, sagte er laut, \WR{Sie müssen mir doch irgendetwas sagen können!}

\par

\WR{Tut mir leid}, wiederholte der Mann. \WR{Das letzte, was wir wissen, ist dass die \EN{Minerva} feindliche Schiffe gesichtet hat. Danach war plötzlich komplette Funktstille.}

\par

Eine Frau mit schlohweißem Haar und hochgradig dekorierter Uniform trat an den Kommandanten heran und sagte so leise, dass nur Burns sie hörte: \WR{Sir, ich muss anmerken, dass ich für einen Fehler halte, die \EN{Regenvogel} samt Begleitschiffen dorthin zu schicken. Wir wissen nicht, was dort vor sich geht und im schlimmsten Fall verlieren wir Kreuzpunkt und den Träger. Dann steht auch keine Verteidigungslinie mehr zwischen Kreuzpunkt und dem Rest der Welt.}

\par

\WR{Zur Kenntnis genommen}, brummte der Grandadmiral. \WR{Wie lange, bis sie springen kann?}

\par

\WR{Die \EN{Regenvogel} erreicht die Hyperraumroute in fünf Minuten. Ich habe die Bestätigung von Captain Fiscale, dass auch die drei Basisschiffe der Phalanx und die Zerstörereskorte bereit zum Angriff sind.} Die kam von einer Offizierin, die wie durch ein Wunder den Überblick über drei große Glasmonitore behalten konnte.

\par

\WR{Sie sollen ihren NZT an lassen. Die ganze Zeit, auch während des Sprungs}, wies Burns an und der Kommunikationsoffizier gab den Befehl weiter.

\par

Während dessen ließ Lertha Akintola ihren Blick über die Reihen an Arbeitsstationen gleiten, die sich fast wie in einem Hörsaal um das Hologramm herum erhoben. Die zahllosen blinkenden Lichter erinnerten sie fast an Weihnachtsbeleuchtung. Die Assoziation damit war jedoch das einzig Besinnliche in diesem Augenblick. Die Offiziere, die im Kommandozentrum ihren Dienst verrichteten, wirkten allesamt angespannt, teilweise sogar verängstigt.

\par

Nun stand die Präsidentin zu Grandadmiral Burns und sprach ihn an. \WR{Norton, wie sieht es aus? Glauben Sie, dieser eine Träger wird einen Unterschied machen können?}

\par

\WR{Leider, Frau Präsidentin}, begann er, \WR{weiß ich derzeit gar nichts. Diese Operation ist der reinste Blindflug. Aber wir können Kreuzpunkt nicht im Stich lassen. Sieben ocin Menschenleben sind in Gefahr. Ich hoffe nur, wir bekommen wieder einen Kontakt mit unsere Streitkräften in Kreuzpunkt, sobald die \EN{Regenvogel} dort ankommt.}

\par

Akintolas Stirn legte sich in tiefe Falten. \WR{Der Geheimdienst hat mich darüber informiert, dass die Shutek in Pollux ein Computervirus verwendet haben, um unsere Kommunikation zu stören. Könnte das hier auch der Fall sein?}

\par

\WR{Darüber denke ich die ganze Zeit nach}, antwortete der Grandadmiral. \WR{Aber schon kurz nach dem Angriff haben wir unionsweit Schutzprogramme gegen diese Schadsoftware in Umlauf gebracht. Außerdem versuchen wir auch zivile Gegenstellen zu erreichen. Aber die scheinen alle gekappt zu sein. Ich halte es für extrem unwahrscheinlich, dass ein Virus es schafft, alle potentiellen Empfänger zu erreichen. Ich glaube, die Shutek verwenden eine neue Technik. Entweder alleine oder im Zusammenspiel mit dem Virus.}

\par

Aktintola sah sich erneut um. Es gab zwar etliche Stühle, doch die waren alle besetzt und sie stand nun schon seit einer Weile in der Kommandozentrale. \WR{Ich würde gerne bald eine weitere Pressekonferenz abhalten. Die Menschen haben mehr Angst, als ich es bisher jemals erlebt habe. Diese Shutek dringen in ein System nach dem anderen ein. Die ersten beiden waren vergleichsweise kleine Kolonien. Aber Kreuzpunkt… Wenn wir unsere zweite Erde verlieren, dann verlieren wir auch gleichzeitig das Vertrauen des Volkes in unsere Fähigkeit zur Gegenwehr.}

\par

Burns atmete schwer. \WR{Bei allem Respekt, Frau Präsidentin. Aber das ist ihr Schlachtfeld. Ich habe mich niemals in politische Belange eingemischt. Alles, was ich tun kann, ist zu versuchen, die Shutek zurückzuschlagen. Sobald ich aber irgendwelche Neuigkeiten habe, die den Menschen Mut machen können, seien Sie sicher, ich überbringe sie Ihnen sofort.}

\par

Akintola nickte langsam. \WR{Danke, Grandadmiral. Momentan ist alles, was ich tun kann, Ihnen viel Glück zu wünschen. Gute Jagd.}

\par

Wie auf das Stichwort der Präsidentin meldete sich die Offizierin am Ausguck zu Wort. \WR{Sir, die \EN{Regenvogel} hat ihrem Hyperantrieb hochgefahren und beginnt mit dem Countdown für den Sprung.}

\par

\WR{Die gute Jagd wünsche ich jetzt ihnen}, entgegnete Burns und deutete auf das kleine Symbol auf dem riesigen taktischen Hologramm, dass den leichten Träger darstellte. Kurz darauf war es verschwunden. Und auch die sechs anderen Piktogramme, welche die Zerstörer und Basischiffe der Bodentruppen zeigten, verloren sich eines nach dem anderen auf ihrem Weg nach Kreuzpunkt Primus, als die Schiffe in den Hyperraum eintraten.

\par

Plötzlich drang ein ohrenbetäubender, schräger Ton aus einem der Lautsprecher. Der Chef der Kommunikationsabteilung drehte sofort die Lautstärke herunter.

\par

\WR{Was ist das?}, fragte Burns verärgert.

\par

\WR{Ich habe keine Ahnung, Sir}, antwortete der Offizier sofort. \WR{Es kommt aber eindeutig vom NZT der \EN{Regenvogel}.}

\par

\WR{Was? Ist sie denn schon angekommen?}

\par

\WR{Negativ.} Der Mann schien sich selbst nicht zu glauben. \WR{Sie ist noch im Hyperraum. Aber wir bekommen trotzdem ein Signal!}

\par

Das Quitschen, das eben noch den ganzen Raum ausgefüllt hatte, erstarb langsam und wurde von leiseren, aber klarer verständlichen Lauten abgelöst. Zunächst klangen diese nach einem Rauschen, erschienen aber schnell zu regelmäßig. Und kurz nachdem der Kommunikationschef die Lautstärke wieder erhöht hatte, wurde den meisten der Anwesenden klar, dass es sich um Stimmen handelte.

\par

Akintola hatte ein ausgezeichnetes Gehör. Doch auch ihr fiel es schwer, zu verstehen, was gesagt wurde. Nur ein paar Sätze konnte sie halbwegs klar wahrnehmen. \WR{Ich kann sie spüren. Sie kommen. Es beginnt.}

\par

Darauf folgte lediglich Stille. Auch in der Kommandozentrale war es noch sehr leise, da jeder versucht hatte, die kaum hörbaren Worte zu verstehen. Dann wurde das Hologramm von einer roten Warnmeldung überlagert, die anzeigte, dass die Verbindung zur \EN{Regenvogel} und ihrer Flotille abgebrochen war.
