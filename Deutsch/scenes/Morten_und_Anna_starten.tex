\WR{Hey. Alles klar?}, fragte eine bekannte Stimme, die Morten wie Kilometer entfernt vorkam.

\par

Noch immer hielt er die Augen geschlossen. Ihm war schlecht und sehr schwindelig. Irgendwie traute er sich kaum, die Lieder zu heben. Kurz nach dem Aufwachen von seinem Traum, hatte Morten diesen immer noch deutlich vor Augen, vergaß ihn aber schnell wieder. Nach dem Sprung war es genau umgekehrt gewesen. Anfangs war er orientierungslos wieder zu sich gekommen, völlig ohne Ahnung, was ihm während des Hyperraumübergangs passiert war. Nun aber, kehrte langsam das Erinnern an den Schrecken ein.

\par

\WR{Ist dort hinten alles in Ordnung?}, fragte eine andere Stimme, die Morten ebenfalls kannte.

\par

Sie schien auch in Wirklichkeit recht weit weg zu sein. Kevin Wilson, dessen Stimme Morten sofort wiedererkannte, antwortete gespielt gelassen: \WR{Ja. Einen Moment dachte ich, Morten sei ohnmächtig aber er wird langsam wieder wach.}

\par

\WR{Der Sprung scheint ihm ganz schön zugesetzt zu haben}, spekulierte Jens Wörg, dem Morten die Worte langsam zuordnen konnte.

\par

Kevin murmelte leise: \WR{Vielleicht sollte er auf die Krankenstation.}

\par

Morten schlug überraschend die Augen auf. Kevin und Kringel wichen sogar ein ganz kleines Stück zurück.

\par

\WR{Nein}, entgegnete Morten sofort und machte sich daran die Gurte langsam zu lösen. Als er ein paar Schritte ging verzog sich langsam die Übelkeit und auch das Schwindelgefühl. Jens Wörg schlug ihm freundschaftlich auf die Schulter und brachte ihn dabei ohne es zu merken beinahe zu Fall. Kevin warf Morten ein paar besorgte Blicke zu, doch er sagte nichts. Es sah ihm auch nicht ähnlich, über seine Befürchtungen zu sprechen. Besonders nicht, wenn es hieß, zuzugeben, dass er sich um andere sorgte. Und Morten selbst war zu stolz, um seine Unpässlichkeit offen zu zeigen. Ein Kampfpilot, der keine Hyperraumsprünge vertrug, war unerhört.

\par

Anna Farley ergriff das Wort und rief über die Köpfe der Piloten, die sich gerade abschnallten: \WR{Also, los geht’s. Folgt der Startreihenfolge und bleibt konzentriert.}

\par

Morten folgte Kevin zur Treppe, die zum Flugdeck führte. Er musste sich sehr bemühen, nicht zu schwanken. Zum ersten Mal seit Stunden kreisten seine Gedanken nicht um Anna. Er dachte an seinen Traum zurück oder was auch immer ihm widerfahren war. War das nur eine seltsame Nebenwirkung eines Überlichtsprunges gewesen? Morten versuchte krampfhaft, sich an die Vorlesungen über Hyperraumphysiologie zu erinnern. Noch lange hatte die Wissenschaft nicht alle Geheimnisse insbesondere des tiefen Hyperraums entschlüsselt und es gab eine Menge vager Hypothesen. Aber von solchen Visionen hatte er noch nie gehört. Was auch immer es gewesen war, es hatte sich für Morten sehr real angefühlt und er hatte große Mühe, es jetzt wieder zu verdrängen.

\par

Auf dem Flugdeck angekommen, bemerkte Morten erst die große Hektik, die herrschte. Überall rannten Techniker umher, Ingenieure gaben Anweisungen und Piloten bestiegen langsam ihre Cockpits. Der Trubel bei der Landung des Transporters am Vortag war nichts im Gegensatz dazu gewesen.

\par

Drei Techniker schoben direkt vor Mortens Nase einen schwebenden Wagen vollbehangen mit Verfolgungsraketen vorbei. Ein anderes Trio befestigte Geschosse derselben Bauart unter den Tragflächen eines Raumjägers.

\par

\WR{Macht schneller beim Auftanken, wir müssen fertig werden. Los, los, los!}, hörte Morten den Chefmechaniker rufen, den alle nur Büffel nannten.

\par

Sein kantiges Gesicht, mit unverwechselbar osteuropäischen Zügen, lief rot an als er einen Techniker sah, der Probleme mit einer Rakete hatte. Der Mann schaffte es kaum, das schwere Geschoss unter die Tragfläche zu hieven und sein Partner, der auf dem Flügel des Jägers kauerte kam nicht auf die Idee ihm zu helfen.

\par

Ein altes Klischee schien sich hier zu bewahrheiten. Wenn es in der Armee der Union Probleme gab, dann wurden diese meistens durch Menschen verursacht. Zu wenig reale Erfahrung und unzureichendes Training waren die Hauptursachen dafür, dass viele Soldaten ins Wanken gerieten, wenn es wirklich ernst wurde.

\par

Der Büffel kam zu den beiden hingerannt und meckerte den unten stehenden lauthals an. Sein Partner, der auf der Tragfläche im Rücken des Deckchefs war, wedelte mit der Hand vor seinem Gesicht herum und erklärte Tukarev somit gestikulierend für übergeschnappt. Morten verkniff sich ein Grinsen.

\par

Die ersten Jäger starteten unter lautem Gedröhn. Es handelte sich dabei um die Bereitschaftsstaffel, die bei jedem Einsatz abhob, um den Träger und seine Geleitschiffe zu beschützen. Seit Morten aufgewacht war, waren mindestens zwei Minuten vergangen. Dass die Staffel jetzt erst startete war ebenfalls ein Zeichen für mangelnde Erfahrung. Das Hangarteam schien mit dem Massenstart einfach kaum fertig zu werden.

\par

Kenji schlug Kevin auf den Rücken und bedeutete ihm somit loszulegen.

\par

\WR{Hey, feiges Huhn}, rief Kevin Morten entgegen. \WR{Deine Mutter ist so fett, Antigravitations-Tragen wurden nur erfunden, um sie überhaupt auf die Wage zu bekommen.}

\par

Der Angesprochene war zu überrascht von dieser willkürlichen Frechheit, dass er kaum dazu kam, Kevin den Mittelfinger zu zeigen. \WR{Hast du einen Schaden!}, blaffte er schließlich und ging alle Beleidigungen durch, die ihm einfielen.

\par

Bevor er sich jedoch für eine entscheiden konnte, fragte Kevin: \WR{Kennst du diesen Brauch etwa nicht?} Augenrollend fuhr er fort: \WR{Ich beleidige deine dicke Mutti, damit du einen Grund hast, leben wieder zurück zu kommen und zu kontern. Kapiert?} Mit diesen Worten rannte er wild entschlossen auf seinen Jäger zu.

\par

Morten spürte wie ihn jemand freundschaftlich gegen den Arm boxte, bevor er den kindischen Humor seines Kameraden verurteilen konnte. Er wandte sich um und erkannte Major Anna Farley. Sie strahlte übers ganze Gesicht und Morten glaubte, dass ihm gleich wieder schwindelig werden würde. So nah waren sich die beiden bisher nicht gekommen.

\par

\WR{Sind sie bereit, Lieutenant?}, fragte Anna elanvoll.

\par

Morten versuchte unbewusst die Haltung einzunehmen, die er selbst am ehesten mit \Wr{bereit} beschreiben würde. \WR{Ja, Madame}, antwortete er zackig.

\par

Anna Farley lächelte matt. \WR{Hier an Bord sollten Sie mich so nennen}, begann sie zu erklären. \WR{Aber da draußen bin ich Anna. Oder \Wr{Fear}, das ist mein Rufname. Wie ist eigentlich Ihrer?}

\par

Morten errötete ungewollt. \WR{Ich habe keinen, Anna. Ich meine natürlich, Madam.}

\par

\WR{Schon in Ordnung}, antwortete seine Vorgesetzte knapp. \WR{Den finden Sie sowieso nicht am Boden. Sie sind Pilot. Irgendwo zwischen den Sternen. Legen wir los?}

\par

Ein entschlossenes Nicken von Mortens Seite folgte. Er folgte Anna zu zwei Aufklärungsjäger, die schon von weitem deutlich kleiner aussahen, als die anderen Flieger. Allerdings waren sie jeweils von einer Gerätschaft umschlossen, die sehr passend als Hyperraumschlitten bezeichnet wurde. Im Wesentlichen handelte es sich dabei um ein abkoppelbares Triebwerk, dass neben Flügen durch den flachen Hyperraum auch Sprünge in andere Systeme erlaubte.

\par

Bevor Anna Farley ihren Aufklärer bestieg schlug sie ihre Faust mit Mortens zusammen. Ein alter Brauch unter den Piloten der Starforce, den auch Morten kann und der schon so lange Bestand hatte, dass sich niemand daran erinnern konnte, wie er entstanden war.

\par

Drei Techniker entfernten gerade die Sicherungen der Bordwaffen seines Jägers. Sonst musste nicht viel getan werden. Ein Aufklärer brauchte nicht aufgetankt zu werden, da er keinen Nachbrenner hatte. Der Hauptreaktor war darauf ausgelegt, kurze Energiespitzen zu erzeugen um dem Antrieb vorübergehend mehr Leistung zu verschaffen. Auch mussten keine Raketen unter den Tragflächen des kleinen Schiffes angebracht werden.

\par

\WR{Viel Glück, Lieutenant}, rief einer der Techniker Morten zu, der daraufhin antwortete: \WR{Danke, Sir.}

\par

Dann schwang er sich in das geräumige Cockpit und lies die Frontscheibe herunterfahren. Die Steuerkanzel des Jägers war angenehm geräumig und machte im Prinzip ein drittel des ganzen Fliegers aus. Der Hauptgrund dafür, das ein Cockpit eines Aufklärers größer war als die Cockpits aller anderen Jäger, bestand darin, dass ein Aufklärer hin und wieder mehrere Tage lang im All verbrachte. Daher sollte dem Piloten so viel Komfort wie möglich verschafft werden. Als Nahrung standen einige Tablette mit den wichtigsten Nährstoffen und ein Trinkschlauch zur Verfügung. An die sanitären Anlagen wollte Morten erst gar nicht denken.

\par

Morten machte sich daran, die Systeme seines Aufklärers hochzufahren. Am längsten dauerte es, bis die hochentwickelten Sensorsysteme und Teleskope einsatzbereit waren. Die Abtaster eines Aufklärers waren fast genauso leistungsstark wie die der \EN{Regenvogel} und die Teleskope konnte ein halbes Sternensystem weit Echtzeitbilder liefern.

\par

Bedächtige rückte sich Morten seinen Helm zurecht und setzte sich sein Zielmonokel auf. Sofort darauf projizierte der Bordcomputer ein Fadenkreuz auf das dünne Glas des Monokels. Die Zielerfassung markierte die Stelle, an welche die Kanonen des Aufklärers schießen würde.

\par

Gerade starteten einige Jäger der Sternenflügel. Es waren Raumüberlegenheitsjäger der \EN{Falken}-Klasse. Mit ihrem länglichen Rumpf, ihren zwei waagerecht abstehenden Tragflächen und der windschnittigen Heckflosse kurz hinter dem Cockpit wirkten sie zwar massiger als ein Aufklärer behielten aber immer noch ein schnelles und wendiges Aussehen.

\par

\WR{In einer Minute sind wir dran, Lieutenant}, erklang Anna Farleys Stimme aus Mortens Kopfhörer. \WR{Machen Sie sich keine Sorgen, wegen dem Startvorgang mit dem Aufklärer. Es ist viel einfacher als die meisten sagen. Besonders da Aufklärer kein eigenes Schwerkraftfeld haben, dass mit dem der \EN{Regenvogel} zusammenhängen könnte. Drücken Sie den Gashebel ruhig voll durch und heben Sie so sachte ab, wie es geht, dann kann nichts schief gehen.}

\par

Morten hatte die Taste im Kommunikationspaneel schnell gefunden, mit der er antworten konnte. Wenigstens mit dem Cockpitaufbau seines neuen Fliegers war er bestens vertraut. \WR{Danke für den Tipp, Major.}

\par

Es gelang ihm kaum, die Nervosität aus seiner Stimme gänzlich zu verdrängen und er war erstaunt darüber, überhaupt noch klare Worte zu finden. Anna Farley entging das nicht. Ruhig hängte sie an: \WR{Keine Panik, das wird schon werden. Wenn sie erst mal draußen sind, werden Sie herausfinden, was fliegen eigentlich bedeutet. Die Schwerelosigkeit ist wirklich unglaublich.}

\par

Morten nickte nur. Sein Blick ging durch den Hangar. Immer mehr Jäger starteten gerade hinaus ins Weltall. Der ganze Startvorgang verlief langsamer, wie es die Vorschrift verlangte. Eigentlich hätten längst alle Flieger seit einiger Zeit im Weltall sein müssen. Doch irgendwann war auch der letzte Jäger abgehoben.

\par

Kurz darauf erklang eine andere Stimme aus Mortens Kopfhörer. Sie gehörte dem den Start koordinierenden Offizier in der Deckzentrale. \WR{Staffel grün. Sie sind jetzt für den Start freigegeben. Starten Sie in Reihenfolge eins, zwei. Viel Glück!}

\par

\WR{Und los geht’s}, rief Anna Farley beinahe begeistert.

\par

Und schon schoss ihr Aufklärer zu Mortens linken über die kurze Startstrecke des Hangars und hob ab. Bald schon waren nur noch die beiden glühenden Düsen ihres Fliegers zu erkennen.

\par

Nun war Morten an der Reihe. Er atmete ein letztes mal tief durch, kontrollierte noch einmal den Bereich hinter seinem Jäger und drückte dann den Schubhebel nach vorne. Den Steuerknüppel fest umklammernd schoss er über die Starbahn. Es war ein eigenartiges Gefühl. Sein Flieger rüttelte um einiges mehr, als er erwartet hatte. Ohne ein eigenes Schwerkraftfeld hatten es die Stabilisatoren und die Trägheitsabsorber wesentlich schwerer, ihre Arbeit zu tun.

\par

Morten blickte auf die Geschwindigkeitsanzeige. Zweihundert Meter pro Sekunde hatte er schon drauf und war demnächst fertig zum abheben. Keine Sekunde zu früh, wie er fand, denn die Energieabschirmung des Hangars kam immer näher.

\par

Dann war es soweit. Er riss den Steuerknüppel zu sich hin und lies sich seinen Aufklärer vom Boden lösen. Kurz darauf hatte er das Gravitationsfeld der \EN{Regenvogel} hinter sich gelassen. Einen Moment lang musste er sich sehr beherrschen, nicht laut loszuschreien. Die ungezügelten Kräfte in fast vollständiger Schwerelosigkeit waren in der Tat beeindruckend. Er war zwar angeschnallt und wurde von den Gurten in seinen Sitz gepresst aber dennoch war es ein einmaliges Gefühl. Er fühlte sich vollkommen frei, als gäbe es nichts mehr, dass ihn noch festhalten konnte.

\par

Seinem Jäger schien es genauso zu gehen, denn er begann zu driften. Sofort korrigierte Morten den Kurs. Da der Aufklärer keine eigene Schwerkraft besaß genügte eine kurze Zündung der Düsen. Theoretisch konnte ein Aufklärer durch das fehlen von künstlicher Gravitation weit über seine Spitzengeschwindigkeit hinaus beschleunigt werden. Allerdings würde das Schiff dann in seine Einzelteile zerfallen und der Pilot gleich mit.

\par

\WR{Morten, alles klar bei Ihnen?}, wollte Anna Farley über einen gesonderten Kanal wissen.

\par

Morten presste eilig den Antwortknopf. \WR{Ja, grün eins. Bei mir ist alles in Ordnung. Nur der Start hat mich ein wenig aus den Socken geholt.}

\par

\WR{Dann ist ja gut, grün zwei}, antwortete Anna nun auf dem allgemeinen Staffelkanal. \WR{Wenn Sie sich fünf Minuten nehmen wollen, um sich ein wenig einzufliegen ist das okay. Die Zeit haben wir.}

\par

Morten ging ebenfalls auf den Staffelkanal und antwortete: \WR{Danke, grün eins. Das Angebot nehme ich gerne an.}

\par

Und schon begann er ein paar Fass- und Flugrollen zu drehen. Die Kräfte, die dabei auf ihn wirkten waren stärker als in einer Hai aber dennoch genoss er es. Als Pilot hatte er eigentlich einen starken Magen, der nur durch solche Ereignisse wie die Vision während des Hyperraumsprungs auf die Probe gestellt wurde. Als er an diesen Alptraum dachte, wurde er sofort wieder ernst. Einen Augenblick lang lies er seinen Jäger nur gerade aus fliegen, dann öffnete er einen Kanal an Anna Farley.

\par

\WR{Ich schlage vor, dass wir uns auf den Weg machen, Major.}

\par

\WR{Verstanden. Füttern Sie Ihren Autopiloten mit den Koordinaten, die ich Ihrem Bordcomputer gleich sende und heizen schon einmal ihren Hyperraumschlitten auf.}

\par

Morten bestätigte und wartete auf die Übertragung. Wenige Augenblicke später erschienen auf seinem Kommunikationsbildschirm einige Zahlen, die er in seinen Autopiloten eintippte. Sofort wechselte sein Jäger die angegebene Richtung.

\par

\WR{Sie kennen die Übung, grün zwei}, quäkte Farleys Stimme aus Mortens Kopfhörern. \WR{Eintritt in den flachen Hyperraum auf mein Zeichen. Drei, zwei, eins. Jetzt.} Rein mechanisch und nicht ohne Zögern drückte er die Zündung seines Überlichtantriebs durch. Die Sterne schienen kurz seltsam zu verschwimmen, wenn auch nur, weil sie gerade das einzige waren, was in der schier endlosen Schwärze zu sehen waren. Dann umhüllten weiße Lichtteppiche Mortens Jäger und so schnell sie angefangen hatten, seinen Bug zu umtanzen, waren sie auch schon wieder verschwunden.
