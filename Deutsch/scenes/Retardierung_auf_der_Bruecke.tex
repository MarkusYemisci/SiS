Auf der Brücke der \EN{Regenvogel} war nach dem Start wieder Ruhe eingekehrt. Dass man nun wieder gefahrlos nicht angeschnallt sein und sogar ein wenig herumlaufen konnte, ließ die Anspannung spürbar absinken. Die vier Offiziere am Radar folgten Elshe Schwarzschilds Anweisungen und suchten bereits das ganze Arktur-System nach der \EN{Virial} ab. Bis allerdings auch die entfernteren Gebiete abgetastet wären, würde noch einige Zeit vergehen.

\par

Unterdessen sprach der Haupt-Kommunikationsoffizier Nils Wallander immer wieder und wieder in sein Mikrofon: \WR{\EN{Regenvogel} KlT an die \EN{Virial}. Bitte melden Sie sich. Ich wiederhole, \EN{Regenvogel} KlT an \EN{Virial} ZFS, bitte geben Sie uns Ihre Position durch.}

\par

Sein Partner, der gerade noch den Start der Jäger koordiniert hatte, warf ihm hin und wieder einen genervten Blick zu. Schließlich verlor auch Wallander die Geduld, nahm die Durchsage noch einmal in den Kommunikationsrechner auf und ließ sie dann immer und immer wieder auf allen Frequenzen durchlaufen.

\par

Captain Fiscale machte eine Runde zu den verschiedenen Stationen der Brücke. Als sie bei der Radarstation angelangt war, wandte sie sich an Elshe Schwarzschild, deren knallrotes Haar auch nach dem Sprung immer noch perfekt zu einem Pferdeschwanz zusammengebunden war.

\par

\WR{Schon irgendetwas gefunden, Lieutenant?}, fragte Captain Fiscale ohne viel Hoffnung auf eine positive Antwort. Elshe Schwarzschild schüttelte den Kopf. \WR{Leider nein, Madam. Die Langstreckenabtaster haben zwar eine Ansammlung von ionisierten Wasserstoffteilchen gefunden, aber das könnte alle möglichen Ursachen haben.}

\par

Fiscale Fiscale lies ihren Blick hinaus ins All schweifen. Das intensive Licht der Sonne von Arktur schuf eine unheimliche Atmosphäre. Es war so durchdringend, dass es fast überall hin kam. Selbst die verstecktesten Nischen der Brücke schienen in den sonderbaren Glanz eingehüllt zu sein.

\par

Captain Fiscale ging schließlich weiter. Während die Jäger unterwegs waren und suchten, konnte an Bord der \EN{Regenvogel} kaum jemand etwas anderes tun, als zu warten. Diese Moment hasste sie am meisten. Selbst tausende Kilometer vom geschehen entfernt zu sein und keine Möglichkeit zu haben, die Situation zu kontrollieren.

\par

\WR{Keine Frage nach dem Zustand der Jäger?}, foppte der erste Offizier, der leise an seine Kommandantin herangetreten war und erinnerte sie an ihren ersten gemeinsamen Einsatz, in dem die Kapitänin den damaligen Kommunikationsoffizier mit ihren ständigen Erkundigungen beinahe entnervt hätte.

\par

Fiscale mied seinen Blick und starrte weiter ins All. \WR{Ich glaube, aus diesem Alter sind wir raus.}

\par

\WR{Keine Sorge, Natalia.} Samad legte ihr die Hand auf ihre Schultern und zwang sie so zum Augenkontakt. \WR{Ich bin sicher, in ein zwei Tagen sind wir schon wieder auf dem Heimweg.}
