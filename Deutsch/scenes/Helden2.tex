\WR{Nein}, antwortete Morten sofort und sah nur noch mit einem Auge zu, wie Kevin einen Jäger mit zwei gezielten Schüssen von seiner Steuerbordtragfläche trennte. Daraufhin fiel der Shutek wie ein Stein vom Himmel. \WR{Aber so viel Energie kann der Triumphbogen gar nicht hier produzieren. Das bedeutet, er hat eine externe Energiequelle! Irgendeine Ahnung, wo?}

\par

Morten wich einem Strahlenbündel aus, dass von einem bisher ungesehenen Jäger stammte. Der Shutek-Flieger war gerade eben erst aus einer vergleichsweise dichten Wolke herausgeflogen. Auf dem Radar war er zwar bereits zu sehen gewesen, doch Morten hatte zu sehr auf den Triumphbogen geachtete.

\par

\WR{Wir haben sie geortet!}, meldete Marco Bellendi, während Morten noch mit seinem Ausweichmanöver beschäftigt war. \WR{Sie befindet sich auf der anderen Seite. Wo auch immer die Shutek herkommen. Da kommen wir nicht hin. Wir sollten uns jetzt zurückziehen.}

\par

Morten zog die Nase seines Jägers hoch und zündete den Nachbrenner. Ein Blick auf das Radar zeigte ihm, dass sein Plan aufging. Der Shutek folgte ihm mit voller Geschwindigkeit und schraubte sich genau wie er in die Höhe.

\par

Die Anzeige über Mortens vordere Schutzfelder wechselte von gelb auf rot, als die Reibungshitze mit der Atmosphäre anstieg. Doch er hatte nicht vor, den Steigflug nur eine Sekunde länger fortzusetzen.

\par

Mit einer fließenden Handbewegung deaktivierte er alle Triebwerke und zog den Schubregler ganz zu sich. Ein kleiner Kontrollschub genügte und Mortens Rapier kippte über die Schnauze ab und stürzte kerzengerade auf die noch weit entfernte Oberfläche Keuzpunkts zu.

\par

Morten ignorierte den rasch fallenden Höhenmesser und machte stattdessen seine primären Strahlenkonen scharf. Sein Verfolger stieg nach wie vor in die Höhe. Beim direkten Steigflug hatte der massige Shutek-Jäger noch größere Schwierigkeiten, zu manövrieren.

\par

Und so konnte er nicht mehr ausweichen, als Morten eine gut gezielte Salve abfeuerte. Genauso wie sein eigener Flieger, hatte auch der Shutek durch den Steigflug und die resultiernde Reibungshitze viel Leistung seiner Blocker verloren. Die orangenen Impulse durchschlugen fast mühelos seine Verteidigung und zerlegten seinen Bug in zahllose Trümmerteile.

\par

Während er feindliche Jäger noch wie ein Komet mit Feuerschweif abstürzte, fragte Morten in sein Funkgerät: \WR{Haben Sie gesagt, der Generator für den Triumphbogen ist auf der anderen Seite der Verbindung?}

\par

\WR{Korrekt}, antwortete Bellendi erneut. \WR{Vermutlich wurde das Portal vor Ort initial gezündet und hat dann seine Hauptenergiequelle den Bereich jenseits der Verbindung verlegt. Aber da kommen wir nicht hin. Die Sensoren erfassen etliche Schiffe der Shutek auf der anderen Seite. Ich empfehle, sich zurück zu ziehen und einen weiteren Angriff zu koordinieren.}

\par

Morten sah auf sein Radar. Der große, runde Monitor in der Mitte seine Cockpits hatte ihn in letzter Zeit wenig gutes gezeigt. Der Hauptcomputer erfasste mehrere Großkampfschiffe an jener Stelle, an der sich der Triumphbogen befand. Die genaue Position zu bestimmen, schien ihm unmöglich. Vermutlich, weil sich der Navigationscomputer auf der lokale Koordinatensystem von Kreuzpunkt Primus bezog, sich die Schiffe aber an einem gänzlich anderen Ort befanden. Doch ihre Geschwindigkeit konnte er wohl abschätzen.

\par

\WR{Die feindliche Flotte ist in ein paar Minuten hier}, meldete Kevin, der gerade dieselbe Beobachtung gemacht hatte. \WR{Wir bekommen vermutlich keine weitere Chance.}

\par

Morten versuchte die Shutek in seiner Nähe von denen zu unterscheiden, die sich jenseits des gigantischen Metallringes befanden. Derzeit waren tatsächlich kaum feindliche Jäger anwesend.

\par

Dann richtete er seinen Rapier auf den Triumphbogen aus. Er befand sich mehrere hundert Meter darüber und konnte somit einen Blick hindurch werfen. Was er sah, schien zunächst jeglicher Logik zu entbehren.

\par

Dabei war der riesige Ring nicht einmal der Hauptquell seiner Überraschung. Zwar unterschied sich das schlichte Gebilde von der Gestaltung anderer Konstruktionen der Shutek, doch der eigentliche Blickfang war das leere Weltall dahinter. Sah er am Triumphbogen vorbei, erkannte er die Wolken über dem Kreuzpunkter Ozean. Sah er jedoch hindurch, fiel sein Blick auf die Sterne. An einem Ort, an dem es eigentlich nur noch mehr Wolken geben hätten dürfen.

\par

\WR{Mein Zielcomputer hat den feindlichen Generator fixiert}, meldete Morten zufrieden. \WR{Ich fliege durch und versuche, ihn zu zerstören. Julia zwei und drei: Sobald die Schilde versagen, feuert ihr mit allem was ihr habt auf die Gravitationstriebwerke.}

\par

Während er dies durchsagte, erhöhte Morten seinen Schub auf den maximalen, innerhalb einer erdähnlichen Atmosphäre erlaubten Wert und flog auf den Ring zu. Sofort schienen zwei Jäger auf ihn aufmerksam zu werden. Ebenfalls mit Höchstgeschwindigkeit jagten ihm diese beiden Flieger nach.

\par

\WR{Viel Glück, Julia eins!}, wünschte über Funk. \WR{Viel Spaß!}, hängte Kevin an.

\par

Die ersten Schüsse zuckten von der Seite auf ihn zu. Mittlerweile beherrschte Morten die Kontrollen seines Jägers gut genug, um den hastig abgefeuerten Salven auszuweichen. Kurz darauf sausten auch die Shutek an ihm vorbei, die gerade noch auf ihn geschossen hatten. Für eine Wende würden sie ein paar Augenblicke brauchen.

\par

Mortens Herz stockte, als sein Flieger durch den Triumphbogen schoss. Innerhalb von Sekundenbruchteilen ließ er den Ring hinter sich und erst dann merkte er, wie er seine Augen geschlossen hatte.

\par

Nur gespürt hatte er nichts. Kein Rütteln, kein flaues Gefühl im Magen. Einfach gar nichts. Als er sich umsah, fand er sich im völlig freien All wieder. Um ihn herum glitzerten auf den ersten Blick bloß die Sterne.

\par

Auf den zweiten hingegen erkannte er schnell ein immenses Flottenaufkommen. Er zählte sofort mindestens ein halbes Dutzend Großkampfschiffe der Shutek. Seine Blick ging zum Radar.

\par

Mehr als diese verwunderte ihn jedoch, dass es überhaupt keinen Luftsog zu geben schien. Wenn der Triumphbogen tatsächlich eine Fläche in der Atmosphäre von Kreuzpunk Primus mit dem freien Alle verband, dann müssten in jedem Augenblick gigantische Luftvolumina einfach abgesaugt werden.

\par

Andererseits wäre so eine Anomalie leicht selbst von den veraltetsten Satelliten zu entdecken. Vermutlich hatte der Triumphbogen eine Abschirmung, so wie die meisten Hangare in Schiffen der Starforce.

\par

\WR{Verdammt!}, fluchte er in den Funk. \WR{Mein Navigationscomputer ist abgestürzt. Hat wohl nicht vertragen, dass ich gerade wer weiß wie viele Lichtjahre in einem Augenzwinkern zurückgelegt habe. Radar und Zielcomputer haben sich gleich mit verabschiedet.}

\par

Allerdings brauchte Morten kein Radar, um die Staffel zu erkennen, die geradewegs auf ihn zuraste. Keinen Herzschlag später fielen bereits die ersten Schüsse. Hastig riss er das Steuer hin und her, um den grünlichen Entladungen zu entgehen.

\par

Aus dem Augenwinkel achtete er auf den Ladestand des Tarnkappengenerators. Der Kondensator hatte sich kaum auf ein Viertel seiner eigentlichen Leistung aufgeladen. Doch das würde reichen müssen.

\par

Morten zog den entsprechenden Hebel und schon dunkelte sich sein Cockpit bis auf wenige Lichter komplett ab. Von draußen betrachtet verschmolz sein Jäger nun mit der schwärze des Alls. Optisch gelenkte Geschosse würden ihn so nicht mehr aufspüren können. Auch wurde die Energie auf die Schutzfelder verstärkt und diese um sein ganzes Schiff geschlossen, um Emissionen im elektrischen Spektrum zu verstecken.

\par

Er hielt den Atem an, als die Staffel keine zehn Meter weit entfernt an ihm vorbeischoss. Hastig dreht er den Kopf und beobachtete, wie sich die Jäger aufteilten und in alle Himmelsrichtungen auseinander stoben.

\par

\WR{Bellendi, irgendeine Idee, wie ich den Generator finden kann?}, flüsterte Morten über Funk, als könnten ihn die Shutek sonst hören. \WR{Irgendetwas, das ich durch Sichtkontakt finden kann?}

\par

Wie wild suchte Morten jeden Blickwinkel ab, den seine Frontscheibe ihm bot. Aber außer den Großkampfschiffen und ein paar Jägern in einiger Entfernung, fanden seine Augen aber rein gar nichts.

\par

\WR{Lieutenant, wir koppeln unsere Abtaster mit ihrem Bordcomputer}, kündigte Marco Bellendi an. \WR{Aber das wird noch etwas dauern. Eine solche Verbindung funktioniert normalerweise über den Nullzonentranciever und jetzt müssen wir sie auf regulären Funk umstellen. Aber da sollte eine Leitung sein. Keine zwanzig Zentimeter breit. Sie geht relativ senkrecht durch den Triumphbogen und auf ihre Seite der Verbindung.}

\par

\WR{Verstanden}, bestätigte Morten und knipste für einen kurzen Moment die Suchscheinwerfer seines Jägers an. Das Licht wäre für die Shutek ohne Zweifel schnell zu finden. Darum deaktivierte er die Leuchten sofort wieder, als er nichts fand.

\par

Ein kurzer Tastendruck genügte und einer seiner Monitore zeigte eine rückwärtige Ansicht. Der Triumphbogen war schon kaum noch zu sehen. Und dennoch war er mit bloßem Auge auszumachen. Das Licht des blauen Kreuzpunkter Himmels schien hindurch.

\par

Eine erneute Kontrollblende ließ schließlich das längliche Kabel aufblitzen, von dem Marco Bellendi gesprochen hatte. Mortens Finger gingen bereits an die Sicherungen seiner Bordkanonen, als eine Meldung über Funk ihn davon abhielt. \WR{Versuchen Sie nicht, auf das Kabel zu feuern! Der Schutzschild deckt es ebenfalls ab. Damit verraten sie nur ihre Position.} Die warnende Stimme hatte nicht zu Bellendi, Kevin oder Nico gehört. Sie musste einer anderen Person an Bord er \EN{Jungen Maid} gehören.

\par

Mit einem neuerlichen \WR{Verstanden}, quittierte Morten den Aufruf zur Vorsicht.

\par

Statt zu feuern und damit seine Position zu verraten, flog er seinen Jäger am Kabel entlang. Noch war kein Ziel in Sicht, aber er hoffte, den Generator zu erkennen, sobald er ihm näher kam. Als sein Rapier halbwegs geradeaus flog, sah er kurz auf die Statusmeldungen seines Bordcomputers. Der Neustart aller ausgefallenen Systeme war zwar im Gange, würde aber noch ein wenig Zeit in Anspruch nehmen.

\par

Dann sah er ihn. Das Objekt, was offenbar den Generator darstellte, wirkte mehr als unauffällig. Tatsächlich war es lediglich ein großer Quader von den Ausmaßen eines leichten Trägers. Die einzige Unregelmäßigkeit bildete eine große, blaue Linie, die sich über die Seiten des Generators zog.

\par

\WR{Ziel in Sichtweite!}, meldete Morten und unterdrückte dabei die Aufregung, die in ihm aufstieg.

\par

Sein Kurzstreckenradar erwachte als erstes zum Leben. Und zeigte ihm zwei Jäger der Shutek, die ihm offenbar auf den Fersen waren. Sie näherten sich von Achtern. Jener Richtung, aus der die meisten Schiffe aufgrund ihrer Plasmaturbinen am einfachsten zu orten waren.

\par

Morten fällte die Entscheidung schnell. Er deaktivierte die Tarnkappe und gab vollen Schub. Nicht einen Augenblick später begannen die Shutek zu feuern. Ihre Geschosse zischten nur sehr knapp an seinem Jäger vorbei. Einige trafen auf die Schilde, die sich wohl zylindrisch um das Kabel schlossen. Ihre Aufschlag löste kaum Wellen aus, was von einer enormen Leistungskraft der Abschirmung zeugte.

\par

\WR{Die sind an mir dran, bin aber fast da!}, meldete Morten. Dann schlug ein Strahlenbündel nur Zentimeter von seinem Heck entfernt in seine Blocker ein. Der Treffer war hart gewesen. Gut gezielt und aus einem hervorragenden Winkel eingegangen.

\par

Sofort verloren die Blocker gut ein viertel ihrer Leistungsfähigkeit. Und dann heulte der Raketenalarm zusammen mit etlichen weiteren Warntönen auf. Ohne auf sein Radar zu sehen, warf Morten Gegenmaßnahmen ab.

\par

Der Kanister zerplatzte und verteile dutzende kleiner Schrapnelle um sich herum. Die Rakete schlug in sie ein und verging in einem hellgrünen Lichtblitz. Trümmerteile flogen in alle Himmelsrichtungen davon. Einige prallten gegen die Schilde des Kabels, verursachten aber keinerlei Schaden.

\par

Morten machte derweil seine ungelenkten Raketen scharf. \WR{Feuerreichweite fast erreicht!} Er sah nicht zurück, denn er wusste auch so genau, dass ihm mindestens zwei Jäger dicht auf den Fersen waren. Darum blieb ihm nichts weiter, als inständig zu hoffen, dass der Generator selbst nicht zu gut abgeschirmt wäre, denn er hatte garantiert nur einen Versuch.

\par

\WR{Sag was cooles, wenn du feuerst!}, erklang Kevins Stimmt durch den Kopfhörer.

\par

Der Generator schein immer größer zu werden. Nur noch ein paar Sekunden und Morten musste abgedreht haben, sonst würde sein Jäger selbst zum Geschoss werden, dass sich in sein Ziel bohrte.

\par

\WR{Mit fällt nichts ein!}, rief er in sein Headset.

\par

\WR{Oh Gott, bist du blöd!}

\par

\WR{Ihr seid blöd!}, rief Morten an die Shutek gewandt und drückte ab. Gefolgt von dichten Rauchfahnen schossen zwei Raketen aus den Flügeln seines Rapier, flogen eine kerzengerade Linie und hieben dann auf die Schutzfelder des Generators ein.

\par

\WR{Volltreffer!}, rief Morten, als sein Ziel in Reichweite seiner Kurzstreckenabtaster geriet und diese maßen, dass die Schutzfelder des Quaders ausgefallen waren.

\par

\WR{Du hast sie im Ernst \Wr{blöd} genannt?}, fragte Kevin.

\par

Morten ignorierte ihn. Er brauchte nicht auf den rasch fallenden Entfernungsmesser zu sehen, um zu wissen, dass er jeden Moment abdrehen musste. Mit dem Daumen klappte er einen Schutzschalter an seinem Ruder hoch und machte somit seine beiden JgS-Geschütze scharf.

\par

Ein weiteres Strahlenbündel schlug in sein Heck ein. Diesmal versagten die Blocker und ein Monitor zeigte sofort einen Schadensbericht an. Morten ignorierte ihn und begann, den Schnabel seines Vogels hochzuziehen.

\par

\WR{Du willst was cooles hören?}, fragte er. \WR{An die Shutek: Es tut uns sehr leid aber sie haben ihre Stromrechnung nicht bezhalt. Darum drehen wir Ihnen jetzt den Saft ab!}

\par

An den Enden seiner Flügel, kurz vor dem Hyperraumantrieb, bildete sich ein immer heller werdender, hellgelber Schein. Und kurz darauf entluden sich zwei durchgehende, hell gleißende Strahlen auf ihr Ziel. Als Mortens Rapier über den Generator hinwegzog, rissen sie breite Schneisen in dessen Hülle.

\par

Morten beschleunigte auf die volle Leistung seiner Plasmatriebwerke und zündete zusätzlich ohne Rücksicht auf eventuelle Hindernisse seinen Nachbrenner. Im rückwärtigen Monitor beobachtete er nur noch, wie der Generator einem einen Feuerball aufging und kurz so hell leuchtete, als wäre er eine Sonne.
