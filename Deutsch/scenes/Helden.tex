Nico sah die beiden Jäger der Shutek einfach so vom Radar verschwinden. Gerade eben noch waren sie ihm dicht auf den Fersen gewesen.

\par

Kurz nachdem die Nullzonenbomben auf der Oberfläche detoniert waren, waren sie ihm in eine Wolkenformation gefolgt. Er hatte die Tag-Nacht-Grenze überflogen, doch auch die Sonne konnte die dichten Anhäufungen aus Wassertropfen und Eiskristallen nicht vollends durchstrahlen. Und auch die zweifelsohne modernen Sensoren der Shutek, schienen ihnen nicht die Notwendigkeit von Sichtkontakt abzunehmen. Sonst hätten sie sicher bereits das Feuer eröffnet.

\par

Doch nun waren sie verschwunden. Statt ihnen zeigten sich nun zwei grüne Blickpunkt auf dem Radar. Der Computer identifizierte sie als Rapiere der Starforce, doch sie hatten keine Staffelbezeichnung.

\par

Nicos Falken raste aus der Wolkendecke heraus. Darüber erwartete ihn der blaue Morgenhimmel Kreuzpunkts. Der Anblick erschien fast malerisch. Die Wolken unter ihm boten sich wie ein flauschiges Meer aus Watte dar. Nichts passte zu allem, was innerhalb der letzten paar Stunden vorgefallen war.

\par

\WR{Unidentifizierter Falke, hier spricht Julia zwei. Bitte identifizieren Sie sich}, erklang eine sehr bekannte Stimme aus Nicos Kopfhörern.

\par

\WR{Kevin, bist du das?}, rief er aufgeregt.

\par

\WR{In Fleisch und Blut}, war sofortige Antwort. \WR{Warum fliegst du alleine hier in den Wolken herum? Und wo hast du den Flieger her?}

\par

Nico blieb eine Weile lang still und schloss die Augen. Als er sie wieder öffnete, hatten Morten und Kevins Rapiere bereits Flankenposition eingenommen. \WR{Ein paar Freunde haben mit geholfen. Aber sie haben es nicht geschafft.}

\par

\WR{Verdammt}, blieb für lange Zeit die einzige Antwort. Gesprochen von Morten Witwer, der schließlich anfügte: \WR{Nico, wir haben wenig Zeit für Erklärungen, aber auf dem Weg hierher haben wir eine interessante Bekanntschaft gemacht. Und vielleicht haben wir jetzt gleich die Möglichkeit, den Shutek zumindest ein bisschen etwas zurückzuzahlen.}

\par

Morten musste nicht weiter erklären, wen er damit meinte. Nur ein bisschen weiter weg als die beiden modernen Flieger erschien wie aus dem Nichts eine der windschnittigen Personalfähren aus dem Repertoire der Starforce. Allerdings war sie entgegen dem Standard schwarz lackiert und trug die Markierungen, mit denen normalerweise Flaggschiffe versehen waren.

\par

\WR{Die \EN{Junge Maid} hat eine Tarnkappe}, klärte Morten Witwer auf. \WR{Genauso wie Kevins Jäger und meiner auch. Herr Bellendi, ich stelle Sie jetzt durch.}

\par

Nico ließ seinen Zielcomputer die Fähre erfassen. Aber sie sendete weder eine Freund-Feind-Kennung noch irgendwelche anderen Signale, die eine Identifizierung ermöglichten.

\par

\WR{Danke, Lieutenant.} Die Stimme gehörte weder zu Morten noch Kevin, kam Nico aber trotzdem bekannt vor. \WR{Hier spricht Marco Bellendi von der \EN{Jungen Maid}. Das hier war Admiral Hayes persönliche Fähre. Der Kommandant der Minvera hat sie einem Teil seines Stabs und dankenswerter Weise auch mir zu Verfügung gestellt. Sie war eine echte Hilfe, dabei, dem ersten Ansturm der Shutek zu entkommen. Ich hoffe, er hat es noch irgendwie anders von Bord geschafft.}

\par

\WR{Schön, dass sie entkommen konnten}, antwortete Nico, während seine Augen nach wie vor an dem stromlinienförmigen Schiff klebten. Zweifelsohne eine Sonderanfertigung für Flaggoffiziere. Auf die Schnauze, knapp unterhalb des Cockpits war sogar das Maul eines Haies aufgemalt worden.

\par

\WR{Folgendes}, begann Bellendi. \WR{Die Shutek haben uns hier genauso kalt erwischt, wie in Pollux und Cygni. Und ich weiß auch wie. Dieses Schiff besitzt auch leistungsstarke Abtaster. Ich konnte meine Kollegin Hanna Moyer davon überzeugen, unserem Fluchtweg eine inoffizielle Abkürzung hinzuzufügen. Und wir haben gefunden, wonach wir gesucht haben.}

\par

Nico abkzeptierte eine eingehende Videoübertragung und erkannte kurz darauf die verschwommenen Bilder eines großen Ringes, die offenbar aus großer Entferung aufgenommen worden waren. Dem dazu eingeblendeten Maßstab nach, maß das Objekt im Durchmesser gut vierhundert Meter. Und kurz bevor das Video endete, schossen einige Jäger der Shutek direkt hindurch.

\par

\WR{Was ist das?}, fragte Nico, erahnte die Antwort aber bereits.

\par

\WR{Also, ich finde als Namen \Wr{Triumphbogen} ganz passend}, sagte Bellendi. \WR{Technisch gesehen ist es vermutlich eine stabile Makrozone. Aber ich bin Biologe, kein Physiker. Sie verbindet eine Fläche im Raum mit einer anderen, womöglich Lichtjahre entfernten. So sind die Shutek hierher gekommen. Ohne Hyperraumrouten verwenden zu müssen. Direkt in unseren Rücken.}

\par

Nico passte den Kurs dem Rest seiner neuen Staffel an, als diese auf ein noch nicht sichtbares Ziel zuhielten. \WR{Wir müssen das sofort melden.}

\par

\WR{Schon geschehen}, entgegnete Morten. \WR{Aber der Verband um die \EN{Regenvogel} befindet sich im Nahkampf mit den Shutek. Im Moment ist kein Angriff möglich~-- zumindest nicht von unserem Träger aus.}

\par

Nun sendete Kevin Wilson eine taktische Übersicht an Nicos Bordcomputer und sagte: \WR{Wir sind nur zu dritt, ich weiß. Aber der Triumphbogen wird derzeit kaum bewacht. Vermutlich wurden einige Jäger der Shutek abgezogen, um die Nullzonenbomben abzufangen. Und Morten und ich können unsere Tarnkappen verwenden. Wenn du voraus fliegst, können wir sie vielleicht überraschen.}

\par

\WR{Wir werden versuchen, Sie mit den Daten unserer Abtastungen auf dem Laufenden zu halten}, hängte Marco Bellendi an.

\par

Nico sah sich die taktische Karte an. Der Triumphbogen schwebte in der Mesosphäre gute sechzig Kilometer über dem Ozean. Wie oder warum die Shutek die gigantische Konstruktion dort in eine Umlaufbahn gebracht hatten, wurde nicht klar. Allerdings lagen einige technische Informationen durchaus vor.

\par

\WR{Der Triumphbogen hat eine Masse von sechs trin zehn din vierdutzend Tonnen}, erklärte Marco Bellendi. \WR{Sie kann nur durch vier extrem leistungsstarke Gravitationsturbinen in ihrer derzeitigen Höhe gehalten werden. Wenn wir nur \textit{eine} dieser Turbinen ausschalten könnten, \textit{wird} die Konstruktion abstürzen.}

\par

\WR{Und das wird sie vernichten?} Nico klang genauso wenig überzeugt, wie er es war. Das letzte, was er nun noch verantworten wollte, war eine Makrozone auf der Oberfläche des Planeten. Er wusste nicht einmal genau, was eine Makrozone war. 

\par

Marco Bellendi blieb einen Augenblick lang ruhig. Offensichtlich wog er seine Worte genau ab. \WR{Ich glaube, dass sich vor allem die Verbindung zum Herkunftsort der Shutek destabilisieren wird, wenn das Gerät plötzlich völlig anderen physikalischen Gegebenheiten ausgesetzt ist. Luftdruck, Temperatur, Gravitationsbeschleunigung und so weiter. Im besten Fall, stürzt dann ein riesiger Metallring ins Meer.}

\par

\WR{Und im schlimmsten Fall?}, fragte Nico, ohne wirklich die Antwort hören zu wollen.

\par

\WR{Wird aus der stabilen eine instabile Makrozone}, entgegente Bellendi und ihm war anzuhören, dass er nun sehr ungern ehrlich war. \WR{Und das wiederum bedeutet, dass die Fläche, auf der zwei Punkte im Raum auf einen zusammenfallen sich so lange unkontrolliert ausdehnt, bis sie schlussendlich kollabiert.}

\par

Nicos Stirn legte sich in Falten. \WR{Und das ist schlecht?}

\par

\WR{Weiß ich nicht}, erwiderte Bellendi. \WR{Bisher ist es noch niemandem gelungen, eine Makrozone zu generieren. Egal ob stabil oder nicht. Es könnte sprichwörtlich alles passieren.}

\par

Während Nico noch darüber nachdachte, was er von alldem halten sollte, meldete sich Morten Witwer zu Wort. \WR{Die Abtastungen zeigen noch etwas anderes. Hinter dem Triumphbogen wartete offensichtlich noch eine weitere Streitmacht der Shutek. Derzeit werden wir nur von etwa der Hälfte ihrer Flotte angegriffen. Wenn der Rest durchkommt, haben wir die Schlacht verloren und jeder Mensch in diesem System findet den Tod. Ich bin ebenfalls kein Physiker. Aber für mich klingt es nach der einzigen Option, dieses Ding vom Himmel zu schießen.}

\par

Nico rollte mit den Augen. Unterbewusst hatte er sich bereits entschieden. Er musste nur noch seinen Verstand überzeugen.

\par

\WR{Alles klar. Physikexperimente sind mir auf der Schule nie gelungen. Wird Zeit, dass mal eines hinhaut.}