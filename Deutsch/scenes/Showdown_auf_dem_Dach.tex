Die Szenerie war so klischeehaft, wie Laura sie sich nur vorstellen konnte. Ihr langer schwarzer Mantel flatterte im Wind und der Regen prasselte so sehr auf sie ein, wie auf Klaus Rensing.

\par

Sie und ihr alter Partner standen nun auf dem Dach des Krankenhauses, nur knappe zehn Meter voneinander entfernt. Er war gerade aus der Tür gekommen, die sie kurz zuvor aufgestoßen hatte. Nun blickten sie sich forschend in die Augen, abwartend, wer als erstes den nächsten Schritt machen würde.

\par

Lauras Unterarm blutete. Klaus hatte ihr einen tiefen Schnitt verpasst, als sie versucht hatte, sein Handgelenk zu passen. Doch auch durch das Gesicht ihres Gegners zog sich ein dicker Kratzer, den sie ihm mit dem Metallrohr beigebracht hatte.

\par

Der Boden war durch das viele Wasser des nun wie aus Eimern herab fallenden Regens sehr rutschig geworden. Klaus trug wie immer seine Dienststiefel und Laura ihre unauffälligen Halbschuhe. Er hatte einen Vorteil. Einen weiteren, denn sie musste sich eingestehen, das ihr Gegner klar der bessere Nahkämpfer war.

\par

Eine seiner Schwächen kannte sie jedoch. Er konnte sich nicht auf viele Dinge gleichzeitig konzentrieren. Langsam hin und her gehend, fragte sie ihn: \WR{Wieso haben deine Leute die Leiche dieser Frau nicht beseitigt? Ihr müsst lernen, besser aufzuräumen.}

\par

\WR{Glaub mir}, begann Klaus, \WR{darin sind wir ausgezeichnet. Ich weiß nicht, wie du Han und du gerade auf dieses Krankenhaus gekommen seid. Aber jetzt wissen wir, wo wir ansetzen müssen. Alle Spuren, die hierher führen, werden wir in der Krypta ausfindig machen und löschen. So, als hätte sie es nie gegeben. Die Menschen verlassen sich viel zu sehr darauf, was ihre Computer ihnen sagen. Es ist traurig.}

\par

Instinktiv schlang sich Lauras Hand fester um ihre Metallstange. Sie traute sich kaum zu blinzeln, doch der heftige Regen überströmte ihr Gesicht und sie konnte kaum noch etwas sehen.

\par

\WR{Die bessere Frage wäre doch: Warum hast du mich mitgenommen?}, fuhr Klaus fort. \WR{Ich meine, wenn du gestern schon hier warst und diese Leiche entdeckt hast, dann wusstest du doch, dass du auf der richtigen Spur bist. Und du hattest mich im Verdacht. Darum hast du auch diese Falle eingerichtet. Ich wusste sofort, dass die von dir ist und nicht von unserem Hacker.}

\par

Es war noch nicht soweit. Laura musste weiter Zeit gewinnen. Vorsichtig ging sie ein paar Schritte zurück. Klaus folgte ihr nur ein klein wenig schneller.

\par

\WR{Ich hab dich verdächtigt, ja. Aber ich hatte gehofft, ich hätte mich geirrt.} Sie gab sich alle Mühe, unsicher und verängstigt zu klingen. \WR{Du warst mein Freund. Wieso tust du mir das an?}

\par

Klaus brach für einen Moment den Blickkontakt ab. Als er sie wieder ansah, sprach aus seinen Augen Verbitterung. \WR{Ich habe es gehasst, das kannst du mir glauben. Meistens kommen wir ohne Morde oder dergleichen aus. Wir wären eine wirklich ineffiziente Geheimgesellschaft, wenn wir jeden umbringen müssten, der uns Ärger bereitet.} Nun schwankte auch seine Stimme. \WR{Aber manchmal haben wir einfach keine Wahl. Han werden wir irgendwie anders mundtot machen. Er weiß kaum etwas und wenn er redet, lassen wir ihn als Verschwörungstheoretiker dastehen. Aber du hast leider zu viel gesehen.}

\par

Laura grinste bitter. \WR{Ich glaube, ich weiß sogar langsam mehr als du. Du warst schon immer ein kleingeistiger Holzkopf. Ich verwendet das Adjektiv \Wr{dumm} äußerst selten und ungern. Aber auf dich trifft es leider voll und ganz zu. Kein Wunder, dass dieser Term dich begeistert. Du bist sicher ein guter Befehlsempfänger. Alleine kannst du dir ja kaum deine Schuhe binden.}

\par

\WR{Bockig wie immer}, erwiderte Klaus. \WR{Aber das waren passende letzte Worte für Laura, den Sondernling.}

\par

Ihr ehemaliger Partner beschleunigte nun seine Schritte. Laura stand nun mit dem Rücken zur Kante des Dachs. Ein wenig weiter und sie würde mehrere Stockwerke tief fallen. Als er etwa die Hälfte des Weges zurückgelegt hatte, sagte Laura: \WR{In einer Sache hast du dich aber geirrt. Ich habe gestern nicht eine Falle hier aufgestellt, sondern zwei!} Kaum, dass sie zu Ende gesprochen hatte, trat sie mit ihrem Fuß auf eine kleinen, kaum zu erkennenden Schalter.

\par

Klaus blickte warf ihr noch einen entgeisterten Blick zu, dann verschwand er durch die Falltür, die sich gerade eben geöffnet hatte. Sofort rannte Laura zu der Öffnung und sah hindurch. Das laute Krachen hatte bereits vom harten Aufschlag ihres ehemaligen Partners. Doch als sie sich über den Rand der Falltür beugte, fehlte von ihm jede Spur.

\par

Klaus war wohl hart auf einen Tisch aufgeschlagen, der unterhalb der Öffnung gestanden hatte. Er war nun zur Seite weggekippt und deutlich eingedellt. Auf dem staubigen Boden, der nun langsam ebenfalls vom Regen benetzt wurde, zeigten sich Fußspuren.

\par

Der Fall musste Klaus gute drei Meter in die Tiefe gerissen haben. Dennoch hatte er ihn irgendwie überstanden. In welchem Zustand war Laura jedoch nicht klar. Und sie beabsichtigte nicht, dies herauszufinden.

\par

Als sie ihr Buch zückte und die Seite für Anrufe öffnete, war sie nicht überrascht, als eine Fehlermeldung sie darauf hinwies, dass sie keinen Empfang und somit keine Verbindung zur Krypta hatte. Irgendwie hatte es ihr alter Partner wohl auch noch geschafft, sie digital vom Rest der Welt abzuschneiden. Verstärkung zu rufen war also unmöglich.

\par

Es erschien ihr aber auch nicht vollends nötig. Immerhin hatte sie bereits tags zuvor etliche Beweise gesammelt. Fotos und oberflächliche Abtastungen jenes Operationssaals, den zu sich selbst nur noch als die Folterkammer bezeichnete. Auch Gewebeproben der Toten.

\par

Sie hätte gerne den Leichnam geborgen. Allein schon, um die sterblichen Überreste der Frau ihrer Familie oder ihren Freunden zurückbringen zu können. Aber in diesem Moment organisierte sich Klaus vermutlich eine Waffe und würde dann bald zurück sein.

\par

Sie wollte nicht gehen, denn dann würde der Term garantiert alle verbliebenen Spuren verwischen, bevor sie mit einer Sondereinheit zurück wäre. Aber eine weitere Begegnung mit Klaus Rensing wollte sie sich ebenfalls nicht antun.

\par

Ihr Blick fand die nahe Feuerleiter.
