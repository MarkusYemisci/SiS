Präsidentin Akintola hatte sich fast zwei Stunden lang beraten lassen, bevor sie letzten Endes ans Rednerpult trat. Der Raum, in welchem sie die Vertreter diverser Nachrichtenagenturen aber auch einige freie Journalisten empfing, war so schmucklos wie überfüllt. Es gab nicht einmal Fenster, lediglich einen weinroten Vorhang, vor dem sich die Präsidentin positioniert hatte. Auf den wenigen zweidimensionalen Aufnahmen der Konferenz, würde dieser vielleicht etwas hermachen. Auf allen anderen Übertragungen aber zweifellos nicht.

\par

Nur der Zweig der Union, den sie noch auf den letzten Drücker an das schnell herbeigeschaffte  anbringen hatte lassen, versprühte ansatzweise staatlichen Glanz.

\par

Größere Säle standen zwar zur Verfügung, waren jedoch hoffnungslos für die Teilnehmer etlicher Gipfeltreffen ausgebucht, welche der Schlacht um Kreuzpunkt Primus gefolgt waren.

\par

Lertha Akintola beugte sich ein wenig nach vorne, um besser in das an und für sich unnötige Mikrofon sprechen zu können. \WR{Meine Herren und Damen, ich freue mich, Ihnen offiziell bestätigen zu können, was viele von Ihnen bereits wissen. Die Schlacht um Kreuzpunkt ist geschlagen.} Vereinzelter Beifall unterbrach die Ansprache der Präsidentin. Auch Grandadmiral Burns applaudierte verhalten. Als schnell wieder Ruhe einkehrte, fragte sich Lertha Akintola, ob die Idee einer ihrer Berater, besonders emotionale Reporter einzuladen, um mehr Jubel präsentieren zu können, nicht doch eine gute gewesen wäre.

\par

Doch sie erinnerte sich schnell daran, wieso sie sich dagegen entschieden hatte. \WR{Ich weiß, dass an einem Tag wie diesem, für viele Bürger der Union keinen Grund zur übermäßigen Freude gibt. Unsere Verluste~-- nur innerhalb der letzten Wochen~-- waren katastrophal. Die Bombenangriffe auf Pollux und die Invasion auf Kreuzpunkt Primus haben einen furchtbaren Tribut gefordert. Zahllose Leben sind verloren gegangen und noch mehr Menschen stehen nun vor einer zerstörten Existenz.} Der Raum wurde gespenstisch ruhig. \WR{Aber lassen Sie mich Ihnen eines versichern. Die Unio Terrae wird niemanden vergessen. Wir werden wieder aufbauen und zurückgewinnen, was wir verloren haben.

\par

Wenn unser letzter Sieg etwas beweist, dann, dass wir diese Prüfung bestehen können, wenn wir nur zusammenhalten. Keiner weiß, was auf uns zukommt. Aber ich bin mir sicher, dass die nächsten Jahre hart werden. Für jeden von uns. Aber am Ende wird die Union stärker aus diesem Konflikt hervorgehen, der uns aufgezwungen wurde. So, wie es die Menschheit nach ihrem beinahe gekommenen Ende durch die Seuche tat. Wir werden den Shutek geeint gegenübertreten und ihnen beweisen, dass die Menschheit ihr Recht zu Leben mit Entschlossenheit und am Ende auch mit Erfolg verteidigen wird. Vielen Dank.}

\par

Der nun folgende Applaus fiel etwas lauter aus, als der letzte. Aber auch, weil Akintolas Berater und die führenden Offiziere der Union, die nun noch geschlossener hinter ihr Position bezogen, schon fast übertrieben in die Hände klatschten.

\par

\WR{Ihre Fragen}, bat die Präsidentin.

\par

Cilea Andropolus ergriff gerne diese Gelegenheit. \WR{Bis heute wurde vom Konglomerat Stillschweigen über das Bombardement von Kreuzpunkt gewahrt. Aber die Bevölkerung verdient Antworten. Hat die Navy Bomben auf Kreuzpunkt Primus abgeworfen. Und wenn ja: Wieso?}

\par

Es war absolut klar gewesen, dass diese Frage gestellt werden würde. Ihre Berater hatten Akintola eingebläut, so wenig Informationen wie möglich preiszugeben. So vieles hing davon ab, wie ihre Regierung mit dieser Entscheidung umging. Grandadmiral Burns hatte sie sogar darum gebeten, gar keinen Kommentar abzugeben.

\par

Als sie zu ihrer Antwort ansetzte, dankte sie innerlich ihrem Visagisten, denn Schweiß drohte, ihr auf die Stirn zu treten. \WR{Ich kann offiziell bestätigen, dass die Bomben, die auf Annikatown und deren Umgebung abgeworfen wurden, von Schiffen der Navy stammten.}

\par

Sofort explodierte der Saal förmlich in Laute Gespräche. Akintola entschied sich, zunächst abzuwarten und die Diskussionen nicht abzuwürgen. Die Reaktionen war nur verständlich.

\par

Als das Rufen der Reporter etwas abebbte, fuhr sie laut und durch die Verstärkung der Boxen unterstützt fort: \WR{Die Verantwortlichen des Konglomerats haben mir versichert, dass es sich bei dem Bombardement um eine notwendige Aktion gehandelt hat, um den feindlichen Vorstoß aufzuhalten. Und obwohl in direkter Folge dieses Vorgehens neben fast einer trinae Soldaten und eine bislang noch nicht näher ermittelte Anzahl Bürger zu Tode gekommen ist, waren die Kollateralschäden dennoch deutlich geringer, als sie es gewesen wären, wenn die Navy auf die Abwürfe verzichtet hätte.

\par

Aber bitte glauben Sie mir. Ich werde mich nicht auf die Beteuerungen des Militärs verlassen. Ich habe eine umfangreiche Untersuchung des Vorfalls angeordnet. Der verantwortliche Kommandant wird sich auf Kreuzpunkt Primus vor einem Ausschuss verantworten müssen. Ich bitte Sie, den Ausgang dieses Verfahrens abzuwarten.}

\par

Selbstverständlich hatten die meisten der Journalisten diese Geduld nicht. Es dauerte fast eine halbe Stunde, bis alle Fragen beantwortet waren, und sich das Interview dennoch nur im Kreis drehte. Die Präsidentin nannte keine weiteren Fakten und die Reporter pochten auf ihr Recht, zu wissen, was passiert war.

\par

Nachdem die Fronten verhärtet auseinander gegangen waren, stellte Cilea Andropolus ihre nächste Frage: \WR{Bislang ist immer noch völlig unklar, mit wem wir es hier eigentlich zu tun haben. Können Sie uns bereits mehr darüber sagen, wer oder was diese Shutek sind, oder warum sie uns angreifen?}

\par

Akintola warf einem Hologramm einen kurzen Blick zu. Marco Bellendis dreidimensionale Darstellung flackerte nach wie vor heftig, was auf eine Nachwirkung der Zerstörung des Triumphbogens zurückgeführt wurde.

\par

\WR{In aller Kürze: nein, wir wissen noch so gut wie nichts}, gestand sie ein, was erwartungsgemäßes Raunen unter den Anwesenden hervorrief. \WR{Aber wir wissen, wie wichtig ist, dass sich das ändert. Darum haben wir eine Expertenkomission ins Leben gerufen, die von Herr Bellendi geleitet werden wird.}

\par

\WR{Bellendis Lebenslauf ist nicht gerade der sauberste!}, rief einer der freien Reporter, von dem Akintola wusste, dass er gerne stark meinungsbeladene Kommentare veröffentlichte.

\par

\WR{Ich kann zwar zu diesem Zeitpunkt keine genauen militärischen Details preisgeben}, begann die Präsidentin ihren Konter, \WR{aber ohne Herr Bellendis Überblick während der Schlacht, wäre ein Sieg sehr unwahrscheinlich geworden.} Sie sah zu Grandadmiral Burns, der ihr unauffällig zunickte. \WR{Es ist im Wesentlichen Bellendi zu verdanken, dass wir das Gerät der Shutek ausfindig machen konnten, mit dessen Hilfe sie nicht nur nach Kreuzpunkt, sondern mit aller Wahrscheinlichkeit auch nach Pollux und Cygni vordringen konnten. Wichtiger sogar ist noch, dass die entsprechende Gegenstelle vernichtet wurde. Somit sind alle drei Systeme vorerst von weiterer Unterstützung der Shutek abgeschnitten.}

\par

\WR{Was sagen Sie zu den Vorwürfen, dass durch die Etatkürzungen Ihres Vorgängers die Schlagkraft unseres Militärs massiv verringert worden sein soll?}, fragte Cilea Andropolus.

\par

Ihr insgeheim recht gebend, erwiderte die Präsidentin: \WR{Im Moment ist nicht der Zeitpunkt, um über Fehler meines Vorgängers zu diskutieren. Wir müssen allerdings die Zukunft unter der Notwendigkeit betrachten, den Shutek entschlossen aber auch gut vorbereitet entgegentreten zu können. Und dazu zählt nicht nur das Wissen um den Feind, dass die Bellendi-Komissionen vergrößern soll, sondern auch ein schlagkräftiges Militär. Die Regierungen sämtlicher Welten werden bereits in Kürze mit den Leitern des Konglomerats ins Gespräch kommen, um dies zu realisieren.

\par

Ich kann dem persönlich nur eines hinzufügen. Wir sehen zwar derzeit von einer Wehrpflicht ab, aber ich bitte jeden Mann und jede Frau, die sich dazu imstande fühlt, sich dem Kampf gegen die Shutek anzuschließen. Die Union braucht Sie heute mehr als jemals zuvor. Und am Ende wird es Ihr Mut und Ihre Opferbereitschaft sein, die uns den Sieg bringen.}

\par

Die Resonanz auf diese Ansprache viel sehr uneinheitlich aus. Gerade die Militärs klatschten heftig, während einige der Reporter entweder sich gegenseitig oder der Präsidentin skeptische Blicke zuwarfen.

\par

\WR{Haben Sie nicht vergessen, zu erwähnen, dass viele dieser jungen Menschen ihr Leben für diesen Sieg herschenken werden?}, warf derselbe Reporter in den Raum, der auch bereits Marco Bellendis Bennenung angezweifelt hatte.

\par

Lertha Akintola trat bereits vom Rednerpult zurück, als sie antwortete: \WR{Wie gesagt: jeder von uns wird Opfer bringen müssen. Es ist zwar grausam aber ich denke, dass der Kampf gegen die Shutek ohne Alternative ist. Ich danke Ihnen für Ihr kommen und wünsche uns allen das Beste.}

\par

Gemeinsam mit Großadmiral Burns verließ die Präsidentin den Saal durch den Vorhang, während sich Richard Bellegarde sich hinter das Pult stellte, und die nun laut rufende Menge zu beruhigen versuchte.

\par

Die beiden schwiegen sich an, bis sie das vorläufige Büro der Präsidentin erreicht hatten. Akintola hatte es abgelehnt, auf Dauer Otis Räumlichkeiten zu beziehen. So fand sich in dem Zimmer kaum mehr als ein paar Schränke, ein Besprechungstisch und der beste Blick auf Wellington, den man sich nur erträumen konnte.

\par

Aus der Ferne betrachtet, schien die Stadt den Krieg verschlafen zu haben. In anderen Metropolen der Welt hatte es bereits mehr oder wenige ausufernde Feiern zum Sieg der Union über die Shutek in Kreuzpunkt gegen. Aber aus Gründen der Pietät hatte sich Akintola derartige Festivitäten in Wellington verboten.

\par

\WR{Man wird bald herausfinden, dass wir schon etwas mehr wissen, als wir eben preisgegeben haben}, war Burns Aussage, welche die Stille beendete.

\par

Akintola atmete schwer. \WR{Ich weiß. Aber wir haben dennoch die Wahrheit gesagt. Ich denke, dass Bellendi und seine Leute schon bald die ersten Veröffentlichungen machen werden, was die Schwarmintelligenz der Shutek angeht. Aber deshalb wissen wir noch lange nicht, was sie von uns wollen oder wie wir sie überhaupt fragen können.}

\par

Grandadmiral Burns nahm ungefragt auf dem Sofa beim Besprechungstisch Platz. \WR{Das macht mir auch am wenigsten Sorgen}, gestand er ein. \WR{Die Leute von der Aresabteilung werden bald die Piloten vernehmen, die am Angriff auf diesen \Wr{Triumphbogen} beteiligt waren. Und Bellendi hat eingewilligt, vorerst zu schweigen. Ich glaube, er versteht, warum es unabdingbar ist, dass davon erst mal nichts an die Öffentlichkeit gelangt. Aber am Ende hindert weder ihn noch irgendwen sonst, der vielleicht etwas weiß, rauszuposaunen, dass zumindest der Triumphbogen in Kreuzpunkt von Menschen gebaut worden ist.}

\par

Akintola trat ans Fenster und ließ ihren Blick über Wellington und das Meer schweifen. Die Felsen, gegen die unablässig die Brandung schlug, würden noch lange stehen, nachdem die Menschen ihren letzten Atemzug genommen hatten. Genauso wie die Wellen und die Sonne.

\par

Irgendwie beruhigte sie dieses Zeugnis der Kurzlebigkeit und Vergänglichkeit. Etwas würde immer bleiben.

\par

\WR{Wir können sicher vieles als Gerüchte abtun}, antwortete sie schließlich. \WR{Aber diese drei Piloten haben den Triumphbogen mit eigenen Augen gesehen. Falls ihnen etwas aufgefallen ist und sie es weitergeben, dann wird man ihnen glauben. Bitte finden Sie einen Weg, sie daran zu hindern.}
