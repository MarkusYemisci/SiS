Der Interkontinentalzug raste fast völlig geräuschlos durch einen der fünf großen atlantischen Tunnels. Klaus Rensing saß seiner Partnerin in Gegenfahrtrichtung entgegen. Er hätte gerne ein wenig mit Laura Gethas geplaudert, doch sie bevorzugte es, während Zugfahrten ein wenig Musik zu hören. In Oslo war es bei der Abfahrt fünf Uhr nachmittags gewesen. Trotzdem döste die Agentin zu einem Klassiker aus dem zwanzigsten Jahrhundert vor sich hin.

\par

Klaus Idee, den Hacker tatsächlich steckbrieflich zu suchen, hatte gefruchtet. In Sucre hatte eine Vermieterin das Phantombild erkannt und sich auf die Anzeige gemeldet. Es gab noch weitere Spuren. Hauptsächlich von Passanten, die den Datendieb gesehen haben wollten, doch die Wohnungsbesitzerin schien die beste Fährte liefern zu können.

\par

Schließlich schaffte es Klaus nicht mehr, ruhig zu bleiben. Er war ein Morgenmensch. Direkt nach dem Aufstehen sofort wach und bereit für den Tag, ganz im Gegensatz zu seiner Partnerin.

\par

\WR{Warum wolltest du eigentlich die Phalanx dabei haben?}, fragte er, um bei etwas dienstlichem zu bleiben.

\par

Laura öffnete seufzend die Augen und nahm sich die Kopfhörer aus den Ohren. \WR{Manus manum lavat}, antwortete sie. \WR{Otis will die Armee ein für alle mal loswerden. Das Hauptargument aus wirtschaftlicher Sicht ist, dass die Offensiven nichts zu tun haben. Darum wollte ich ihnen einen Auftrag verschaffen.} Klaus nickte, doch er wirkte skeptisch. \WR{Aber irgendwer musste dieselbe Idee gehabt haben. Es war keine Einheit verfügbar, als ich eine angefordert habe. Anscheinend sind große Teile der Phalanx ausgerückt.}

\par

\WR{Seltsam. Es war doch erst eine große Übung. Ich frage mich, warum der Oberkommandant soviel Geld verschwendet.}

\par

Klaus hätte noch weiter laut gedacht, hätte nicht sein Buch eine Eilmeldung angezeigt. Und dann noch eine. Und noch zehn weitere. Anscheinend war das Programm auf fast allen DDV-Sendern unterbrochen worden, um irgendetwas wichtiges dazwischenzuschieben.

\par

Klaus Wahl fiel auf Spectare State Vision~-- der Sender, der ihm am seriösesten erschien~-- und schon erschien das dreidimensionale Bild eines Nachrichtenstudios knapp oberhalb seines aufgeschlagenen Buches.

\par

\WR{Guten Tag, liebe Zuschauer}, begrüßte Ron Tucker, der schnauzbärtige Nachrichtensprecher des Senders. \WR{Wir unterbrechen unsere aktuellen Traumfilm für eine Ansprache des Forums aus Neuseeland. Präsident Otis.}

\par

Das Bild verschwamm kurz und als es wieder klarer wurde, war der Pressesaal des Präsidialbüros zu erkennen, den Klaus und Laura schon in zahllosen Übertragungen zu sehen bekommen hatten. Doch ein Detail war anders, was beide sofort wahrnahmen. Die Flagge der Union hing nun auf Halbmast. Auch die sonst übliche Kühle und Gelassenheit auf dem Gesicht des Präsidenten fehlte vollkommen.

\par

\WR{Liebe Mitbürgerinnen und Mitbürger}, begann er und schien dabei wesentlich weniger sicher, als man es von diesem Präsidenten gewohnt war. \WR{Ich trete heute mit sehr schlechten Nachrichten vor sie. Ich muss Ihnen leider mitteilen, dass die Unionskolonie auf Pollux Primus angegriffen und zerstört worden ist.}

\par

Lauras und Klaus Blicke trafen sich. In ihren Augen stand Fassungslosigkeit geschrieben. Ein Raunen ging durch die Ansammlung von Journalisten, die sich im Pressesaal eingefunden hatte. Es verstummte schnell, als der Präsident fortfuhr.

\par

\WR{Falls Sie Verwandte oder Freunde auf Pollux haben}, Otis vermied explizit eine Vergangenheitsform, \WR{dann sollten Sie sich an das Bürgeramt wenden. Wir haben eine Beratungsstelle eingerichtet, die Ihnen Auskunft über Ihre Bekannten oder Angehörigen geben wird.}

\par

Wieder ging ein lautes Raunen durch den Raum. Sicher hatten auch einige der Reporter jemanden auf Pollux gekannt. Einer von Ihnen erhob sich und warf die Frage in dem Raum: \WR{Wie viele Überlebende gab es?}

\par

Henry Otis kämpfte gegen den Drang zu seufzen und antwortete stattdessen so nüchtern er konnte: \WR{Etwa zweitrin vierdutzend. Eine genaue Zahl steht nicht fest. Einige der Überlebenden sind schwer verletzt und werden derzeit in den medizinischen Einrichtungen des Wega Stützpunktes behandelt.}

\par

Aus dem Raunen wurde schnell ein aufgeregtes Gerede. Durch die DDV-Übertragung nur schwer zu erkennen, verließen einige der Journalisten den Pressesaal.

\par

Der Präsident fuhr fort: \WR{Wir wissen nichts über die Motive der Angreifer. Aber die vorläufige Analyse der Daten lässt darauf schließen…} Eine lange Pause folgte. Der Raum wurde gespenstisch still und explodierte gleich wieder in tobende Diskussionen, als Otis fortfuhr, \WR{dass der Angriff von einer nichtirdischen Macht ausgeführt wurde.}

\par

\WR{Scheiße}, hauchte Klaus. Laura sagte nichts. Sie wusste nicht, was ihr mehr Angst machte. Das die Frage nach fremdem Leben nun auf die denkbar schlechteste Art beantwortet worden war. Oder der düstere Schleier, der die Angreifer umgab. Wozu waren sie imstande? Was wollten sie von den Menschen und warum begrüßten sie sie mit Gewalt? Sie kannte das Gefühl sehr gut, nicht zu wissen, womit man es zu tun hatte. Als Jugendliche hatte sie den Aufstieg der Capital Fellowship miterlebt. Es hatte nur ein paar Anschläge gegeben. Aber mit einem mal hatten alle hinter jeder Ecke Bomben und in jedem böse dreinschauenden Passanten einen Attentäter gesehen.

\par

Es war die Angst vor dem Unbekannten, das wurde Laura klar. Und in diesem Fall schien sie mehr als berechtigt zu sein.

\par

\WR{Wir sind nicht alleine}, fuhr der Präsident fort. \WR{Viele haben es schon lange vermutet. Doch jetzt können wir sicher sein. Unser Intellekt ist keine einsame Laune der Natur, sondern hat zumindest ein Gegenstück. Diese Erkenntnis ist vielleicht eine der wichtigsten in unserer Geschichte. Und ich wünschte sie wäre unter anderen Umständen eingetreten.}

\par

Einige der Journalisten warfen dem Staatsoberhaupt Fragen entgegen. Doch Henry Otis hob nur beschwichtigend die Hand. \WR{Dieses neue Wissen macht mir Angst. Genauso wird es Ihnen gehen. Ich habe gestern Abend von den Neuigkeiten erfahren und erst jetzt wird mir klar, was sie wirklich bedeuten. Dort draußen denken nun andere über uns nach. Sie sehen vielleicht auf unsere Kultur und fragen sich, was wir sind. So wie wir uns fragen, was sie sind. Und ich halte es für ausgesprochen wichtig, dass wir darauf schnellstens eine Antwort finden.}

\par

Klaus Hände umklammerten sein Buch. Er atmete schnell und Schweißperlen traten auf seine Stirn. Laura hatte ihn schon oft unter Stress beobachtet. Doch so aufgeregt hatte sie ihn nur selten gesehen. \WR{Kleine grüne Männchen}, murmelte ihr Partner. \WR{Ich kann es einfach nicht glauben. Ich meine, als ich geboren wurde, da sind wir schon zu den Sternen gereist. Ich habe mir immer gedacht, dass wir irgendwann mal auf mehr als nur ein paar Tiere stoßen. Aber das passiert so plötzlich. Der Tag ist so schnell gekommen.}

\par

Laura lauschte weiter den Ausführungen des Präsidenten. \WR{Es gehört zu unserer Natur, dass wir nun nach Vergeltung verlangen. Aber ich halte es für sehr wichtig, besonders jetzt, sehr vorsichtig zu sein. Für diese Situation gibt es kein Beispiel. Es ist das allererste mal. Wir wissen nichts über die Gründe. Gerade darum denke ich, dass wir nicht mit Waffen auf diesen Angriff reagieren sollten.}

\par

Das schien einem der Reporter zu viel zu werden. Er sprang auf und schrie: \WR{Die bringen uns um und wir sollen uns nicht wehren? Das ist Wahnsinn!}

\par

Otis wartete ein wenig mit seiner Antwort. Der Mann blieb stehen. \WR{Ich kann sie verstehen. Dieser Angriff hat über vier Pinae Menschen das Leben gekostet. Frauen, Kinder, Alte. Und ich versichere Ihnen, diese Tat wird nicht unbeantwortet bleiben. Aber wenn wir jetzt einen Fehler machen~-- wie blindlings um uns zu schlagen~-- dann wird es noch weitaus mehr Tote geben.}

\par

Eine erwartungsvolle Stille legte sich über den Pressesaal, denn jeder wusste, was folgen würde.

\par

\WR{Wenn es eine Chance gibt, diese Situation friedlich zu lösen}, begann der Präsident, \WR{dann müssen wir sie nutzen. Dafür stehe ich ein, auch wenn es das schwierigste ist, was ich in meiner Laufbahn bisher getan habe. In diesen Stunden wird ein wissenschaftlicher Ausschuss unserer klügsten Köpfe gebildet. Mit dem Ziel, irgendwie eine Möglichkeit zu finden, mit den Angreifern reden zu können. Wenn uns das gelingt, können wir vielleicht ein elementares Missverständnis mit einem einzigen Gespräch lösen, anstatt uns auf einen unvorstellbaren Krieg einzulassen.}

\par

\WR{Dieser Schwachkopf}, donnerte Klaus. \WR{Unsere Leute werden niedergemetzelt und er steckt immer noch in seiner Das-Militär-ist-scheiße-Kampagne. Darum macht er das doch. Jeder vernünftige Mensch würde diesen Bastarden die Starforce auf den Hals hetzen. Aber er glaubt, wenn er diesen Mist ohne Waffen löst, drückt er seine Überzeugungen durch.}

\par

\WR{Er hat Recht}, entgegnete ihm Laura energisch. Ihr Blick verfolgt schon seit einiger Zeit nicht mehr die Übertragung sondern verlor sich im weiten Ozean. Das Meer war in vielerlei Hinsicht dem Weltall sehr ähnlich, fand sie. Es war riesig, und je tiefer man kam, umso dumpfer und dunkler wurde es. Und es gab einen Haufen Fische mit starren Gesichtern. Die keine Mine verzogen, egal was geschah. Für sie waren sie ein Sinnbild der Natur. Seit Laura von der Seuche erfahren hatte, kam sie ihr gleichgültig vor. Sie schenkte zwar den Wesen in ihr durch Nahrung und Wärme das Leben. Aber sie interessierte sich auch nicht für ihr Schicksal. Wenn ein Leben endete, dann trauerte Mutter Natur nicht. Sie schickte bloß ein paar Bakterien um den Müll wegzuräumen.

\par

Würde es etwas bedeuten, wenn diese Außerirdischen die Menschheit ausrotteten so wie sie es anscheinend mit der Bevölkerung von Pollux gemacht hatten? Machte es einen Unterschied ob sich die Menschen selbst durch ihre eigene Kurzsicht ausradierten, wie sie es beinahe mit dem Routenkrieg getan hätten oder ob dies ein paar UFOs übernehmen würden. Würde das Universum die Menschen vermissen?

\par

\WR{Laura}, rief ihr Partner überrascht. \WR{Gerade eben hast du mir noch erzählt, du wolltest die Marineinfanteristen beschäftigen um sie vor Otis Streichkonzert zu retten. Und jetzt gibst du dem alten Holzkopf recht?}

\par

Laura sah Klaus wieder in die Augen. \WR{Er hat recht. Denk an den Routenkrieg. Denk an den Geschichtsunterricht. Unzählige Kriege hätten nicht sein müssen. Sie sind ausgebrochen, weil ein paar Generäle ihre Finger nicht ruhig halten konnten, nachdem das Volk \Wr{Feuer!} gebrüllt hat.}

\par

\WR{Aber da hatten wir es mit Menschen zu tun}, gab Klaus fast wütend zurück. Er glaubte seine Partnerin nicht wieder zu erkennen. \WR{Weiß Gott, was das für Dinger sind! Vielleicht denken Sie nicht wie wir. Vielleicht sind es zwei Meter große Insekten, die kleine Babys fressen. Was denkst du, was jemand zu dir sagen würde, der einen Angehörigen auf Pollux verloren hat?}

\par

Laura antwortete nur leise: \WR{Ich habe vielleicht jemand auf Pollux verloren. Mein Großneffe. Als ich ihn zuletzt gesehen habe, da war er noch ein kleiner Junge. Mittlerweile ist er fast erwachsen.}

\par

Sie konnte sich noch genau an das letzte mal erinnern, als sie Marcello gesehen hatte. Seine Eltern waren damals noch nicht geschieden gewesen. Er leckte an seinem Eis, als er mit seiner Mutter und seiner Tante durch Rio spazierte.

\par

Laura und ihre Großcousine hatten sich damals über die problematische Ehe unterhalten und den Plan, nach Pollux zu ziehen. Den kleinen Jungen hatte das nicht interessiert. Er hatte einfach sein Eis genossen und fasziniert die alten Gebäude und Straßen bewundert. Auf seinem Hemd war ein Heißluftballon abgebildet gewesen~-- ein Souvenir aus dem Luftfahrtmuseum. \WR{Ich find Ballons viel besser als Raumschiffe}, hatte er gesagt. \WR{Ich würd gerne mal mit einem fliegen.}

\par

Laura wollte weinen, doch irgendetwas hielt sie zurück. Klaus war es nicht, das wusste sie. Zwar zeigte sich ihm gegenüber ungern Gefühle, doch hatte sie sich in seiner Gegenwart auch immer am schlechtesten zusammenreißen können.

\par

Ihr Partner, der eine Weile lang nur geschockt durch sie hindurch gesehen hatte, forderte aufgeregt: \WR{Melde dich schleunigst bei diesem Amt. Vielleicht wissen die…}

\par

Doch Laura unterbrach ihn sofort: \WR{Nein! Ich kann nicht. Ich kann einfach nicht. Nicht schon wieder! Wir greifen uns diesen Typen und dann fahren wir heim..}

\par

Klaus sagte eine Weile lang nichts. Er wusste mittlerweile sehr gut, wann seine Partnerin Hilfe wollte und wann er sie besser in Ruhe ließ. So nickte er nur und antwortete: \WR{Okay.}