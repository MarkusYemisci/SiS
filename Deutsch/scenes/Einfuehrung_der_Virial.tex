\lettrine[lines=3]{F}{\"ur} einen Menschen wäre es unmöglich gewesen, doch der Supercomputer der \EN{Virial} hatte keine großen Schwierigkeiten damit, den Sprung durch den Hyperraum zu berechnen, der das Forschungsschiff ins Arktur-System brachte. Mühelos addierte er sechsdimensionale Vektoren und multiplizierte sie mit Matrizen um den Kreuzer der \EN{Ereignishorizont}-Klasse auf Kurs zu halten. Richtungskorrekturen waren überlebenswichtig. Insbesondere bei einem Sprung, der in ein System führte, in dem noch nie ein Mensch zuvor gewesen war. Lediglich einige Testdrohnen hatten die Reise nach Arktur bislang unbeschadet überstanden.

\par

Die \EN{Virial} würde die ersten Menschen in die Nähe dieses fernen Sterns tragen.

\par

Die sechzig Männer und Frauen an Bord hätten nun vielleicht um ihr Leben gebangt, wenn sie den Sprung tatsächlich bewusst erlebt hätten. Doch der menschliche Geist schien nicht geschaffen, um die tief liegende, fremde Welt des Hyperraums wahrnehmen zu können. Viel Hilfe wären sie dem Computer auch bei vollem Bewusstsein aber ohnehin nicht gewesen.

\par

Es war schwer, ein Segelschiff selbst auf einem weitestgehend zweidimensionalen Ozean zu steuern. Schwerer war es, ein Flugzeug durch das dreidimensionale Reich der Lüfte zu manövrieren. Aber die Reise durch den tiefen Hyperraum, bei dem sich selbst die klügsten Physiker nicht sicher waren, wie viele Dimensionen er eigentlich hatte, war ohne die Hilfe komplexer Algorithmen unmöglich zu bewerkstelligen.

\par

Es galten eiserne wie einfache Regeln. Ein Sprung musste auf einer nahezu geraden Linie stattfinden und unter keinen Umständen durften sich größere Hindernisse in der Flugbahn befinden. Zwar existierten beispielsweise herumirrende Meteoriten im euklidischen Raum und konnten so zwangsläufig nicht mit einem Schiff auf Überlichtgewschwindigkeit kollidieren. Doch auch sie bildeten einen Massenpunkt, der die Trajektorie der \EN{Virial} entscheidend hätte verändern können. War ein fremdes Objekt nah genug, um eine nennenswerte Ablenkung von der idealerweise enormen Gravitationskraft eines Ziels zu bewirken, lag es am Computer, eine Kurskorrektur vorzunehmen.

\par

Schaffte er dies nicht schnell genug~-- und viel Zeit blieb ihm selbst bei den längsten Sprüngen von ein paar Sekunden nicht~-- würde das Schiff am Ziel vorbeischießen und schnell in Gefilde des Hyperraums gelangen, für die es keine Karten gab und ein Austritt generell unmöglich schien.

\par

Was mit einem solchen, verlorenen Flugkörper passieren würde, wusste niemand, denn weder verschollene Sonden noch Raumschiffe waren jemals wieder im erforschten Universum aufgetaucht. Der Mannschaft der \EN{Virial} blieb es jedoch erspart, dieses Mysterium zu ergründen.

\par

Am neunzehnten März zweitausend siebenhundert acht verließ das Forschungsschiff den Hyperraum um neunzehn Uhr acht zur allgemeinen Erdzeit. Im weitaus üblicheren Datumsformat der Union entsrpach dies dem Jahr vierdin elfdutzend und zwei.

\par

Abed Nidal schreckte auf und griff sofort mit beiden Händen fest die Konsole, hinter der er saß. Sofort stieg in seinem Hals die altbekannte Übelkeit auf, die ihn bei fast jedem Intersystemsprung begleitete. Er versuchte krampfhaft, sich zu konzentrieren und mit seinen ganz persönlichen Atemübungen zu beginnen, die ihm halfen, den Brechreiz früher oder später immer in den Griff zu bekommen.

\par

Diesmal fiel es ihm deutlich schwerer als sonst.

\par

Noch immer hatte er die wirren Bilder vor Augen. Formlosen Erscheinungen gegen die selbst die Muster, die entstanden, wenn man sich fest die Hände auf die Augen drückte, wirkten wie ein photorealistisches Bild. Und immer wieder Geräusche, die in seinen Ohren nach wie vor wie eine Kreissäge nachhallten. Ein Klangteppich wie aus dumpfen Trommeln und hellem Kreischen.

\par

All diese abstrusen Eindrücke war Abed längst gewöhnt. Der Flug nach Arktur war längst nicht sein erster Sprung gewesen. Und zu oft hatte er sich bereits seines Mageninhalts entledigt, weil der Hyperraum auf seinen Geist eingewirkt hatte, wie es nur ein Schraubstock mit einem darin eingeklemmten Kopf tun könnte.

\par

Aber diesmal war es anders gewesen. Deutlich schlimmer. Vielleicht, weil die Informationen, die sein Gehirn von seinen Sinnen und tausend anderen Quellen während einer Überlichtreise geliefert bekam, diesmal ein klein bisschen weniger chaotisch aber dafür beängstigender gewesen waren.

\par

Er erinnerte sich wie meistens nicht an Details. So, wie auch ein Traum kurz nach dem Aufwachen sofort verschwand, verblassten auch die Erinnerungen an die Phase während des Sprungs nach wenigen Augenblicken. Aber er konnte diesmal noch deutlich die ausgestreckte Hand vor seinen geistigen Augen sehen. So, als wolle sie immer noch nach ihm greifen. Und der Schrei klang ebenso in seinen Ohren nach.

\par

Abed entschloss sich, sofort die Augen aufzureißen, um schnellstmöglich wieder in der Wirklichkeit anzukommen. Das Licht Arkturs brannte ihm dabei fast in den Augen. Der riesenhafte Stern besaß mehr als den fünfundzwanzigfachen Radius der irdischen Sonne und sandte gleißendes Licht durch die Fenster der Brücke.

\par

Die \EN{Virial} befand sich noch auf einem sicheren Abstand von drei astronomischen Einheiten. Dennoch hatten die Schutzfelder welche die Brücke umgaben bereits ihre liebe Mühe, die Strahlung abzufangen. Ohne diese und die spezielle Verglasung des Kommandozentrums, wäre es unmöglich gewesen, den Stern direkt anzusehen.

\par

Dennoch wandte Abed seine Augen wieder vom Gestirn ab und versuchte sich eher auf sein deutlich näherliegendes Arbeitspult zu konzentrieren und endlich wieder klar zu werden. Die Instrumente, Schalter und Glasmonitore erschienen ihm deutlich greifbarer und wirklicher, als der weit entfernte Stern. Und sie glänzten sehr schön, wie er aufs neue feststellte. Die Pinnacle Science Group rüstete seine Schiffe gut aus. Doch bei dem Umsatz des Megakonzerns waren ein paar der neuesten Rechner der Union vermutlich trotzdem noch ein Pappenstiel.

\par

Trotz allem empfand es Abed als eine Schande, sollten sie verunreinigt werden, falls die Pfannenkuchen vom Vortag noch einmal zutage treten sollten.

\par

Nach einiger Zeit kam er, wie auch die meisten anderen Menschen in seiner Nähe, wieder voll zu sich. Er ballte seine Hände kräftig zu Fäusten und erfreute sich über das bekannte Gefühl der Anspannung.

\par

\WR{Warum tue ich mir das eigentlich immer wieder an?}, murmelte er leise vor sich hin, immer noch mit der Übelkeit kämpfend. Sein Nebensitzer, Eddie Carlyle, begann sofort zu grinsen, als er Abeds Selbstgespräch mitbekommen hatte. \WR{Weil du die Knete brauchst, natürlich}, stichelte er.

\par

Abed grunzte und entgegnete sofort: \WR{Ach, halt doch die Klappe. Du stehst auch nicht besser da. Besonders nicht nach deinen ganzen Sportwetten.}

\par

Eddie antwortete irgendetwas, fand aber kein Gehör. Abed sah aus dem fünf mal zehn Meter großen Frontfenster der Brücke. Der riesige, orangeroten Stern des Arktur-Systems bot einen atemberaubenden Anblick, nun, da er das nötige Bewusstsein aufbringen konnte, um ihn genauer zu betrachten. Das Gestirn war weit mehr als eine einfache leuchtende Kugel. Durch die filternden Scheiben war ein direkter Blick auf die Korona möglich. Selbst aus der riesigen Entfernung, welche die \EN{Virial} von Arktur trennte, waren die unzähligen Sonnenflecken, Konturen und Eruptionen zu sehen, die teilweise größer waren als drei Erden zusammen.

\par

Abed erstarrte kurz im Angesicht dieses Schauspiels. Regungslos saß er da und blickte den orangeroten Riesen an. Er hätte sich wahrscheinlich noch eine ganze Weile nicht aus dem Bann des Sterns befreien können, wenn ihn nicht ein anderes Besatzungsmitglied unsanft angerempelt hätte. Als Abed dem Übeltäter einen bösen Blick zuwerfen wollte, war dieser schon weiter gerannt. Die Hand fest vor den Mund gepresst lief der Mann in Richtung Aufzug. Abed konnte sich ein kurzes Grinsen nicht verkneifen. Unbewusst griff er sich in die Tasche seines Gehrocks und ertastete die kleine Papiertüte, die er immer bei sich trug. Für den Fall, dass er seinen Magen einmal nicht rechtzeitig beruhigen konnte. Er verfuhr schon seit Jahren so. Schon seit seinem ersten Sprung. Damals hatte er eben keine Tüte dabei gehabt und seinen Mageninhalt über die Konsolen seiner Sitznachbarn verteilt. Die Beziehung zu seinen damaligen Kollegen hatte sich durch diesen Vorfall nicht gerade verbessert.

\par

\WR{Verdammte Anfänger}, fing Eddie an zu schimpfen. \WR{Was essen diese Jungspunde bloß, dass ihnen derart schnell die Galle hochkommt?}

\par

Abed belächelte die aufbrausende Art seines Sitznachbars. Wenn irgendwo irgendetwas nicht richtig lief, dann konnte man sich darauf verlassen, dass Eddie sich als erster darüber beschwerte. Dennoch hatte er ihn gerne als Kollegen. Seine Sprüche und Meckereien brachten etwas Abwechslung in das sonst recht eintönige Leben an Bord der \EN{Virial}.

\par

Kurze Zeit später hörte Abed den Mann am Steuer melden: \WR{Hyperraumssprung abgeschlossen, Herr Krüger. Wir haben das Arktur System erreicht.}

\par

Johann Krüger, ein durchtrainierter Mann in seinen Fünfzigern mit breiten Schultern, löste die Sicherheitsgurte seines Sessels und erhob sich. Wie die meisten höheren Firmenangestellten trug auch er einen tadellos geschnittenen, dunklen Gehrock aus kostspieligem Stoff. Seinen Kragen hatte er hochgestellt und das Stoffband, das an eine Krawatte aus dem zwanzigsten Jahrhundert erinnerte, einwandfrei zurechtgerückt. Die Adrettheit seiner Kleidung täuschte. Krüger nutzte den visuellen Eindruck, den er hinterließ lediglich, um seinen Ehrgeiz zu unterstreichen, nicht, um seine höhere Position zum Ausdruck zu bringen.

\par

Abed sah an sich selbst hinab. Seine Kleidung wirkte nicht annähernd so formell und korrekt wie die des Kommandanten. Er selbst hatte einen Pullover einem Hemd vorgezogen und auf ein Halstuch gänzlich verzichtet. Von dem hohen Qualitätsunterschied zwischen den Stoffen der Gehröcke ganz zu schweigen.

\par

Einen noch krasseren Gegensatz bildete jedoch sein Kollege Eddie. Dieser trug an diesem Tag nicht einmal seinen abgewetzten Anzug, sondern lediglich ein Flanellhemd und eine durchgesessene Cordhose. Und obwohl die Gleichstellungsnormative Benachteiligung, beispielsweise aufgrund von Kleidung, verbot, wunderte es Abed dennoch, wie Eddies Bewerbung bei einem renommierten Unternehmen wie der Pinnacle Science Group durchgekommen sein konnte.

\par

Johann Krüger zückte sein Buch und machte einige Notizen. Das Gerät war etwas größer und breiter als die üblichen Versionen, damit man problemlos im Stehen mit einem Nanotintenfüller darauf schreiben konnte, anstatt sich für jede Notiz eine holographische Tastatur anzeigen lassen zu müssen.

\par

Abed hatte zunächst vermutet, dass es sich bei den Aufschrieben, die Krüger so oft machte, um Belanglosigkeiten handelte, die früher oder später in irgendeiner Firmenakte verschwinden würden. Jedoch hatte sich schnell herausgestellt, dass sie eher ein Ausdruck von Krügers extremem Disziplinverständnis war. Die Dateien in seinem Buch enthielten massenhaft selbstgesetzte Zielvorgaben, Zeitpläne und andere Dinge, welche die meisten Menschen einfach den Tag über im Kopf behielten.

\par

Eddie starrte Gedankenverloren durch die Fenster hinaus ins Weltall. Leise murmelte er: \WR{Ich glaube, so weit draußen im Nichts war ich noch nie. Wieso können wir nicht eigentlich mal ein paar interessante Kolonien besuchen?}

\par

Sein Nebenmann lies seinen Blick ebenfalls durch die Unendlichkeit des Raums schweifen. Da das gesamte Kommandozentrum, bis auf die rückwärtige Wand, von Fenstern begrenzt wurde, war es nicht schwierig sich in den Bann des Universums ziehen zu lassen. Als Abed seinen Kopf in den Nacken legte und direkt nach oben sah, gab es nichts mehr, dass seinen Blick behinderte und er fühlte sich, als würde er selbst zwischen den Sternen schweben. Nur gelegentliche Lichtreflexionen an den Scheiben erinnerten ihn daran, dass der Kommandoraum durch Spezialglas von einem halben Meter Dicke vom Rest des Kosmos getrennt war. Den eigentlichen Schutz boten jedoch Schildpanzer. Die \EN{Virial} besaß, wie die meisten Schiffe, zwei Generatoren dafür. Einer, der im Bedarfsfall Schutz für den gesamten Rumpf bot und ein weiterer, der dies nur für die Brücke tat.

\par

Eddie hatte sich bereits wieder seiner Arbeitsstation zugewandt, als Abed ihm antwortete: \WR{Sieh's positiv. Während des Sprungs haben wir uns innerhalb von wenigen Sekunden weiter fortbewegt, als irgendein Mensch in seinem ganzen Leben zu Fuß}

\par

Der Angesprochene sah nur gedankenverloren auf seine Bildschirme. Abed entschied, es ihm gleich zu tun. Kapitän Krüger würde sowieso schon bald einen Lagebericht haben wollen und dann müsste er ihm Rede und Antwort stehen können. Das war immerhin seine Aufgabe im Kommandoraum.

\par

Er überwachte die passiven Suchsysteme der \EN{Virial}. Das bedeutete einerseits das Echtzeitradar und andererseits die Emissionssensoren, die überall auf der Hülle des Forschungsschiffes verteilt waren. Die Technik des Radars beeindruckte Abed immer wieder~-- genauso, wie die fehlerhafte Verwendung des historischen Begriffs. Die Geräte, die nichts mit einem Radar aus der Vorseuchen-Ära zu tun hatten, waren in der Lage, eine etwa zweieinhalb astronomische Einheiten große Sphäre beinahe augenblicklich abzusuchen. Die dem Radar eigene Abstrahlung war dabei von anderen kaum noch zu orten, was allerdings ein taktisches Detail war, das seit dem Routenkrieg fast keine Bedeutung mehr besaß.

\par

Um einiges empfindlicher noch als das Radar, war jedoch das Netz der Emissionssensoren. Diese konnten selbst geringste Abweichungen zur normalen Hintergrundstrahlung aufspüren. Da sich Strahlung aber nur im schnellsten Fall mit Lichtgeschwindigkeit fortbewegte, arbeiteten die Sensoren mit einer Verzögerung. Diese konnte länger oder kürzer sein, je nachdem wie weit die Strahlungsquelle entfernt lag.

\par

Als Abed damals die Stellenanzeige für den Radarposten auf der \EN{Virial} in die Hände gefallen war, hatten ihn gerade diese technischen Raffinessen gereizt. Er hatte sich häufig gefragt, ob er die Stelle auch angenommen hätte, wenn ihm schon damals klar gewesen wäre, dass seine Arbeit hauptsächlich aus stumpfsinnigem Starren auf einen Bildschirm bestehen würde. Aber jedes Mal kam ihm auch in den Sinn, was er alles verpasst hätte, wenn er sich nicht beworben hätte.

\par

Er hatte Teile des Universums gesehen, die anderen Menschen niemals zu Gesicht bekommen würden. Fremde Planeten, die unglaublich bizarr aber gleichzeitig auch wunderschön waren. Eine Nebelbank die sich über hunderte von astronomischen Einheiten erstreckte. Und nicht zuletzt die Sonne. Was von der Erde aus nur eine leuchtende Scheibe war, wirkte aus der Nähe wie ein niemals enden wollendes Feld aus weißem Feuer. Abed erinnerte sich daran, wie er durch eine Absorberscheibe auf den Stern gesehen hatte und sich vorgekommen war, als würde er in reines, lebendiges Licht sehen.

\par

Solche Erlebnisse waren wahrscheinlich der Hauptgrund dafür, warum er sich nicht mittlerweile einen anderen Job gesucht hatte. Und er vermutete, dass es sich bei den meisten Anderen auf der \EN{Virial} genauso verhielt. Immer wieder wurde die Bezahlung vorgehalten aber in Wirklichkeit waren sie alle nur neugierige Romantiker, die sich die Welt ansehen wollten.

\par

Was Krüger anging, war sich Abed allerdings nicht so sicher. Wie alle leitenden Angestellten der Pinnacle Science Group hatte auch er wenig mit einem Naturwissenschaftler gemein. Die PSG betraute meistens Firmenfunktionäre mit dem Kommando über ein Forschungsschiff. Wahrscheinlich weil man in der Führungsriege nicht annahm, dass ein experimentierfreudiger Forscher, der bei jeder Anomalie anhielt und nachsah, ein Raumschiff genauso ökonomisch befehligen konnte, wie ein Betriebswirt. Obwohl die PSG an sich ein sehr magnates Unternehmen war, kannten ihre Geldmittel dennoch Grenzen. Und selbst im achtundzwanzigsten Jahrhundert war fast nichts teuerer als die Raumfahrt.

\par

Abed dachte dabei an sein letztes Gehalt. Über neuntausend Naira Quartalsverdienst waren ein Wort. Die Bezahlung war sicherlich das Letzte, worüber er sich ärgerte. Eher über das, was er für sein Geld tat. Er hatte acht Jahre lang studiert und besaß ein Klasse-A-Diplom in Physik und zwei C-Diploma in Mathematik und Hyperraum-Ingernieurswesen. Zu der Zeit, als dieser Titel nicht nur für Ärzte verwendet worden war, hätte er sich als \textit{Doktor} Nidal bezeichnen können.

\par

Natürlich brauchte Abed sein Fachwissen gelegentlich um die Funktionsweise des Radars nachvollziehen zu können aber die meiste Zeit saß er nur vor einem Monitor und sah einer Linie beim Rotieren zu.

\par

Herr Krüger ging im Kommandoraum auf und ab. Hin und wieder sah er einigen der Forscher über die Schulter, kaum verstehend, was auf den Bildschirmen vor sich ging. Wie Abed und Eddie beobachtet hatten, schien die Teleskopüberwachung die Lieblingsstation des Kommandanten zu sein. Entweder weil die Bilder der Fernrohre oft imposant aussahen und auch für einen Laien leicht zu verstehen waren, oder aufgrund der jungen Spezialistin für Optik, die dahinter beflissen ihren Dienst tat.

\par

Schließlich forderte Herr Krüger einen Statusbericht ein und notierte die immer gleich lautende Antwort gewissenhaft in sein Buch. Als es an Abed war, seine Meldung zu machen, sagte auch er: \WR{Alle Systeme arbeiten normal.}

\par

\WR{Wie sieht es mit den aktiven Suchsystemen aus?}, fuhr Krüger fort und richtete seinen Blick auf Eddie.

\par

Dieser entgegnete leichthin: \WR{Alles Paletti}, ohne sich umzudrehen.

\par

\WR{Wie bitte?}, fragte Herr Krüger unmittelbar. Seine Stimme klang zwar freundlich aber alle im Kommandoraum spürten deutlich, dass dem Kapitän die Antwort missfiel. Eddie seufzte über die Vehemenz, mit der Krüger an seinem Schema F festhielt. Nachdem alle Blicke erwartungsvoll auf ihn gerichtet waren, antwortete der Sensorspezialist erneut: \WR{Alle Systeme arbeiten funktionsgemäß.}

\par

Herr Krüger nickte nur kurz und fuhr dann mit seiner Befragung fort. Eddie lehnte sich zu seinem Nachbarn und flüsterte leise aber hörbar verärgert: \WR{Oh Mann. Ich glaub, er hat sogar einen Plan dafür, wenn er auf die Toilette geht.}

\par

\WR{Denk dir nichts}, antwortete Abed sofort. \WR{Es ist normal, dass ein Konzern ganz genau wissen will, was los ist. Das Problem ist, dass die Jungs und Mädels von den oberen Rängen keine Ahnung haben, was wirklich wichtig ist und was nicht. Und so wird eben jeder Mist aufgeschrieben und dokumentiert.}

\par

Eddie rollte mit den Augen. \WR{Stimmt. Trotzdem ist es nicht zu viel verlangt \Wr{Alles Paletti} interpretieren zu können, oder?}

\par

\WR{Krüger ist eben so}, entgegnete ihm Abed sofort. \WR{So sehr wie er es liebt, sich selbst reden zu hören, gibt er bestimmt auch gerne vor, was alle anderen sagen sollen. Oder er will sich einfach absichern. Wenn er aus deinem Mund hört, dass alle Systeme funktionierten ist er aus dem Schneider. Aber wenn du sagst \Wr{Alles Paletti}, dann kann das viel heißen. Dass es deiner Seele gut geht, zum Beispiel. Oder, dass dir die Entwicklung deiner Geldanlagen gefällt. Verstehst du, was ich meine?}

\par

Abeds Nebensitzer wandte seinen Blick ab und grinste verbittert. \WR{Ich denke es wäre besser, wenn einer von uns Wissenschaftlern das Kommando hätte. Nicht so ein aufstrebender Schreibtischhengst, der zwar alle Regeln kennt, dafür aber keine Ahnung von wissenschaftlicher Methodik hat.}

\par

Abed brummte leise. Eigentlich konnte er seinem Sitznachbarn nur zustimmen. Aber irgendwie hatte er sich schon damit abgefunden, wie die Hierarchie in der PSG funktionierte.

\par

\WR{Weißt du}, begann er mit einer Antwort, \WR{ich glaube, wir verschwenden nur unsere Energie, wenn wir uns jetzt über alles aufregen. Im Prinzip haben wir doch, was wir wollen. Wir reisen durchs Universum! Wir können uns Sachen anschauen, den sich der Rest der Menschheit nicht einmal vorstellen kann.}

\par

Eddie sah auf sein Computerterminal. Auf dem Monitor, der nur aus einer dünnen Glasschicht und einem metallenen Rahmen bestand, war ein Halbkreis zu sehen, der sich immer wieder um eine Achse in der Mitte des Bildschirms drehte.

\par

\WR{Du denkst also, der Rest der Menschheit kann sich keine einfallslose drei-D-Animation vorstellen?}

\par

Kurz darauf war Krüger mit der Einholung der Statusberichte fertig. Er atmete angestrengt aus und begann dann, laut genug, um von jedem gehört zu werden, zu reden: \WR{Meine Damen und Herren. Wir haben Arktur erreicht und werden jetzt mit unserem Auftrag beginnen. Bitte behalten Sie im Hinterkopf, dass die PSG von jedem erwartet, dass er seine Aufgabe gewissenhaft und pflichtbewusst durchführt.} So manch einer musste sich einen mürrischen Seufzer verkneifen. \WR{Wir werden nun als aller erstes mit einer groben Erfassung sämtlicher Himmelskörper in diesem System beginnen. Wie wir die folgende Vermessung vornehmen, besprechen wir heute Abend en Detail. Wir haben eine Menge Arbeit vor uns, also lassen Sie uns anfangen.}

\par

Eddie grummelte frustriert. Abed warf ihm einen beschwichtigenden Blick zu. Im Gegensatz zu seinem Kollegen hatte er sich schon seit geraumer Zeit mit Krügers Art arrangiert. Er hatte sich vorgenommen, den Kapitän nicht für seinen Ehrgeiz zu verurteilen. Auch wenn ihm dies dessen fast schon klinischer Narzissmus nicht gerade leichter machte. Zumindest schien dieser nur selten durch. Vorausgesetzt, man gehörte nicht zu Krügers engerem Mitarbeiterkreis.

\par

So machte sich Abed einfach an die Arbeit. Er würde die nächsten Stunden genießen, denn während dieser Zeit hing vieles an ihm. Danach, wenn alle Körper des Systems zumindest oberflächlich erfasst wären, würde er nur noch starr auf seinen Monitor sehen und hoffen, dass sie ihm doch noch irgend etwas neues zeigen würden. Ein Wunsch, der bisher niemals in Erfüllung gegangen war.