\WR{Morten, hau deinen Nachbrenner rein!}, brüllte Kevin. Der Angesprochene nahm sich keine Zeit, um zu antworten. Natürlich ließ er bereits jedes verfügbare Capezinmolekül in seine Plasmatriebwerke einspritzen.

\par

Die digitale Nadel seines Geschwindigkeitsmessers zitterte am Anschlag, während er auf das Gegenstück des Kreuzpunkter Triumphbogens zu flog. Dieser hielt seine Position noch, während Morten bereits erkannte, wie sein Pendant auf der anderen Seite zu schwanken begann.

\par

Durch die Verbindung war zu erkennen, wie sich der Triumphbogen bewegte, über seine eigene Achse abkippte. Auch die Ränder der Makrozonen schienen auszufransen sich auszudehnen.

\par

\WR{Die Verbindung bricht zusammen}, sagte Marco Bellendi. \WR{Morten, Sie haben noch zwei Dutzend Sekunden.}

\par

Er überschlug die Entfernung zum Ziel im Kopf. Selbst mit der Rechenfunktion eines Buches oder Handcomputers wäre das Ergebnis das gleiche gewesen. Seine Zeit reichte hinten und vorne nicht aus.

\par

\WR{Meine Güte. Das muss das Paradies für einen Physiker sein}, sagte der Bellendi. \WR{Diese Messwerte bekommt man normalerweise nur bei einem Hypersprung zu sehen. Es ist, als würde sich der tiefe Hyperraum von hier aus bis hin zu… wo auch immer sie sind öffnen.}

\par

Morten dachte darüber nach, abzudrehen. Aber er befand sich tief im Raum der Shutek und hatte kein Ziel, zu dem er flüchten konnte. Und die Jäger, die sich nun kaum mehr hundert Meter hinter seinem Heck befanden, eröffnete bereits in wilder Wut das Feuer. Nun hatten sie keine wertvolle Einrichtung mehre, die sie nicht treffen durften.

\par

Aber mit seinen konventionellen Triebwerke bräuchte er noch anderthalb Minuten. Und bis dahin wäre die Verbindung definitv verschlossen. Schon jetzt hatte das Phänomen, dass Bellendi als Makrozone benannt hatte, kaum mehr die Hälfte ihres ursprünglichen Durchmessers.

\par

Morten wollte gerade schon sein Funkgerät anwerfen und Kevin und Nico irgendetwas zum Abschied sagen. Auch wenn er nicht sicher war, was. Da drang eine andere Stimme an sein Ohr.

\par

\WR{Flacher Hyperraum. Du musst jetzt springen.}

\par

\WR{Was?}, fragte Morten verwirrt. \WR{Bitte wiederholen, \EN{Junge Maid}.}

\par

\WR{Wir haben nichts gesagt}, war Bellendis nicht weniger konfuse Antwort.

\par

\WR{Vertrau mir. Spring jetzt!}

\par

Jetzt hörte Morten deutlich, dass die Stimme gar nicht mal aus seinem Kopfhörer stammte. Sein Navigationscomputer war wieder hochgefahren. Aber die Anzeigen ergaben keinen Sinn. Der Text auf dem Bildschirm war reiner Kauderwelsch. Das einzige, was halbwegs gewollt aussah, war eine Gleichung, die eine Trajektorie zu einem von Kreuzpunkts Monden anzeigte.

\par

\WR{Sprung einleiten!}

\par

Die Zeit war fast um und die Verbindung beinahe geschlossen. Ein feuerroter Ring umgab nun die Aussicht auf Kreuzpunkts blauen Himmel.

\par

Fast unterbewusst ging Mortens Hand zu den Kontrolle für den Hyperantrieb seines Rapiers. Im Gegensatz zu allen anderen Fliegern im Repertoire der Starforce mit Ausnahme der schweren Jäger und Bomber verfügte sein Vogel über einen bordeigenen Sprungantrieb.

\par

Der Computer warnte ihn sofort, als er diesen scharf schaltete. In roten Lettern schrieb er, dass kein eindeutiger Kurs berechnet war und zu wenige Daten für die Bestimmung einer sicheren Passage vorlagen.

\par

Fünf Sekunden.

\par

Morten überging die Warnung durch drehen eines kleinen Schlüssels am Griff des Hyperantriebs. Ohne weitere Worte zog er diesen hastig zu sich hin und wurde augenblicklich von der Wucht der Beschleunigung in seinen Sitz gedrückt. Der Raum ihn herum schien zu verschwimmen und mit den weißglühenden Flächen um sein Schiff herum zu verschmelzen.
