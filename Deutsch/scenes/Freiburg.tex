\WR{Was denkst du}, begann Klaus Rensing, \WR{wie lange wird der Schnee noch liegen?}

\par

Laura Ghetas sah aus dem Fenster. Die Aussicht war gelinde gesagt schön. Sie und ihr Partner aßen in einem Restaurant zu Mittag, dass auf einem Berg in der Nähe von Oslo. Die Sonne stand schon recht tief, obwohl es erst dreizehn Uhr war. Die Küste und die Stadt wurde in ein schwaches, fast zerbrechliches Licht getaucht. Zahllose hin und her rasende Lichter zeugten von der Hektik der Menschen in der Stadt, während das Meer wie der ruhige Kontrapunkt wirkte. Die Wellen, die gegen die Küste schlugen, wirkten im Gegensatz zur Wirklichkeit vom Restaurant aus nur wir geräuschlose kleine Erhebungen im Wasser.

\par

Laura genoss den Anblick, bevor sie antwortete: \WR{Ich weiß nicht. Der Winter dauert hier lange. Aber was soll's. Ich komme ja wochenends nach Hause.}

\par

\WR{Ich mag keinen Winter}, dachte Klaus laut. \WR{Alles ist kalt, nass und eklig. Ständig muss man sich dick anziehen um nicht zu frieren und irgendwas außerhalb der eigenen vier Wände unternehmen, das kann man vergessen.}

\par

Laura lächelte bitter. \WR{Du findest es zu kalt? Was denkst du, wie es den Menschen vor ein paar hundert Jahren ging, bevor es künstliche Immunsysteme gab. Die müssen ständig erkältet gewesen sein.}

\par

\WR{Aber dafür hat sie ein kleiner Schnupfen noch nicht gleich ins Krankenhaus gebracht}, gab Klaus zurück und nahm einen weiteren Bissen von seinem Kotelett.

\par

Laura war bereits fertig mit essen. Sie hatte nicht viel gehabt. Nur eine Suppe und ein paar Scheiben Brot. Als er selbst fertig gegessen hatte, musterte Klaus den Teller seiner Partnerin skeptisch.

\par

\WR{Sag mal, wieso isst du so wenig}, begann er mit tiefem Unverständnis in der Stimme zu fragen. \WR{Mit deiner Figur kannst du noch gut fünf bis zehn Kilo vertragen. Und kritisch wird’s erst bei fünfzehn oder zwanzig.}

\par

Seine Partnerin zuckte mit den Schultern. \WR{Du weißt doch, ich spar mein Geld. Essen bekomm ich auch in den öffentlichen Versorgungsstationen. Und da muss ich nicht mal was bezahlen.}

\par

Klaus Rensing leerte sein Glas vom letzten Schluck des alkoholfreien Biers. \WR{Also das werd ich nie verstehen. Du bist Vizeberaterin beim Geheimdienst. Du brauchst bestimmt nicht zu diesen komischen Fressstellen zu gehen. Da geht doch nur hin, wer fast nix in der Tasche hat.}

\par

Laura seufzte. Diese Diskussion hatte sie mit ihrem Partner schon häufig geführt. Sie glaubte mittlerweile nicht mehr daran, ihren Kollegen jemals von seinen Vorurteilen befreien zu können. Zumindest konnte sie vom tatsächlichen Grund ihrer Appetitlosigkeit ablenken, indem sie es noch einmal versuchte.

\par

\WR{Wie ich dir schon tausend mal gesagt habe, Klaus. Du unterschätzt völlig den Wert dieser Stationen. Sie ermöglichen zum Teil die Freiheit, die alle so sehr an der Union lieben. Kein Künstler muss heute noch am Hungertuch nagen, nur weil seine Kunst brotlos ist. Wer von diesen Menschen kann schon seine Träume und seinen Lebensunterhalt gleichzeitig bezahlen? Und für die persönliche Erfüllung ist der Staat doch da.}

\par

\WR{Diese Typen sollen sich einfach eine anständige Arbeit suchen, so wie wir}, tat Klaus die Argumente seiner Kollegin einfach ab.

\par

Laura wollte gerade etwas erwidern, da piepte ihr Buch. Sie klappte es beim blauen Lesezeichen und bemerkte mit überraschend starkem Herzklopfen, dass es der Kommandant des Berliner Starforce Stützpunkts war, der ihr zur Unterstützung eine Fähre zugesagt hatte.

\par

\WR{Klaus, es geht los. Unser Mann greift die Krypta Scientia von einer Station in Freiburg aus an. Die Fähre landet in einer Minute auf dem Bergplateau.}

\par

\WR{Was?}, brachte dieser schockiert hervor und spähte auf seine Armbanduhr. \WR{Der Verteilerabsturz ist doch erst seit einer Minute im Gang. Der Kerl ist echt fix.}

\par

Hastig sprangen beide auf, verabschiedeten sich so schnell sie konnten bei der Bedienung und bezahlten per Fingerabdruck die Rechnung für ihr Essen.

\par

Die kalte Winterluft brannte Klaus wie Laura in der Lunge und im Rachen, als sie ein Hecheln nicht unterdrücken konnte, während sie auf die Bergspitze zu rannte.

\par

Laura hatte mit dem Kommandanten des Stützpunkts vereinbart, dass eine Fähre zu einem kleinen Landeplatz auf dem Berg kommen, und sie und Klaus abholen würde, sobald es losging. Aber dass die versteckten Anti-Hacking Programme so schnell nach dem Systemabsturz einen Einbruch melden würden, hätte niemand angenommen. Sonst hätten die beiden nicht noch eine gemütliche Mittagspause eingelegt, sondern hätten gleich auf dem Flugplatz des Stützpunkts gewartet.

\par

Die Triebwerke der Fähre verursachten bereits einen enormen Lärm, als sie noch einen Kilometer weit entfernt war. Als das Shuttle in den Landeanflug ging, spielte sich die Geräuschkulisse knapp unterhalb der Schmerzgrenze ab. Das Schiff der Hermes-Klasse setzte sanft mit seinen Landeklappen auf der kleinen Plattform auf, die vor Jahren nahe der Bergspitze installiert worden war.

\par

Klaus sprang als erstes durch die schmale Luke auf der Steuerbordseite und half anschließend seiner Kollegin einzusteigen. Sie sah ihm dabei nur kurz in die Augen aber die Zeit reichte ihr aus, um das obligatorische Jagdfieber bei ihrem Partner zu erkennen. Innen nahm sie der Kopilot in Empfang.

\par

\WR{Sind alle Agenten so langsam?}, scherzte der Mann lachend, als sich die Tür schloss und die Fähre abhob. \WR{Ich hoffe, Sie haben nicht auch noch Ihre Knarren vergessen. Ihr Vorgesetzter hat nämlich gerade bestätigt, dass der Kerl, der gerade in Freiburg hackt, euer Mann ist. Er benutzt dasselbe Programm, wie auch beim letzten und vorletzten Mal.}

\par

Laura nickte zufrieden. Wie aufs Stichwort zog ihr Partner seine Waffe aus der Mantelinnentasche und überprüfte die Munition. Er hatte eine ordinäre Kugelpistole bei sich, wie sie bei der Polizei und teilweise auch beim Geheimdienst üblich war. Nur die Munition war etwas besonderes. Die Geschosse bestanden aus einem Schaummaterial, in dem ein kleiner Injektor voller Betäubungsmittel eingelagert war. Das machte Kugelwaffen im Gegensatz zu Strahlern bei Einsätzen der Polizei zur beliebteren Wahl. Außerdem war es erst vor kurzem gelungen, Linsen für Strahlenwaffen so klein herstellen zu können, damit sie in eine Pistole passten.

\par

Klaus war sehr überrascht gewesen, als er zum ersten mal bemerkt hatte, dass seine sonst so sanfte Partnerin gleich zwei Waffen mit sich führte. Zumindest wenn sie im Einsatz war. Sie besaß im Gegensatz zu den meisten Polizisten eine Strahlenpistole.

\par

Beide pirschten ins Cockpit vor. Laura wurde, wie immer, ein wenig schlecht, als sie aus dem Fenster sah. Die Fähre flog einen parabelförmigen Kurs, der sie zuerst dem Himmel entgegen und dann ans Ziel bringen würde. Obwohl die Trägheitsabsorber auf voller Leistung liefen, waren die großen Beschleunigungskräfte, die auf alle Insassen wirkten, noch deutlich zu spüren. Nur aufgrund der künstlichen Schwerkraft konnte Laura aufrecht stehen. Ohne sie wäre sie schon kurz nach dem Start auf die Rückwand des Passagierbereichs gefallen.

\par

Der Kopilot hatte sich gerade angeschnallt, als er sich umdrehte und sagte: \WR{Keine Sorge. Wir sind in Freiburg gelandet, bevor Sie überhaupt anfangen können, sich zu übergeben.}

\par

Wie aufs Stichwort kippte die Fähre nach vorne ab und begann auf die Wolkendecke zu zu stürzen. Als das Shuttle die weiße Schicht durchflogen hatte, zündete der Pilot die Bremsraketen. Düsen am vorderen Teil des Shuttles spieen capezingenährtes Feuer aus und bremsten das Fluggerät abrupt ab. Noch recht klein und unscheinbar war die Stadt Freiburg bereits durch die Frontscheibe zu erkennen. Sie erstreckte sich um die ersten Bergmassiven des Schwarzwalds. Nur in der Innenstadt fanden sich noch Häuser im Stil der Gründerjahre der Stadt. Die Seuche hatte das Gesicht Freiburgs für immer verändert. Die meisten Bauten waren im Gusssteinstil des späten zweiundzwanzigsten Jahrhunderts errichtet worden aber es gab auch einige Wolkenkratzer, die aus späteren Epochen stammten. Vor allem die Gebäude, die an der Grenze der Rheinebene standen, waren perfekte Beispiele für die bevorzugte Glasarchitektur der Union.

\par

Die Fähre bremste noch um einiges stärker ab. Ein Vorgang, bei dem Laura Ghetas zunehmend übler wurde. Doch die Tatsache, dass das Schiff gleich gelandet sein würde, beruhigte sie etwas.

\par

\WR{Achtung. Hier Shuttle Alpha neun vom Kresimir-Stützpunkt in Berlin. Wir befinden uns auf einem Polizeieinsatz und erbitten Landeerlaubnis}, sprach der Kopilot in das Mikrofon seines Headsets.

\par

Die Antwort der Flugleitung bekam Laura nicht mit aber kurz darauf landete die Fähre in der Nähe des Altbahnhofs auf einer öffentlichen Plattform.

\par

\WR{Viel Glück}, rief der Kopilot Laura und Klaus entgegen, als diese mit gezogenen Waffen aus der Fähre sprangen.

\par

Das Shuttle hob sofort wieder ab, nachdem es die beiden verlassen hatten, denn die Plattform, auf der es gelandet war, diente normalerweise als Landeplatz für medizinische Notfälle. Daher war eines der zentralen Krankenhäuser auch nicht fern.

\par

\WR{Hier lang!}, brüllte Klaus seiner Kollegin zu. Er musste aus voller Kehle schreien um den Lärm der startenden Fähre zu übertönen.

\par

Beide rannten die Wendeltreppe der Plattform hinab. Ihre Schritte klangen laut auf dem Metall der Stufen und das Gestänge schwang so heftig, dass Laura glaubte, es könne jeden Moment auseinanderfallen. Klaus hatte sich den Weg zu jedem Verteilerknoten genau eingeprägt, als der ganze Einsatz noch in der Planung gewesen war. Sein überdurchschnittlich gutes Gedächtnis war Laura nach wie vor ein großes Rätsel.

\par

Einige Leute traten mit überraschten, teils sogar erschrockenen, Minen zur Seite, als Laura und Klaus über die Straße rannten. Die schwarzen Mäntel der beiden flatterten im Winterwind, als sie eine Häuserecke bogen.

\par

Laura erkannte sofort die offene Luke, die den Zugang zum unterirdischen Datenverteilerknoten markierte. Eines stand fest, der Hacker war gut in seinem Fach. Die Öffnungsautomatik der tonnenschweren Stahlplatte war nur schwer zu umgehen. Aber der Datendieb hatte es offensichtlich geschafft, denn aufgestemmt konnte er sie kaum haben.

\par

\WR{Polizei! Kommen Sie raus! Hände nach oben!}, schrie Klaus Rensing, als er mit der Waffe im Anschlag in den unterirdischen Zugang peilte. Laura zog sofort nach. Sie ließ den Suchstrahl ihrer Pistole nie aus den Augen, als sie in den Schacht blickte.

\par

\WR{Scheiße!}, fluchte Klaus lautstark, als er und Laura erkannten, dass der Zugang leer war.

\par

Eine Klappe am Terminal des Datenverteilers war aufgebrochen worden und ein Verbindungskabel hing an einer Seite lose heraus.

\par

\WR{Wir kommen zu spät}, hängte Klaus etwas leiser an. \WR{Der Kerl arbeitet ja wahnsinnig schnell.}

\par

Laura seufzte und lies ihre Waffe sinken. Sie hatte versagt. Der Täter hatte fliehen konnten. Je nachdem, wie lange er schon weg war, konnte er bereits in irgendeiner Gasse verschwunden sein. Entmutigt ließ sie ihren Kopf hängen.

\par

Aber als sie auf den schmutzigen, halb geschmolzenen Schnee hinab blickte, erkannte sie etwas. Einige Fußspuren führten vom Schacht weg und in Richtung Straße. Der Schnee in den Abdrücken hatte begonnenzu schmelzen, war aber noch nicht völlig zu Wasser geworden. Jemand, der die Verflüssigungschemikalie von den Straßen an den Schuhen gehabt hatte, war aus dem Schacht gekommen. Lange konnte es nicht her sein, denn sonst wäre der Schnee schon vollkommen weg.

\par

Aufgeregt packte Laura ihren Kollegen am Arm und sprach hastig: \WR{Er kann noch nicht weit sein! Stell Kontakt zur hiesigen Sicherheitsstelle her. Wir brauchen Verstärkung!}

\par

Während Klaus sein Funkgerät zückte, rannte Laura zurück in Richtung Straße. Zügig aber nicht hektisch ließ sie ihren Blick durch die Menschenmassen gleiten. Die meisten Passanten ignorierten sie völlig. Nur einige erwiderten den Blickkontakt. Aber keiner von ihnen wirkte verdächtig.

\par

Klaus kam ebenfalls herbei und stellte sich neben Laura. \WR{Ich hab die Polizeistelle informiert}, berichtete er atemlos. \WR{Aber die brauchen ein bis zwei Minuten um hierher zu kommen.}

\par

Laura nickte konzentriert. Irgendeiner der vielen Menschen auf der Straße war ihr Mann. Aber Laura schätzte die Menge auf mindestens fünfzig Personen, was das Problem nicht vereinfachte.

\par

\WR{Da!}, gab Klaus plötzlich laut von sich und zeigte auf eine Person, die zügig in Richtung des Altbahnhofs lief.

\par

Eigentlich unterschied er sich nicht sehr von den anderen Menschen, die auf der Straße umhergingen aber Laura war sich sofort sicher, dass ihr Kollege den Täter erspäht hatte.

\par

\WR{Hinterher}, forderte sie und rannte los.

\par

Die Person wandte sich kurz um und schien Ausschau zu halten. Es war ein mittelgroßer unauffälliger Mann, mit Schnurrbart und Seitenscheitel, der einen dicken, grauen Gehrock trug. Als er Laura sah, beschleunigte er seinen Schritt.

\par

Laura und Klaus begannen ebenfalls schneller zu Rennen. Noch war sich der Täter offenbar nicht sicher ob er bereits aufgeflogen war. Immer wieder sah er unauffällig in die Richtung der beiden heran rennenden Ordnungshüter.

\par

Dann schien ihm die Situation zu gefährlich zu werden. Er warf den Gehrock beiseite und rannte los. Seine Geschwindigkeit beeindruckte Laura und Klaus gleichermaßen und er hatte einen Vorsprung von fast hundert Metern.

\par

\WR{Stehen bleiben!}, rief ihm Laura hinterher~– erwartungsgemäß erfolglos.

\par

Der Flüchtende überquerte eine weiter Straße und rannte in den Altbahnhof hinein. Ein Gebäude, dass noch aus dem einundzwanzigsten Jahrhundert stammte und aus nostalgischen Gründen renoviert und stehen gelassen worden war. Die Station war natürlich viel zu klein für eine Stadt mit den Ausmaßen Freiburgs, daher war ein zeitgemäßerer Bahnhof im ökonomischen Viertel errichtet worden.

\par

Nur wenige Sekunden später rasten auch Laura und Klaus in den Bahnhof. Ein paar Passanten sprangen schreiend zur Seite, als die beiden mit gezogener Waffe durch die Tür gerannt kamen. Einige blieben wie versteinert stehen und fünf oder sechs Personen begannen laut zu schimpfen, nachdem sie sich vom ersten Schrecken erholt hatten.

\par

Laura hielt nach dem Hacker Ausschau. Er konnte noch nicht weit gekommen sein. Einen Augenblick später entdeckte sie einen fallen gelassenen Pullover, der wahrscheinlich vom Täter stammte. Er hatte ihn mit Sicherheit abgeworfen um wenigstens ein bisschen anders auszusehen und in der Menge abtauchen zu können.

\par

Doch Lauras wache Augen fanden ihn schnell wieder. Sie zeigte mit dem Finger in die Richtung des Flüchtenden und sagte zu Klaus: \WR{Er läuft auf die Gleise zu.}

\par

Der Angesprochene rief einige Anweisungen in sein Funkgerät, während er Laura folgte. Die Bahnhofshalle war keine siebzig Meter lang, darum hatten sie den ersten der neun Bahnsteige schnell erreicht.

\par

Nachdem Klaus die Informationen an die lokale Polizei weitergegeben hatte, stellte er Kontakt zur Bahnhofsleitung her. Nachdem der Kanal stand begann er: \WR{Hier spricht Klaus Rensing. Ich bin Polizist im Einsatz. Ich verfolge einen Flüchtigen. Er versucht wahrscheinlich in einen der Züge zu kommen. Stoppen Sie sofort den ganzen Verkehr.}

\par

\WR{Was?}, quäkte eine perplexe Stimme aus dem Funkgerät. \WR{Wieso? Was ist hier eigentlich los? Haben sie überhaupt eine Befugnis für so eine Anweisung?}

\par

Während Klaus begann mit der Person an der anderen Leitung zu streiten, lief Laura den ersten Bahnsteig entlang. Ein Zug hielt gerade dort und versperrte den Blick auf die anderen Gleise. Als sie das Ende der Magnetbahn erreicht hatte, erkannte sie den Täter sofort wieder. Er stand am Bahnsteig zwischen dem ersten und zweiten Gleis und sah hektisch um sich.

\par

Laura tat, woran sie schon seit einiger Zeit immer wieder gedacht hatte und sprang auf die Gleise. Noch während sie aufkam, fragte sie sich, ob es ihr wirklich nichts ausmachte, dass jeden Moment ein Zug aus der anderen Richtung kommen konnte. Üblicherweise hätte sie sich nun eine Zeit lang ihren düsteren Gedanken gewidmet. Doch der Adrenalinschub, den ihr die Verfolgung gab, ließ dies nicht zu. Ihr ganzes sportliches Können aufwendend, schwang sie sich über die schweren Pfeiler wie über einen Bock im Sportunterricht.

\par

\WR{Stehen bleiben!}, schrie Sie und zielte mit ihrer Waffe auf den Hacker.

\par

Dieser machte große Augen, als er sie sah und begann den Bahnsteig entlang zu rennen. Laura, die die Schienen gerade überquert hatte, überlegte ob sie versuchen sollte, den Flüchtenden zu betäuben. Aber es standen zu viele andere Menschen in der Nähe und selbst wenn sie gut zielte, konnte sie einen von ihnen treffen. Ihre Hände zitterten und sie wusste nicht, ob es die Kälte, die Anstrengung oder etwas anderes war.

\par

So verfolgte sie den Flüchtenden zu Fuß weiter. Erleichtert sah sie, wie ein weiterer Schwebezug gerade dabei war, auf dem dritten Gleis einzufahren. Er konnte also nicht mehr über die Schienen springen.

\par

Umso überraschter war Laura, als er es doch tat und direkt vor die Lok geriet.

\par

Für einen Moment hielt sie den Atem an, als sie dachte, dass die Lokomotive ihn erfasst hätte. Der Zugführer stieg sofort in die Eisen und ließ seine Bahn ruckartig abbremsen.

\par

Aber Laura erkannte kein Blut oder irgend ein anderes Zeichen, das darauf hindeutet, dass der Hacker vom Zug erfasst worden wäre. Für einen Sekundenbruchteil blieb sie wie festgewachsen stehen. Sie hatte schon ein paar Verfolgungen hinter sich. Aber war noch nie jemandem gefolgt, der so ohne Rücksicht auf Verluste floh. Besonders nicht bei einem Datenverbrecher, denn viele von ihnen kamen sogar nach wenigen Monaten wieder frei, wenn sie einen geschickten Anwalt hatten. Ein Hacker war die letzt Art von einem Kriminellen, dem Laura so eine Flucht zugetraut hatte. Nur bei einem Handtaschendieb hätte es sie noch mehr überrascht.

\par

Dann rannte sie weiter, auf das Ende des Zugs zu. Es war nur eine kurze Nahverkehrsbahn, daher hatte sie den letzen Wagon schnell umrundet. Nun lag es an ihr, ebenfalls keine Rücksicht auf Verluste zu nehmen.

\par

In vager Hoffnung, dass Klaus, den sie unterdessen völlig aus den Augen verloren hatte, es bereits geschafft hatte, den Zugverkehr zu unterbrechen, hielt sie Ausschau nach anderen Zügen und rannte schließlich über die Schienen. Sie überquerte alle neun Gleise, fand den Flüchtigen aber nicht wieder.

\par

Erst eine halbe Minute später, erkannte sie einen Stofffetzen, der an einem aufgebrochenen Kanalzugang, jenseits der nächsten Straße hing. Abermals ihre ganze Geschicklichkeit aufbringend, schwang sich Laura über den Begrenzungszaun des Bahnhofs und rannte zu diesem Zugang. Aber schon auf dem Weg dorthin, wurde ihr klar, dass sie den Hacker verloren hatte. In der Kanalisation gab es Hunderte von Abzweigungen. Sie abzusuchen war so langwierig wie sinnlos.

\par

Erst jetzt spürte Laura, wie aufgedreht sie war. Ihr Herz schlug wie eine zehn polternde Trommeln und ihr Puls raste schneller als die Züge, die sie fast überfahren hätten. Mit kontrolliertem ein und ausatmen versuchte sie, sich wieder zu beruhigen.

\par

Plötzlich erklang Klaus Stimme aus ihrem Buch. Sie schlug es auf und ging auf Empfang. \WR{Laura, bist du in Ordnung? Wo steckst du?}, fragte ihr Partner. Er klang ernsthaft besorgt.

\par

Niedergeschlagen stellte sie eine Verbindung zu ihm her und antwortete: \WR{Ich bin auf der Straße hinter den Gleisen. Ich hab ihn verloren. Ich glaub ich hab’s versaut.}

\par

Einen Moment lang blieb alles Still. Dann antwortete Klaus nachdenklich: \WR{Vielleicht nicht ganz. Ich hab zwischen dem ersten und dem zweiten Gleis was gefunden. Der Blödmann hat sein Buch verloren.} 
