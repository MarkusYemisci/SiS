Es war viel zu warm. Den Herbst verband die junge Frau stets mit fallenden Blättern in die buntesten Farben, mit kürzer werdenden Tagen und mit grauem, regnerischem Wetter. All das hätte in jener Situation auch viel besser gepasst. Doch an diesem späten Septembertag schien die Sonne fast wie im Sommer. Der Wald war erfüllt von einer Mischung aus den unterschiedlichsten Gerüchen. Manche waren modrig und passten zu den allgegenwärtigen Pilzen und dem Humus, durch den die Agentin schritt. Andere waren frisch und hatten vielleicht einmal jemanden zu dem Ausdruck \Wr{guter Luft} inspiriert. Doch nichts von alledem hätte Laura Gethas mit ihrer Lage assoziiert.

\par

Sie war nun so allein, wie sie sich an jenem Septembertag gefühlt hatte. Damals war sie von etlichen Polizisten umschwärmt worden, die hin und wieder nach Instruktionen fragten. Nun aber war weit und breit niemand auf der kleinen Lichtung zu sehen.

\par

Der Wald an sich war sehr schön. Eine bunte Mischung als Tannen, Birken und auch einigen Buchen. Am Boden reckten sich vereinzelt Sträucher in die Höhe. Doch alles in allem war das Kronendach zu dicht, um viel Vegetation am Boden zuzulassen. So lag dort eine dicke Schicht von gefallenen Blättern, die teilweise schon von Käfern und kleineren Organismen zersetzt wurden.

\par

Nur an einer einzigen Stelle, etwa rechteckig und gut einen Meter lang, lag so gut wie nichts am Boden. Die Blätter waren entfernt worden und der Untergrund hatte eine andere Farben. Es war eindeutig, dass hier jemand vor kurzem gegraben hatte. Laura wusste auch wer und sie wusste auch, was dort eingebettet unter einer Tonne von schwerer, nasser Erde lag.

\par

Das Wissen darum bereitete ihr Übelkeit. Vor allem aber fühlte sie sich wirklich und wahrhaftig hilflos. In dem Moment, in dem sie in das kleine Waldstück eingestiegen war und das Grab gesehen hatte, war die Hoffnung aus ihrem Leben gewichen.

\par

Laura schrak auf und warf dabei ihre Kaffeetasse um. Leise plätschernd ergoss sich die braune Flüssigkeit auf den Boden. Doch sie hatte nicht die Energie, aufzuspringen und einen Lappen zu besorgen. Stattdessen sah dem Kaffe einfach zu, wie er sich der Gravitation und der Thermodynamik ergab. So wie der Rest der Welt.

\par

Der Traum hatte sie noch viel zu sehr im Griff, als dass sie besonnen hätte reagieren können. Ein Blick auf ihr Buch zeigte ihr, dass es bereits fünf Uhr in der Frühe war. Sie erinnerte sich nicht daran, wann sie eingeschlafen war. Daher wusste sie auch nicht, wie lange sie eigentlich geschlafen hatte.

\par

Sicherlich nicht lange genug, das war ihr klar. Aber es lohnte sich kaum, nach Hause zu gehen. Normalerweise war sie spätestens um acht Uhr Ortszeit im Büro. Und bis sie sich in ihrer Osloer Wohnung eingerichtet hatte, konnte sich praktisch auch gleich wieder aufstehen.

\par

Widerwillig roch sie an ihren Achseln. Ein Besuch im Keller bei den Duschen war unumgänglich, wäre aber dafür auch schnell erledigt. Ihr Blick traf die Couch in der Ecke ihres Büros. Dort konnte sie sicher noch zwei Stunden schlafen. Sie fühlte sich so müde, dass das Einschlafen an sich kein Problem darstellen sollte. Aber sie wagte es trotzdem nicht, die Augen zu schließen. Denn wenn sie das tat, würde sie schon bald wieder auf der Lichtung sein. Und dann würde sie anfangen zu graben.