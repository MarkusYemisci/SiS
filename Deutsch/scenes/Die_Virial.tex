Morten und Anna flogen schon seit einiger Zeit in einer Standardformation per Autopilot. Die Schwerelosigkeit begann Morten langsam aber sicher etwas unheimlich zu werden. Allerdings kaum mehr, als der Flug durch den falchen Hyperraum. Auch wenn dieser bislang kaum spürbar geblieben war. Nachdem er es anfangs sehr genossen hatte, musste er nun mit seinem Magen kämpfen. Die Null-g-Simulatoren der Flugschule waren nicht annähernd in der Lage gewesen den Zustand des freien Falls wirklich korrekt nachzustellen.

\par

Hinzu kam noch das merkwürdige Licht des Sterns, das so intensiv war, dass es sogar manchen Planeten zum leichten Glimmen brachte.

\par

Abgesehen davon war der Flug bislang ereignislos geblieben. Das Interessanteste, das Morten getan hatte war, seine Abtaster nach dem verloren Forschungsschiff suchen zu lassen.

\par

Die Funkkanäle der einzelnen Staffeln waren mittlerweile tot. Während eines echten Einsatzes galt Funkstille, die nur als bestimmten Gründen durchbrochen werden durfte. Umso mehr verwunderte es Morten, als sich Anna Farley plötzlich über einen gesonderten Kanal meldete: \WR{Okay Morten. Wir fliegen noch mindestens zweieinhalb Stunden hier draußen herum. Wollen Sie vielleicht ein bisschen von sich erzählen, damit ich weiß, wer hinter der Bezeichnung grün zwei steckt.}

\par

Morten war gelinde gesagt überrascht. War das eine Art Test? Wollte Anna wissen ob er sich an die Regeln hielt, egal unter welchen Umständen?

\par

Unsicher antwortete er auf demselben Kanal: \WR{Ähem, Anna. Verstehen Sie mich nicht falsch aber ist es nicht untersagt Privatgespräche während eines Einsatzes zu halten?}

\par

\WR{Da hast du Recht, Morten}, antwortete seine Vorgesetzte sofort. \WR{Und weshalb ist das untersagt?}

\par

Morten brauchte nicht lange nachzudenken. \WR{Nun, man wollte vermeiden, dass Feinde den Funkverkehr auffangen.}

\par

\WR{Ganz recht}, stimmte Anna zu. \WR{Aber momentan tastet jedes Schiff der Starforce in diesem System nach der \EN{Virial} ab. Wenn wirklich Gegner hier wären, würden unsere Abtasterstrahlen auf ihrem Radar aufleuchten wie ein Hochleistungsscheinwerfer.}

\par

Morten blieb ruhig. Er fragte sich, ob Annas Antwort ein Tadel gewesen sein sollte. Doch als sie weitersprach war ihr Ton gelassen und ruhig. \WR{Keine Sorge Morten. Du wirst bald feststellen, dass es sinnvolle und weniger sinnvolle Regeln gibt. Und bald hast du ein Gespür dafür, welche Regeln wirklich wichtig sind. Mit anderen Worten: Du kriegst schnell raus, wie der Hase läuft.}

\par

Morten atemte erleichtert auf, stellte dabei aber sicher, dass Mikrofon nicht sendete. Anna gefiel ihm immer besser und besser. Aber er musste sich konzentrieren und konnte sich nun keine Gedanken an sie erlauben. Immer noch ernst klingend fragte er in sein Mikrofon hinein: \WR{Aber unsere Gespräche werden vom Flugschreiber mitgeschnitten. Werden diejenigen, die unsere Datenbänke auswerten, das mit den Regeln genauso sehen?}

\par

Kurz glaubte Morten Anna beherrscht lachen zu hören. \WR{Womöglich nicht}, antwortete sie amüsiert. \WR{Aber ich kann dir versichern, Captain Fiscale reißt niemandem wegen solchen Lappalien den Kopf ab. Sie kann ein richtig harter Knochen sein und Strenge steht bei ihr auf der Tagesordnung. Aber sie weiß, worauf es ankommt.}

\par

\WR{Okay}, antwortete Morten immer noch ein wenig nervös.

\par

Anna fuhr fort: \WR{Also. Dein Name ist Morten. Ziemlich ungewöhnlich. Das ist ein Wort aus einer der alten Sprachen oder? Ich glaube Norwegisch.}

\par

\WR{Richtig}, antwortete der Gefragte. \WR{Oder Dänisch. Aber bitte fragen Sie nicht, was er bedeutet. Ich komme nicht von der Erde. Ich habe mir daher nie viel Wert darauf gelegt.}

\par

\WR{Ich habe in deiner Akte gelesen, dass du von Corna stammst. Ist das wirklich so eine schlimme Ecke, wie man immer sagt?}

\par

\WR{Na ja}, fing Morten jetzt ruhiger an zu antworten. \WR{Das Leben ist recht rau dort, das stimmt. Es ist eben ein unfruchtbarer, felsiger Wüstenplanet. Mann nennt ihn nicht umsonst den Planet auf dem man die Bäume zählen kann. Und die Leute haben ihren...}, Morten unterbrach sich, \WR{ganz eigenen Charakter. Prügeleien an der Schule und zum Teil auch mitten auf der Straße waren keine Seltenheit. Aber ich habe niemals erlebt, dass jemand zusammenschlagen wurde, der einer Rauferei aus dem Weg gehen wollte. Dafür haben die Schulen und die Sicherheitsleute schon gesorgt.}

\par

Nach einigen Augenblicken warf Anna ein: \WR{Ich habe gehört, dass man Cornaer schon von weitem erkennt. Es heißt, Größe und kantige Gesichtsformen wären für den Planeten typisch. Aber in deinem Fall stimmt das wohl nicht.}

\par

\WR{Vorurteile, schätze ich. Aber ich bin kein gebürtiger Cornaer}, entgegnete Morten schnell. \WR{Wo ich eigentlich herkomme weiß ich nicht. Meine leiblichen Eltern haben ich nie kennen gelernt. Ein Pärchen auf Corna hat mich adoptiert, als ich ein halbes Jahr alt war.}

\par

Morten stockte. Warum er Anna dieses erzählt hatte, war ihm selbst nicht klar. Er fand ausgesprochen langsam neue Freunde. Und selbst wenn, dauerte es normalerweise Monate, bis er ihnen solch persönliche Dinge anvertraute. Irgendwie kam Morten nicht gegen den Drang an, Anna einfach zu vertrauen und zu versuchen, ihr näher zu sein. Dabei hoffte er, sich ihr nicht zu sehr aufzudrängen.

\par

\WR{Das war bestimmt hart}, entgegnete Anna nach einer längeren Pause. Selbst durch den Kopfhörer ging das Mitgefühl in ihrer Stimme nicht verloren.

\par

Morten antwortete: \WR{Stimmt schon. Aber meine Pflegeeltern haben sich wirklich Mühe gegeben, dass ich mich bei ihnen wohlfühlen konnte. Trotzdem kam ich ihnen nie wirklich nah. Sie haben mir von Anfang an nichts verheimlicht und ich wusste, dass ich nicht ihr Kind war. Bis heute hab ich übrigens keine Ahnung wer meine leiblichen Eltern sind… oder gewesen sind.}

\par

\WR{Hast du deine Pflegefamilie nie gefragt?}, wollte Anna wissen.

\par

Morten seufzte. \WR{Ich wollte. Aber irgendwie habe ich mich nicht überwinden können. Und als ich siebzehn wurde und vielleicht langsam mal den Mut entwickelt hatte, sind meine Stiefeltern leider bei einem Unfall gestorben. Obwohl ich mich nie als ihr Sohn gefühlt hatte, hat mich das ziemlich hart getroffen. Ich hab dann bei meinem Paten gewohnt bis ich achtzehn war. Den hat es nicht sonderlich interessiert, was ich so trieb. Er sah sich eher als so eine Art Verpflegungsmaschine und war, glaube ich, recht froh als ich mir eine Arbeit gesucht hatte und ausgezogen bin.}

\par

\WR{Was hast du gearbeitet?}, fragte Anna interessiert.

\par

\WR{Ich hab mich in einer Herstellungshalle für Schwebezugmotoren verdingt. Das war ganz schöne Knochenarbeit und das schlimmste war, ich dachte ich müsste das für den Rest meines Lebens tun. Aber dann kam plötzlich ein Typ namens Martin Siegel in die Firma. Er sagte, er würde das nur als Zwischenstation sehen um später was anderes zu lernen. Er hat richtig daran geglaubt und härter gearbeitet, als alle anderen. Das hat mich beeindruckt.}

\par

\WR{Hatte er die Idee, zur Flugschule zu gehen?}, erklang Annas Stimme aus Mortens Kopfhörer.

\par

\WR{Nein}, antwortete dieser sofort. \WR{Die kam von mir. Irgendwie hat mich sein Tatendrang  angesteckt und ich habe angefangen zu träumen.}

\par

Einen Moment blieb es still, dann empfing Morten Anna: \WR{Du bist also so eine Art Vagabund?}

\par

Morten wollte gerade eine Antwort geben, als auf seinem Radar etwas anfing zu blinken. Der Blickpunkt, dargestellt durch einen Kreis mit einem Fragezeichen in der Mitte, war gelb gefärbt was bedeutete, dass der Computer zwar die Anwesenheit eines vermeintlich künstlichen Objekts registrierte, es aber nicht identifizierne konnte.

\par

Hastig ging Morten auf die übliche Frequenz der Staffel und rief Anna: \WR{Grün eins, ich habe hier etwas auf dem Radar. Nicht identifiziertes Flugobjekt, Peilung eins drei acht Grad zu neun eins neun. Sehen Sie das auch?}

\par

\WR{Hier grün eins}, antwortete Anna sofort in viel ernsterem Ton. \WR{Ich habe es ebenfalls auf dem Schirm. Meine Abtaster führen gerade eine Analyse durch. Ich sage der \EN{Regenvogel} Bescheid, vielleicht ist das unser großer Fang.}

\par

Nur einen Wimpernschlag danach erklang Annas Stimme auf dem Flottenkanal: \WR{Hier ist Major Farley. Ich rufe die \EN{Regenvogel}. Wir haben hier etwas gefunden, könnte die \EN{Virial} sein. Bitte um Instruktionen.}

\par

Morten sah sich unterdessen die Peilung etwas genauer an. Erst traute er seinen Augen nicht aber nachdem er die Koordinaten ein zweites mal mit einer Karte des Systems verglichen hatte, war er sich sicher.

\par

\WR{Grün zwei an grün eins}, sprach Morten in sein Mikrofon. \WR{Das Signal kommt aus einem Asteroidenfeld und zwar ziemlich aus der Mitte. Kommt ein Schiff der \EN{Ereignishorizont}-Klasse da überhaupt hin?}

\par

Einen Moment geschah nichts, dann meldete sich Anna und wies an: \WR{Rufen Sie sie, grün zwei.}

\par

Morten programmierte seinen Kommunikationssystem darauf, auf maximaler Leistung auf allen Kurzstreckenfrequenzen zu senden. \WR{Hier ist second Lieutenant Wittwer von der \EN{Regenvogel} KlT. Ich rufe die \EN{Virial}, bitte melden.}

\par

Nachdem nach einer halben Minute keine Antwort erklang, wiederholte Morten seinen Ruf. Ebenfalls ohne beantwortet zu werden. Er wollte sich gerade wieder bei Anna melden, als diese durchgab: \WR{Grün zwei, ich erreiche die \EN{Regenvogel} nicht. Haben Sie Kontakt zur \EN{Virial}?}

\par

\WR{Nein, Madam}, antwortete Morten kurz angebunden.

\par

Seine Brust verengte sich. Das ein Träger nicht mehr zu erreichen war, war nicht wenig ungewöhnlich. Die \EN{Regenvogel} hatte einen Nullzonentranciever, mit dem sie spielend ins nächste Planetensystem funken konnte. Dass sie zu ein paar Jägern nicht mehr durchkam, die nur wenige tausend Kilometer von ihr entfernt kreuzten, war eigentlich unvorstellbar.

\par

Anna Farley wiederholte ihren Ruf den herkömmlichen Frequenzen: \WR{Grün eins an \EN{Regenvogel}. Bitte antworten.}

\par

Doch auch diesmal gab es keine Antwort. Morten fragte sich, ob er es ebenfalls versuchen sollte. Aber ihm wurde klar, dass dies ziemlich sinnlos wäre. Wenn Anna den Träger nicht erreichen konnte, dann war er, der den tupfengleichen Jäger flog, wohl ebenfalls nicht dazu imstande. Außerdem wollte er nichts über ihren Kopf hinweg unternehmen.

\par

Dennoch blieben seine Augen auf das Radar geheftet. Immer noch blinkte der Blickpunkt in gelber Farbe auf. Die \EN{Regenvogel} war schon vor einiger Zeit vom Radar verschwunden, was Morten aber nicht überrascht hatte. Selbst die hochentwickelten Abtaster seines Aufklärers hatten ihre Grenze. Der Träger befand sich außerhalb der Reichweite passiver Systeme.

\par

\WR{Morten}, begann Anna. Dass sie ihn auf dem offiziellen Geschwaderkanal mit Namen ansprach beunruhigte ihn noch mehr. \WR{Ich habe immer noch keinen Kontakt zur \EN{Regenvogel}. Aber die Abtaster sind mit der Analyse durch. Der Kontakt, den wir sehen, ist ein Unions-Transpondersignal, das ziemlich wahrscheinlich von der \EN{Virial} stammt.}

\par

Ein Moment der Ruhe folgte. Morten vermutete, dass Anna gerade angestrengt nachdachte. Er hatte zwar keine Ahnung, was gerade vorging aber er spürte deutlich, dass etwas nicht stimmte.

\par

\WR{Wir fliegen nach Hause und holen Hilfe}, verkündete Anna. \WR{Um mehr zu erfahren müssten wir ins Asteroidenfeld fliegen und nachsehen aber das kann ich mit ihnen nicht riskieren. Seien Sie mir nicht böse aber Sie sind noch Anfänger, Sie schaffen das nicht.}

\par

\WR{Nein!}, rief Morten aus Reflex in sein Mikrofon und bereute sofort, dass er einer Vorgesetzten gegenüber laut geworden war.

\par

Doch zu seiner Überraschung schien Anna keineswegs verärgert. \WR{Morten, das würde wirklich hart werden. Das Asteroidenfeld ist verdammt dicht. Wie dicht wissen wir nicht mal, wir waren schließlich noch nie hier.}

\par

\WR{Madam}, begann Morten eindringlich, \WR{Sie dürfen jetzt keine Rücksicht auf mich nehmen. Wenn die \EN{Virial} tatsächlich in Schwierigkeiten steckt, dann dürfen wir keine Zeit verlieren. Sie haben mich als Flügelmann ausgewählt. Bitte lassen Sie mich meine Arbeit machen.}

\par

Wieder folgte ein langer Moment der Ruhe. Vermutlich war Anna hin und her gerissen zwischen Rücksichtnahme und dem Gedanken an die Besatzung der \EN{Virial}. Es kam Morten vor, als sei eine Ewigkeit vergangen, als sich Anna wieder meldete: \WR{In Ordnung, wir fliegen. Aber seien Sie vorsichtig. Denken Sie daran, wir erreichen die \EN{Regenvogel} nicht. Wir müssen es auch wieder zu ihr zurückschaffen, wenn wir das Forschungsschiff tatsächlich finden. Es hilft nichts, wenn wir danach an ein paar Felsbrocken kleben.}

\par

\WR{Bestätigt, ich gebe mir alle Mühe, grün eins}, antwortete Morten entschlossen.

\par

\WR{Verlasse flachen Hyperraum. Abkopplung der Schlitten vorbereiten.} Anna lies ihren Jäger auf Unterlichtgeschwindigkeit zurückfallen und sprengte die Zusatztriebwerke ab, die daraufhin mit sämtlichen Positionsleuchten anfingen zu blinken. Auf diese Weise sollten spätere Andockvorgänge erleichtert werden. Ohne den Schlitten war ihr Schiff noch um einiges wendiger und auf kürzere Distanz besser zu beschleunigen.

\par

\WR{Folgen Sie mir, grün zwei}, befahl sie tonlos.

\par

Morten hängte mimte seine Flügelfrau und gab ebenfalls Vollschub. Das Gefühl war mehr als unangenehm. Es kam ihm vor, als hätte sich gerade ein tonnenschwerer Bulle auf ihn gesetzt, der ihn nun in seinen Sitz presste. Doch nach kurzer Zeit hatte er eine konstante Geschwindigkeit erreicht und sauste zusammen mit Anna auf das Asteroidenfeld zu, dass immer deutlicher im Lichte Arkturs zu erkennen war.

\par

Erstmals war er froh, einen Aufklärer zu fliegen. Wenn es nur ein Jägertyp durch ein Feld voller gefrorenem Eis und messerscharfen Felsen schaffen konnte, dann war es ein Aufklärer der Argus-Klasse. Die Wendigkeit und Schnelligkeit seines Fliegers würde Morten nun sicherlich sehr zu gute kommen.

\par

\WR{Denken Sie daran Morten}, erinnerte Anna eindringlich. \WR{Fliegen Sie nicht zu schnell an und wenn es zu eng wird drehen Sie ab! Das letzte das die \EN{Regenvogel} braucht ist ein toter Anfänger auf seiner ersten Mission. Davon hatten wir schon einen und das reicht.}

\par

Morten schluckte. Schlagartig wurde ihm wieder bewusst, dass er die Ausrüstung eines Gefallenen benutzte. Inständig hoffte er, dass es wirklich kein schlechtes Omen war aber daran zu glauben war mittlerweile recht einfach geworden.

\par

Das Asteroidenfeld schien näher zu kommen. Auch die Felsbrocken, die das Feld bildeten, glänzten seltsam durch das intensive Licht des Sterns Arktur. Es musste wirklich ziemlich groß sein, denn Morten musste seinen Kopf drehen um den Anfang und das Ende erkennen zu können und das aus einer Entfernung von mehreren Millionen Kilometern.

\par

Während die beiden Jäger auf das Feld zu brausten, programmierte Morten sein Radar so, dass sein Zielmonokel jeden Asteroiden markierte, der mit ihm auf Kollisionskurs war. So würde er es einfacher haben, den Steinen und Eisklötzen auszuweichen. Natürlich war das Durchqueren eines Asteroidenfelds Lehrinhalt auf Mortens Flugschule gewesen. Aber sein damaliger Fluglehrer hatte jeden Studenten einzeln gewarnt niemals auch nur in die Nähe eines solchen Feldes zu fliegen wenn es sich irgendwie vermeiden lies.

\par

Plötzlich kam Morten ein Gedanke. Im Moment schien nur Kurzstreckenkommunikation auf aller nächste Nähe zu funktionieren. Die \EN{Regenvogel} war aber unerreichbar. Genauso unerreichbar wie die \EN{Virial}. Was auch immer den Kontakt zum Träger störte, konnte auch dafür verantwortlich sein, dass das Forschungsschiff nicht antwortete.

\par

Während er noch grübelte kamen die beiden Jäger dem Asteroidenfeld immer näher. Mittlerweile waren einzelne Felsen bereits deutlich zu erkennen. Viele  erschienen in einem kalten grau, manchmal auch eher braun. Die wirkliche Farbe war schwer zu erahnen, da das orange braune Licht der Sonne einfach alles anzustrahlen schien. Die Eisbrocken wirkten durch die Beleuchtung ein wenig wie Bernsteine.

\par

Irgendwann hielt es Morten nicht mehr aus und er fragte in sein Funkgerät: \WR{Anna, haben Sie irgendeine Idee, wieso wir keinen Kontakt mehr zur \EN{Regenvogel} haben. Oder was das Forschungsschiff in einem Asteroidenfeld gesucht haben könnte?}

\par

\WR{Nein, wirklich nicht}, antwortete Anna nach einer ganzen Weile. \WR{Meine Abtaster haben kurzzeitig ein schwaches Hyperraumfeld aufgespürt. Woher es kommt oder was es genau ist weiß ich nicht aber vielleicht ist es die Ursache für unseren Kommunikationsverlust.}

\par

Morten ersparte es sich, seine Einschätzung zum besten zu geben. Dieses Feld war zwar eine neue Entdeckung, konnte aber so gut wie alles bedeutet. Vielleicht war es nur eine Auswirkung der überaktiven Sonne. Außerdem hatte Anna so geklungen, als wäre sie gerade nicht auf fröhliches Rätselraten aus.

\par

Die beiden Jäger hatten das Asteroidenfeld fast erreicht. Mortens Zielmonokel begann bereits, einige Felsbrocken zu markieren, mit denen er zusammenstoßen würde, wenn er nicht auswich. Erst jetzt erkannte er, dass der ganze Meteoritengürtel von einer auslandenden Staubwolke umgeben war, die im Schein der Sonne blaugrün leuchtete. Es sah gespenstisch aus und erinnerte Morten an ein Bild, dass er einmal gesehen hatte. Darauf war ein Friedhof zu sehen gewesen über dem ein dichter Nebel gehangen war.

\par

Was die ganze, unheimliche Atmosphäre noch verstärkte war die absolute Ruhe, die meistens im Weltall herrschte. Ohne ein Übertragungsmedium war keine Klangquelle in der Lage, gehört zu werden. Und so erklang für Morten lediglich das Brummen seines Antriebs.

\par

Kurz vor dem Anfang des Asteroidenclusters veränderte sich die Farbe des Blickpunkts auf Mortens Radar. Der Computer identifizierte es nun eindeutig als Unionssignal und die Farbe veränderte sich von gelb nach blau, was für einen zivilen Kontakt stand.

\par

\WR{Letzte Chance noch umzudrehen}, warnte Anna über Funk.

\par

Morten schüttelte den Kopf, obwohl Anna ihn nicht sehen konnte und antwortete: \WR{Nein, wir sind fast da. Mein Computer hat das Signal gerade als eines aus der Union identifiziert.}

\par

Ohne ein weiteres Wort flog Anna geschickt in das Asteroidenfeld ein. Behände ließ sie ihren Jäger über einen Eisbrocken gleiten und verringerte die Geschwindigkeit ihres Jägers. Morten tat es ihr gleich. Er ging auf ein drittel der maximalen Schubleistung und schlingerte um einen Meteoriten herum. Aus etlichen Simulationen wusste er, wie schwer es war, in einem Asteroidenfeld zu manövrieren aber an die Wirklichkeit kamen diese Trainingseinheiten niemals heran. Morten musste praktisch bereits auf den nächsten Felsen achten, der auf ihn zukam, während er den ersten umrundete. Immer wieder schnellten weiße Klammern auf seinem Zielmonokel hin und her und markierten Brocken, mit denen er zusammenstoßen konnte.

\par

Plötzlich musste Morten scharf hochziehen, als ein kleiner Splitter auf ihn zu raste und er hoffte dabei, dass er dadurch nicht gleich mit dem nächstbesten Felsen zusammenprallen würde. Tatsächlich steuerte er nach dem Ausweichmanöver auf einen Asteroiden zu. Aber er hatte noch genügend Zeit, um sich wieder in eine bessere Fluglage zu bringen.

\par

Seine Augen gingen immer wieder auf das Radar, wo er Ausschau nach Annas Jäger hielt. Wenn er ihre Spur verlieren würde, wäre die Aktion gelaufen, denn dann müsste sie zunächst nach ihm suchen und das könnte dauern. Der Blickpunkt, der ihren Jäger darstellte verschwand immer wieder durch allerhand Störquellen die zuhauf in einem Asteroidenfeld vorkamen, was es nicht leichter machte, ihr zu folgen.

\par

Immer wieder musste Morten seinen Jäger die unmöglichsten Ausweichmanöver durchführen lassen um nicht mit einem der Fels- oder Eisbrocken zusammenzuprallen. Hätten sich die Asteroiden nicht bewegt, wäre es nicht derart schwierig gewesen, denn dann hätte er sich einfach langsam hindurchschlängeln können. Aber so musste er ständig aufpassen, nicht von zwei der Brocken zusammengepresst zu werden.

\par

Gerade ließ Morten seinen Jäger nach Steuerbord wegkippen um einem Felsen auszuweichen, der wegen seiner tiefschwarzen Farbe kaum zu sehen war, da stellte er überrascht fest, dass das Asteroidenfeld anscheinend dünner wurde. Schlagartig wurden ihm keine Objekte mehr angezeigt, die mit ihm auf Kollisionskurs waren. Alles was blieb waren einige große Brocken und der grünlich schimmernde Staub, der sich auf Mortens Cockpitscheibe absetzte. Das Transpondersignal das die Radare der beiden Jäger immer noch auffingen waren jetzt ganz nah. Nur noch wenige hundert Kilometer entfernt.

\par

\WR{Ist das normal, dass ein Asteroidenfeld plötzlich so unterschiedliche Formen annimmt?}, fragte Morten verwirrt in sein Funkgerät.

\par

\WR{Das kann ich nicht sagen}, antwortete Anna ehrlich. \WR{Vielleicht ist hier ein Asteroid durchgerauscht, der eine Schneise freigeschlagen hat. Aber das ist nur eine Möglichkeit.}

\par

Morten nickte symbolisch und lies seinen Blick durch die breite Cockpitscheibe gleiten. Eigentlich müsste die Signalquelle bereits in Sichtweite sein, jedoch erkannte er nichts. Kein Licht, kein Triebwerksglimmen nicht mal eine Form, die an ein Raumschiff erinnert hätte. Morten lies seine Abtaster nach Energiesignaturen suchen aber die vielen Störquellen durch die Asteroiden und durch die Sonne würden die Suche einige Zeit lang dauern lassen.

\par

So suchte Morten die Gegend lieber mit bloßem Auge ab. Wenn das Forschungsschiff irgendwo in der Nähe wäre, würde es kaum zu übersehen sein. Die Asteroiden machten ihm mittlerweile keine Probleme mehr. Keiner der Felsen, die weit voneinander verstreut langen, war auf einem Kollisionskurs mit ihm oder bewegte sich besonders schnell. Nur eines fiel Morten auf. Die meisten Brocken hatten auffallend merkwürdige Formen. Auf der einen Seite schien sie eher Glatt zu sein, während sie auf der anderen von Felsvorsprüngen und Kratern nur so strotzten. Sah so wirklich eine Schneise aus, die von einem anderen Asteroiden gezogen worden war?

\par

Morten hielt pausenlos Ausschau. Seinen Jäger hatte er ein wenig beschleunigt um der Signalquelle schneller näher zu kommen. Aber zu sehen gab es nichts. Die meisten Asteroiden und Meteoriten hatten eine graue Färbung. Wäre das Licht der Sonne nicht gewesen hätte das Bild genauso schwarz weiß sein können.

\par

Nur noch einige Dutzend Kilometer trennten Mortens Jäger von der Quelle des Signals. Wenn dort wirklich ein anderes Schiff war, dann würde er spätestens in einer halben Minute damit zusammenstoßen aber da schien einfach nichts zu sein.

\par

Morten kniff die Augen zusammen in der Hoffnung damit mehr erkennen zu können. Aber alles was er sah war, dass der Staub, den er bereits überall im ganzen Feld gesehen hatte, mehr zu werden schien. Aber dann erkannte er, dass die zusätzlichen Partikel kein Staub, sondern eine gefrorene Flüssigkeit waren. Die Farbe war nur schwer zu erkennen, da auch sie angestrahlt wurden. Aber Morten vermutete, dass es Treibstoff war.

\par

\WR{Was zum Teufel ist hier bloß los?}, fragte Anna auf dem Kanal der Staffel.

\par

Das sie ganz offen eine solche Frage stellte, machte ihren Flügelmann noch nervöser. Gerade, als er dachte, das Maximum seiner Beunruhigung sei erreicht, machte er eine neue Entdeckung. Zuerst traute er seinen Augen kaum aber er konnte sich nicht irren. Das menschliche Skelett trieb nur wenige Meter vor seinem Jäger herum. Es war bereits so nahe, dass Morten sogar die Fetzen von Kleidung erkennen konnte, die an den Knochen hingen.

\par

\WR{Verdammte Scheiße!}, brüllte er förmlich in sein Funkgerät.

\par

Kurz darauf trieb er direkt unter dem Skelett hindurch. Er würgte, als er sah, wie Reste von vereistem Fleisch an den Knochen der Leiche hingen. Die Überreste des Körpers waren schwarz und verkohlt. Wahrscheinlich war der Unglückselige bei einer Explosion ins Weltall geschleudert worden. Morten musste sich mehrmals beherrschen und seinen Mageninhalt wieder zurückwürgen als er sich bildlich vorstellte, wie der Mann oder die Frau gestorben war.

\par

Noch schlimmer war aber der Gedanke daran, warum der Mensch gestorben war. Ein Unfall? Morten schloss die Augen um sich das Skelett nicht weiter ansehen zu müssen. Zwar war es ausdrücklich verboten, sich selbst während eines Einsatzes die Sicht zu nehmen aber das war ihm im Moment egal.

\par

\WR{Was ist los, Morten? Was haben Sie gefunden?}, fragte Anna energisch in ihr Mikrofon.

\par

Morten berichtete ihr von der Leiche und den Verbrennungsspuren und seine Vorgesetzte sagte lange Zeit nichts. Dann befahl sie: \WR{Weitersuchen. Das muss jemand von der Besatzung der \EN{Virial} gewesen sein. Keine anderen Menschen waren sonst hier. Sie kann nicht weit sein.}

\par

Morten bestätigte und flog einige Bögen. Seine Augen verengte sich zu schmalen Schlitzen als er den Himmel krampfhaft nach dem Forschungsschiff absuchte. Aber so sehr er sich anstrengte, er fand einfach nichts. Schließlich entschloss er sich, wieder auf die Signalquelle zuzusteuern. Wenn das Schiff irgendwo war, dann konnte es eigentlich nur dort sein.

\par

Es wunderte Morten kaum, weswegen er die \EN{Virial} nicht schon früher erkannte hatte, als er sie schließlich sah. Der Forschungskreuzer sah nicht mehr im Geringsten nach einem Raumschiff aus. Die Hülle war an vielen Stellen eingedellt oder aufgeplatzt. Das lange, halsartige Mittelsegment schien abgeknickt und nirgends gab es auch nur die kleinste Lichtquelle. So war das Schiff nur schwer von einem Asteroiden zu unterscheiden, zumindest aus der Ferne. Je näher Morten kam umso eher erkannte er die Formen eines Raumers der \EN{Ereignishorizont}-Klasse aber es bestand kein Zweifel, dass es nicht mehr war als ein Wrack.

\par

Sofort ließ Morten seine Abtaster nach Wärmesignaturen suchen, die vielleicht menschliche Lebenszeichen sein konnten. Während dessen erklärte er Anna über Funk, was er entdeckt hatte. Ohne zu antworten ließ Anna ihren Jäger auf Mortens Position zuschießen.

\par

Die Abtastung, die Morten durchgeführt hatte, war schnell zu Ende und ließ keinen Zweifel, dass es keine Überlebenden gab. Sein Jäger war dem Wrack mittlerweile schon recht nahe gekommen. Er erkannte leicht, dass die Hülle an unzähligen Stellen aufgebrochen war. Rußpartikel hatten sich um die verschieden großen Löcher angesammelt. War das Schiff in einen Meteoritenschauer geraten? Aber wieso war es überhaupt hier? Für Morten ergab das alles keinen Sinn.

\par

\WR{Sehen Sie das?}, fragte Anna über Funk und aktivierte die Suchscheinwerfer ihres Jägers.

\par

Die Lichtkegel richteten sich auf einen besonders großen Riss in der Hülle. Die Außenwand des Schiffes schien eingedrückt zu sein und die Ränder des Bruchs wirkten wie ausgefranst. Morten kannte diesen Anblick von zahllosen Bildern, die er auf der Flugschule gezeigt bekommen hatte. Dieses Loch war eindeutig der Überrest eines Waffeneinschlags. Wahrscheinlich, wie Morten sich überlegte, der eines AGKS-Torpedos oder einer starken Rakete.

\par

\WR{Die \EN{Virial} wurde also angegriffen}, schlussfolgerte er über Funk.

\par

Diese Erkenntnis ließ ihn schaudern. Selbst die Capital Fellowship hatten sich mit Übergriffen auf die Zivilbevölkerung zurück gehalten. Schließlich hatten sie es auf die Unterstützung der Menschen gegen den Staat abgesehen. Außerdem befand sich die \EN{Virial} zum Zeitpunkt des Angriffs bereits nicht mehr im Raum der Union. Jeder menschliche Angreifer wäre mit dem Sprung nach Arktur ein großes Risiko eingegangen, nur um ein Forschungschiff zu vernichten. Möglicherweise war irgendein Piratenstamm verantwortlich. Überreste der Armee des Commonwealth, die seit achtzig Jahren durch die Randgebiete zogen und jeden ausraubten, der sich nicht wehren konnte. Aber Morten hatte noch nie davon gehört, dass Piraten so weit außen operiert hätten.

\par

Von Annas Jäger ging ein gefächerter, hell leuchtender Strahl aus, der über die Überreste der \EN{Virial} huschte. Nahbereichabtaster wie Morten wusste. Nur Aufklärer waren mit ihnen ausgerüstet und sie lieferten Ergebnisse von der Genauigkeit eines Labors aus dem einundzwanzigsten Jahrhundert.

\par

Anna Stimme klang fast abwesend als sie die Untersuchungsergebnisse durchgab: \WR{Reste einer Nullzonenreaktion. Immer noch Restwärme zu finden. Dieses große Loch stammt eindeutig von einer Anti-Schiff-Waffe, wie sie eigentlich nur die Starforce verwendet. Die kleineren Risse weißen eine Strahlung auf, die auf Strahlenwaffenbeschuss schließen lässt.}

\par

Morten schluckte. Nicht einmal die Capital Fellowship, die für eine Terrorgruppe hervorragend ausgerüstet gewesen war, hatte jemals diese Waffen besessen. Atombomben waren der Schlimmste Bestandteil ihres Arsenals gewesen. Das alles konnte eigentlich nur eines bedeuten aber Morten wagte gar nicht, daran auch nur zu denken.

\par

Aber gerade in diesem Moment fiel ihm etwas wieder ein. Am Vorabend hatte er vom Protokoll Omega dreizehn gehört. Er hatte keine Ahnung gehabt, wo er diese Worte schon einmal vernommen hatte aber nun konnte er sich wieder erinnern. Ein Wissenschaftler namens Marco Bellendi, der davon überzeugt gewesen war, außerirdisches Leben gefunden zu haben, hatte dieses Protokoll entwickelt. Es war ein sehr einfaches Modell zur Kommunikation, das über mathematische Grundsätze eine Verständigung ermöglichen sollte. Mithilfe dieses Codes sollten Außerirdische in die Lage gebracht werden, die menschliche Sprache Basal zu verstehen.

\par

Annas Funkspruch riss Morten aus seinen Überlegungen und erschreckte ihn so sehr, dass er zusammenfuhr: \WR{Ich habe das Schiff vollständig abgetastet. Es ist völlig zerstört. Kein Sauerstoff mehr und auch keine Energiequellen. Die \EN{Virial} ist nur noch ein großes Stück Metall.}

\par

\WR{Aber woher kommt dann das Transpondersignal}, fragte Morten fast reflexartig.

\par

Die Antwort fiel nicht sofort. \WR{Es scheint als käme es von der anderen Seite des Wracks. Ich werde hinfliegen und es mir genauer ansehen. Folgen sie mir bitte und…} Sie unterbrach sich, \WR{und halten sie den Zeigefinger am Abzug.}

\par

Die beiden Flieger glitten über das Wrack des Forschungsschiffes. Morten versuchte in alle Richtungen gleichzeitig zu spähen. Jeder der weit entfernten Felsen kam ihm mittlerweile Verdächtig von und sein Finger schlang sich tatsächlich um den Feuerknopf seines Steuerknüppels. Die beiden Strahlenkanonen erster Ordnung, die sich rechts und links neben seinem Cockpit befanden, hatte er bereits scharf geschaltet.

\par

Hinter dem Wrack verblasste das unheimliche Licht der Sonne größtenteils und es wurde dunkel. Nur die Suchscheinwerfer Annas Jäger, das Glühen dessen Turbinen und das matte Licht der Sterne erhellten die Gegend noch. Dann fiel Morten ein schwaches Blinken auf. Rot und grün immer abwechselnd. Er erkannte diesen Code sofort. Es waren die Positionslichter einer Logbuchboje, wahrscheinlich jener der \EN{Virial}.

\par

Anna hatte es ebenfalls entdeckt. Sie ließ ihren Jäger nahe herangleiten. \WR{Wunderbar}, sagte sie über Funk klang dabei aber alles andere als erfreut. \WR{Das ist eine Drohne, wenn wir Glück haben, enthält sie Aufzeichnungen über den Angriff. Dann finden wir vielleicht heraus, was hier los war.}

\par

Morten konnte beobachten, wie Anna ihren Jäger näher an die Boje heran brachte. Für seine Augen unsichtbar, baute sich eine Funkverbindung zwischen Annas Flieger und dem quaderförmigen Objekt auf.

\par

\WR{In einer Minute werde ich alle Daten heruntergeladen haben}, verkündete sie angespannt über Funk.

\par

Morten betätigte und begann Kreise zu drehen. Das Asteroidenfeld, dass sie umgab wirkte nun noch bedrohlicher als je zuvor. Ihm wurde nun schnell klar, woher die Schneise in diesem sonst sehr dichten Feld kam. Wahrscheinlich hatte die Explosion der \EN{Virial}, die das Schiff in ein Wrack verwandelt hatten, auch einige Asteroiden aus dem Weg gesprengt. Das erklärte auch die merkwürdigen Formen einiger dieser Brocken. Für Morten vergingen die sechzig Sekunden viel zu langsam. Er wünschte sich nun nur noch, so schnell wie möglich wieder auf die \EN{Regenvogel} zurückzukehren. Dort würde er dann erst einmal mit den Schrecken fertig werden müssen, die er gesehen hatte. Er kannte nicht die genaue Besatzungsstärke der \EN{Virial} aber ihm war klar, das sehr viele Menschen ihr Leben verloren hatten. Und das alleine reichte schon aus um ihn fast wieder zum würgen zu bringen. Die im All treibende Leiche, die er gesehen hatte, setzte dem ganzen lediglich noch die Krone auf.

\par

Immer wieder sah Morten abwechselnd auf die Uhr und dann aufs Radar. Zu erkennen war jedoch nichts. Nur das Transpondersignal und Annas Jäger. Die Sekunden vergingen wie in Zeitlupe.

\par

Dann endlich war es soweit. Anna ließ ihren Jäger aufsteigen und sich von der Boje weg schweben. \WR{Alle Daten gesichert}, meldete sie erleichtert über Funk.

\par

Morten fuhr zusammen, als er plötzliche die blauen Lichtblitze von Annas Bordkanonen aufblitzen sah. Kurz darauf zerplatzte die Signalboje und zurück blieb nur ein glühender Metallklumpen.

\par

\WR{Wieso haben Sie das getan}, wollte Morten mehr als überrascht klingend wissen.

\par

\WR{Damit niemand anders als wir die Daten in die Finger bekommt}, antwortete sie mit einer ungewohnten Härte. \WR{Falls sie nicht schon längst von jemand anderem geborgen wurden.}

\par

Morten formierte sich wieder an Annas Seite. Für ihn war ihre Vorgehensweise nicht ganze nachvollziehbar aber er vertraute ihr. Sie würde schon wissen, was sie tat.

\par

Gerade wollte er fragen, ob eine Rückkehr zur \EN{Regenvogel} angebracht wäre, als plötzlich ein gleißend helles Licht aufblitzten. Innerhalb eines Sekundenbruchteils konnte Morten nichts mehr sehen aber er spürte, wie sein Jäger von etwas weggedrückt wurde und wild zu trudeln begann.
\ortswechsel
Das Wrack der \EN{Virial} war explodiert. Soviel wurde Morten schnell klar. Als er nach einer, wie es ihm schien, Ewigkeit wieder etwas erkennen konnte, sah er, dass sein Aufklärer wild hin und her schlingerte. Mit aller Kraft versuchte er den Steuerknüppel zu sich hin zu ziehen und seinen Flieger abzufangen. Während dessen brannten mindestens ein Dutzend Warnsymbole in seinem Cockpit auf. Seine Blocker waren ausgefallen und der Computer markierte einige beschädigte Stellen am Rumpf seines Schiffs. Wahrscheinlich hatte ein Schrapnell seinen Jäger gestreift, nachdem die Schutzfelder zusammengebrochen waren.

\par

Langsam bekam Morten seinen Aufklärer wieder unter Kontrolle. Er drehte ihn hastig in Richtung \EN{Virial}. Doch wo zuvor noch das Wrack geschwebt war, gab es nun nur noch einen gleißenden Feuerball, dessen Leuchten sogar das orangene Licht der Sonne überschattete.

\par

Morten gab Vollgas, denn wenn er nur eine Sache von seinen Fluglehrern gelernt hätte, dann sicher, dass man in einer solchen Situation lieber nicht zu lange an einer Stelle bleiben konnte. Seine Augen huschten zum runden Radarmonitor. Erleichtert erkannte er, dass Annas Jäger immer noch da war und sich gerade wieder gefangen hatte.

\par

Aber er erschrak, als plötzlich zwei nicht identifizierte Kontakte auftauchten und auf ihn zusteuerten. Dann erschien noch ein dritter und der flog auf Anna zu. Waren das Asteroiden, die durch die Explosion aus der Bahn geraten waren? Aber Morten gestand sich schnell ein, dass das nur Wunschdenken war. Irgendjemand war da draußen, hatte gerade die \EN{Virial} in die Luft gejagt und war jetzt hinter ihm her. Panik drohte Morten zu übermannen aber er dachte nur noch daran, seinen Jäger möglichst schnell nach unten wegtauchten zu lassen

\par

\WR{Anna!}, brüllte er in sein Mikrofon. \WR{Bogey auf dem Radar. Was sollen wir jetzt machen?}

\par

Die Antwort seiner Vorgesetzten kam schon nach wenigen Sekunden aber Morten kam es wie eine Ewigkeit vor. \WR{Raus aus dem Asteroidenfeld und zurück zur \EN{Regenvogel}. Wir teilen uns auf und versuchen unser Glück alleine. Wenigstens einer von uns muss es zurückschaffen.}

\par

Wenigstens einer! Morten schluckte. Eines wurde ihm sofort klar, die Lage war ernst. Sehr ernst sogar. Anna hatte offen ausgesprochen dass sie schon bald tot sein konnten. Morten hatte niemals angenommen, sich dem Tod so schnell stellen zu müssen. Natürlich hatte man ihm auf der Flugschule erklärt, dass seine Arbeit eventuell lebensgefährlich sein könnte. Aber er hatte niemals geglaubt, wirklich in einen Kampf geraten zu können. Die Capital Fellowship war zerschlagen und um Raumpiraten kümmerten sich die Eliteeinheiten. Wieso also hätte er davon ausgehen sollen, vielleicht selber kämpfen zu müssen?

\par

Morten Magen verkrampfte sich und er begann am ganzen Körper kalt zu schwitzen. Ihm wurde schwindelig und er glaubte, jeden Moment das Bewusstsein zu verlieren. Nur Annas Stimme hielt ihn bei der Stange: \WR{Morten, was ist los? Wir müssen hier weg, gib Gas verdammt!}

\par

\WR{Ich kann dich doch nicht alleine lassen!}, schrie er in sein Mikrofon.

\par

Anna klang offen genervt, als sie antwortete: \WR{Wir haben keine Zeit für diesen Machoscheiß! Bring deinen Arsch hier raus und zwar sofort!}

\par

Morten musste kräftig schlucken um sich am Erbrechen zu hindern. Er riss seinen Steuerknüppel zu Seite und begann zu schlingern. Wenn jemand ihn aufs Korn nehmen wollte, würde er es ihm nun so schwerer fallen. Während dessen rief er die Koordinaten ab, an denen er und Anny ihre Hyperschlitten zurückgelassen hatten und programmierte sein Radar so, dass es ihm den Weg dorthin wies.

\par

Aber er konnte sich nicht sofort auf den Weg machen. Instinktiv ging er mit seinem Antrieb auf volle Leistung und zog seinen Jäger steil nach oben. Ein Alarmton verkündete ihm, dass auf ihn geschossen wurde.

\par

Und tatsächlich zuckten kurz darauf einige Strahlen an Mortens Jäger vorbei. Sie glänzten grünlich blau und sahen nicht nach irgendeiner Entladung aus, die Morten jemals gesehen hatte. Mit einem Blick über die rechte Schulter erkannte Morten, dass eine weitere Salve auf ihn zukam. Er hatte praktisch keine Zeit und drückte die Nase seines Jägers nach unten. Noch ehe er überhaupt realisiert hatte, dass etwas passiert war, zuckten schon die Strahlen wenige Zentimeter über seiner Cockpitscheibe vorbei. Er war sich sicher, wenn die Entladungen getroffen hätte, wäre er jetzt tot.

\par

Nur noch an sein eigenes Überleben denkend drückte Morten seinen Schubregler ganz durch und gab seinen Düsen noch mehr Leistung.

\par

Als er nach hinten sah, erkannte er zum ersten mal einen der Angreifer. Was auch immer es war, es hatte weit ausladende Flügel und einen länglichen Rumpf. Zwei Paare langer Rohre verliefen dicht am Rumpf entlang. Zwei weitere waren an den Enden der langen Tragflächen angebracht. Diese schienen über zwei massige, dreieckige Körper mit dem Hauptrumpf verbunden zu sein. Zwischen ihnen fanden sich zwei Zylinder, die wahrscheinlich Turbinen waren. Morten war davon überzeugt, besonders da ihre Enden rot leuchteten.

\par

Alles in allem sah der Raumer wie ein Jäger für ihn aus. Aber so sehr er sich auch anstrengte, er konnte kein Cockpit erkennen. Was war dieses Ding also? Eine Drohne.

\par

Plötzlich schrillte der Kollisionsalarm los und Morten sah nach vorne. Im ersten Moment hatte er angenommen, auf einen Asteroiden zuzufliegen aber er zuckte zusammen, als er erkannte, dass er auf einen weiteren Angreifer zusteuerte. Der Jäger, der genauso aussah wie jener der ihn verfolgte, hatte anscheinend einen Bogen gezogen um ihn abzufangen. Und schon sah Morten die tödlichen Strahlen auf sich zu rasen. So schnell er konnte, lies er seinen Jäger ausweichen. Außerdem zog er den Abzug an seinem Steuerknüppel durch und lies so einige bläuliche Entladungen los. Die Salve verfehlte ihr Ziel aber um einiges.

\par

Dann musste Morten erneut ausweichen. Er hatte den Bereich des Asteroidenfeldes erreicht, in dem die Felsen wieder dichter beieinander lagen. Aber dieses mal war er mit vollem Karacho hinein geflogen und nicht mit der Sicherheitsgeschwindigkeit. Im Viertelsekunden Takt musste er irgendwelche Felsbrocken ausweichen. Er dachte schon gar nicht mehr nach, sondern bewegte einfach nur noch den Steuerknüppel rein nach Gefühl.

\par

Ein lauter Knall ließ Morten aufschrecken. Ein Blick über die linke Schulter genügte um zu erkennen, dass er bei einer Kollision mit einem Meteor einen Teil seiner Tragfläche verloren hatte.

\par

Doch bevor er genauer darüber nachdenken konnte, schoss schon die nächste Salve der grünlich blauen Strahlen seiner Gegner über ihn hinweg. Den Asteroiden, den sie trafen, zerrissen sie wie einen Fetzen Papier. Und dann schrillte ein Geräusch in Mortens Ohr, dass ihn bis aufs Mark erzittern ließ. Ein lautes Heulen aus seinem Kopfhörer warnte ihn davor, dass eine Rakete auf ihn abgefeuert worden war. Dem Warnton nach eine wärmesuchende Rakete, die sich an seinen Antrieb hängen würde. Die Starforce verwendete diese Geschosse zwar nicht, weil sie eine recht einfache Konstruktion waren und einen Gegner nur von Achtern anpeilen konnten aber deswegen waren sie nicht weniger tödlich.

\par

\WR{Scheiße!}, brüllte Morten und riss seinen Jäger herum.

\par

Haarscharf kreiste er um einen größeren Einsbrocken und bemerkte nur noch, wie die Rakete ihre Peilung verlor und in einen Asteroiden krachte. Der Felsen zerplatzte in einer grellen Explosion.

\par

Morten hätte beinahe laut aufgeschrieen, so erleichtert war er. Aber dann schrillte der Raketenalarm erneut auf. Zitternd sah er auf sein Radar. Das Geschoss, ebenfalls ein Wärmesucher, war nur noch gute sechzig Meter von ihm entfernt. Er musste sofort reagieren, denn in einem Wimpernschlag wäre er tot.

\par

Ohne zu überlegen presste er auf eine Taste, die Gegenmaßnamen abwerfen ließ. Ein kleiner Zylinder wurde aus dem Heck seines Jäger ausgestoßen und zerplatzte wenig später. Dabei verteilte er eine Wolke von superheißen fingerlangen Schrapnellen, die auseinander zu treiben begannen. Die Rakete schoss hinein und die messerscharfen Gegenmaßnahmen schnitten sich in ihr Inneres. Daraufhin explodierte sie in einem hellen Lichtblitz. Die Detonation war so nah an Mortens Jäger passiert, dass ihn die Druckwelle aus der Bahn warf.

\par

Immer noch überrascht davon, dass er es geschafft hatte, der Rakete auszuweichen sah Morten auf seinen Radarmonitor. Die beiden Kontakte, die zweifellos seine Verfolger darstellten hatten ihre Farbe in rot geändert. Der Bordcomputer war intelligent genug, Raumer die auf einen schossen, als Gegner zu erkennen. Die Distanz zu beiden Blickpunkten nahm langsam aber sicher zu. Wahrscheinlich war Mortens Jäger schneller als seine beiden Verfolger. Bei einem Aufklärer keine Überraschung, denn nur Abfangjäger waren noch schneller.

\par

Doch die hohe Geschwindigkeit verlangte Morten alles ab. Immer wieder prasselten kleinere Felssplitter gegen sein Schiff. Glücklicherweise waren die Blocker wieder aufgeladen und voll funktionsfähig. Aber jeder Einschlag kostete sie einen Teil ihrer Leistungsfähigkeit. Morten riss mit einer Hast am Steuerknüppel, dass es eher nach einem Anfall als nach einer kontrollierten Bewegung aussah.

\par

Als er einen Moment lang Gelegenheit hatte, wagte es Morten wieder einen Blick auf sein Radar zu werfen. Annas Jäger und der dritte Kontakt waren mittlerweile verschwunden. Das Asteroidenfeld verursachte dermaßen viele Interferenzen dass es schwer war, überhaupt irgendetwas zu orten. Aber Morten wurde das üble Gefühl nicht los, dass Anna vielleicht etwas zugestoßen sein könnte. Während dessen schwelte der grünlich schimmernde Staub überall herum und trübte die Sicht zusätzlich ein.

\par

Morten wollte gerade aufatmen als er erkannte, dass seine beiden Verfolger vom Radar verschwunden waren. Offenbar hatte er sie tatsächlich abgehängt. Eine unglaubliche Erleichterung machte sich in ihm breit, wie er sie noch nie gespürt hatte. Wenn er nun nur die Nerven behielt und es schaffte, den Fels- und Eisbrocken so geschickt auszuweichen würde er es vielleicht wirklich zur \EN{Regenvogel} zurückschaffen. Alleine hatte er keine Chance gegen seine Verfolger aber mit den vielen anderen Jägerstaffeln und den Bordkanonen des Träger und seiner beiden Corvetten sähe die ganze Sache schon ganz anders aus.

\par

Aber plötzlich kam ihm ein anderer, unguter Gedanke. Anna hatte den Kontakt zur \EN{Regenvogel} verloren. Vielleicht war sie vernichtet worden. Vielleicht waren diese drei Jäger nur die Vorhut einer viel größeren Streitmacht. Die \EN{Virial} jedenfalls war durch einen Torpedotreffer zerstört worden und Jäger trugen normalerweise solche Waffen nicht. Also musste es noch größere feindliche Schiffe geben.

\par

Das Aufleuchten von grünlich blauen Strahlenimpulsen holte Morten in die Wirklichkeit zurück. Ein Asteroid explodierte direkt vor ihm. Aber ihm wurde schnell bewusst, dass er nicht das Ziel des Angriffes gewesen war. Ein Aufklärer der Argus-Klasse raste durch sein Blickfeld. Das konnte nur Anna sein. Aber ihr Flieger wurde von einem anderen Jäger verfolgt. Morten konnte nicht sagen, ob es der dritte Kontakt war, oder einer seiner Verfolger, denn der Jäger unterschied sich äußerlich nicht von den beiden anderen, die er gesehen hatte.

\par

Die Hülle des Fliegers hatte eine merkwürdige Farbe. Irgendwie eine Mischung aus rostigem braun und gräulichem grün. Das Licht Arkturs reflektierte ungewöhnlich stark auf der glänzenden Hülle.

\par

Morten erkannte seine Gelegenheit. Der Gegner schien ihn nicht bemerkt zu haben, so hängte er sich an ihn dran. Während er seinen Jäger unter großer Mühe und ständig Hindernissen ausweichend, auf den Flieger ausrichtete, sah er ihn sich genau an. Aber er konnte einfach kein Cockpit erkennen, obwohl er fast senkrecht auf ihn hinab sah. Aber dann fiel ihm etwas anderes auf. Zwei dreieckige, stromlinienförmige und blau leuchtende Flächen befanden sich auf einer kleinen Wölbung direkt vor den Düsen. Sie sahen fast wie Augen aus.

\par

Morten hatte es endlich geschafft, Annas Verfolger in sein Fadenkreuz zu bekommen.

\par

\WR{Friss Blei!}, brülle er laut und ließ seine beiden Kanonen sprechen.

\par

Die Salve traf den Jäger und dort wo die Strahlen eingeschlagen hatten, schienen sich die Entladungen in kreisrunde Wellen aufzulösen.

\par

Der getroffenen Gegner machte jedoch keine Anstallten abzudrehen. Anstatt dessen löste sich ein Geschoss, dass einen grünlichen Schweif hinter sich herzog von seiner Steuerbordtragfläche und jagte auf Annas Raumer zu. Obwohl sie schon recht weit weg war, würde die Rakete ihren Flieger bald erreicht haben. Ohne zu überlegen gab Morten Gas. Er sah nur noch, wie Anna ihre Gegenmaßnahmen abwarf. Aber vielleicht viel zu früh, denn der Zylinder explodierte lange bevor die Rakete ihm nahe gekommen war. Die Schrapnellwolke würde sich schnell so stark verteilt haben, dass die Rakete sie einfach durchfliegen würde, ohne Schaden zu nehmen.

\par

Ohne sich wirklich bewusst zu sein, was er tat, drückte Morten ab. Die Strahlensalve streifte die Rakete und brachte sie zur Explosion. Anna klang mehr als erstaunt, als sie sich über Funk meldete: \WR{Danke für die Hilfe, aber jetzt nichts wie weg!}

\par

Die beiden Aufklärer jagten durch das Asteroidenfeld. Viel Distanz zu überwinden hatten sie nicht mehr und wenn sie erst einmal im freien Weltall wären, hätten sie eine echte Chance, ihre Gegner abzuhängen.

\par

Aber ihre Verfolger gaben sich nicht einfach so geschlagen. Der Angreifer, den Morten kurzzeitig zum Abdrehen gezwungen hatte, kehrte nun zu einem zweiten Anflug zurück. Und dieses mal wirkte er noch aggressiver. Der Jäger schoss nun mit anderen Waffen. Wie Morten durch einen Blick über die Schulter schnell erkannte, schossen die kurzen rötlichen Schimmerblitze aus den Kanonen die dem Rumpf am nächsten waren.

\par

Anna riss ihren Steuerknüppel zur Seite und flog einen Looping. Morten ahnte schon was sie vorhatte und begann wild hin und her zu pendeln. Zweimal kam er einem Asteroiden so nahe, dass er glaubte demnächst tot zu sein. Während dessen hatte sich Anna hinter ihren Gegner manövriert und sie eröffnete das Feuer. Morten richtete seine Zielerfassung auf den Gegner aus. Der Computer erkannte den Schiffstyp nicht, was ihn nicht verwunderte aber er sein Zielmonitor zeigte eine schematische Darstellung des Jägers an. Unter anderem zeigte sie auch die gemessene Leistung seiner Blocker, die durch den Beschuss kaum nachließ.

\par

Daraufhin wendete Morten seinen Jäger ebenfalls. Knapp schlingerte er um einen Felsbrocken herum und brachte den feindlichen Jäger ins Visier. Der Flieger war gerade dabei, Annas Beschuss auszuweichen. Morten eröffnete ebenfalls das Feuer, verfehlte seinen Gegner jedoch.

\par

Dann kam ihm eine Idee. Den Jäger abzuschießen würden sie mit ihren Aufklärern kaum schaffen. Und selbst wenn, wären die beiden anderen schnell zur Stelle und dann würde es ungemütlich werden. Denn selbst mit der Wendigkeit eines Aufklärers würden sie dem Feuersturm nicht entkommen. Also entschloss sich Morten seinen Nahbereichabtaster auf den Angreifer auszurichten.

\par

\WR{Dann wollen wir doch mal sehen, was du für einer bist}, murmelte er eher zu sich selbst, als der gefächerte Strahl den Gegner traf.

\par

Der Angreifer schien sich nicht sicher zu sein, womit er es gerade zu tun bekam. Jedenfalls flog er hinter einen größeren Asteroiden, der die Abtastung unterbrach. Morten war es egal. Er hatte einige Daten gesammelt und gab nun Vollgas. Anna tat es ihm gleich und nun war sie es, die ihm folgte.

\par

Morten blickte auf sein Radar und erkannte, dass es nur noch wenige hundert Meter aus dem Asteroidenfeld heraus waren. Aber er hatte einen Moment zu lange auf den Monitor geblickt, denn plötzlich schlug einer kleiner Meteor auf seine rechte Tragfläche auf. Er durchschlug mühelos die dünnen Blocker des Aufklärers und riss ein klaffendes, funkendes Loch in den getroffenen Flügel.

\par

Mortens blick raste zur Schadensanzeige. Wenn der Antrieb irgendwie beschädigt worden war, dann hatte er kaum noch eine Chance. Sein Jäger würde liegen bleiben und wäre leichter zu treffen als eine sitzende Ente.

\par

Gerade als er erleichtert sah, dass der Antrieb wohl noch intakt war, schrillte erneut der Kollisionsalarm auf. Durch die Cockpitscheibe war zu erkennen, das Mortens Jäger auf einen riesigen Felsbrocken zu raste. Verzweifelt zog er seinen Steuerknüppel zurück und sah nur noch, wie die Oberfläche des Felsens an ihm vorbeizog.

\par

Gerade als er dachte, er würde jeden Moment aufschlagen, flog sein Jäger endlich parallel zum Asteroiden und wenig später war er aus dem Asteroidenfeld heraus. Atemlos richtete er seinen Jäger auf die Koordinaten der abgeworfnenen Schlitten aus.

\par

Anna tat es ihm gleich. Ein kurzer Blick auf die Radaranzeige ließ sie wissen, dass ihre drei Verfolger das Asteroidenfeld ebenfalls gerade verlassen hatten. Sie blickte über ihre Schulter und sah, wie die drei Jäger wild zu feuern begannen. Einer der Stahlen streifte ihre Steuerbordtragfläche und entzog ihren gesamten Blockern gleich die Hälfte ihrer Energie.

\par

Hastig zündete sie das Äquivalent eines Nachbrenners. Der Reaktor stellte den Düsen mehr Energie zur Verfügung und verlieh dem Jäger so mehr Schub. Lange würde das Schiff diese Geschwindigkeit, die auch noch immer größer wurde, nicht aushalten.

\par

Aber es reichte, um sich außer Feuerreichweite zu bringen. Beide Jäger rasten nun auf die Hyperraumschlitten zu, während Mortens Jäger eine lange Rauchspur hinter sich zurückließ. Die drei Angreifer blieben ihnen auf den Versen, waren aber einfach nicht schnell genug, um sie noch einzuholen.

\par

\WR{Beginne Andockmanöver}, gab Anna durch und ließ ihren Jäger gekonnt in ihren wild blinkenden Hyperraumschlitten gleiten. Morten folgte ihrem Beispiel und hoffte dabei inständig, dass die Schäden an seinem Schiff sich nicht auf die Kopplung auswirken würden. Es konnte viel schiefgehen und das metallische Knirschen und Kreischen, das erklang, als er seinen Jäger in den Schlitten schob, machten ihm keinen Mut. Einige Anzeigen warnten von struktureller Überbelastung, doch am Ende rastete das Konstrukt ein.

\par

Der Autopilot hatte noch die Route durch den flachen Hyperraum eingespeichert und es dauerte nicht lange, sein Schiff auf Kurs zu bringen. Kaum, dass Anna \WR{Sprung einleiten!} befohlen hatte, drückte er den Beschleunigungshebel ganz nach vorne und ließ sich von dem Druck des Hyperraumübergangs in den Sessel pressen.

\par

Zum ersten mal atmete Morten ansatzweise hoffnungsvoll auf. Aber seine Erleichterung schwand sofort wieder, als Anna auf dem allgemeinen Flottenkanal \WR{Grün eins an \EN{Regenvogel}, bitte kommen. Bitte melden Sie sich \EN{Regenvogel}} rief und doch keine Antwort erhielt.

\par

Was war bloß mit der \EN{Regenvogel} los? Bis sie an den Koordinaten angelangt wären, würde es noch eine knappe Stunde dauern und bis dahin konnte noch viel passieren. Vielleicht waren noch anderswo Feinde, die sie nur abzufangen brauchten. Dann würden ihnen auch ihre schnellen Aufklärers nicht viel helfen.

\par

Eine halbe Stunde lang geschah nichts, außer dass die drei Verfolger vom Radar verschwanden weil sie außer Reichweite waren. Immer wieder setzte Anna einen allgemeinen Notruf ab, bekam aber keine Antwort. Morten wurde wieder Angst und Bange. Und als er dann drei neue Kontakte auf dem Radar hatte, fuhr er wild schnaufend hoch.

\par

Aber sobald er erkannte, dass es drei grüne Blickpunkte waren, die also das Transpondersignal der Starforce aussendeten, brach er fast in Jubelschreie aus. Annas Stimme, die nun über Funk erklang, wirkte genauso begeistert: \WR{Hier grün eins. Wir werden verfolgt, Gegner sind hier in der Umgebung. Alle Schiffe müssen sofort in Alarmbereitschaft versetzt werden.}

\par

\WR{Hier Bravo eins}, erklang eine Stimme, die Morten nicht wiedererkannte. \WR{Verstanden grün eins. Wir wissen von der Bedrohung. Eine Staffel aus Henningtons Geschwader ist verschwunden. Keine Ahnung wie. Sie waren nur wenige hunderttausend Kilometer von der \EN{Regenvogel} entfernt, gerade außer Sichtweite, da haben wir sie plötzlich verloren. Wir haben das Areal abgesucht aber nur noch Trümmer gefunden. Dann haben wir versucht alle Staffeln zurückzurufen aber irgendwas blockiert unsere Kommunikation, deswegen wurden wir ausgesandt um möglichst viele Staffeln selbst zu erreichen.}

\par

Anna antwortete schnell: \WR{Hören Sie, wir haben die \EN{Virial} gefunden. Oder das was von ihr übrig war. Sie wurde von Nullzonenwaffen zerstört. Egal wer das war, er könnte auch der \EN{Regenvogel} gefährlich werden.}

\par

\WR{Keine Sorge}, antwortete die Stimme. \WR{Der Träger und die beiden Corvetten sind in voller Alarmbereitschaft. Wir sind auf alles vorbereitet.} Morten glaube nicht daran.

\par

Die drei neuen Jäger kamen jetzt in Sichtweite. Es waren Abfangjäger der Apace-Klasse. Sehr schlank und mit zwei massigen Turbinen an den Enden ihrer, mit Raketen beladenen, Tragflächen ausgestattet. Am Ende des säulenförmigen Rumpfes waren zwei dicke Zusatztriebwerke angebracht. Diese zwei Düsen konnten einen Abfangjäger stärker als jedes andere Schiff beschleunigen.

\par

Als Morten und Anna ihre drei Kameraden erreicht hatten, wendeten die Abfangjäger und nahmen Flankenpositionen ein um die beiden Aufklärers nach Hause zu begleiten.

\par

\WR{Alles in Ordnung?}, fragte die Morten immer noch unbekannte Stimme über Funk. \WR{Ihr rechter Flügel dampft ganz schön heftig.}

\par

Morten antwortete mit Erleichterung: \WR{Ja, Bravo eins. Nur ein Kratzer. Ich hoffe nur der Büffel bringt mich nicht um, wenn er sieht, was ich mit seinem Flieger gemacht hat.}

\par

\WR{Nur wenn Sie ihn wirklich Büffel nennen, Grün zwei.}

\par

\WR{Hat man schon irgendeine Ahnung, wieso unsere Nullzonentranciever nicht mehr funktionieren?}, wollte Anna schließlich wissen. \WR{Ich meine, es sollte ziemlich schwierig sein, sie zu blockieren.}

\par

Morten hatte zwar nur schwerlich einen klaren Gedanken fassen können, während sein Jäger sich auf dem Heimflug befunden hatte, doch er hatte sich bereits dasselbe gefragt. Zwei Nullzonentranciever verbanden sich über den Hyperraum und reduzierten die Verbindung zwischen zwei Punkte auf Null. Ab diesem Moment war die Verbindung nicht mehr von außen zu trennen. Der Prozess musste zwar zunächst initiiert werden, doch dies funktionierte oft sogar über viele Lichtjahre hinweg problemlos.

\par

\WR{Nein}, war die rasche Antwort des Anführers der Bravo-Staffel. \WR{Zumindest nicht genau. Weder die Kommunikations- noch die Radarabteilung hat irgendetwas außergewöhnliches festgestellt.}

\par

Dieses mal freute sich Morten besonders darüber, dass ein Flug absolut ereignislos verlief. Nachdem es noch etwa fünfzehn Minuten bis zur Ankunft bei der \EN{Regenvogel} waren, funktionierte der Funkkontakt wieder. Lieutenant Wallander, der blauhaarige Kommunikationsoffizier, verkündete mürrisch dass er schon bald einen Weg finden würde, die Funkblockade zu durchbrechen. Was er davon halten sollte, wusste Morten nicht, er konnte sich nur vorstellen, dass Wallander recht erbost darüber war, dass man seine Kommunikationswege unterbrach.

\par

Wenig später hatten sie die \EN{Regenvogel} erreicht. Im All um das Schiff herum herrschte reger Betrieb. Einige andere Staffeln waren gerade zurückgekehrt und Morten atmete auf, als er Kevin und Kenjis Jäger an den aufgeklebten Farbbändern erkannte.

\par

Gerade startete ein Schwadron Verteidigungsjäger auf dem Flugdeck der \EN{Regenvogel}. Dann bekamen Morten und Anna endlich Landeerlaubnis, zusammen mit dem Befehl, sich sofort auf der Brücke zu melden. Major Farley ließ ihrem Flügelmann den Vortritt beim Landen. Morten war darüber sehr erleichtert und brachte seinen Vogel beim ersten Versuch sicher auf den Boden.
