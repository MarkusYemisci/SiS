Eine Bar, wie es so manche gab auf Sigma Draconis II, einem Planeten, bei dem viele auf die zweite Erde gehofft hatte. Allerdings hatten bereits erste Untersuchungen ergeben, dass es auf dem Himmelskörper zwar eine atembare Atmosphäre und sogar Wasser und Regen gab, doch dass sich keinerlei Leben entwickelt hatte. Weder Einzeller noch Vorformen von DNA, ganz zu schweigen von größeren Organismen wie Tieren oder Pflanzen. Sigma Draconis II war vom All aus betrachtet nichts weiter, als ein warmer, wässriger Felsen.

\par

So stammte die wenige Vegetation, die sich in den letzten einhundert Jahren herangebildet hatte, vollständig aus den Laderäumen von Kolonialschiffen. Der Planet bot abgesehen von seinem minimal grünhaltigen Himmel und seinen enormen Bergmassiven kaum etwas interessantes. Das war einer der Gründe gewesen, weswegen er zu einem Industrieplaneten geworden war. Wo kein Ökosystem existierte, konnte auch keines geschädigt werden.

\par

Jene Bar, die so gut lief, wie jede andere auf Sigma Draconis II~-- kaum~-- befand sich am Straßenrand der einzigen Autobahn zwischen der Hauptstadt und den entfernteren Siedlungen. Ganz in der Nähe befand sich ein Stützpunkt der Phalanx, dessen Personal das Gros der Gäste in der langsam auseinander fallenden Spelunke ausmachte. Die Barracken und die Trainingsplätze waren allesamt mit Generatoren für künstliche Schwerkraft ausgestattet, um Muskelabbau bei den Soldaten vorzubeugen. Wie alle Industrieplaneten hatte Sigma Draconis II eine, verglichen mit der Erde, deutlich geringere Gravitation, was die Arbeiten in den Fertigungsanlagen sehr vereinfachte.

\par

Hinter einer nahen Bergkette lagen drei Basisschiffe der Phalanx vor Anker. Sie ruhten so lange in ihren planetaren Landebuchten, bis ein Einsatz oder eine Übung ausgerufen wurde. Dass sie nicht direkt neben dem Stützpunkt im Dock lagen, hatte seinen Grund in der Sicherheit. Die Capezinreserven, die nötig waren, um diese Schiffe starten zu lassen, waren im Fall einer Panne ein untragbares Risiko.

\par

Genauso hatte der Kommandant des Stützpunkts auch schwarzer Bär schon einige male genannt. Der angehende Marineinfanterist stand kurz vor dem Ende seiner Ausbildung. Und wie er es bis dahin geschafft hatte, war vielen schleierhaft. Die meisten schrieben seinen Erfolg dem guten Training seiner Ausbilderin zu.

\par

Nun genoss er das dritte Bier in seiner Lieblingskneipe, die er deswegen favorisierte, weil sie ihn am wenigsten anekelte. \Wr{Eddies Pub} war im wesentlichen ein Holzbungalow, der nur deswegen noch stand, weil die Winde auf Sigma Draconis II nicht allzu stark waren. Neben ein paar Tischen und der Theke gab es nur noch die Toiletten und einen öffentlichen Computer, der in ein Museum gehört hätte.

\par

Der Rest aus Schwarzer Bärs Einheit lag bereits in den Gemeinschaftsschlafsälen des Vergara-Stützpunkts. Ein Zwanzig-Kilometer-Lauf mit vollem Gepäck hatte sie müde gemacht. Schwarzer Bär selbst fühlte sich auch nicht viel besser. Doch ohne seine Abendbiere glaubte er nicht, noch einen Tag länger im Ausbildungslager verbringen zu können.

\par

Was ihn geritten hatte, gerade zu den Marineinfanteristen zu gehen, wusste er selbst nicht mehr genau. Trotz seiner nicht unbedingt höchstklassigen Bildung hätte er sicher eine angenehmere Beschäftigung finden konnte. Wahrscheinlich, so glaubte er, wollte er einfach nicht in die Fußstapfen seines Vaters treten, der stets den Weg des geringsten Widerstands gegangen war. Leider auch, als sich die ersten, ernsteren Probleme in seiner Ehe eingestellt hatten. So waren Schwarzer Bär und dessen Mutter eines Tages in einem verlassenen Haus aufgewacht.

\par

Durch dieses Ereignis hatte er auch eine starke Abneigung gegen die Bürokratie der Unio Terrae aufgebaut. In einer Zeit, in der man täglich mindestens drei mal gefilmt wurde, und der Staat ein volles Bildregister seiner Einwohner besaß, musste irgendwer wissen, wohin sich sein Vater geflüchtet hatte. Doch Auskunft wollte niemand geben.

\par

Verlassen zu werden hatte für Schwarzer Bärs Mutter eine massive Lebenskrise eingeläutet. Der Verlust ihrer Arbeitsstelle, Depressionen und Alkoholabhängigkeit waren die Folgen gewesen. Irgendwann war sie in eine Heilanstalt für Suchtkrankheiten eingeliefert worden. Doch selbst mit der modernen Medizin der Union war die Heilung einer Abhängigkeit nur möglich, wenn der Patient das auch wollte. So hatte seine Mutter immer wieder Rückfälle erlitten.

\par

Irgendwann hatte er es nicht mehr ertragen, sie so leiden zu sehen und beschlossen, seine Heimat Corna zu verlassen. Da Cornaer mit dem Klischee des raubeinigen Draufgängers in Verbindung gebracht wurden, hatte man ihn bei der Phalanx gerne aufgenommen. Der praktische Teil mit seinen vielen Übungen, Orientierungsläufen und Gefechtssimulationen hatte für ihn nur ein überwindbares Problem dargestellt. Der theoretische Unterricht über Ethik, Moral und Phalanx-Methodologie hingegen, brachte ihn regelmäßig an die Grenzen seiner Geduld. Es interessierte ihn nicht, welche Richtlinien irgendein Politiker für Soldaten aufgestellt hatte. Oder was irgendein Schlipsträger glaubte, welche Werte einen guten Befehlsempfänger ausmachten. Er machte es aus demselben Grund, aus dem in seinen Augen auch jeder andere arbeitete~-- um seinen Lebensunterhalt zu verdienen.

\par

Bär, wie er von seinen ungeduldigeren Freunden auch genannte wurde, leerte sein letztes Bier und zückte sein Buch. Der Barkeeper kam gelangweilt heran getrottet und tippte den Preis für die Getränke in die Kasse ein, die daraufhin Kontakt zu Schwarzer Bärs Buch aufnahm. Fünf Naira waren zumindest für drei Bier nicht zu viel verlangt.

\par

\WR{Bis Morgen}, verabschiedete sich der angehende Infanterist und hob kurz die Hand zum Gruß.

\par

Gerade als er durch die Tür ins Freie ging, stieß er mit einem Mann zusammen, der gerade hineingehen wollte. Bär brummte ein leises \WR{Tschuldigung} und wollte schon weitergehen, als ihn der andere an der Schulter packte und herumriss.

\par

\WR{Hey! Ihr hirnlosen Knobelbecher glaubt wohl euch gehört die ganze Welt!}, spie ihm der Mann entgegen, der nach seiner Arbeitsuniform zu einer der Schwermetallwerken gehörte. Seinen Worten folgte eine gehörige Alkoholfahne.

\par

Schwarzer Bär seufzte. \WR{Ich hab doch schon Entschuldigung gesagt.}

\par

\WR{Das reicht mir aber nicht}, lallte der Mann und schubste sein Gegenüber so fest er konnte. \WR{Ich will, dass ihr Phalanx Vollidioten eure Sachen packt und verschwindet. Oder sucht euch gleich einen \textit{richtigen} Job.}

\par

\WR{Pack mich ja nicht mehr an!}, fauchte Bär, dessen Geduld ihrem jähen Ende entgegenging.

\par

Doch der Mann dachte nicht daran und schubste ihn erneut. Und es kam, wie es nicht zum ersten mal gekommen war. Er gab seiner eigenen Wut nach und schlug dem Arbeiter ins Gesicht, der daraufhin mit blutender Nase zu Boden ging.
\ortswechsel
\WR{Ich hab langsam die Schnauze voll von Ihnen, Rekrut!}, brüllte Legat van Nyst. Der in die Jahre gekommene Leiter des Stützpunkts bildete einen seltsamen Kontrast zu Schwarzer Bär. Er ging dem Marineinfanteristen höchstens bis zu den Schultern. Doch jeder im Büro des Kommandanten nahm ihn in diesem Moment als größer wahr.

\par

\WR{Ich wäre längst zu Hause, wenn Sie nicht wie jede verdammte Woche Ärger machen würden!} Der Legat knallte die Polizeiakte so heftig auf seinen Schreibtisch, dass alle Anwesenden mit Ausnahme von Bär zusammenfuhren. \WR{Was glauben Sie eigentlich, was Sie hier machen? Ein bisschen rumgammeln und dann und wann Dampf ablassen, so wie es ihnen passt? Das hier ist die Phalanx und kein scheiß Schießbudenverein!}

\par

Van Nyst trat so nah an Schwarzer Bär heran, dass sich seine Nase demnächst an der breiten Brust des Rekruten reiben würde. Um das zu verhindern ging der Legat einen Schritt zurück und schrie weiter: \WR{Es vergehen keine sieben Tage in denen Sie nicht wieder irgendeine Hurenpisse bauen. Im Forum wollen einige, dass wir arbeitslos werden. Glauben Sie vielleicht, prügelnd durch die Gegend zu rennen, poliert unsere öffentliche Wahrnehmung auf?} Als Bär nicht antwortete brüllte der Legat weiter. \WR{Verdammt noch mal, machen Sie ihre Fresse auf!}

\par

\WR{Ich glaube der Kerl war eh kein Freund der Armee}, antwortete der Marineinfanterist laut. \WR{Ich hab ihm bloß bei seiner Wahlentscheidung geholfen.}

\par

\WR{Wollen Sie mich verarschen?}, donnerte der General. \WR{Sehe ich aus, wie ein verdammter Vollidiot?}

\par

Centurio Roland Hügle räusperte sich und meldete sich vorsichtig zu Wort. \WR{Wenn Sie erlauben, Legat. Rekrut Schwarzer Bär gab an, provoziert worden zu sein.}

\par

Der Centurio war ein Ausbilder der alten Schule. Die Richtlinien für das Trainieren von Rekruten, die Beleidigungen, Kollektivstrafen und körperliche Disziplinierung verbot und Respekt auch gegenüber den untersten Rängen vorschrieb, waren ihm ziemlich egal. Er hatte einmal Ärger bekommen, weil er einen Neuling gegen eine Wand geworfen hatte. Doch sein Standbein in den Streitkräften war, auch wegen seiner hohen Ausbildungsquote, so gefestigt, dass sich keine ernsten Konsequenzen ergeben hatten. In Hinsicht auf Hügles Ansichten gegenüber vorlauten Zivillisten, hatte es Bär nicht überrascht, dass er ihn nun zu verteidigen versuchte.

\par

\WR{Wissen Sie, wie oft ich schon provoziert wurde, Centurio?}, entgegnete der Legat leiser aber immer noch bestimmt. \WR{Ich glaube, es gibt keine Beleidigung, die mit der Armee in Verbindung steht, die ich noch nicht gehört habe. Aber deswegen habe ich noch nie jemanden geschlagen.}

\par

An Schwarzer Bär gewandt fuhr der Legat fort: \WR{Das Konglomerat hat das Gewaltmonopol. Das heißt, dass jeder von ein glänzendes Vorbild dafür sein muss, wann Gewaltanwendung angebracht ist und wann nicht. Das sollten Sie längst wissen. Diese Lektion gehört zu Lehrplan, gleich im ersten Jahr.}

\par

Schwarzer Bär seufzte in sich hinein. Die Befragung durch die Polizei und die Untersuchungen hatten bis nach Mitternacht gedauert und er wollte nur noch ins Bett, anstatt sich eine Belehrung über die Verantwortung der Armee anzuhören.

\par

\WR{Keine Angst}, antwortete der Legat, der scheinbar Gedanken lesen konnte. \WR{Ich werde Ihnen jetzt keine Nachhilfestunde geben. Ich bin nicht sicher, ob überhaupt irgendetwas davon in ihren sturen Schädel gehen würde. Merken Sie sich bloß eines: Egal, welche Gründe Sie hatten, sich der Armee anzuschließen. Wir haben Erwartungen und wenn Sie die nicht erfüllen, dann sind sie raus!}

\par

Nun ergriff die einzige Person das Wort, die bisher noch als einzige nichts gesagt hatte. Präfektin Rana Sali, Schwarzer Bärs Ausbilderin. Eine Frau in den Dreißigern. Die pechschwarzen Haaren vorschriftsmäßig zu einem Zopf zusammengebunden. Ihr Bindi~-- der rote Punkt zwischen ihren Augenbrauen~-- verstieß ebenfalls nicht gegen die Regelungen, da er als Teil ihrer Religionsfreiheit akzeptiert wurde.

\par

\WR{Heißt dass, Rekrut Schwarzer Bär darf seine Ausbildung abschließen, mein Herr?}, fragte sie tonlos.

\par

\WR{Nein, das heißt es nicht}, entgegnete der Legat sofort. \WR{Es heißt nur, dass ich Ihn nicht sofort rauswerfe. Aber wenn er ein einziges mal einen Zivillisten auch nur schief ansieht, dann werde ich ihn persönlich dermaßen in den Arsch treten, dass ihn die Starforce wieder einsammeln muss.}

\par

Präfektin Sali atmete innerlich auf. Sie war keinesfalls einverstanden mit dem Verhalten ihres Auszubildenden. Aber sie war auch der Ansicht, dass dieser vielleicht nur in der Armee eine Chance im Leben haben würde. Dass es in der Union keine Arbeitslosen gab, hieß nicht, dass jeder auch automatisch ein glückliches Dasein führte. Sie nahm sich vor, Bär später noch einmal selbst ins Gebet zu nehmen.

\par

Von keinem erwartet, betrat die Adjutantin des Legaten den Raum. Van Nyst hatte ihr schon vor langem erlaubt, ohne Ankündigung einzutreten, es seihe denn, er verlangte ausdrücklich das Gegenteil. In Ihrer Hand hielt sie einen Handcomputer, den sie noch einmal ungläubig ansah, bevor sie meldete: \WR{Mein Herr, wir haben eine Priorität eins Nachricht direkt von Admiral Burns erhalten. Für sämtliche Streitkräfte gilt ab sofort Alarmstufe Gamma drei. Die volle Reserve wird reaktiviert und alle Basisschiffe sollen sofort starten. Anscheinend haben wir Schnee im September.}

\par

\WR{Wer?}, fragte der Legat völlig außer Fassung.

\par

\WR{Laut diesem Bericht weiß man noch nichts genaues}, antwortete die Adjutantin. \WR{Admiral Burns hat eine Durchsage für null null null fünf Erdzeit angekündigt.}

\par

Van Nyst, der gerade im Begriff gewesen war, sich hinzusetzen, drückte nun eine Taste auf dem Kontrollpult seines Schreibtisches. \WR{Hier ist der Kommandant. Eine Notsituation ist eingetreten und es gilt ab sofort die höchste Alarmstufe. Das ist natürlich keine Übung. Ich will jeden einzelnen in den nächsten zwei Stunden an Bord der Schiffe sehen. Ende der Durchsage.}

\par

Alarmstufe Gamma drei sah vor, dass sich jede militärische Organisation des Konglomerats auf baldige Kampfhandlungen bereitzumachen hatte. Das bedeutete auch, dass die Marineinfanteristen, jener Teil der Phalanx, der nicht am Boden, sondern an Bord von Raumschiffen stationiert war, diese sofort bemannte. Allein die enormen Ausmaße der Union machten es nötig, im Ernstfall mobile Einsatzbasen im All zu haben.

\par

\WR{Wegtreten}, befahl der Legat und ging zu seinem Umkleideraum, in dem er seine Galauniform gegen die übliche Einsatzkleidung eintauschen würde.