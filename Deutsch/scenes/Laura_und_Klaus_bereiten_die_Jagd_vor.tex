Der Geruch von frischem Essen durchzog die ganze Kantine.
Laura Gethas und Klaus Rensing hatten ihr gemeinsames Essen gerade abgeschlossen.
Es war zwar schon sechs Uhr Abends, also eine merkwürdige Zeit für ein Mittagsmahl, aber die Essensausgabe hatte durchgehend geöffnet.
Viele der Mitarbeiter im Geheimdienstzentrum fingen erst Abends an und hörten früh am Morgen wieder auf, da sie eine Wohnung auf der anderen Seite der Erde hatten.
Darum konnte keine einheitliche Zeit für Essenspausen festgelegt werden.
Da die Union aber immer bemüht war, Arbeitsplätze zu schaffen und eine ganztägige Kantine solche ermöglichte, versuchte auch niemand, eine gemeinsame Pause zu finden.

\par

\WR{Dann legen wir mal los}, gab Klaus Rensing von sich und versuchte, dabei elanvoll zu klingen.

\par

Er überragte seine Kollegin um mehr als einen Kopf und war auch von der ganzen Statur her das genaue Gegenteil zu Laura.
Klaus war stämmig und hatte ein erwachsenes Gesicht voller Kanten, zu dem seine dunkelblonden Haare gut passten.
Gemeinsam hatten die beiden nur einen wachen und energischen Blick.

\par

Laura nickte beiläufig und zog ihr Buch Lesegerät des Fahrstuhls vorbei.
Ausnahmsweise war der Lift leer, als die beiden eintraten.
Klaus wählte das Stockwerk aus, in dem sich Lauras Büro befand.

\par

\WR{So, wie geht's dir}, fragte Lauras Partner um die, für ihn, unangenehme Stille zu beenden.

\par

Sie antwortete, ohne ihn anzusehen: \WR{Gut. Und dir?}

\par

Klaus hatte sonst stets eine sehr stolze Haltung.
Doch nun wirkte er geknickt, als er auf den Boden sah und sagte: \WR{Das stimmt nicht und das weißt du.
Es geht dir nicht gut.}

\par

\WR{Du möchtest wirklich jetzt darüber reden?}, fragte Laura und gab sich gar keine Mühe, ihre blank liegenden Nerven zu verbergen.

\par

\WR{Ich möchte schon seit einer Ewigkeit darüber reden}, schoss es förmlich aus ihm heraus.
\WR{Oder denkst du, ich würde sonst mit einem Lahmarsch wie dir Joggen gehen wollen?
Ich will, dass du endlich aufmachst.
Wir haben beide das gleiche erlebt.
Aber ich hab geschafft es abzuhaken.
Du nicht.}

\par

Laura, die sonst bemüht gewesen war, Augenkontakt zu vermeiden, blitzte ihren Partner nun wütend an.
\WR{Genau das ist das Problem.
Du hast es abgehakt.
Mein Boss hat es abgehakt.
Jeder hat es abgehakt.
Und ich hab keine Ahnung, wie ihr das macht.
Wie ihr weitermachen könnt, wenn etwas so derart falsch läuft.
Und wir haben nicht das gleiche erlebt.
Ich habe sie gesehen, Klaus.
Ich habe ihren Körper aus dem Dreck gezogen.}

\par

Der Lift hatte sein Ziel längst erreicht, aber Klaus drückte auf den Knopf, der die Tür zuhielt.
\WR{Es geht nicht um's Können.
Es geht darum, dass ich es hinter mir lassen \textit{muss}.
Wie willst du sonst weitermachen?
Erklär mir das mal.
So wie jetzt?
Ich hab dich im Auge.
Du isst kaum etwas, du kümmerst dich nicht mal mehr um deine Haare.}

\par

\WR{Bin ich dir also nicht mehr hübsch genug}, gab Laura prompt zurück.

\par

\WR{Darum geht es nicht und das weißt du}, fuhr Klaus unbeirrt fort.
\WR{Dieser Fall hat genug Leben gekostet.
Deines muss es nicht auch sein.
Du hast die Wahl, auch wenn du das nicht hören willst.}

\par

Laura blieb still.
Sie hatte sich wieder von ihrem Partner abgewandt und konnte förmlich spüren, wie er sie mit seinen Blicken durchbohrte.
Doch sie hielt stand und starrte stattdessen für Minuten auf die geschlossene Tür des Aufzugs.

\par

\WR{Also um was geht es diesmal eigentlich?}, fragte der Polizist schließlich.

\par

Laura war dankbar über das Aufgeben ihres Kollegen und antwortete: \WR{Um unsere Bezahlung natürlich.
Wir müssen sie uns verdienen.}

\par

\WR{Schon klar}, entgegnete Klaus sofort und seufzte.
\WR{Ich würde gerne wissen, was wir diesmal dafür tun müssen, wenn die Frau Agentin bereit wäre, es mir zu sagen.}

\par

Die Tür des Fahrstuhls öffnete sich endlich und die beiden traten auf den Gang hinaus.
Obwohl abends und nachts fast genauso viele Agenten im Osloer Geheimdienstzentrum arbeiteten, wie am Tag, war um diese Zeit immer deutlich weniger Betrieb.
Nur gelegentlich liefen andere Mitarbeiter durch die langen Korridore.

\par

Wahrscheinlich war es das ganze Ambiente, dass die Agenten zur Ruhe animierte.
Die matte Beleuchtung schuf, zusammen mit der größtenteils hölzernen Innenausstattung, eine gemütliche Atmosphäre.
Besonders die Wandverkleidungen aus Mahagoni wirkten in der abendlichen Beleuchtung nobel und schön.
Tagsüber traten der ein oder andere Kratzer etwas prominenter hervor.

\par

\WR{Es geht um einen Datendieb}, begann Laura Ghetas zu Rede und Antwort zu stehen.
\WR{Bis auf weiteres als \Wr{Hacker} bekannt.
Wir wissen seinen Namen nicht aber man geht davon aus, dass er sich zurzeit auf der Erde aufhält.
Näheres steht in der Akte, die natürlich auch dir nicht vorenthalten wird.}

\par

Klaus nickte bedächtig.
Beide betraten das Büro.
Laura hielt immer einen extra Schreibtisch für ihren Partner bereit.
Allerdings war dieser deutlich weniger üppig ausgestattet als ihr eigener, auf dem Bilder, ein paar Origamifiguren und gleich fünf Zimmerpflanzen standen.
Klaus fragte immer wieder, wie sie es schaffte, mit dem wenigen Platz zurande zu kommen.
Allerdings schien  es neuerdings so, als würden die Pflanzen bald keinen Platz mehr brauchen, so braun waren manche ihrer Blätter schon.

\par

Die beiden nahmen auf zwei Bürostühlen platz, während Laura die Akte aufrief, die sie von ihrem Vorgesetzten erhalten hatte.
Ein mehrseitiger Eintrag erschien kurz darauf auf dem gläsernen Bildschirm.
Aus Erfahrung wusste Laura, dass solche Berichte zwar immer recht wortreich aber häufig informationsarm ausfielen.
Manchmal wäre es ihr lieber gewesen, an die entsprechenden Stellen hätten man dick gedruckt geschrieben: \Wr{Wir haben keine Ahnung, findet den Rest bitte selbst heraus.}
So musste sie sich durch viele Seiten von Text kämpfen um eine Handvoll Informationen herauslesen zu können.

\par

Sie und Klaus kamen meistens ertwa gleich schnell voran.
Darum schwiegen die beiden und lasen sich den Text aufmerksam durch, während sie das Scrollen übernahm.
In dem Bericht des Geheimdienst-Hauptquartiers auf Kreuzpunkt Primus war von einer ernsten Bedrohung der Datensicherheit die Rede. 
Und tatsächlich schien der unbekannte Hacker in der Vergangenheit recht aktiv gewesen zu sein.
Einige versuchte Attacken auf die Zentralbank, die man mit ihm in Verbindung brachte, lagen schon länger zurück.
Weitere Angriffe auf die Krypta Scientia, die ebenfalls mindestens drei Jahre zurücklagen, schienen dem Computerverbrecher zur Selbstbereicherung gedient zu haben.
Manipulation von Essensmarken, Bevorzugung bei der Vergabe von Plätzen auf interstellaren Reisen und gefälschte Überweisungen auf das Konto einer Firma, die nachweislich nicht existierte.

\par

Dann änderte sich das Schema der Taten.
Die elektronischen Angriffe richteten sich nun plötzlich gegen geheime Daten des Konglomerats.
Dreimal hatte der Hacker versucht, Dienstpläne, Versetzungslisten und Ereignisprotokolle aus allen Bereichen des Konglomerats zu knacken und herunterzuladen.

\par

Dabei hatte der Mann oder die Frau eine direkte Kabelverbindung zu einem Verteilerknoten verwendet.
Eine sinnvolle Vorgehensweise denn mit einer Funkverbindung hätte er oder sie niemals die Verbindungsgeschwindigkeit erreicht, die viele moderne Hacking-Programme brauchten um schnell genug die zahlreichen Schutzfunktionen zu überwinden.

\par

Nach einer halben Stunde waren beide mit dem Lesen des Berichtes durch.
Laura schätzte sich als kaum schlauer als zuvor ein.
Klaus schien es nicht besser zu gehen.
Er brummte leise vor sich hin und sinnierte dann: \WR{Wieso macht er das bloß? Wenn wir ihn erwischen, kommt er vielleicht jahrelang in den Knast.
Und die Daten, die er gestohlen hat, sind vielleicht geheim aber kaum den Aufwand wert.}

\par

\WR{Keine Ahnung}, antwortet Laura abwesend.
Ihr Verstand begann bereits alle Möglichkeiten zu erforschen und in imaginäre Schubladen abzulegen.
\WR{Aber ich denke, es gab hier eine Entwicklung.
Hacken scheint mir anfangs kaum mehr als eine Freizeitbeschäftigung für ihn gewesen zu sein.
Der finanzielle Schaden, den er angerichtet hat, ist kaum der Rede wert.

\par

Aber dann ist er in die vollen gegangen.
Staatliche Daten als Ziel zu wählen.
Auch seine Methoden sind professioneller geworden.
Am Anfang hat er hauptsächlich bestehende Programme für seine Bedürfnisse angepasst.
Aber die Software, mit der er den Knoten in Recife angezapft hat, ist unbekannt.
Er muss sie selbst geschrieben haben.
So etwas bedeutet einen ziemlichen Aufwand.

\par

Es wäre möglich, dass er die gewonnenen Daten an einen fremden Geheimdienst verkaufen will.
Wir wissen, dass die autonomen Welten nicht unsere Mittel haben.
Sie sind auf das Ausnutzen von Sicherheitslecks angewiesen.}

\par

Klaus nickte.
\WR{Könnte sein.
Aber vielleicht hat er auch nur plötzlich ein persönliches Interesse an irgendetwas bekommen.
Vielleicht ist er einer dieser Verschwörungsspinner, der versucht irgendeine abgehobene Theorie zu untermauern.
Wusstest du schon, dass man in der Zeit vor der Seuche geglaubt hat, die Menschheit sei von reptilienartigen Außerirdischen unterwandert gewesen.}
% Oh, wenn du wüsstest...

\par

\WR{Wie dem auch sei}, schob Laura ein, bevor die große Runde der Spekulationen beginnen würde.
\WR{Ich denke, am ehesten erfahren wir von ihm selbst, was er wollte.
Wir müssen Ihn erwischen.
Und dafür brauchen wir Ideen.
Wie schnappen wir also jemanden, von dem wir weder wissen, wie er aussieht, wie groß er ist, ob er eine Frau oder ein Mann ist…
Wir wissen gar nichts.}

\par

Klaus sah sich noch einmal eine Stelle des Berichts an, bis er er triumphierend zu lächeln begann.
\WR{Eine verwertbare Sache wissen wir über ihn.}
Laura sah ihren Partner gespannt an.
\WR{Er benutzt eine direkte Kabelverbindung zu einem Verteilerknoten.}

\par

Die Agentin nickte zustimmend.
Aber der angespannte Ausdruck wich nicht aus ihrem Gesicht.
\WR{Leider hilft uns das nicht wirklich weiter}, warf sie ein.
\WR{Es gibt keine Sicherheitsprogramme, die schnell genug sind um unseren Freund aufzuspüren, bevor er mit seiner Arbeit fertig ist.
Auch wenn er schon jetzt plant an einem bestimmten Ort zur Tat zu schreiten, wissen wir noch lange nicht, wo das sein wird.
Es gibt über zehntrin Verteilerstationen auf der Erde.}

\par

Klaus Rensing blieb einen Moment lang ruhig.
Seinen Gesichtsaudruck kannte Laura gut.
Hin und wieder bedeutete er etwas gutes.

\par

Nach einer Minuten schlug Klaus vor: \WR{Wie wäre es, wenn wir einen Verteilerknoten besonders \Wr{schmackhaft} machen.
Wir könnten eine Finte legen und es so aussehen lassen, als würde einer der Knoten gerade stillliegen weil er überholt wird.}

\par

\WR{Müsste lecker klingen, für ihn}, gestand Laura ein.
\WR{Ein abgeschalteter Knotenpunkt hat keine aktive Abwehr.
Es wäre ein Kinderspiel, ihn in einen niedrigen Zustand hochzufahren und auszunutzen.
Ich befürchte nur, unser Mann würde den Braten riechen.}

\par

Klaus nickte abermals zur Zustimmung.
Mit seinen Fingern trommelte er nervös auf den Tisch.
Das war, anders als ein sehr gespannter Gesichtsausdruck, kein gutes Zeichen für Laura.
Es bedeutete normalerweise, dass sie nun die geistige Arbeit übernehmen musste.
Doch plötzlich schreckte ihr Kollege auf.
\WR{Und was ist mit einer unfreiwilligen Abschaltung?}

\par

\WR{Einem Absturz?}, fragte Laura ehrlich interessiert.

\par

\WR{Genau.
Wir fingieren ein technisches Problem in der Krypta Scientia, von dem jeder Hacker weiß, dass es eine schwere Fehlfunktion auslösen wird.
Am besten ein Problem, dass mehrere Stunden anhält.
Dann hätte unser Mann genügend Zeit, den betroffenen Verteilerknoten zu erreichen.}

\par

\WR{Das könnte klappen}, gestand die Agentin beeindruckt ein.
\WR{Ich denke es wäre klug, gleich mehrere Verteilerknoten zu Absturz zu bringen.
Dadurch stellen wir sicher, dass es der Kerl auf jeden Fall schafft, einen davon zu erreichen um ihn per Kabelverbindung zu hacken.
Außerdem wirkt es glaubwürdiger.}

\par

Klaus fuhr kurz darauf aus seinem Sessel auf.
Voller Elan schlug er sich mit der Faust auf die flache Hand.
Wie Laura ihn kannte, würde er gleich losstürmen um Vorkehrungen zu treffen.
Obwohl Klaus andere Fehler hatte, konnte man ihm eines ganz sicherlich nicht vorwerfen: Faulheit.

\par

\WR{Ich rede mal mit den Damen und Herren von der elektronischen Verwaltung, damit sie etwas entsprechendes in die Wege leiten. 
Bestimmt stellen sie sich stur. Drei vier Verteilerstationen sind zwar kein großer Verlust aber mit unserem Hacker gehen natürlich ein paar Nutznießer einher.
Also besorg uns am besten schon mal eine Anordnung von deinem Boss.}

\par

\WR{Mach ich}, bescheinigte Laura, bevor sich ihr Kollege aufmachte.
\WR{Ich denke, ich unterhalte mich mal mit der Starforce.
Wir können eine von deren Fähren gebrauchen, wenn wir schneller am Ziel sein wollen, wie unser neuer Liebling.}

\par

Klaus nickte eifrig, winkte zum Abschied und sauste in den Korridor hinaus.
Laura bemühte sich zu lächelte.
Es war spät in der Nacht.
Sie war müde, sie hatte einen Jet-Lag.
Aber die Aussicht, mit Klaus auf Verbrecherjagd zu gehen, erschien ihr verlockend und vor allem ablenkend.

\par

Also nahm sie einen großen Schluck Kaffee und sich noch einmal die Akte vor.
Auf den ersten, den zweiten und wahrscheinlich auch den zehnten Blick gab sie nicht viel her.
Aber auch die Taten eines scheinbar Unsichtbaren warfen ein gewisses Licht auf seinen Charakter.
Sie musste es nur sehen lernen.