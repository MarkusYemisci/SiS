Die Innenräume des Phalanx-Basisschiffs, auf dem er stationiert war, erinnerten schwarzer Bär immer wieder an die engen Gänge in einem U-Boot. Die \EN{Heinlein} KBs bot einem Bataillon Soldaten ein mobiles Zuhause. Sie war ein sehr großes Raumschiff mit voluminösem Innenraum für Sportanlagen, Übungskursen, Waffenlagern und Mannschaftsquartieren. Da aber all diese Räume viel Platz in Anspruch nahmen, blieb nur wenig für die Korridore übrig.

\par

Wie alle Schiffe der \EN{Castrum}-Klasse war auch die \EN{Heinlein} so riesenhaft wie unbeweglich. Sie überragte einen mittleren Träger in der Länge um gut dreihundert Meter, verfügte aber nur über wenige Plasmaturbinen.

\par

Offiziell gehörten diese Schiffe zur Navy und wurden auch von einer kleinen Besatzung dieser geführt. Doch im Wesentlichen hatte an Bord der \EN{Heinlein} Legat van Nyst das Kommando, solange das Schiff nicht in einen Raumkampf geraten würde.

\par

Bär hatte gerade sein abendliches Training absolviert und war nun auf dem Rückweg zum Gemeinschaftsschlafsaal. Der Sportpark war zwar groß, doch die Anlagen um die Heimatbasis herum hatten ihm besser gefallen. Dort hatte die Phalanx im Prinzip einen ganzen Planeten gehabt, um Fußballfelder, Trainingsparcours und Schießplätze zu bauen.

\par

Der Schlafsaal, in dem gut einhundert Männer und Frauen Platz hatten, war so gut wie leer. Die meisten tranken noch etwas in der Messe oder übten auf dem Schießplatz. Im Gegensatz zu den beiden anderen Schiffen, die mit der \EN{Heinlein} im Verband flogen, waren auf ihr die Rekruten der Basis untergebracht, die ihre Ausbildung noch nicht abgeschlossen hatten.

\par

Schwarzer Bär setzte sich auf sein Bett. Die wackelige Metallkonstruktion begann zu wanken, als sein muskulöser Körper darauf traf. Der Rekrut in der oberen Etage des Doppelbettes wurde wach und grunzte. Fischauge gähnte und beugte sich über die Kante nach unten. Bär war jedes mal versucht zu lachen, wenn er seinen Kameraden sah. Fischauge~-- eigentlich Kacper Piecek~-- wirkte nicht im geringsten wie ein typischer Phalanx-Marine. Er war weder groß, noch muskulös noch hatte er in irgendeiner Weise ein respekteinflößendes Auftreten. Aber er war der beste Scharfschütze in Bärs Ausbildungsgruppe.

\par

Am Anfang waren die beiden nicht besonders gut miteinander ausgekommen, was hauptsächlich daran lag, dass er Fischauge einmal geschlagen hatte. Mittlerweile tat es ihm leid. Schwarzer Bär war sich sicher, dass sein Kamerad keinen Streit hatte anfangen wollen, sondern bloß nicht gewusst hatte, wie man mit Leuten wie ihm umging. Seither hatte er den Winzling unter seine Fittiche genommen und ihn auch gegen einige Attacken unzufriedener Rekruten verteidigt.

\par

Damit hatte er sich unbewusst am Lehrsystem der Union beteiligt. Bereits in der Schule sollten besonders begabte Kinder oder solche, die in der Ausbildung bereits weiter waren, Verantwortung für einen oder zwei Kameraden übernehmen und sie als Mentoren durch den Alltag begleiten. Da über diese Verantwortung Rechenschaft abgelegt werden musste, wurde nicht nur Schikane von Gleichaltrigen, sondern auch von den Mentoren weitestgehend verhindert. Dieses System fand Anwendung von der Grundschule bis hin zu jeder Universität. Nach einigen Anlaufschwierigkeiten hatte es gezeigt, dass es Verantwortung und Gerechtigkeit bereits früh förderte.

\par

In der Phalanx wurde dieses Vorgehen nicht ganz so wörtlich angewendet aber dennoch genutzt. Schwarzer Bär hätte sicher einen Posten als Mentor bekommen, wenn er nicht das Talent besessen hätte, sich selbst wöchentlich in Schwierigkeiten zu bringen.

\par

\WR{Was ist los?}, fragte Fischauge müde. \WR{Eine Übung?}

\par

Schwarzer Bär schwang sich auf die Matratze. \WR{Nein}, antwortete er. \WR{Tut mir leid. Wollte dich nicht wecken.}

\par

Fischauge grunzte. \WR{Hast du die Rede von Präsident Oschiss gehört?}

\par

\WR{Jep}, machte Schwarzer Bär. \WR{Kleine grüne Männchen. Ich weiß nicht, wieso sich alle so sehr darüber aufregen. Wir latschen schon seit vielen Jahrdinae durchs Weltall. Klar, dass wir irgendwann die Nachbarn treffen.}

\par

\WR{Aber wir sind die, denen sie irgendwann vis-à-vis gegenüberstehen werden.}

\par

\WR{Ich hab dir schonmal gesagt: Lass mich mit dem Fremdwörterscheiß in Ruhe}, meckerte Schwarzer Bär.

\par

\WR{Tut mir leid, hab ich vergessen}, neckte ihn Fischauge. \WR{Du hattest ja kein Französisch auf der Schule, denn du warst auf keiner Schule.}

\par

Schwarzer Bär lachte nur. Als er mit seiner Ausbildung angefangen hätte, wäre ein solcher Satz wahrscheinlich darin geendet, dass es ein blaues Fischauge gegeben hätte. Nun hatte sich aber nicht nur sein Kamerad an ihn, sondern auch er an Kacper gewöhnt.

\par

\WR{Nein im ernst}, redete Fischauge weiter. \WR{Kennst du diese Horrorfilme und Zukunftsbücher von vor der Seuche, die es jetzt als DDV-Videos gibt? Schon damals haben sich die Leute Außerirdische vorgestellt. Und viele von denen sehen echt fies aus. Ich hab keine Lust, von einem drei Meter großen Monster gefressen zu werden.}

\par

Schwarzer Bär lachte. \WR{Blödsinn. Die Leute damals waren Idioten. Besonders die, die sich diese Filme und Bücher ausgedacht haben. Ich weiß auch nicht wie diese Typen aussehen. Aber als sich die Phalanx die Capital Fellowship vorgeknöpft hat, wussten sie über praktisch jeden dieser Terroristen genau Bescheid. Bin sicher, wenn wir gegen die bösen Jungs antreten, wissen die Forscher-Eierköpfe längst, wo deren Hintern sind, damit wir reintreten können.}
