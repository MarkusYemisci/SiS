\WR{Zwei Schiffe auf Abfangkurs, Madame}, meldete Elshe Schwarzschild, nachdem sie ihre Abtastung gegengeprüft hatte. Captain Fiscale hob beide Augenbrauen, als ihr gewahr wurde, wie professionell die junge Frau mit der Situation umging. Ein Blick in die Gesichter der anderen zeigte hingegen immer das gleiche Bild. Eine Mischung aus Angst, Wut und Anstrengung. Doch Elshe Schwarzschild wirkte gänzlich unbeeindruckt von allem, was um sie herum geschah.

\par

Fiscale Fiscale beugte sich über Schwarzschilds Rücken und besah sich die Anzeigen. Der Computer verglich gerade die Abtastergebnisse mit den Aufzeichnungen, fand aber nichts, das den beiden Kontakten auch nur im entferntesten entsprach.

\par

\WR{Für mich sieht das eine wie ein Kreuzer aus}, spekulierte Schwarzschild. \WR{Das andere könnte ein Zerstörer sein. Aber ich kann da ehrlich gesagt bloß raten, Frau Kommandantin.}

\par

\WR{Wie ist der Zeitplan}, fragte Fiscale an Maas Petrarca gerichtet. Der Einsatzleiter prüfte die etlichen digitalen Stoppuhren, die er sich gestellt hatte und antwortete dann: \WR{Unsere Abfangjäger und die \EN{Haie} haben den verstreuten Konvoi erreicht. Sie müssten ihn in den nächsten zwei dutzend Minuten zu uns eskortieren können. Den Sprungpunkt erreichen wir in genau zweidutzend acht Minuten.}

\par

\WR{Zum Glück liegen sich die Hyperraumrouten nach Arktur und nach Wega so eng beieinander}, sagte Commander Samad.

\par

\WR{Konnten Sie Kontakt zur Oberfläche herstellen?}, rief Captain Fiscale an Lieutenant Wallander gerichtet in den oberen Brückebereich hinauf. Doch dieser verneinte. Elshe Schwarzschild hängte dem an: \WR{Madame, die Angreifer haben das Bombardement des Planeten abgebrochen. Ich habe eindutzend sechs Atombomben gezählt. Da unten ist sicher niemand mehr am Leben.}

\par

Captain Fiscale schluckte. Samad bemerkte, dass sie sich krampfhaft an einen Handlauf klammerte und trat an sie heran. \WR{Wir hätten da nichts mehr tun können. Es waren zu viele. In den Orbit einzuschwenken wäre Selbstmord gewesen.}

\par

Die Kommandantin erwiderte den Blick ihres ersten Offiziers nur flüchtig und griff nach dem Hörer, der an einer Säule neben der Tür angebracht war. Auf den ersten Blick wirkte die schiffseigene Sprechanlage ein wenig wie antiquiert. Doch ein Hörer hatte den Vorteil, dass man ein Gespräch führen konnte, ohne das jeder im Umkreis von zwei Metern die Antworten mithörte. Captain Fiscale wählte den Maschinenraum an und sprach: \WR{Lieutenant Bashir, wir brauchen mehr Geschwindigkeit. Bringen Sie den Reaktor auf eindin eindutzend Prodin.}

\par

Die Chefingenieurin, deren Gesicht nur wenige zu sehen bekamen, da sie den Maschinenraum nur selten verließ, antwortete gehetzt: \WR{Die Turbinen sind alt, Madame. Ich bin nicht sicher, ob sie den Ausstoß überstehen.}

\par

\WR{Keine Sorge}, gab Fiscale zurück. \WR{In ein paar Minuten wissen wir es. Eins eins null Prodin, wenn ich bitten darf.}

\par

\WR{Aye, Madame}, antwortete Dilara Bashir und legte auf.
\ortswechsel
Die Meldungen auf dem Gefechtskanal der \EN{Regenvogel} beunruhigten Morten zunehmend. Jens, Kevin und er eskortierten Flug dutzend eins und vier weitere Schiffe in Richtung der Route nach Wega. Sie bildeten die Vorhut. Der Großteil der Jäger der \EN{Regenvogel} wehrte gerade das anrückende feindliche Geschwader ab oder beschützte die langsameren Schiffe.

\par

Anna Farley hatte die graue Staffel weggeschickt und sich mit ihren roten Jungs den etlichen Kontakten angenommen, die mit einem mal auf dem Radar aufgetaucht waren. Morten war klar, dass sie diesen Kampf auf keinen Fall gewinnen konnte und die Funksprüche bestätigten seine Einschätzung. \WR{Verdammt, Rot fünf, wo bist du?}, hörte man einen Piloten verzweifelt rufen. Dann verstummte er plötzlich. Ein anderer meldete ruhiger: \WR{Zwei neue Kontakte sind gerade reingekommen. Das sind zu viele, Rot eins.}

\par

\WR{Bravo vier, die schießen schon wieder Raketen ab. Wenden und abfangen}, hörte Morten den Kommandanten der Abfangjägerstaffel befehlen, der ihn bereits in Arktur zurück zur \EN{Regenvogel} begleitet hatte. Dann schallte ein Schrei voller Angst und Verzweiflung durch den Äther. Morten Hand schnellte zu den Kommunikationskontrollen und er schalteten wieder auf den Kanal seiner eigenen Staffel zurück.

\par

Seit einiger Zeit hatte selbst Kevin kein Wort mehr gesagt, der sonst selbst bei absoluter Funkstille kaum Ruhe gab. Morten versuchte, ruhig zu atmen, doch sein Blick huschte immer wieder voller Nervosität zum Radar, ständig befürchtend es könnten neue Feinde auftauchen. Die Minuten vergingen quälend langsam und während die meisten anderen Staffeln der \EN{Regenvogel} in heftige Kämpfe verwickelt waren, konnte Morten nichts tun, als kontrollierende Blicke auf die Transporter zu werfen, die er mit seiner Staffel bewachte.

\par

Nicht zum ersten mal fiel ihm auf, wie Jens Jäger immer wieder abdriftete und er ständig Korrekturmanöver flog, so als hätte er das Steuer gar nicht in der Hand. \WR{Alles in Ordnung, Grau eins}, sprach er in sein Mikrofon. Eine Zeit lang gab es keine Antwort, doch dann entgegnete Jens endlich: \WR{Ja. Alles bestens.} Aber er klang ganz und gar nicht danach.

\par

Nach einer unerträglichen Viertelstunde kam die \EN{Regenvogel} und ihre beiden Eskortschiffe wieder in Reichweite des Radars. Morten atmete erleichtert auf, doch seine Entspannung hielt nicht lange an. Das Radar zeigte auch etliche Feindkontakte. Zwei größere Bogeys befanden sich noch am äußeren Ende des Erfassungsbereichs, doch etwa ein Dutzend kleinerer Schiffe umschwirrten das Trägerschiff. Schon von weitem konnte Morten das Lichtspiel der ausgetauschten Salven erkennen.
\ortswechsel
Captain Fiscale betrachtete das Bild eines der feindlichen Schiffe, das von einem Echtzeitteleskop geliefert wurde. Der Raumer war stromlinienförmig gebaut, sein Rumpf schien aus einem einzigen Guss zu bestehen. Das vordere Ende erinnerte Fiscale auf eine seltsame Weise an die Zwiebeldächer einer orthodoxen christlichen Kirche. Doch der Bauch des Schiffes wirkte mehr wie der Brustkorb eines kranken, abgemagerten Pferdes, aus dem die Rippen bereits deutlich heraustraten.

\par

Für mehr als einen flüchtigen Blick hatte die Kommandantin allerdings keine Zeit. Ihr Schiff lag unter Beschuss. Schwarzschild hatte elf Jäger gezählt, die um die \EN{Regenvogel} herum flogen und immer wieder Schüsse auf den Träger und seine Belgeitkorvetten abfeuerten. Noch hielten die Schutzfelder, doch die Angreifer gingen sehr methodisch vor. Ihr Ziel schienen die Kanonentürme zu sein. Anders als bei planetaren Schildkuppeln konnten die Geschütze nicht durch die Panzerfelder des Trägers schießen. Darum wurden sie selbst von der Abschirmung ausgespart und waren somit angreifbar.

\par

Das Deck erbebte, als einer der angreifenden Jäger ein paar gut gezielte Salven auf einen Geschützturm steuerbords abfeuerte und ihn damit schnell zur Explosion brachte.

\par

\WR{Die Charlie Staffel soll das Feuer…}, begann Captain Fiscale, doch sie kam nicht weit. Völlig außerstande irgendwie zu helfen, musste sie mit ansehen, wie einer der Verteidigungsjäger der \EN{Regenvogel} von den Angreifern in die Zange genommen und abgeschossen wurde. Das Schiff trieb als brennendes Wrack davon. Von einer Rettungskapsel gab es keine Spur.

\par

\WR{Petrarca, wie lange, bis die Transporter springen?}, fauchte Fiscale.

\par

\WR{Die ersten Schiffe haben den Hyperraumknoten fast erreicht. Die größeren Transporter brauchen noch etwa fünf Minuten. Mit ihnen kommen auch unsere ausgesandten Staffeln zurück.}

\par

\WR{Na gut}, begann die Kommandantin. \WR{Steuer, wenden Sie und nehmen Sie Kurs auf diese beiden größeren Schiffe. Wallander, sagen Sie unseren Korvetten, sie sollen zur Wega-Route fliegen und sie um jeden Preis halten. Wir machen ihnen den Rücken frei.}
