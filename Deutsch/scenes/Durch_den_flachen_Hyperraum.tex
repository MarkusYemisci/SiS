\WR{Rote Staffel, blaue Staffel}, war Maas Petrarcas Stimme über Funk zu hören. \WR{Nehmen Sie sofort Kurs auf die \EN{Heinlein}. Fliegen Sie mit ihr nach Kreuzpunkt Primus und begleiten Sie ihre Landetransporter! Wir binden die feindlichen Kräfte hier. Sie sollten keine Schwierigkeiten haben, in den Orbit zu kommen. Aber passen Sie auf! Nur weil noch keines der feindlichen Schiffe in eine tiefere Umlaufbahn eingeschwenkt hat, bedeutet das nicht, dass es dort friedlich zugeht! Bis auf weiteres erhalten Sie ihre Befehle von General van Nyst. Er kommandiert die \EN{Heinlein}. Gute Jagd!}

\par

Anna Farleys Staffel befand sich seit dem Abfangen der Raketen etwas abseits vom Kampfgeschehen und so war es kein Problem, den Kurs zu ändern und sich zu den Basisschiffen der Marineinfanteristen durchzuschleichen. Zwei Jägerstaffeln würden zwar sicher auffallen aber auf dem Radar waren keine Shutek zu erkennen, die nahe genug wären, um sie noch abzufangen, bevor sie die \EN{Heinlein} erreichen würden.

\par

\WR{Na toll, Babysitting}, meckerte Kevin über Funk und entlockte Morten dabei sogar ein Lächeln. Selbst die Gefahr in der sich nicht nur sein Flügelmann selbst, sondern auch seine Lebensgefährtin befand, taten seiner Art keinen Abbruch. Er war ein kindsköpfiger Draufgänger. Aber dabei zumindest authentisch.

\par

\WR{Sieh's positiv, Rot vier}, antwortete Morten schließlich und achtete dabei darauf, dass er auch wirklich nur an Kevin sendete. \WR{Wenn du Glück hast, kann dich deine Freundin beim Fliegen beobachten. Das muss doch Eindruck machen, oder?}

\par

Zur Linken wurde langsam Staffel blau erkennbar. Eine Maschine stach dabei besonders hervor. Dexter Hennington flog für diesen Einsatz ebenfalls einen Falken, statt dem schweren Jäger, den er sonst bevorzugte. Statt mit den üblichen blauen Streifen, war sein Schiff mit goldenen Mustern verziert, die an Tätowierungen erinnerten.

\par

\WR{Und? Wie viele habt ihr, Schatz?} Morten kam fast die Galle hoch, als er Henningtons Stimme auf dem allgemeinen Staffelkanal hörte. Die Frage war zwar an Anna Farley gerichtet aber definitiv für die Ohren aller gedacht.

\par

\WR{Ich hab nicht mitgezählt}, war Annas sofortige Antwort. \WR{Das mache ich, wenn wir wieder \textit{sicher} gelandet sind, was das allerwichtigste ist.}

\par

Natürlich konnte es Kevin nicht lassen. \WR{Drei auf meinem Konto~-- und das nur bis jetzt.}

\par

\WR{Halt die Klappe, Kotzbrocken. Mach dir erst mal einen Namen, dann reden wir weiter.}

\par

Morten hätte fast etwas eingeworfen, da kam ihm Anna Farley zuvor. \WR{Protzen könnt ihr später. Jetzt müssen wir uns erst mal unsere Aufgabe konzentrieren. Haltet den Kanal frei für wichtige Dinge!}

\par

\WR{Oho, da kehrt jemand den Wing Commander raus.} Dexter versuchte neckisch zu klingen, blieb jedoch ruhig, nachdem weder Anna, noch jemand anderes darauf einging.

\par

Langsam kamen die \EN{Heinlein} und zwei weitere Basisschiffe in Sichtweite. Die länglichen Vehikel waren keine Schönheiten. Ihr quaderförmiger Hauptrumpf erinnerte an Containerschiffe, die in der Zeit vor der Seuche auf den Meeren der Erde unterwegs gewesen waren. Gegen den massigen Körper, der sogar einen mittleren Träger überragte, wirkte die Brücke genauso winzig, wie die an den Seiten in Reih und Glied angebrachten Abwurfschiffe.

\par

Bei dem Anblick dieser Riesen konnte Morten kaum glauben, dass sie in der Lage waren, auf einem Planeten zu landen.

\par

In geringerem Maße verwunderte ihn, dass eine eingehende Videoübertragung auf seinem Kommunikationsmonitor angezeigt wurde. In Situationen wie dieser beschränkten sich die meisten auf Funksprüche. Nachdem er sie bestätigt hatte, war General van Nysts Gesicht auf dem Schirm zu erkennen.

\par

\WR{Piloten der roten und blauen Staffel, Flankenposition einnehmen. Wir werden demnächst in den flachen Hyperraum eintreten und sie im Kielwasser mitnehmen. Die \EN{Crossguard} schickt uns ständig die Ergebnisse ihrer Abtastungen und es sieht schlimm aus. Im Orbit um Kreuzpunkt Primus wimmelt es nur so von feindlichen Jägern. Rechnen Sie damit, dass wir sofort angegriffen werden.}

\par

Die Stirn des Generals lag in so tiefen Falten, dass man sie bereits als Schützengräben hätte verwenden können.

\par

\WR{Der Plan ist einfach. Legat Gajjar berichtet von zwei Fronten am Boden. Eine im Norden, eine im Süden. Wir werden Landungsschiffe runter schicken, die beide unterstützen sollen. Die sind erstens schneller und zweitens müssen sich die Shutek so auf mehrere Ziele konzentrieren. Zu verteidigender Raum ist sehr weitläufig, Feindpräsenz hoch. Teilen Sie sich also auf. Und denken Sie daran: wenn ein Soldat sterben muss, dann wenigstens auf dem Schlachtfeld mit einem Gewehr in der Hand und nicht in einer Blechbüchse beim Landeanflug. Das ist alles. Gutes Gelingen!}

\par

\WR{Ihr, habt's gehört, Jungs und Mädels}, begann Anna Farley. \WR{Dex, du und deine Hampelmänner fliegt runter zur Südfront. Wir nehmen uns den Norden vor. Uns wird nicht so schnell kalt.}

\par

Morten sah zu Kreuzpunkt Primus. Aus der Entfernung wirkte der Planet unglaublich still und friedlich. Nichts deutete darauf hin, dass aus seiner Oberfläche gerade ein tödlicher Kampf tobte. Die Shutek schienen sich nicht für die drei kleineren Kontinente zu interessieren. Zumindest hatte van Nyst sie nicht erwähnt. Aber auch auf dem großen Hauptkontinent, der sich von knapp unterhalb des Äquators bis hin zu den Polen erstreckte, war nichts vom den entbrannten Gefechten zu sehen.

\par

\WR{Navigationsdaten kommen vom Leitstand der \EN{Heinlein}}, begann Farley. \WR{Nehmt eure Positionen ein und dann \Wr{Bon Voyage}.}

\par

Verschiedene Bestätigungen wurden über den Staffelkanal gesendet, während Morten schon seinen Navigationsomputer mit dem der \EN{Heinlein} verband. Eine Checkliste ratterte auf einem freien Monitor herunter und zeigte, den Zustand der Vernetzung. Fast jede dritte Zeile wurde in gelb dargestellt, was sie als Warnung des Computers auswies. Unter Gefechtsbedingungen wurden nur meistens nur die nötigsten Überprüfungen für einen berechneten Kurs durchgeführt. Der Flug durch den flachen Hyperraum war zwar deutlich weniger gefährlich als durch sein tieferes Gegenstück, doch fühlte sich Morten mit der berechneten Lösung nicht sonderlich wohl.

\par

Langsam bewegte er seinen Jäger in die unmittelbare Nähe des Basisschiffes, bis er förmlich das Gefühl hatte, am Rumpf zu kleben. Tatsächlich konnte er ansatzweise ein paar Besatzungsmitglieder hinter den Fenstern des Schiffes erkennen.

\par

Das Zielmonokel, das Morten nach wie vor trug, zeigte ihm, dass sich sein Schiff im vorgesehenen Bereich befand. Damit übergab er die Kontrolle an den Leitstand der \EN{Heinlein}.

\par

Als er den Steuerknüppel losließ und sich ein wenig zurücklehnte, spürte er, wie angespannt er war. Obwohl er sich in seinem engen Cockpit kaum bewegen konnte, fühlte er, wie ihn fast jeder seiner Muskeln schmerzte.

\par

\WR{Jäger in Position}, meldete Farley knapp an die \EN{Heinlein}.

\par

\WR{Verstanden}, war die prompte Antwort einer anonymen Stimme eines Offiziers auf der Brücke des fliegenden Kastens. \WR{Eintritt in den flachen Hyperraum in dutzend acht Sekunden.}

\par

Morten straffte seine Haltegurte.

\par

\WR{Viel Glück, Rot drei}, wünschte Kevin und klang dabei nicht ansatzweise albern, was sich jedoch schnell änderte. \WR{Falls ich auf der anderen Seite nicht wieder rauskomme, sag deiner Mutter, ich liebe sie.}

\par

Morten seufte und öffnete dann einen Kanal nur an Kevin gerichtet. \WR{Mach ich. Und wenn ich es nicht schaffe, sag deinem Vater das gleiche.}

\par

\WR{Sprung in drei, zwei, eins...}

\par

Die Beschleunigung auf Sprunggeschwindigkeit war dank der lebenswichtigen Trägheitsabsorber nicht zu spüren gewesen. Aber der Übergang in den flachen Hyperraum brachte Morten Übelkeit.

\par

Fast wie Nebel wirbelten Lichterscheinungen um die Basisschiffe und die Jäger um sie herum. Kreuzpunkt Primus schien mit jedem Augenblick anzuschwellen, während sich das Sternenfeld im Hintergrund nicht veränderte.

\par

\WR{Bitte wiederholen}, sprach Morten in sein Headset, nachdem er glaubte, einen Funkspruch überhört zu haben.

\par

Eine Zeit lang antwortete ihm niemand.
Dann hörte er Anna Farley sagen: \WR{Rot drei, es gab keine Übertragungen mehr, seit wir in den Hyperraum eingetreten sind. Was meinen Sie?}

\par

Doch Morten hörte es nach wie vor deutlich. Jemand sprach, sehr leise aber trotzdem klar auszumachen. Die Worte wurden überlagert von einer Reihe von Geräuschen, die Morten aus einer Dokumentation über die Zeit vor der Seuche bekannt war. Damals waren Telefonleitungen dafür benutzt worden, eine krude Vorform der Crypta Scientia zu betreiben und Modems hatten die analogen Signale digitalisiert. Die dabei entstehenden Geräusche waren als überaus unangenehm beschrieben worden.

\par

So ähnlich klang, was Morten nun hörte. Hastig setzt er sein Headset ab, nur um festzustellen, dass sich nichts änderte. Aufgeregt blicke er um sich, sah aber nur die \EN{Heinlein} und die anderen drei Jäger der roten Staffel. Außerdem war ihm klar, dass sich Geräusche nicht durch ein Vakuum verbreiten konnten und die Wechselwirkung seines Vehikels mit dem Hyperraum klang völlig anders.

\par

Angestrengt versucht er, sich auf die Worte unter dem leisen Lärm zu konzentrieren. Je genauer er hinhörte, umso eher verstand er. Zwei Stimmen redeten besonders intensiv miteinander. Aber die Worte waren viel zu leise, als dass er einen zusammenhängenden Satz ausmachen konnte. Irgendwie klangen die Stimmen für ihn wie eine singende Menge von Menschen, bei denen die einzelnen aber nicht laut genug sangen und der Text somit unterging. Zwei Worte hörte er aber trotzdem heraus. \Wr{Stratosphäre} und \Wr{Ring}.

\par

Dann klang eine andere Stimme immer deutlicher an sein Ohr. Als er realisierte, dass sie aus seinem Headset kam, setzte er dieses hastig wieder auf.

\par

\WR{Rot drei, bitte melden!}, forderte Anna Farley. \WR{Morten, bist du in Ordnung?}

\par

\WR{Ja Madam. Entschuldigung!}, erwiderte der Gefragte hastig. \WR{Ich… Ich weiß nicht, was gerade passiert ist.} Sollte er ihr sagen, dass er Stimmen hörte? Damit wäre ihm ein Flugverbot sicher. Besonders, wenn er dann noch Auskunft über seine Alpträume während der tiefen Sprünge geben musste.

\par

\WR{Ich wollte bloß mein Headset überprüfen, da ist es mir runterfallen}, log er. Anna Farley blieb still. Natürlich glaubte sie ihm nicht und er konnte es ihr nicht verübeln.

\par

Doch der baldige Austritt aus dem Hyperraum unterbrach weitere Gespräche darüber. \WR{Bereithalten für Austritt. Unsere Abtaster zeigen \textit{etliche} Kontakte. Machen Sie sich auf etwas gefasst!}, warnte der Leitstand der \EN{Heinlein}. \WR{Verlassen Hyperraum in drei, zwei, eins…}

\par

Kaum war Morten in den Sessel seines Cockpits hineingepresst worden, zuckten schon die ersten Strahlen um ihn herum und er wurde einige male getroffen.