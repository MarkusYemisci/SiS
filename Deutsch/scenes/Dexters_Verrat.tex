Simon Maddeux und Dexter Hennington waren weit hinter ihre beiden Staffeln zurückgefallen. Es war ein harter Kampf gewesen und ihre beiden Jäger hatten einige Treffer eingesteckt. Simons Falke hatte die Spitze seiner Backbordtragfläche verloren und Dexters schwerer Jäger hatte einen Streifschuss an der rechten Rumpfhälfte erhalten. Immer wieder sprühten Funken aus dem klaffenden Streifen. Die Schadensdiagnostik lief noch, aber es schienen einige Bordsysteme getroffen worden zu sein. Unter anderem war der Flugschreiber völlig zerstört.

\par

Die Treffer, die die beiden einstecken hatten müssen, waren jedoch nicht umsonst gewesen. Gemeinsam mit dem Rest ihrer Staffeln, der nun mit den größeren und langsameren Transportern in Richtung Wega-Route unterwegs waren, hatten sie es geschafft, die Angreifer auf Distanz zu halten und einen Großteil des Konvois zu retten.

\par

Doch Simon konnte das Bild der brennenden \EN{Pales fünf} nicht aus dem Kopf bekommen, die schließlich explodiert war und dabei Mehr als dreitausend Menschen mit in den Tod gerissen hatte.

\par

Nun bildeten die beiden die Nachhut, um den Hauptteil des Geschwaders vor nachrückenden Feinden zu warnen. Collonel Maddeux sah auf sein Radar, doch noch zeigte es keine Verfolger. In seiner Reichweite befand sich lediglich Dexters Schiff.

\par

Ferne Lichtblitze zeugten von dem Kampf, in den sich die \EN{Regenvogel} gestürzt hatte. Der Träger hatte mittlerweile sein Transpondersignal zugeschaltet. So war es sogar zu sehen, wenn es sich außerhalb der Radarreichweite befand. Bei Kampfeinsätzen blieb es meistens aus, um dem Feind die Ortung zu erschweren, doch nun schien sich die \EN{Regenvogel} ohnehin im Gefecht zu befinden. Simon Maddeux klinkte sich in den Nahbereichkanal des Trägerschiffs ein, der dazu eingesetzt wurde, die Verteidigung des Schiffes zu koordinieren.

\par

\WR{Bravo eins und zwei, weg von unserem Bug. Wir feuern gleich Raketen ab!}, hörte er Lieutenant Wallander plärren.

\par

Wenig später schrie ein Pilot, den Simon als Brat Ganon kannte: \WR{Kann Anflug nicht fortsetzen. Abwehrfeuer zu dicht. Scheiße, ich bin getroffen.} Der Rest der Übertragung ging in chaotischem Rauschen unter.

\par

Simons Blick kehrte zum Radar zurück. Er hatte gar nicht bemerkt, wie Dexter immer weiter hinter ihn zurückgefallen war. Nun befand sich sein Jäger direkt achtern. Maddeux blickte verwundert auf die Entfernungsanzeige. Zuerst fragte er sich, ob Dexter Feinde auf dem Radar hatte und einen waghalsigen Alleingang hinlegen wollte. Aber obwohl er ihn als mutigen Draufgänger mit großer Klappe kannte, wusste er doch, dass sich der Anführer der Eisenhämmer zumindest während eines Einsatzes fast immer absolut professionell verhielt.

\par

\WR{Dex, was ist los?}, fragte Maddeux in sein Mikrophon, nachdem er auf einen Kanal zu seinem Flügelmann geschaltet hatte. \WR{Du fällst zurück. Hast du Probleme mit deinem Triebwerk?}

\par

Eine Antwort erhielt Simon in Form von vier hell gleißenden Strahlenbündeln, die sich problemlos durch die Achterschilde seines Falken fraßen und das Heck des Jägers in einen Feuerball einhüllten. Nur einen Augenblick später gellten in Simons Kopfhörern so ziemlich alle Alarmsignale auf, die er im Verlauf seiner Ausbildung kennen gelernt hatte. Die Hälfte aller Anzeigen auf dem Armaturenbrett war ausgefallen. Der Rest warnte vor jeder nur vorstellbaren Katastrophe, die einem Piloten passieren konnte.

\par

Ein Blick von Simon Maddeux über seinen Rücken genügte, um zu erkennen, dass das Heck seines, wie wild trudelnden, Jägers fast komplett abgerissen war. Es musste an ein Wunder grenzen, dass die Capezintanks der Nachbrenner nicht längst in die Luft geflogen waren. Ein Sicherungssystem blies den übrigen Treibstoff ins All, um eine Entzündung zu verhindern, was das Trudeln von Simons Jäger noch verstärkte.

\par

Und alles, an was er nun denken konnte war, dass sein Flügelmann, ein Pilot mit dem er seit Jahren zusammengearbeitet hatte und dem er trotz seiner Art stets vertraut hatte, seinen Tod einleiten würde. Keine Frage nach dem Warum, keine Überlegung über den Grund für diesen Angriff schoss ihm durch den Kopf. Nur den Drang, die anderen zu warnen.

\par

Simons Hand griff nach den Kommunikationskontrollen, gerade als der Alarm für den Ausstieg ertönte. Doch Simon blieb weder Zeit, eine Nachricht abzusetzen, noch das Cockpit seines Jägers abzusprengen. Er erkannte ein letztes Mal den Jäger seines Mörders, bevor Dexter erneut seine Waffen sprechen ließ.
