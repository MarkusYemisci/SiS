\WR{Ich dachte mir schon, dass ich dich hier finde.} Laura erkannte die Stimme ihres Partners sofort, hatte aber bereits davor geahnt, dass es Klaus Rensing war, der gerade durch die Tür der kleinen Kneipe gekommen war. \WR{Aber außer mir findet dich hier niemand. Kroatien hat weitaus schönere Flecken.}

\par

Laura drehte sich nicht um, sondern sah weiter auf den kleinen Holoprojektor am Ende des Thresens. Der Wirt kannte sie bereits gut genug, um sie alleine zu lassen, während er mit seiner Familie im Hinterzimmer zu Mittag aß. Dass ihn seine Bar nicht übermäßig interessierte, erkannte man leicht an ihrem Zustand. Zwar war der kleine Raum absolut sauber und der Boden sogar frisch gewienert, doch die Wände hätten längst einen neuen Anstrich verdient gehabt. Die vielen Flecken und anderen Unschönheiten der Tapete hatte der Besitzer durch Aufhängen von Bildern, Figuren, Blumensträußen, Steinen und anderem Ramsch überdeckt. Auch eine Fensterscheibe im Eingangsbereich war zerbrochen und nur durch aufgeklebte Pappe ersetzt worden. Der Blick nach draußen auf den Strand, das kristallklare Meer und die steinige Insel Krk waren jedoch so sehr Blickfang, dass viel Zierde in der Bar und dessen Veranda Verschwendung gewesen wäre.

\par

\WR{Wenn du dich verstecken willst, darfst du nicht in deine Stammkneipe gehen}, sagte Klaus nach einer längeren Ruhe. \WR{Auch, wenn du mir gesagt hast, die wäre in Hong Kong. Dafür kenne ich dich einfach zu gut.}

\par

Nach wie vor sah Laura ihren Partner nicht an, sondern hielt sich ihren Lindenblütentee unter die Nase. Sie trank ihn schon seit einiger Zeit nur noch in dieser Bar. Als Grund, immer wieder dorthin zurückzukommen. Nicht, weil der Wirt ihn dort besonders gut machen würde. Er roch intensiv, doch hatte er nun fast jeden Geschmack verloren. So fühlte es sich an. Als hätte das ganze Leben seinen Geschmack verloren.

\par

\WR{O'Shea ist immer noch ziemlich sauer}, gab Klaus zu bedenken. \WR{Aber er wird sich sicher wieder einkriegen. Du bist trotz allem nach wie vor seine Beste. Er braucht dich sicher mehr als umgekehrt.}

\par

Laura konnte sich kaum noch daran erinnern, was ihr Vorgesetzter ihr am gestrigen Abend gesagt hatte. Nachdem ihrer Begegnung mit dem Scharfschützen war ihr ihre ganze Existenz mit einem mal wie der Abspann eines Traumfilmes vorgekommen. Er war noch nicht vorbei, doch außer einem dunklen Hintergrund gab es nichts mehr zu sehen.

\par

Klaus seufzte. \WR{In Ordnung, heute kommst du mir nicht davon. Wir bereden das. Hier und jetzt.}

\par

Laura ignorierte ihren Kollegen und erhob sich stattdessen, ohne ihn eines Blickes zu würdigen. Schnell kritzelte sie etwas in ihr Buch und bezahlte auf diese Weise. Dann verschwand sie durch die quietschende Tür auf die Veranda, dicht gefolgt von Klaus.

\par

Sofort drang den beiden das unterschwellige Aroma von Bratfett in die Nase. Laura war sich nicht sicher, ob er aus der Küche des Wirts oder aus einem nahen Hotel stammte. Doch er erinnerte sie auf eine subtile Weise mehr an die Gegend, als es das Rauschen der Wellen oder der Geruch des Meeres tat.

\par

Vor ihr erstreckte sich ein langer Strand voller Kies. Zu ihrer Rechten erhob sich ein felsiger Abhang, auf das man ein Hotel von vor der Seuche wiedererichtet hatte. Die Adria zu ihrer Linken. Es gab kein Entkommen. Klaus würde nicht locker lassen, das wusste sie.

\par

\WR{Okay.} Sie klang bei weitem nicht so wütend oder hart, wie sie es beabsichtigt hatte. Vielmehr klang ihre Stimme brüchig. \WR{Was kann ich dir noch sagen? Johanna Bell ist tot. Sie ist in einem Sarg erstickt und es war meine Schuld.}

\par

\WR{Das war es nicht. Du konntest…}

\par

Laura unterbrach ihren Partner sofort. \WR{Nein. Hör auf. Wenn du mit mir reden willst, dann mach es richtig. Und dazu gehört auch, die Dinge zu aktzeptieren wie sie sind.}

\par

Sie konnte nicht mehr weitergehen und Klaus Blick hielt sie auch nicht stand. Also setzte sie sich einfach auf die nahe Mauer und ließ ihre Füße über dem Wasser baumeln.

\par

\WR{Ich kenne Typen wie Anthony Hannibal. Den Leuten möglichst viel Schmerz zu bereiten ist alles, was sie interessiert.} Klaus legte Laura die Hand auf die Schulter und war überraschte, dass sie nicht sofort aufstand. \WR{Es gab nichts, was du hättest tun können.}

\par

Tränen waren in ihre Augen getreten, ohne dass sie es bemerkte hatte. \WR{Er wusste, wo Johanna begraben war. Er hat sie dort verscharrt. Ich hätte es irgendwie aus ihm herausholen müssen. Aber ich konnte nicht. Ich weiß nur noch wie er mich angegrinst hat und meine Sicherungen durchgebrannt sind. Bestimmt hat er jeden Schlag genossen, den ich ihm verpasst habe. Und mit jeder Sekunde, die meine Kollegen damit verbacht haben, dieses verdammte Verhör irgendwie vor mir zu retten, ist Johanna mehr und mehr die Luft zum Atmen ausgegangen.}

\par

Klaus nahm neben Laura platz. Seine Füße waren zu lang, um über dem Wasser hin und her zu schwingen. Die Wellen schwappten immer wieder über seine teuren Kunstlederschuhe hinweg, doch es schien ihn nicht weiter zu kümmern.

\par

\WR{Du weißt, dass er genau das wollte, oder?}, fragte er rhetorisch. \WR{Er hatte nie vor, dir~-- oder irgendwem~-- zu sagen, wo er Johanna vergraben hatte.}

\par

Laura schüttelte heftig den Kopf und schluchzte. \WR{Na und? Darum geht es doch gerade. Ich bin Agentin. Ich war für diesen Fall verantwortlich. Es war meine Aufgabe, herauszufinden, wo sie war. Ob er es mir sagen wollte, oder nicht. Nicht nur, hab ich das nicht gepackt, ich habe sein Scheißspiel auch noch mitgespielt.}

\par

Klaus antwortete für gewöhnlich immer sehr schnell. Manchmal auf Kosten der Bedachtheit. Doch dieses mal war er es, der lange schwieg. Der Klang der Wellen hatte etwas beruhigendes, auf sie beide. Wie das Wasser immer wieder gegen die Mauer gespült wurde und dabei im Rhythmus rauschte.

\par

\WR{Es tut mir so leid, dass ich nicht da war}, sagte er schließlich und seine Stimme klang dabei nun selbst sehr brüchig.

\par

Laura schluckte. Den Verdächtigen zu stellen war nicht einfach gewesen. Am Ende war er ihnen zwar in die Falle gegangen, hatte aber mehr Widerstand geleistet, als erwartet. Klaus war angeschossen worden und hatte die Nacht im Krankenhaus statt dem Verhörraum verbracht.

\par

\WR{Hannibal hatte damals schon vier Kinder auf dem Gewissen}, fuhr Klaus schließlich fort. \WR{Alle grausam umgebracht. Er wäre mit oder ohne Johanna für immer im Knast gelandet. Ich bin mir sicher, er hätte dir nie gesagt, wo sie ist. Du hast ihn besiegt. Du warst klüger als er. Darum hatte er eine Wut auf dich und wollte zumindest einen kleinen Sieg einfahren, indem er dich aus der Fassung bringt.}

\par

Klaus Hand griff nach Lauras und drückte sie fest. \WR{Das musst du verstehen. Er wollte dich fertigmachen, weil du gewonnen hattest. Wärst du nicht gewesen, wäre er vielleicht immer noch da draußen und würde Kinder ermorden. Nachdem wir ihm Handschellen angelegt haben, war alles, was er noch tun konnte, dir das Gefühl zu geben, dass du versagt hast.}

\par

Laura versuchte wie immer, nicht an den Tag zurückzudenken. Und wie immer versagte sie. Sie sah ihr Buch vor sich aufgeklappt. Den Permutare-Bericht über verkaufte Särge. Nicht sehr umfangreich, da diese Art der Bestattung nicht mehr üblich war. Auf Miranda sowieso nicht. Und noch seltener waren Särge in Kindergröße.

\par

Dann die geologische Übersicht von Neu Tokyo und dem Umland. Der Boden war fast überall sehr felsig. Löcher waren kaum mit leichtem Werkzeug auszuheben. Größere Maschinerie würde aber auffallen. Es blieben nur noch wenige Waldstücke übrig, in denen man relativ problemlos einen Sarg vergraben konnte.

\par

Die Suche war schnell vorangekommen. Tatsächlich hatte Johanna eine Socke verloren. Und selbst im achtundzwanzigsten Jahrhundert gab es nichts, dass so schnell Witterung aufnahm, wie die Nase eine Spürhundes.

\par

Das Meer verschwand und Laura fand sich wieder in dem Waldstück. Das golendene Sonnenlicht fiel durch das Blätterdach. Der Wind rauschte. Bis auf das Rascheln der Schuhe der Suchmannschaft war alles ruhig. Selbst die Vögel waren außergewöhnlich stumm. Es war viel zu warm für einen Herbsttag, doch Laura interessierte das nicht.

\par

Ihre Hand war nich wie vor blau vom ständigen Aufprall in Hanniballs Gesicht. Doch auch diesen Schmerz nahm sie kaum war, denn beim Anblick des kleinen Hügels fühlte sie sich wie gelähmt. Als wäre er ein Raubtier, dass sie jeden Moment fressen würde. Zwar würde ihr das erspart werden, doch als sie ihn erklomm, wurde ihr klar, dass sie mit jedem Schritt dem Ende ihres Lebens ein bisschen näher kam.

\par

Eine Gruppe von Polizisten waren bereits dabei zu graben. Sie beeilten sich, als ginge es um Leben und Tod. Im Sekundentakt flog der Dreck von ihren Spaten. Doch Laura war klar, dass es bereits zu spät war. Ihr Buch zeigte ihr elf uhr und zweieinhalbdutzend. Hanniball hatte ihr gesagt, Johannas Sauerstoff würde um acht Uhr ausgehen. Am Abend des Vortages.

\par

Dann stieß einer der Spaten auf etwas hartes. Wenige Minute später war der Sarg freigelegt. Er war mit Metall ummantelt, um ihn hermetisch abzuschließen. Keine Luft kam hinein und keine kam hinaus.

\par

Laura schrie ihre Helfer an zur Seite zu gehen und fiel vor dem Sarg in die Knie. Sie riss das Schloss auf und hob den Deckel an, nur, um  sofort in Tränen auszubrechen. Vor ihr lag der Leichnam eines kleinen Mädchens in einem Pullover mit einer gelben Ente darauf. Wie auf den Fahndungsfotos. Die Augen friedlich geschlossen, als würde sie schlafen. Doch ihre blasse Haut verriet, dass sie tot war. Das Beatmungsgerät, dass ihr Hanniball mit Gewalt umgeschnallt hatte, schlang sich immer noch wie ein Fremdkörper um ihr Gesicht.

\par

Laura versuchte, es abzureißen und spürte dabei die unnatürliche Kälte auf Johannas Haut.

\par

\WR{Ich weiß nicht, was ich dir noch sagen kann. Nur, dass du nichts dafür kannst}, beschwörte Klaus.

\par

Seine Partnerin brauchte eine Weile, um zu verstehen, was er sagte. Dann antwortete sie: \WR{Ich hätte sie retten müssen. Das hätte nicht passieren dürfen. Es ist wie ein Fehler in der Geschichte dieser Welt. Danke, dass du versucht hast, mir zu helfen.}

\par

Klaus hielt nach wie vor ihre Hand und griff nun noch fester zu. \WR{Laura, bitte versprich mir, dass du nichts dummes tust.}

\par

Nun riss sie sich los und begegnete seinem mit einem eigenen, flammenden Blick. In ihrem Augen spiegelte sich die Wut, erwischt worden zu sein.

\par

Klaus ließ sich nicht beirren, als er fortfuhr. \WR{Ich habe mit ein paar der Zeugen in Oslo geredet. Du hast dich dem Heckenschützen förmlich in die Schusslinie gestellt. Bitte sei ehrlich. Wolltest du, dass er dich erschießt?}

\par

Laura sah wieder auf das Wasser. Zu aufgewühlt, um eine Reflexion ihres Gesichts zu zeigen. \WR{Eine Welt, die keinen Platz für Johanna Bell hat, hat auch keinen Platz für mich.}

\par

Klaus atmete schwer. \WR{Bitte, tu das nicht. Ich weiß, es ist hart. Ich weiß dass, weil ich nicht für dich da war. Ich hab mir eine Kugel eingefangen. Darum musstest du dieses Verhör alleine bestreiten.}

\par

\WR{Also denkst du, dass es anders hätte laufen können?}, fragte Laura tonlos.

\par

\WR{Was weiß ich.} Ihr Partner blaffte fast. \WR{Ich bin kein Hellseher. Darum geht es mir nicht. Was ich sagen wollte, war, dass ich nicht da war. Ich wollte nicht, dass du alleine bist. Aber wenigstens jetzt, kann ich für dich da sein. Diese Zeit wird vorbeigehen, das schwöre ich dir. Du wirst wieder lachen, mit einer hübschen Dame Wein trinken. Was auch immer du willst. Es liegt an dir. Gib nicht auf. Das ganze wird Zeit brauchen, das weiß ich ja. Aber eines Tages…}

\par

Laura schwang sich förmlich aus Klaus Nähe. Für sie überraschend blieb er sitzen. Für einen Moment hatte sie gedacht, er würde ihr nachlaufen wollen.

\par

\WR{Weißt du, ich werde nicht heute sterben, oder morgen. Oder in vier Jahren. Nicht an Alterschwäche, einer Kugel oder einen Messer zwischen den Rippen. Ich bin in dem Moment gestorben, als ich Johanna ein letztes mal aus den Armen gegeben habe.}
