\WR{Einschlag in drei Minuten und zwei dutzend Sekunden!}, rief Elshe Schwarzschild aufgeregt ihrer Kommandantin entgegen. Diese hatte keine Gelegenheit, zu reagieren, denn Maas Petrarca meldete sofort darauf: \WR{Staffel Bravo ist nicht in Position zum Abfangen!}

\par

Captain Fiscale ging schnell aber scheinbar gelassen zum Geländer, von dem aus sie den unteren Bereich der Brücke besser sehen konnte. An beiden Flanken zeigten die Glasmonitore der Kanoniere Flugbahnen und Schussrichtungen von feindlichen Jägern und den eigenen Kanonen. Die Shutek schwirrten um die \EN{Regenvogel} herum und versuchten immer wieder, ihre Geschütze zu treffen. Wie die meisten Jäger, waren auch die Maschinen der Shutek zu schwach bewaffnet, um den Schutzfeldern eines größeren Kriegsschiffs wirklich gefährlich werden zu können. Aber diese sparten die eigenen Kanonen aus, damit diese ungestört Feuern konnten, ohne die eigenen Schilde zu treffen. Damit waren sie offen für feindliche Salven.

\par

\WR{Staffel Charlie soll ihre Nachbrenner aufheizen, verdammt noch mal!}, rief Fiscale zu ihrem Einsatzleiter hinunter. \WR{Falls sie es nicht schaffe, soll Staffel rot heranrücken und alles abfangen, was durchkommt.}

\par

Petrarca gab die Befehle sofort an die Abfangjäger durch. Wallander, der sonst die gesamte Kommunikation überwachte, hatte ihm längst eine Standleitung zu den einzelnen Jägerstaffeln eingerichtet, sodass er sich selbst eher darum kümmern konnte, Kontakt zu den Jägern der \EN{\EN{Artiglio} de Leone} herzustellen und diese ins Kampfgeschehen einzubeziehen.

\par

Ein Blick zu ihrer Linken zeigte Captain Fiscale, wie der Zerstörer \EN{Vulkan} langsam in eine Flankenposition kam und seinen Hauptgeschützturm ausrichtete. Das Schiff war klein, aber man konnte ihm bereits an seinen fast überproportional wirkenden Kanonentürmen ansehen, welche Feuerkraft es entfachen konnte. Allerdings blieb dadurch kaum Platz auf seiner Oberfläche für kleineren Geschütze, weswegen die \EN{Vulkan}, wie alle Schiffe ihrer Art, sehr verwundbar für Jägerangriffe waren.

\par

Fiscale ließ sich die taktische Situation auf ihrem Handcomputer anzeigen. Allein das nähere Umfeld um die \EN{Regenvogel} war bereits ein unüberschaubares Chaos. Mehrere dutzend Jäger rasten in alle Himmelsrichtungen umher. Die Kapitänin erkannte kaum ein Muster hinter den Bewegungen, doch realisierte sofort, dass es nicht genügend Maschinen gab, um auch nur in diesem kleinen Bereich Raumüberlegenheit zu erreichen.

\par

\WR{Mister Wallander}, sagte sie energisch, \WR{unsere Zerstörer sollen noch näher herankommen. Wir fliegen in möglichst enger Formation. Auf diese Weise müssen unsere Jäger einen viel kleineren Bereich abdecken.}

\par

Commander Samad war unterdessen an sie herangetreten und mahnte: \WR{Wenn eines unserer Begleitschiffe getroffen wird, sind wir viel zu nah dran, um rechtzeitig…}

\par

\WR{Ich weiß!}, schnitt ihn die Kommandantin gereizt ab. \WR{Darum darf das eben nicht passieren.} Der erste Offizier wollte bereits einen weiteren Einwand bringen, doch Fiscale kam ihm zuvor. \WR{Stellen Sie Kontakt zur \EN{Heinlein} her. Während wir diese Shutek-Flotille bekämpfen, können sie sich vielleicht bis nach Kreuzpunkt Primus durchschleichen. Sie sollen einen Kurs durch den flachen Hyperraum errechnen. Aber noch nicht anfliegen! Das würde zu viel Aufmerksamkeit erregen. Sobald wir diese drei Schiffe in einen Nahkampf verwickelt haben, schicken wir ihnen ein paar Jäger mit, die bei der Landung helfen.}

\par

Samad griff Fiscale am Oberarm, als diese schon weiterlaufen wollte. \WR{Natalia, das ist zu gefährlich. Wir haben nicht genügend Jäger, um an zwei Schauplätzen zu kämpfen. Wir sollten warten, bis die \EN{Crossguard} oder die \EN{Artiglio} uns zur Seite stehen können.}

\par

\WR{Dann ist es zu spät!}, donnerte Fiscale. \WR{Die Shutek haben jetzt schon ihre Bodentruppen zum Einsatz gebracht. Wenn sie es schaffen, unten genügend Silos lahmzulegen, dann braucht ihre Flotte nur noch in den Orbit einzuschwenken, und man kann Kreuzpunkt nur noch anhand der Größe von Pollux unterscheiden. Wir müssen die Shutek aufhalte. Und zwar jetzt!}

\par

Commander Samad ließ sie widerwillig los und sagte weiter nichts mehr. Trotz der vielen Jahre, die Fiscale  bereits mit ihrem ersten Offizier gedient hatte, wusste sie nicht, ob sie ihn wirklich überzeugt hatte und er auf ihrer Seite stand.

\par

Kurzerhand ließ sie ihn stehen und begab sich eine Treppe in den unteren Brückenbereich zu Lieutenant Schwarzschilds Station hinab. Mehrfach musste sie sich am Handlauf festhalten, da das Schiff unter dem permanenten Beschuss wankte. Dabei verfluchte sie die Ingenieure und Planer der Navy, die bekannt dafür waren, dass sie ihre Schiffe mehr nach ästhetischen Gesichtspunkten, als nach Funktionalität gestalteten. Doch sie gestand ihnen zu, dass wohl niemand mit einer Invasion durch die Shutek gerechnet hatte.

\par

Unten angekommen, beugte sie sich über Elshe Schwarzschilds Schulter, die gerade durch die Binokulare ihrer Station sah. Vermutlich analysierte sie die Echtzeitbilder der feindlichen Schlachtschiffe. Dabei durch zwei Zylinder zu sehen, half dabei, sich auf die angezeigten Fotos anstatt auf irgendetwas anderes zu konzentrieren. Insbesondere in einer Situation, in der praktisch im Sekundentakt ein metallisches Kreischen durch das Schiff ging. Jedes mal, wenn ein Geschoss auf die Schutzfelder prallte.

\par

\WR{Miss Schwarzschild, die Zerstörer sind jeden Augenblick in Feuerposition. Können Sie mir ein Ziel geben?}, fragte Fiscale.

\par

Ohne von ihren Binokularen aufzuschauen, antwortete Schwarzschild: \WR{Die \EN{Aeshma} scheint Gefechtsschäden erlitten zu haben. Aber die \EN{Amon} ist ungünstig positioniert. Wenn wir sie treffen, dann zwingt das ihre beiden Begleiter vielleicht zum ausweichen.}

\par

Die Kommandantin rief zum Kommunikationschef hinauf: \WR{Alle Zerstörer sollen mit allen Geschützen auf die \EN{Amon} zielen. Feuer frei!}
