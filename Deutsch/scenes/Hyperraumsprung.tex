Während dessen hatte sich auf der Brücke hektische Betriebsamkeit ausgebreitet. Zwar lief niemand mehr durch die Gegend, da jeder schon angeschnallt in seinem Sessel saß aber dennoch plapperte alles wild durcheinander. Der Einsatzleitoffizier forderte Zwischenberichte ein, die vier Kommunikationsoffiziere gaben Anweisungen an die Besatzung oder die beiden Begleitkorvetten weiter und der Steuermann meldete jede Minute, wie lange es noch bis zum Sprung dauern würde.

\par

Captain Fiscale wirkte in diesem Chaos wie ein Fels in der Brandung. Sie ließ sich vom Stress der anderen nicht im geringsten anstecken. Blickte hin und wieder auf ihre Taschenuhr oder warf Commander Samad einen freundlichen Blick zu.

\par

\WR{Noch drei Minuten bis zum Sprung, Madam}, meldete der Steuermann, ohne seinen Blick von der Konsole vor ihm zu lassen.

\par

Der Navigator lies seinen drehbaren Stuhl in Captain Fiscales Richtung rotieren, bevor er seinen Bericht abgab. \WR{Der Kurs ist so genau berechnet worden, wie es möglich war. Es gibt da drei bis vierdin Variablen. Aber der Computer hat ihren Einfluss durch Schätzungen kompensiert und ich glaube, wir können den Sprung wagen.}

\par

\WR{Glauben ist nicht wissen, Lieutenant}, warf ihm Commander Samad sofort streng zu. \WR{Der Ausdruck \Wr{glauben} sollte aus dem Wortschatz der Starforce gestrichen werden.}

\par

In der matten Brückenbeleuchtung, die hauptsächlich von den Konsolen und den Sternen kam, war nur ansatzweise zu erkennen, wie der Navigator mit den Augen rollte.

\par

\WR{Sir, ich sage nur wie es ist. Wenn ich behaupten würde, ich wüsste zu din Prodin, dass der Sprung vollkommen sicher ist, dann wäre ich ein Lügner. Und wenn man in ein nicht erforschtes System springt, dessen Sprungroute bislang nur ein mal verwendet wurde, dann bleibt eben ein gewisses Risiko.}

\par

Commander Samad blieb still. Einige Brückenoffiziere waren der Diskussion aufmerksam gefolgt. Noch vor ein paar hundert Jahren wäre eine solche Auseinandersetzung innerhalb der Armee undenkbar gewesen. Aber zur Zeit der Union wurde jeder Offizier absichtlich dafür ausgebildet, den höherrangigen Offizieren Kontra zu geben, falls dies sinnvoll oder nötig erschien, was allerdings nicht bedeutete, dass den Betroffenen das auch gefiel. Daher waren Diskussionen mit Vorgesetzten immer eine Gradwanderung.

\par

\WR{Ist die Mannschaft zum Sprung bereit?}, fragte Captain Fiscale den Einsatzleiter Maas Petrarca.

\par

Der Lieutenant Commander warf einen Blick auf einen seinen Monitore. Eine Anzeige verriet ihm, dass dreihundert einundfünfzig Alkoven und Sicherheitssessel in Betrieb und besetzt waren~-- so viele, wie es Menschen an Bord gab.

\par

\WR{Ja, Madam}, antwortete der Einsatzleiter knapp.

\par

Captain Fiscale nickte zufrieden und klappte ihre Taschenuhr zusammen. Commander Samad musste sich ein ausgedehntes Gähnen verkneifen. Er war schon seit über einem Tag auf den Beinen. Anders als der größere Rest der Mannschaft hatte er keine Zeit gehabt, sich ein paar Stunden auszuruhen. Die ganze Zeit über hatte er so ziemlich jede Informationsquelle über das Arktur-System gewälzt. Außerdem hatte er sich alle Daten eingeprägt, die ihm über Schiffe der \EN{Ereignishorizont}-Klasse zur Verfügung gestanden hatten. Wenn die \EN{Regenvogel} angekommen wäre, würde es seine Hauptaufgabe sein zu versuchen an möglichst alle Eventualitäten zu denken und Captain Fiscale mit Optionen zu versorgen.

\par

\WR{Stellen Sie mich zur Mannschaft durch, Mister Wallander}, bat Fiscale Fiscale, mit festem Blick zum Kommunikationsoffizier.

\par

Der blauhaarige Mann gab die Anweisung an einen seiner drei Mitarbeiter mit nickte dann zur Bestätigung. Jeder Lautsprecher des Schiffes übertrug nun die Stimme vom Mikrofon der Kapitänin.

\par

\WR{Achtung, an die gesamte Besatzng. Wir werden in einer Minute springen. Behalten Sie die Nerven und bleiben Sie ganz ruhig, dann kann nichts schief gehen.}

\par

Nun, da die Crew informiert war, befahl Commander Samad an den Steuermann gewandt: \WR{Mister Senkethi, beginnen Sie mit dem Sprung und zählen Sie herunter.}

\par

\WR{Aye, Sir}, antwortete der junge Offizier knapp und bediente seine Instrumente.

\par

Captain Fiscale warf ihren beiden Offizieren einen zuversichtlichen Blick zu und sah dann wieder nach vorne. Durch die vollverglaste Brückenfront war zu erkennen, wie sich die Sterne kaum merklich zu verschieben schienen, als die \EN{Regenvogel} ihren Kurs geringfügig korrigierte. Ein weit entfernter Nebel, der trotz der enormen Distanz noch riesig aussah, hüllte die ganze Umgebung in ein blutrotes Licht. Schon als sich die \EN{Regenvogel} vor einigen Stunden in seine Richtung gedreht hatte, hatte der ein oder andere Offizier auf der Brücke zusätzliche Lampen eingeschaltet um das rötliche Glühen zu übertünchen.

\par

Von einer Hyperraumleitboje war nirgends etwas zu sehen. Die Route nach Arktur war vor kurzem erst entdeckt worden und daher war noch keine installiert worden. Dadurch wurde der Sprung nicht gerade leichter zu berechnen, denn eine Boje mit Gegenstück auf der anderen Seite, hätten praktisch die ganze Arbeit übernommen.

\par

\WR{Zwanzig Sekunden bis zum Sprung, Madam}, meldete der Steuermann, dessen Haarpracht Commander Samad mehr und mehr an einen schwarzen Pilz erinnerte.

\par

Captain Fiscale machte einen völlig entspannten Eindruck, während die meisten anderen auf der Brücke sich entweder an die Lehnen ihrer Stühle klammerten oder ihre Gurte immer fester und fester anzogen.

\par

\WR{Zehn Sekunden bis zum Sprung}, meldete der Steuermann, nun selbst etwas aufgeregt klingend.

\par

Ein schriller Alarm erklang. Commander Samad sah sich erschrocken um, nur um sich daran zu erinnern, dass es sich um den Alarmton vor einem ungelenkten Hyperraumübergang handelte. Langsam wurde auch er mit seiner ganzen Erfahrung zunehmend unruhig. Ein so weiter Sprung ohne Leitboje und nur mit dem Bericht eines einzigen anderen Schiffes im Navigationscomputer war eine der besten Möglichkeiten, für immer zu verschwinden.

\par

Abdel Samad hatte dem Blick des Todes schon mehrere male standgehalten. Die Angst davor, zu sterben, hatte er nie abgelegt. Doch das Schlimmste, was er sich nach seinem Ableben vorstellen konnte, war, dass die verbreitetste atheistische Auffassung zutraf und sein Bewusstsein schlicht endete. Doch noch viel schauerlicher klangen für ihn die Hypothesen über ewiges Gefangensein in den Untiefen des Hyperraums.

\par

Es hatte bereits Unfälle gegeben. Mehr als ein Schiff war durch die Lichtmauer gesprungen, ohne dass jemals wieder jemand etwas von ihm oder seiner Besatzung gehört hatte. Und selbst der Seemansgarn, der sich um diese Vorfälle rankte war nichts verglichen mit den Augenzeugenberichten von Schiffe, die nur ein Stückweit vom Kurs abgekommen waren und dann eine viel längere Zeit im tiefen Hyperraum verbracht hatten, als es geplant gewesen war.

\par

Das einer der Brückenoffiziere gerade jetzt etwas von der \EN{Kristallsee} flüsterte, beunruhigte ihn nicht im geringsten.
\ortswechsel
Morten Wittwer fing langsam an zu bereuen, nicht wirklich einen Eimer vor sich gestellt zu haben, denn sein Magen begann sich zu verkrampfen. Es war ruhig geworden im ersten von drei Sicherheitsräumen der Piloten. Im Geiste hatte er jede Sekunde mitgezählt, seit Captain Fiscale gemeldet hatte, dass es noch eine Minute bis zum Sprung sei. Und er nun war er bei fünf Sekunden angekommen. Gleich würde es losgehen. Sein erster Sprung auf einem Kriegsschiff. Jeder, der das bereits hinter sich gebracht hatte, sagte, dass es eine unvergessliche Erfahrung sei.

\par

Morten kam bei Null an. Gleich würde es passieren. Und wie sich herausgestellte, hatte er genau gezählt. Die \EN{Regenvogel} beschleunigte auf Sprunggeschwindigkeit. Er spürte es deutlich, als starke Kräfte ihn nach Rechts pressten. Irgendwie konnte sich Morten kaum vorstellen, wie es die Gurte schaffen sollten, ihn bei dieser Beschleunigung festzuhalten. Ohne die Trägheitsabsorber des Schiffes wäre jedes Genick auf diesem Schiff kurz nach Anfang der Beschleunigungsphase gebrochen.

\par

Ein seltsames Grollen stellte sich ein. Der Hyperraumübergang begann. Das Geräusch war so seltsam und fremdartig, dass Morten glaubt, kein Synthesizer der Welt könnte es künstlich nachstellen. Es klang irgendwie tief. Ein bisschen nach einer Bassgitarre. Aber das beschrieb es nicht richtig. Auf eine merkwürdige Weise schien es gar nicht wirklich da zu sein. Es kam ihm so vor, als würde es sich allein in seinem Kopf abspielen. Aber dann hörte er es wieder so deutlich als würde es direkt vor ihm entstehen.

\par

Plötzlich schienen Mortens Augen ihn zu täuschen. Das Bild des Raumes, den er gerade sah, verzog und verformte sich. Alles schien auf einmal enorm weit weg zu sein, als würde er auf eine bemalte Leinwand oder einen Monitor schauen. Grelle Lichterscheinungen durchzuckten den Raum. Der weiße, nebelhafte Schimmer hinterließ etwas, das wie ein Kondensstreifen aussah. Alles was Morten sehen konnte schien gänzlich zu verschwinden.

\par

Und dann kam es ihm vor, als würde sein Körper in tausend Stücke zerrissen werden. Alles in ihm drängte von ihm weg, als würde er in einer gigantischen Zentrifuge stecken. Morten wusste nicht einmal ob er die Augen noch offen oder schon geschlossen hatte. Das Gefühl für seinen Körper war verschwunden. Er konnte sich nicht mehr bewegen oder auch nur einen Finger krümmen. In seinem Kopf entstand etwas, dass ihm wie eine tausendfach heftigere Variante von Schwindel vorkam. Als würde sein Gehirn in einem Strudel absaugt werden. Dieser Zustand schien ewig anzudauern und dabei aber immer schlimmer und schlimmer zu werden. Doch dann hörte er abrupt auf.

\par

Es war ein seltsames Gefühl. Morten glaubt, nichts mehr sehen zu können. Er hatte auch das Gefühl gar keinen Körper mehr zu haben. Als würde sein Geist frei irgendwo umher schweben. Allerdings an irgend einem Ort, an dem es unheimlich beengt war. Überall schienen Grenzen zu sein. Und eigentlich, so kam es Morten vor, gab es überhaupt keinen Raum mehr. Er fühlte sich, als seien er und der ganze Rest der Welt auf einen unendlich kleinen, einsamen Punkt zusammengeschrumpft.

\par

Morten bemühte sich, irgendwelche Eindrücke wahrzunehmen. Tatsachlich glaubte er, verzerrte Bilder zu sehen und seltsame Geräusche zu hören. Aber die schienen eher von seinem eigenen Geist zu stammen. Es war irgendwie ähnlich, wie das Gefühl, wenn man die Augen schloss, nur das es hinter den Liedern keine Welt mehr zu geben schien.

\par

Es war Morten, als würde plötzlich alles hell erleuchtet werden. Als würde mit einem mal wieder Raum bestehen. Sein erster Gedanke war, dass der Sprung vorbei und er wieder in seinem Körper zurück war. Aber das konnte nicht sein. So sah kein Innenraum eines Trägers aus.

\par

Tatsächlich schien es Morten als sei der Raum um ihn herum nicht mehr unendlich klein sondern vielmehr endlos groß. Und helles Licht schien von überall her zu kommen. Es war auf eine seltsame Weise wirklich schön. Eine ganz andere Welt.

\par

Dann kam Morten ein anderer, weniger erfreulicher Gedanke. Helles Licht und Endlosigkeit. War er etwa tot? Hatte er den Sprung nicht überlebt und befand sich nun in dem was gemeinhin als der Himmel bezeichnet wurde? So konnte man sich den Himmel tatsächlich vorstellen. Die uralten Vorstellungen von einem Platz hoch über den Wolken der unendlich groß war, trafen auf den Ort, an dem sich Morten gerade glaubte, recht gut zu.

\par

Aber das konnte es nicht sein. Morten war nicht gläubig. Er lehnte die Vorstellung eines Gottes zwar nicht strikt ab, fühlte sich aber auch keiner zugetan. Genau genommen, so dachte er sich, hatte er niemals wirklich darüber nachgedacht.

\par

Und nicht nur das wurde ihm schlagartig klar. Es war ihm, als würden ihm alle Erinnerungen seines Lebens offen stehen und nicht mehr vernebelt, verfälscht oder vergessen sein. Spielend einfach erinnerte er sich an seine Kindheit auf Corna. An jeden Tag in der Schule und mit seinen Freunden. An jedes Abenteuer in den spärlich gesäten Wäldern seines Heimatplaneten und an jeden Kopfsprung in das viel zu warme Wasser des Schwimmbades. All diese Erinnerungen schienen jetzt sonnenklar und ganz einfach zu erreichen.

\par

Morten versuchte drüber nachzudenken, wie lange er sich nun schon in diesem Zustand befand. Er hatte jegliches Zeitgefühl verloren. Es konnte Stunden sein oder nur der Bruchteil einer Sekunde.

\par

Und dann veränderte sich wieder etwas. Der Raum wurde wieder kleiner und das Licht dunkler. Aber nicht so sehr wie zum Beginn. Er Morten hatte immer noch das Gefühl, nicht eingeengt zu sein und sich irgendwie bewegen zu können. Aber er fühlte sich unwohl. Schmerzen durchzogen ihn und er fühlte sich fast, als würde er verbrennen. Von so etwas hatte ihm noch niemand berichtet. Jeder hatte ihm gesagt, dass ein Sprung unangenehm sei aber von starken Schmerzen hatte er noch nie gehört.

\par

Die Erinnerungen an sein frühes Leben verblassten allmählich und schienen ganz zu verschwinden. Und kurz darauf hatte Morten Mühe damit, sich überhaupt zu erinnern, wer er war.

\par

Bilder zuckten vor ihm vorbei. Genaues konnte er nicht erkennen aber eines war ihm sofort klar. Es war nichts schönes, das er ansah. Eher etwas furchtbares und schreckliches. Ein bedrohlicher, dumpfer Ton hüllte ihn ein. Er wurde immer lauter und lauter und Morten glaubte, dass allein der Klang ihn jeden Moment zerquetschen würde. Jedes mal, wenn er dachte, der Ton könnte nicht noch lauter werden, hämmerte er noch stärker und schneller.

\par

Doch durch das Dröhnen hindurch vernahm Morten eindeutig Schreien. Das erste nicht Abstrakte, was er in dieser furchtbaren Umgebung wahrnahm. Schließlich erkannte er, dass es nicht nur ein einziger Schrei war sondern das angsterfüllte Kreischen und Brüllen von vielen verschiedenen Stimmen. So etwas hatte er noch nie gehört. In den Schreien hallte so viel Verzweiflung und Panik mit, dass Morten nicht glaubte, es sich noch länger anhören zu können.

\par

Langsam nahmen auch die Bilder Form an. Es war wie in einem Alptraum. Morten konnte sich selbst nicht beschreiben was er da sah aber es war eindeutig das schlimmste, das ihm jemals vor die Augen gekommen war. Bilder, die von Schrecken zeugten, an die er sonst nicht einmal zu denken wagte. All die furchtbaren und widernatürlichen Dinge, gegen die sich sein Verstand normalerweise wehrte.

\par

Das Dröhnen und Schreien wurde immer heftiger und immer mehr Bilder strömten auf Morten ein. Er hatte das Gefühl, jeden Moment sein Leben zu verlieren. Es kam ihm so vor, als wäre bald alles vorbei und er würde einfach aufhören zu existieren.

\par

Dann hörte er noch etwas anderes. Es war eindeutig eine Stimme. Aber was sie sagte hörte sich nicht wie gesprochene Worte an. Morten glaubt eher, dass die Worte direkt in seinen Geist drangen.

\par

Die Stimme sagte deutlich und unmissverständlich: \WR{Sie ist es!}

\par

Und dann dachte Morten an gar nichts mehr. Alles wurde dunkel und er hatte das Gefühl, zu verschwinden.