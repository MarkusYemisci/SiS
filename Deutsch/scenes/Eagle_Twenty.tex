\WR{Meldung von der \EN{Heinlein}}, berichtete Lieutenant Wallander. \WR{Ihr Navigator hat einen Kurs durch den Flachen Hyperraum berechnet. Eintrittspunkt liegt bei eins elf neun zu vier vier sieben zu ein Zwölftel Pi.}

\par

Captain Fiscale rannte nunmehr seit einiger Zeit von einer Station der Brücke zur anderen. Nun stand sie vor Maas Petrarcaas Übersichtshologramm und versucht sich ein Bild über die Lage zu machen. Gerade waren fünf gelbe Blickpunkte vom Radar verschwunden. Die letzten Marschflugkörper waren damit ausgeschaltet.

\par

\WR{Jetzt sind wir dran}, sagte die Kommandantin in Gedanken versunken und mehr zu sich selbst. Doch tatsächlich dauerte es nur noch wenige Augenblicke, bis die Zerstörer ihre eigenen Anti-Schiff-Raketen abfeuerten. Die Marschflugkörper drangen scheinbar langsam aber unter gewaltiger Rauchentwicklungen aus den Startrampen an den Flügeln der keilförmigen Schiffe.

\par

Auf dem Übersichtsradar waren nun zwölf türkisfarbene Blickpunkte zu sehen, die allesamt auf ein gemeinsames Ziel zusteuerten.

\par

Die \EN{Amon}, wie der Computer der \EN{Regenvogel} das größte der drei feindlichen Schiffe automatisch betitelt hatte, reagierte noch nicht auf den Abschuss. Der Kreuzer~-- zumindest wurde ausgehend von der Bewaffnung, Panzerung und Manövrierbarkeit darauf geschlossen, dass es einer war~-- würde bald alle seine Geschütze auf die anfliegenden Marschflugkörper ausrichten.

\par

Auf Commander Samads Gesicht zeichnete sich eine Spannung ab, die Fiscale von ihrem ersten Offizier nicht kannte, als er den Kurs der Raketen verfolgte, als sie auf ihr Ziel zuschossen. Langsam kam Bewegung in die feindlichen Reihen. Einige der Jäger zogen sich aus dem äußeren Zirkel um die \EN{Regenvogel} zurück und rasten den Marschflugkörpern entgegen. Diese begannen mit Ausweichmanövern, als sie sowohl von von allen Seiten aus unter Beschuss gerieten. Und schon vergingen die ersten in hellen Lichtblitzen.

\par

\WR{Starkes Abwehfeuer}, meldete Lieutenant Commander Petrarcaa unnötigerweise. \WR{Neun Fische sind noch im Wasser.} Trotz ihrer Anspannung runzelte Captain Fiscale die Stirn über diesen anochronistischen Begriff.

\par

Fast ein dutzend feindlicher Jäger stürzten sich nun auf die abgefeuerten Geschosse und beharkten sie mit ihren Bordkanonen. Die Navigationssysteme der Marschflugkörper hatten einige Zeit gehabt, ein umfangreiches Profil des feindlichen Abwehrfeuers zu berechnen. So führen augenblicklich kleinere Kurskorrekturen aus, um die Strahlenentladungen zu gehen.

\par

Fiscales Augen klebten förmlich am Hologramm und an den Flugbahnen der Raketen, die sich in den äußeren Zirkel der \EN{Amon} vorarbeiteten. Nun eröffneten auch die Zerstörer das Feuer auf die Marschflugkörper. Einer nach dem anderen wurde getroffen, lange bevor sie ihr Ziel erreichen konnten.

\par

\WR{Verdammt!}, rief Einsatzleiter Petrarca. \WR{Nur noch fünf übrig! Ihr Geschütze arbeiten unglaublich effizient.} Ein weiterer Lichtblitz folgte dieser Aussage.

\par

Doch auf dem Hologram war nun zu erknnen, wie vier der verbleibenden Raketen den inneren Zirkel erreichten und die \EN{Amon} ein Ausweichmanöver wagte. Dadurch wurde Steuerbordflanke des Schiffes entblöst und die Geschosse rasten mit voller Geschwindigkeit in die Schutzfelder.

\par

Selbst durch das Aussichtsfenster der Brücke war die Detonation zu sehen. Ein orangefarbener Schein erstrahlte und hielt sich fast eine Minute lang.

\par

Beifall und kurze Jubelrufe hallten durch die Brücke. Selbst Captain Fiscale konnte sich eine kleine Triumphgeste nicht verkneifen. \WR{Wie ist der Zustand des Ziels!}, rief sie an Elshe Schwarzschild gerichtet und hängte sofort ein \WR{Lieutenant!} an, als diese nicht augenblicklich antwortete.

\par

\WR{Immer noch in der Luft}, meldete diese schließlich und schien sich dabei kaum selbst glauben zu können. \WR{Alle vier Geschosse kamen durch, und ihre Schutzfelder sind definitiv zerstört. Aber die Abtaster registrieren nur oberflächliche Schäden. Es könnte natürlich sein, dass die Waffen mehr Schaden angerichtet haben. Aber wenn, dann registrieren ihn unsere Sensoren nicht.}

\par

\WR{Verdammt!}, entfuhr es Fiscale.

\par

Doch Commander Samad schied die Situation etwas positiver zu sehen. Er deutete auf die feindliche Formation. \WR{Der Ausweichkurs des Kreuzers hat auch einen ihrer Zersöterer zu einer Bewegung gezwungen. Ihre Aufstellung ist durcheinander. Außerdem haben sie etliche Jäger aus unserem mittleren und weiten Zirkel abgezogen. Das gibt uns etwas Luft.}

\par

Fiscale rannte zu Lieutenant Wallander. \WR{Befehl an die Zerstörer: Antrieb auf dreiviertel. Dasselbe gilt für uns, Steuermann! Wir nutzen die Chance und gehen mit diesen Kerlen mal auf Tuchfühlung.}

\par

\WR{Madam, ich bin nicht sicher, ob wir einen Nahkampf mit den Shutek überstehen. Wir wissen zu wenig über ihre Bewaffnung.} Abdel Samad war für diese Warnung nicht wie üblich unauffällig an Fiscale herangetreten, sondern hatte sie praktisch durch die halbe Brücke gerufen.

\par

Die Kommandantin ignorierte ihn vorläufig. \WR{Mister Petrarca. Henningtons und Farleys Staffeln sollen Kurs auf die Basisschiffe nehmen. Die \EN{Heinlein} kann sie durch den flachen Hyperraum mitnehmen. Sie sollen bei der Landeoperation Luftunterstützung geben.}

\par

Der Einsatzleiter wartete einen Moment ab, bevor er den Befehl bestätigte. Commander Samad äußerte hingegen seine Zweifel, diesmal aber wieder aus gewohnter Nähe. \WR{Captain, ich halte das für keine gut Idee. Erstens haben Rot und Blau schon die Hälfte an Munition und Treibstoff verbraucht. Und zweitens brauchen wir sie hier, um unsere Raumüberlegenheit zu halten.}

\par

Natalia Fiscale sah zunächst auf den Boden, bevor sie einen festen Augenkontakt mit ihrem ersten Offizier einging. \WR{Abdel, ich hasse es, das sagen zu müssen, aber es ist tausendfach wichtiger, dass diese Landetruppen den Boden erreichen, als das dieses Schiff hier in einem Stück bleibt. Legat Gajjar kann diesen Ansturm nicht alleine aufhalten. Ich kenne die stärke der Bodentruppen auf Kreuzpunkt. Da unten hat es im Sommer durchschnittlich ein dutzend Grad Celsius. Wohnraum ist knapp und niemand baut eine Kaserne, wenn er den Platz auch als Wohnraum nutzen kann.}

\par

Abdel Samad rieb sich angetrengt die Stirn. Fiscale spürte deutlich, wie gerne er widersprochen hatte. Aber schließlich nickte er nur und rief: \WR{Achtung! Auf Einschläge vorbereiten. Es gibt gleich einen heftigen Schlagabtausch und wir bekommen garantiert ein paar Fäuste ins Gesicht.}
