Die Mittagssonne stand hoch über Wellington als Techniker von Spectare Vision in den Ecken des präsidialen Büros Sonden für holographische Übertragungen aufstellten. Mindestens drei wurden für die Triangulierung und Bilderzeugung gebraucht aber es war ratsam weitere aufzustellen, um die Bildqualität des Hologramms zu verbessern oder um den Ausfall einer Sonde auszugleichen.

\par

Als die Mitarbeiter von SV insgesamt sechs Sonden aufgestellt hatten, brachten sie einige akustischen Sensoren an den Wänden an. Präsident Otis nahm sich vor, zu überprüfen ob auch alle dieser Rekorder wieder entfernt werden würden. Diese Geräte waren recht klein, hatten eine hohe Übertragungsreichweite und waren schon das ein oder andere mal in Büros von Regierungsmitarbeitern \Wr{vergessen} worden.

\par

Nur eine Mitarbeiterin des Public Broadcasting Service blieb scheinbar untätig. Cilea Andropolus, eine der geschätztesten Journalistinnen der Union, sah schon seit einiger Zeit konzentriert in ihr Buch. Der rote Einband bildete einen krassen Kontrast zu ihrer eher farblosen Kleidung. Sie würde im Gespräch der Politikstudenten mit dem Präsidenten eine moderierende Rolle einnehmen. Das öffentliche Interesse an der Politik war in den letzten Jahren nach einem Einbruch wieder angestiegen und so genossen DDV-Übertragungen mit politischem Inhalt große Beliebtheit. Sie wurden, wie die Fußballweltmeisterschaft über die gesamte Union ausgestrahlt. Jeder Nullzonentranciever im von Menschen besiedelten Raum auf der richtigen Frequenz wäre in der Lage, die Ausstrahlung augenblicklich zu empfangen.

\par

Die Studenten und ihre beiden Professorinnen hatten bereits Platz genommen. Henry Otis bemühte sich zu lächeln, als er einen Blick auf die wild zusammengewürfelte Klasse warf. Es waren Frauen und Männer aus den unterschiedlichsten Kulturkreisen und mit verschiedensten Hintergründen vertreten. Soviel der Präsident wusste, besuchten sie eine Universität auf Wega Primus, die einen recht guten Ruf besaß.

\par

Einige der Studenten wirkten ein wenig nervös. Andere schienen völlig ruhig und gelassen zu sein. So unterschiedlich sie auch waren, eines hatten sie aber alle gemeinsam. Ihre Kleidung war das absolute Gegenteil von konservativ. Keiner trug etwas das auch nur entfernt nach einem Gehrock aussah. Das Spektrum reichte von Pullovern über T-Shirts bis hin zu Kleidern, die wie aus einem Gemälde des siebzehnten Jahrhunderts entliehen schienen.

\par

Schließlich erhob sich Andropolus und ging zum Leiter der Übertragungsmannschaft. Nachdem sie sich einen Moment lang leise mit ihm unterhalten hatte, erklärte sie mit einem professionellen Lächeln auf den Lippen: \WR{Die Technik ist bereit, Herr Präsident. Wir können beginnen, sobald sie fertig sind.}

\par

Henry Otis nickte und versuchte ein ähnlich unverbindliches Lächeln aufzusetzen wie die Reporterin. Er fragte sich, ob er vielleicht noch einmal zu den Damen und Herren der Maske hätte begeben sollen. Aber die holographische Übertragung bei SV war so detailreich, dass jedes bisschen Schminke zu erkennen sein würde. So blieb der Präsident lieber wie er war.

\par

\WR{Wir können beginnen}, sagte er kurz darauf.

\par

Andropolus nickte dem Leiter der Übertragung zu. Dieser ging zu einen portablen Terminal und machte einige Eingaben. Der Mann wirkte konzentriert und angespannt. Präsident Otis beneidete ihn in diesem Moment. Er fühlte sich nämlich genauso, konnte sich aber nicht außerhalb der Aufnahmsphäre zurückziehen. Und ihm war klar: selbst zu Zeiten der Union, in denen Darbietung nicht mehr der wichtigste Faktor in der Politik war, würde es nicht gut aussehen, wenn sich der Präsident von einigen Jugendlichen nervös machen ließ.

\par

\WR{Wir beginnen mit der Übertragung in fünf Sekunden}, berichtete der Übertragungsleiter und zeigte mit seinen Fingern einen Countdown an.

\par

Präsident Otis richtete nun seinen Blick auf die Klasse und versuchte unverbindlich zu wirken. Cilea Andropolus nahm einen Platz zwischen der Klasse und dem Schreibtisch des Präsidenten ein. Da das Hologramm der Übertragung von allen Seiten aus betrachtet werden konnte, hatte die Moderatorin keine Möglichkeit, den Zuschauer direkt anzublicken. Deswegen richtete sie ihr Augenmerk immer abwechselnd auf Henry Otis und die Klasse.

\par

Als der Countdown des Übertragungsleiters beendet war, begann Cilea Andropolus mit ihrer Einleitung. \WR{Liebe Zuschauer, ich grüße Sie und heiße Sie herzlich willkommen. Danke, dass Sie wieder zum \Wr{inneren Zirkel} eingeschaltet haben, auch wenn Sie vielleicht wie die meisten noch ganz im Fußballfieber sind.} Die Reporterin lehnte sich locker an den Schreibtisch des Präsidenten, was ihr einen kurzen, nicht gänzlich erfreuten, Blick von Henry Otis einbrachte. \WR{Wir senden heute in Echtzeit aus dem Büro des Präsidenten, im Regierungsgebäude der Union hier in der Nähe von Wellington. Ich freue mich, ihnen dieses mal die Diskussionsstunde der fünften Klasse der Domitian Hochschule auf Wega Primus und Präsident Henry Otis präsentieren zu dürfen.}

\par

Andropolus wandte sich nun dem Präsidenten zu. \WR{Ich möchte mich hiermit schon einmal bei Herr Henry Otis bedanken, dass er sich bereiterklärt hat, diese Diskussionsrunde im Rahmen meiner Sendung stattfinden zu lassen.}

\par

\WR{Gern geschehen}, gab der Präsident zurück. Von seiner Anspannung war nicht das geringste zu hören. \WR{Ich freue mich, dass die Bürger der Union ein so großes Interesse an politischen Belangen entwickelt haben, die uns schließlich alle betreffen, und stehe natürlich auch öffentlich gerne Rede und Antwort.}

\par

Die Reporterin richtete nun ihren Blick der Klasse entgegen. \WR{Auch bei ihnen möchte ich mich für ihr Kommen bedanken. Es erfordert Courage eine solche Unterredung in der Öffentlichkeit zu tätigen.}

\par

Die beiden Lehrkräfte wechselten fragende Blicke, nicht wissend wer nun eine Antwort geben sollte. Ein junger Student, groß mit krausen Haaren und einem Gesicht auf dem ein fordernder Ausdruck geschrieben stand, ergriff diese Gelegenheit beim Schopf und erhob sich. \WR{Meine Klasse und ich sind gerne gekommen. Aber ich denke, Courage werden nicht nur diejenigen aufbringen müssen, die noch keine gefestigte Position in der Politik innehaben.}

\par

Andropolus nickte bedeutungsvoll. Henry Otis konnte seine Verwirrung über die Aussage des Studenten nicht völlig verbergen. Die Moderatorin ergriff kurz darauf wieder das Wort: \WR{Bevor wir mit der Diskussionsrunde beginnen, würde ich gerne Frau Saahla fragen, wie ihr Schule auf die Idee einer derartigen Aktion kam.}

\par

Präsident Otis bemerkte beunruhigt, dass er ihren Namen gar nicht gekannt hatte, bevor Andropolus ihn genannt hatte. Er hoffte inständig, dass sie auch die andere Lehrerin noch mit Namen anreden würde, bevor er es müsste. Es konnte zwar niemand erwarten, dass er sich die Namen von dreißig Studenten merken konnte, die er noch nie zuvor gesehen hatte aber er bereute es mittlerweile, nicht zumindest die ihrer Professorinnen vor dem Gespräch in Erfahrung gebracht zu haben.

\par

Die harten orientalische Züge der Frau, die eigentlich viel zu jung wirkte, um an einer Universität zu lehren, passten gute zu ihrer tiefen und kraftvollen Stimme. \WR{Natürlich sind unsere Studenten sehr an der aktuellen Politik der Union interessiert. Ein Gespräch mit einem leitenden Mitglied der Regierung ist dabei eine einmalige Gelegenheit für die Schüler, einmal Kontakt zur realen Regierungsarbeit zu bekommen und sich nicht immer mit dem zufrieden geben zu müssen, was man in den Lehrbüchern lesen kann.}

\par

\WR{Ihre Klasse, Frau Saahla, war die erste die mit einer Echtzeitübertragung einverstanden war}, fuhr Celia Andropolus fort. \WR{In wie weit, glauben Sie, kommt das ihren Studenten zugute?}

\par

Der Student, der auch zuvor schon einmal das Wort ergriffen hatte, hob die Hand und sprach: \WR{Wenn Sie gestatten, würde ich die Frage gerne beantworten.}

\par

Die Lehrerin machte zunächst ein überraschtes Gesicht, nickte dann aber einverständig. Andropolus umlief die Stuhlreihung, die für die Gäste aufgestellt worden war. Jedem, der die Szenerie verfolgte, fiel sofort ihr stolzer Gang auf. \WR{Natürlich Herr Petersson. Die Frage betrifft, wenn man es sich genau überlegt, ja hauptsächlich Sie als Schüler.}

\par

Präsident Otis kam nicht umhin, kurz beide Augenbrauen hochzuziehen. Diese listige Journalistin kannte tatsächlich die Namen des Studenten. Und wahrscheinlich auch die der anderen. Otis befürchtete langsam, ihre Professionalität unterschätzt zu haben.

\par

\WR{In der Politik ist es wichtig, sich in der Öffentlichkeit präsentieren zu können}, erklärte der Student. Von Nervosität oder Anspannung war ihm nichts anzumerken. \WR{Wir möchten später auch einmal in diesem Gebiet tätig werden und deswegen glaube ich, eine Feuertaufe wie diese ist eine wirklich wertvolle Erfahrung.}

\par

Präsident Otis nickte und richtete seine Hand auf den Studenten. \WR{Da haben Sie völlig recht. Kurz nachdem ich graduiert hatte, musste ich für einen Redner einspringen, der ausgefallen war. Ich hatte noch niemals vor so vielen Menschen gesprochen. Es war ein wirklich prägendes Erlebnis.}

\par

Andropolus trat ein wenig in den Hintergrund. Sie faltete die Hände und suchte die Blicke der Anwesenden. \WR{Ich denke, wir steigen nun in die Diskussionsrunde ein. Die fünfte Klasse hat während ihres Studiums einige Themenpunkte herausgearbeitet, die sie nun gerne ansprechen möchte.}

\par

Mit einer über das Publikum schweifenden Handbewegung lud die Journalistin die Gäste ein, nun zu beginnen. Es verwunderte Otis nicht, dass es Petersson war, der aufstand um die erste Frage zu stellen. Aber es überraschte den Präsidenten wohl, dass er sie aus einem Buch der neusten Bauart ablas, wie das Logo auf dem Einband schnell verriet. Solche Modelle waren sehr kostspielig. \WR{Herr President, wir haben die aktuelle Außenpolitik der Union verfolgt und an einem Thema kommt man da kaum vorbei. Die Handelsbeziehungen mit den autonomen Welten. Empfinden Sie die Situation als gerecht oder denken Sie, dass die selbstständigen Systeme durch die Union in ihrer Entwicklung behindert werden?}

\par

Henry Otis räusperte sich zunächst. Mit einer Frage dieser Tragweite hatte er nicht so früh gerechnet. Zumal eine unglücklich gewählte Antwort schnell zu einem Problem für ihn werden könnte. Schließlich erwiderte er: \WR{\textit{Gerechtigkeit} ist ein Begriff, der nur sehr schwer zu fassen ist. Das weiß ich seit meiner Justizreform. Aber ich denke die Beziehungen zwischen der Union und den Welten des Sinistra-Sektors ist in so weit fair, dass beide Parteien bemüht sind und dies auch immer schon waren, gemeinsam Lösungen zu finden.}

\par

Peterssons Augenbrauen formten ein V. Otis hatte seinen Blick schon zahllose male gesehen. Es waren die Augen eines Mannes, der bereit war, sich auf einen verbalen Schlagabtausch einzulassen. Nach einem kurzen Blick in sein Buch ergriff er wieder das Wort. \WR{Nun, das war eigentlich nicht ganz das, was ich wissen wollte. Immer wieder werden Beschwerden in der Bevölkerung der autonomen Welten laut, die Union würde sie bewusst klein halten. Wie stehen Sie dazu?}

\par

\WR{Ich denke, dass kann ich guten Gewissens verneinen}, antwortete Henry Otis entschlossen. \WR{Die ungehinderte Entwicklung \textit{aller} Menschen ist ein großes Maxim der Union. Natürlich ist es für die Autonomen schwerer, neue Planeten zu erschließen oder eine stabile Wirtschaft zu etablieren, als es für uns ist. Aber ich glaube, diese Probleme waren den Regierungen der selbstständigen Welten bewusst, als sie ihre Unabhängigkeit erklärten.}

\par

Der junge Student schien auch mit dieser Antwort nicht wirklich zufrieden zu sein. \WR{Viele empfinden aber das Handelsembargo durchaus als Behinderung für die autonomen Welten. Ohne das Sagittae System als Zwischenablage zu verwenden, könnte sie überhaupt nicht mit uns handeln und wären vollkommen auf sich gestellt.}

\par

Otis suchte nun bewusst den Blickkontakt zu Petersson. Der junge Mann hielt dem problemlos stand. Er blickte den Präsidenten fordernd an.

\par

\WR{Ich denke}, entgegnete dieser, \WR{Sagittae Primus als \Wr{Zwischenablage}, wie Sie es nennen, zu benutzen ist ein absolut legitimes Mittel. Das Wirtschaftssystem der Union ist davon abhängig, dass unsere Währung nicht direkt an einer anderen gemessen werden kann. Nur so kann der Staat bestimmte Leistungen wie, Wohnungen, Nahrung oder Schulbildung kostenlos anbieten. Ich persönlich sehe die derzeitige Lösung als gute Möglichkeit. Und die Tatsache, dass sie schon seit Jahrzehnten funktioniert, spricht für sie.}

\par

Tom Petersson nickte und setzte sich wieder. Kurz darauf erhob sich eine junge Frau mit aufwendig geflochtenen Haaren und einer selbstgemachten Strickjacke um die nächste Frage zu stellen. Sie wirkte harlmos. Darum überraschte es Otis umso mehr, mit welcher Leichtigkeit sie ihre Frage vorbrachte. \WR{Herr Präsident, Sie hatten von einigen Jahren eine wirklich erfolgreiche Justizreform angeregt. Als Reaktion auf die Vorfälle in Zusammenhang mit der Verbannung von Schwerverbrechern auf VT drei drei vier, wenn ich mich nicht irre.}

\par

\WR{Gestatten Sie mir, Sie zu verbessern}, bat Henry Otis, als die Studentin eine Pause einlegte. \WR{Meine Justizreform war keine Reaktion auf die Tragödie, die sich auf dem \Wr{Blutmond} abgespielt hat. Ich war schon davor der Meinung, dass das Justizsystem der Union eine gründliche Verbesserung nötig gehabt hatte.}

\par

Die Studentin schien durch den Einspruch des Präsidenten ein wenig aus der Fassung geraten zu sein, sammelte sich aber recht schnell wieder. \WR{Was ich fragen wollte}, fuhr sie fort. \WR{Halten Sie diese Reform zurückblickend für etwas gutes? Ich meine, es gab nicht nur positive Resonanzen auf die Veränderungen.}

\par

\WR{Einige ihrer Kollegen bemängeln, dass Richter nun nach Gefühl entscheiden müssten}, vervollständigte Petersson die Ausführung seiner Kommilitonin.

\par

Otis wurde bewusst, dass der junge Student wohl gerne im Mittelpunkt stand. Eine Eigenschaft, die einige Politiker bei sich finden konnten, und die auch mit den Jahren kaum nachließ.

\par

Der Präsident holte zu einer Antwort aus: \WR{Ich denke nicht, dass Richter nun nur noch nach ihrem eigenen Gefühl entscheiden können. Diese Justizreform wurde durchgeführt um Willkür in der Rechtsfindung abzubauen, nicht um sie zu begünstigen. Ich habe selbst erlebt, wie ein Firmenchef ungestraft geblieben war, weil es bei der Beweisfindung zu Unstimmigkeiten gekommen war. Alle Beweise waren legitim, konnten aber nicht mehr verwendet werden. Die Justizreform von vier acht sieben hatte die Aufgabe solche Missstände zu beheben, auch wenn sie bewusst zulässt, dass völlig diametrale Rechtsauffassungen nebeneinander bestehen. Man kann eben nicht alle Menschen mit demselben Maß messen. Das wäre, als würde man Capezin nach seinem Geschmack und Olivenöl nach seiner Eignung, Raumschiffe anzutreiben, bewerten.

\par

Von den Zuständen auf VT drei drei vier ganz zu schweigen.}

\par

Henry Otis erinnert sich sehr lebhaft an die heftigen Debatten, die er vor zweiundzwanzig Jahren und davor geführt hatte. Bis zu diesem Zeitpunkt hatte die Union Schwerverbrecher, die wegen Vergewaltigung oder mehrfachem Mord verurteilt worden waren, auf den unbewohnten Planetoiden VT drei drei vier verbannt. Der Himmelskörper an sich war zumindest in den terraformierten Gegegenden recht fruchtbar und hatte ein mildes Klima. Aber die Gefangenen hatten keine Möglichkeit ihn zu verlassen. Sie waren einfach dort zurückgelassen und vergessen worden~-- bis zu dem Tag an dem sich herausstellte, dass es zu einem Justizirrtum gekommen war. Einer der Verurteilten war unschuldig. Er wurde befreit. Aber die anderen Gefangenen wollten ihren Mithäftling nicht einfach gehen lassen. Vier der Inhaftierten starben bei der Befreiungsaktion. Anschließend verlor die Union mehrere Millionen Naira an den zu Unrecht verurteilten, der mit einer Schadensersatzklage erfolgreich gewesen war.

\par

Damit hatte die ganze Misere aber ihr Ende noch lange nicht gefunden. Ein Großteil der Verbannungsfälle waren neu aufgerollt worden. Für die meisten Verhandlungen war das Beisein des Angeklagten aber unerlässlich. So wurden in einer großen Militäraktion einundzwanzig Gefangene von VT drei drei vier abgeholt, der nicht erst seit damals den Beinamen \Wr{Blutmond} getragen hatte. Unglücklicherweise hatten sie es aber fertig gebracht, den Landetransporter der Phalanx zu kapern. Sie hatten angegeben, die Mannschaft als Geiseln gefangen zu halten. In Wirklichkeit war aber niemand aus der Besatzung mehr am Leben gewesen. Am Ende war der Transporter abgeschossen worden.

\par

Diese schlimmen Tage waren gleichzeitig auch Henry Otis große Stunde gewesen. Sein Vorschlag eines völlig neuen Justizsystems war noch wenige Wochen zuvor belächelt worden. Nach diesen Ereignissen jedoch, waren seine Ideen in aller Munde geraten. Und kurz darauf stand das neue System. Verbannungen oder gar die Todesstrafe, wenngleich niemals angewendet, waren verboten, die Gesetzestexte radikal verkürzt oder umgeschrieben worden.

\par

Diese Reform hatte sich bewährt und Henry Otis sogar siebzehn Jahre später noch Rückenwind im Präsidentschaftswahlkampf verschafft.

\par

Obwohl die meisten der Studenten zu jung waren, um diese Ereignisse bewusst miterlebt zu haben, war die Gruppe gut genug informiert um, wie der Präsident, ebenfalls kurz in Gedanken zu versinken.

\par

Kurz darauf meldete sich Petersson wieder zu Wort. Henry Otis zweifelte nicht daran, dass er von ihm noch einige Fragen zu hören bekommen würde. \WR{Herr Präsident}, begann der Student, \WR{in der Präambel des Gesetzbuchs der Union ist eindeutig festgehalten, dass jedes Leben wichtig und kostbar ist und keinesfalls zerstört werden darf. Wie rechtfertigt sich denn dann unsere Armee? Die hat schließlich die Aufgabe, unter Umständen auch Leben auszulöschen.}

\par

Henry Otis kam nicht darum herum zufrieden zu lächeln als er diese Frage hörte. Sie würde ihm eine Plattform um erneut seinen Standpunkt zur Prudentium Phalanx Defensionum und der Starforce vorzubringen. Voller Vorfreude begann er: \WR{Meiner Meinung nach hat sie gar keine Legitimation. Sie sprechen mir aus der Seele wenn sie sagen, dass Menschen sich nicht gewaltsam gegen andere Menschen richten sollten. Und ich freue mich, dass es Leute gibt, die der Armee kritisch gegenüberstehen.}

\par

Nun war es an Petersson zu lächeln. Allerdings kaum merklich und auf eine siegesgewisse, schadenfrohe Art. \WR{In der Tat, Herr Otis, sind Sie nach eigenen Aussagen zwar gegen die Armee aber weiterhin für eine Polizeipräsenz. Und die Sicherheitskräfte haben ebenfalls die Aufgabe notfalls auch tödliche Gewalt anzuwenden~-- gegen Menschen.}

\par

Der Präsident schluckte. Petersson hatte es tatsächlich geschafft ihn in eine Falle zu locken. Er hatte sich so darauf gefreut, sich wieder über die Armee auslassen zu können, dass er nicht bemerkt hatte, was der junge Student vorgehabt hatte. Jetzt galt es, schnell eine schlüssige Antwort zu finden.

\par

\WR{Wissen Sie}, begann Henry Otis, \WR{es ist manchmal notwendig eine Abwägung zu treffen zwischen einem Ideal und der Notwendigkeit. Ganz ohne Polizeikräfte geht es nicht. Aber ich denke, eine Armee, die jährlich Milliarden an Naira verschlingt hat einfach keinen Platz mehr.}

\par

Die Entgegnung gefiel dem Präsidenten nicht voll und ganz aber sie würde genügen, um ihn nicht allzu dumm dastehen zu lassen. Petersson dachte wahrscheinlich gerade nach, wie er die Gunst der Stunde nutzen konnte, um den Präsidenten erneut aufs Glatteis zu führen. Die Denkfalten auf der Stirn des jungen Studenten wurden größer und größer. Aber eine seiner Kommilitoninnen ergriff das Wort und kam ihm so, wenn gleich unbeabsichtigt, zuvor: \WR{Herr Präsident, ich habe eine Frage zur Finanzpolitik der Union.}

\par

Henry Otis nickte, als Zeichen seiner Bereitschaft zuzuhören. \WR{Es geht um das Gesetz zur Kapitalbegrenzung}, fuhr die Studentin fort. \WR{Es wurde zwar lange vor ihrer Amtszeit verabschiedet aber sie haben es immer offen unterstützt. Finden Sie es nicht ungerecht, einem Menschen eine Obergrenze für Reichtum aufzuerlegen, nur weil er deutlich mehr verdient als andere? Ich meine, der Betreffende arbeitet ja in dem Fall auch sehr viel mehr als der Rest. Verdient er dann nicht auch eine gewisse finanzielle Unabhängigkeit? Ich persönlich empfinde das Gesetz als einen Verstoß gegen die Grundfeste der Union. Man fühlt sich sogar an die Irrlehren aus der vorkatastrophalen Ära erinnert.}

\par

Der Präsident räusperte sich ausgiebig. Dieses Thema zu besprechen würde nicht ganz einfach werden. Nicht zuletzt auch, weil einige Zuschauer dem Kapitalgesetz wahrscheinlich genauso kritisch gegenüberstanden, wie die Studentin auch.

\par

Vorsichtig entgegnete er: \WR{Sie haben natürlich auf einer Seite Recht. Ich denke ebenfalls nicht, dass der Staat willkürlich die Verdienste seiner Bürger an sich reißen darf. Aber darum geht es bei diesem Gesetz nicht. Ich denke wir sind uns einig wenn ich sage: Was sich ein Mensch erarbeitet, dass steht ihm auch zu.} Henry Otis bemerkte, wie wenig Wirkung seine beschwichtigenden Worte erzielten. Das Gesicht der Studentin wirkte nun sogar etwas verärgert. Trotzdem fuhr er fort: \WR{Wenn jemand viel Arbeitet, viel Verantwortung trägt und so weiter, dann steht ihm natürlich auch eine hohe Bezahlung zu. Aber irgendwann kommt der Punkt, an dem das Geld beginnt für sich selbst zu arbeiten. Und genau das versucht das Gesetz zur Kapitalbegrenzung zu verhindern. Nicht umsonst gibt es in der Union den Grundsatz, dass sich Leistung lohnen muss, Lohn aber auch Leistung voraussetzt. Es ist daher wichtig, zu verhindern, dass einzelne Wirtschaftsakteure ein zu massives Kapital erwirtschaften und sich am Ende vielleicht ein Monopol einrichten können. Als Privatperson kommt irgendwann der Moment, in dem man außerstande ist, sein Geld sinnvoll einzusetzen. Unser Abgabensystem wird oft als zu vereinfacht und primitiv bezeichnet~-- \textit{genau} der Anteil, den ein Wirtschaftsakteur am Gesamtkapital hat, steuert er für gemeinschaftliche Investitionen bei. Dadurch wird es transparent.}

\par

Zufriedenheit mit der Antwort des Präsidenten war das Letzte, dass sich auf dem Gesicht der Studentin widerspiegelte. Aber Henry Otis hatte nicht damit gerechnet, es jedem Recht machen zu können. Irgendwann kam der Punkt, so glaubte er, an dem man sich auf etwas festlegen musste.

\par

Die nächste Frage, stammend von einem jungen, locker gekleideten Sudenten, sollte den Präsidenten vor weniger Probleme stellen. \WR{Die Union feiert in ein paar Monaten ihren sieben dutzendsten Jahrestag}, begann der Mann. \WR{Das ist eine lange Zeit. Ich denke es hat noch nie einen menschlichen Staat gegeben, der so stabil und gleichzeitig so umfassend war aber über eine derart lange Zeitspanne für Frieden gesorgt hat. Worin liegt, Ihrer Meinung nach, das Erfolgsgeheimnis der Union?}

\par

\WR{Ich finde, das ist eine sehr wichtige Frage}, begann Henry Otis unmittelbar zu antworten. \WR{Wir alle müssen uns der vielen Errungenschaften der Union immer bewusst sein. Denn Sie sind sicherlich alles andere als selbstverständlich. Ich denke der Hauptgrund, wieso unsere Union die stabilste Organisation ist, die bisher von Menschen geschaffen wurde, ist, dass wir uns genauso mehr denn je auf den Wert des individuellen Lebens besinnen, wie auf die Wichtigkeit von gesellschaftlichem Engagement. Bereits die Allianz, die als erste Staatsform nach der Seuche entstanden ist, hatte erkannt, wie wichtig Leben ist und war schon nahe dran an dem, was wir heute als \Wr{Gleichgewichtsnormative} kennen. Die Union hat diesen Gedanken noch um einiges weiter geführt. Der Routenkrieg, der schließlich indirekter Auslöser für die Gründung der Union ist, hat uns noch einmal erkennen lassen, wie wichtig Einigkeit ist. Hätten die Überlebenden der Gabbot-Seuche damals nicht zusammengefunden, säßen wir heute nicht hier.}

\par

Einige der Studenten wirkten sehr nachdenklich. Die Themen, die der Präsident anschnitt waren für keinen einfach zu verdauen. Die Seuche war, über sechshundert Jahre nach ihrem Ende, noch fest in den Köpfen der Menschheit verankert. Wie eine Art kollektives Trauma, dessen Überwindung bis in die Zeiten der Union hinein Hauptthema vieler Kunstformen war.

\par

\WR{Ich denke, wir sollten uns ein Beispiel an den Menschen von damals nehmen}, fuhr Präsident Otis fort. \WR{Leute haben sich zusammengefunden, die sich gegenseitig vor dem Virus nicht einmal das Recht zu Leben eingeräumt hatten. Ich frage mich, ob wir uns jemals darüber klar geworden wären, wie sehr zerbrechlich wir sind, wenn die Dinge nicht gekommen wären, wie sie sind. Die Seuche war das schlimmste was der Menschheit jemals zugestoßen ist. Aber sie war vielleicht auch unsere letzte Rettung.}