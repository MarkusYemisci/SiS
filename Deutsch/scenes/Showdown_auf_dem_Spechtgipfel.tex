\WR{Sie werden uns garantiert gleich sehen}, meldete Kacper Piecek mit konzentriertem Blick durch sein Fernglas.

\par

Nico lag neben Schwarzer Bär im Schnee und war dabei besonders dankbar um seine Jacke. Die dunklen Rüstungen der Marineinfanteristen waren auf dem hellen Untergrund gut zu erkennen. Lediglich die Dunkelheit schützte die Soldaten vor der sofortigen Entdeckung. Aber Nico hatte die Shutek und ihre Technologie selbst erlebt und er war sich sicher, dass seine Gegner sich nicht alleine auf ihre tief sitzenden Augen verließen.

\par

Das Gelände war relativ übersichtlich. Die kleine Mannschaft kauerte nur knappe zweihundert Meter vor einem Gebäude, dass früher offensichtlich einmal nicht zu der militärischen Basis gehört hatte. Es sah viel mehr nach einem alten Hotel aus. Nur drei Stockwerke hoch und mit einem mit Ziegeln gedeckten Dach. Dahinter erstreckte sich das Flugfeld des Stützpunktes auf dem Hochplateu des Spechtgipfels. Ein gigantischer aber völlig leer wirkender Hangar wirkte auf der Wiese seltsam deplaziert.

\par

Doch er war Nicos eigentliches Ziel. Die Basis war zwar offenkundig von den Shutek überfallen worden, doch sicher nicht, ohne vorher in Alarmbereitschaft gehen zu können. Das wiederum bedeutete, dass einige der Flieger vielleicht bereits startklar gemacht worden waren. Mit etwas Glück hatten die Shutel sie nicht demoliert und es war auch eine Fähre für die Marineinfanteristen dabei.

\par

Die Körperpanzer der Shutek blitzten im Mondlicht immer wieder bedrohlich auf. Die Lichter der Basis blieben aus. So schimmerten sie in einem dunklem grün.

\par

\WR{Da sind mindestens vier dutzend von denen}, mahnte Minx, die nun ebenfalls durch ihren Fernstecher sah. \WR{Vielleicht können wir uns durch dieses Waldstück an ihnen vorbei schleichen.}

\par

Schwarzer Bär besah sich die dichte Reihe an Tannen, die an der Nordflanke des Berges hinauf wuchs. Zwischen ihnen war es in der Tat stockdunkel.

\par

\WR{Nein}, entschied er jedoch rasch. \WR{Erstens wissen wir nicht, ob sie uns nicht vielleicht trotzdem sehen und zweitens haben wir kaum noch Zeit.} Er verschaffte sich mit eigenen Augen einen Überblick über die Bewaffnung seines Stoßtrupps. Der Anzeige auf seinem Zielmonokel traute er nicht, denn einige Waffen waren dafür bekannt, kaputt zu sein und dennoch Bereitschaftssignale auszusenden.

\par

\WR{In Ordnung. Fischauge: du und Nico. Ihr rennt zum Hangar. Denkt dran, ihr müsst beide unentdeckt bleiben. Tiger und ich nehmen sie uns mit den schweren Waffen vor. Wir wollen sie hierher locken und sie beschäftigen, damit unser Fliegerbubi eine Maschine finden und starten kann. Aber haltet sie euch auf Distanz! Wir müssen nachher auch noch zum Hangar rennen können.}

\par

Nicos fragender Blick fand Schwarzer Bär. \WR{Das sind über ein dinmeter. Ihr schafft es niemals zum Hangar.} Er war bemüht, leise zu sprechen, um die anderen nicht zu beunruhigen. Das Klacken ihrer Waffen, die in Bereitschaft gebracht wurden~-- auch wenn bewusst leise ausgeführt~-- halfen, seine Worte zu übertönen.

\par

\WR{Müssen wir auch gar nicht}, entgegnete Schwarzer Bär. \WR{Vorausgesetzt, du schaffst es, eine Fähre \textit{hier} zu landen.}

\par

Nico holte unwillkürlich tief Luft. Das Gelände war uneben genug, um eine Landung zu erschweren und dennoch flach genug um von so gut wie jeder erdenklichen Seite aus ins Kreuzfeuer genommen werden zu können. \WR{Hab ich eine Wahl?}, fragte er schließlich.

\par

Die prompte Antwort überraschte ihn nicht. \WR{Scheißt Präsidentin Akintola in den Wald?} Stille folgte.

\par

\WR{He, ihr habt nicht zufällig noch eine Knarre übrig, die etwas mehr Wumms hat?}, fragte Nico schließlich.

\par

Minx und Schwarzer Bär tauschten einen kurzen Blick aus. Dann drückte sie ihm eine längliche Pistole mit schmalem, fast kastenförmigen Lauf in die Hand. \WR{Eine Strahlenkanone. Fast kein Rückstoß und schmilzt sich durch fast jede Panzerung. Aber du hast nur gute fünf Schuss, bis du sie aufladen musst.} Dann reichte sie ihm noch ein kleines Magazin, dass an den Rändern blau schimmerte. \WR{Mein letztes. Geh sparsam damit um.}

\par

Nico nickte. Kacper Piecek griff in an der Schulter und zog ihn mit sich. Widerwillig folgte ihm der Pilot geradewegs auf die Reihen der Feinde zu.

\par

\WR{Lasst uns abrocken!}, rief Schwarzer Bär schließlich und sprang aus der Deckung. Sofort begann sein Maschinengewehr zu rattern. Zumindest einmal an diesem Tag war er der Angreifer. Und so schrie er auch.

\par

Die verbrauchten Patronenhülsen spritzten nur so glühend aus der Seite der Waffe heraus und brachten den Schnee zischend zum schmelzen. Allerdings war dieses Geräusch nicht zu hören im Trommelfeuer der zahllosen Kanonen, die nun Feuer in Richtung der Shutek spuckten.

\par

Die erste Salve ließ einige der feindlichen Soldaten schier zerbersten. Dann begann der Gegenangriff. Als wären sie gerade erst zum Leben erwachte, begannen viele der Feinde in einem kaum zu erkennenden roten Licht zu flimmern. Kurz darauf zuckten die Strahlenentladungen der Gegner über die Köpfe von Bärs Infanteristen hinweg.

\par

Nico sah kurz hinter sich, erkannte aber nur, wie einer der Männer einen schweren Granatenwerfer in Anschlag brachte und ein hell leuchtendes Geschoss in hohem Bogen und mit gleißend hellem Schweif auf die feindlichen Reihen abfeuerte. Noch bevor es den Boden erreichte, zerplatzte es. Die Einzelteile gingen wie Feuerregen über die Shutek nieder und detonierten ein zweites mal. Obwohl mindestens ein halbes Dutzend Shutek im Flammenmeer untergingen, nahm der feindliche Ansturm kaum einen Abriss.

\par

Dann erwachte auch noch ein feindliches Abwehrgeschütz zum Leben und spie den Marineinfanteristen Entladungen von der Dicke eines Unterarms entgegen.

\par

\WR{Verdammt!}, fluchte Fischauge und ging sofort in die Hocke. \WR{Das kann ich mir nicht mit ansehen!} Er legte sein Gewehr an und musste nicht lange zielen. Sofort nach dem schier ohrenbetäubenden Knall fiel der Shutek aus seinem Sitz, der die Kanone bedient hatte.

\par

\WR{Weg hier! Dass haben die garantiert nicht überhört!}, forderte Piecek und Nico kam der Aufforderung nur zu gerne nach.

\par

Gemeinsam stapften sie durch den Schnee, rechts an dem Gebäude vorbei, dass nach einem ehemaligen Hotel aussah. Obwohl er mit jedem Schritt nach Luft rang und am ganzen Körper schwitzte, hatte Nico das Gefühl, kaum voran zu kommen. Der Schnee war tief und durch den schneidenden Wind fror er trotz der Anstrengung.

\par

Kaum hatten sie die Ecke des Hauses erreicht, kamen von einem nahen Parkplatz zwei Shutek heran gerannt. Nico zögerte nicht lange. Er brachte seine Pistole in Schussposition und ließ den Laserstrahl zur Zielhilfe seinen Gegner finden. Als er abdrückte, stellte er fest, dass die Waffe in der Tat kaum einen Rückstoß erzeugte. Nur einen hell leuchtenden Strahl, der sich durch die Brust seines Ziels fraß und den Shutek scheppernd zu Boden fallen ließ.

\par

Sein Partner schien fast zeitgleich in die Knie zu gehen. Nicos Blick schwang herum und er erkannte, wie Fischauge sein Gewehr bereits wieder sinken ließ. Dabei bewunderte er, wie der doch recht kleine Mann, seine Waffe so mühelos anheben konnte.

\par

\WR{Stopp!}, rief dieser und hielt Nico mit ausgestrecktem Arm zurück. Sie waren noch nicht an der Häuserecke vorbei gelaufen und Piecek wollte sichergehen, dass sie nicht sofort in die nächste Gruppe Gegner hinein rennen würden.

\par

Nico staunte nicht schlecht, als sich eine Art Miniaturfernrohr aus dem Mittelfinger des Handschuhs ausfuhr. So brauchte der Scharfschütze nur noch seine Hand auf die Häuserecke zu legen und konnte die Bilder des kleinen Spionagegeräts auf seinem Zielmonokel einsehen.

\par

\WR{Die Luft ist rein}, meldete er und zog Nico sofort hinter sich her.

\par

Der Parkplatz war offensichtlich mit der in der Union üblichen, schneeschmelzenden Beschichtung überzogen, denn der Asphalt lag frei. Auf diesem Untergrund kamen die beiden viel schneller voran. Der Hangar lag nur noch gute fünfzig Meter von ihnen entfernt.

\par

Trotzdem bremsten beide ihren Lauf schnell ab. Eine Gruppe Shutek kam eine Straße zu ihrer rechten hinauf gerannt und schnitt ihnen so den Weg ab. Es waren zu viele, als dass die beiden etwas gegen sie hätten ausrichten können. So gingen sie hinter einem Findling in Deckung und beobachteten, wie sich die neuen Gegner dem Kampf gegen Schwarzer Bärs Stellung anschlossen.

\par

Als ein neuer Granateneinschlag eine Schützenreihe der Gegner in Staub verwandelte und so einiges an Aufmerksamkeit auf sich zog, rannten Nico und Kacper Piecek wieder weiter.

\par

\WR{Ich hoffe wirklich, die haben dort überhaupt ein paar funktionierende Maschinen stehen}, sagte Nico, der mittlerweile kaum noch Luft bekam.

\par

Beide hatten die Straße erreicht, über die gerade noch die Verstärkung der Shutek gerannt gekommen war. Piecek wollte gerade etwas sagen, als er wie aus dem Nichts in die Luft gerissen wurde. Nico fiel vor Schreck von den Beinen, als er realisierte, dass sein Kamerad gerade von einer riesigen Hand in die Höhe gezogen wurde.

\par

Diese gehörte zu einem Kampfläufer der Shutek. Kleiner und weniger filigran als die Riesenspinnen aber noch einmal um ein Vielfaches furchteinflößender. Der massige Torso schien keinen wirklichen Kopf zu besitzen. Lediglich ein gigantisches Maul, dessen Zähle aus demselben Material zu bestehen schienen, wie die Bajonette der Shutek-Gewehre.

\par

Kacper Piecek brüllte einen markerschütternden Schrei in die Nacht, kurz bevor er im Schlund dies Menschenfressers verschwand. Nico blieb einen Moment lang liegen. Völlig unfähig, sich zu rühren. Hilflos sah er mit an, wie der Kampfläufer auf den Überresten seines Kameraden herum kaute und diese schließlich auch verschluckte.

\par

Sein Blick ging zu seiner Strahlenpistole. Er riss die Waffe förmlich vom Boden und gab alle seine vier verbleibenden Schüsse direkt auf das Maul des Menschenfressers ab. Erneut stockte ihm der Atem, als er erkannte, wie die Entladungen scheinbar wirkungslos auf die Blocker des Kampfläufers trafen.

\par

Auch ohne Augen schien er auf Nico aufmerksam zu werden. Dieser erhob sich sofort und rannte auf den Hangar zu. Ein besonders heftiger Adrenalinstoß erlaubte es ihm dabei, gleichzeitig zu laufen und seine Pistole nachzuladen.

\par

Ein Blick über die Schulter verriet ihm, was er bereits befürchtete. Der Menschenfresser war nun hinter ihm her. Der gigantische Kampfläufer stützte sich bei Rennen wie ein Gorilla auf den Fäusten ab. Er kam so schnell näher, dass Nico gar nicht merkte, wie er gegen die Hintertür des Hangars prallte.

\par

Mit der Tür gemeinsam fiel er ins Innere des Hangargebäudes. Ein einziges mal dankte er den Shutek für ihre offensichtliche Zerstörungswut. Denn die dicke Panzerung der Tür war von einigen Treffern mürbe gemacht und sie selbst aus den Angeln gehoben worden.

\par

Gerade noch schaffe er es, in Deckung zu gehen, als der Menschenfresser mit seiner riesigen Faust in den Durchgang schlug. Die Überreste der Tür landeten scheppernd im Innenraum.

\par

Nico achtete nicht darauf, sondern zwang sich zurück auf die Beine. Mit der feuerbereiten Pistole stürmte er den Korridor entlang. Er kannte die Grundrisse der Starforce-Gebäude zwar nicht auswändig, doch er wusste, dass diese meistens nach demselben Prinzip aufgebaut waren und wo er ungefähr den Startbereich finden konnte.
\ortswechsel
Schwarzer Bärs Nebenmann wurde von einem gut gezielten Strahl aus einem Sturmgewehr der Shutek von den Beinen gerissen. Er hatte dies an jenem Tag schon oft genug gesehen, um abzustumpfen. Doch das scherzverzerrte Gesicht des Mannes und die blanke Angst, die in seine nun leeren Augen geschrieben stand, ließen ihn dennoch im Inneren so kalt werden, wie die Umgebung.

\par

Dann übernahmen seine gut eingeübten Kampftaktik für seinen eingefrorenen Verstand. Der Mann, der gerade eben den Tod gefunden hatte, hatte ein Gatling-Gewehr in beiden Armen getragen. Die Waffe war von enormer Bedeutung für diesen Kampf auf weiter Flur.

\par

Schwarzer Bär spurtete zu seinem gefallenen Kameraden, ließ sein Gewehr fallen und hob die Gatling-Kanone auf. Kaum dass sein Handschuh den Griff der Waffe berührt hatte, übertrug die Zielkamera des Gewehres ein Bild mit geschätzter Projektilflugbahn auf Bärs Monokel.

\par

Einen passenden Gegner zu finden, fiel ihm Leicht. Einer der besser gepanzerten Shutek mit persönlichem Blocker schien den feindlichen Vorstoß anzuführen. Immer wieder gab er Schüsse in Richtung der Marineinfanteristen ab, ohne sich dabei wirklich um Deckung zu bemühen.

\par

\WR{Achtung. Heißes Blei!}, brüllte Schwarzer Bär unnötigerweise~-- seine Kameraden hielten sich bestensfalls neben aber nicht vor ihm auf.

\par

Dann ließ er seine neue Waffe sprechen. Innerhalb einer Sekunde verfeuerte sie achtzig glühend heiße Metallkugeln in Richtung des Panzershuteks. Nur einige davon trafen ihn tatsächlich, doch das reichte, um ihn von den Beinen zu holen. Die Salve trennte seinen Torso vom Unterleib und er fiel wild mit den Armen rudernd zu Boden.
\ortswechsel
Nico hatte Mühe, zu bremsen, als er um eine Ecke rannte und sich plötzlich im Flugbereich des Hangars wiederfand. Die Halle war zwar beinahe komplett dunkel, doch er konnte die zwei Shutek mühelos erkennen, die neben einem Falken standen und diesen auseinandernahmen.

\par

Instinktiv legte er an und schoss. Die Entladung hüllte die Halle für einen kurzen Augenblick in tiefes orange, bevor sie auf ihr Ziel traf und sich durch dessen skelletartige Brust fraß. Der andere Shutek erwiderte sofort das Feuer. Nico hatte keine Zeit zu staunen, wie schnell der Soldat seine Pistole gezogen hatte, die zwar kleiner wirkte, wie die regulären Gewehre der Shutek aber verglichen mit den Waffen der Phalanx nach wie vor überproportioniert aussah.

\par

Ein stechender Schmerz fuhr durch Nicos Schulter, nachdem er sich unsanft auf den Boden geworfen hatte.

\par

Als er sich wieder aufgeraffte hatte, kam der Shutek bereits mit metallisch scheppernden Schritten auf ihn zu.

\par

\denken{Jetzt oder nie}!, sagte er in Gedanken zu sich selbst und schwang seinen Oberkörper auf den fahrbaren Instrumententisch, den er als Deckung benutzt hatte. Der Shutek sah bereits in seine Richtung, doch er war schneller. Der hellgelbe Strahl seiner Pistole löste den Kopf seines Gegners in Rauch auf. Wild zuckend fiel der Rest seines Körpers in sich zusammen.

\par

Nico sah sich hastig um. Die Halle schien leer und so wie er die Reaktionszeit der Shutek einschätzte, hätte man ihn bereits erschossen, wenn sich ein weiterer Gegner irgendwo in Deckung befand.

\par

Schnell hatte er sich eine Übersicht gemacht und klatsche vor Freude in die Hände. Neben der demontierten Falken stand noch ein weiterer in der Halle. Genauso wie zwei Personalfähren. Diese waren zwar nicht allzu geräumig, würden aber alle Marineinfanteristen aufnehmen können.

\par

Widerwillig stieg Nico über die Leiche des Shuteks, den er gerade eben erschossen hatte. Nach wie vor zuckten seine Gliedmaßen unkontrolliert hin und her und ließen den kopflosen Gegner wie einen Besessenen wirken.

\par

\WR{Du hörst dich ziemlich durchgerostet an}, sagte Nico zu den Überresten des Shutek. \WR{Zeit für eine Ölung. Die letzte!} Der Strahl aus seiner Pistole durchborte den Brustkorb der Leiche, ließ sie sich noch einmal aufbäumen und dann regungslos hinabsinken.

\par

\WR{An meinen Sprüchen muss ich noch arbeiten}, sinnierte Nico, als er auf den augenscheinlich intakten Jäger zuging.

\par

Er schloss unwillkürlich die Augen, als er die toten Körper dreier Menschen sah, die zwischen den beiden Jägern lagen. Zwei von ihnen trugen die Uniformen von Piloten, während die dritte eine Galauniform der Phalanx trug. Vermutlich war sie einmal der Kommandant des Stützpunktes gewesen und hatte die Basis verlassen wollen. Die beiden Piloten hätten in diesem Fall wohl Geleitschutz fliegen sollen.

\par

Nico zog daraus zwei Schlussfolgerungen, die ihn hoffnungsvoller stimmten. Zum einen verließen Kommandanten zumeist als letzter die Stellung, was bedeutete, dass der Rest der Besatzung der Basis vielleicht mit dem Leben davon gekommen war. Zumindest waren in der großen Halle keine weiteren Leichen zu sehen, welche an einem derart umtriebigen Ort aber klar zu erwarten gewesen wären.

\par

Zum anderen bedeutete dies, dass zumindest einer der beiden Jäger noch startklar sein sollte.

\par

Nico griff nach dem Helm einer der Toten und schloss ihm so schnell und sanft er konnte ein letztes mal die Augen. Dann schob er eine bereitstehende Leiter an den Rumpf des Jägers und stieg hinein. Sie würde ihn zwar später beim Abheben stören, doch damit würde er leben müssen.

\par

So zügig es ging, schmiss er eine handvoll Schalter in die An-Stellung und sah zu, wie die Bildschirme seines Jägers zum Leben erwachten. Sofort wählte er die kürzestmögliche Checkliste und griff nach dem Funkgerät, dass ihm einer der Soldaten gegeben hatte.

\par

\WR{Bär, hier Snoopy, bitte kommen.}

\par

\WR{Snoopy?}, war die kurze Antwort, begleitet von donnerndem Geweherfeuer.

\par

\WR{Mein Rufzeichen. Ich habe einen Falken gefunden und auch eine Fähre. Ich starte jetzt und übernehm die Kontrolle über den größeren Vogel per Fernsteuerung. Es wird ein bisschen haarig aber ich denke, wir haben so bessere Chancen. Ich kann euch Feuerschutz geben, während wir…}

\par

\WR{Ja, ja!}, entgegnete Schwarzer Bär hörbar angespannt. \WR{Egal was du vorhast, mach schnell! Wir werden überrannt.}

\par

\WR{Durchhalten!}, befahl Nico und sah dann wieder auf die Checkliste. Die meisten Einträge waren abgearbeitet und er schlug förmlich auf die Zündung ein. Ohne Bodenbesatzung war ein Kaltstart ein extrem heikles Manöver. So atmete Nico hörbar aus, als die beiden Haupttriebwerke seines Vogels zum Leben erwachten und Feuer zu spucken begannen.

\par

Keinen Augenblick später fuhr er aber zusammen, als er ein ihm erst seit kurzem bekanntes stampfen vernahm. Er wusste, wozu die Laute gehörte, noch ehe er den Menschenfresser vor das Tor des Hangar treten sah.

\par

Inständig hoffend, dass die, verglichen mit dem Rumpf des Jägers wie brüchige Stelzen wirkenden, Räder nicht blockieren würden, schob er den Schubregler nur ein winziges Stück weit nach vorne. Dieser stand ohnehin bereits nur auf Atmosphäremodus, doch die Kraft der Plasmaturbinen schoben seinen Flieger trotzdem gleich mit einer ordentlichen Geschwindigkeit auf die Startposition zu. Das Klappern der umfallenden Leiter ging im Gebrüll des Antriebs unter.

\par

\WR{Kommt schon!}, brüllte Nico seinen Waffenmonitor an. Die Strahlenkanonen seines Jägers waren noch nicht warm geworden, doch spätestens nun hatte ihn der Menschenfresser entdeckt. Überraschend schnell kam er auf den Jäger zu gerannt.

\par

Nico drückte seinen Steuerknüppel in Richtung des Läufers, bis sich das Fadenkreuz nach einer schier endlos erscheinenden Zeit endlich über den gigantischen Shutek schob.

\par

In der Hoffnung, dass der Strahlenkondensator bereits ausreichend voll geladen war, löste er die Sicherung und schob seinen Finger über den Abzug. \WR{Das ist für Kacper, du Hurensohn!}

\par

Und schon spien die ratternden Kanonen seines Jägers Feuer. Der Widerhall der Waffen war bereits innerhalb einer Atmosphäre laut. Doch in einem geschlossenen Raum dröhnten sie mit fast ohrenbetäubender Wirkung.

\par

Der Menschenfresser explodierte nicht nur sprichwörtlich. Sein massiger Körper platzte auseinander, nachdem er von den Strahleninpulsen praktisch auseinandergerissen worden war.

\par

\WR{Ja, da bist du platt, was?}, rief Nico dem brennenden Wrack des Kampfläufers entgegen.

\par

Dann riss er seinen Steuerknüppel mit aller Gewalt in die entgegengesetzte Richtung und drückte den Schubregler fast bis zur Mitte. Die Triebwerke des Falken stießen nun Stichflammen aus und drückten den Flieger aus der Halle hinaus. Beinahe hätte Nicos Backbordtragfläche den zerstückelten Menschenfresser touchiert.

\par

\WR{Verdammt, ich hätte: \Wr{Kein Grund, gleich in die Luft zu gehen} sagen sollen}, schalt er sich selbst und konzentrierte sich sogleich wieder auf seine Startbahn.

\par

Diese erschien ihm sofort viel zu kurz. Hinter dem Rollfeld konnte er bereits eine Reihe Tannen wie eine bedrohliche dunkle Wand aufragen sehen. Ein kurzer Blick auf die Geschwindigkeitsanzeige verriet ihm, dass sein Falke viel zu langsam war, um abzuheben.

\par

Mit einem routiniert eingeübten Handgriff fasste er an die manuellen Triebwerkskontrollen und gab volle Leistung auf die Manövriertriebwerke unter der Schnauze seines Jägers. Die beiden Triebwerke stießen augenblicklich weißglühende Flammen auf. Unterdessen raste er weiter auf den Wald zu.

\par

Die Nase des Falken neigte sich durch den vertikalen Schub langsam~-- quälend langsam~-- nach oben. Kurz bevor das Rollfeld in eine Heide einmündete, zündete er die Nachbrenner und drückte sein Schiff somit in die Luft, ohne, dass es den nötigen Druckunterschied unter- und oberhalb der Tragflächen erfahren hatte.

\par

Nico war nicht überrascht, als gleich mehrere Warntöne ihn auf das bevorstehende Versagen der künstlichen Schwerkraft sowie des schieren Zusammenhalts jeder einzelnen Schraube und Niete seines Vogels hinwiesen.

\par

\WR{Ja, ja, ja!}, brüllte er ohne wirklichen Ansprechpartner seinen Anzeigen entgegen. \WR{Das Ding fällt bald auseinander. Das hatte ich heute schon einmal!} Ein plötzliches Knarzen ließ ihn sofort wieder verstummen. \WR{Halt einfach noch ein bisschen durch, Schätzchen!}

\par

Ein Blick durch die Cockpitscheibe verriet ihm, dass er gerade so über die Baumwipfel streichen würde. Und kurz darauf hörte er tatsächlich die Spitzen der Nadelbäume gegen den Bauch seines Jägers schlagen. Der Monitor, der eine Sicht nach hinten offerierte, zeigte ihm den kleinen Waldbrand, den er gerade angerichtet hatte, als seine Feuer speienden Plasmaturbinen über die Tannen gezogen waren.

\par

Langsam atmete Nico auf, als er an Höhe gewann und seine Flugbahn sich zu stabilisieren begann.
\ortswechsel
\WR{Nein!} Schwarzer Bär hatte den Schrei selbst ausgestoßen. Doch es war ihm, als würde er jemand anderem zuhören.

\par

Vor wenigen Minuten noch hatte er Kacper Pieceks Vitalwerte auf der Übersicht seines Zielmonokels verschwinden sehen. Nun sackte Minx vor seinen Augen in sich zusammen. Ihre Brust glühte noch an der Stelle, an der die Strahlenentladung sie getroffen hatte.

\par

Schwarzer Bär hielt nach dem Schützen Ausschau, musste aber schnell erkennen, dass gleich ein halbes Dutzend Shutek in Frage kamen, die gerade dabei waren, die Stellung seines Stoßtrupps zu überrennen.

\par

Und schon kam ein weiterer Gegner um einen Felsen gerannt. Bär durchlöcherte den Neuankömmling mit dem letzten Rest an Minution, den sein Gatling-Gewehr noch hergab. Doch dann spürte er einen stechenden Schmerz in seinem rechten Bein.

\par

Er hatte den Strahl noch aus dem Augenwinkel erkannte. Unwillkürlich ging sein Blick am eigenen Körper hinab. Vom Knie abwärts fehlte ihm sein komplettes Bein. Zurückgeblieben war nur ein glühender Stummel.

\par

Dass er den Halt verlor und zu Böden stürzte, rettete ihm das Leben, als eine weitere Entladung geradewegs an seinem Kopf vorbei zog und in den Felsen hinter ihm einschlug. Der Geruch von schmelzendem Gestein stieg ihm in die Nase und erinnerte ihn an seine frühen Jahre als Pfadfinder. Beim Feuermachen hatte es oft ähnlich gerochen, wenn er die richtigen Steine aneinander gerieben hatte, um einen Funken zu erzeugen.

\par

Der Shutek, der gerade noch auf ihn geschossen hatte, stand nun geradewegs neben ihm. Die tief sitzenden Augen auf sein vermeintliches nächstes Opfer gerichtet. Die grässliche Fratze, die sein Gesicht bildete, blieb völlig unbewegt.

\par

\WR{So nicht!}, rief Schwarzer Bär und hechtete plump in die Richtung des Feindes. Gerade als dieser seine Tötung vollenden wollte, schwang er das Bajonett, dass er von einem der erbeuteten Gewehre entfernt hatte und hieb damit dem Shutek dessen eigene Beine ab.

\par

Ein weiterer Schlag trennte seinen harten, knorpelartigen Torso in zwei Hälften. \WR{Stirb du verdammtes Arschloch!}, brüllte er ihm entgegen und schlug mit der letzten Kraft, die sein schwächer werdener Körper noch aufbieten konnte, auf den Shutek ein, bis sich dieser nicht mehr rührte.

\par

Auf dem Rücken liegend sah er jetzt geradewegs in den nun sternenklaren Himmel. Der Anblick erinnerte ihn an die Urlaube, die er mit seiner Mutter verbracht hatte. Insbesondere jene, nachdem sein Vater aus der Ehe geflohen war. Sie hatte ihn stets vermisst. Er selbst weniger. Und an den Stränden von Thessaloniki hatten sie beide zumindest für ein paar Tage Ruhe und Frieden gefunden. Hier war alles weit weg, was sie beide im Alltag belastet hatte. Die bohrenden Blicke der Nachbarn, die eigentlich nur den verschwundenen Vater suchten. Oder die Probleme in der Schule und im Freundeskreis. Nur ein der warme Sand, das Rauschen der Wellen und Eiscreme.

\par

Schwarzer Bär kniff die Augen zusammen. Die Anzeige seiner Division war mittlerweile bis auf wenige Einträge komplett ausgegraut. Und die Shutek rannten nun von allen Seiten auf sie zu. Er hatte nicht mehr viel Zeit.

\par

\WR{Bär an Snoopy}, rief er in sein Funkgerät. \WR{Wir wurden überrannt. Es ist vorbei. Schaffen Sie Ihren Arsch hier raus!}

\par

Zunächst erklang keine Antwort. Nicos Stimme klang eine ganze Tonlage höher, als er schließlich sagte: \WR{Vergessen Sie das! Ich bin schon in der Luft, ich kann Ihnen in einer Minute Unterstützung von oben geben.}

\par

Der Klang nach dem Auf- und Abwickeln von Ketten war bereits dezent zu hören. Schwarzer Bär musste nicht erst den Kopf drehen, um zu wissen, dass er von einer anrückenden Riesenspinne stammte.

\par

\WR{Ein Kampfläufer ist im Anmarsch und wir sind hier nur noch zu dritt. Am Ende holt sie dieser Drecksack noch aus der Luft. Und bis Sie hier sind, sind wir sowieso erledigt. Hauen Sie ab, so lange sie noch können!}

\par

Der Schnee dämpfte seine Schritte ab. Doch Bär konnte den nächsten Shutek dennoch hören, der geradewegs auf ihn zu lief. Beinahe genüsslich langsam richtete er sein Gewehr auf den sterbenden Soldaten aus. Dieser griff sich lächelnd an den Hosenbund und zog eine Zylindergranate heraus. Wie einen Taschenregenschirm ließ er das Sprengladung aufspringen und scharf werfen.

\par

\WR{Tun Sie mir einen Gefallen, Fliegerbubi}, flüsterte er in sein Funkgerät, als könnte der Shutek ihn hören oder etwas mit dem anfangen, was sagen wollte. \WR{Sagen Sie meiner Mutter, ich liebe sie und sie muss irgendwie…}
\ortswechsel
Nico flog in Rückenlage und sah mit vor Angst geweiteten Augen über Kopf auf das Schlachtfeld auf dem Spechtgifel. Es fielen kaum noch Schüsse und die Stelle, an der er Schwarzer Bär und seine Kameraden zuletzt gesehen hatte, wurde für einen Augenblick von einem hellen Feuer erleuchtet, noch bevor der Soldat seine Bitte zu Ende gesprochen hatte. Nun klang nur noch statisches Rauschen aus Nicos Kopfhörern.

\par

Widerwillig drehte er seinen Vogel wieder in eine horizontale Position. Er war längst über den Berggipfel hinweg gezogen, doch schon warnte ihn sein Bordcomputer vor einer laufenden Raketenerfassung. Er zog die Nase des Falken hoch und schoss dem Himmel entgegen.

\par

Als er außer Reichweite der Riesenspinne war, die gerade noch versucht hatte, ihn aufs Korn zu nehmen, brüllte er förmlich in sein Headset: \WR{Ich rufe den Spechtgipfel. Ist da irgendjemand?} Die Frequenz stimmte. Es war zwar eine Weile her, seit er sich die gängigen Kanäle für die Starforce und die Phalanx eingeprägt hatte, aber er war sich sicher, auf dem Hauptgefechtskanal zu senden. \WR{Ich wiederhole. Hier ist Nico Curiosa. Ich rufe den Stoßtrupp auf dem Spechtgipfel. Ist da irgendwer? Bitte antworten!} Der Äther schwieg ihn an. \WR{Bitte!}

\par

Nico seufzte nicht oder hielt den Atem an. Er sah nur durch die Scheibe seiner Kanzel und gestand sich ein, dass auf der kleiner werdenden Oberfläche wohl kein Mensch mehr am Leben war.

\par

Als er Morten Witwer vor gerade mal ein paar Stunden das Leben gerettet hatte, war diesem ein Shutek auf den Fersen gewesen und hatte ihn angefunkt. Es fiel ihm nicht schwer, sich an die drei Zahlen der Frequenz zu erinnern, auf welcher der feindliche Flieger damals gesendet hatte. Er stellte sein Funkgerät auf den entsprechenden Kanal und sagte: \WR{Wisst ihr, ihr dreckigen Flachwichser, wir Menschen haben ein Sprichwort. Alles gute kommt von oben.}
