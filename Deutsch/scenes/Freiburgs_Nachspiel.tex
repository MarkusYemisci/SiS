Die Stimmung in Berater O’Sheas Büro hätte besser sein können. Laura Ghetas und Klaus Rensing starrten beide gleichermaßen auf den Boden als wären ihre Schuhe die interessantesten Gegenstände der Welt. O’Shea, Lauras direkter Vorgesetzter und Leiter der gesamten Aktion, saß hinter seinem Schreibtisch und sah auf den Bericht, der vor ihm ausgebreitet lag.

\par

Laura sah auf, mied aber den Blick ihres Vorgesetzten und blickte lieber aus dem Fenster auf das winterliche Oslo. Seit ihr Vorgesetzter sie und Klaus zu sich bestellt hatte, war kein einziges Wort gesagt worden. Die Stille dauerte nun schon einige Minuten an. Schließlich beendete O’Shea die Ruhe: \WR{Ich mache Ihnen keinen Vorwurf, Frau Ghetas. Trotzdem frage ich mich, wie das passieren konnte.}

\par

Gemächlich griff er nach einem Blatt aus dem Papierstoß vor sich und zitierte: \WR{Der Verdächtige floh, trotz vehementer Verfolgung und entkam schließlich in einen Kanalschacht. Die großangelegte Suche brachte bislang keine Ergebnisse ein.}

\par

Wieder wurde es ruhig. Klaus Rensing machte ein oder zweimal den Eindruck, etwas sagen zu wollen brach aber ab, ehe er ein Wort hervorbrachte. Schließlich war es Laura, welche die Stille brach und etwas zu ihrem eigenen Bericht sagte: \WR{Ich muss zugeben, dass unser Mann derart schnell entkommen konnte und uns so spielend abgehängt hat, damit habe ich nicht gerechnet. Ich übernehme die volle Verantwortung dafür. Wir hätten gleich mit einer Eingreiftruppe aufwarten sollen.}

\par

O’Shea hob beide Augenbrauen und las im Bericht weiter. \WR{Hier steht, der Flüchtige sei über mehrere Bahngleise gerannt, obwohl Zugverkehr herrschte. Das ist seltsam für einen einfachen Hacker.}

\par

\WR{Das dachte ich mir auch, als ich das gesehen habe, das können Sie mir glauben}, kommentierte Klaus Rensing mit aufgeregter Stimme. \WR{Er ist gerannt als hätten wir ihn töten wollen.}

\par

Lauras Vorgesetzter las weiter. Der Bericht schien ihn mehr und mehr zu verwundern. Schließlich legte er die Papiere weg, faltete beide Hände und sah Laura durchdringend an.

\par

\WR{Ich sehe keine Veranlassung, Ihnen den Fall zu entziehen. Aber bitte versprechen Sie mir, dass sie das nächste mal keine Fehler mehr machen.}

\par

\WR{Das hatte ich sowieso nicht vor, mein Herr}, antwortete Laura Ghetas schnell.

\par

Ihr Vorgesetzter schien über irgend etwas nachzudenken, dass ihn sehr beschäftigte. Sein Gesichtsausdruck wirkte alles anderes als gelassen.

\par

\WR{Na ja, ich schätze nicht nur wir haben das Problem, dass wir hin und wieder etwas nachlässig sind}, sinnierte er schließlich. \WR{Allgemein ist das Konglomerat etwas eingeschlafen, seitdem die Capital Fellowship nicht mehr aktiv sind.}

\par

Klaus Rensing räusperte sich und lächelte bitter. Laura wusste, wie schlecht er manchmal mit Kritik umgehen konnte und war froh, dass er sich nicht lauthals beschwert hatte. Stattdessen lies er seinen Blick durch O’Sheas Büro gleiten und blieb irgendwann bei einigen exotischen Fischen hängen. Die Geschöpfe stammten höchst wahrscheinlich aus einem Ozean auf Kreuzpunkt Primus. Der Planet an sich war lange nicht so artenreich, wie die Erde. Allerdings tummelten sich in den Meeren der kalten Welt zahllose Bewohner zum Teil bis unter die Eisdecken.

\par

Einer der Fische hatte zwei kleine Ärmchen anstatt Vorderflossen. Einen Moment lang kam es Klaus vor, als würde er ihm mit geballter Faust drohen.

\par

\WR{Eines verstehe ich nicht}, sprach O’Shea nachdenklich. \WR{In Ihrem Bericht steht, dass die Sicherheitskameras im Bahnhof keine verwertbaren Bilder geliefert haben. Wie kommt das, Frau Ghetas?}

\par

Laura konnte der Versuchung, sich im Kopf zu kratzen nicht widerstehen. \WR{Wie es aussieht, hat unser Mann alle Sicherheitssysteme im Umkreis des Verteilers gehackt und außer Betrieb gesetzt. Wie er das geschafft hat, ist mit schleierhaft. Überhaupt verstehe ich nicht, wie er so schnell mit dem Angriff fertig sein konnte.}

\par

Klaus Rensing meldete sich wieder zu Wort: \WR{Es ist zwar eine Weile her, dass unsere Dienststelle das letzte mal ein Phantombild \textit{gezeichnet} hat aber wir haben über zwanzig bestätigte Zeugen. So müssten wir schon bald ein sehr realitätsnahes Bild fertig stellen können.}

\par

Councellor O’Shea nickte zufrieden.

\par

\WR{Außerdem haben wir noch das}, sagte Laura mutiger, als sie sich fühlte und reichte ihrem Chef das Buch des Hackers.

\par

Die Spurenabteilung hatte bereits alle relevanten Hinweise auf den Täter von dem Gerät entnommen und so konnten Laura den Computer bedenkenlos in der Hand herumtragen. Ein leises Piepsen erklang, als O’Shea das Buch aufschlug. \WR{Gesperrt}, murmelte er verärgert.

\par

Laura begann zu erklären: \WR{Wir haben bisher nicht versucht, den Code zu knacken. Dieser Hacker ist wirklich gut. Wir befürchten, wenn wir auch nur den geringsten Fehler machen, zerstören sich alle Daten selbst.}

\par

\WR{Gibt es eine Möglichkeit, trotzdem an die Daten heranzukommen?}, fragte ihr Vorgesetzter hoffnungsvoll.

\par

Laura zuckte mit den Schultern. \WR{Schwer zu sagen. Unsere Elektronikabteilung hat da ein neues System entwickelt. Dabei wird der Speicherkern nicht angezapft sondern mit einem Scanner abgetastet. Wir würden die Daten sozusagen aus zweiter Hand erhalten. Kein Sicherheitsmechanismus kann dagegen gefeit sein.}

\par

\WR{Was ist das für eine Zahl?}, fragte O’Shea und deutete auf eine Nummer, die auf jeder Seite zu erkennen war und stetig kleinere Werte annahm.

\par

Klaus Rensing gab seine Hypothese zum besten: \WR{Ich denke, dass ist eine Art Zeitschaltuhr. Wenn die Zeit um ist, werden alle Daten gelöscht. Zumindest hätte ich so ein System eingebaut, wenn ich der Hacker gewesen wäre.}

\par

\WR{Und wenn du auf die Idee kommst, hat er das sicher geschafft}, murmelte Laura leise, während sich ihr Vorgesetzter das Gerät genauer ansah.

\par

Schließlich nickte O’Shea entschlossen und wies an: \WR{Wenn das so ist, sollten wir uns beeilen. Ich denke, diese neue Abtastungsmethode ist unsere beste Chance. Ich werde den Computer sofort zu Miles oder Pesscala bringen. Wie haben sie vor, weiter gegen unseren Mann vorzugehen?}

\par

Laura schoss sofort darauf los, als hätte sie diese Frage schon die ganze Zeit über beantworten wollen: \WR{Als erstes melden wir uns bei der Phalanx. Das nächste mal, wenn wir auf diesen Typen Treffen werden wir ein paar Soldaten dabei haben, die uns helfen.} O’Shea nickte anerkennend. \WR{Was das Aufspüren des Mannes angeht, da sehe ich unsere beste Gelegenheit darin, Hinweise aus der Bevölkerung abzuwarten und ihn dann erst einmal zu observieren. Wenn wir uns sicher sind, dass er nicht entkommen kann, schlagen wir erneut zu und schnappen ihn uns. Wenn es nötig ist, schießen wir auch mit Betäubungsmunition.}

\par

\WR{Gut}, begann ihr Vorgesetzter, \WR{aber wenn jemand feuert, stellen Sie sicher, dass keine Unbeteiligten getroffen werden. Wenn ein Passant getroffen wird und einen epileptischen Anfall bekommt, hängt in Null Komma nichts SSV um uns herum. Die sind sowieso bereits auf uns aufmerksam geworden. Einen weiteren Fehler sollten wir uns nicht erlauben.}

\par

Laura und Klaus erhoben sich. \WR{Es wird keinen geben}, gab Laura entschlossen von sich. Tatsächlich war sie sich keineswegs sicher. Sie wusste aber, dass sie selbst schlimmstenfalls noch einen machen würde.
