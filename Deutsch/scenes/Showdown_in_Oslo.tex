Die Bilder wiederholten sich und wiederholten sich. Laura sah genervt auf die Projektion über Klaus Buch. Ihr Partner hatte sein Gerät auf den Schreibtisch gelegt, den ihm seine Partnerin zur Verfügung gestellt hatte. Bereits den ganzen Tag lief ein Bericht über den Angriff auf Pollux nach dem anderen. Und obwohl die Nachrichtensprecher und Reporter immer wieder das Gleiche berichteten, schien Klaus Neugier nicht nachzulassen.

\par

Wieder wurden die Fotos gezeigt, die mithilfe eines Handcomputers aufgenommen, irgendwie den weg durch die Datenkontrolle der Starforce gefunden hatten. Sie schienen aus der Kabine eines zivilen Schiffes geschossen worden zu sein. Undeutliche Abbildungen von dunklen Raumschiffen mit weiten Schwingen und feuernden Geschütztürmen eines Navy-Schiffs überfluteten mittlerweile nicht mehr nur die DDV-Sendern, sondern die gesamte Krypta Scientia. Gefolgt von Videokommentaren etlicher Bürger, die zwar keine Ahnung zu haben schienen, aber dennoch zu allem ihren Senf gaben.

\par

Die offiziellen Nachrichten machten jedoch keinen viel besseren Eindruck. Es wurden zahllose, entnervte Flüchtlinge interviewt, genauso wie Experten, von denen zuvor noch niemand etwas gehört hatte. Aber die Kernfrage, wer den Angriff ausgeführt hatte und warum, konnte niemand beantworten.

\par

Auch der Bericht über angebliche Sichtungen von Außerirdischen in der näheren Vergangenheit, der gerade lief, konnte kein Licht ins dunkel bringen. Der Moderator einer Sondersendung spracht gerade mit einem Experten für extraterrestrisches Leben.

\par

\WR{Wir haben bis zum gestrigen Tag noch auf keinem der drei dutzend Planeten etwas gefunden, das über kleinere Tiere oder simple Vegetation hinausging}, erklärte der Diplomträger. \WR{Und im Gegensatz zu dem, was uns Marco Bellendi vor einiger Zeit weismachen wollte, gab es bislang auch noch keine Hinweise auf eine andere Zivilisation neben unserer eigenen.}

\par

Laura erhob sich seufzend und warf ihrem Partner einen kurzen Blick zu. Klaus schien sich nicht entscheiden zu können, was ihn mehr interessierte. Das Laura das Zimmer verließ, die Hinweise aus der Bevölkerung von Sucre oder die hastig zusammengeschnittene Reportage über Außerirdische. Schließlich fragte er: \WR{He, Laura. Gehst du schon?}

\par

\WR{In einer Stunde vielleicht}, antwortete sie. \WR{Es ist schon spät. Morgen ist auch noch ein Tag. Aber jetzt hol ich mir erst mal einen Tee.}

\par

Klaus nickte gedankenverloren. Sie sah ihn noch einen Moment lang an, dann verschwand sie. Es fiel ihr immer schwerer, ihn nicht einzuweihen. Bisher, so glaubte sie, hatte sie ihn noch täuschen können. Er war launisch und frustriert, so wie jedes mal, wenn eine Spur im nichts verlief. Die Sondereinheit, die immer noch die Wohnung des Hackers durchsuchte, hatte bislang nichts verwertbares gefunden und sich stattdessen bloß einen Haufen Computerviren eingefangen.

\par

Müde erreichte sie die Messe. Außer ihr waren nur zwei Agenten von der Drogenabteilung und Yoshi Tagori anwesend, einer von Lauras Kommilitonen auf der Akademie. Trotz der hohen Fenster drang kaum Licht in den Raum. Draußen war es stockdunkel und die Laternen der Stadt drangen kaum durch den dichten Schneefall.

\par

Laura gefiel der Aufenthaltsraum. Seine Ausstattung war zwar so schlicht, dass es schon fast traurig war. Aber dennoch fand sie ihn sehr gemütlich. Das Licht war abends gedämmt und der Teppichboden schluckte fast jedes Geräusch. Auf den niedrigen Tischen in der Mitte des Raums lagen allerlei Magazine und Zeitungen aus. Einige davon gehörten sogar zu den guten.

\par

Laura betrat die Küche und ging zielstrebig zum Wasserkocher. Sie hätte auch die Kombimaschine verwenden können, die Kaffee, Tee und Suppen kochen konnte. Aber irgendwie widerstrebte ihr die Massenware großer Nahrungsmittelfirmen. Es gab einen Teeladen in Zermatt, in den sie regelmäßig ging, um sich mit ausgefalleneren Varianten einzudecken. Sie entschied sich für einen Kräutertee von Chi Capella Primus. Einem der wenigen Planeten der Union auf dessen Planeten es genießbare Vegetation gab.

\par

Das Aroma stieg Laura wohlig in die Nase, als sie heißes Wasser durch das Sieb mit dem Teesatz goss. Yoshi Tagori kam durch die Tür, als sie sich einen Löffel suchte, um ihr Getränk umzurühren. Er tat so, als wäre er überrascht und sagte: \WR{Es tut mir leid, Laura. Ich wollte nicht stören.}

\par

Sie schenkte ihm ein Lächeln. Seit der Akademie hatte er ein Auge auf sie geworfen und ließ kaum eine Gelegenheit aus, um alleine mit ihr zu reden. Auch wenn er es nicht zugab, fand sie ihn leicht zu durchschauen. \WR{Kein Problem. Du störst nicht.}

\par

Yoshi fand schnell eine Entschuldigung in die Küche gekommen zu sein. Er suchte sich ein Glas und ließ es mit Leitungswasser volllaufen. Laura war nicht gerade in der Stimmung für eine Unterhaltung. Aber ihren alten Kommilitonen wollte sie auch nicht einfach fortschicken. Also fragte sie ihn: \WR{Yoshi, kann ich mir dein Buch mal für eine Weile ausborgen?}

\par

\WR{Klar}, antwortete er ihr. \WR{Was ist mit deinem passiert?}

\par

Lauras Gehirn fand schnell eine Lüge. \WR{Ist mir heute morgen ins Waschbecken gefallen. Wasserdicht sind diese Dinger bis heute nicht.}

\par

Yoshi reichte ihr sein Buch. Ein hervorragendes Modell aus der freien Wirtschaft, an dem Laura nur störte, dass es ihr zu groß war. Sie dankte Yoshi und ging in einen der Waschräume. Die Toiletten der meisten Amtsgebäude auf der Erde waren hervorragend ausgestattet. So war jede Kabine ein eigenständiger, schallisolierter Raum.

\par

Laura meldete sich mithilfe des Handcomputers bei Sermo an, der staatlichen Plattform für Kommunikation jeder Art. Aus ihrem Adressbuch wählte sie ihre Schwester und ihren Neffen an und begann mit der Aufzeichnung eines DDV-Videos. Ihre eigenes Buch hätte das nicht gekonnt, aber für Yoshis Modell war das kein Problem.

\par

Angestrengt versuchte sie sich vorzustellen, sie hätte Marcello und Linda vor sich, anstatt in die unsichtbare Linse Buches zu starren, dessen Seiten sich mit intelligenter Tinte selbst beschrieben. Sie hätte einiges darum gegeben, bei ihren Verwandten sein zu können.

\par

\WR{Hallo}, fing sie etwas unbeholfen mit der Nachricht an. \WR{Ich habe gehört, was passiert ist. Es läuft überall in den Nachrichten. Es tut mir so leid. Ich hoffe, euch geht es gut. Ich war so froh, als mir das Bürgeramt bestätigt hat, dass ihr es geschafft habt. So viele hatten nicht das Glück.} Sie seufzte. Wären ihr nun Marcello und Linda gegenübergestanden, hätte sie keine Probleme damit gehabt, ihre Gefühle in Worte zu fassen. So bat sie nur: \WR{Linda, drück Marc von mir. Denk dran, deine Schwester denkt an dich. Ich hoffe, wir sehen uns bald.} Sie hoffte und war sich doch nicht so sicher. Selbst, wenn alles gut laufen würde, wäre es längst nicht klar, dass sie irgend jemand, den sie kannte, bald wiedersehen würde.

\par

Dann schickte sie die Nachricht ab und ging in die Messe zurück. Dort wartete Yoshi bereits auf sein Buch. Er ließ es täglich kaum aus der Hand. Wenn ein Hacker darauf jemals Zugriff bekäme, wüsste er wohl alles über den Agenten und seine Arbeit, das es zu wissen gab.

\par

Dann bekam Laura einen Einfall. Wie bei den meisten guten Ideen, die sie gehabt hatte, wusste sie auch  diesmal nicht so genau, woher er gekommen war. Klaus nahm sie kaum wahr, als sie in ihre Büro zurückkehrte. Er sah sich nach wie vor die Berichte von Personen an, die den Hacker in Sucre gesehen haben wollten.

\par

\WR{Glaubst du, einer von denen hat ihn wirklich gesehen?}, fragte Laura beiläufig.

\par

Klaus schüttelte den Kopf, ohne sie anzusehen. \WR{Und wenn schon. Der Blödmann ist längst weg. Ich glaube, er sitzt jetzt schon in einem Raumschiff in die autonomen Welten.}

\par

\denken{Oder ich greife ihn mir heute Abend}, dachte sich die Agentin. \denken{Und dann bekommt er meinen Frust der letzten vier Monate ab.}

\par

Laura setzte sich hinter ihren Schreibtisch und öffnete die unterste Schublade. Darin fand sie ein Geschenk, dass sie vor zwei Jahren von ihrer Freundin bekommen hatte. Ein in Leder gebundenes, echtes, nicht digitales Buch mit karierten Seiten und einem einfachen Kugelschreiber. In Zeiten der Union gehörten solche Utensilien zur Vergangenheit. Ihre Freundin hatte sie an die Detektivarbeit vor ein paar hundert Jahren erinnern wollen. Als nicht jede Beobachtung direkt über einen Computer in die Kriminaldatenbank des Konglomerats floss, sondern noch auf Papier aufgeschrieben wurde. Das Buch war eine Art Antiquität, nur dass es nicht wertvoll war.

\par

Laura konnte sich beim besten Willen nicht vorstellen, dass der Hacker in der Lage sein sollte, herauszufinden, was sie dort hineinschrieb. Also warf sie ihrem Partner einen verstohlenen Blick zu und notierte in kurzen Worten, wo sie hinwollte und wen sie dort erwartete, als ihr klar wurde, dass Klaus seine Augen nicht vom Bildschirm ließ. Falls ihr tatsächlich etwas passieren sollte~-- was ein Opfer darstellte, dass sie gerne bringen würde~-- dann hätte man zumindest eine Spur.

\par

Laura verbrachte den Rest der späten Stunden damit, in der Krypta nach einem neuen Buch zu suchen. Ihr altes hatte sie vom Geheimdienst gestellt bekommen und es war selbst für ihre Ansprüche ein wenig zu sparsam ausgestattet gewesen. Ein schwarzes, in Kunstleder gebundenes Modell von Riethink gefiel ihr am besten. Mit ein wenig von ihrem Gesparten, müsste sie es sich leisten können. Mittlerweile fragte sie sich sowieso jeden Morgen, worauf sie eigentlich noch sparte.

\par

Schließlich fiel ihr Blick auf die Uhr. Es war schon fast zwölf Uhr. Um ihr Rendezvous noch einhalten zu können, musste sie sich beeilen. Sie gähnte und stand auf. \WR{Geisterstunde}, sagte sie, an Klaus gewandt und warf sich ihren schwarzen Mantel über. \WR{Ich gehe jetzt nach Hause. Wir sehen uns morgen.}

\par

\WR{Ich geh heute Abend noch mit Karl Bowlen und ein bisschen was trinken. Willst du mit?}, fragte ihr Partner.

\par

Doch Laura schüttelte nur den Kopf. \WR{Ich hatte heute zu viel Stress um noch Spaß haben zu können. Vielleicht am Wochenende. Sei morgen bloß fit!}, mahnte sie. Doch sie wusste, dass Klaus es sein würde. Seine Fähigkeit bis spät in die Nacht zu feiern und am nächsten Tag trotz Schlafentzugs so pünktlich wie wach im Büro zu sein erstaunte sie immer wieder.

\par

\WR{Du kennst mich doch}, antwortete er und wünschte ihr eine gute Nacht.

\par

Im Eingangsbereich wandte sich Laura an den Nachtwächter. Ihre Schritte auf dem Granitboden hallten laut wieder und kündigten bereits ihr Kommen an. \WR{Samir, ich nehme heute meine Pistole mit}, warnte sie den Sicherheitsmann vor. \WR{Vielleicht sehe ich mir morgen früh etwas an und ich möchte nicht extra hier vorbeikommen um sie zu holen.}

\par

\WR{Ist gut}, versicherte ihr der Nachtwächter und machte eine Notiz in der Waffenbestandsakte. Bei einem neuen Mitarbeiter hätte er vielleicht noch ein paar Kontrollfragen gestellt. Aber Laura kannte er schon lange genug, um zu wissen, dass sie kein Blutbad mir ihrer Dienstwaffe anrichten wollte.

\par

\WR{Gute Nacht!}, wünschte sie dem Mann, als sie durch den Metalldetektor trat. Er meldete lediglich ihre Strahlenwaffe und ließ sie passieren.

\par

Der Nachtwächter erwiderte ihren Gruß, als sie auf die kalte und verschneite Straße vor dem Geheimdienstzentrum trat. Der frische Schnee knirschte unter ihren Füßen, als sie sich auf die Suche nach der nächsten Straßenbahn machte. Die Dunkelheit beunruhigte sie. Nicht nur, wegen dem, was sie vorhatte. Sondern auch, weil sie für Laura ein Sinnbild für die Unkontrollierbarkeit des Lebens war. Wenn es Nacht war, konnte man noch weniger wissen, was sich hinter der nächsten Ecke verbarg, als bei Tageslicht. So entstand zumindest kein Gefühl falscher Sicherheit.

\par

Laura stieg in die nächstbeste Bahn ein, die in die richtige Richtung fuhr. Sie fror trotzt ihres dicken Mantels und war froh, endlich in den warmen Wagon einsteigen zu können. Der Fahrgastbetrieb hielt sich erstaunlicherweise in Grenzen. Normalerweise waren die Nahverkehrsmittel selbst in den späten Abendstunden noch voll ausgelastet.

\par

Einer der schlichten Sitze kam Laura gerade recht. Sie hielt ihre Füße vor einen Heizungsschlitz und ließ die Warme Luft ihre nassem Schuhe umströmen. Als sie die Augen schloss, sah sie den Stadtplan Oslos vor sich, den sie sich am Nachmittag eingeprägt hatte. Kurz nachdem die Stimme des Fahrtcomputers die Station angesagt hatte, auf die sie wartete, stieg sie widerstrebend aus.

\par

Die kalte Luft vom Meer schlug ihr sofort entgegen, als sie in den Schneefall heraustrat. Die Straße, die ihr der Hacker genannt hatte, lag in der Nähe des Hafens. Der Klang der ankommenden Wellen war schnell wieder zu hören, nachdem sich die brummende Straßenbahn entfernt hatte. Auf den Straßen war kein Mensch zu sehen. Die ganze Gegend wirkte völlig verlassen. Sie war nicht ganz so dicht besiedelt, wie die Innenbezirke der Stadt. Auch gab es keine Läden oder Restaurants. Nur Wohnhäuser voller schlafender Menschen.

\par

Laura wünschte sich nun langsam auch ins Bett. Der Tag war anstrengend gewesen, nicht nur wegen der Angst um ihren Neffen. Auch weil sie zweimal um den halben Globus gefahren war.

\par

Der Anblick einer dunklen Silhouette ließ sie aufschrecken und herumfahren. Sie hatte gar nicht bemerkt, dass sie bereits am verabredeten Treffpunkt angekommen war. Ein großer Mann in einem dunklen Mantel stand plötzlich hinter ihr. Ihre Hand ging reflexartig in Richtung ihrer Manteltasche. Der Hacker hatte nichts davon gesagt, dass sie unbewaffnet hätte kommen sollen.

\par

Doch der Mann, der ihr gegenüberstand, hob sofort die Hände. \WR{Keine Waffen sind notwendig, Frau Gethas}, sagte er leise.

\par

Laura zog die Pistole dennoch aus ihrem Halfter. Vorsichtig blickte sie sich um. Sie stand in einer kleinen Gasse zwischen zwei gewöhnlichen Wohnhäusern. In ihren verschneiten Gärten war nichts besonderes zu erkennen. Doch Laura wusste, dass das nichts heißen musste. Sie wünschte sich in diesem Moment eine der Wärmebildkameras der Phalanx.

\par

\WR{Ich habe mich an unsere Abmachung gehalten}, sprach sie. \WR{Niemand weiß, dass ich hier bin. Jetzt nehmen Sie die Hände hoch. Ich verhafte sie wegen Datendiebstahl in mehr als dutzend zwei Fällen, Widerstand gegen das Konglomerat und Sachbeschädigung.}

\par

Der Mann lachte kurz auf und lehnte sich an einen Gartenzaun. Laura bemühte sich, sein Gesicht zu erkennen, doch es war zu dunkel und eine Kapuze verdeckte es fast komplett. \WR{Langsam, Frau Agentin. Ich bin doch hier um zu beichten.}

\par

\WR{Ich weiß schon mehr über Sie, als Sie sich vielleicht denken}, kündigte Laura an und griff nach ihren Magnetschellen. \WR{Ich weiß, dass sie für die Agentur für verdeckte Ermittlung gearbeitet haben~-- oder es noch tun.}

\par

Der Mann blieb ruhig und das erste mal an diesem Tag hatte Laura das Gefühl, dass etwas gut lief. \WR{Ich habe es mir schon gedacht, als wir keine Filmaufnahmen von Ihnen gefunden haben. Agenten der Verdeckten werden automatisch von der Bildverfolgung ausgenommen. Und dann ihre Fingerabdrücke. Es ist bei Ihrem Verein Usus, sie verändern oder entfernen zu lassen. Ich habe langsam die Nase voll von Ihrer Abteilung.}

\par

Die Agentur für verdeckte Ermittlung genoss keinen guten Ruf in der Union. Laura kannte viele Thriller aus dem zwanzigsten Jahrhundert. Während sie im FBI, das in ihnen oft vorkam, ihre eigene Abteilung wiedererkannte, erinnerten sie die Verdeckten eher an die CIA. Eine kleine, trotz der gegenseitigen Überwachung, gut abgeschirmte Einheit, die sehr viel mehr tat, als sie zugab.

\par

Der Mann reagierte nicht und Laura wurde allmählich nervös. Sie hob ihre Pistole drohend an und stützte ihre Waffenhand mit der anderen ab. Ihre Finger wurden langsam steif vor Kälte und sie bereute es, keine Handschuhe angezogen zu haben.

\par

Dann wurde ihr noch mehr klar und sie seufzte enttäuscht. \WR{Sie sind nicht der Hacker, stimmt das nicht? Sonst hätten Sie keine Hemmungen, mir Ihr Gesicht zu zeigen. Ich weiß, wie er aussieht. Und sie haben keine Ahnung, wovon ich rede.}

\par

Der Mann trat einen Schritt vor und Laura schreckte sofort zurück. Als er stehen blieb, beruhigte sie sich wieder. Schließlich sagte der Vermummte: \WR{Sie haben Recht. Der, den Sie suchen ist in diesem Augenblick ziemlich weit weg. Aber in einer Sache haben Sie sich getäuscht. Ich weiß, dass mein Partner bei der V-Agentur war.}

\par

\WR{Ich hatte eine Vereinbarung mit ihm}, fauchte die Agentin. \WR{Ich habe meinen Teil erfüllt. Falls Sie glauben, dass ich noch mal auf Ihren Blödsinn reinfalle, haben Sie sich getäuscht. Ich nehme Sie jetzt fest.}

\par

\WR{Dann verlieren Sie aber das hier!}, warnte der Mann und zog vorsichtig einen älteren Handcomputer aus seiner Jackentasche. \WR{Es ist mit einem Code versehen. Darauf ist einiges, das sie interessieren dürfte. Wenn Sie mir Ihr Ehrenwort geben, dass Sie mich gehen lassen, dann bekommen Sie ihn.}

\par

Laura schnaubte verächtlich. \WR{Ich glaube Ihnen kein Wort mehr. Vielleicht hätte ich, wenn Ihr \Wr{Partner} persönlich aufgetaucht wäre, so wie vereinbart. Aber so haben Sie nicht gerade mein Vertrauen verdient. Geben Sie mir jetzt Ihre Hände.}

\par

Doch der Mann tat nichts dergleichen. \WR{Wir mussten sicher sein, ob Sie wirklich die sind, die wir erwartet haben. Und ob Sie vertrauenswürdig sind und tatsächlich niemandem etwas verraten haben.}

\par

\WR{Ich glaube}, begann Laura, \WR{dass sie ein zwei Gauner sind, die ein bisschen Geld durch Datenklau machen. Nichts weiter. Woher wollen Sie wissen, dass ich niemandem etwas erzählt habe? Sagen Sie mir nicht, sie hätten sich in die Überwachungssysteme meiner Arbeitsstelle eingeklinkt. Damit würden Sie, wie man so schön sagt, den Vogel abschießen.}

\par

Der Vermummt entriegelte den Handcomputer und lud ein Programm. Die Anzeige bestand im Wesentlichen aus fünf Punkten. \WR{Bis jetzt, wissen wir nicht, ob sie jemandem von heute Abend erzählt haben oder nicht. Das hier ist ein Lügendetektor. Nicht einhundertprozentig sicher, aber sehr zuverlässig.}

\par

\WR{Und Sie glauben, den mache ich mit?} Lauras Ton wurde spöttisch.

\par

\WR{Wenn ja, dann werde ich Ihnen eine Kostprobe von dem geben, was mein Partner Ihnen erzählen kann}, versprach der Mann. \WR{Wir wollen Sie nicht hereinlegen. Wir sind nur vorsichtig. Wenn Sie noch ein bisschen mitspielen, dann werden Sie nicht mit leeren Händen gehen müssen. Wenn Sie mich festnehmen, sage ich kein Wort.} Er legte den Handcomputer auf eine nahe Säule, nachdem er behutsam den Schnee davon herunter gewischt hatte. \WR{Es wird Ihre Stimmlage und Ihre Gestik auswerten, während ich Ihnen Fragen stelle. Ihre Waffe können Sie weiterhin auf mich richten.}

\par

Laura kämpfte mit sich selbst. Es schien nicht, als hätte sie viel zu verlieren. Aber so hatte es meistens ausgesehen, bevor sie wirklich etwas verloren hatte. Vorsichtig, ohne ihre Pistole sinken zu lassen, nahm sie das Gerät in Augenschein. Es war ein alter und recht einfacher Handcomputer mit einer simplen Glasoberfläche. Anzeichen für einen Mechanismus der ihr schaden hätte können, fand sie nicht. Sie war darin ausgebildet worden, Fallen zu erkennen und es gehörte zu ihren speziellen Talenten. Trotzdem war ihr klar, dass ein Risiko blieb.

\par

\WR{Na gut}, antwortete Laura. \WR{So lange Sie nicht wollen, dass ich das Ding anfasse.}

\par

\WR{Sind Sie Laura Gethas?}, fragte der Mann sofort.

\par

\WR{Ja}

\par

\WR{ Geboren vierdin siebendutzen sieben. Tag eindin siebendutzend neun.}

\par

\WR{Nein. Tag siebendutzend und acht.}

\par

\WR{Arbeiten Sie für den Geheimdienst des Konglomerats?}

\par

\WR{Ja. In der Argus-Abteilung.}

\par

\WR{Sind Sie eine Verräterin?}

\par

Laura stockte. Dann erwiderte sie entschieden: \WR{Nein!}

\par

\WR{Wurden Sie in den letzten zehn Monaten wegen einer Erkrankung des Gehirns, irgendeiner Art behandelt?}

\par

\WR{Nein.}

\par

Die letzte Frage formulierte der Mann sehr langsam und bedächtig. \WR{Haben Sie mit jemandem über dieses Treffen gesprochen, bevor Sie hierher gekommen sind?}

\par

\WR{Nein}, entgegnete ihm Agentin Gethas und musste dafür nicht einmal lügen. Sie hatte lediglich eine kleine Notiz in ihrem Schreibtisch zurückgelassen.

\par

Der Mann nickte schließlich und strich sich die Kapuze vom Kopf. Laura war überrascht, was sie sah. Die Person, die sich ihr zeigte, war noch nicht sehr alt. In der Mitte seiner zwanziger, wie sie schätzte. Er hatte violette Haare und seltsam hellblaue Augen. Laura vermutete dahinter eine Mutation durch die Seuche, wie sie immer noch vorkamen.

\par

\WR{Ich wusste, dass Sie vertrauenswürdig sind}, sagte der Unbekannte. \WR{Ich habe Sie als Kontaktperson ausgesucht.}

\par

Laura tat einige vorsichtige Schritte rückwärts, um dem Mann keine Angriffsmöglichkeit zu geben. Der Schnee gab nur langsam unter ihren Füßen nach und sie spürte, dass ihre Haare vollkommen zugeschneit sein mussten. \WR{Jetzt sind Sie dran}, begann die Agentin. \WR{Was haben Sie mir zu sagen.}

\par

Der Unbekannte beantwortete ihre Frage bereitwillig: \WR{Der Mann, den Sie suchen hat bei der Agentur für verdeckte Ermittlungen gearbeitet. Seine früheren Vergehen, bei denen es hauptsächlich um die Beschaffung von Geld ging, waren nur inszeniert. Er war ein Köder, um große Auftraggeber im Hackergeschäft anzulocken. Ich gehöre zur Polizei. Ich wurde ihm zugeteilt, um ihn zu unterstützen~-- genauso wie Sie auch einen Partner bei den Sicherheitskräften haben.}

\par

Die Augenbrauen der Agentin formten ein V. \WR{Das klingt nach einer normalen, verdeckten Aktion. Warum wurden die anderen Abteilungen des Geheimdienstes nicht darüber informiert? Wenn Sie die Wahrheit sagen, hätte Ihnen das viel Ärger erspart.}

\par

\WR{Weil wir uns mit den falschen Leuten eingelassen haben}, antwortete ihr der Mann. \WR{Vor etwa zwei Monaten kamen ein paar seltsame Auftraggeber zu meinem Partner. Wir haben keine Ahnung, was das für Typen sind. Aber Sie gehören jedenfalls nicht zur Hackerszene. Ihr Auftrag hat uns in die Bredouille gebracht. Auf einmal waren alle hinter uns her. Man hat den Chef meines Partners ersetzt und versucht, ihn umzubringen.}

\par

\WR{Langsam}, forderte die Agentin. \WR{Was war das für ein Auftrag?}

\par

\WR{Es ging um Dienstakten und Versetzungspläne. Mein Partner sollte sich in den Teil der Krypta Scientia hacken, der dem Konglomerat gehört und einige personenbezogene Dokumente stehlen. Eigentlich keine große Sache. Ich meine, was nützen einem schon ein paar hundert Lebensläufe von Soldaten, in denen sowieso nichts brisantes drinsteht. Jeder weiß doch, dass die interessanten Sachen in abgetrennten Datenzentren liegen, die keinen Kontakt zur Krypta haben. Aber die haben meinem Partner eine Menge Geld angeboten. Zwei Pinae Naira Anzahlung plus weitere acht, wenn er den Auftrag zu Ende gebracht hätte.}

\par

\WR{Zehn Pinen}, murmelte Laura erstaunt. Das war wirklich eine Menge Geld für diese Art von Drecksarbeit.

\par

Der Mann grinste. \WR{Oho, Sie können kopfrechnen}, feixte er.

\par

\WR{Nicht frech werden!}, warnte ihn Laura Gethas. \WR{Ich nehme an, er hat den Auftrag angenommen. Was ist dann passiert?}

\par

\WR{Anfangs schien alles super zu laufen. Er hat die zwei Pinae sogar bekommen und beschlossen, gegen die Typen zu ermitteln. Aber dann lief alles aus dem Ruder. Er fand heraus, dass diese Kerle nicht an die Daten interessiert waren. Als er sich in die Krypta gehackt hatte, haben die diese Sicherheitslücke genutzt, um die Dateien zu ersetzen. Und als er das herausfand und mir erzählte, hatte er auf einmal diese Killer am Hals. Und sein Chef~-- der einzige, der etwas davon wusste~-- war auf einmal weg.}

\par

\WR{Killer?}, fragte Laura ungläubig.

\par

\WR{Ja. Und zwar nicht diese Idiotenschläger von der Mafia oder den Piraten}, antwortete ihr der Mann und die Agentin glaubte, Angst in seinen Augen erkennen zu können. \WR{Das waren Profis. Ich hab in meiner Karriere schon einiges erlebt. Aber das war echt heftig. Mein Partner hat mir das Leben gerettet. Es gab eine Schießerei in Neu Karlsruhe auf dem Mars. Seit dem sind wir untergetaucht und suchen jemanden, dem wir vertrauen können.}

\par

\WR{Und wie kamen Sie auf mich?}, fragte Laura verwirrt.

\par

\WR{Ganz einfach. Erstens: Sie haben uns verfolgt. Zweitens: Ihre Dienstakte wurde nicht ersetzt. Glauben Sie mir, wir wissen noch viel mehr. Aber ich kann Ihnen das nicht einfach so hier sagen. Auf diesem Handcomputer}, der Mann griff danach und hielt ihn in Lauras Richtung, \WR{sind Datum und Ort für unser nächstes Treffen vermerkt. Dort stellt er sich. Wenn Sie mir versprechen, mich gehen zu lassen, dann gebe ich ihnen...}

\par

Laura blieb wie vereist stehen. In ihr Gesicht war etwas Blut gespritzt, als der Kopf des Mannes an einer Seite aufplatze und sein Körper lautlos und leblos in den Schnee fiel. Eine Sekunde lang stand sie einfach nur da und beobachtete, wie der Schnee um den Unbekannten herum vom Blut getränkt und geschmolzen wurde.

\par

Heckenschütze, fuhr es ihr durch den Kopf. Sie brauchte nicht nachzudenken. Ihre forensische Ausbildung arbeitete nun unbewusst für sie. Sie ließ sich die letzten Momente des Mannes noch einmal durch den Kopf gehen und rief sich ins Gedächtnis, dass die linke Schläfe des Unbekannten explodiert war. Der Schuss war also von rechts gekommen.

\par

Laura warf sich gegen eine Mauer, von der sie hoffte, dass sie ihr vor dem Schützen Deckung bot. Dann flogen ihr ohne Vorwarnung elektrische Einzelteile um die Ohren. Sie begriff sofort. Der Schütze hatte auf den Handcomputer geschossen, den ihr der Unbekannte hatte geben wollen. Das machte seine Geschichte um einige glaubhafter.

\par

Im Schnee glaubte die Agentin die Überreste des Projektils erkennen zu können. Das lange Geschoss und der Zustand in dem sich die Leiche des Unbekannten befand, ließen sie wissen, dass der Attentäter ein militärisches Scharfschützengewehr verwendete.

\par

Wenn sie ihren Kopf nur einen Millimeter über die Mauer bewegte, wäre sie tot, das war ihr klar. Aber sitzen bleiben konnte sie auch nicht, denn solche Munition könnte unter Umständen auch Beton durchbrechen. Dann bot ihr auch die Mauer keinen Schutz.

\par

Und in dieser Lage wurde ihr klar, wie sehr sie von Technologie abhing. Mit ihrem eigenen Buch hätte sie Hilfe rufen können. Aber den hatte sie selbst zertreten. Um Hilfe schreien wollte sie auf keinen Fall. Selbst wenn jemand wach wurde, und sie sah, begab er sich vielleicht aus Versehen ins Schussfeld des Attentäters.

\par

Ihr Blick suchte die Umgebung ab. Doch sie fand keine Möglichkeit zur Flucht. Die Mauer endete ein paar Meter weiter. In den Gärten der umliegenden Häuser war es zwar dunkel, doch die Agentin zweifelte nicht daran, dass der Attentäter einen Wärmebildsucher verwendete.

\par

Hätte sie sie es nicht besser gewusst, hätte Laura denken können, die ganze Szenerie wäre vollkommen ruhig. Der Schnee fiel geräuschlos auf den Boden und ein Baum in der Nähe neigte sich im leichten Wind.

\par

Ein entkommen war unwahrscheinlich, selbst, wenn sie versuchen würde, im Zickzack zu rennen. Die Behauptung des jungen Mannes, es mit professionellen Auftragskillern zu tun zu haben, war soeben deutlich glaubhafter geworden. Und ein solcher hätte keine Schwierigkeiten, eine zwar gut trainierte, aber aus der Übung gekommene Agentin zu treffen, selbst wenn sie geschickt floh.

\par

Ihr nächster Gedanke war ihre Strahlenwaffe. Die oragefarbenen Entladungen würden gut sichtbar sein und sofort die Polizei auf den Plan rufen. Aber dann wären die Beamten, die zu ihrer Rettung geeilt kämen, vermutlich das nächste Ziel des Heckenschützen und das war etwas, das Laura nicht verantworten wollte. An ihren Händen klebte ihr bereits zu viel Blut.

\par

Sie konnte Christine Bell praktisch vor sich liegen sehen. Mit friedlich geschlossenen Augen, so als sei sie bloß eingeschlafen, statt qualvoll erstickt zu sein. Sie war damals nicht schnell genug gewesen und es sah ihr sehr nach ihrem Schicksal aus, auch jetzt nicht schnell genug sein zu können. Mittlerweile roch sie den eisernen Geschmack von Blut, das aus dem Schädel des Unbekannten strömte, doch es bereitete ihr keine Übelkeit mehr.

\par

Sie hatte nicht gedacht, dass ihre Mattigkeit in einer Lage wie dieser zurückkehren würde. Vielleicht war es die konzentrierte Hoffnungslosigkeit, die sich sonst subtiler durch ihr Leben zog. Es war vorbei, dass wusste sie, seitdem sie Christine Bells leblosen Körper aus dem vergrabenen Sarg geholt hatte. Der Tod dieses Kindes war genauso ihr eigener gewesen. Man verlangte von ihr, dass sie weitermachte. Aus selbstsüchtigen Gründen. Damit ihre Familie und ihre Freunde nicht trauern mussten, damit man bei der Arbeit nach wie vor auf ihre Fähigkeiten zurückgreifen konnte, damit sie vielleicht einmal einen Partner glücklich machen konnte.

\par

Sie konnte nicht mehr. Jeder Muskel in ihrem Körper, jede Zelle schien sich gegen einen Versuch des Weiterkämpfens zu wehren und sie wunderte sich, warum ihr Herz eigentlich noch schlug. Es gab nur eine Bewegung, zu der sie noch genügend Kraft hatte. Also stand sie einfach auf.
\ortswechsel
Der Attentäter traute seinen Augen kaum. Die Agentin stand da und sah in seine Richtung. Er war zu weit weg. Sie konnte ihn nicht erkennen, das wusste er. Und doch lies ihn der Ausdruck in ihren Augen glauben, sie könne ihn sehen. Sie hatte definitv keine Angst. Würde er nun abdrücken, wäre das ihr Ende. Das wusste er und das wusste sie. Warum versuchte sie nicht, wegzurennen oder rief nach Hilfe? Natürlich hatte sie keine wirklich sinnvolle Option. Doch jeder andere Mensch, den der Mörder bereits getötet hatte, hatte zumindest versucht, sich zu wehren, oder zu retten. Ob das Opfer nun wie wild an dem Klavierdraht gerissen hätte, der sich um seine Kehle geschlungen hatte, oder vor einem sicheren Schuss davon gelaufen war. Eines hatten alle gemein. Der Überlebensinstinkt ließ sich nicht ausschalten.

\par

Der Mörder drehte die Restlichtverstärkung seiner Waffe auf und schaute erneut durch die Zielhilfe seines Gewehrs.

\par

Der helle Kontrast die Leiche des Unbekannten deutlich erkennbar werden. Früher oder später würde eine Polizeieinheit nach dem rechten sehen. Er hätte die Ahnungslosen Sicherheitsmänner relativ leicht ausschalten können, allerdings würden auch diese bald vermisst werden. Dann wäre es fast unmöglich die Agentin noch unauffällig aus dem Verkehr zu ziehen. Er musste es also jetzt tun.

\par

Sein Finger spannte sich um den Abzug. Nach wie vor sah sie ihm entgegen. Sie schien zu wissen, woher der Schuss gekommen war und in welchem verlassenen Haus er sich aufhielt. Das Abdrücken viel im schwerer, als er erwartet hätte. Etwas bremste ihn. Die Hilflosigkeit seines Ziels hatte ihn noch nie gestört. Vielleicht war es die Art, wie sie sich ihrem Schicksal ergab.

\par

Dann heulte mit einem mal ein Alarm auf und jede einzelne Straßenlaterne erstrahlte nunmehr mit voller Leistung. Obwohl er eine schallgedämpfte Waffe verwendete und es mitten in der Nacht war, musste jemand die Lage erkannt und Hilfe gerufen haben.

\par

Zeitgründe ließen ihn den Rückzug antreten. Er würde ein paar Minuten brauchen, um seine Ausrüstung zusammenzupacken. Und wenn die Polizei schlau genug war, mithilfe des Einschusswinkels auf den Handcomputer seine ungefähre Position zu raten, dann müsste er vielleicht schnell fliehen. Und das würde gefährlich werden. Darum begann er zügig aber nicht gehetzt, sein Gewehr abzubauen.
\ortswechsel
Laura schrie sofort als sie das Hovercraft der Polizei sah: \WR{Heckenschütze! Hier ist ein Heckenschütze!} Beim zweitem Mal verstand einer der beiden, was sie ihnen zurief und gab eine Meldung über das Kommunikationssystem des Fahrzeugs durch. Sein Partner parkte es hinter einem Haus, so dass es vorerst geschützt sein sollte.

\par

Einer der beiden Polizisten stieg aus und rief ihr aus seiner Deckung heraus zu: \WR{Wissen Sie, aus welcher Richtung er geschossen hat?}

\par

\WR{Vom Hafen aus!}, brüllte Laura so laut sie konnte. \WR{Passen Sie auf. Er benutzt eine Waffe des Militärs.}

\par

In der Union gab es nur zwei Möglichkeiten, an Waffen zu kommen. Man besorgte sie sich aus den Beständen des Konglomerats oder man kaufte sie von Piraten, die sie sich ihrerseits aus Resten des Routenkriegs bedienten. Diese waren zwar alt, doch deswegen nicht unbrauchbar. Die gesamte Raumflotte der autonomen Welten bestand im wesentlichen aus alten Schiffen des Routenrkiegs.

\par

Es kam Laura wie eine Ewigkeit vor, bis endlich die Spezialeinheit der Polizei eingetroffen war. Tatsächlich hatte es bloß einige Minuten gedauert. Aus dem gepanzerten kastenförmigen Hovercraft sprangen zwei Männer, die drei graue Schilde bei sich trugen. Diese waren mehr als zwei Meter hoch und speziell zum Schutz vor Scharfschützen entwickelt worden. Sie bestanden aus drei Schichten. Die ersten würde bei einem Treffer leicht durchschlagen werden. Doch dabei würde sie das Projektil stark abbremsen, damit es sich im Idealfall in der zweiten Schicht verfing. Eine Art Vorhang aus dünnen aber stabilen Seilen. Würde es die Kugel auch dort hindurch schaffen, würde sie durch den Energieverlust spätestens in der dicken dritten Schicht stecken bleiben.

\par

Die beiden Elitepolizisten hielten die Schilde stets so, dass sie ihren ganzen Körper verdeckten und arbeiteten sich vorsichtig zu Laura vor. An der Mauer angelangt, stellten sie das dritte Schild ab.

\par

\WR{Können Sie das halten?}, fragte einer der Polizisten.

\par

Laura nickte. Sie war auf der Akademie ausgebildet worden, solche Schilde zu benutzen. Vorsichtig ergriff sie es und folgte im Rückwärtsgang den beiden Männern. Als sie sich im Schutz des nahen Hauses befand, legte sie das Schild beiseite und griff nach ihrem Dienstausweis.

\par

\WR{Ich bin Vizeberaterin Laura Gethas}, stellte sie sich dem ranghöchsten Polizisten vor. \WR{Ich habe mich hier mit einem Verdächtigen getroffen. Sein Leichnam liegt dort unten. Ein Scharfschütze hat vom Hafen aus auf uns geschossen.}

\par

\WR{Unser Eingreiftrupp ist bereits auf der Suche, meine Dame}, versicherte einer der Polizisten mit den Schildern.

\par

Laura nickte, doch sie ahnte bereits, dass man den Schützen nicht finden würde. Er hatte nicht mehr versucht, auf sie zu feuern. Wahrscheinlich war er längst über alle Berge.

\par

Ein Mann trat ans Fenster seines Hauses und rief irgend etwas unverständliches. Die Angst stand ihm ins Gesicht geschrieben. Laura ging es nicht anders. Jedes mal, wenn sie an einen Freitod gedacht hatte, hatte sie sich gefragt, ob sie wirklich den Mut dazu aufbringen konnte, es zu tun. Nun wusste sie es. Das sie vielleicht nur mit Glück überlebt hatte, spielte für sie keine Rolle. Als sich ihr Atem beruhigte, begann sie, Enttäuschung zu fühlen.

\par

Doch die verschwand kurz darauf wieder. Ihr Vorgesetzter, Berater O'Shea kam die Straße entlang gerannt. Er wirkte aufgebracht und gleichzeitig beunruhigt, das konnte Laura bereits von weitem sehen. Sie würde ihm nun alles erklären müssen und da sie nach der verpatzten Aktion in Freiburg bereits zum zweiten Mal versagt hatte, würde ihre Geschichte ihm nicht gefallen.