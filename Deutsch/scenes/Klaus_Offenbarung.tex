Klaus zog legte den Hitzeschneider auf den Boden. Das kleine Gerät war noch bei weitem zu heiß, um wieder in die Jackentasche gesteckt zu werden, aus welcher er es gerade gezogen hatte. Unter mäßigem Scheppern fiel ein kleiner Teil der Plexiglasscheibe auf den Boden.

\par

Vorsichtig schob Klaus zunächst seine Beide und dann den Rest seines Körpers an den geschmolzenen Schnitträndern vorbei, die vermutlich ebenfalls noch eine nennenswerte Restwärme aufwiesen.

\par

Kaum stand er wieder aufrecht, brachte er sein Gewehr in Anschlag. Doch bevor er Laura in das Treppenhaus folgte, entschloss er sich, die Steuerung der Falle genauer in Augenschein zu nehmen. Wie seine Partnerin gesagt hatte, fand er einen Handcomputer und eine Batterie. Und wie er bereits befürchtet hatte, fehlte von einer Sprengkapsel jede Spur. Nur zur letzten Gewissheit zog er die beiden Drähte ab und sah zu, wie sich die Plexiglasschotten wieder öffneten.
\ortswechsel
\WR{Wenn Sie wussten, dass das hier nur eine Falle ist, wieso sind Sie dann hier?}, fragte Laura, wenig überzeugt von den Behauptungen des Mannes, auf den sie immer noch mit ihrer Waffe zielte.

\par

\WR{Damit ich mit Ihnen reden kann, natürlich.} Der Hacker klang müde. \WR{Ich wollte das lieber zu meinen Bedingungen durchziehen. Aber nach der Sache in Oslo musste ich andere Alternativen in Betracht ziehen. Dass mich früher oder später jemand findet, war klar. Dann sollten es wenigstens Sie sein.}

\par

Lauras Augen verengten sich zu kleinen Schlitzen. \WR{Ihr Freund in Oslo sagte, ich sei vertrauenswürdig. Bedeutet dass, das sie nicht glaube, dass ich zu dieser… \Wr{Sektion} gehöre?}

\par

\WR{Nicht so sarkastisch, bitte}, erwiderte der Hacker und nahm auf einem der Stühle platz. \WR{Ich wünschte, ich hätte Ihnen früher vertraut. Mikhaels Tod hätte sich vermeiden lassen.}

\par

\WR{Allerdings}, blaffte Laura. \WR{Sagen Sie mir jetzt nicht, dass er für Ihre verdrehte Verschwörungstheorie gestorben ist.}

\par

Der Hacker wirkte mit einem mal völlig locker. Dass er seine Füße auf dem nächsten Tisch ablegte, hätte noch als einziges gefehlt. \WR{Er war Teil dieser Verschwörung~-- genauso wie ich. Ich weiß, dass das, was Sie als \Wr{Sektion} bezeichnen, real ist, denn ich habe dafür gearbeitet?}

\par

Lauras Griff um ihre Waffe lockerte sich ein wenig. Dennoch hielt sie die Strahlenpistole weiter auf den Kopf ihres Gesprächspartners gerichtet.

\par

\WR{Und ich bin mir ziemlich sicher}, fuhr der Hacker fort. \WR{Dass mich der Term nicht mehr lange leben lässt. Aber vorher werde ich Ihnen sagen, was ich weiß. Gesetz dem Fall, sie wollen es hören. Denn wenn sie erst wissen, was ich Ihnen sagen will, dann ist es zu spät. Dann wird man auch Sie aufs Korn nehmen und letzten Endes in irgendeiner Weise aus dem Verkehr ziehen.}

\par

Tatsächlich kostete es Laura einiges an Überwindung, um nicht einfach umzudrehen und die Beine in die Hand zu nehmen. Seit sie in Oslos einem Mord durch einen Scharfschützen mitansehen hatte müssen, war sie zumindest bereit zu glauben, dass der Hacker verfolgt wurde. Und wenn jemand auf ihn Jagd machte, warum dann nicht auch auf eine niedere Vizeberaterin beim Geheimdienst?

\par

\WR{Reden Sie schon!}, forderte sie schließlich, klang dabei aber deutlich weniger entschlossen, als sie gewollt hatte.

\par

\WR{Wir nennen uns \Wr{den Term}. Wir sind wie ein Ausdruck in einer Gleichung. Sie müssen verstehen, die Rechnung um unsere Union geht nicht auf. Jeder Mensch soll frei sein und am besten ohne jede Kontrolle oder Bevormundung leben. Klar, an jeder Ecke hängt mittlerweile eine Kamera und Gnosis und Permutare wissen ständig, wo wir sind, was wir machen und welche Güter wir eintauschen. Aber letzten Endes sollen das doch Geheimnisse dieser beiden Computer bleiben, oder?}

\par

Laura reagierte nicht. In diesem Moment wollte sie einfach nur zuhören.

\par

\WR{Wir haben zahllose Gesetze, die unsere Anonymität und unsere Bürgerrechte schützen. In der Crypta und im echten Leben. Gleichzeitig ist unsere Regierung vielleicht die transparenteste in der Geschichte unserer Welt.

\par

Aber das ganze hat einen Haken. Gesetze schützen nicht nur die Rechtschaffenen, sondern eben auch jeden vom Kleinkriminellen, der Leute ausraubt und dann eine schwere Kindheit vorschiebt, bis hin zum Wirtschaftsverbrecher, der dinae betrügt und sich dann wegen einer Gesetzeslücke mit dem Erstohlenen davonmacht.

\par

Und da kommt der Korrekturterm ins Spiel. Wir sind die Notlösung in einer Gleichung, die schon von Grund auf fehlerhaft ist. Jedes mal, wenn jemand die Spielregeln zu sehr verbiegt oder zu hartnäckig bricht, treten wir auf und eliminieren die Gefahr. Egal mit welchen Mitteln. Was eben gerade nötig ist. Von hintergründigen Manipulationen über Erpressung und Gewaltanwendung.}

\par

\WR{Und Mord}, unterbrach ihn Laura. \WR{Sie haben also zu diesem Verein gehört und jetzt jagt man sie, weil sie ausgestiegen sind. Seit Oslo habe ich mir schon so etwas gedacht.}

\par

Der Hacker lachte kurz und sein verschmitztes Grinsen hätte dabei kaum klischeehafter sein können. \WR{Nein. Ganz im Gegenteil. Um ehrlich zu sein, liebe ich den Term. Wenn sie es wirklich absolut nüchtern betrachten und Dinge wie Wahrheit, Ethik oder Moralvorstellungen als genau die subjektiven Dogmen betrachten, die sie auch sind, dann werden sie feststellen, dass diese Organisation eine gute Idee ist.

\par

Denken Sie doch mal nach. Von wie vielen Fällen wissen Sie, in denen Verbrecher straffrei davongekommen sind, weil sie Zeugen eingeschüchtert oder sich ihnen sogar einfach ganz entledigt haben? Gesetze können genauso gut ein Hindernis sein, wie sie uns dienen. Eines können Sie mir glauben. Die Feinde unserer Demokratie, seien es denn nun Mafiabosse, Radikale oder Terroristen. All diese Leute kennen sich plötzlich sehr gut mit unseren Regeln aus, wenn sie ihnen helfen.

\par

Ich habe ihre Akte gelesen. Der Fall Bell. Sie hatten den Entführer schon. Schauen Sie mir in die Augen und sagen Sie mir, dass unsere Bürokratie \textit{nicht} daran gehindert hat, dieses Mädchen zu retten.}

\par

Laura blieb still. Krampfhaft zwang sie sich, nicht zu schlucken.

\par

\WR{Sie brauchen nicht zu antworten, keine Sorge. Ich weiß, dass sie tief in ihrem Inneren längst verstanden haben, dass unsere heile Welt nur funktionieren kann, wenn jemand den Dreck wegräumt. Und es ist doch besser, wenn sich nur ein paar wenige die Hände richtig schmutzig machen, wenn dafür zahllose Menschen in ihrer ach so hoch entwickelten Kultur friedlich aufleben können. Sie wissen es, hab ich recht? Irgendwer muss die Toiletten putzen.}

\par

Am liebsten hätte Laura sich nun auch einen Stuhl genommen. Sie brachte jedoch all ihre Kraft auf, um aufrecht stehen zu bleiben. Auch wenn sie tatsächlich etwas an dem fand, was ihr der Hacker sagte, kannte sie die Situation in der sie nun steckte, nur zu gut.

\par

\WR{Das reicht. Ich nehme Sie hiermit fest. Derzeit bin ich nicht offiziell für diesen Fall verantwortlich, darum darf ich Sie auch nicht verhören. Ich bringe sie aber auf die nächste Polizeidienststelle, wo Sie von unbefangenen Staatsdienern befragt werden.} Sie hielt einen Moment inne, als sie bemerkte, wie blanke Angst in das Gesicht des Hackers stieg. \WR{Es ist wie Sie gesagt haben. Die Gesetze helfen manchmal auch den Verbrechern. Darum brauchen Sie auch keine Aussage mir gegenüber zu machen.}

\par

\WR{Warten Sie!}, bat der Hacker, mit einem mal deutlich weniger selbstsicher. \WR{Da ist noch mehr, was Sie wissen sollten. Wenn Sie der Union weiterhin vertrauen wollen, bitte. Aber damit schaufeln Sie nicht nur sich selbst, sondern auch dem Rest der Menschheit ein Grab.}

\par

Laura war drauf und dran, die Bemerkung einfach abzutun. Die Geschichte des Mannes, der ihr nun kaum merklich zitternd, gegenübersaß, erschien ihr so hahnebüchen, dass sie sich darüber ärgerte, sie auch nur für einen Moment ernst genommen zu haben. Aber alles, was geschehen war, bewies zumindest, dass mehr hinter dem steckte, was der Hacker sagte.

\par

Dieser fuhr hastig fort: \WR{Die Invasion der Shutek kam nicht so unerwartet, wie sie vielleicht glauben.}

\par

\WR{Was?}, platzte es aus Laura heraus. \WR{Wollen Sie mir jetzt erzählen, dass die Menschen, die auf Pollux gestorben sind und die, die nun in Kreuzpunkt kämpfen, Opfer einer Verschwörung, statt von Außerirdischen geworden sind?}

\par

\WR{Sie wurden Opfer von beidem!}, brüllte der Hacker nun. \WR{Ich weiß nicht, wieso. Und ich weiß auch nicht, warum. Aber der Term ebnet den Shutek den Weg. Glauben Sie etwa, es war Zufall, dass der erste Großangriff stattgefunden hat, als das Endspiel der Weltmeisterschaft war? Zu der sich jeder entbehrliche Soldat freigenommen hat? Das war es, was ich beweisen will. Darum habe ich versucht, mich in gesperrte Bereiche der Crypta einzuklingen. Es geht um Versetzungpläne, Flotten- und Armeebewegungen. Einfach alles!}

\par

\WR{Na gut}, erwiderte Laura. \WR{Wenn das stimmt, dann haben Sie doch sicher Beweise, oder? Es wäre vielleicht klug gewesen, Ihr Buch nicht mit einer Selbstzerstörung auszurüsten. Dann wüssten wir, was Sie glauben zu wissen.}

\par

Der Hacker lachte erneut. Diesmal deutlich hysterischer. \WR{Damit hätte ich dem Term doch nur in die Hände gespielt. Dann wäre klar gewesen, was alles in der Crypta gefälscht werden hätte müssen.}

\par

Laura winkte energisch mit ihrer freien Hand. \WR{Eine Sekunde. Die Crypta Scientia ist sozusagen unser Spiegel der Wirklichkeit. Die kann man nicht so einfach fälschen. Jeder benutzt sie und gestaltet sie mit. Es würde sofort auffallen, wenn dort falsche Daten eingespielt würden.}

\par

\WR{Ach wirklich?}, fragte der Hacker rhetorisch. \WR{Wann hat Ihnen die Crypta denn das letzte mal widersprochen? Wo nehmen wir heute unsere Informationen her? Es gibt zu viele Menschen auf der Welt, als das wir noch wirklich aufeinander hören, wenn wir miteinander reden. Die Crypta ist nicht der Spiegel der Wirklichkeit, sie ist zum größten Teil unsere Wirklichkeit! Wussten Sie, dass dort jahrelang stand, dass es auf Keuzpunkt Primus auch Dinosaurierfossilien gäbe? Jeder hat das geglaubt. Niemand hat es nachgeprüft. Nicht mal die Kreuzpünktchen selbst. Wenn da etwas drinsteht, dann glauben wir das, ohne es zu hinterfragen.

\par

Ich kann Ihnen jetzt von Angesicht zu Angesicht sagen, welche Funde ich gemacht habe. Und glauben Sie mir, dass ist mittlerweile die einzige verlässliche Methode, wie sie noch an Informationen über diese Verschwörung kommen können.}

\par

Laura seufzte. \WR{Na gut. Aber verhaftet sind Sie trotzdem. Schießen Sie los, es ist Ihre Entscheidung.}

\par

Der Hacker holte tatsächlich tief Luft, bevor er begann. \WR{Als erstes kam ich Ihnen auf die Schliche, als Legat Khumalo getötet wurde. Ich bin mir sicher, dass…}

\par

Es war zu schnell und vor allem zu unverhofft passiert, als das Laura realisieren hätte können, was geschehen war. Und dennoch hing der Kopf des Hackers nun in Fetzen und der laute Knall einer Feuerwaffe hallte durch die dunkle Cafeteria. Das Blut des Mannes war bis auf Lauras Mantel gespritzt. Zumindest wusste sie, welche Waffe das angerichtet hatte.

\par

So schnell sie konnte, wirbelte sie herum, doch sie erkannte nur die immer noch schwingende Doppeltür des Saals. Als sie wieder zurückdrehte, fragte sie sich als allererstes, wie es Klaus geschafft hatte, sich unbemerkt so nah an sie heran zu schleichen.

\par

\WR{Alles klar, Laura}, begann er, \WR{bleib jetzt bitte erst mal ganz locker.} Er versuchte wohl, beruhigend zu klingen, doch die Tatsache, dass er Laura mit seiner Schrotflinte und sie ihn mit ihrer Strahlenpistole bedrohte, brachte ihn sichtlich aus der Fassung.

\par

\WR{Ich verspreche dir, dass das nicht so hätte laufen sollen}, fuhrt er fort. \WR{Du bist geschockt und das ist in Ordnung. Es tut mir leid, dass es nicht anders gegangen ist aber ich hab dich gebraucht um an ihn heranzukommen. Ich bin ein guter Attentäter aber ein lausiger Ermittler.}

\par

Laura war über ihre eigene Fassung erstaunt, als sie antwortete: \WR{Du gehörst also zu denen.}

\par

Klaus nickte langsam. \WR{Nicht, dass du das nicht geahnt hättest. Sonst hättest du mich kaum in dieser Falle eingeschlossen um mit ihm hier alleine sein zu können. Ich bin neugierig. Wie bist du auf mich gekommen.}

\par

Erneut festigte sich Lauras Griff um ihre Waffe. \WR{Weil du ein Idiot bist. Woher hast du gewusst, dass ich mich dem Attentäter in Oslo in die Schussbahn gestellt habe? Ich hab alle Akten überprüft. Es gab keine Zeugen. Du warst es. Darum hast du auch nicht auf mich geschossen.}

\par

Ein verschmitztes Lächeln glitt über Klaus Gesicht. Aus irgendeinem Grund wirkte er nun sogar erleichtert. \WR{Dieses kleine Stimmchen in deinem Kopf, weswegen du so gut bist, in dem, was du tust. Hätte mir denken können, dass du es so schnell rausbekommst. Um ehrlich zu sein, habe ich befürchtet, dass du es sogar noch viel früher wissen würdest. Wer außer mir hätte denn das Memo in deinem Schreibtisch finden können, auf dem du Ort und Zeitpunkt deines kleinen Osloer Rendezvous aufgeschrieben hast.}

\par

\WR{Jeder der eine Tür und eine Holzschublade aufbrechen kann, du Trottel}, entgegnete ihm Laura. \WR{Aber es hat dich verdächtig gemacht, das stimmt.}

\par

Klaus ließ seine Schrotflinte zumindest ein kleines Stück weit sinken. \WR{Tja, wie dem auch sei. Ich bin froh, dass wir wieder offen zueinander sein können. Mach dir keine Sorgen. Die Aktion mit der Schleuse nehm ich dir nicht krumm. Du wusstest es ja nicht besser. Aber jetzt leg bitte erst mal deine Waffe weg, dann erklär ich dir alles.}

\par

Laura verzog keine Miene, als sie antwortete: \WR{Du hast sie ja nicht mehr alle.}

\par

\WR{Pass auf}, forderte Klaus, \WR{das geht einfach oder schwer. Wir sind Freunde, auch wenn das im Moment nicht so aussieht. Ich hab dich wirklich gerne, das schwöre ich dir. Aber ich werde dich ausschalten, wenn es denn sein muss.}

\par

\WR{Oder ich knall dich vorher ab}, gab Laura zu bedenken. \WR{Ich werde meine Pistole ganz sicher nicht ablegen. Du legst deine Waffe weg, dann hast du zwei drei Minuten Zeit mir zu sagen, was du mir sagen wolltest~-- ganz im Vertrauen, weil wir Freunde sind~-- und dann bist du verhaftet. Ende der Geschichte.}

\par

Klaus schüttelte den Kopf. \WR{So einfach ist das nicht, Laura, bitte glaub mir das.}

\par

\WR{Dir glauben?} Laura legte allen Sarkasmus in diese Worte, den sie aufbringen konnte.

\par

\WR{Ja, ich gehöre zum Term, aber wir helfen ganz bestimmt nicht diesen außerirdischen Arschlöchern!} Klaus lachte gespielt. \WR{Weißt du, warst eine Kandidatin. Du bist es noch! Der Term rekrutiert überall, nicht nur im Geheimdienst. Egal ob du im Lebensmittelgeschäft kassierst, eine private Fluggesellschaft leitest oder eben Vizeberaterin in der Argus-Abteilung bist. Du kannst dabei sein, wenn dein Charakter passt, wenn du die Qualifikation hast und wenn du keine Angst davor hast, dir die Hände schmutzig zu machen.}

\par

Laura war schon die ganze Zeit kleine Schritte zurückgetreten. Ihre Pistole schoss auf hunderte von Metern gerade aus. Klaus Schrotflinte hingegen verlor schnell an Effektivität. Er erkannte ihr Spiel jedoch und sagte: \WR{Schön stehen bleiben. Eines sag ich dir, auf der Schießbahn habe ich mich immer zurückgehalten. Ich kann sicher schneller abdrücken als du, verlass dich drauf.}

\par

\WR{Warum hast du ihn erschossen?}, wollte Laura wissen. Sie erwartete keine vernünftige Antwort, hoffte aber, ihren Gegner ablenken zu können.

\par

\WR{Er war ein Verräter. Er wollte den Term auffliegen lassen. Frag mich nicht wieso. Persönlich würde ich vermuten, dass er eifersüchtig war. Auch im Term gibt es eine Hackordnung und er stand trotz seiner Leistungen beim Geheimdienst ziemlich weit unten. Hat ständig zu viele Fragen gestellt und wollte sich wichtig machen.

\par

Nicht so wie wir beide. Ich kenne dort nicht mal meine Vorgesetzten und du hättest ihn fast festgenommen, ohne auch nur seine Geschichte anzuhören. Wir haben beide verstanden, dass wir eben manchmal nur Zahnräder sein dürfen, damit das große ganze funktionieren kann. Jeder hat seinen Platz. Bla bla bla. Nicht zu viel denken. Gedanklich aufblühen kann man dann im Privatleben.}

\par

\WR{Er sagte, er hätte Beweise}, sagte Laura langsam und bedacht. \WR{Was hat es damit auf sich?}

\par

\WR{Gutes Stichwort}, antwortete ihr Klaus und klang dabei in der Tat erfreut. Er schritt zum Leichnahm des Hackers, stets mit der Schrotflinte auf ihre Ziel gerichtet und durchwühlte mit der freien Hand die Jack des Getöteten. Schnell wurde er fündig. Ein kleines Buch kam zum Vorschein, dass er Laura vor die Füße warf.

\par

\WR{Was da drin steht, gibt es noch an zwei anderen Orten. Seinem Gehirn}, Klaus deutete auf die Leiche. \WR{Alle klar, an \textit{zwei} Orten. Seinem Buch und meinem Kopf. Wenn ich so drüber nachdenke: du hast recht. In dieser Situation ist es absolut zu viel verlangt, dass du deine Waffe weglegst. Aber ich brauche trotzdem einen Vertrauensbeweis von dir. Du hast jetzt zwei Optionen. Erstens: versuch zu fliehen oder mich zu erschießen und geh dabei drauf. Oder zweitens: schieß auf dieses Buch, werde damit in den Term aufgenommen und hör dir dann von mir an, was ich über den Quatsch weiß, den dieser Kerl verzapft. Und glaub mir, das ist einiges. Du wirst es glaubhaft finden, vertrau mir.}

\par

Laura zögerte. Könnte sie es schaffen, Klaus zu entwaffnen? Die bedeutendere Frage war allerdings: Kannte er jene ihrer Schwächen, die sie in dieser Situation tatsächlich unterlegen machte? Vielleicht war er schneller im Schießen. Aber darauf sie war bereit, es darauf ankommen zu lassen. Das Problem war, dass sie nicht auf einen Freund schießen konnte. Wenn Klaus das wusste, dann war sie erledigt.

\par

\WR{Wenn du sie nicht besiegen kannst, dann schließ dich ihnen an}, zitierte Laura und zielte mir ihrer Waffe auf das Buch. Klaus nickte ihr mit ernstem Gesichtsausdruck zu.

\par

Sie drückte ab, das Buch verbrannte fast sofort und das letzte was sie sah, bevor sie eine Ladung Schrot traf, war das dreckige Grinsen ihres Partners.
