Zur Pilotenausrüstung gehörte kein persönliches Funkgerät. Etwas, dass Nico Curiosa bislang niemals gestört hatte. Nun, da er aber durch die geisterhaft leeren Straßen von Chapelwood lief, wünschte er sich nichts mehr als das.

\par

Die kleine Stadt, die laut ihrem Ortsschild gerade mal sechzehntausend Einwohner beherbergte, hätte auch die Kulisse eines Gruselfilms sein können. Alle Lichter der an sich schon sterilen Häuser waren ausgeschaltet und nirgendwo war eine Menschenseele zu sehen.

\par

Ein Haus, das im Grunde genommen ein großer Glaskasten war, auf dem die Flagge von Kreuzpunkt Primus wehte, stand völlig offen. Nico überlegte für einen Moment, ob er versuchen sollte, darin Schutz zu suchen. Doch er verwarf den Gedanken sofort, als ihm klar wurde, weswegen das Gebäude vermutlich verlassen war.

\par

Andere hatten es wohl geschafft, zu fliehen und die dabei die Türe zu schließen. Vielleicht gingen sie aber auch automatisch zu.

\par

Nico fror. Der Fliegeroverall war zwar warm und seine Uniform beinhaltete sogar Handschuhe aber gegen die Temperaturen, selbst im Süden von Kreuzpunkt Primus, kamen sie einfach nicht an.

\par

Er kannte die Geographie des Planeten kaum, doch wusste, dass sich Chapelwood am Fuß des Spechtgipfels befand. Die besten Chancen, Hilfe zu holen, hatte er von dort oben aus. Zwar dürften sich in den Häusern etliche Nullzonentranciever befinden, doch er hatte keinen Grund anzunehmen, weshalb diese nicht nach wie vor blockiert sein sollten. Auf dem Berg jedoch lag eine Basis der Starforce, die auch noch nennenswerte konventionelle Kommunikationsmittel haben sollte.

\par

Nico schlich um eine Straßenecke. Der Schnee dämpfte den Klang seiner Schritte ab, wofür er mittlerweile sehr dankbar war. Zwar hatte er noch keinen Shutek~-- oder sonst jemanden~-- aus der Nähe gesehen, doch er würde sich auch nicht beschweren, wenn das so bliebe.

\par

Vor ihm erstreckte sich etwas, das wohl die Hauptstraße Chapelwoods bildete. Neben einigen Boutiquen gab es auch einen großen Marktplatz mit zahlreichen Bars und Restaurants. Das Leben schien aus der Szenerie gewichen.

\par

Nico bemerkte erst, dass er die Luft angehalten hatte, nachdem er reflexartig einatmete. Auf dem großen Platz lagen mindestens zwanzig Personen regungslos im Schnee. Es bestand kein Zweifel daran, dass sie alle tot waren. Die Glieder mancher waren in unnatürliche Posen verbogen. Andere lagen nur mit dem Gesicht nach unten und großen Schusswunden im Rücken da.

\par

Unter den Toten waren auch mindestens zwei Kinder. Die Shutek hatten offenbar keinen Unterschied gemacht und einfach auf alles geschossen, was sich bewegt hatte. Nico hatte so etwas schon einmal gesehen. Aber damals waren die meisten der Toten zumindest keine Zivilisten gewesen, sondern Soldaten, die sich hatten wehren können.

\par

Je mehr er aber darüber nachdachte, umso mehr glaubte er, dass es keinen großen Unterschied machte.

\par

Ihm wurde schlecht und er lehnte sich an eine Hausmauer. Doch nachdem er Schritte hörte, geriet er sofort wieder Bewegung. Eine kleine Gasse an der Seite des Marktplatzes diente ihm als Deckung. So fest er konnte presste er sich in einen Hauseinang. Die Tür öffnete sich nicht und so musste er darauf hoffen, dass das Sternenlicht und die fernen Laternen nicht hell genug waren, um ihn zu verraten.

\par

Nico war drauf und dran, aufzuatmen, als er drei Personen erkannte, die den Marktplatz hinunter rannten. Er wusste nicht genau, wie ein Shutek aussah, aber die drei waren definitiv Menschen. Einer von ihnen trug einen langen schwarzen Mantel, die anderen beiden dicke Jacken. Offenbar zwei Erwachsene mit einem Kind.

\par

Gerade wollte er aus der Deckung springen und auf sie zu rennen, als ein gleißender roter Strahl das Mädchen in den Rücken traf. Nico zwang sich, nicht laut zu schreihen, als das Kind plump in den Schnee fiel. Die beiden anderen~-- vermutlich ihren Eltern~-- schafften es nicht. Beide bleiben stehen und sahen den blanken Horror mit angstverzerrten Gesichtern an.

\par

Dann traf der zweite Strahl den Mann, der sich an die Brust griff. Die Entladung war heftig gewesen, denn sein ganzer Torso schien zu glühen. Schließlich fiel er dampfend zu Boden.

\par

Die Frau rannte in Nicos Richtung. Er unterdrückte den Reflex, sofort aus der Deckung zu stürmen und ihr zu helfen. Man hatte ihm keine Waffe gegeben und obwohl er in diesem Augenblick mehr Wut uns Hass verspürte, als in seinem ganzen Leben zuvor, blieb er vernünftig genug, um zu wissen, dass er es nicht mit einem Soldat der Shutek aufnehmen konnte.

\par

Er zuckte zusammen, als er realisierte, dass ihn die Frau gesehen hatte. Sie winkte ihm zu und er blieb wie angewurzelt stehen. Noch bevor sie nach Hilfe rufen konnte, trat eine dunkle Figur um die Ecke und schoss. Der dritte Strahl traf auch sie und sie ging mit ungläubigem Blick in die Knie.

\par

Nico konnte förmlich sehen, wie das Leben aus ihrem Körper wich, als sie im Schnee in sich zusammenfiel und mit weit aufgerissenem aber leerem Blick in seine Richtung starrte.

\par

Dafür lasse ich euch Hurensöhne bluten, schwor er sich und drückte sich mit aller Kraft, die er aufbringen konnte, noch weiter in den Hauseingang.

\par

Seine Wut wich aber zumindest teilweise einer tiefen Angst, als er aus dem Augenwinkel sah, wie der Shutek auf ihn zukam. Er lief langsam und hatte sein Gewehr scheinbar nicht im Anschlag. Also hatte er Nico noch nicht entdeckt, denn er hatte keinen Augenblick gezögert, eine fliehende und unbewaffnete Person zu erschießen.

\par

Sollte er selbst versuchen, zu rennen? Vielleicht war er schneller, als die Frau, die keine zehn Meter von ihm entfernt am Boden lag. Aber er glaubte dennoch nicht, entkommen zu können. Die einzige andere Option war, zu kämpfen. Aber womit?

\par

Nico riskierte einen Blick auf den Shutek. Er war groß. Mindestens zwei Meter und trug ein Gewehr bei sich, das zwei Drittel seiner Höhe maß. Unter dem Lauf war eine lange Klinge befestigt, die man bereits nicht mehr als Bajonett bezeichnen konnte. Dafür war sie viel zu massiv. Mit einem Griff würde sie bereits ein mittellanges Schwert abgeben.

\par

Der Torso des Shutek schien auf den ersten Blick von Schläuchen überzogen. Aber dann erkannte Nico, dass es Wülste seines stabil wirkenden Panzers waren.

\par

Der Shutek wirkte in keiner Weise lebendig. Nicht nur sein Gesicht, das einem Totenschädel glich ließ einen leblosen, mechanischen Eindruck entstehen. Sein ganzer Körper schien metallisch. Aber er bewegte sich zu geschmeidig, um eine Maschine zu sein. Die Union kannte Androiden. Aber keine dieser künstlichen Personen ging derart natürlich.

\par

Und an seinem Hals pulsierte etwas, das wie eine dicke Vene aussah.

\par

Nico hatte zu genau hingesehen, das wurde ihm nun klar. Der Shutek hatte ihn entdeckt. Er richtete sein Gewehr aus und schoss sofort.

\par

Als Nico sich am Boden wiederfand, fragte er sich zuerst, ob er bereits tot war. Er war instinktiv zur Seite gehechtet. Das Bröckeln von Mauerwerk hinter ihm verriet ihm jedoch, dass der Schuss nicht in ihn eingeschlagen war.

\par

Nun lag er aber am Boden und der Shutek baute sich direkt vor ihm auf. Seine Augen verrieten nichts. Sie waren nur dünne rote Lichter im Inneren von tiefen Höhlen. Heller leuchtet der Lauf des Gewehrs, dass das Skelett auf ihn ausrichtete. Nico schloss die Augen.

\par

Dann ein weiterer Knall und er riss sie wieder auf. Der Kopf des Shutek war zerplatzt und Rauch sowie eine scheinbar sehr dickflüssige, grünlich blaue Flüßigkeit drangen aus seinem Hals. Doch sein Körper blieb unbeirrt stehen.

\par

Kurz darauf kam ein großer Mann hinter dem toten Gegner zum Vorschein. In der Hand hielt er noch einen qualmenden, großkalibrigen Revolver. Seine Rüstung~-- vollständig mit Brustpanzer, Funktionshelm und Schulterstücken~-- wiesen ihn als Marineinfanteristen aus.

\par

Er steckte den Revolver weg und reichte Nico die Hand. Dieser ergriff sie zögerlich und ließ sich mit einem kräftigen Ruck auf die Beine ziehen.

\par

\WR{Centurio Schwarzer Bär}, stellte sich der Mann vor. \WR{Sind sie Pilot?}

\par

Nico sah an sich hinab. Nach wie vor trug er die Kluft der Flieger. \WR{Ja. Starforce.}

\par

Schwarzer Bär nickt. Mit einem kräftigen Tritt brachte er den regungslosen Leib des Shutek zum Umfallen. Lediglich das riesenhafte Gewehr nahm er auf.

\par

\WR{Blöde Arschlöcher}, gab er abfällig von sich und es hätte genauso gepasst, wenn er auf die Körper des Shutek gespuckt hätte.

\par

Schwarzer Bär drückte ab und sofort löste sich ein roter Strahl aus der Waffe. Schnee stob von der Stelle auf, die er getroffen hatte. Nicht einmal wegen der Entladung selbst, sondern alleine wegen der Druckwelle, die sie erzeugt hatte.

\par

\WR{Ich bin sehr froh, Sie kennen zu lernen}, sagte Nico vorsichtig. Der Schrecken steckte ihm noch wie vor in den Knochen.

\par

Der Soldate nickte. \WR{Ganz meinerseits. Insbesondere, weil wir gerade dringend jemand brauchen, der eine Fähre oder so etwas fliegen kann. Wir wollen nämlich von diesem verdammten Schneeball runter. Sind Sie der Kerl, der hier in der Nähe abgestürzt ist? Ach, ist doch egal. Gut, dass Sie hier sind.}

\par

Nico zuckte kurz zusammen, als jemand um die Ecke kam. Doch er erkannte schnell, dass es sich um weitere Soldaten handelte. Eine junge Frau trat an Schwarzer Bär heran. \WR{Die Leute auf dem Platz sind alle tot. Alte Leute, Kinder… Warum machen die das? Wollen Sie uns einschüchtern?}

\par

\WR{Keine Ahnung}, antwortete der Soldat. \WR{Denken Sie an Pollux. Eine ganze Kolonie innerhalb von Stunden mit Atomwaffen dem Erdboden gleich gemacht. Ich glaube, es ist ihnen einfach egal. Sie wollen uns nur tot sehen~-- und zwar alle.} Dann sah er Nico an. \WR{Was ist denn nun? Fühlen Sie sich fit, eine Fähre zu steuern.}

\par

\WR{Alles, was mich einer heißen Dusche näher bringt}, entgegnete der Gefragte sofort, während ihm Minx eine Pistole in die Hand drückte.

\par

\WR{Wissen Sie, wie man mit so etwas hantiert?}, fragte sie skeptisch.

\par

Tatsächlich war ihr Unmut gerechtfertigt. Piloten absolvierten zwar eine Ausbildung an Schusswaffen, doch die fand ganz am Anfang des Studiums statt und wurde nicht wiederholt. Glücklicherweise war er kein Pilot der Starforce.

\par

Über sich selbst überrascht, kontrollierte er behände Magazin und Sicherung. \WR{Ich weiß es besser, als mir lieb ist}, kommentierte er.

\par

Schwarzer Bär setzte sich schon wieder in Marsch und winkte seine Leute hinter sich her. Ein paar seiner Gefolgsleute waren schneller, wie der Mann, der ein großes Präzisionsgewehr mit sich trug. Andere waren langsamer, was hinsichtlich ihrer Waffenlast nicht überraschte. Ein besonders stämmiger Soldat hatte ein mehrläufiges Maschinengewehr über den Rücken geschlungen.

\par

\WR{Unser Plan ist folgender}, erklärte Schwarzer Bär Nico, der aufgrund des Fehlens, jeglichen Gepäcks, leicht mit ihm mithalten konnte. \WR{Wir begeben uns zur Starforce-Basis auf dem Spechtgipfel. Wir haben dort mit Infrarot-Fernrohren raufgeschaut und werden dort sicher nicht alleine sein. Ich hoffe sie können schnell rennen. Auf jeden Fall schnappen Sie sich irgendetwas, das fliegt und uns aufnehmen kann. Dann hauen wir gemeinsam ab und hoffen, dass uns die Shutek nicht erwischen. Vor hier aus brauchen wir etwa zwei Stunden auf den Berg, wenn wir uns beeilen. Also: Bewegung.}
