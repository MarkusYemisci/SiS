\WR{Brückenschilde wieder oben}, begann Petrarca, \WR{aber die restlichen Schutzfelder verlieren immer mehr an Leistung! Der Beschuss ist einfach zu stark!}

\par

Fiscale sah zu Wallander. Dabei fand ihr Blick die \EN{Impervious}. Der Zerstörer musste immer mehr und mehr Treffer schlucken. Für den Abschuss von Marschflugkörpern warne die Shutek zu nah, aber ihrer Geschütze stellten selbst ein nicht zu vernachlässigendes Risiko dar.

\par

Dann durchbrach ein besonders dicher Strahl die Schutzfelder der \EN{Impervious} und schlug ungehindert auf der Hüllenpanzerung ein. Diese löste sich an der gertroffenen Stelle sofort in geschmolzenes Metall auf. Weitere Treffer folgten und hüllten den Mantel des Zerstörers in Feuer ein.

\par

\WR{Meldung von der \EN{Impervious}}, gab Wallander durch. \WR{Ihr Reaktor hat Schaden genommen. Schutzfelder und Kanonen verlieren an Energie. Sie müssen abdrehen.}

\par

Fiscale sah auf den Boden. Die Konsequenzen ihres nächsten Befehls waren ihr genauso klar, wie die die taktische Situation. Wenn der Zerstörer zur Steuerbordseite abdrehte, dann läge die \EN{Regenvogel} komplett offen da. Blieb er an Ort und Stelle, würde er nicht mehr lange durchhalten.

\par

Mit brüchiger Stimme wies sie an: \WR{Negativ. Wir versuchen Jäger abzustellen. Aber es ist unabdinglich, das die \EN{Impervious} die Stellung hält!}

\par

Wallander sah seine Kommandantin eine Sekunde lang völlig entgeistert an. Dann schloss er seinen Mund wieder, drehte sich zu seiner Station und gab die Befehle durch.

\par

Fiscale brauchte nicht erst auf ihren Handcomputer zu schauen, um zu wissen, dass die \EN{Regenvogel} derzeit nicht einmal genügend Jäger zur Verfügung hatte, um sich selbst zu schützen. Der \EN{Impervious} Feuerschutz zu geben war ausgeschlossen.

\par

\WR{Captain, wie können diesen Kreuzer nicht besiegen!}, rief Petrarca wütend. \WR{Wir haben nicht die Waffen dafür. Und unsere Zerstörer spielen gerade unsere erweiterte Panzerung.}

\par

Vielleicht könnte sich die \EN{Vulkan} oder die Vaillaint freikämpfen und dann mit Marschflugkörpern auf die \EN{Amon} feuern. Aber ein solches Manöver war ohne Deckung durch Jäger schwer zu bewerkstelligen. Eine andere Option war, vollen Schub zu geben, um eine Stelle zu erreichen, von der aus ein Flug durch den flachen Hyperraum möglich wurde. Dabei musste die \EN{Regenvogel} aber sehr nah mit ihren Jägern zusammenkommen, um diese in ihrem Kielwasser mitnehmen zu können. Und dabei konnte keine der Maschinen ausweichen.

\par

\WR{Der Kern der \EN{Impervious} bricht!}, warnte Wallander, gerade noch rechtzeitig für Fiscale, um aus dem Fenster zu sehen. Hilflos musste sie beobachten, wie das keilförmige Schiff explodierte. Wenn es jemand an Bord zu den Fluchtkapseln geschafft hatte, dann würde er in den Ausläufern des gigantischen Feuerballs verbrennen, der einmal die \EN{Impervious} gewesen war.

\par

Einhundert Frauen und Männer hatte von einem Moment auf den nächsten aufgehört zu existieren. Fiscale war einmal Schöffin bei einer Verhandlung gewesen. Damals war ein Techniker bei einem Drucksturz ums Leben gekommen. Der Vorfall hatte ein jahrelanges Nachspiel gehabt, in dem sich der Vorgesetzte des Opfers hatte verantworten müssen. Der Tod des Technikers hatte praktisch die Welt zum stehen gebracht.

\par

Nun waren viele weitere Leben verloren gegangen. Aber die Schlacht ging einfach weiter, als sei nichts geschehen.

\par

Doch das stimmte nicht voll uns ganz, wie Captain Fiscale schnell feststellte. Durch den Verlust der \EN{Impervious} würde die Flanke der \EN{Regenvogel} offenliegen, sobald sich der Rauch verzogen hatte.

\par

\WR{Sensorenalarm!}, meldete Elshe Schwarzschild. \WR{Sechs Kontakte auf viertrin Kilometer auf elf Uhr. Sind gerade aus dem flachen Hyperraum aufgetaucht.}

\par

\WR{Die haben Verstärkung angefordert}, brachte der Steuermann fast weinend hervor. \WR{Das müssen Bomber sein, sonst hätten sie keinen Sprung machen können!}

\par

\WR{Tja, wenigstens nehmen sie uns ernst}, warf ihm Captain Fiscale entgegen.

\par

Schwarzschild klang nun viel hoffnugnsvoller, als sie weiters meldete: \WR{IFF-Kennung: Kon. Das sind unsere eigenen! Die müssen von der \EN{\EN{Artiglio} de Leone} stammen.}

\par

\WR{Kurzstreckenfunkspruch}, hängte Nils Wallander an. \WR{Ich stelle durch.}

\par

Wenig später quäkte die Stimme eines Piloten aus den Lautsprechern der Brücke. \WR{Hier ist Collonel Blair. Kappa eins von der \EN{\EN{Artiglio} de Leone}. Wir hatten gehofft, bei Ihnen aufmunitionieren zu können. Aber so wie es aussieht, brauchen Sie erst mal was von uns.}

\par

Natalia Fiscale hing ihrem Kommunikationschef praktisch auf der Schulter, als sie antwortete: \WR{Wir haben den feindlichen Kreuzer~-- Zielbezeichnung \Wr{\EN{Amon}}~-- einige Treffer beigebracht. Können Sie ihn vom Himmel holen?}

\par

\WR{Mit Vergnügen, Madam}, war die prompte Antwort. \WR{Torpedoerfassung läuft schon, halten Sie nur ein paar Sekunden durch.}

\par

Einige der Brückenoffiziere suchten mit ihren Blicken fieberhaft den Himmel jenseits der \EN{Amon} ab. Dort musste sich die Bomberstaffel befinden, doch noch waren sie zu weit weg, um mit bloßem Auge erkannt werden zu können.
