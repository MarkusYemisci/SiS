Die weiße Farbe trocknete Kevin eindeutig zu langsam. Kunstvoll hatte er eine Skizze des Shutek-Jägers auf den Hai gezeichnet, den er während des letzten Einsatzes geflogen hatte. Der stark beschädigte Flieger ruhte derzeit unter dem zentralen Flugdeck der \EN{Regenvogel} und wartete auf seine Reparatur. Da die meisten Mitglieder der Mechanikermannschaft gerade dabei waren, den Träger an sich wieder zu flicken, musste die Reparatur des Jägers vermutlich noch lange warten.

\par

Morten sah betreten auf den Boden. Funken hatte auch ihm einen metallenen Becher ihres selbstgebrannten Schnaps vor die Füße gestellt. Doch ihm war nicht nach Trinken zumute. Seine Begleiter, die Chefingenieurin selbst, Kevin und Nico Curiosa nahmen es mit dem allgemeinen Alkoholverbot weniger genau.

\par

\WR{Jetzt trockne endlich!}, forderte Kevin die Skizze des zerstörten Fliegers auf. In seiner Hand hatte er bereits einen Pinsel mit roter Farbe. \WR{Ich will dich durchstreichen.}

\par

Morten verstand nicht, wieso Kevin derart enthusiastisch bleiben konnte. Die Kapitänin hatte ihm praktisch gerade erst Flugverbot erteilt und dennoch war sein Kampfgeist scheinbar ungebrochen. Tatsächlich schien ihm die Situation sogar zu gefallen. Eine Vermutung, die sich bestätigte, als er Morten bat: \WR{Kannst du nachher mit deinem Handcomputer ein Foto von mir und dem Jäger machen? Ich möchte es meiner Freundin auf Kreuzpunkt Primus schicken. Als das mit Pollux passiert ist, hat sie mir einen romantischen Abend pro abgeschossenem Fein versprochen. Und das hier war schon mal Nummer eins.}

\par

\WR{Sicher}, antwortete Morten tonlos.

\par

Ein Moment der Stille folgte, in dem jeder außer er einen Schluck von Funkens Schnaps nahm. Nicos war der größte, weswegen er hustete und sich auf die Brust klopfte, nachdem das merkwürdig riechende Gebräu gänzlich in seinem Hals verschwunden war. Als er die Augen wieder öffnen konnte, nickte er der Chefingenieurin anerkennend zu.

\par

\WR{Was ist eigentlich mit dir, Morten?}, fragte Dilara Bashir an den einzigen Nichttrinker gerichtet. \WR{Hast du auch eine Freundin, mit der du auf tote Shutek anstoßen kannst.}

\par

Der Angesprochene schüttelte sofort den Kopf. \WR{Nein. Und wenn, wäre es mir zu makaber.}

\par

Die Chenfingenieurin mied seinen Blick kurz und sagte dann: \WR{Du solltest auch mal ein bisschen locker lassen. Glaub mir, das ganze könnte eine längere Sache werden. Und es ist schön, jemanden zu haben, den man nach einem Einsatz besuchen kann.}

\par

\WR{Streng dich nicht an, Funken}, sagte Kevin sofort. \WR{Das versuche ich ihm auch schon die ganze Zeit über beizubringen. Aber er hört nicht auf mich. Er reitet eben Paragraphen und keine Frauen.}

\par

Morten wollte seinem Stubenkameraden gerade etwas böses zurufen, als dies Dilara bereits tat. \WR{He, dir hab ich nicht erlaubt, mich Funken zu nennen.}

\par

\WR{Entschuldigung, Madam}, erwiderte Kevin sofort. \WR{Ich dachte nicht, dass das noch eine Frage wäre, nachdem du uns schwarzgebrannten Schnaps geschenkt hast.}

\par

Während sich die Chefingenieurin und der Pilot noch etwas weiter kabbelten, wandte sich Nico an Morten. \WR{Noch mal: ziemlich schöner Flugstil da draußen.}

\par

Der Angesprochene nickte dem älteren Piloten freundlich zu. \WR{Das Kompliment gebe ich gerne zurück. Ich weiß, einen Transporter zu steuern ist nicht ganz so glamourös wie einen Raumjäger. Aber du hast es den Kerlen echt schwer gemacht, das konnte man sehen. Ich wünschte, dein Freund hätte es auch geschafft.}

\par

\WR{Oder die drei din anderen Leute in seinem Schiff}, hängte Nico sofort an. \WR{Ich kann noch gar nicht fassen, dass er weg ist. Wir haben gemeinsam so viel durchgemacht.}

\par

Morten hielt es für besser, nichts zu sagen. In den Tagen nach der Schlacht von Pollux hatte er das ein oder andere Wort mit Nico Curiosa gewechselt, doch er hatte nicht das Gefühl, ihn bereits gut genug zu kennen, um irgendetwas persönliches sagen zu können, dass ihm helfen würde.

\par

Schließlich war es Nico, der wieder das Wort ergriff. \WR{Auf Ed Steiner und alle anderen Freunde, die wir verloren haben.}  Nach einem großen schluck stellte er seinen Becher ab und erhob sich. \WR{Ich gehe schlafen. Captain Fiscale hat mich morgen früh einbestellt. Keine Ahnung, was sie von mir will. Davor brauche ich noch ein bisschen Ruhe.}

\par

Die Schritte des Piloten hallten in dem großen Raum unter dem Hangar laut nach, als er sich verabschiedet hatte und ging. Morten ließ seinen Blick durch die Halle schweifen. Zunächst hatte er den Hangar selbst als blank und unschön empfunden. Aber das Lager für Raumjäger und Fähren war noch um einiges spartanischer. Einige der Stahlträger, die ohnehin von sich aus eine braune Farbe hatten, zeigten Rosteflecken, die vermutlich von den undichten Leitungen herrührten, die völlig offen lagen und an etlichen Stellen tropften.

\par

\WR{Ja, darum muss ich mich kümmern}, sagte Funken, als sie Mortens Blick an die Decke gehen sah. \WR{Aber auch eine Chefingenieurin braucht mal Feierabend.}

\par

\WR{Tut mir leid. Ich wollte dir nicht das Gefühl geben, dass du deine Arbeit nicht richtig machst},  entschuldigte sich Morten sofort.

\par

Dilara lachte nur und klopfte dem Piloten gegen die Schulter. \WR{Du machst dir viel zu viele Sorgen. Wenn du mich \textit{während des Dienstes} kritisierst, dann rauscht es. Aber hier bin ich nur eine Freundin.}

\par

Morten bemühte sich, Dilaras Lächeln zu erwidern. Wieder blieb es eine ganze Weile lang ruhig, bis Kevin schließlich lautstark seinen Pinsel über das Abbild des feindlichen Jägers zog. Sein Gesichtsausdruck erinnerte nicht wenig an den eines Kindes, das gerade seine Weihnachtsgeschenke auspackte.

\par

Dilara schenkte ihm keine weitere Beachtung, sondern wandte sich wieder an Morten. \WR{So ziemlich jeder findet, dass du ein kleinlicher Korinthenkacker bist. Da überrascht es mich doch, dass du so gut mit Nico klar kommst.}

\par

Nun war auch Kevin ganz Ohr. Morten fragte: \WR{Wieso sollte ich nicht? Ich meine, ich kenne ihn nicht wirklich, aber er scheint ein netter Kerl zu sein.}

\par

Dilaras Gesicht wurde mit einem mal sehr ernst. Auch Kevins Blick verfinsterte sich und er schien bereits ungutes zu erwarten. \WR{Du weißt nicht, wer dieser Mann ist, oder?}, fragte er.

\par

\WR{Ein ehemaliger Starforce-Pilot, der nach Pollux Primus gezogen ist?}, entgegnete Morten vorsichtig und bereits ahnend, dass mehr dahinter steckte.

\par

Funken sah auf den Boden, und umklammerte fast ihren Becher, als sie erklärte: \WR{Nico Curiosa war Terrorist. Er ist für die Capital Fellowship geflogen.}

\par

Morten zuckte fast zusammen und erhob unterbewusst die Hände in eine Abwehrhaltung. \WR{Das kann nicht sein. Auf so etwas steht lebenslänglich.}

\par

Kevin fuhr mit Dilaras ursprünglicher Erklärung fort und genoß es dabei sichtlich, einmal derjenige zu sein, der mehr wusste. \WR{Du hast doch sicher von der \EN{Majestic}-Tragödie gehört, oder? Der Träger, der vor einem Jahrdutzend von der Fellowship zerstört worden ist. War damals überall im DDV.}

\par

Natürlich kannte Morten die Geschichte. \WR{Die \EN{Majestic} war das erste Schiff der Maria Stuart-Klasse. Sie sollte die Umsiedlung von Bürgern von Aquarii Minor leiten. Aber die Capital Fellowship hat einen der Transporter gekapert und unter dem Vorwand eines medizinischen Notfalls eine Landung auf Impervious vorgenommen. Dafür musste diese ihre Schutzfelder ganz herunterfahren. Und diese Gelegenheit hat eine Bomberstaffel der Fellowship genutzt, um den Antrieb des Schiffes zu zerstören. Da sie mangels Kurs keinen Flug durch den flachen Hyperraum durchführen konnte und in einem tiefen Orbit um Aquarii Minor lag, ist sie abgestürzt.}

\par

\WR{Aber jeder Träger hat auch immer Verteidigungsjäger an Bord}, fuhr Kevin fort. \WR{Nico Curiosa hat damals mit einigen anderen Begleitschutz für die Bomber geflogen und dabei auch einen Jäger der Starforce abgeschossen. Der Pilot ist ausgestiegen. Aber dass er das geschafft hat, war reines Glück.}

\par

Mortens Entsetzen war nun Verwunderung gewichen. \WR{Aber wenn Nico das getan hätte, dann würde jetzt immer noch in einem Gefängnis sitzen.}

\par

\WR{Er hat nach dem Angriff auf die \EN{Majestic} wohl Gewissensbisse bekommen und die Geiseln befreit}, führte Dilara weiter aus. \WR{Dann hat er dabei geholfen, die Capital Fellowship zu infiltrieren, was letztendlich zu deren Zerschlagung und Laurentia Batanides Tod geführt hat. Darum und weil er niemanden getötet hatte, hat man seine Haftstrafe stark verkürzt und ihm lediglich verboten, jemals wieder einen Beruf im tiefen Weltall anzunehmen.}

\par

Kaum das die Chefingenieurin ihre Erklärung beendet hatte, setzte Morten zum ersten Schluck aus seiner Schnapstasse an.
