\WR{Steaks sind fertig!}, rief Ed Steiner, der sich sogar die Mühe gemacht hatte, eine echte Kochmütze für die Grillpartie zu besorgen.

\par

Sein Freund Nico Curiosa lag absolut entspannt in einem Liegestuhl im Garten seines Hauses. Er war am Ende seiner Kräfte, denn er hatte einen anstrengenden Arbeitstag hinter sich gebracht. Seit er seine Strafe in einer Besserungsanstalt auf Miranda abgesessen hatte, arbeitete er als Landwirt auf Pollux Primus und das kam ihm mindestens genauso anstrengend vor, wie einen Raumjäger zu steuern.

\par

Etwas ungeschickt griff Curiosa nach der Wasserflasche, die er neben sich ins Gras gestellt hatte. Schließlich fand er sie und nahm einen genüsslichen Schluck daraus. Erfrischt vom Wasser überwand er seine eigene Trägheit und setzte sich auf.

\par

Obwohl er anfangs große Schwierigkeiten damit gehabt hatte, keiner Arbeit im tiefen Weltall mehr nachgehen zu dürfen, hatte er sich inzwischen damit abgefunden. Der Wechsel in seinem Leben hatte auch viel gutes mit sich gebracht. Anstatt der kalten Metallwände eines Raumschiffs umgab ihn nun ein großes Haus und die wunderbare Landschaft von Pollux. Am Horizont erstreckte sich ein riesiger See von den Ausmaßen eines kleinen Landes in dem man das ganze Jahr über baden konnte, denn ein mildes Klima war eine der angenehmsten Eigenschaften des Planeten. Schnee und damit die Gelegenheit zum Skifahren gab es in den Bergen, die sich an diesem Tag nur als undeutliche Schemen am Horizont abzeichneten.

\par

Das umliegende Land gehörte Nico und seiner Familie beinahe alleine. Aber er bestellte nur einen kleinen Teil davon. Dennoch war es mehr als genug um ihm, seiner Frau und seinen zwei Kindern ein angenehmes Leben zu sichern. Obwohl der überwältigende Großteil von Pollux Primus Vegetation aus dem Ökosystem der Erde stammte, galten sein Obst und sein Getreide als sehr hochwertig. So war es für ihn kein Problem, mit dem Verkauf von Gerste seinen Lebensunterhalt zu bestreiten.

\par

Ähnlich gut ging es Ed. Auch er war ein ehemaliger Pilot, der seine Arbeit allerdings freiwillig aufgegeben hatte. Nun arbeitete er als Volkswirt in der nahen Hauptstadt des Planeten, deren Lichter in der Abenddämmerung langsam sichtbar wurden.

\par

Nico wollte sich gerade erheben um sich eines von Eds Gemüseschnitzeln zu sichern, als es am Himmel plötzlich hell blitzte. Das Licht schien so intensiv, wie das eines Sterns. Selbst Nicos Kinder, die sich sonst durch nichts vom wilden Herumtollen abbringen ließen, sahen nun gebannt gen Himmel. Doch genauso schnell wie er gekommen war, verschwand der Blitz auch wieder.

\par

Nico erinnerte sich an die Gelegenheit, als er selbst ein derartiges Aufblitzen gesehen hatte und dabei kam ihm eine düstere Vorahnung.

\par

\WR{Was war denn das?}, fragte Nicos Frau Vivien gedankenverloren. \WR{Ein Gewitter?}

\par

Doch Ed schüttelte sofort den Kopf. \WR{Selbst auf diesem Planeten gibt es kein Gewitter ohne Wolken.} Dann sah er Nico an. Beide teilten einen Blick, der den anderen wissen ließ, dass sie wohl gerade den gleichen Gedanken hatte.

\par

\WR{Die \EN{Kaukasia}}, murmelte Nico leise. Doch Vivien konnte die Besorgnis in seiner Stimme deutlich heraushören.

\par

Nur einen Moment später erklang ein lautes Alarmsignal. Es ging von einem Durchsagemast aus, der irgendwo zwischen den Bauernhöfen und der Stadt aufgestellt worden war.

\par

Niemand~-- mit Ausnahme von Nico und Ed~-- konnte etwas mit dem Alarmsignal anfangen. Die beiden wussten jedoch, dass es sich um den Warnton für Alarmstufe eins handelte.

\par

\WR{Vivien}, begann Nico ruhig und beherrscht zu sprechen, \WR{bring die Kinder bitte in den Wagen.}

\par

Seine Frau trat an ihn heran. Ihre Stimme klang genauso beherrscht wie seine und doch konnte er ihre Angst wahrnehmen. \WR{Was ist denn los?}

\par

\WR{Ich weiß nicht}, antwortete Nico. \WR{Aber dieses Signal bedeutet, dass die Stützpunkte auf volle Alarmbereitschaft gehen.}

\par

\WR{Vielleicht ist es nur eine Übung}, dachte Ed laut.

\par

Doch Nico schüttelte sofort den Kopf. \WR{Das glaube ich kaum. Nicht während eines Fußballendspiels. Die wissen, dass da jeder vor dem Holographen hängt. Und derzeit sind schlechte Reaktionszeiten das letzte, was man den Kontrollgremien vorlegen will.}

\par

Vivien nahm ihre beiden Kinder an die Hand und ging mit ihnen ins Haus. Nico nickte ihr entschlossen zu, als sie noch einmal zurücksah.

\par

\WR{Was sollen wir denn jetzt machen?}, fragte Ed Steiner ratlos.

\par

Nico wollte ihm gerade antworten, als eine Stimme aus dem Nachrichtenmast klang. \WR{Alle Menschen müssen sich sofort in die Stadt begeben. Das ist keine Übung. Bitte bewahren Sie Ruhe und kommen sie zügig in die Stadt. Es gelten die Notfallregeln.}

\par

\WR{Scheiße}, murmelte Nico und machte sich ebenfalls auf, in sein Haus zu gehen. Ed folgte ihm auf dem Fuß.

\par

Als beide das gemütliche Anwesen durchquert hatten, war Vivien gerade dabei das Hovercraft aus dem Unterstand zu fahren. Ihre beiden Kinder saßen auf dem Rücksitz des Cabriolets.

\par

\WR{Ich fürchte, dass war’s mit der Grillparty}, murmelte Ed und setzte sich in den Wagen.

\par

Nico war nicht in Stimmung für Galgenhumor. Sein alter Kumpan Ed hatte keine Familie, um die er sich Sorgen machen musste. Früher war Nico Curiosa nicht anders gewesen. Doch seit er geheiratet und Kinder bekommen hatte, sah er viele Dinge ernster.

\par

Er setzte sich gerade auf den Beifahrersitz, als plötzlich ein neuer Alarmton erklang. Nico erkannte das Signal sofort und richtete seinen Blick auf den Horizont. Dort begann sich bereits eine gigantische Energiekuppel zu formen, welche die Stadt und die Farmen unter sich einschloss. Die Schilde der Kolonie wurden von einem unterirdischen Generator aus erzeugt und bedeckten nun den Himmel von Pollux Primus.

\par

Die Barriere wirkte fast wie die Oberfläche einer Seifenblase, weswegen Nicos Kinder das Schauspiel staunend verfolgten. Ein Gewirr aus den unterschiedlichsten Farben und abstrakten Formen huschte über die Oberfläche der Schilde. Doch Nico stieß nur seine Frau an und sagte: \WR{Gib Gas, wir müssen hier weg.}

\par

Vivien ließ den Motor an und steuerte das Hovercraft auf die Landstraße. Sie warf ihrem Mann einen besorgten Blick zu und wollte wissen: \WR{Nico, sag es mir. Was ist hier los?}

\par

\WR{Ich hab keine Ahnung}, antwortete dieser in Gedanken versunken

\par

Kurz darauf gesellte sich zu den grellen Lichtern der Schilde ein weiteres Leuchten, dass zuerst nur schwach und dann immer stärker am Himmeln zu erkennen war. Der Punkt wuchs immer mehr an und zog einen hellen Schein hinter sich her. Dann ging alles sehr schnell. Noch ehe jemand erkennen konnte, was dieses Leuchten gewesen war, stürzte das Geschoss auf die Schilde und ging in einem grellen Feuerball auf.

\par

Nico verstand sofort was passierte. \WR{Nicht hineinsehen!}, schrie er seiner Frau Vivien entgegen und stürzte von seinem Beifahrersitz auf. Er wandte sich der Rückbank zu und versuchte, seinen Kindern die Augen zuzuhalten, die wie versteinert in den Himmel starrten. Noch im selben Moment folgte ein Donner, der so laut war, dass sich niemand unter den Schilden noch denken hören konnte. Nico presste sich wie all die anderen auch beide Hände flach auf die Ohren, konnte das Dröhnen damit aber kaum dämpfen.

\par

Dann sah er, wie eine kompakte Feuerwand die Schildblase entlang hinab glitt und alles außerhalb der Abschirmung verbrannten. Ganze Wälder fingen in Sekundenbruchteilen Feuer und verwandelten sich in ein rotschwarzes Gewisch. Abgesehen von dem Donnern spürte man innerhalb der Schilde jedoch nichts von der Verwüstung.

\par

\WR{Was war das?}, schrie Vivien noch immer, als der Lärm schon langsam nachließ.

\par

Ihr Mann berührte jedoch nur ihren Arm und sagte ihr so gefasst, wie er nur konnte: \WR{Fahr so schnell es geht.}

\par

Widerstrebend tat sie wie geheißen und trat aufs das Beschleunigungspedal. Das Hovercraft schoss los und raste mit einhundertfünfzig Stundenkilometern auf die Hauptstadt zu. Unterdessen beobachtete Nico gebannt, wie die Geschütztürme des nahen Kitty-Hawk-Stützpunktes zu feuern begannen. Ihr Ziel war nicht zu erkennen, schien sich aber irgendwo in hohen Schichten der Atmosphäre zu befinden.

\par

Ihr Feuerspeien blieb jedoch vergebens. Eine weitere leuchtende Kugel schoss aus dem langsam dunkler werdenden Himmel herab und traf trotz Abwehrfeuer auf die Schildhalbkugel. Eine neue, verheerende Explosion folgte. Diesmal war ein anderer Teil des Schutzfelds getroffen worden. Nico wusste, dass sich dahinter der Tiefgrundsee befunden haben musste. Doch von diesem blieb nichts als ein Krater übrig, als die Glut und die Hitze der Explosion das Wasser verdampften und jedwede Vegetation im Umkreis innerhalb kürzester Zeit zu Asche verwandelten.

\par

Nicos Frau schrie vor Schmerz, als der Donner zu grollen begann und sie sich nicht die Ohren zuhalten konnte, weil sie das Steuer des Wagens im Griff behalten musste. Schließlich hielt sie es nicht mehr aus und trat auf die Bremse. Kurz bevor er die Straßen des Vororts Ulmengrad erreicht hatte, blieb der Hovercraft stehen.

\par

Vivien presste sich beide Hände auf die Ohren während sie mit schmerzverzerrtem Gesicht die Augen schloss. Nico sah sie an und schrie ihr entgegen \WR{Es ist vorbei. Der Donner ist weg.}, doch sie reagierte nicht. Er musste ihre Hände vorsichtig aber bestimmt wegziehen. Nach ein paar endlos erscheinenden Sekunden öffnete sie wieder die Augen und sagte viel lauter als nötig: \WR{Ich hör fast gar nichts mehr! Verdammt, das tut so weh!}

\par

\WR{Ich werde fahren}, sagte Nico und sah seinen ehemaligen Fliegerkollegen Ed an. Beide stiegen aus dem Hovercraft und befreiten Vivien aus ihrem Fahrersitz. Nico unterdrückte dabei gewaltsam die fast übermächtige Angst um seine Frau und seine Kinder. Hastig riss er sich Teile der Ärmel seines Hemds ab und versuchte, sich irgendwie damit die Ohren zu verstopfen.

\par

Nachdem Ed Vivien vorsichtig auf dem Beifahrersitz untergebracht hatte, gab Nico wieder Gas. \WR{Das sind Nullzonenbomben, oder?}, fragte Ed Steiner und klang dabei fast ängstlich. \WR{Deine alten Freunde sind wieder da.}

\par

Nico Curiosa steuerte sein Hovercraft deutlich langsamer durch die Straßen von Ulmengrad. Nicht wegen der Geschwindigkeitsbegrenzung, sondern weil anscheinend alle Familien des kleinen Ortes offensichtlich gleichzeitig ihre Häuser verließen und mit dem nötigsten bepackt in ihre Hovercrafts sprangen und damit alle Wege hoffnungslos überfüllten.

\par

Nico sah sich um. Ein Elternpaar schrie sich verzweifelt gegenseitig an, während ihre beiden Kinder weinend am Straßenrand standen. Der Vater drängte darauf, den restlichen Weg zur Stadt zu laufen, doch seine Frau wollte unbedingt das Hovercraft nehmen. Aus dem Haus gegenüber trug ein junges Pärchen im halben Minutentakt allerlei Habseligkeiten in seinen fahrbaren Untersatz. Wäre die Situation nicht so ernst gewesen, hätte Nico gelächelt. Trotz der nichtmaterialistischen Grundhaltung der Union hatten sich die Menschen nicht von ihren Vorstellungen von Eigentum entfernt. Er erkannte darin wieder, wie weit Politik und Philosophie von der Wirklichkeit entfernt waren.

\par

\WR{Wie weit ist es noch?}, fragte er an Ed Steiner gewandt. Sein Fliegerkollege überlegte nur kurz. \WR{Vielleicht noch drei Kilometer.}

\par

Nico fluchte laut. Er war Pilot bei der Starforce gewesen und wusste in ungefähr, wie viel die Schilde der Stadt aushalten würden. Wenn sein ehemaliger Staffelkamerad recht gehabt hatte, und wirklich Nullzonenbomben auf Graustadt fielen, dann würden die Schutzfelder schon bald nachgeben.

\par

Wie aufs Stichwort traf schon das nächste Geschoss auf die Schildkuppel. Diesmal an einer weiter entfernten Stelle. Der resultierende Donner erklang jedoch nur unwesentlich leiser. Nico stoppte den Wagen und presste sich beide Hände so fest er konnte gegen den Kopf.

\par

Erst als der Lärm langsam abklang, war die neueste Lautsprecherdurchsage deutlich zu verstehen. \WR{Bitte bewahren Sie Ruhe. Es ist eine Notsituation eingetreten und wir werden Graustadt evakuieren. Begeben Sie sich bitte geordnet zum Raumhafen. Dort werden Transportschiffe zum Abflug bereitgestellt.}

\par

Nun brach flächendeckend Panik aus. Menschen schrien und begannen von Angst getrieben davonzurennen. Das Hupkonzert auf den Straßen nahm bald Lautstärken an, die dem Einschlag der Bomben Konkurrenz machen konnten. Das junge Pärchen, das eben noch damit beschäftigt gewesen war, seine Habe zu retten ließ nun alles fallen und sprang in sein Hovercraft.

\par

Auch Nico konnte dem Drang nicht widerstehen, wie wild auf den Hupknopf zu hämmern. Schließlich erkannte er aber, dass es keinen Sinn hatte. Zwischen Ulmengrad und Graustadt schien es einen kilometerlangen Stau zu geben und er fragte sich, ob es zu Fuß vielleicht nicht doch schneller gehen würde.

\par

\WR{Nico, es gibt sicher nicht genug Plätze für alle hier}, gab Ed Steiner zu bedenken und klang dabei so leise wie beunruhigt. Nico sagte nichts, doch er wusste, dass sein Freund recht hatte. Auf Pollux Primus lebten fast vier Millionen Siedler. Doch die wenigen Transporter und Frachter, die hin und wieder für den Personentransport und die das Verschicken von Getreide verwendet wurden, würden keinesfalls genügend Raum für alle bieten. Er konnte nur beten, dass er mit seiner Familie noch irgendwie an Bord einer der Transporter kommen würde, bevor die Schilde versagten.

\par

Hoffnung flammte wieder auf, als Nico sah, wie die Kanonen von Kitty Hawk eine herabfallende Bombe trafen, die daraufhin in einem hellen Lichtblitz verging. Jubel brach aus, als die Menschen auf der Straße das Schauspiel mit ansahen. Doch Nicos Hoffnungen wurden gleich wieder gedämpft, als er am Horizont einige Kondensspuren erkannte. Die Flieger, die sie hinterließen, kamen schnell näher. Nur einen Augenblick später eröffneten sie das Feuer auf die Schildkuppel.

\par

Und Nico wusste auch wieso. Ihre Kanonen waren zwar wahrscheinlich zu schwach, um das Schutzfeld zu durchbrechen, doch es würde es genügend aufweichen, damit sie selbst hindurch fliegen konnten. Die Einschläge ihrer Strahlenwaffen klangen dumpf, als würde man auf eine Metallplatte trommeln.

\par

Dann trafen die ersten der Jäger auf die Schildblase. Eine helle Explosion flammte auf und als Nico den Feuerball sah, glaubte er, sie wären alle am Schutzfeld zerschellt. Doch eine Sekunde später war zu sehen, dass wohl nur einer der Flieger zerstört worden war.

\par

Die Jäger waren zu weit entfernt, als das Nico etwas hätte erkennen können. Doch sie schossen mit einem ohrenbetäubenden Gedröhn über Ulmengrad hinweg, direkt auf die Kitty Hawk Stützpunkt zu, der gerade noch unterhalb der Schildkuppel lag.

\par

Für einen Sekundenbruchteil konnte Nico einen der Flieger erkennen, als er im Tiefflug über die Vorstadt hinwegbrauste. Seine Hülle schimmerte im Licht der untergehenden Sonne wie der Panzer eines Käfers. Lange Kanonenrohre ragten von den Enden seiner Flügeln hervor.

\par

Das Hupen hinter Nicos Hovercraft wurde lauter und lauter. Er hatte so gebannt in den Himmel gesehen, dass ihm gar nicht aufgefallen war, dass die Straße vor ihm langsam leerer wurde. Wenigstens ein bisschen erleichtert trat er auf das Beschleunigungspedal und reihte sein Fahrzeug wieder in die Warteschlange ein. Für einen Moment bereute er es, nach seiner Entlassung aus der Haftanstalt, nicht auf eine der Kernwelten gezogen zu sein. In den riesigen Städten der Erde, Centauri, Sirius oder Kreuzpunkt Primus gab es keine Hovercrafts mehr. Dafür wären allerdings nicht die Straßen aber die Untergrundbahnen und die Züge völlig überfüllt gewesen.

\par

Ein weiteres Lichtspiel erleuchtete den Himmel, als ein paar Falken der Starforce, die Nico am Klang ihrer Turbinen erkannte, sich ein Feuergefecht mit den Angreifern lieferten. Nico wollte schon in den Jubel der Menschen in seiner Umgebung mit einstimmen, als einer der Falken getroffen wurde und brennend in einen nahen Wald stürzte. Eine Feuerwelle schoss von der Absturzstelle schnell in alle Richtungen und der Donner der Explosion übertönte für kurze Zeit alles andere.

\par

Angstschreie und wütende Ausrufe folgten. Nico hätte den Unbekannten am liebsten selbst die übelsten Flüche nachgejagt. Doch er musste sich aufs Fahren konzentrieren. Mühsam schob er die Angst und die Wut beiseite, als er auf die Schnellstraße zur Stadt einbog und ein wenig mehr Gas gab. Der Stau schien sich aufgelöst zu haben und Nico fragte sich, wie es so schnell dazu gekommen sein konnte.

\par

Dann erkannte er wie ein Pulk Panzer der Phalanx in einiger Entfernung von der Straße in ein nahes Feld rollte. Diese schweren, mit doppelläufigen Granatwerfern bestückten Kampfkolosse schienen vorher die Fahrbahn blockiert zu haben.

\par

Als Nicos Hovercraft an der Einheit vorbeifuhr blickte er kurz die angsterfüllten und vor Anstrengung rot gewordenen Gesichter der jungen Soldaten, die neben den Panzern her rannten. Obwohl die Fahrzeuge langsam fuhren, hatten sie Mühe, mitzuhalten. Einige der Autofahrer jubelten ihnen zu und feuerten sie an, als sie an der Einheit vorbeikamen. Nico hörte den Mann im Hovercraft vor ihm laut \WR{Macht die Schweinehunde fertig!}, brüllen. Gewehre schienen für ihn wohl ein Symbol für Wehrhaftigkeit zu sein. Doch Nico war klar, dass ein paar Salven aus einer Handwaffe nichts gegen die Jäger ausrichten konnten, die wie zornige Hornissen unter der Schildkuppel umherjagten.

\par

\WR{Wer immer das ist, dumm ist er nicht}, rief Ed Steiner vom Rücksitz aus. \WR{Seitdem diese Flieger hier sind, haben sie aufgehört Bomben zu werfen. Diese verdammten Verbrecherschweine sind schon wieder zurück.}

\par

\WR{Das ist nicht die Capital Fellowship}, entgegnete ihm Nico. Er kannte sich bestens mit dieser Terrorgruppe aus, die schon seit einem Jahrzehnt nicht mehr aktiv gewesen war. Aus eigener Hand wusste er, wie schlimm ihre Angriffe gewesen waren. Aber eine ganze Kolonie unter Beschuss zu nehmen und eine Raumstation der Starforce zu vernichten, dass hätte er ihnen auch zu ihrer Blütezeit nicht zugetraut.

\par

Der Stadtrand war nun schon in greifbarer Nähe und Nico beschleunigte wie die anderen Fahrer vor ihm noch ein bisschen mehr. Kurz hallte ein neuer Knall auf. Er war nicht so laut wie der vorherige, schien aber einen viel näheren Ursprung zu haben. Nico sah in den Rückspiegel und seine Befürchtungen bewahrheiteten sich. Eine rotschwarzer Feuerball hüllte den Ort ein, an dem zuvor der Kitty Hawk Stützpunkt gestanden hatte.
