Kurze Zeit später hatten Morten und Kevin ihr Gepäck vorläufig in Bashirs Büro verstaut und sie in die Backbordflügel der \EN{Regenvogel} begleitet. Nach wie vor schwer beladen mit zahllosen Ersatzteilen. Morten musste seinen Kopf am tiefsten einziehen. Bashir hingegen konnte sich in den engen Zusatzgängen mühelos bewegen. Die Wände waren nicht abgeschottet und legten den Blick frei auf die Eingeweide des Schiffes. Schier unzählige Rohre verliefen in chaotischem Muster und mindestens die Hälfte davon schien mehr oder weniger undicht zu sein.

\par

Morten atmete innerlich auf, als die drei einen breiteren, Raum erreichten, der an das Innere eines Brunnens erinnerte. Über ihren Köpfen musste sich besagter Geschützturm befinden.

\par

\WR{Wir hätte öfter damit schießen sollen}, gab Lieutenant Bashir zu bedenken. \WR{Das Öl in den Drehräumen trocknet aus. Und dann werden die Stoßdämpfer zu heiß.} Als ob die \EN{Regenvogel} die Worte ihrer Chefingenieurin untermauern wollte, fiel den dreien eine kaputte Variante der Bauteile entgegen, die Morten gerade mit sich trug.

\par

\WR{Ach so. Vorsicht!}, sagte Bashir, als das laute Scheppern des herunterfallenden Metallstücks verklungen war. Der Stoßdämpfer schien derart verzogen, dass sein Zustand ohne Frage war.

\par

Die Chefingenieurin nahm sich ein Ersatzteil aus Mortens Armen und schob es in die nun leere Röhre. Nachdem sie kurz mit einem elektrischen Schraubenschlüssel an dem Stoßdämpfer gewerkelt hatte, griff sie nach ihrem Handcomputer. Das Militär verwendete selbst Jahre nach der Markteinführung der ersten Bücher keine. Ihre Vorgänger galten nach wie vor als weniger fehleranfällig und waren leichter zu warten.

\par

\WR{Rufe Petrarca}, sagte die Chefingenieurin zu ihrem Gerät. Wenig später quäkte die Stimme ihres Gesprächspartners aus dem Handcomputer. \WR{Was gibt's, Bashir? Ich bin beschäftigt.}

\par

\WR{Ich auch, du Held}, war die prompte Antwort. \WR{Ich hab gerade einen defekten Stoßdämpfer in G drei ausgetauscht. Die anderen sehen gut aus. Lass das Ding doch einfach mal ein bisschen rotieren und simulier einen Schuss, damit wir sehen, ob es wieder geht.}

\par

\WR{Verstanden. Geht in Deckung.}

\par

Bashir scheuchte die beiden Piloten mit wedelnden Händen zurück in den Zugangstunnel, aus dem sie gerade gekommen waren. Nur kurze Zeit später begannen sich die Zahnräder im über dem \Wr{Brunnen} zu drehen und ließen dabei die Innereien des Flügels der \EN{Regenvogel} erbeben. Richtig laut wurde es aber erst danach, als ein langer Bolzen unvermittelt nach unten schoss und sofort darauf wieder nach oben glitt.

\par

\WR{Alles klar, Petrarca}, bestätigte Bashir, während Kevin und Morten noch mit offenem Mund auf das Geschehen starrten. \WR{G drei ist hiermit offiziell einsatzbereit. Ende.}

\par

Die Chefingenieurin nahm die restlichen Stoßdämpfer aus Mortens Armen und verstaute sie in einer metallenen Vorratskiste. \WR{Was ist?}, wollte sie wissen, als sie bemerkte, wie entgeistert Kevin Wilson sie ansah.

\par

Dieser räusperte sich zunächst. \WR{Das war ziemlich gefährlich. Wenn dieses Ding runtergekommen wäre, während wir noch drunter standen…}

\par

\WR{Dann hätte es deinen Holzkopf zertrümmert}, bestätigte Bashir bereitwillig. \WR{Merk dir eines: wir haben hier draußen keine Sicherheitsschilder, fast keine verschlossenen Türen und schon gar keine Hundeleinen für Welpen. Also wenn du irgendwo hinlatschst, wo du nicht hingehörst und dann draufgehst, beschwer dich nicht.}

\par

\WR{Ist das legal?}, wollte Morten wissen, glaubte die Antwort aber bereits zu kennen.

\par

\WR{Natürlich nicht}, blaffte die Chefingenieurin. \WR{Aber wir sind meistens außerhalb des Territoriums der Union unterwegs. Kein vereinigter Boden, keine Unionsgesetze. Nur die Statuten der Navy. Und die sind doch eher Richtlinien.}

\par

Bashir zwängte sich an den beiden Männern vorbei.

\par

\WR{Gab es denn schon mal Unfälle?}, fragte Kevin fast ängstlich.

\par

\WR{Keine tödlichen zumindest}, war Dilaras prompte Antwort. \WR{Liegt daran, dass wir hier draußen aufeinander Acht geben und mitdenken. Das könnt ihr doch sicher auch, oder? Ihr seid doch große Jungs. Kommt mit!}

\par

Weitere zehn Minuten später hatten die drei den Hauptmaschinenraum entdeckt. Er befand sich in einem Aufbau hinten und unterhalb des Hauptrumpfes. Der Anblick war überwältigend. Die vielen Techniker und Ingenieure, die dort arbeiteten, wirkten winzig, verglichen mit den Maschinen. Allem voran der Hauptenergiequelle, die sich längs durch den ganzen Maschinenraum zog und von Zugangsgerüsten gesäumt war.

\par

\WR{Das ist \Wr{der dicke Dirk}. Unser Perpetuum Mobile}, präsentierte Bashir stolz das Herz der \EN{Regenvogel}, das an eine Induktionsspule erinnerte. Allerdings eine mit gut fünfzehn Metern Durchmesser. Es war außerdem eines der wenigen Bauteile der inneren Sektionen, auf denen Morten nicht ansatzweise Rost erkennen konnte. Die goldenen und silbernen Oberflächen glänzen metallisch, während sie sich in einem merkwürdigen Rhythmus hin und her bewegten und dabei fast geräuschlos blieben.

\par

Die Wände hingegen wirkten aufgeraut und abgenutzt. Ein Eindruck, der täuschte. Sie bestanden aus einer extrem widerstandsfähigen Legierung, die zusätzlichen Schutz für den verwundbarsten Teil der \EN{Regenvogel} bot. Selbstverständlich waren die Metallplatten daher nur schwer zu berarbeiten und wirkten darum fast wie Rohmaterial.

\par

\WR{Ich hab nie richtig kapiert, wie diese Dinger funktionieren}, gab Kevin zu. \WR{Ich meine, es kann kein echtes Perpetuum Mobile sein, oder? Die sind physikalisch unmöglich.}

\par

Bashir antwortete, ohne ihren Blick von dem riesigen zylindrischen Generator abzuwenden. \WR{Das ist eigentlich ganz einfach. Unser Universum ist…} Dann piepte ihr Handcomputer. Die Nachricht entlockte ihr ein Seufzen. \WR{Jeff braucht Hilfe. Unterhalb des Hangars klemmt ein Aufzug zu den Jägerbuchten. Dass sollten wir unbedingt vor dem Sprung hinbekommen, sonst dauert es ewig, bis ihr mir euren Vögeln im All seid. Ihr geht besser zum Quartiermeister. Im Hangar steht ihr nur im Weg. Der Lift ist da hinten.} Dilara deutete auf einen einen Ausgang, der neben dem Hauptreaktor winzig wirkte. \WR{Wie gesagt: C-Deck. Raum zweidutzend eins. Viel Spaß noch!}

\par

\WR{Danke, Lieutenant Bashir}, schloss Morten, froh darüber, nichts weiter tragen zu müssen.

\par

\WR{Bitte}, erwiderte die Chefingenieurin. \WR{Nenn mich \Wr{Funken}.}