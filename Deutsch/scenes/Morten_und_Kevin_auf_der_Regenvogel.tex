Morten Wittwer schlug hastig beide Augen auf. Er war eingedöst, doch irgendetwas sagte ihm, dass er längst hätte aufwachen müssen. Ein kurzer Blick auf den Monitor, der in den Sessel seines Vordermannes eingelassen war, bestätigte seine Befürchtungen. Es war bereits neunzehn Uhr Dutzend. Die \EN{Galere vier sieben} hätte die \EN{Regenvogel} längst erreichen müssen. Noch mehr Verwunderung folgte dem Blick aus dem Fenster. Grell leuchtende Lichter, die ein wenig an Wolken erinnerten, zuckten immer wieder am Fenster vorbei. Der helle Schein musste Morten geweckt haben. Wieso befand sich der Transporter im flachen Hyperraum?

\par

\WR{Krass, oder?}, hörte Morten es von rechts sprechen. Unwillkürlich riss er seinen Kopf herum und bemerkte einen Piloten, der ihm während der Reise schon öfter aufgefallen war. Er war hoch gewachsen und hatte rotblonde Haare. Seine stämmige Figur rundete ein langer, schwarzer Ledermantel ab, den er über seiner Uniform trug und der in keiner Weise passend erschien.

\par

\WR{Die \EN{Regenvogel} ist einfach weg}, fuhr er fort. \WR{Keine Ansage aus dem Cockpit und niemand hat etwas gesehen. Einer der Piloten glaubt, wir sind auf dem Weg zur Route nach Arktur.}

\par

Morten wurde nur langsam wieder wach. Er musste wesentlich tiefer geschlafen haben, als er zunächst geglaubt hatte. Noch bevor er zu einer halbwegs sinnvollen Antwort ansetzen konnte, streckte ihm der andere Pilot seine Hand entgegen. \WR{Kevin Wilson.}

\par

Widerwillig schüttelte er sie. Tatsächlich war es auch auf Corna nicht unüblich, diesen altertümlichen Gruß zu verwenden. Er galt als persönlicher als die obligatorische Verbeugung. Doch Morten mochte ihn trotzdem nicht.

\par

\WR{Second Lieutenant Wittwer}, antwortete er schlicht. Halb unwillig, sich schon jetzt auf eine informelle Bekanntschaft einzulassen, halb im Schlaf und sich nur an die formelle Vorstellung erinnernd.

\par

\WR{Und wie ist dein Rufzeichen?}, fragte Kevin Wilson und setzte sich auf den freien Platz neben Morten. Dieser konnte ein Seufzen nicht unterdrücken und sagte: \WR{Ich habe noch keines. Wie auch? Ich bin ja noch nicht einen Einsatz geflogen.}

\par

Wilson wandte wegen dieser forschen Erwiderung seinen Blick kurz ab. \WR{Das hat nichts zu sagen}, entgegnete er schließlich. \WR{Mich nennt man seit meinem ersten Jahr an der Akademie \Wr{Murphys Law} oder einfach nur \Wr{Law}, wenn dir das zu lange ist.}

\par

Englisch. Morten fragte sich, wieso diese altertümliche Sprache noch nicht ausgestorben war. So wie der Rest der Sprachen aus der Zeit vor der Seuche. Nicht einmal die autonomen Welten verwendeten sie noch flächendeckend. Aber besonders in der Starforce war sie weit verbreitet. Während des Routenkriegs war die Armada des Commonwealth für lange Zeit die dominante Kraft im All gewesen. Und damals hatte man auf Kreuzpunkt noch Englisch gesprochen. Nach der Gründung der Union hatte sich die schwierige Aufgabe gestellt, ein gemeinsames Heer aufzubauen. Und dabei hatten sich Sitten, Gebräuche und sogar längst tote Sprachen in die moderne Welt eingeschlichen. Darum sagten Offiziere der Starforce und der Navy \Wr{Sir} oder \Wr{Madam}, statt \Wr{mein Herr} oder \Wr{meine Dame}, wie es die Mitglieder der Bodentruppen taten. Auch die Bezeichnung der Ränge war nach wie vor Englisch.

\par

\WR{Ich habe so viele glückliche Treffer in Übungsgefechten gesetzt, dass man irgendwann mal Murphys Gesetz auf mich angewandt hat}, erläuterte Kevin Wilson. \WR{Mir selbst wäre ja \Wr{Glückskind} lieber gewesen. Aber ich denke, man kann sich seinen Spitznamen nicht aussuchen. Und immerhin ist das doch eine kleine Ehre, oder?}

\par

Morten lächelte. \WR{Ich bin kein Experte für die Kultur vor der Seuche. Aber so viel ich weiß, besagt Murphys Gesetz, dass alles, was schief gehen kann auch schief gehen wird.}

\par

Kevin Wilson blieb einen Moment ruhig. Dann antwortete er: \WR{Für dich finden wir auch noch einen. Pass nur auf, dass es nicht \Wr{Miesepeter} oder was in der Richtung wird.}

\par

Wieder folgte eine längere Pause. Schließlich fuhr Kevin Wilson fort: \WR{Was denkst du, wieso sind wir noch nicht da?}

\par

\WR{Keine Ahnung}, antwortete Morten sofort.

\par

\WR{Vielleicht wurde die \EN{Regenvogel} angegriffen. Ich habe gehört, die Raubzüge von Piraten sind gerade in den Randbereichen deutlich häufiger geworden.}

\par

\WR{Dann hätten wir ein Notsignal empfangen und diese Fähre befände sich bereits auf höchster Alarmstufe}, gab Morten zurück. \WR{Außerdem glaube ich nicht, dass Piraten einen Träger des Konglomerats direkt angreifen würden.}

\par

Kevin wirkte fast überrascht. \WR{Da vergisst du aber die \EN{Majestic}.}

\par

\WR{Das ist erstens fast ein Jahrdutzend her und zweitens wurde sie nicht von Piraten vernichtet, sondern von der Capital Fellowship. Es gibt keine Berichte darüber, dass die Schwarzflaggen irgendetwas größeres als einen Zerstörer als Ziel gewählt hätten. Dafür sind sie einfach zu schlecht ausgerüstet.}

\par

\WR{Dann waren es vielleicht die autonomen Welten, oder ein natürliches Phänomen.} Kevin Wilson schien kaum zu bremsen, so aufgeregt war er. \WR{Irgendwas stimmt hier nicht, da bin ich mir sicher.}

\par

Noch ehe jemand etwas weiteres sagen konnte, erklang die Stimme der Pilotin aus der Bordsprechanlage. \WR{Achtung, Achtung. Wie Sie sicher bereits bemerkt haben, wurde eine Kurskorrektur durchgeführt. Wir sind nach wie vor auf dem Weg zu \EN{Regenvogel}. Diese hat aber einen neuen Auftrag erhalten und befindet sich bereits auf dem Weg nach Arktur. Wir treffen sie daher dort. Bitte beachten Sie, dass wir wenig Zeit haben, Sie alle von Bord zu bringen, also trödeln Sie gefälligst nicht beim Aussteigen!}

\par

Kevin nahm neben Morten Platz und schnallte sich an. Dieser vermied bewusst den Augenkontakt und ersparte sich jeden Kommentar. Auch ein \WR{Jetzt wird's ernst} von Kevins Seite konnte kein weiteres Gespräch in Gang setzen.

\par

Der Rest des Fluges ging schnell vorbei. Die Fähre passierte in einiger Entfernung einen Gasriesen, dessen blaue Atmosphäre durch die zuckenden Lichter, welche den Transporter umgaben, nur vage zu erahnen war. Der Planet wirkte wie eine Glaskugel, in die abwechselnd heller und dunkler Sand eingefüllt worden war. Die Bänder hatten unterschiedliche Breiten, ihre Grenzen verliefen jedoch seltsam parallel.

\par

Der flache Hyperraum war eine mittlerweile gut erforschte Schicht, ohne die interplanetares Reisen so gut wie unmöglich war. Selbst die Entfernungen innerhalb eines Planetensystems waren bereits so gigantisch, dass es auch mit den leistungsstärksten Unterlichtantrieben Monate dauerte, von einem Himmelskörper zum anderen zu kommen.

\par

Die Navigation im flachen Hyperraum war deutlich unkomplizierter, als durch die tieferen Gefilde. Kurskorrekturen waren vergleichsweise leicht zu bewerkstelligen und Unfälle blieben extrem selten. Etwas, das sowohl Kevin als auch Morten stets beruhigte.

\par

Und dann verzogen sich die züngelnden Lichtteppiche um die Fähre herum so plötzlich, dass es viele fast schon mehr überraschte als der Ruck und der dumpfe Knall, welche mit dem Zurückfallen auf Unterlichtgeschwindigkeit einhergingen.

\par

Mortens Neugierde gewann schließlich gegen seine Wunsch nach einer neutralen Haltung und er sah angestrengt aus dem Fenster. Irgendwo da draußen schwebte nun die \EN{Regenvogel} KlT durch den Raum. Wenn sie sich bereits in offiziellem Einsatz befand, dann waren ihre Positionsleuchten und Suchscheinwerfer gemäß dem Protokoll der Navy deaktiviert. Lediglich in einem hochfrequentierten Bereich wäre sie wie jedes Schiff verpflichtet, diese eingeschaltet zu lassen. Aber die Region um den Sprungpunkt nach Arktur war Niemandsland.

\par

\WR{Du schaust in die falsche Richtung!}, mahnte Kevin und entlockte Morten damit einen weiteren entnervten Seufzer. Schließlich folgte sein Blick dem Zeigefinger seines Sitznachbarn und ging durch eines der gegenüberliegenden Fenster.

\par

Die \EN{Regenvogel} war bereits sehr nah. Jemand musste es in der Tat sehr eilig haben, um den Hyperraumflug einer Fähre derart dicht bei einem anderen Raumschiff enden zu lassen. Zu dem Träger konnte die Distanz nur noch wenige Kilometer betragen, denn auf dem Rumpf waren bereits deutlich Fenster zu erkennen.

\par

Der längliche Rumpf erinnerte an einen Flugzeugträger aus der Zeit vor der Seuche. Ein langes Start- und Landedeck nahm den Großteil der Fläche ein und endete am Heck mit einem großen Tor, das ins Innere der Aufbauten führte. Diese wirkte für ein Schiff der Größe der \EN{Regenvogel} fast schon überproportional hoch, was allerdings nicht überraschte, da darin nicht nur die Mannschaftsquartiere, sondern auch die Kommando- und Kommunikationseinrichtungen untergebracht waren.

\par

Hinter den Aufbauten fand sich ein weiteres Tor, das über den Plasmaturbinen lag und das Startdeck durchgängig machte. Trotzdem fanden auf Trägern der \EN{Deprisa}-Klasse alle Start- und Landeoperationen durch das Vordertor statt. Von hinten aus zu landen war wegen der geringen Strecke im Innenbereich schlichtweg unmöglich~-- außer für Senkrechtstarter wie der neuen F achtundneunzig oder Fähren wie der \EN{Galere vier sieben}.

\par

Die Flügel, deren kurze Verbindung zum Hauptrumpf fragil wirkte, beinhalteten weitere Plasmaturbinen, welche zu der enormen Höchstgeschwindigkeit der \EN{Regenvogel} beitrug. Außerdem dienten sie als Plattform für insgesamt sechs Strahlengeschütze.

\par

Einige Träger der \EN{Deprisa}-Klasse verfügten vor dem Brückenturm über ein Anti-Schiff-Geschütz, welches bei der \EN{Regenvogel} komplett fehlte. Stattdessen fand sich an dieser Stelle eine gut bestückte Lafette, mit mehreren Raketenwerfern. Wie Morten bereits aus seinem Studium der technischen Daten seiner zukünftigen Arbeitsstelle wusste, zur Abwehr von Marschflugkörpern sowie Jägern und Bombern. Marschflugkörper waren eine der Waffengattungen, die Großkampfschiffen schon mit wenigen Treffern schwer schaden konnten.

\par

\WR{Achtung, Achtung. Wir haben Landefreigabe von der Leitstelle \EN{Regenvogel} erhalten und bitten Sie, sich anzuschnallen und eine sichere Position einzunehmen}, erklang es aus dem Lautsprecher.

\par

Ein Raunen ging durch die Reihen, als die Fähre zu einer weiten Kurve ansetzte und dabei einen noch genaueren Blick auf den leichten Träger bot. Der Schriftzug \Wr{\EN{Regenvogel}} war erst kürzlich neu auf die Hülle lackiert worden. Die Lettern hoben sich durch ihre leuchtend silberne Farbe deutlich vom Rest der gebraucht wirkenden Hülle ab. Eher dem sichtbaren Dienstalter der Panzerplatten entsprach die Zeichnung eines tatsächlichen Vogels am Bug des Trägers, der mit weit aufgerissenem Schnabel und wütendem Blick einen stillen Schrei ausstieß. Derartige Dekorationen zeugten davon, dass die \EN{Regenvogel} die meisten ihrer Jahre auf Patrouille der äußeren Planetensysteme verbracht hatte. Schiffe der Kernflotten wirkten in der Regel wesentlich gepflegter aber auch nüchterner.

\par

Das hintere Hangartor der \EN{Regenvogel} öffnete sich und machte den Blick auf das Innere der Flugdecks frei. Am Rand des Tores machte sich das atmosphärische Schutzfeld nur durch die weiß glänzenden Streifen bemerkbar, welche die Öffnung einfassten.

\par

Ein dumpfes Grollen erklang, als \EN{Galere vier sieben} ihre Landestutzen ausfuhr und die Gegengravitationstriebwerke zuschaltete. Dies ermöglichte es ihr, für kurze Zeit im Schwerkraftfeld der \EN{Regenvogel} zu schweben und somit langsam und kontrolliert in den Hangar zu gleiten. Die künstliche Schwerkraft, die auch zum Abfangen der Massenträgheit genutzt wurde, hatte allerdings auch den Nachteil, dass Raumschiffe selbst in Schwerelosigkeit ständig beschleunigt werden mussten, um ihr Tempo zu halten.

\par

Der Hangar wirkte von innen betrachtet noch größer als von außen. Die Mitte des Startdecks war bemalt wie eine Straße und von Singallampen gesäumt. An den Rändern waren Mitglieder der Deckmannschaften beschäftigt. Die Routiniertheit der Techniker war bereits daran zu erkennen, dass kaum jemand aufsah und der landenden Fähre Beachtung schenkte. Das Versorgen der zahlreichen Kisten schien wichtiger zu sein.

\par

Morten griff reflexartig nach seinen Armlehnen, als die Fähre auf das Flugdeck der \EN{Regenvogel} aufsetzte und ihre Insassen dabei heftig gen Decke hievte. Im selben Moment, in dem er ausatmete, hörte er aus dem hintere Bereich der Fähre bereits von mindestens einer Stelle ein lautes Würgen, gefolgt vom sofortigen Aufsteigen eines säuerlichen Geruchs. Die Landung war nicht übermäßig ruckartig gewesen. Darum konnte es kaum an etwas anderem als dem Alkohol gelegen haben.

\par

Wenig später meldete sich wieder die Pilotin aus dem Cockpit. \WR{Achtung. Ich habe gerade von der Brücke der \EN{Regenvogel} mitgeteilt bekommen, dass der Steuermann gerne die Haupttriebwerke hochfahren würde. Mit anderen Worten, das faule Pack soll seine billigen Plätze verlassen und sich umgehend an Deck melden. Also raus hier und zwar plötzlich!}

\par

Morten und Kevin lösten fast zeitglich ihre Sitzgurte. Wilson mochte vielleicht ein Schwätzer sein, aber den militärischen Drill hatte er genauso angenommen, wie Wittwer selbst. Zügig verließen die Piloten den Passagierraum und traten nacheinander über eine viel zu steile und viel zu enge Treppe an Bord der \EN{Regenvogel}. Das grelle Licht der riesigen Leuchtstangen an der Decke des Hangars ließ Morten sofort die Augen zukneifen.

\par

Als er sie wieder öffnete, erkannte er, wohin die Gruppe der Piloten ging. Am steuerbordseitigen Rand des Hangars gab es eine Treppenkonstruktion, die mehrere Decks zugänglich machte und dabei so baufällig wirkte, dass Morten erwartete, sie würde anfangen hin und her zu wanken, sobald sie jemand betrat. Als zwei Personen in den Uniformen der technischen Abteilung durch eine schwere Panzertür kamen, zeigte sich allerdings, dass dem nicht so war. Vor der Treppe wartete ein Mann, offensichtlich ein Pilot, in Bereitschaftsuniform. Dieselbe dunkelblaue Jacke mit Schulterpolstern und Stehkragen, die auch Morten und die anderen Piloten trugen. Allerdings mit den Rangabzeichen eines Majors.

\par

Vorschriftsgemäß nahmen die Piloten in einer Reihe Aufstellung und richteten ihren Blick starr nach vorne. Kevin Wilson war einer der wenigen, die ein Lächeln aufsetzten. Morten stand zu seiner Linken und atmete tief durch. Offizielle Anlässe ließen ihn nervös werden und der Knoten, den er in seiner Magengegend zu spüren glaubte, zog sich durch den Gestank nach verbranntem Capezin und Motoröl noch enger zusammen.

\par

Der Major begann mit den Armen hinter dem Rücken verschränkt vor der Reihe auf und ab zu gehen. Er hatte dunkelblonde, kurz geschorene Haare, die so eben wirkten, als könnte man darauf ein Haus bauen. Seine Uniform saß ebenfalls perfekt. Weniger vorschriftsmäßig erschien sein Bart, der entlang seines Unterkiefers nur eine dünne Linie war und eigentlich überhaupt nicht dort sein durfte. Genauso wenig wie seine Sonnebrille, die zwar sportlich wirkte, aber auf einem Raumschiff mehr als unnötig war.

\par

\WR{Ach du scheiße!}, brüllte er den Neuankömmlingen unvermittelt entgegen, als diese salutierten. \WR{Man kann euch mit einem Blick ansehen, wie verdammt miserabel eure Ausbildung war.}

\par

Morten stöhnte lautlos. Bereits auf der Akademie hatte er Offiziere wie diesen kennen gelernt, die glaubten, dass Neulinge als aller erstes eine gehörige Abreibung brauchten, um sich gut einfinden und funktionieren zu können.

\par

Etwas leiser aber immer noch schreiend fuhr er fort. \WR{Nehmt gefälligst Haltung an, wenn ich mit euch rede!} Sofort zuckten die Piloten zusammen und korrigierten ihre, weitestgehend ohnehin bereits ordnungsgemäße Haltung. \WR{Ich bin Major Dexter Hennington und ich führe das Sternenflügel-Geschwader an. Normalerweise würde euch jetzt meine Kollegin Commander Anna Farley begrüßen. Aber Sie macht im Gegensatz zu euch etwas sinnvolles und fliegt Kampfpratrouille.} Obwohl niemand der Neuankömmlinge irgendeine Reaktion zeigte, tat Hennington so, als ginge ein leises Flüstern durch die Reihe. \WR{Ihr habt richtig gehört! Kampfpatrouille. Hier draußen spielt das echte Leben, nicht den reglementierten Schwachsinn, den euch eure Lehrer auf der Akademie verzapft haben. Ist das klar?}

\par

Ein lautes \WR{Ja, Sir!} folgte. Selbstverständlich war es Dexter Hennington zu leise, weswegen er es prompt wiederholen ließ.

\par

\WR{Da ihr Blödmänner euch nicht alleine zurecht findet, habe ich jetzt die Ehre, mir zu merken, welche dämlichen Fressen ab heute hier an Bord dienen. Also sagt mir euren Rang, Namen und Rufzeichen.}

\par

Hennington fing von rechts an. Die Pilotinnen und Piloten bellten ihm entgegen, was er wissen wollte. Wenig überraschend verwarf der Major sämtliche Spitznamen und vergab eigene Kreationen, die teilweise sogar recht originell aber ausnahmslos gemein waren.

\par

Bei einem Piloten blieb er etwas länger stehen, bevor er etwas sagte. Morten schielte hinüber und erkannte den unglücklichen Trinker, dem sein Erbrechen nach wie vor anzusehen war.

\par

\WR{Sag mal, hast du dich etwa vollgekotzt?}, fragte Hennington und bemühte sich dabei, besonders angeekelt zu wirken.

\par

\WR{Sir, ich bitte um Entschuldigung. Der Flug war etwas ruckartig und ich…}

\par

\WR{Du hast gesoffen!}, plärrte ihm der Geschwaderführer entgegen. \WR{Was immer noch besser ist, als ein Kampfpilot, dem der Flugstil einer Raumfähre den Magen umdreht.} Wie aufs Stichwort hob in diesem Moment \EN{Galere vier sieben} ab. Morten rang wie der Rest um Haltung, denn die Hitze der Turbinen war noch bis zum Rand des Hangars zu spüren und der Lärm war schlichtweg ohrenbetäubend.

\par

\WR{Ich versichere Ihnen, Sir, ich werde ab sofort keinen Alkohol mehr anrühren.} Die Stimme des jungen Mannes zitterte förmlich.

\par

\WR{Labern Sie keinen Scheiß!}, donnerte Hennington. \WR{Sie saufen gefälligst weiter! Ein echter Soldat muss das können und zwar ohne sich dabei einzusauen!} Morten rollte mit den Augen und war sofort dankbar darüber, dass der Major nicht gerade in seine Richtung sah. Er konnte nun keine Lehrstunde über das altertümliche Bild eines Militärangehörigen brauchen.

\par

\WR{Was mich zu deinem Rufzeichen bringt}, fuhrt Hennington fort. \WR{Kotzbrocken.}

\par

Nur kurze Zeit später war Kevin Wilson an der Reihe. Aus einem Grund, den Morten nicht verstand, lächelte er nach wie vor. Hennington baute sich vor dem Piloten auf, denn er überragte ihn nur um wenige Zentimeter. \WR{So so, wir haben hier also einen modebewussten jungen Mann.} Der Major strich forsch über Wilsons Lederjacke. \WR{Echt. Vermutlich ein Import aus den autonomen Welten, da wir braven Unionsbürger uns ja nicht mehr die Finger beim Töten von Tieren schmutzig machen.}

\par

\WR{Da haben Sie Recht. Ich habe sie von…}

\par

Morten hatte gewusst, dass Hennington nun anfangen würde, zu brüllen. Trotzdem fuhr er zusammen. \WR{Das ist mir scheißegal! Du trägst deine Uniform genauso wie jeder andere Frischling auf diesem Kahn, kapiert?}

\par

Die Frage, die sich allen stellte, als Wilson antwortete, war, ob er sich seines Fehlers bereits von vorne herein bewusst war. \WR{Sie tragen doch auch eine Sonnebrille. Und ich glaube weniger, dass die laut Statuten erlaubt ist.}

\par

\WR{Wie lautet dein Name?}, fragte Hennington gefährlich ruhig und leise.

\par

\WR{Ensign Kevin Wilson, Sir. Rufzeichen \Wr{Murphys Law}.}

\par

\WR{Mhm}, begann Hennington. \WR{Von jetzt ab nenne ich dich \Wr{Witzfigur}. Weil du ein kleiner Klugscheißer bist, der keine Ahnung hat, wie das echte Leben ist.} Die Lautstärke des Majors stieg mit jedem Wort exponentiell. \WR{Die Tage an denen du dich bei deiner Mama ausheulen konntest sind vorbei! Und du musst dir ab jetzt auch selbst den Arsch abwischen. Aber vor allem stellst du nie wieder die Kleidung deiner Vorgesetzten infrage, ist das klar?}

\par

Wilson musste einen halben Liter Speichel im Gesicht haben, so nah war ihm Hennington gekommen. \WR{Nein, Sir, ist es nicht}, antwortete er unbeeindruckt. \WR{Und Sie dürfen mich nicht anschreien.}

\par

\WR{Was?}, brachte der Major hervor und klang dabei erneut unangenehm entspannt. \WR{Sag das noch mal.}

\par

\WR{Laut den geltenden Gesetzen der Union~-- die für uns genauso gelten, wie für jeden Zivilisten~-- ist das Anschreien einer Person eine aggressive Handlung und somit untersagt. Was meinen Spitznamen betrifft: \Wr{Militärangehörige sind mit Rang und Nachnamen anzusprechen}. Frei zitiert aus den Statuten des Konglomerats, bezüglich der Behandlung von Kameraden. Ich denke, ich habe mich klar genug ausgedrückt.}

\par

Hennington atemte durch, explodierte jedoch nicht in einen Schreikrampf, wie es einige vorhergesehen hatten. \WR{Zieh deine Jacke aus. Sofort.} Wilson tat wie geheißen und reichte sie dem Major. Dieser pfiff durch die Finger und sicherte sich somit die Aufmerksamkeit einer der Techniker. \WR{Majit. Sei so gut und schmeiß das hier in den Müll.} Der Mann griff sich wortlos die Lederjacke und ging davon. Wilson wollte ihm folgen, doch Hennington packte ihn am Kragen und drückte ihn relativ mühelos in die Reihe zurück.

\par

\WR{Witzfigur, du gehst mir jetzt schon mehr auf den Sack als der Rest deiner traurigen Truppe. Wir sind noch nicht fertig miteinander. Im Moment ist die Zeit nicht da aber wir sprechen uns noch einmal.}

\par

\WR{Mit Vergnügen}, entgegnete Kevin Wilson mit einem breiten Grinsen. \WR{Sir.}

\par

Hennington biss sich auf die Unterlippe, ließ dann aber von Mortens Nebenmann ab. Noch ehe dieser sich überlegte, wieso der Geschwaderkommandant so kampflos aufgegeben hatte, hatte sich dieser bereits vor ihn gestellt.

\par

Wie aus der Pistole geschossen, bellte der Pilot. \WR{Second Lieutenant Morten Wittwer, Sir. Ein Rufzeichen habe ich keines. Tut mir leid.}

\par

Hennington lächelte schief. \WR{Ah, sehr gut. Ein unbeschriebenes Blatt. Formbar. Das ist mir am liebsten. Lass mal sehen…} Er musterte sein Gegenüber gründlich, wobei es Morten erleichterte, dass er seine Uniform so vehement in Schuss hielt. \WR{Da hat man ja die Qual der Wahl. Du bist so hässlich, dass mir fast gar nichts dazu einfällt. Eine Knollennase, viel zu breite Lippen. Wenigstens hast du nicht dieselbe große Klappe wie die Witzfigur neben dir.} Er  warf Kevin Wilson einen bösen Blick zu, den dieser gespielt freundlich erwiderte. \WR{Du bist schon Second Lieutenant, was?}

\par

\WR{Aye, Sir}, entgegnete der Angesprochene tonlos. \WR{Entsprechende Leistung in der Flugschule.}

\par

\WR{Ein echter Profi also}, schlussfolgerte Hennington spöttisch. \WR{Ich denke \Wr{Stümper} passt ganz gut zu dir. Was meinst du.}

\par

Morten schluckte jede halbwegs schlagfertige Antwort hinunter und sagte stattdessen: \WR{Gute Wahl, Sir.}

\par

\WR{Wunderbar. Wir beide verstehen uns}, entgegnete Hennington und tätschelte Morten spielerisch aber fest auf die Wange. Nachdem er zu seiner Ausgangsposition zurückgekehrt war, sagte er an alle gerichtet: \WR{Jetzt, wo ich euch Arschlöcher endlich persönlich kennen lernen durfte, kann ich mich ja wichtigeren Dingen widmen. Eigentlich sollte ich euch jetzt eure Quartiere und ein bisschen was vom Schiff zeigen. Aber ehrlich gesagt: Ich hab keinen Bock. Ich hab keinen Bock mich für euch anzustrengen und ich hab keinen Bock auf eure blöden Visagen. Also schnappt euch euer Gepäck, dass vermutlich sowieso nur aus Pornoheften, Gleitgel und Taschentüchern besteht, und verpisst euch!}

\par

Das darauf unisono folgende \WR{Ja, Sir} sagte deutlich mehr als nur zwei Worte.

\par

Morten atmete auf und blieb noch eine Weile stehen. Er bemerkte erst, wie abwesend er wohl gewesen war, als er die ersten Piloten bereits mit ihren Reisetaschen auf die Treppe steigen sah. Die Besatzung der Fähre hatte das separat verstaute Gepäck in einigem Abstand sogar mehr oder weniger ordentlich auf dem Flugdeck aufgereiht. Morten hatte keine Probleme, seine Tasche zu finden, da nur noch zwei dort standen.

\par

Die andere gehörte offensichtlich Kevin Wilson. \WR{Was für ein Wichser}, spie der junge Pilot aus, als er sich sein Gepäck über die Schulter warf. Als von Morten keine Antwort kam, fragte er: \WR{Was hältst du von unserem Begrüßungkomitee?}

\par

\WR{Es war erwartungsgemäß}, gab der Angesprochene kurz angebunden zurück.

\par

\WR{Du redest nicht viel, oder?}, wollte Kevin Wilson wissen, was ihm einen skeptischen Blick seines Gesprächspartners einbrachte.

\par

\WR{Kommt darauf an, worüber}, antwortete dieser schließlich. \WR{Beispielsweise würde mich im Moment am brennendsten interessieren, wo unsere Kabinen sind und wo wir uns melden sollen.}

\par

Noch bevor Kevin Wilson eingestehen konnte, dass er ebenfalls überfragt war, gab eine helle Stimme die Antwort: \WR{Der Quartiermeister hat sein Büro auf dem C-Deck. Von ihm bekommt ihr alles, was ihr braucht. Ersatzuniformen, Zimmernummern und eure Handcomputer für den Dienstgebrauch.}

\par

Die Stimme gehörte zu einer jungen Frau, die so zierlich gebaut war, dass sie praktisch in ihrer Mechanikerlatzhose schwamm. Zumindest schienen die Handschuhe, die sie sich ihre Kleidung und den Gürtel geklemmt hatte ihre Größe zu haben. Ihr Gesicht war ölverschmiert, allerdings nicht genug, um die vielen Sommersprossen zu verstecken, doch ihre Frisur war hingegen tadellos. Sie hatte sich ihre pechschwarzen Haare zu einem Zopf zusammengebunden.

\par

Morten und Kevin stellten sich vor.

\par

\WR{Lieutenant Dilara Bashir}, entgegnete die Frau. \WR{Besser bekannt als \Wr{Funkenschlag}. Aber das täuscht. Ich habe die Geräte dieser Mühle hier gut im Griff.}

\par

\WR{Da bin ich mir sicher}, antwortete Morten sofort. \WR{Lieutenant ist ein hoher Rang für eine Mechanikerin.}

\par

Bashir grinste breit. \WR{Aber nicht für eine Chefingenieurin.}

\par

Morten sog den Atem ein. \WR{Entschuldigung, Madam. Ich wollte nicht respektlos sein.}

\par

\WR{Waren Sie nicht}, entgegnete die Ingenieurin. \WR{Mechaniker arbeiten sowieso härter. Also war das sozusagen ein verunglücktes Kompliment.}

\par

\WR{Wieso tragen Sie eigentlich keine Offizieruniform?}, wollte Kevin wissen und klang dabei verwunderter, als er es sein sollte.

\par

\WR{Damit Sie nicht dreckig wird, Schlaukopf}, war Dilaras antwort. \WR{Außerdem verwirre ich gerne die Neuen.}

\par

Morten hatte sich nach wie vor noch nicht ganz von seinem Tritt in das Fettnäpfen erholt, was in eine gewisse Unsicherheit in seiner Stimme mündete, als er sagte: \WR{Sie wirken sehr beschäftigt, Madam. Ich denke, wir begeben uns jetzt lieber zum Quartiermeister.}

\par

Dilara winkte ab. \WR{Der hat Mittagspause und danach eine Unterredung mit der Kapitänin. Außerdem haben Ihre Kameraden sein Büro vermutlich schon gefunden. Sie sind also als letzte in der Reihe und können dort Stunden warten. Oder Sie machen gleich mal einen guten Eindruck und helfen mir in der Zwischenzeit tragen.}

\par

Ohne eine Antwort abzuwarten, drückte Dilara Morten drei Zylinder entgegen, von denen dieser keine Ahnung hatte, was sie eigentlich darstellten. \WR{Stoßdämpfer für ein Schnellfeuergeschütz}, erklärte die Chefingenieurin, als sie Mortens fragenden Blick erkannte. \WR{Einmal damit geschossen und schon kaputt. Echte Wertarbeit der Union. Zum Glück haben wir einen echten Auftrag bekommen. Dadurch haben wir jetzt höchste Priorität, was neue Ausrüstung betrifft.}

\par

Kevin nahm einen Werkzeugkoffer, auf den Bashir deutete und fragte dabei: \WR{Weiß man schon, worum es geht?}

\par

\WR{Nur, dass wir schnellstmöglich nach Arktur sollen. Mehr weiß ich nicht.} Die Chegfingenieurin hob einen sehr schwer aussehenden Sack voller Gerätschaften auf, während sie antwortete, klang dabei aber kein bisschen angestrengt.