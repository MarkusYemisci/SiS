\WR{Sprunggeschwindigkeit in drei dutzend Sekunden, Captain}, meldete der Steuermann der \EN{Regenvogel}. Die Brücke bot einen fantastischen Anblick auf die grellweißen Lichterscheinungen, die nun um den Träger züngelten wie Flammen eines Lagerfeuers. 

\par

Captain Fiscale überprüfte ein letztes mal ihre Gurte. Dabei traf ihrer auf den freundlichen Blick Abdel Samads. Ihr erster Offizier wirkte genauso zuversichtlich wie immer, als er in eine bequemere Sitzposition rutschte. Auch er nestelte an seinem Anschnallgurt herum. Das Schiff wankte zwar kaum, doch jeder, der einmal einen Hyperraumsprung auf einem militärischen Schiff mitgemacht hatte, wusste, dass ein wenig zusätzlicher Halt nicht das letzte war, was man wollte.

\par

Hier und da mischten sich Farben in das Gleißen, dass die \EN{Regenvogel} nunmehr fast vollständig umschloss. Natalia Fiscale genoss den Anblick. Er hatte jedes mal etwas unwirkliches auf sie. So, als drängten die komplexen mathematischen Formen, welche die Grundlage jeder Sprungberechnung darstellten, tatsächlich in die Realität. Als ob sich das Abstrakte mit dem physikalischen mischen würde.

\par

\WR{Ich hasse das}, flüsterte Abdel Samad.

\par

Fiscale warf ihm einen verwunderten Blick zu. \WR{Sprünge?}

\par

\WR{Ja}

\par

\WR{Das hast du mir niemals gesagt. Ich dachte immer, die würden dir nichts ausmachen.}

\par

Ihr erster Offizier lächelte nun breit. \WR{Schön, dass ich dich noch überraschen kann.}

\par

Der Steuermann setzte sich ein Headset auf und öffnete dann eine Leitung, die auf dem ganzen Schiff hörbar war. \WR{Achtung, hier ist die Brücke. Wir haben Sprunggeschwindigkeit jetzt erreicht Hyperraumsprung in vier, drei, …}

\par

Fiscale streckte fast schon unwillkürlich ihre Hand aus und ergriff die ihres ersten Offiziers. Dabei waren beide erleichtert darüber, dass der Rest der Brückebesatzung nun geschlossen zum Bug hin sah und diese Geste der Nähe nicht wahrnahm.

\par

\WR{Zwei, eins, \textit{jetzt}!}, schloss der Steuermann und zog einen großen, silbernen Hebel ganz zu sich heran. Dann verschwand die \EN{Regenvogel} in einem Blitz.
\ortswechsel
\WR{Du bist so ruhig}, stellte Kevin Wilson unnötigerweise fest, während sein Flügelmann noch einmal die Gurte seines Alkovens festzog.

\par

\WR{Sagen wir einfach, ich bin nicht gerade froh darüber, mit einem Terroristen in einer Staffel zu fliegen}, antwortete Morten schließlich.

\par

Kevin seufzte. \WR{Ich bin jedenfalls froh, mit \textit{dir} zu fliegen. Wir zeigen es diesen Drecksäcken gleich so richtig! Wenn wir mit denen fertig sind, suchen die auf Kreuzpunkt an Ostern das nächste mal keine Eier, sondern Shutek-Trümmer im Schnee.}

\par

\WR{Auf Kreuzpunkt wird kein Ostern gefeiert}, erinnerte Morten kühl.

\par

Kevin lachte nur und begann kontrolliert ein und wieder aus zu atmen. Morten hatte mittlerweile zumindest ansatzweise erkannt, was sich hinter der heißen Luft verbarg, die sein Flügelmann permanent versprühte. Angst.

\par

So wie er sie selbst hatte. Und damit schaffte er es kaum noch, sich über Kevins draufgängerisches Verhalten zu ärgern. Sie beide versteckten ihre Furcht auf ihre ganz eigene Weise.

\par

\WR{Hyperraumsprung in vier, drei, zwei, eins, \textit{jetzt}!}

\par

Sofort fühlte Morten einen Druck auf seinem Kopf. Er kniff die Augen zu und fragte sich sogleich, wieso er das überhaupt konnte. Wie beim Sprung nach Pollux verschwand die gewohnte Wirklichkeit um Morten herum und es erfüllte ihn mit Panik, dies mitzuerleben. Als würde die ganze Welt um ihn herum auf tiefster Ebene einfach auseinandergebaut werden, wichen die Innenräume der \EN{Regenvogel} der Dunkelheit.

\par

Doch Morten wurde schnell klar, dass es keine Schwärze war, wie er sie kannte. So wie die Nacht und der Schatten, die ihrerseits nur die Abwesenheit von Licht darstellten. Stattdessen schien nun einfach alles zu fehlen. Es gab keinen Weg mehr, den man bloß wegen der ausgegangenen Straßenlaterne nicht mehr sah und der früher oder später im Sternenlicht wiederfand. Man war verloren. Der einzige Rest im Nichts.

\par

Unter Umständen hätte Morten diese Vorstellung sogar angenehm empfinden können. Für einen Moment alleine sein. Abgeschnitten von der ganzen Welt. Frei schwebend in der Leere. Doch die Leere war erdrückend. Ihr fehlte es nicht nur an Substanz, sondern auch an Raum.

\par

Dann kehrte ein Umfeld zurück. Aber es war nicht die \EN{Regenvogel}. Die Umgebung war für Morten auch nicht wirklich zu erkennen. Er kam sich vor, als lese er ein Buch und die beschriebene Welt nehme nur langsam und unvollständig Gestalt an. Als könne sich sein Verstand nicht entscheiden, welche Instanz der Begriffes \Wr{Fels}, \Wr{Wasser} oder \Wr{Luft} es sich nun vorstellen sollte.

\par

Aber das Gefühl, beobachtet und verfolgt zu werden wurde mit einem mal sehr real und ausgesprochen eindeutig. Hastig versuchte er, sich umzusehen. Aber er sah nichts. Er konnte nicht wahrnehmen, was ihn verfolgte, weil er keinen Begriff kannte, um es zu beschreiben. Alles, was er in diesem Moment tatsächlich kannte, war seine Angst. Seine Reaktion auf dieses Unbekannte.

\par

Schließlich fragte er sich, ob er davonrannte, konnte aber nicht einmal seine Beine sehen. Dann stoppte er hastig. Vor ihm tat sich ein Abgrund auf an dem nicht das fürchterlichste war, dass er wie ein Schlund auf Morten wirkte, sondern eher, dass er tatsächlich \textit{sehen} konnte, dass das Loch vor ihm keinen Boden hatte. Es ging einfach weiter und weiter und am Ende starrte ihm die Unendlichkeit entgegen. Er war in die Enge getrieben worden von etwas, dass er weder verstehen noch erkenne konnte. Wütend fuhr er herum.

\par

Dann riss die zurückkehrende Wirklichkeit den Abgrund und das jagende Unbekannte davon. Der Pilotenraum der \EN{Regenvogel} nahm wieder Form an. Im Gegensatz zu Pollux war Morten jedoch sofort ganz bei sich. Er riss sich die Gurte förmlich vom Leib und stampfte auf den Boden.

\par

Noch vor allen anderen stand er bereit da und setzte sich seinen Helm auf. Dann half er einem noch etwas benommenen Kevin aus seinem Alkoven. Im selben Moment grellte ein lauter Alarmton durch das ganze Schiff und einige Lampen sprangen an und tauchten Korridore und Räume in ein tiefes rot. Weder Kevin noch Morten mussten etwas dazu sagen. Ihnen war beiden klar, was diese Signale bedeuteten. Der Feind stand ihnen praktisch schon auf den Zehen.

\par

\WR{Alles klar, Murphy}, begann Morten, griff fest nach Kevins Hand und zog ihn aus seinem Alkoven. \WR{Lass uns ein paar Shutek abknallen!}