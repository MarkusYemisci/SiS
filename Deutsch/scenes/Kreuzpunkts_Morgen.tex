\WR{Wissen wir endlich, was das für eine Explosion war?}, fragte Captain Fiscale, diesmal deutlich wütender.

\par

Wieder einmal war der Raum um die \EN{Regenvogel} herum von heftigen Schusswechseln erfüllt. Doch auch in weiterer Ferne war zu erkennen, wie sich Kreuzpunkt Primus wehrte. Von der Oberfläche des Planeten aus, stiegen permanent neue Strahlenbündel und Boden-Raum-Raketen auf. Gemeinsam schlugen sie immer wieder auf den Schutzfeldern und Rümpfen der Shutek ein, die nun im tiefen Orbit des Planeten in der Falle saßen.

\par

Ihre Schiffe waren in Reichweite der planetaren Abwehranlagen geraten. Vermutlich in der Hoffnung, dass diese mittlerweile von der Bodentruppen ausgeschaltet worden wären. Doch die Front der Shutek war fast ausnahmslos im blauen Feuer der Nullzonenbomben verbrannt. Trotz der Höhe von gut einhundertfünfzig Kilometer, welche die \EN{Regenvogel} von der Oberfläche trennte, war das Nachglühen der Explosionen nach wie vor zu sehen. Langsam schob sich eine dichte Oberfläche über das verheerte Land, als wolle der Planet seine klaffende Wunde zudecken.

\par

Elshe Schwarzschild klebte förmlich an ihren Binokularen. Schließlich sagte sie: \WR{Ja und nein. Ich kann mittlerweile sicher sagen, dass es sich nicht um ein Schiff gehandelt hat. Was auch immer da hochgegangen ist, war stationär.} Die Radarchefin streifte sich durch ihre feuerroten Haare. \WR{Und im Moment der Detonation gab es einen heftigen Energieanstieg, den nur unsere Hyperraumsensoren aufgefangen haben.}

\par

Natalia Fiscale war versucht, zu Schwarzschilds Station hinab zu steigen, auch wenn Naturwissenschaften aller Art niemals ihre Stärle gewesen waren. Doch eine Meldung des Kommunikationsoffiziers ließ sie an Ort und Stelle verharren. \WR{Madam, unsere Nullzonentraciever funktionieren wieder.} Der junge Mann schien sich selbst kaum zu glauben. \WR{Ich habe gerade Kontakt zum Admiralsstab auf der Erde aufgenommen.}

\par

Keinen Augenblick später hing Captain Fiscale Wallander Stellvertreter praktisch auf der Schulter. \WR{Geben Sie Ihnen sofort einen Lagebericht durch!}

\par

\WR{Schon geschehen}, meldete der Kommunikationsoffizier. \WR{Gemeldet wird: die Panta Rhei ist samt Geleitschutz auf dem Weg hierher! Voraussichtliches Eintreffen in dutzend sechs Minuten.}

\par

Wer mitgehört hatte, atmete nun erleichtert auf. Captain Fiscale hingegen wandte sich erneut der taktischen Übersicht zu. Nach wie vor tobte der Kampf. Aber die Shutek hatten sich durch ihren Angriff zwischen eine Zange, deren eine Seite die Blockade der Starforce und die andere die planetaren Geschütze bildeten. Mit einem mittleren Träger wie der Panta Rhei könnte sich das Blatt in der Schlacht schnell wenden.

\par

\WR{Befehl an die Flotte}, begann sie in Richtung von Wallander Vertreter. \WR{Angriffsformation einnehmen. Langsamer Vorstoß. Wir müssen jetzt vor allem ihre Zerstörer und Kreuzer binden. Zusammen mit den Kräften der Panta Rhei können wir sie überwältigen.}

\par

Der Kommunikationschef gab die Befehle durch, während Maas Pertraca schon meldete: \WR{Captain, vier neue Kontakte sind gerade eben in unsere Radarreichweite eingetreten. Zwei unserer Jäger, einer der planetaren Staffeln und ein Schiff ohne Markierungen.}

\par

Es musste kein Funkkontakt hergestellt werden, um die Kapitänin wissen zu lassen, wer sich da gerade im Anflug befand.

\par

\WR{Holen Sie sie an Bord und schicken Sie Witwer und Wilson auf die Brücke.}
