Schon wenige Minuten später hatten sich Anna und Morten auf der Brücke eingefunden. Morten hatte gerade genügend Zeit gehabt, seinem Kameraden Kevin zu erklären, dass es ihm gut ging. Dieser schien sich wirklich Sorgen gemacht zu haben und Morten hatte sich sogar dazu hinreißen lassen, sein Kostüm der Gleichgültigkeit abzulegen.

\par

Für Morten war es der erste Besuch auf der Brücke. Er konnte nicht anders als sich erst einmal neugierig umzusehen. Der freie Blick ins All war wirklich eine Aussicht, die es in sich hatte. Aber auch die vielen Stationen verliehen dem Kommandozentrum eine scheinbare Größe, die der Raum so gar nicht hatte. Auch wenn der zwanzig Jahre alte Träger nicht mehr die neusten Apparaturen hatte und viele Monitore Flachbildschirme anstatt durchsichtiger Glasscheiben waren, gaben die Geräte der Brücke den Anschein eines echten Nervenzentrums.

\par

Zum ersten Mal erkannte Morten den blauhaarigen Kommunikationsoffizier, der gerade eifrig damit beschäftigt war, seine drei Untergebenen zu delegieren. Zwei von ihnen hatten ihre Station auf der anderen Seite der Brücke und waren hauptsächlich für die Intersystemkommunikation und Verständigung unter Grokampfschiffen verantwortlich.

\par

Der Offizier, der sich um den Kontakt über die Sternensysteme hinaus kümmerte, meldete mit unverhohlener Nervosität: \WR{Immer noch kein Kontakt zum Rest der Flotte und zum Hauptquartier, Madam. Lediglich die Kurzstreckenblockade haben wir mittlerweile geknackt.}

\par

Die Angesprochene, Captain Fiscale selbst, war gerade dabei eine Runde durch die Brücke zu drehen und ihren Offizieren über die Schulter zu schauen. Den einen schien das egal zu sein, andere wurden dadurch offenbar nervös, was Morten gut nachvollziehen konnte.

\par

Aber als Captain Fiscale Morten und Anna sah, kam sie sofort auf die beiden zu. Auch Commander Samad hatte den Neuling erspäht und erhob sich nun aus seinem Sessel.

\par

\WR{Major}, begann Captain Fiscale, \WR{bitte berichten Sie, was dort draußen geschehen ist.}

\par

Anna erzählte in allen Einzelheiten, wie sie die \EN{Virial} gefunden hatten und anschließend vor den Angreifern geflohen waren. Dabei ließ sie es nicht aus, Morten für sein fliegerisches Können zu loben und zu erwähnen, dass er einen halbfertigen Scan einer der Gegner angefertigt hatte. Commander Samad sah ihn mit einem Blick an, den er so noch von wenig anderen Offizieren an Bord kannte. Offenbar war er wirklich beeindruckt.

\par

Captain Fiscale schien über diese Ereignisse zumindest teilweise informiert gewesen zu sein. Anna hatte sich wahrscheinlich bereits über Funk mit der \EN{Regenvogel} unterhalten, sobald der Kontakt wieder hergestellt gewesen war.

\par

\WR{Irgendeine Ahnung, wer Sie angegriffen haben konnte?}, fragte sie abschließend.

\par

Anna schüttelte den Kopf: \WR{Leider nicht. Aber Lieutenant Wittwers Scan könnte uns weiterhelfen. Auch wenn er nicht vollständig ist.}

\par

\WR{Gute Arbeit}, lobte Captain Fiscale beiläufig in Mortens Richtung.

\par

Commander Samad trat an die beiden heran. Er wirkte seltsam gefasst. Fast so, als ob gar nichts gewesen wäre. Captain Fiscale schien wiederum nicht sonderlich ruhig, gab sich aber alle Mühe, gelassen und entspannt zu wirken.

\par

\WR{Wie genau haben die Angreifer ausgesehen?}, wollte Commander Samad wissen.

\par

Morten antwortete vor Anna: \WR{Ihre Jäger~-- falls es welche waren~-- sahen merkwürdig aus. Es fing schon bei den Hüllen an. Ihr Glanz. Eine derartige Legierung habe ich bisher noch bei keinem anderen Schiff gesehen. Außerdem hatten sie keine Cockpits aber irgendwelche Dinger, die wie Augen aussahen…}, er unterbrach sich aufgeregt, um die richtigen Worte zu suchen. Zu seiner Lehrzeit hatte er keine Schwierigkeiten gehabt, sich gleich eloquent zu artikulieren. \WR{Ich glaube nicht dass es menschliche Konstruktionen waren, Sir.}

\par

\WR{Was sollen Sie denn sonst gewesen sein?}, blaffte Lieutenant Wallander, der Mortens Ausführungen angehört und sich der Gruppe angeschlossen hatte.

\par

Abdel Samad wandte sich ihm zu und gab zu bedenken: \WR{Vielleicht war an Marco Bellendis Außerirdischen doch mehr dran, als wir dachten.}

\par

\WR{So ein Unsinn!}, bellte Wallander fast wütend. \WR{Die Union hat Siedlungen auf mehr als zwanzig Welten. Und nirgendwo haben wir etwas gefunden, dass größer als ein Adler war. Bellendi hat gefälschte Beweise für die Existenz von Aliens vorgelegt und damit hat er sich in der wissenschaftlichen Welt zum Gespött gemacht.}

\par

Auch Commander Samad war nun alles andere als ruhig als er entgegnete: \WR{Allein die Vorstellung, Menschen wären die einzigen intelligenten Wesen ist lächerlich. Und wenn einer unserer Piloten sagt, er habe etwas gesehen, dann hat er etwas gesehen. Oder waren Sie da draußen?}

\par

Morten gab sich Mühe, neutral zu wirken. Er kannte die Dynamik zwischen den beiden Brückeoffizieren nicht. Aber es hätte ihn auch überrascht, wenn es auch im Kommandozentrum der \EN{Regenvogel} keine Animositäten zwischen der ein oder anderen Person gegeben hätte.

\par

Samads Gesicht begann indessen rot anzulaufen. Captain Fiscale machte sich bereits zum einschreiten bereit, als jemand die Treppen zur Brücke hinaufgerannt kam.

\par

Sein Overall wies ihn als Techniker vom Flugdeck aus. Der Mann, wie Morten sofort erkannte war es Tukarev, wedelte schon von weitem mit zwei Zippo-Disketten in der Hand herum. Polternd kam er zu den Gruppe hinauf gestürmt. Bei den breiten Schultern des Mannes wunderte sich Morten, dass er die Treppe nicht seitwärts hinaufgehen musste.

\par

Als er oben angekommen war begann er, ohne jemanden vorher zu grüßen, zu sprechen: \WR{Hier sind die Daten der Logbuchboje der \EN{Virial} und der Abtasterergebnisse von grün zwei.}

\par

Captain Fiscale nahm die Disketten in Empfang. Gerade als sie etwas sagen wollte, ob zum Dank oder zum Tadel donnerte Tukarev bereits wieder davon.

\par

\WR{Entschuldigen Sie mich bitte}, rief er laut zum Abschied. \WR{Unten im Hangar, ist die Hölle los.}

\par

Morten konnte sich vorstellen, dass der Büffel tatsächlich die Wahrheit sagte. Von der Brücke aus war zu erkennen, wie permanent Jäger starteten und landeten. Außerdem war zu sehen, dass die doppel- und einzelläufigen Geschütztürme der \EN{Regenvogel} begonnen hatten, ständig hin und her zu schwenken, was eine aktive Zielsuche bedeutete.

\par

Im unteren Bereich des Brücke waren die vier Kanoniere damit beschäftigt, die Waffen zu kontrollieren. Sie saßen an vier, in nach Back- und Steuerbord angeordneten Konsolen, welche die Kanonen der jeweiligen Richtung kontrollierten.

\par

Morten und Anna folgten Captain Fiscale zur Kommunikationskonsole, wo sie Lieutenant Wallander die Zippodiskette überreichte. \WR{Das sind die Flugschreiber-Daten der \EN{Virial}. Major Farley hat sie von einer Logbuchboje in der Nähe des Wracks des Schiffes heruntergeladen. Bitte starten sie die Wiedergabe}, wies Captain Fiscale an.

\par

Wallander, der sich mittlerweile zumindest wieder halbwegs beruhigt hatte, nickte und schob den Datenträger in ein Lesegerät, das in seiner Konsole integriert war. Auf der Monitorwand erschienen verschiedene Anzeigen. Anscheinend war die Logbuchboje voll mit den verschiedensten Daten gewesen. Die niedergeschriebenen Aufzeichnungen des Kommandanten, Abtastergebnisse, Flugdaten und zu guter Letzt auch einige Kommuniqués.

\par

\WR{Was sind das für Nachrichten?}, wollte Captain Fiscale wissen und deutete auf die gespeicherten Videoübertragungen.

\par

Wallander fuhr sich durch seine blauen Haare und strich sich dann durch seinen, ebenfalls blauen, Jägerbart. \WR{Diese Nachrichten wurden auf einem langwelligen Hyperraumband versendet. Sie sollten wahrscheinlich an einen Empfänger weit außerhalb von Arktur gehen. Aber wenn ihre Kommunikation genauso gestört wurde, wie unsere, dann wundert es mich nicht, warum niemand die Rufe erhalten hat.}

\par

\WR{Spielen Sie sie ab!}, befahl Captain Fiscale hastig.

\par

In Morten wuchs die Anspannung. Anna schien es genauso zu gehen. Wahrscheinlich enthielten diese Kommuniqués Antworten auf die Frage, was mit der \EN{Virial} genau passiert war und Morten hatte bereits genug davon gesehen.

\par

Und schon begann die Wiedergabe. Sie zeigte zwei Männer, die in eine Kamera blickten. Da die Forscher keine Uniformen trugen war es schwer zu sagen, wer der Kommandant war. Aber nach den Gehröcken der beiden zu urteilen, war der stehende Mann der Kapitän. Er war es auch, der sprach: \WR{An die Leitung der Pinnacle Science Group. Hier ist Johann Krüger, Kommandant der \EN{Virial}. Wir haben hier draußen etwas gefunden! Rein zufällig als wir das System mit unseren Teleskopen abgesucht haben. Es muss ein Raumschiff sein! Aber es ist auf keinen Fall ein menschliches. Vielleicht haben wir tatsächlich den ersten Kontakt zu einer außerirdischen Spezies geknüpft. Wir werden versuchen, mit dem Omega dutzendeins Protokoll eine Kommunikation zu ermöglichen. Ende des Berichts.}

\par

Morten schluckte. Damit waren wohl die letzten Zweifel ausgeräumt. Wer oder was auch immer auf ihn geschossen hatte, es war kein Mensch gewesen.

\par

Das zweite Kommunique folgte sofort: \WR{Hier ist Johann Krüger von der \EN{Virial}. Wir haben festgestellt, dass die Kommunikation aus dem System heraus durch irgendwas gestört wird. Unsere Techniker glaube, es liegt daran, dass die Nullzonen-Gegenstelle auf Pollux aus unbekannten Gründen nicht erreichbar ist aber wir werden periodisch versuchen diese Nachricht zu senden. Wir hoffen, am Ende erhalten Sie diesen Bericht. Wir sind jetzt sehr dicht an das andere Schiff herangeflogen. Der Abstand beträgt nur noch knapp einhundert Meter. Wir können es bereits deutlich sehen.}

\par

Tatsächlich waren auf der Aufzeichnung unscharfe Umrisse von etwas großem zu erkennen, das in der Nähe der \EN{Virial} schwebte.

\par

\WR{Das Schiff ist groß. Etwa fünfhundert Meter lang. Wir glauben, dass wir gleich zu Beginn von ihnen abgetastet wurden, sind uns aber nicht sicher. Die Technologie, die sie dafür verwenden, gibt uns noch Rätsel auf. Das Schiff selbst ist genauso fremdartig. So wie wir das sehen sind die einzigen Dinge, die unseren Schiffen ähnlich sind, die Turbinen und einige Einlassungen, die wie Fenster aussehen.

\par

Wir versuchen bereits seit einigen Stunden, eine Verbindung zu ihnen aufzubauen. Wir haben bereits alle Register gezogen. Selbst die Daten von Omega dutzendeins im Binärcode mit unseren Positionslichtern durchzugeben, haben wir versucht. Aber bislang haben sie keine Versuche gemacht, zu antworten.}

\par

Anna sah Morten besorgt an. Beide hatten schon eine düstere Vorahnung, wie die Antwort der Außerirdischen ausgefallen sein würde.

\par

Wieder begann eine Übertragungsaufzeichnung. Diesmal schien Johann Krüger und auch der Kommunikationsoffizier angespannt. Die Umrisse des anderen Schiffes waren nicht mehr zu erkennen.

\par

\WR{Dritter Zwischenbericht. Wir haben eine beunruhigende Entdeckung gemacht, als wir eine Abtastung der Unbekannten gewagt haben. Zunächst einmal haben wir riesige Tanks vollgefüllt mit einer Flüssigkeit mittschiffs entdeckt. Das alleine hat uns noch keinen Schrecken eingejagt und war allenfalls für unsere Ingenieure interessant. Aber dann haben wir mehrere Vorrichtungen erfasst, die wahrscheinlich Kanonentürme sind. Genau wissen wir es nicht, denn niemand hier hat Erfahrung mit militärischer Ausrüstung. Aber egal, was diese Geräte sind, sie haben sie auf uns ausgerichtet! Wir kehren jetzt zum Sprungpunkt zurück. Sie folgen uns zwar aber scheinen geringfügig langsamer zu sein als wir, daher werden wir es wohl schaffen. Ende des Berichts.}

\par

Mortens Augen verengten sich. Er ertappte sich dabei, wie er der Besatzung der \EN{Virial} die Daumen drückte und hoffte, dass sie es schaffen würden. Aber er selbst wusste am besten, dass dies nicht passieren würde.

\par

Eine weitere Übertragung begann. Johann Krüger hatte dicke Augenringe und wirkte wie zwanzig Jahre älter. Diesmal saß er selbst vor der Kommunikationskonsole. Der andere Mann war verschwunden. \WR{Vierter Bericht. Wir haben den Sprungpunkt beinahe erreicht. Aber dieses andere Schiff hat kleinere Raumer gestartet, wahrscheinlich Jäger. Sie werden bald bei uns sein.

\par

Das ist nicht alles. Wir haben mehrere stark komprimierte Stoßsignale vom unbekannten Schiff erhalten. Ich glaube, sie versuchen endlich eine Verständigung zu uns aufzubauen. Es war nicht leicht aber ich habe die Besatzung überredet, das Schiff anzuhalten und eine Antwort auf ihre Signale zu versuchen. Bislang hatten wir noch keinen Erfolg damit, die fremde Syntax zu verstehen aber ich hoffe, dass wir es bald schaffen werden.

\par

Die Besatzung wird ungeduldig. Sie glauben, wir könnten springen, bevor uns die kleineren Raumer erreichen aber ich sehe das anders. Wir müssen versuchen, Kontakt mit ihnen zu bekommen. Ich bin überzeugt davon, dass wir sonst ein schlimmes Missverständnis verursachen. Ende des Berichts.}

\par

Anna hielt sich vor Schreck die Hand vor den Mund. Sie stöhnte leise vor Entsetzen. Als die nächste Übertragung einsetzte, weitete sich der Schrecken auch auf ihre Augen aus. Die \EN{Virial} schien ständig erschüttert zu werden. Entladungen von Strahlenwaffen zuckten an den großen Fenstern der Brücke vorbei und der Eindruck als würde das Bild von Wellen durchzogen deutete auf Einschläge gegen die Schutzfelder hin.

\par

\WR{Notruf, Notruf! Hier ist die \EN{Virial}. Wir werden von den Unbekannten angegriffen. Bitte schicken Sie sofort Hilfe!}

\par

Diesmal hatte jemand anderes gesprochen. Er wirkte jünger als Krüger und trug einen abgenutzten Gehrock. Doch der Kommandant der \EN{Virial} drängte sich sogleich zur Konsole. Die Stimmen klangen gedämpft, als sich die beiden, weit vom Mikrofon entfern, zu streiten begannen.

\par

\WR{Verdammt machen Sie die Konsole frei, Nidal. Wir müssen endlich Kontakt zu Ihnen herstellen! Das ist unsere einzige Chance!}

\par

\WR{Vergessen Sie’s} brüllte der andere Mann und stieß Krüger beiseite. Wieder an auf die Konsole gerichtet schrie er förmlich ins Mikrofon: \WR{Notruf an alle! Hier ist das Forschungsschiff \EN{Virial}. Wir werden angegriffen. Unsere Koordinaten lauten: Eins drei acht Grad, zu zwei drei zwei Grad zu neun eins acht. Bitt schicken Sie sofort…}

\par

Die Übertragung brach ab. Mortens Augen wurden feucht und er ballte die Hände zu Fäusten. Das war also das der Anfang des Endes der \EN{Virial}. Aber es schien noch eine weitere Aufzeichnung zu geben.

\par

\WR{Hier ist Jonathan Nidal von der \EN{Virial}}, im Hintergrund waren Asteroiden zu erkennen. \WR{Krüger ist tot. Kiowa ebenfalls. Die kennen keine Gnade. Wir haben sie angebettelt das Feuer einzustellen aber das hat sie nicht interessiert. Ich glaube nicht, dass es ein Missverständnis war. Wir haben alles getan, um sie nicht zu provozieren.

\par

Jetzt verstecken wir uns im Asteroidenfeld. Ich hoffe irgendjemand erhält diese Übertragung, auch wenn das Asteroidenfeld nicht gerade dabei hilft, die Kommunikationsblockade zu überwinden. Es ist noch viel schwerer, von hier aus eine NZT-Verbindung herzustellen. Wir versuchen jetzt, andere Systeme. Kreuzpunkt Primus ist am besten ausgestattet und daher unsere erste Wahl. Aber hier sind wir vorerst sicher, glaube ich. Man wird die Starforce anweisen nach uns zu suchen. Ich war immer Pazifist aber jetzt bin ich ganz froh, dass wir hoffentlich ein paar Strahlenkanonen auf unserer Seite sehen werden. Ich hoffe nur, sie brauchen nicht so lange.}

\par

Im Hintergrund geschah etwas. Ein Mann rannte auf Nidal zu und deutete dabei durch das Glas der Brücke. Irgendetwas bewegte sich im Asteroidenfeld aber es war zu unscharf, um es zu erkennen. Nidal wirbelte herum.

\par

\WR{Oh nein}, sagte er nur. Dann hämmerte er förmlich auf seine Konsole ein. \WR{Ich lade alle Daten in die Logbuchboje und schieße sie raus.}

\par

Nach einem weiteren Blick durch das Fenster krächzte er: \WR{Das war’s, jetzt sind wir erledigt.}

\par

Und schon begann die \EN{Virial} zu erbeben. Selbst durch die Unschärfen hindurch war zu erkennen, wie eine riesige Explosionswolke Teile des Schiffes einzuhüllen schien. Nidal wurde aus dem Sitz gerissen. Irgendetwas schoss auf die Brücke. Grünliche Strahlen durchbohrten die Glaswände. Da die \EN{Virial} kein Kriegsschiff war, hatte sie auch keine eigenen Schutzfelder für die Kommandozentrale. Das Sicherheitsglas zerbrach unter der zu groß gewordenen Belastung und einige Besatzungsmitglieder wurden ins All hinaus geblasen.

\par

Nidal schaffte es gerade noch, eine Taste zu drücken, bevor er mit einem Schrei voller Verzweiflung ebenfalls in den Weltraum gerissen wurde. Danach erlosch das Bild.

\par

Lange Zeit blieb es ruhig. Captain Fiscale hatte inzwischen Halt an der Rückenlehne von Wallander Stuhl gesucht. Alle starrten nur ungläubig auf den Monitor, der gerade noch die schrecklichen Bilder gezeigt hatte.

\par

Es kam allen wie eine Ewigkeit vor, bis Commander Samad die Stille endlich beendete. \WR{Wir sollten das System sofort verlassen, Madam. Wir wissen jetzt, was mit der \EN{Virial} passiert ist und damit ist unsere Mission zu Ende.}

\par

\WR{Das kommt überhaupt nicht in Frage}, donnerte Wallander sofort. \WR{Wir müssen zuerst genau wissen, wer das getan hat. So lange wir die Identität der Angreifer nicht kennen, ist unser Auftrag nicht beendet.}

\par

\WR{Und was wollen Sie tun?}, fragte Samad den Kommunikationsoffizier provokant. \WR{Diese Kerle fragen wo sie herkommen. Sie um die Koordinaten ihres Heimatsystems bitten?} An Captain Fiscale gewandt, hängte er an: \WR{Madam, es spielt keine Rolle, wer diese Angreifer sind. Wir haben keinen Kontakt zum Hauptquartier. Wir müssen das System sofort verlassen und berichten, was hier passiert ist. Sonst wird der nächste Träger geschickt, der hier ebenfalls in die Falle geht.}

\par

Fiscale schien dem still zuzustimmen. Morten war insgeheim auch sehr dafür, schnellstens zu verschwinden. Am liebsten hätte er sich gleich in irgendeinen sprungfähigen Jäger geschwungen um sich abzusetzen. Aber das war nicht möglich und außerdem wollte er seine Freunde auch nicht alleine lassen.

\par

\WR{Steuermann}, begann die Kommandantin, \WR{setzen Sie Kurs auf den Sprungpunkt nach Pollux. Maximale Geschwindigkeit. Wir springen sofort, wenn wir dort sind, keine Verzögerung.} An den Kommunikationsoffizier Nils Wallander gewandt befahl sie: \WR{Informieren Sie alle Jägerstaffeln, dass sie sofort zurückkehren und landen sollen. Verteidigungs- und Abfangjäger bleiben aber im Einsatz und landen erst kurz bevor wir springen. Die beiden Korvetten sollen ihre Geschwindigkeit nicht auf die unsrige drosseln sondern sich so schnell wie möglich zum Sprungpunkt begeben. Informieren Sie die Kommandanten der \EN{Celestina} und der \EN{Marianna} über die Situation und weisen Sie sie an, die gewonnen Daten ans Kommando zu übertragen, sobald sie in Pollux sind.}

\par

Lieutenant Wallander bestätigte den Befehl und gab dann einige Instruktionen an einen seiner Untergebenen weiter, der für die Schiff-zu-Schiff-Kommunikation zuständig war.

\par

Indes ging Captain Fiscale zur Offizierin für Radarüberwachung und reichte ihr die zweite Disc mit den Worten: \WR{Frau Schwarzschild, untersuchen Sie bitte die Daten auf dieser Diskette. Darauf befindet sich ein Scan von einem der Angreifer. Er ist nicht vollständig aber vielleicht hilft er uns trotzdem weiter.}

\par

\WR{Aye, Madame}, antwortete die Offizierin eifrig und legte die Diskette ein. Sie wirkte noch sehr jung, obwohl sie mit ihrem Rang mindestens fünfundzwanzig sein musste. Aber ihre glatte Haut, ihr jugendliches Gesicht und die, säuberlich zu einem Zopf gebundenen, knallroten Haare ließen sie keinen Tag älter als achtzehn aussehen. Immer wieder rückte Elshe Schwarzschild auf ihrem Stuhl hin und her, während sie die Daten bearbeitete.

\par

Morten und die anderen, die Captain Fiscale zur Hauptradarstation gefolgt waren, sahen nun erwartungsvoll auf den Bildschirm. Gleich würden sie einen Blick auf die Angreifer werden können.

\par

Kurze Zeit später erschien eine schematische Darstellung der Jäger, die Morten mit eigenen Augen gesehen hatte. Was aber länger dauerte war Elshes Analyse der Daten. Immer wieder sah sie Archivdaten im Computer ein, während sie die einzelnen Bauteile auf der vereinfachten Darstellung des Jägers zu identifizieren versuchte. Morten war bestimmt kein Spezialist für die Deutung von Abtastungen aber was er wusste reichte aus um ihm zu sagen, dass der Aufbau dieses Jäger nichts ähnelte, was die Starforce bisher kannte.

\par

\WR{Eigenartig}, begann Elshe schließlich nach mindestens einer Viertelstunde. \WR{Ich kann zwar einige Bauteile erkennen, wie zum Beispiel diesen Reaktor hier}, sie deutete auf einen großen, rot blinkenden Zylinder, \WR{aber vieles was man in einem Jäger vermuten würde, fehlt hier einfach.}

\par

\WR{Wie zum Beispiel ein Cockpit}, murmelte Morten.

\par

\WR{Tief im Rumpf, hinter einer besonders starken Panzerplatte gibt es eine Art Tank, der nicht zum Antrieb zu gehören scheint. Der ist mit einem System von, teils größeren, teils kleineren Rohren verbunden. In diesen Leitungen~-- meine Güte, einige sind nicht größer als eine Ader~-- fließt eine Flüssigkeit.}

\par

\WR{Wie wird dieses Ding gesteuert}, wollte Commander Pearson ungeduldig wissen. \WR{Ich sehe hier nirgendwo einen elektronischen Kreislauf oder so etwas wie einen Computer.}

\par

Lieutenant Schwarzschild sah noch einige male auf ihre Anzeigen, bevor sie unsicher antwortete: \WR{Das ist verdammt seltsam. Selbst ein Hovercraftmobil hat mehr Elektronik in sich wie dieser Jäger. Ich erkenne nur einige, nicht vernetzte Komponenten. Aber es scheint keinen Hauptrechner oder etwas in dieser Richtung zu geben. Keine Ahnung, wie dieser Jäger kontrolliert wird. Auf jeden Fall nicht so, wie es bei unseren Fliegern passiert.}

\par

Captain Fiscale beugte sich über den Monitor. Ungläubig sah sie auf die Anzeigen als wäre das Dargestellte ein dämlicher Jux. \WR{Vielleicht sieht es nur so aus, als würde etwas fehlen, weil der Scan nicht vollständig ist}, überlegte sie laut.

\par

Elshe Schwarzschild schüttelte sofort den Kopf. \WR{Das glaube ich nicht, Madam. Der Querschnitt ist komplett. Nur die Analysen der einzelnen Geräte fehlen noch.}

\par

\WR{Was ist mit den Waffen}, fragte Commander Samad ungeduldig.

\par

Die Radaroffizierin zuckte mit den Schultern, ohne noch einmal auf die Anzeigen zu sehen. \WR{Das kann ich beim besten Willen nicht sagen}, erklärte sie. \WR{Es könnte alles mögliche gewesen sein. Die Bauart der Geschütze findet sich in keiner Datenbank. Genauso wenig wie das schwache Hyperraumband, dass zwischen zwei der Angreifer bestanden zu haben schien.}

\par

\WR{Vielleicht eine Art Kommunikationssystem}, dachte Commander Samad laut. \WR{Wir benutzen Direktverbindungen über NZTs. Aber es gab schon etliche Wissenschaftler, die behaupteten, man könnte auch einfach durch den Hyperraum funken.}

\par

Captain Fiscale richtete sich wieder auf. Nach kurzem überlegen verkündete sie: \WR{Miss Farley, Mister Wittwer. Sie haben wirklich gute Arbeit geleistet. Die Daten der \EN{Virial} und der Scan helfen uns enorm. Von Ihrem fliegerischen Können ganz zu schweigen.}

\par

Nur ganz kurz warf sie Commander Samad einen zufriedenen Blick zu. Dieser erwiderte ihn mit einem echten Lächeln.

\par

\WR{Mister Wittwer}, begann Captain Fiscale erneut, \WR{hiermit erhalten Sie eine Belobigung. Ich rate Ihnen, sich erst einmal etwas auszuruhen, bis zu Ihrem nächsten Einsatz könnte die Zeitspanne kürzer sein, als Sie vielleicht denken. Wegtreten.}

\par

Morten salutierte zackig, warf Anna einen letzten Blick zu und verließ dann die Brücke.
