Morten hatte gerade geglaubt, den Sprung nach Pollux dieses mal ohne ein derartig albtraumhaftes Erlebnis wie auf dem Hinflug hinter sich gebracht zu haben. Zwar hatte ihn auch sein jüngster Sprung durch die Lichtmauer nicht unversehrt gelassen~-- sein Magen probte wie immer den Aufstand und sein Kopf kam ihm vor, als würde er bald platzen, wie ein Ballon im Vakuum~-- doch er hatte dieses mal keine grauenvollen Stimmen gehört oder schreckliche Bilder gesehen.

\par

Doch ein Albtraum blieb auch dieses mal nicht aus. Das erste was er sah, als er die Augen öffnete, war Anna Farley, wie sie Dexter Hennington leidenschaftlich küsste. Diese Szene ereignete sich derart nah an seinem Sicherheitsalkoven, dass Morten sich schon bedrängt fühlte.

\par

Dafür, dass es gerade erst einen Hyperraumsprung mitgemacht hatte, schaltete sein Gehirn ausgesprochen schnell. Anna Farley war wohl der Vorgesetzte, der Dexter immer wieder aus dem Schlamassel zog, wenn er mal wieder Mist gebaut hatte. Üblicherweise versuchte er es aus Rücksicht den Reinigungsmannschaften gegenüber zu verhindern, aber dieses mal hätte er am liebsten direkt auf den Boden gekotzt.

\par

Mortens Blicke mussten ihn verraten haben, denn als sich Anna wieder aus seiner Umarmung gelöst hatte, grinste ihn Dexter nur schadenfroh an und verschwand.

\par

Erst jetzt bemerkte Morten, dass irgendetwas nicht stimmte. Der Hauptalarm des Schiffs plärrte und die Warnleuchten tauchten den Sprungraum wie den Rest der \EN{Regenvogel} in ein tiefrotes Licht. Das bedeutete Alarmstufe eins und diese wiederum implizierte eine unmittelbare Bedrohung.

\par

Morten Nebenmann Kevin hatte sich gerade aus seinem Alkoven befreit und half nun Morten, seine Gurte abzunehmen. Leise flüsterte er seinem Freund zu: \WR{Tut mir leid, Mann. Man kann nicht immer gewinnen.} Ein leiser Seufzer blieb Mortens einzige Reaktion.

\par

Anna Farley schien ihren Staffelkameraden erst jetzt wahrzunehmen. Mit ernstem Gesichtsausdruck erklärte sie: \WR{Morten, wir sind in Schwierigkeiten. Pollux Primus liegt unter heftigem Beschuss. Ich habe noch nichts von der Brücke gehört aber ich wette, es sind dieselben, die auch uns in Arktur angegriffen haben. Gerade läuft der Alarmstart. Ich muss ebenfalls gleich raus. Sie wissen, was Sie in so einem Fall zu tun haben?}

\par

Morten nickte, immer noch ein wenig unter dem Einfluss des Sprungs. \WR{Ja, Madam. Ich bin keiner Notfallstaffel zugeteilt. Also halte ich mich auf dem Flugdeck bereit.}

\par

Anna nickte zufrieden und schenkte ihrem Staffelkameraden ein flüchtiges Lächeln, bevor sie die Treppe zum Startbereich hinab rannte. Kevin klopfte Morten freundschaftlich auf die Schulter und schubste ihn vorwärts. \WR{Auf geht's. Vielleicht lassen die uns auch raus. Dann können wir den Mistkerlen in den Arsch treten.}

\par

\WR{Du hast ja keine Ahnung}, entgegnete Morten bloß.

\par



\par

Gerade als sich die meisten erhofft hatten, der Stress und die Aufregung würden nachlassen, nachdem die \EN{Regenvogel} nach Pollux gesprungen wäre, wurden sie nun bitter enttäuscht. Die beiden Belgeitkorvetten des leichten Trägers hatten den Hyperraumeintrittspunkt aufgrund ihrer überlegenen Geschwindigkeit wesentlich früher erreicht. Sie waren es gewesen, die der \EN{Regenvogel} den Angriff auf Pollux gemeldet hatten, direkt nachdem diese das System ebenfalls erreicht hatte.

\par

Captain Fiscale lehnte sich gerade auf die Lehne von Lieutenant Wallander Sessel. Der Kommunikationsoffizier war gerade dabei, sämtliche Frequenzen auf Hilferufe zu untersuchen. Seinen Untergebenen hatte er aufgetragen, Kontakt mit der Raumstation \EN{Kaukasia} oder der Bodenkontrolle auf Pollux Primus herzustellen. Bisher hatten die beiden keinen Erfolg gehabt.

\par

\WR{Ich weiß nicht, Madam}, begann Wallander. \WR{Entweder der Feind stört schon wieder unsere Kommunikation wie schon in Arktur. Oder da unten ist einfach niemand mehr, der uns antworten könnte.}

\par

\WR{Bis vor kurzem jedenfalls hat es von Notrufen nur so gewimmelt}, hängte einer seiner Helfer an. \WR{Aber jetzt ist alles stumm geworden.}

\par

Unterdessen sprach Commander Samad mit Elshe Schwarzschild am Radar. \WR{Was meinen Sie damit, die \EN{Kaukasia} ist weg?}

\par

Die junge Offizierin seufzte und antwortete so sachlich, wie es ihr möglich war: \WR{Damit meine ich, dass sie vernichtet worden ist. Da wo die Station sein sollte, finden die Abtaster nur noch einen glühenden Trümmerhaufen.}

\par

Abdel Samad unterdrückte jede Reaktion, die er hätte geben können. Er war sich ziemlich sicher, dass keine davon den anderen viel Mut gemacht hätte. Schließlich fragte er: \WR{Was ist mit den Angreifern? Wie viele sind es?}

\par

\WR{Das versuche ich gerade herauszufinden, Sir}, antwortete Schwarzschild und drehte einen Regler an ihrem Instrumentenpult voll auf. \WR{Ich habe fünf bestätigte Kontakte. Alle haben die Ausmaße von Großkampfschiffen. So wie es aussieht bombardieren sie die Siedlung auf der Oberfläche. Außerdem erfasse ich mehrere Dutzend kleinerer Signale. Wahrscheinlich Jäger.}

\par

\WR{Wie zum Teufel sind die hier hergekommen? Es gibt nur zwei Sprungwege nach Pollux. Einen haben wir gerade genommen und der zweite führt tiefer in die Union. Jemand müsste diese Kampfgruppe doch bemerkt haben.}

\par

\WR{Vielleicht haben Sie hier gewartet und sich versteckt}, sinnierte Abdel Samad. Doch Elshe Schwarzschild belehrte ihn: \WR{Das ist unwahrscheinlich. Auf Pollux gibt es eine sehr gut ausgerüstete Überwachungsstation. Die hätten eine so große Flotte sicher bemerkt.}

\par

\WR{Madam!}, rief Wallander Captain Fiscale kurz darauf zu. \WR{Ich habe Funkkontakt mit einem Raumschiff der Union. Es ist die \EN{Pales fünf}, ein Getreidefrachter.}

\par

\WR{Lassen Sie das hören}, befahl Fiscale Fiscale, die es kaum erwarten konnte, endlich mehr zu erfahren.

\par

Die Lautsprecher neben Wallander Arbeitsplatz begannen zu quäken. \WR{\EN{Pales fünf} an \EN{Regenvogel}, bitte kommen.}

\par

\WR{Wir hören Sie}, erwiderte der Kommunikationsoffizier. \WR{Wir brauchen sofort Ihre Position und Ihren Status.}

\par

Der Kapitän der \EN{Pales fünf} zögerte keinen Sekundenbruchteil. \WR{Wir sind im Orbit um Pollux. Höhe: zweit Trinmeter. Unsere Situation ist beschissen! Die sind hinter uns her. Ich glaube nicht, dass da unten noch jemand lebt. Sie haben die Siedlung einfach gefressen und jetzt wollen sie uns auch noch abschießen.}

\par

\WR{Gibt es noch andere Überlebende?}, wollte Captain Fiscale sofort wissen.

\par

\WR{Kaum. Ein dutzend drei Schiffe haben den Start geschafft. Wir fliegen auf den Sprungknoten nach Wega zu. Aber ohne Hilfe schaffen wir es nicht.}

\par

Captain Fiscale nahm einen tiefen Atemzug. Das war eine Lage, auf die sie weder die Akademie, noch ihre jahrelanger Dienst vorbereitet hatte. Ein ganzer Planet wurde von einer überlegenen Streitmacht bombardiert und der einzige Fluchtweg war weit entfernt. Dennoch brauchte sie nicht lange, um sich zu entscheiden.

\par

\WR{Ist die Bravo Staffel draußen?}, fragte sie an Einsatzleiter Petrarca gewandt. Dieser brauchte nur einen kurzen Blick auf eine seiner Konsolen zu werfen, um dies nickend zu bejahen. \WR{Gut. Wallander, geben Sie Befehle durch. Sie sollen ihre Zusatzraketen anwerfen und so schnell wie möglich zur \EN{Pales fünf} fliegen. Das gilt für alle Staffeln außer unseren Geleitjägern und den zwei Bombern. Die bleiben in unserer Nähe, während wir den Weg nach Wega absichern.}

\par

\WR{Madam, ich glaube nicht dass die Bravo Staffel ausreicht}, warf Abdel Samad der Kommandantin entgegen. \WR{Und andere Staffeln schaffen es vielleicht nicht rechtzeitig.}

\par

Während er das sagte gab gerade Lieutenant Wallander die Befehle an die Abfangjäger weiter, die daraufhin mit maximaler Geschwindigkeit auf Pollux zu rasten. Ihre grell leuchtenden Zusatzraketen waren noch zu erkennen, als die Jäger selbst schon lange nicht mehr zu sehen waren.

\par

\WR{Vielleicht sollten wir eine der Corvetten abstellen}, schlug Samad vor. \WR{Sie könnte einen Kurs durch den flachen Hyperraum zu den Frachtern berechnen und Jäger in ihrem Kielwasser mitnehmen.}

\par

Die Kommandantin zog den Vorschlag einen Augenblick lang in Betracht, verwarf ihn aber wieder. \WR{Unsere eigene Position darf nicht zu angreifbar werden. Und wenn der Feind sieht, dass wir ihm Großkampfschiffe entgegen schicken, wird er sich sofort auf uns stürzen.} Sieh deutete auf die holografische Übersicht des Raums um Pollux, die gerade vor der Station des Einsatzleiters dargestellt wurde. \WR{Unsere beste Chance ist es, den Feind erst einmal nicht zu sehr auf uns aufmerksam zu machen. Trotzdem müssen wir die Transporter besser schützen.}

\par

\WR{Haie}, dachte Samad laut. \WR{Sie sind unsere drittschnellsten Jäger. Nur Aufklärer sind noch fixer. Aber die wären für Verteidigungsaufgaben zu schwach bewaffnet.}

\par

\WR{Madam, die meisten unserer Piloten haben seit Jahren keinen Hai mehr geflogen}, gab Petrarca zu bedenken.

\par

\WR{Samad, auf Flugschulen wird doch fast nur mit Haien trainiert, oder?}, wollte Fiscale wissen. Der Commander nickte, wissend, worauf sie hinaus wollte. Die Kommandantin wandte sich erneut an Wallander. \WR{Rufen Sie die Lieutenants Wörg, Wilson und Wittwer aus. Sie haben mit Haien noch am meisten Erfahrung. Sie werden fliegen. Sagen Sie der Deckmannschaft das Staffel Grau beim Start Vorrang hat.}

\par

Abdel Samad trat an Captain Fiscale heran. Leiser als üblich sprach er: \WR{Madam, diese drei Piloten haben keine zehn Einsätze hinter sich. Sie haben keine Erfahrung mit realen Gefechten.}

\par

\WR{Soll ich Ihnen etwas sagen?}, fragte Fiscale Fiscale rhetorisch und ihr war deutlich anzumerken, dass sie sich in ihrer Haut nicht wohl fühlte. \WR{Das haben sie nicht, genauso wenig wie die meisten anderen hier an Bord. Heute müssen wir einfach erwachsen werden.}