\WR{Nein}, hauchte Nico Curiosa, als er den kleinen Lichtblitz erkannte, der vom Tod seines besten Freundes und all seinen Passagieren deutete. Flug dutzend zwei war verloren, jedoch zu weit entfern, als das von Nicos Flieger aus Einzelheiten zu erkennen gewesen wären~-- und er war froh darum.

\par

Kurz darauf fiel sein Blick auf das blinkenden Radar. Es besaß keine militärische Programmierung, so hatten die beiden Blickpunkte, die auf seinen Transporter zusteuerten keine rote Einfärbung. Doch Nico war klar, dass es sich dabei um zwei der Angreifer handeln musste. Die beiden Kontakte befanden sich laut Richtungsbestimmung auf einem direkten Abfangkurs.

\par

Dann fielen ihm zwei weitere Blickpunkte auf. Der Computer identifizierte sie als Haie der Starforce. Hastig setzte er sich sein Headset auf und wählte den allgemeinen Notfallkanal aus. \WR{Hier ist Flug dutzend eins. Ich rufe die Haie Grau zwei und drei.}

\par

Nur wenig später traf eine Antwort ein. \WR{Hier ist Lieutenant Morten Wittwer. Wie können wir helfen.}

\par

\WR{Hören Sie zu Morten}, sprach Nico Curiosa atemlos. \WR{Da sind zwei Bogeys, die haben es auf uns abgesehen.}

\par

\WR{Nur nicht ins Hemdchen machen}, erklang eine andere Stimme auf dem selben Kanal. \WR{Die haben wir längst gesehen.}

\par

Kurz darauf piepte in Nicos Kopfhörer laut der Raketenalarm auf. Einem der beiden Angreifer musste eine Erfassung gelungen sein. Bei einem Schiff von der Größe von Flug dutzend eins war dies auch keine Kunst gewesen.

\par

Auf dem Radarbildschirm war das Geschoss als gelber Blickpunkt genau in der Mitte zu erkennen. Seine Entfernungsanzeige nahm schnell immer kleinere Werte an. Nico entschloss sich, nicht gleich auszuweichen. Im Moment flog die Rakete frontal auf sein Schiff zu und hatte somit eine kleine Fläche zu treffen. Würde er wenden, würde sein Schiff dem Geschoss die viel längere Flanke präsentieren.

\par

Er würde abwarten, bis die Rakete näher als einen Kilometer herangekommen wäre und dann hart abdrehen. Wenn er Glück haben würde, würde das Geschoss am Schiff vorbeisausen und sich im weiten Nichts verlieren.

\par

Seine Augen immer auf das Radar gerichtet, hielt sich Nico bereit. Doch ein heller Blitz ließ seinen Blick nach vorne gehen. Dort, wo eben noch die Rakete auf ihn zugeschossen war, brannte jetzt das Feuer ihrer Explosion. Einige rot glühenden Strahlen durchzuckten den Feuerball, kurz bevor ein Hai kurzerhand hindurch schoss.
\WR{Ich hab's doch gesagt}, quäkte es aus Nicos Kopfhörern. \WR{Wir haben alles im Griff.}
\ortswechsel
Morten versuchte erneut sein Glück. Während Kevin den beiden Angreifern direkt vor die Rohre geflogen war, um ihre Rakete abzufangen und damit die Aufmerksamkeit auf sich zu lenken, machte er seine eigenen scharf. Der Computer schaffte es, einen der beiden Angreifer ins Visier zu nehmen und begann mit einer Raketenerfassung. Zu spät bemerkte der Unbekannte, dass er ins Fadenkreuz geraten war und Morten ließ die Rakete mit den Worten \WR{Grau drei, B zwei} starten Mit einem Ruck löste sich das Geschoss von der Tragfläche und raste auf sein Ziel zu.

\par

Der Jäger konnte nicht mehr ausweichen oder Gegenmaßnahmen starten und wurde vom Einschlag der randvoll mit Capezin gefüllten Rakete in seine Einzelteile zerlegt. Nachdem das Feuer der Explosion abgeklungen war, blieben nur noch einzelne Wrackteile des Schiffes zurück.

\par

Der zweite Angreifer geriet schnell ins Kreuzfeuer von Mortens und Kevins Bordkanonen. Viel schneller als einer der beiden geglaubt hatte, wirbelte der kleine Jäger herum und seine Düsen begannen, grünes Feuer auszustoßen. Die Beschleunigung, mit der der Angreifer davon raste überraschte beide gleichermaßen. \WR{Ja, hau nur ab du kleines Arschloch!}, schrie ihm Kevin wütend hinterher und vergaß dabei, dass seine Geschwaderkameraden die einzigen waren, die ihn hören konnten.

\par

Doch als Morten auf das Radar sah, wurde ihm schnell klar, dass der Feind wahrscheinlich nicht aus Angst, sondern eher aus strategischen Gründen den Rückzug angetreten hatte. In einiger Entfernung erfasste das Radar etliche neue Kontakte, die von Pollux Primus aus zu kommen schienen. Schnell wechselte Morten auf den allgemeinen Notrufkanal und sagte: \WR{Flug dutzend eins, da kommt mehr Ärger auf uns zu. Nehmen Sie Kurs auf den Hyperraumknoten. Unser Träger ist auch dorthin unterwegs. Wir versuchen, diese anrückende Staffel aufzuhalten.}

\par

\WR{Verstanden}, antwortete ein dankbarer Nico Curiosa. \WR{Und vielen Dank.}

\par

\WR{Gern geschehen}, gab Morten zurück und hoffte, dass Anna Farley wirklich nur ein paar Minuten hinter ihm war. Sonst würden er und Kevin die Gegner sicher nicht lange aufhalten können.