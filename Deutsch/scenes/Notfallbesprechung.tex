In Wellington war es bereits Vormittag. Präsident Otis hatte sich gerade erst von der Preisverleihung auf Praslin Island verabschiedet und noch die mitternächtliche Luft der Insel eingeatmet, da hatte ihn eine Fähre der Starforce schon nach Neuseeland gebracht. Üblicherweise litt er an Reiseübelkeit, doch dieses mal hatte er nichts davon bemerkt, denn seine Gedanken hingen an ganz anderer Stelle.

\par

Noch wusste er nicht viel. Einen Angriff auf Pollux sollte es gegeben haben. Die Zahl der Toten gehe in die Millionen. Otis sah sie vor sich. Eine gesichtslose, anonyme Masse, die viel zu groß war, um noch als eine Ansammlung von Individuen wahrgenommen zu werden. So hatte es bereits einmal angefangen, als die Capital Fellowship ein Regierungsgebäude auf Kreuzpunkt Primus in die Luft gesprengt hatte. Damals hatte es nur ein paar hundert Todesfälle gegeben. Doch hatten diese Opfer bereits ausgereicht, um Otis Kampagne gegen die Armee zu einem raschen Ende zu bringen.

\par

Im größten Besprechungsraums des Präsidialbüros herrschte ein heilloses Chaos. Der lange Konferenztisch war mit Akten und Handcomputer dermaßen übersät, dass das feine Kiefernholz  kaum noch zu erkennen war. Die Lautstärke und Intensität der Gespräche erinnerten den Präsidenten genauso an die Siegesfeier seines Wahlerfolg,s wie die schiere Anzahl der Menschen, die sich in diesem Raum gezwungen hatte. Von außen prasselte ein heftiger Regen gegen die hohen Fenster. Das Wetter war in der Tat so schlecht, dass man vom nahen Wellington kaum die Lichter erkennen konnte.

\par

\WR{Schätzungsweise gab es nur an die vier trin Überlebende}, begann die sichtlich überforderte Verteidigungsministerin Marina Rosso. Doch sie wurde nahezu übertönt.

\par

Henry Otis konnte sich nur vage vorstellen, wie die junge Frau sich jetzt fühlte. Er hatte sie kurz nach seiner Wahl persönlich für die Stelle der Ministerin für äußere Sicherheit vorgeschlagen. Sie hatte ihm sehr passend gewirkt. Zum einen, weil sie neben einem Klasse A Diplom in Betriebswirtschaftslehre auf ein B Diplom in Volkswirtschaftslehre auch ein persönliche Erfahrungen mit der Armee gemacht hatte. Eine kurze Laufbahn beim militärischen Nachrichtendienst hatte ihr ein Mindestmaß an Einblick verschafft. Außerdem war sie dafür bekannt, die Meinungen der ihr nahestehenden Personen aufzunehmen wie ein Schwamm. So war sie eine wichtige Verbündete in Otis Kampf gegen das Militär geworden.

\par

\WR{Ruhe bitte!}, forderte der Präsident so laut wie die Verteidigungsministerin bleich war.

\par

Als die Gespräche leiser wurden, fuhr Marina Rosso fort: \WR{Das Raumschiff \EN{Regenvogel} KlT hat einen Konvoi aus dem System eskortiert. Laut Bericht des Kapitäns haben von ursprünglich dutzend drei Schiffen nur ein dutzend die Route nach Wega erreicht.}

\par

Der oberste Minister Richard Bellegardè, der zur Linken des Präsidenten am Kopfende saß, musste um Fassung ringen, als er fragte: \WR{Gibt es eine Chance, dass es weitere Überlebende gibt?}

\par

\WR{Auf keinen Fall}, antwortete ihm Grandadmiral Burns sofort, der bisher überraschend ruhig geblieben war. \WR{Die Aufzeichnungen der \EN{Regenvogel} zeigen, wie die Angreifer die Kolonie etwa eine Stunde lang mit nuklearen Sprengköpfen bombardiert haben. Dort gibt es nur noch Krater, sonst nichts mehr.}

\par

Der Finanzminister brach in Tränen aus. Otis wusste, dass er Verwandte auf Pollux gehabt hatte. \WR{Wir sollten sofort eine Liste erstellen, wer unter den Überlebenden ist}, regte der Präsident an. \WR{Bald muss ich die Union über den Angriff informieren und dann will ich den Angehörigen eine Möglichkeit bieten können, so schnell wir möglich zu erfahren, ob jemand aus ihrer Familie zu den Opfern gehört. Admiral Burns, leiten Sie bitte entsprechende Anweisungen an die Basis in Wega weiter.}

\par

\WR{Ist bereits geschehen, mein Herr}, antwortete der Militärchef schlicht, ohne den Präsidenten dabei anzusehen.

\par

Lertha Akintola stellte die Frage, die alle beschäftigte. \WR{Wer hat uns da angegriffen?}

\par

Grandadmiral Burns bemühte sich, nicht mit den Schultern zu zucken. \WR{Wir sind uns nicht sicher. Als wir vom Einsatz nuklearer Waffen erfuhren, hielten wir zunächst eine terroristische Vereinigung oder Piraten als Täter für möglich. Aber auf die \EN{Regenvogel} wurde mit Nullzonenwaffen geschossen. Unser Nachrichtendienst hält es für unmöglich, dass es außer der Starforce jemand ein solches Arsenal besitzt. Nicht einmal die Armee der autonomen Welten.}

\par

\WR{Der Nachrichtendienst ist bekannt dafür, Fehler zu machen}, gab die Verteidigungsministerin gereizt zurück. \WR{Vielleicht haben sich die Autonomen weiterentwickelt. Nichts hindert sie daran, solche Waffen zu bauen.}

\par

\WR{Abgesehen von den fehlenden technischen Möglichkeit, meine Dame}, antwortete Burns nüchtern.

\par

Marina Rosso schnaubte lediglich verächtlich. Um weitere Feindseligkeiten zu verhindern hakte Lertha Akintola nach: \WR{Wenn es keine Piraten und auch nicht die autonomen Welten waren…}

\par

Der Grandadmiral erahnte bereits die Frage. \WR{Die Auswertung aller Daten der \EN{Regenvogel} ist noch nicht abgeschlossen. Aber der Träger hat uns noch mehr Informationen geschickt, nachdem er Wega erreicht hatte. Er war ursprünglich auf der Suche nach dem Forschungsschiff \EN{Virial}. Dieses wurde von den gleichen Angreifern vernichtet, wie die Kolonie auf Pollux. Und dem Logbuch der \EN{Virial}  nach zu urteilen waren die Unbekannten außerirdischen Ursprungs.}

\par

Sofort explodierte der ganze Saal in wilde Diskussionen. Es war nicht das erste mal, dass ernst gemeinte Spekulationen über intelligentes Leben außerhalb der Menschheit aufgekommen waren. Vor ein paar Jahren hatte ein Biologe geglaubt, tatsächlich Außerirdische gefunden zu haben.

\par

\WR{Ruhe!}, forderte Otis erneut. \WR{Grandadmiral Burns, gibt es Beweise für diese These.}

\par

Der Militärchef würdigte den Präsidenten nach wie vor keines Blickes, sondern erklärte nüchtern. \WR{Wir haben etliche Abtastungen der feindlichen Raumschiffe vorliegen. Allein ihre Bauweise zeigt, dass sie nicht von Menschenhand gefertigt wurden. Obwohl ihre Waffen und Vehikel den unsrigen nicht vollkommen unähnlich sind, fehlen dennoch die typisch terranischen Eigenheiten. Beispielsweise scheinen die Schiffe keine Besatzung, dafür jedoch riesige Gastanks zu haben, die nichts mit dem Antrieb zu tun zu haben scheinen.}

\par

\WR{Vielleicht waren es doch die autonomen Welten}, spekulierte Marina Rosso. \WR{Ich meine, die arbeiten doch viel mit automatisierten Schiffen, weil ihnen das Personal fehlt.}

\par

Admiral Burns schüttelte entschieden den Kopf. \WR{Nein, meine Dame, ich denke Sie irren sich. Nicht einmal die Union wäre in der Lage solch ausgereifte Roboterschiffe zu entwickeln und wir haben einen Computer, der Wirtschaft für fünfunddreißig Milliarden Personen steuert. Außerdem ist Pollux zu weit von der Grenze der autonomen Welten entfernt. Ihre Versorgungswege wären für einen Angriff zu lang. Ich denke, wir können sicher sein, dass wirklich Außerirdische für diesen Angriff verantwortlich sind.}

\par

\WR{Mich hätte es überrascht, wenn es keine Außerirdischen geben würde}, gestand Richard Bellegardè ein. \WR{Allein in ein paar hundert Lichtjahren um die Erde gibt es schon Planeten, auf denen sich Leben entwickeln könnte.}

\par

Der Chef des wissenschaftlichen Beraterstabs, Joachim Lesc-Bubleth hob beide Augenbrauen. \WR{Bei Ihrer Wahrscheinlichkeitsrechnung vergessen Sie aber, dass intelligente Zivilisationen wahrscheinlich nicht länger als einige Jahrtausende leben. Abgesehen davon hätten sie gigantische Distanzen zu überwinden, die selbst mit einem Hyperantrieb noch ein riesenhaftes Hindernis darstellen. Und würden sie aus unserer Nähe stammen, dann hätten wir längst Funksignale von ihnen empfangen müssen. Ob konventionell oder per Nullzone.}

\par

Präsident Otis lächelte bitter. \WR{Ich denke, wir haben gestern Nacht eine unmissverständliche Nachricht von ihnen erhalten. Unser erstes Zusammentreffen mit einer fremden Zivilisation und wir schießen schon aufeinander. Ein weiterer Beweis, wozu Waffen führen.}

\par

Nun platzte der Ärger aus Admiral Burns heraus. Er blickte den Präsidenten mit hasserfüllten Augen an und donnerte: \WR{Verdammt! Tausende sind gestern gestorben und Sie denken an nichts anderes als an ihren vorgespielten Pazifismus. Hätten Sie und Ihre Kabinett der Armee nicht wieder und wieder die Mittel zusammengestrichen, hätten wir die Kolonie vielleicht besser verteidigen können. Wäre nicht zufällig ein Träger in der Nähe gewesen, hätte es niemand lebend aus diesem Hexenkessel geschafft.}

\par

\WR{Was schlagen Sie vor, Admiral?}, fragte Marina Rosso den immer noch kochenden Militärchef.

\par

Dieser brauchte nicht lange für eine Antwort: \WR{Einen entschiedenen Gegenschlag, Ma'am. Ich habe höchste Alarmstufe für unsere gesamte Flotte angeordnet. Wir werden bald bereit sein~-- auch wenn unsere Mannschaften kaum Gelegenheit zum Üben hatten.}

\par

Otis schüttelte sofort den Kopf. \WR{Abgelehnt. Wir kennen noch nicht einmal den Grund für ihren Angriff. Wir sollten versuchen, Kontakt mit ihnen aufzunehmen.}

\par

\WR{Und wie soll das funktionieren?}, fragte Lesc-Bubleth. Er klang mehr als verärgert. \WR{Wenn das wirklich Außerirdische sind~-- und ich bin noch nicht bereit, das zu akzeptieren~-- dann können wir bestenfalls raten, auch welche Art sie kommunizieren. Ich bin nicht bereit, mich in so einer Krise an diesem Unsinn zu beteiligen.}

\par

\WR{Ich genauso wenig, mein Herr}, hängte Burns an.

\par

Das war üblicherweise der Moment an dem Lertha Akintola schlichtend eingriff. Doch nun hielt sich die Vizepräsidentin einfach nur heraus.

\par

\WR{Sie werden}, blaffte Otis. \WR{Ich muss sie sicher nicht belehren, dass das Präsidialbüro in den ersten achtundvierzig Stunden eines Septemberschnees die volle Weisungsgewalt innehat. Also erfüllen Sie ihre Pflicht oder ich finde jemand, der es für sie tut.}

\par

\WR{Ja, mein Herr}, bestätigte Burns zackig und salutierte mit dem Gesichtsausdruck eines Gladiators vor dem Präsidenten. Dann stand er auf und sagte: \WR{Entschuldigen Sie mich jetzt. Jemand muss für unsere Sicherheit sorgen.}