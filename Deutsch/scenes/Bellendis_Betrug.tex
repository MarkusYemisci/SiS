\WR{Es ist mir egal, wie vielversprechend ihre Bilder sind}, entgegnete Shen Zi einem jüngeren Marco Bellendi. Der damalige Direktor der Pinnacle Science Group war direkt. Viele empfanden ihn darum als unhöflich. Doch Bellendi schätzte seine Offenheit. So verschwendete er wenigstens keine Zeit mit langen Sondierungsgesprächen, die doch mit einer Absage endeten. \WR{Sie wollen von mir, dass ich Ihnen ein Schiff meines Unternehmens zur Verfügung stelle. Und es soll nicht irgendwo nach Außerirdischen suchen, sondern mehr als dreißig Lichtjahre von der Erde entfernt. Haben Sie eine Ahnung, was so ein Flug kostet?}

\par

\WR{Natürlich}, erwiderte Marco Bellendi und unterdrückte seine Gereiztheit. \WR{Gerade weil ich weiß, dass mir etwa eins Komma vier Octinae Naira fehlen, komme ich zu Ihnen.}

\par

Shen Zi blieb einen Moment ruhig und starrte einfach durch seinen Gesprächspartner hindurch. Hinter dem Konzernchef erstreckte sich eine üppige Anzeigefläche über die ganze Wand, die Bilder der letzten Sondierung von Iota Persei zwei präsentierte. Es war kein direktes Luftbild, denn ausgerechnet über der interessantesten Stelle der Oberfläche hatte sich zum Zeitpunkt der Aufnahme eine dichte Wolkendecke befunden. So hatte es Marcos Team mit einer Komposition sämtlicher nicht optischer Untersuchungsresultate versucht und dieses Bild generieren lassen. Leider zeigte es nur ein paar unscharfe Linien auf einer weitläufigen Grasfläche. Marco hielt diese für eindeutige Zeichen von Ackerbau.

\par

Doch die monströse Leinwand war es nicht, der Bellendi im Büro des Pinnacle Vorsitzenden am meisten beeindruckte. Der Raum war ausgesprochen klassisch eingerichtet. Die Wände waren mit einem Holzverschlag versehen und der dunkelblaue Teppichboden konnte nur eine Antiquität sein. Genauso wie die grüne Schreibtischlampe, und der Tisch selbst.

\par

Der Ausblick aus dem Fenster war trostlos, denn der Firmensitz der Pinnacle Science Group war in einer der äußersten Städte Kreuzpunkt Primus erbaut worden~-- wenige Kilometer von einem Gletscher entfernt. Durch den Schneesturm, der derzeit tobte war aber weder dieser noch die atemberaubenden Berge zu erkennen.

\par

Die einzige Farbenfreude spendete eine Werbeanzeige des Konzerns, die gerade auf Shen Zis DDV-Gerät abgespielt wurde. Sie zeigte Bilder von gut aussehenden Männern und Frauen, die sich enthusiastisch unterhielten oder konzentriert auf glänzende chemische Apparaturen schauten. Bellendi fragte sich, ob er gerade wirkliche Forscher des Unternehmens oder doch nur Schauspieler sah.

\par

Der Ton war ausgeschaltet, doch Marco kannte die Reklame bereits in- und auswendig. Er konnte den Sprecher im Hintergrund förmlich sagen hören: \WR{Zusammenarbeiten heißt, sich ganz und gar hinzugeben. Teil eines Teams zu sein.

\par

Bei PSG treffen Neugier auf Intuition. Bewährtes auf Neuerung. Jahrhunderte an Erfahrung auf jungen Forschergeist. Ihr Potential auf unsere Mittel.

\par

Bewegen Sie etwas.

\par

Pinnacle Science Group. Weil Wissen Werte schafft.}

\par

Dass Shen Zi diesen Werbefilm auf seinem Schreibtisch laufen ließ, obwohl es in der Krypta zahllose interessantere Videos zu finden gab, zeugte davon, dass seine Verbundenheit zu seinem Unternehmen nach wie vor ungebrochen war. Zi war ein Forscher. Sein Lebenslauf sprach eine eindeutige Sprache und strotzte nur so vor Nennungen bekannter Veröffentlichungen und Ehrungen. Doch er war auch ein geschickter Geschäftsmann. Sein Auftreten verlieh ihm trotz seiner schlichten Erscheinung eine Präsenz, die einen Konferenzraum in kürzester Zeit ausfüllte.

\par

Bellendi glaubte, dass es im Wesentlichen an seiner Art, sich zu bewegen lag. In jedem Handgriff~-- und wenn er nur eine Akte auf den Tisch legte~-- fand sich eine unerschütterliche Kompromisslosigkeit, die jedem Geschäftspartner sofort zeigte, wer von beiden den Pareto-effizienteren Tausch machen würde.

\par

Marcos Anliegen war daher ein verlorener Posten. So verwunderte es ihn nicht, als Shen Zi den Augenkontakt zu ihm suchte und sagte: \WR{Ich bewundere, wie wichtig Ihnen Ihre Forschung ist. Jedes mal, wenn Sie mir eine Idee vorstellen, wünschte ich, Sie wären bei uns fest angestellt. Aber Sie müssen auch verstehen, dass ich Sicherheiten brauche. Bringen Sie mir irgendetwas. Nur mehr als bloß ein paar unscharfe Fotos. Dann verspreche ich Ihnen, werde ich jedes Risiko auf mich nehmen um Ihnen Ihre große Entdeckung zu ermöglichen.}

\par

Und dabei einen Gewinn einstreichen, mit dem sich die PSG für alle Zeit an die Spitze der Unionsunternehmen setzen würde, fügte Bellendi in Gedanken an. Er rang sich ein Lächeln ab. Zi war ein wichtiger Förderer und schon fast so etwas wie ein Freund für ihn geworden. Beides gute Gründe für ihn, den Firmenchef nicht zu verärgern.

\par

\WR{Ich verstehe.} Ganz schaffte er es nicht, seine Enttäuschung zu verbergen. Shen Zi nahm es ihm nicht übel. Er verbeugte sich tief, nachdem er sich aus seinem Sessel erhoben und Marco damit signalisiert hatte, dass das Treffen zu Ende war.

\par

\WR{Ich glaube an Sie, Herr Bellendi.} Marco war verwundert, so klare und aufmunternde Worte von jemanden zu hören, der selten Lob verteilte. \WR{Ich bin sicher, dass Sie finden werden, was Sie suchen.}
\ortswechsel
Es gibt keine Wunder, außer denen, die man sich selbst macht. Dieser Satz ging Bellendi auf dem gesamten Rückweg zur Biologischen Fakultät der Universität von Kreuzpunkt Stadt durch den Kopf. Der Zug fuhr deutlich langsamer, als er es ohne den wild tobenden Schneesturm gekonnt hätte. Dichte Netze, die schlimme Steinschläge an der steilen Felswand verhindern sollten, waren zur rechten zu sehen. Würden diese reißen, hätte der Zug noch eigene Blocker, mit denen sich das Schlimmste vermeiden ließ.

\par

Trotzdem fühlte sich Marco Bellendi nicht wohl auf dieser Strecke. Sie ging über felsiges Terrain an der Ostseite eines riesigen Berges vorbei, der ganze vier Kilometer in den Himmel ragte. Seine vereisten, grauen Wände, wirkten wie eine unterschütterliche Mauer.

\par

Die noch weitestgehend unberührte Natur Kreuzpunkts war der Grund für viele Menschen, sich auf dem insgesamt doch eher kargen Planeten anzusiedeln. Hier waren die vielen Fehler von Anfang an vermieden worden, die Menschen im Umgang mit ihrer Umwelt oft begingen.

\par

Bellendi jedoch bevorzugte ausgebaute Städte. Und Strände. Überhaupt alles, was warm und sonnig war. Orte, an denen man sich morgens die Umhängetasche auf den Gepäckträger spannte und dann ohne Jacke, Mantel oder Regenschirm unbeschwert zu Arbeit radelte. All das bot Kreuzpunkt Primus nicht. Höchstens in den äquatorialen Gebieten stiegen die Temperaturen im Sommer auf mehr als dreißig Grad. Und selbst dort war dies eine Seltenheit.

\par

Das Spiegelbild in der von innen leicht angelaufenen Fensterscheibe zeigte seinem einen bedrückten Gesichtsausdruck. Dieser rührte jedoch nicht von Bellendis Versagen, Li zu überzeugen her. Sondern eher von dessen Frage an sich selbst, ob er sich selbst nach wie vor im Spiegel anschauen können würde, wenn er durchzog, was er vorhatte.

\par

Er war sich sicher, dass es eine Zivilisation auf Iota Persei zwei gab. Vermutlich keine weit entwickelte. Die Fernaufklärung hatte weder Satelliten noch irgendwelche anderen künstlich geschaffenen Objekte im Orbit jener Welt entdeckt. Aber die Muster mussten  Bewässerungsanlagen auf der Oberfläche darstellen. Die Linien alleine hätten Bellendi auch nicht überzeugt. Aber zu ihnen kamen Quellen von infraroter Strahlung, die sich offensichtlich zielgerichtet bewegten. Nicht erratisch, wie es Fehlmeldungen der Fernaufklärungsdrohnen normalerweise taten.

\par

Zunächst hatte Marco vorgehabt, seine Kollegen anzuweisen, die Bilder vom Computer neu berechnen zu lassen. Wendete man genauere Verfahren an und kontrollierte man die Rahmenbedingungen erneut durch, ließ sich möglicherweise eine bessere Darstellung erzeugen. Aber das war ein Umweg, der Bellendi nun mehr und mehr als überflüssig erschien.

\par

Das bildgebende Verfahren erzeugte Resultate, die nur schwer von Fälschungen zu unterscheiden sein würden.
