\lettrine[lines=3]{A}bed Nidal und Eddie Carlyle saßen wie immer nebeneinander am Esstisch des viel zu klein geratenen Erholungsraums der \EN{Virial}. Außer dem Tisch fanden sich in dem steril und metallisch anmutenden Raum nur noch ein paar Schränke, einige Sofas und ein paar Dartscheiben. Zwei Fensterscheiben, deren enorme Dicke leicht zu erahnen war, boten einen Ausblick ins Weltall. Da die Dartscheiben direkt neben den Fenstern hingen, war es nicht verwunderlich, dass sich hier und da einige Kratzer am Glas entlang zogen.

\par

Abed war es immer wieder ein Rätsel, was sich die Konstrukteure des Schiffes beim Entwerfen des Erholungsraums gedacht hatten. Höchstens zwanzig Leute hatten darin Platz bei einer Besatzung von beinahe achtzig. Diesmal gab es allerdings kein Problem, denn Herr Krüger war dabei die Lage und die Aufgaben zu besprechen und es war eine Überraschung, dass sich überhaupt zwanzig Interessenten gefunden hatten. Die Besprechungen folgten immer dem gleichen Schema. Krüger erklärte, welche Zielsetzungen das Unternehmen verfolgte, wie er sie zu erreichen gedachte und was jeder einzelne zu tun hätte. Dann folgten meistens einige Appelle an die Arbeitsamkeit und den Fleiß der Besatzung und einige lobende Worte über frühere Operationen.

\par

Kapitän Krüger ging am Tisch auf und ab. In der Hand hielt er eine Tasse Tee. Der Leiter des Schiffes wartete immer ein paar Minuten, in denen er den Anwesenden Zeit ließ, sich untereinander ein wenig zu unterhalten.

\par

Abed und Eddie redeten an diesem Abend aber kaum miteinander. Beide fühlten sich ausgelaugt, waren müde und zugleich aufgekratzt. Vielleicht, weil sie den ganzen Tag im Prinzip gar nichts getan hatten, außer ihre Radarbildschirme zu begutachten. Die physischen Belastungen eines Hyperraumsprungs taten das ihrige dazu.

\par

Abed begnügte sich damit, hin und wieder an dem Croissant herunterzubeißen, dass er vor sich liegen hatte. Dafür, dass es aus den Nahrungsreserven des Schiffes stammte, schmeckte es ziemlich gut. Aber vielleicht, so befürchtete Abed, hatte er auch schon vergessen wie ein richtiges Croissant schmeckte. Eines, dass nicht in einer Gefrieranlage haltbar gemacht worden war.

\par

Eddie war dabei mit seinem Buch herumzuspielen. Dass Modell, dass er bevorzugte, stammte von Bolchaal und bot um einiges mehr Funktionen als Abeds Modell. Der fingerdicke Apparat war sogar mit einem kleinen aber hochauflösenden holographischen Projektor ausgestattet, mit dem Eddie drei-D-Übertragungen anschauen konnte.

\par

\WR{Was machst du da?}, fragte Abed. Die Müdigkeit des Forschers schwang unüberhörbar in der Frage mit und es grenzt an ein Wunder, dass sie nicht mit einem Gähnen endete.

\par

Eddie seufzte. \WR{Ich will das Streitgespräch zwischen dem alten Otis und dieser Studentenmeute angucken. Aber irgendwie finde ich die Aufzeichnung nicht in der Krypta.}

\par

Abed kam um ein hämisches Grinsen nicht vorbei. Dass sein Kollege in der Krypta Scientia etwas nicht fand, wäre nicht das erste Mal gewesen. Allerdings konnte sich eine Suche auch als sehr schwierig erwiesen, da die Krypta im Prinzip ein unionsweiter Datenspeicher war, der so gut wie alles enthielt. Drei-D-Übertragungen, Werke nichtkommerzieller Musiker, Maler und Dichter, Enzyklopädien, Archive, Programme und vieles mehr. Der Speicherbedarf der ganzen Krypta konnte mittlerweile nur noch geschätzt werden.

\par

\WR{Ist die Übertragung überhaupt schon oben?}, fragte Abed vorsichtig. \WR{Das ganze war doch erst gestern Mittag. Vielleicht hat Spectare sie noch nicht hochgeladen.}

\par

Eddie reagierte gar nicht auf die Frage seines Kollegen. Er würde sich wahrscheinlich bald eine Sehnenscheidenentzündung zuziehen, so heftig kritzelte er mit seinem Stift auf den beiden aufgeschlagenen Seiten herum. Doch egal in welcher Kombination und Genaugikeit er seine Anfrage formulierte, sein Buch lieferte keine Resultate. Die intelligente Tinte formierte sich immer wieder zu demselben Satz~-- Keine Ergebnisse zu ihrer Suche.

\par

\WR{Verdammt!}, gab Eddie schließlich verärgert von sich. \WR{Jetzt kriege ich gar keinen Kontakt mehr zur Krypta. Wehe der Nullzonentranciever von diesem dämlichen Kübel ist schon wieder im Eimer. Dann dreh ich hohl! Ich halts hier draußen doch nicht ohne Krypta aus.}

\par

Abed hob beschwichtigend beide Hände. Er hatte bereits keine Lust auf den Kapitän. Aber auf einen verärgerten Eddie Carlyle schon zweimal nicht. \WR{He, bleib jetzt mal ganz ruhig. Der Ärger geht bestimmt nicht ewig. Wir haben einen echt weiten Sprung hinter uns. Da kann die Verbindung schon mal abbrechen, oder?}

\par

Eddie grunzte und drückte weiter auf seinem Handcomputer herum. Sein Freund hatte vorgesorgt und sich vor dem Sprung ins Arktur System, das gute sechsunddreißig Lichtjahre von der Erde entfernt lag, ein paar neue Bücher und etwas Musik in sein Buch gespeichert.

\par

Mit einem mal wurde Eddie sehr nachdenklich. Sein Gesicht wirkte wie eingefroren. \WR{Moment mal}, begann er leise. \WR{Was du sagst, kann gar nicht sein. Wenn unser NZT tatsächlich etwas durch den Sprung abbekommen hätte, dann hätten wir schon den ganzen Tag keine Verbindung mehr. Die Verbindung müsste entweder stehen oder eben nicht. Aber ein andauerndes An und Aus kann so nicht vorkommen.}

\par

In der Theorie hatte Carlyle recht. Der Nullzonentranciever baute zu einem Gegenstück eine direkte Verbindung im Raum auf. Der Abstand zwischen zwei bestimmten Punkten, innerhalb der Geräte betrug damit Null. Elektromagnetische Wellen konnten die winzige Öffnung passieren und somit Informationen in Echtzeit transportieren.

\par

Noch ehe Eddie sich weiter um sein Buch kümmern konnte, ergriff Krüger das Wort. Der Leiter des Schiffes nahm einen letzten Schluck aus seiner Teetasse, stellte sie auf den Tisch und sagte: \WR{Guten Abend, meine Damen und Herren. Ich bin froh, dass sie alle hier sind. Besonders da ich weiß, dass diese Besprechungen immer etwas trocken ausfallen.}

\par

In den Gesichtern aller Anwesenden fand sich stille Zustimmung. \WR{Herr Kiowa, was haben wir bis jetzt?}, fragte Krüger schließlich einen Forscher, der quasi seine Rechte Hand darstellte.

\par

Der angesprochene Mathematiker verhielt sich meistens still. Er galt unter der Besatzung als der Besonnenste, denn er begegnete Krüger immer ohne jeden Vorbehalt, bemühte sich aber trotzdem zu den Forschern dazu zu gehören und ein ganz selbstverständliches Gemeinschaftsgefühl zu etablieren.

\par

Kiowa warf einen kurzen Blick in sein Buch. Er besaß ein recht breites Modell, das zwar unhandlich war aber dafür auch deutlich mehr Platz zu schreiben bot.

\par

\WR{Nun, wie es aussieht}, begann der Mathematiker, \WR{ist das Arktur System das, was es sein sollte. Wie mit Teleskopen bereits beobachtet wurde, haben wir es hier mit einem Typ acht Stern zu tun, der von fünf Planetenanwärtern umkreist wird. Keiner ist ohne Terraformierung bewohnbar. Es gibt ein Asteroidenfeld ganz in der Nähe des Hyperraumknotens nach Pollux, das eventuell reich an Erzen sein könnte. Aber nach dem was unser Radar sagt, bewegen sich einige Fragmente sehr schnell, es wird also sicher schwierig dort Bodenschätze abzubauen.}

\par

Einer der Forscher brummte frustriert. Abed vermutete, dass er das Asteroidenfeld entdeckt hatte und nun enttäuscht war, weil es praktisch niemandem etwas nützte.

\par

Kiowa sprach unbeirrt weiter: \WR{Das Strahlungsspektrum des Sterns ist, aus der Nähe betrachtet, ein wenig anders als erwartet. Der Neutrinoausstoß ist höher als es die Theorie vorhersagt. Auch die Ultraviolettte Strahlung ist stärker als es die Fernmessung ergeben hat. Bis zum dritten Planeten beträgt die Strahlungsleistung durchschnittlich zehntausend Hincarr. Wie vermutet, können die ersten drei auf gar keinen Fall besiedelt werden. Aber in der Nähe des fünften Planeten, der sich im Moment nah am Hyperraumpunkt befindet, könnte es einen Sprungpunkt nach Capella geben. Zumindest hat das unser Computer errechnet, nachdem wir ihn mit den vorläufigen Massedaten des Capella Systems gefüttert haben.}

\par

Herr Krüger nickte zufrieden. \WR{Das könnte interessant sein. Wir haben noch dutzenddrei Sprungdrohnen an Bord. Wir könnten überprüfen, ob sich wirklich eine Hyperraumverbindung aufbauen lässt. Ein neuer Sprungknoten kann für die Expansion der Union nur hilfreich sein. Auch wenn Arktur dafür nur als Zwischenstation herhalten würde.}

\par

Abed bemerkte wie Eddie die Augen verdrehte. Er konnte verstehen wieso. Einen neuen Sprungpunkt ausfindig zu machen war Sache der Mathematiker und Physiker. Passive und aktive Radarüberwachung war dafür absolut belanglos. Trotzdem freute sich Abed schon darauf die Köpfe bestimmter Mannschaftskameraden ein wenig rauchen zu sehen. Einen Sprungpunkt zu einem System zu berechnen, dessen Massenverteilung man nur abschätzen konnte war eine schwere Aufgabe. Eine Hyperraumroute musste auf einem bestimmten Eintrittsvektor und mit einer bestimmten Entfernung zum Ziel benutzt werden. Vektor und Entfernung hingen von vielen Faktoren ab, die man bei einem unerforschten Ziel nur raten konnte. Das bedeutete, dass die Rechenkünstler einfach einige Möglichkeiten ausprobieren mussten, was eigentlich gegen das Ehrgefühl jedes gestandenen Mathematikers ging. Und da nur noch fünfzehn Drohnen übrig waren, hatten die Verantwortlichen nicht gerade viele Versuche. Abed sah schon vor sich, wie einigen der Schweiß auf die Stirn treten würde.

\par

Er wurde allerdings aus seinen Gedanken gezogen, als Krüger das Wort an ihn richtete: \WR{Herr Nidal. Sie sind an der Reihe. Was hat die Überwachung mit den passiven Suchsystemen ergeben?}

\par

Abed brauchte nicht in sein Buch als Gedächtnisstütze zu blicken um eine Antwort darauf zu geben. \WR{Nun, es gab keine unerwarteten Sichtungen. Vielleicht einige nicht näher bestimmbare Strahlungsspitzen. Aber die kommen von einem der Planeten. Sie sind nicht ungewöhnlich bei einer so großen solaren...}

\par

Eine laute, quäkende Stimmte unterbrach unvermittelt Abeds Ausführungen. Sie drang aus dem Lautsprechersystem des Gemeinschaftsraums und gehörte dem wachhabenden Wissenschaftler der Nachtschicht. \WR{Der Kapitän soll bitte \textit{sofort} in die Leitzentrale kommen. Wir haben hier eventuell etwas interessantes gefunden. Gegen so etwas wäre ein schwarzes Loch ein Witz!}

\par

Einen Moment lang war alles still und einige verwunderte und fragende Blicke trafen sich. Dann redete alles durcheinander. Einige Forscher standen aufgeregt auf und machten sich auf den Weg zu Aufzug. Herr Krüger ging zum nächsten Kommunikationsterminal und stellte eine Verbindung zur Brücke her, während die meisten anderen bereits den Raum verließen. Abed und Eddie gehörten zu den ersten, die sich in den viel zu engen Aufzug zwängte.

\par

\WR{Hier ist Krüger}, begann der Leiter der \EN{Virial} in das Richtmikrofon des Terminals zu sprechen. \WR{Was haben Sie gefunden?}

\par

Die Stimme des Forschers am anderen Ende der Leitung klang so aufgeregt und gespannt, wie es auf der \EN{Virial} noch niemals jemand gewesen war. \WR{Das müssen Sie selbst sehen. Es ist einfach unglaublich. Ich kann nicht fassen, dass sich das gerade auf dem Bildschirm vor mir abspielt. Kommen Sie auf die Brücke, verdammt!}

\par

Krüger zuckte vom Kommunikationsterminal zurück. So sprach selten jemand mit ihm. Auf dem Kommandodeck musste wirklich die Hölle los sein, so erregt wie der Wissenschaftler klang. Aber je mehr Krüger über alles nachdachte, umso mehr wurde ihm bewusst, dass es eigentlich nur zu seinem Guten sein konnte. Egal was die Forscher gefunden hatten, es schien bahnbrechend zu sein. Und diese Entdeckung war unter \textit{seiner} Leitung gemacht worden. Die Firmenführung würde ihn mit Sicherheit in Erinnerung behalten. Schnell machte er sich ebenfalls auf den Weg zur Leitzentrale.

\par

Abed und Eddie waren längst angekommen. Sie und dutzende andere aufgeregte Forscher bildeten einen weiten Ring um die Teleskopstation. Die Spezialistin, die vor der Kontrollkonsole saß, hackte förmlich auf die Tastatur ein. Die beiden drängten sich etwas nach Vorne um dem Schauspiel besser folgen zu können. Auf einem der gläsernen Monitore zeigte sich immer und immer wieder dieselbe Aufnahme eines Teleskops. Sie war zwar nur eine Sekunde lang aber sorgte dafür, dass sowohl Abed als auch Eddie minutenlang auf den Bildschirm starrten, während sie mal aufs mal wiederholt wurde. Der andere Monitor zeigte den Ablauf einer Bildvergleichsroutine die einfach keine Übereinstimmung mit der Datenbank fand.

\par

\WR{Könnte es das wirklich sein?}, fragte Eddie leise vor Aufregung. \WR{Hatte dieser Marco Bellendi am Ende doch Recht?}

\par

Abed nahm die Frage seines Kollegen kaum noch war. Er starrte nur auf die Aufnahme des Teleskops in deren Mitte sich immer wieder ein Schimmer aus Licht zeigte, der größer wurde und dann wieder verschwand. Eigentlich war nicht viel zu sehen aber ihm war dennoch klar, was die Bilder bedeuteten.

\par

Kurz darauf betrat Herr Krüger die Brücke. Er musste gerannt sein, denn er schwitzte und keuchte. Den Leiter des Schiffes hatte bisher noch niemand jemals laufen sehen.

\par

Krüger drängte sich durch die Masse und stellte sich neben die Teleskopspezialistin.

\par

\WR{Was ist das?}, wollte er wissen und zeigte auf die Aufnahme, die nach wie vor in einer Endlosschleife wiederholt wurde.

\par

Die Spezialistin antwortete ihm ohne ihren Blick von ihren Anzeigen abzuwenden: \WR{Das wurde zufällig aufgenommen. Hätten wir das Suchmuster des Teleskops nur um ein paar Grad anders eingestellt, hätte der Computer das komplett übersehen.}

\par

\WR{Kurs setzen, sofort!}, rief Krüger durch die halbe Brücke.

\par

Der Mann am Steuer hatte schon damit gerechnet, dass dieser Befehl folgen würde. Er nickte nur und bediente dann die Ruderkontrollen während ihn der Navigator mit Informationen versorgte. Krüger begab sich zur Kommunikationsstation. Er sprach etwas leiser aber der Trubel auf der Brücke zwang ihn, immer noch so laut zu reden, dass Abed ihn verstehen konnte.

\par

\WR{Laden Sie alle Daten zum Omega dutzendeins Protokoll herunter. Wir werden es vielleicht bald brauchen}, instruierte Krüger den Kommunikationstechniker.

\par

Abed wusste nur zu gut, was es mit diesem Protokoll auf sich hatte. Bedächtig drehte er sich zu Eddie um, der seinen Blick nicht von der Aufnahme lassen konnte, und sagte: \WR{Es ist wahr. Bellendi hatte wirklich Recht.}