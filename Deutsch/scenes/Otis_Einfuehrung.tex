Getrampel im Gleichschritt. Jeder der Soldaten, den der Beobachter durch sein antiqiuertes Fernrohr besah, ging starr hinter dem nächsten her. Den Blick immer nach vorn gerichtet, die Arme entweder dicht am Körper oder eine Gewehrattrappe haltend. Keine einzige Gefühlsregung zeigte sich in den Gesichtern der jungen Männer und Frauen, die durch die Straßen von Wellington marschierten.

\par

Der Beobachter richtete sein Fernrohr auf die Spitze der Parade. Zwei Offiziere, ein Mann und eine Frau, hielten die Flagge der Unio Terrae in den Himmel. Selbst im vollen Sonnenschein war das schwarz-weiße Banner~-- der Zweig der Einigkeit~-- noch gut zu sehen. Hinter den beiden marschierte eine deutlich jüngere Offizierin, die eine weitere Fahne trug. Darauf das Emblem der Prudentium Phalanx Defensionum, meist kurz als \Wr{Phalanx} bezeichnet. Der Mann, der durch das Fernrohr blickte fragte sich, ob die Bannerträger überhaupt über die Bedeutung der Insignien Bescheid wussten, die sie so stolz präsentierten.

\par

Eine knapp dreihundert Mann starke Truppe folgte den drei Leitern an der Spitze. Allesamt Infanteristen der Phalanx, die entschärfte Maschinengewehre mit sich führten oder gänzlich unbewaffnet daherkamen. Der Beobachter sah in der Parade ein Zeichen von militärischer Dummheit. Irgendein General hatte wohl befürchtet, dass es seinen Untergebenen zu langweilig werden könnte. Und so musste die halbe Stadt zusehen, wie die Phalanx wieder einmal ihre Muskeln spielen ließ.

\par

Das öffentliche Interesse schien sich jedoch in Grenzen zu halten. Durch sein Fernrohr erkannte der Mann einige Menschen, die am Straßenrand standen und den Soldaten irgend etwas zuriefen. Die meisten Besucher derartiger Paraden interessierten sich mehr für die anschließenden Feiern und die kostenlosen Getränke, die in deren Zuge ausgeschenkt wurden. Unter den Zuschauern fanden sich aber auch regelmäßig einige Jugendliche, die sich schwärmend aus den Soldaten ihren persönlichen Liebling heraussuchten.

\par

Der Beobachter war einer der wenigen, dem bewusst war, wie sehr sich das Bild einer Armee im Lauf der Zeit verändert hatte. Anfang des einundzwanzigsten Jahrhunderts, was nun knappe siebenhundert Jahre her war, hatte man mit den Soldaten vor allem harte Männer, im Extremfall sogar ehemalige Kriminelle in Verbindung gebracht. Militärangehörige waren bemüht gewesen, gemäß dem Willen ihrer Vorgesetzten, als raubeinige Kämpfer gesehen zu werden, denen kein Feind zu stark und kein Wall zu hoch war.

\par

Doch dann war der Schnitt gekommen. In dessen Verlauf das Gabbot-Virus, die gefährlichste Krankheit in der Geschichte der Menschheit, freigesetzt worden war. Zu Beginn des einundzwanzigsten Jahrhunderts war ein Großteil der damaligen Weltbevölkerung an den Folgen der Krankheit gestorben. Nur gut siebzig Millionen Menschen hatte die Krankheit überlebt. Und infolge dessen hatten Nationen und natürlich auch deren Armeen aufgehört zu existieren.

\par

Starke fünfhundert Jahre lang hatte es keine nennenswerten Streitkräfte mehr gegeben. Und als dann im Vorspiel des Routenkrieges wieder militärische Strukturen entstanden waren, hatten diese völlig andere Zielsetzungen und Stellenwerte entwickelt. Am Ende waren es einige Befehlsverweigerungen auf beiden Seiten gewesen, die das Ende des Routenkriegs und den Fall der damaligen Supermächte eingeläutet hatten.

\par

Deshalb hatte die Armee, über die vergangenen Jahrzehnte, ein völlig neues Gesicht bekommen. Härte, Disziplin und unbedingter Gehorsam waren in den Hintergrund getreten. Dinge wie moralische Werte, die Fähigkeit zum reflektierten Denken und fundierte Entscheidungen treffen zu können standen nun an der Spitze eines jeden Ausbildungsplans der Phalanx.

\par

So erkannte der Beobachter in den Soldaten keine knallharten Draufgänger sondern vielmehr junge Leute, die sich die Welt ansehen und Abenteuer erleben wollten. Aber wirklich kämpfen würden die allerwenigsten müssen. Die meisten hätten nach dreißig Jahren ereignislosen Militärdienstes genügend Nairas zusammengespart um sich und ihren Familien ein sehr gutes Leben zu ermöglichen.

\par

In den Augen des Mannes hinter dem Fernrohr gab es keine Zukunft für die Armee. Seit zehn Jahren war auch die letzte Bedrohung für den Frieden, die Capital Fellowship, völlig verschwunden. Die drohende Ausrottung durch das Gabbot-Virus hatte den Menschen schon vor einigen Jahrhunderten knallhart vor Augen geführt, das ihr Überleben nicht selbstverständlich war. Die Menschen hatten begriffen, dass ihre Spezies auch als Ganzes sterblich war und das ihre Taten sie jederzeit einholen konnten. Wahrscheinlich aufgrund dieses Umdenkens, hatte es in den vergangenen drei Zenturien nur einen einzigen Krieg gegeben. Dieser, der Routenkrieg, hatte jeden noch einmal an die Schrecken eines solchen Konflikts erinnert. Fünfzehn Millionen Menschen waren in diesem Konflikt gestorben. Und, nach der einhelligen Meinung eines jeden modernen Historikers, völlig umsonst.

\par

\WR{Was sehen Sie sich an, Henry?}, fragte den Beobachter eine ihm wohlbekannte Stimme.

\par

Präsident Otis wandte sich von seinem Fernrohr ab und sah seinem Besucher entgegen. Er war so sehr in seinen Gedanken über die Phalanx versunken gewesen, dass er gar nicht gehört hatte, wie sich die Glastür seines geräumigen Büros geöffnet hatte. Die Tür glitt leise wieder zu, als er Richard Bellegardè einen ersten Blick zuwarf. Beide begrüßten sich mit einer knappen, informellen Verbeugung.

\par

\WR{Guten Morgen, Richard}, hieß Otis seinen Besucher willkommen. \WR{Ich wollte nur einen Blick auf die Zinnsoldaten werfen. Kommt mit das nur so vor, oder werden die Rekruten immer jünger?}

\par

Senator Bellegardè räusperte sich. \WR{Ich weiß nicht. Vielleicht sollten Sie Herr Burns fragen, der weiß es bestimmt.}

\par

\WR{Nu ja}, begann der Präsident lächelnd, \WR{er wird in jedem Fall verärgert sein, wenn ich ihn mit \textit{Herr} und nicht mit \textit{Großadmiral} Burns anspreche.}

\par

Richard Bellegardè deutete ein Lachen an. Präsident Otis ging zu seinem Schreibtisch und versuchte eine der Schubladen zu öffnen. Sie klemmte~-- wie immer. Manchmal bereute Henry Otis es, einen Holzschreibitsch im barocken Stil als Arbeitstisch gewählt zu haben. Andererseits würde ein moderneres Modell seinen Hang zur Klassik kaum unterstreichen.

\par

Schließlich hatte es der Präsident geschafft, die Schublade zu öffnen. Er entnahm ihr eine ellenlange Holzkiste und öffnete sie. Darin fanden sich zwei Wasserpfeifen und eine kleine Packung mit Rauchbarem.

\par

\WR{Möchten Sie auch eine, Richard?}, fragte der Präsident unverbindlich.

\par

Der Oberminisiter schüttetelte den Kopf und gab zurück: \WR{Nein, danke. Daraus mache ich mir nichts, wie Sie wissen.}

\par

Henry Otis nickte und begann eine der Pfeifen zu befüllen. Die zweite war für den unwahrscheinlichen Fall gedacht, dass tatsächlich irgendwann jemand das Angebot annahm. Meinungen über Rauchen gingen in der ganzen Union weit auseinander. Obwohl im Gegensatz zur entfernten Vergangenheit keinerlei gesundheitlicher Schaden zu befürchten war, wurde das Rauchen immer noch als ein charakterliches Laster angesehen.

\par

Zu Zeiten der Unio Terrae exestierte kein Tabak mehr. Es wurde nur noch ein Gemenge geraucht, dass ätherische Öle oder Gewürze aber hauptsächlich wässrigen Nebel freisetzte. Solche Mischungen waren in den unterschiedlichsten Variationen zu bekommen. Diese reichten von Menthol bis hin zu absurden Geschmacksrichtungen wie Fisch oder Paprika.

\par

Präsident Otis entzündete die Pfeife und nahm den ersten Zug. Er genoss es, wie das Wasser seine trockene Nase wieder benetzte und die Öle seine Nebenhöhlen öffneten.

\par

Als er einmal tief ein und wieder ausgeatmet hatte, fragte er den Senator: \WR{Wie ist ihr Treffen mit Kanzler Mubarek gelaufen?}

\par

\WR{Offen gesagt bin ich ziemlich erledigt}, gestand Richard Bellegardè ein. \WR{Man weiß erst wirklich, was \Wr{Jetlag} bedeutet, wenn sie am Abend von einem Planeten abfliegen und auf einem ankommen auf dem gerade Morgen ist. Es ist, als gäbe es keine Nacht.}

\par

Präsident Otis bedachte seinen Besucher mit einem mitleidvollen Blick. Der Oberminister fuhr mit seiner Erzählung fort. \WR{Na ja, und Sie kennen ja Bill. Er hat eine Ausdauer beim essen, dass man nur noch schauen kann. Böse Zungen behaupten schon, er würde irgendwann einmal mehr essen, als er selbst wiegt.}

\par

Henry Otis lachte sofort auf und entgegnete: \WR{Das wird schwierig.}

\par

Schnell senkte der Senator sein Haupt ein wenig um zu verbergen, dass er über beide Ohren grinste und Mühe hatte, einen Lacher zu unterdrücken. Es gab sicher nur wenige Personen auf den zwanzig bewohnten Welten der Union, mit denen er auf diese Art über einen Politiker sprechen konnte, ohne sein Amt zu verlieren.

\par

\WR{Na ja, im Ernst}, kehrte Richard Bellegardè wieder zur Frage des Präsidenten zurück. \WR{Kanzler Mubarek ist recht zufrieden mit der Unterstützung welche die Union im vergangenen Jahr an Sagittae Primus geleistet hat. Er war bei bester Laune und hat dem Stab und mir ein opulentes Abendessen angeboten. Wenn ich da nicht schon anderthalb Dutzend Stunden auf den Beinen gewesen wäre, hätte ich das Essen sicherlich auch genossen.}

\par

Präsident Otis Nicken wirkte zwar gleichgültig aber Senator Bellegardè wusste, dass er zufrieden war. Er kannte Henry schon vor dessen Berufung zum Präsidenten der Union und wusste um die kleinen Macken seines Kollegen. In der Tat passten die gelegentlichen Schrullen sogar zu ihm. Man könnte sogar fast auf sie schließen, wenn man Otis markantes Aussehen besah. Sein Gesicht war genauso lang und kantig, wie der Rest seiner Erscheinung und zwei ausgedehnte Geheimratsecken hatten sich bereits über der hohen Stirn gebildet. Die schwarz-grau melierten Haare und der dunkle Gehrock mit aufgestelltem Kragen vollendeten die seriös autoritäre Ausstrahlung des Präsidenten.

\par

Richard Bellegardè selbst war äußerlich eher unscheinbar. Er war um einiges kleiner und untersetzter als der Präsident. Aber seine Augen strahlten unterschwellig eine harmlose Freundlichkeit aus, die noch von seiner in die Breite ausgedehnten Nase vervollständigt wurde.

\par

\WR{Vielen Dank noch mal, dass Sie mich auf dieser Reise vertreten haben. Ich weiß, dass der Flug sicherlich kein Zuckerschlecken war}, bedankte sich Otis erneut.

\par

Richard Bellegardè nickte nur. \WR{Jederzeit wieder.}

\par

\WR{Eine gute Beziehung zu Sagittae Primus ist unerlässlich}, erinnerte der Präsident seinen Kollegen unnötigerweise. \WR{Diese Kolonie ist unsere einzige Möglichkeit mit den autonomen Welten Handel zu betreiben. Und obwohl Sie bei dieser Reise nur eine repräsentative Rolle innegehabt hatten, haben Sie trotzdem dazu beigetragen, dass alles das auch so bleibt.}

\par

Der Senator war sich durchaus bewusst, dass Sagittae Primus als wirtschaftlicher Außenposten sehr wichtig war. Und das aufgrund eines Sachverhalts, der sich zehn Jahre vor seiner Geburt ergeben hatte. Damals hatten die Welten im Sinistra-Sektor ihre Unabhängigkeit von der Union erklärt, beziehungsweise sich nach dem Routenkrieg erst gar nicht angeschlossen und damit eine politische Krise ausgelöst. Das Problem hatte hauptsächlich darin bestanden, dass sich die Interstellare Terranische Union geweigert hatte, mit diesen nun autonomen Welten Handel zu treiben. Das ganze Finanzsystem der Union stützte sich unter anderem darauf, dass ein Megacomputer sämtliche Transaktionen~-- egal wie umfangreich oder unbedeutend~-- zusammenfasste und bewertete. Somit ergab sich ein Geldwert, welcher stets der Summe der Sachwerte der Union entsprach. Handel mit den autonomen Welten hätte aufgrund deren unregulierten Wirtschaft zu einer starken Unschärfe bei der Erfassung des Geldwertes geführt.

\par

Die autonomen Systeme hatten wiederum zu wenig eigene Ressourcen besessen, um sich selbst ohne Handel mit anderen versorgen zu können.

\par

Schließlich hatte sich die Welt Sagittae Primus bereiterklärt eine Art Pufferzone zu bilden. Als eine Kolonie der Union waren in ihr alle Währungen gültig. Somit konnten die selbstständigen Welten Waren über Sagittae Primus an die Union verkaufen und umgekehrt. Die Grenzkolonie ging dabei ein gewisses Risiko ein, dass schlecht zu kalkulieren aber dafür recht lukrativ war. Zahlten sie weniger Geld an die autonomen Welten, als deren Waren laut dem Megacomputer \Wr{Perumatre} wert waren, blieb der Mehrwert in auf Sagittae. Zahlten sie aber mehr, blieb ihre Wirtschaft auf dem Minus sitzen. Bislang hatte die Kolonie jedoch davon profitiert, der Korrekturfaktor in einer äußerst komplexen Gleichung zu sein.

\par

\WR{Viele Bürger sehen es als Problem, dass die autonomen Welten Sagittae Primus gegenüber die Preise für den Weiterverkauf an uns fast beliebig festlegen kann}, resümierte Richard Bellegardè. \WR{Aber ich persönlich mache mir darüber weniger Sorgen. Die Kolonie Sagittae Primus lebt gut von dem Handel mit den autonomen Welten. Wenn sie anfangen würden, Wucher zu treiben, würde die Union nichts mehr von ihnen einkaufen und sie bliebe auf ihren erworbenen Waren sitzen.}

\par

Präsident Otis nickte zustimmend und nahm einen neuen Zug aus seiner Pfeife. \WR{Ja, die freie Marktwirtschaft. Ich bin froh, dass unser System, zumindest oberflächlich, wesentlich unkomplizierter ist.}

\par

\WR{Ob unsere Wirtschaftssenatorin dem zustimmen würde ist fraglich}, entgegnete Richard Bellegardè. \WR{Frau Bertram versucht schon, das System zu vereinfachen, seitdem sie in der Politik ist.}

\par

Der Präsident nahm hinter seinem Schreibtisch Platz, während ihm der oberste Senator antwortete. Er klappte sein ellenlanges Buch auf, dass bereits vor ihm lag. Sofort erschienen einige Hologramme, die Otis Tagesplanung anzeigten. Bellegardè hatte sich oft gewundert, wieso der Präsident es vorzog, diese so offen anzeigen zu lassenl, owbohl er sie auch einfach auf einer Seite in seinem Buch hätte lesen können. Dann allerdings, wäre sicher viel weniger aufgefallen, welch ein beschäftigter Mann er war.

\par

Seinen Sinn für Stil vermittelte bereits der aufwendige Ledereinband.

\par

Als Henry Otis den altmodischen Tastaturmodus zuschaltete, erzeugte eine kleine Linse zweidimensionale Umrisse von Tasten auf einer weichen Unterlage auf dem Schreibtisch. Der Präsident begann zu tippen.

\par

\WR{Weswegen wollten Sie mich eigentlich sprechen, Richard?}, fragte Henry Otis, während er seinen Terminkalender überprüfte.

\par

Der Senator faltete seine Hände und wippte mit seinen Füßen auf und ab, wie er es häufig tat, wenn er etwas vorbrachte, das er nicht gerne ansprach. \WR{Es geht um den Besuch der Universitätsklasse. Ich habe gehört, Sie wollen sie hier in ihrem Büro empfangen?}

\par

Präsident Otis war nun ganz Ohr und sah zu Richard Bellegardè auf. \WR{Das ist richtig}, begann er: \WR{Stört Sie etwas daran?}

\par

Der Senator räusperte sich. \WR{Nun ja. Ich habe wirklich nichts gegen die Einrichtung hier aber finden Sie nicht, dass der Konferenzsaal im zwölften Stock optisch etwas mehr hermacht?}

\par

\WR{Schon}, antwortete Otis, \WR{aber ich glaube, dieses Arbeitszimmer schafft eine persönlichere Atmosphäre. Ich will vermeiden, dass sich die Studenten vorkommen, als würden sie in irgendeinen Raum gesteckt, um dort abgefertigt zu werden. Auch wenn die Damen und Herren mit Sicherheit wieder versuchen werden, mich mit ihren vorlauten Fragen in die Bredouille zu bringen. Ganz egal, wo das treffen stattfindet.}

\par

Richard Bellegardè lachte kurz auf. \WR{Das dürfen Sie ihnen nicht übel nehmen. Alle Politikstudenten sind Rebellen~-- genauso wie wir zu unserer Zeit.}

\par

Der Präsident gestand dies seinem Gesprächspartner zu. \WR{Meine Wortgefechte mit meiner Schulleiterin waren damals in der ganzen Universität bekannt…}, Otis unterbrach sich, \WR{und sie hätten mir fast einen Verweis eingebracht.}

\par

Der Senator nickte. Dass der Präsident hin und wieder quer trieb, war ihm schon seit Otis Kampagne zur Abschaffung der Armee bekannt. \WR{Aber gut, dass Sie mich daran erinnern}, merkte er an. \WR{Ich muss unbedingt noch den Tisch entfernen und genügend Sitzmöglichkeiten hier rein bringen lassen.}

\par

In seinem Buch und der entsprechenden Nachrichten, die darin abgebildet war, las Henry Otis die Anzahl der Studenten ab, die er zu erwarten hatte. Außerdem vergewisserte er sich noch einmal über die Uhrzeit. Er hatte noch fünf Stunden um alles vorzubereiten.

\par

Richard Bellegardè zückte sein eigenes Buch und trennte vorsichtig eine Seite heraus, die er dem Präsidenten reichte. \WR{Der Abschlussbericht zu meiner Reise nach Sagittae Primus}, kommentierte er dazu. Der Präsident hatte es schon vor langem aufgegeben, Bellegardè darum zu bitten, ihm die Dateien einfach in seinen Krypta-Raum zu speichern. Der Collonele Minister hatte ihm jedes mal die gleiche Antwort gegeben: Sicherheit.

\par

Otis nahm das Pseudopapier dankend an. \WR{Ich werde mich dann mal wieder auf den Weg machen. Im Büro der Hauptversammlung wartet noch eine ganze Menge Arbeit}, verabschiedete sich der Senator.

\par

Der Präsident sah sofort ruckartig auf. Er schüttelte den Kopf und wendete ein: \WR{Vielleicht sollten Sie sich erst einmal ein paar Tage frei nehmen. Ich bin sicher Ihre Stellvertreterin kann Sie so lange ersetzen.}

\par

\WR{Na ja}, entgegnete ihm Richard  Bellegardè, \WR{sie musste mich ja schon während der ganzen Reise vertreten.}

\par

\WR{Denken Sie darüber nach}, forderte der Präsident noch einmal nachdrücklich.

\par

Senator  Bellegardè nickte und verabschiedete sich. Als er das Büro des Präsidenten verlassen hatte, lehnte sich dieser erst einmal zurück und nahm noch einen Zug aus seiner fast erloschenen Pfeife. Das Gespräch mit der Politik-Klasse würde bestimmt kein Spaziergang werden. Die letzte Gruppe an Studenten und Studentinnen hatte es richtig genossen Henry Otis mit kritischen Fragen zu bombardieren. Sie hatten sich gut auf das Gespräch vorbereitet und die meisten Fragen waren äußerst schwierig zu beantworten gewesen. Er fand großen Gefallen daran, dass das Interesse an der Politik der Union auch nach achtzig Jahren nicht verschwunden war. Aber er verabscheute die altkluge und aggressive Art, die viele der Studenten an den Tag legten. Obwohl er selbst nicht sehr viel anders gewesen war, bevor er vor gut dreißig Jahren aktiv in die Politik eingestiegen war. Schon fast automatisch schrieb seine Hand die wichtigsten Stichpunkte, die mit seiner derzeitigen Politik in Verbindung gebracht wurden, auf eine noch leere Seite seines Buches. Sofort wurden einige Artikel aus der Krypta Scientia zu diesem Thema angezeigt. Er begann, sie nur halb aufmerksam zu überfliegen und betrachtete sich währenddessen auf der Spiegelseite, die sein Buch mittlerweile obligatorisch anzeigte. Erst vor kurzem hatte er sich die Haare schneiden lassen und sein Kragenzwirn saß wie angegossen. Ein selbstsicheres Lächeln kam mittlerweile schon fast von selbst.

\par

Langsam erhob sich Henry Otis aus seinem, mit künstlichem Leder bespannten, Ohrensessel und begab sich zurück an sein Fernrohr. Nachdem er es ein wenig anders ausgerichtet hatte, erkannte er, dass die Parade nach wie vor im Gange war. Die Armee ließ ihn einfach nicht los. Schon seit Jahren nicht. In der Abstimmung von zweitausend dreihundert achtzig zur Abschaffung der Armee vor zwanzig Jahren hatte er eine herbe Niederlage einstecken müssen. Er war damals Vizekanzler der Erde gewesen und hatte sich dafür stark gemacht alle offensiven Kräfte des Verteidigungskonglomerats abzuschaffen. Die Polizei und der Geheimdienst hätten seiner Meinung nach völlig ausgereicht. Die Kampagne hatte keinen schlechten Start gehabt. Es hatte ausgesehen als würden fast sechzig Prozent für eine Abschaffung der Armee stimmen. Aber dann war es zum ersten Anschlag der Capital Fellowship gekommen. Laurenzia Batanides, die mittlerweile verstorbene Anführerin dieser Terrorgruppe, hatte Vizekanzler Otis einen dicken Strich durch die Rechnung gemacht. Auf einmal hatte jeder nach einer starken Hand gerufen, welche die Capital Fellowship vom Himmel hätte Fegen sollen.

\par

Henry Otis bemerkte, wie er unbewusst die Zähne zusammengebissen hatte. Die Erinnerung an den Tag der Abstimmung und das vernichtende Ergebnis ließen dem Präsidenten immer noch die Hände zu Fäusten werden.

\par

Otis Stimmung wurde aber deutlich angehoben, als er durch sein Fernrohr beobachtete, wie eine Bande kleiner Kinder mit Eiern und anderen Lebensmitteln auf die Soldaten warfen. Diese mussten den Scherz ertragen. Der Plan bei einer Parade sah vor, dass niemand aus der Reihe tanzte. Und die Armee hatte sich nicht so stark gewandelt, dass ein Soldat einfach vom Plan abweichen durfte. Der Wurfangriff endete erst, als die Kinder von einigen Sicherheitsleuten vertrieben wurden.

\par

Als der Präsident im Begriff war, einen weiteren Zug aus seiner Pfeife zu nehmen, erschien ein zweidimensionales Hologramm über dem noch immer aufgeschlagenen Buch und zeigte das Gesicht der Sekretärin aus dem Vorzimmer. Eilig legte Henry Otis die Pfeife beiseite und begab sich zu seinem Terminal.

\par

\WR{Frau Pudy, ich hatte Sie doch darum gebeten, mich vorerst nicht zu stören.} Der Präsident achtete darauf, nicht verärgert aber bestimmt zu klingen. Der Gesichtsausdruck der Sekretärin war schwer zu deuten. Er schwankte zwischen mürrischer Sturheit und Gleichgültigkeit.

\par

\WR{Mein Herr, Grandadmiral Burns ist hier. Er möchte Sie sprechen und meint, es sei dringend}, entgegnete die junge Frau in teilnahmslosem Ton.

\par

Henry Otis schüttelte sofort den Kopf. \WR{Sagen Sie dem Admiral, er soll sich gedulden, ich bin gerade beschäftigt.}

\par

Noch bevor der Präsident zu Ende gesprochen hatte, schoss die Sekretärin aus ihrem bequemen Polstersessel hoch und die Kamera begann die Wand hinter ihr scharf zu stellen.

\par

Otis hörte sie nur noch sagen: \WR{Halt! Sie können da nicht hinein! Der Präsident arbeitet…}

\par

Kurz darauf öffnete sich die gläserne Tür und Grandadmiral Norton Burns betrat mit großen Schritten das präsidiale Büro. Schon auf den ersten Blick erschien der Admiral auffällig. Er hatte breite Schultern und wirkte muskulös. Trotz seines fortgeschrittenen Alters, dass durch die graue Farbe seiner kurz geschorenen Haare, zutage trat, machte Burns einen kräftigen und vitalen Eindruck.

\par

Der Oberkommandant des Konglomerats marschierte durch den langen Raum, machte vor dem Schreibtisch halt und salutierte gemäß der Vorschrift. Dann begann er Henry Otis mit seinen Blicken förmlich zu durchbohren. Seine groben Gesichtszüge verharrten wie in einer Starre.

\par

Der Präsident hielt dem Stand und sah gelassen zurück. An sich wirkte der Admiral kontrolliert und ruhig aber Henry Otis kannte Burns schon lange genug, um zu wissen, dass dieser innerlich kochte und er genoss den Anblick. Zur Krönung setzte er ein Lächeln auf und begrüßte den Neuankömmling übertrieben freundlich: \WR{Guten Morgen, Grandadmiral. Stets zu Diensten. Was kann ich für Sie tun?}

\par

Norton Burns nahm die Schmähung mannhaft hin. Sein Gesicht blieb regungslos. \WR{Herr Präsident, Sie haben erneut den Militäretat gekürzt. Und zwar um ein Achtel!} Die leichte Erhebung seiner Stimme am Ende der Aussage des Grandadmirals war das erste bemerkbare Anzeichen seiner Erregung.

\par

Präsident Otis seufzte und faltete die Hände, als müsste er einem Kind etwas erklären, dass es doch nicht verstehen würde. \WR{Nicht ich habe den Etat gekürzt, sondern der Senat. Es war eine mehrheitliche, von den Bürgern unterstützte, Entscheidung. Daher besteht kein Grund weiter darüber zu diskutieren. Des weiteren bin ich überrascht, dass Sie mich erst jetzt deswegen ansprechen. Immerhin war die Abstimmung vor drei Wochen.}

\par

Grandadmiral Burns knirschte dermaßen heftig mit den Zähnen, dass der Präsident das Geräusch vage hören konnte. \WR{Wie Sie vielleicht wissen}, begann Burns, heftig darum bemüht sich im Zaum zu halten, zu antworten, \WR{war ich zu dieser Zeit auf einem Flottenmanöver. Und das ist ein Zweidutzend-Stunden-Job bei dem die meiste Zeit über strikte Funkstille galt.}

\par

Henry Otis schüttelte den Kopf als eine Geste der Unverständnis. Nach dem er den Admiral eine Zeit schmoren gelassen hatte, setzte er zu einer Antwort an: \WR{Grandadmiral Burns, Sie sollten sich wirklich etwas mehr um die Politik der Union scheren. Immerhin sind Sie der Oberkommandant des Konglomerats. Die Bürger erwarten von Ihnen, dass Sie den politischen Aspekt der Armee wahrnehmen. \textit{Kümmern} Sie sich um diese Aufgabe, sie ist wichtig.}

\par

Der Präsident hatte erwartet, dass Norton Burns nach dieser Aussage die Fassung verlieren würde. Ihm hätte es nur Recht sein können. Ein tobender Militär in seinem Büro wäre genau das richtige gewesen um die Wähler mit seiner armeefeindlichen Einstellung anzustecken.

\par

Allerdings wurde der Grandadmiral nur gefährlich ruhig. Er blickte an seiner grauen Uniform in Gala-Ausfertigung hinunter. Ein klares Zeichen um dem Präsidenten seine Kompetenz zu demonstrieren. Die drei wichtigsten Orden, die vom Konglomerat überhaupt vergeben wurden, schmückten Norton Burns linke Brusthälfte. Nicht, dass er nicht noch weitere Orden innehätte, aber das Protokoll sah vor, dass nur immer drei gleichzeitig getragen wurden. Direkt unter den Auszeichnungen hatte der Grandadmiral seine Einsatzabzeichen aufgereiht. Er brachte es auf stolze vier Zeilen.

\par

\WR{Ich denke, ich bin dieser Aufgabe durchaus gewachsen, Herr Präsident}, antwortete er ruhigen Tons. \WR{Aber das ist auch nicht das Thema. Die Etatkürzung ist indiskutabel. Ich musste den Bau von vier mittleren Trägern zurückstellen um die rechtzeitige Fertigstellung der \EN{Königin} zu gewährleisten.}

\par

\WR{Die \EN{Königin}. Ihre neue Gespielin}, gab Otis in einem Tonfall zurück, der den Einwurf des Admirals lächerlich machen sollte.

\par

Dieser lies sich aber auf kein Kräftemessen mit dem Präsidenten ein. Anstatt dessen entgegnete er ernst: \WR{Jeder der mittleren Träger, die nun wegen der Etatkürzung nicht gebaut werden können, hat dreidun fünfdutzend Mann Besatzung. Das sind eintrin eindun und elfdutzend Offiziere, die nun einen anderen Arbeitsplatz brauchen. \textit{Ihre} Wähler, Herr President.}

\par

Henry Otis lehnte sich weiter in seinen Sessel zurück. Er griff nach seiner Pfeife und nahm einen Zug. Ganz bewusst im Beisein des Admirals, um die Geringschätzung ihm gegenüber zu unterstreichen. Nun war es am Präsidenten seinen Gesprächspartner mit Blicken zu durchbohren. Er musste vorsichtig sein. Alles konnte selbst er sich nicht erlauben. Der Admiral konnte ihm problemlos wegen Unsachlichkeit in Amtsangelegenheiten sein Misstrauen aussprechen. Und dies hätte selbst für jemanden, der so fest im Sattel saß, wie Otis, unangenehme Folgen.

\par

\WR{Die Etatkürzung ist nun mal beschlossen}, begann Präsident Otis nun deutlich bedachter zu antworten. \WR{Wie ich schon erwähnte. Es war eine mehrheitliche Entscheidung. Ich habe Sie angeregt aber das heißt nicht, dass ich alleine dafür verantwortlich bin. Falls Sie in Zukunft eher in solche Entschlüsse eingebunden werden möchten, dann müssen Sie Ihren Beruf und ihre politischen Ambitionen eben unter einen Hut bringen. Ich denke, das wäre alles.}

\par

Grandadmiral Burns erhob sein Haupt und verabschiedete sich schlicht mit: \WR{Auf Wiedersehen, Herr Präsident.}

\par

Kurz darauf war er verschwunden. Henry Otis seufzte. Er nahm noch einen Zug aus seiner Pfeife. Später hätte er dazu keine Zeit mehr. Dann müsste er sich mit einer Horde Kinder herumschlagen.