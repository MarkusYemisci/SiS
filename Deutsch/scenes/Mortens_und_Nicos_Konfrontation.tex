Das Dröhnen der Maschinen erstarb. Morten war schon länger aus seinem neuen Jäger ausgestiegen. Sein beiden Flügelmänner waren gerade jetzt erst gelandet und die Triebwerke ihrer Flieger verloren langsam ihr charakteristisches rotes Glühen.

\par

Verglichen mit den Haien, von denen Morten vor kurzem noch einen geflogen war, waren diese Jäger der \EN{Falken}-Klasse wahre Schönheiten. Von oben betrachtet, wirkten sie wir fast wie ein Pfeil. Unter ihren stromlinienförmigen Flügeln trugen sie eine beachtliche Raketenlast. Diese und ihre vier Strahlenkanonen machten sie zur perfekten Wahl, wenn es um die Bekämpfung anderer Jäger ging. Auch für Einsätze innerhalb einer Atmosphäre eigneten sie sich gut.

\par

Zwei der vier Strahlenwaffen waren zum Schnellfeuern gedacht. Diese hatte Mortens Staffel tatsächlich während des Trainingsfluges abgefeuert. Die Rohre, die neben der Pilotenkanzel in den Rumpf eingelassen waren, hatten sich durch die enorme Hitze an ihren Enden bläulich bis bräunlich verfärbt.

\par

Die beiden anderen Strahlenwaffen waren mittig an den Tragflächen montiert. Sie wiesen eine deutlich geringere Feuerrate auf und waren während des letzten Übungseinsatzes unbenutzt geblieben.

\par

Morten winkte seinen beiden Flügelmännern zu. Er kannte sie noch nicht sehr gut und nahm sich vor, sich demnächst ausführlicher mit ihnen zu unterhalten. Beide waren auf die \EN{Regenvogel} versetzt worden, nachdem diese ihren Reparaturaufenthalt im Raumdock begonnen hatte.

\par

Morten erinnerte sich noch gut daran, wie er sich gefühlt hatte, als er den Helm eines Toten aufgesetzt hatte. So würde es nun den meisten der neuen ergehen.

\par

Zumindest innerhalb des Hangars waren die meisten Schäden behoben worden. Einige der freiliegenden Rohre hatte man tatsächlich verkleidet. Offiziell galt noch nicht das Kriegsrecht. Doch besonders nach dem Fall von Gygni war dem Konglomerat vom Permutare voller Zugriff auf die Geldreserven der Union erteilt worden.

\par

Kurz darauf ergatterten zwei Dinge Mortens Aufmerksamkeit. Das eine waren zwei neue Jäger, die noch deutlich moderner wirkten, als die Falken. Sie gehörten zum Modell F dutzend sechs und hatten die Bezeichnung \WR{Rapier}. Ihre Konstruktion wirkte optisch weniger ansprechend. Fast langweilig, wie Morten fand. Der Rumpf war ein gerader Zylinder mit rundem Bug auf dessen hinterer Seite das Cockpit angebracht war, umgeben von zwei leistungsstarken aber unauffälligen Triebwerken. Die Flügel zeigen nicht wie bei den Falken nach hinten, sondern senkrecht zu den Seiten hinaus. Sie waren außerdem eher Quader, als wirkliche Tragflächen. Im Falle des Rapier befanden sich die Raketen in Boxen innerhalb der Flügel. Ein Ausdruck der modularen Bauweise dieses Jägertyps. Statt in einem Stück in eine Werkstatt verfrachtet werden zu müssen, konnten Teile des Rapier ausgebaut und an anderer Stelle repariert werden.

\par

Doch was Mortens Blick noch mehr einfing, war Anna Farley, wie sie mit Nico Curiosa sprach. Er selbst hatte nicht mehr mit dem Ex-Piloten gesprochen, nachdem ihn Funken über dessen Vergangenheit in Kenntnis gesetzt hatte. Auch die anderen schienen ihn zu meiden.

\par

Schließlich interessierte es ihn so sehr, was sie mit ihm zu schaffen hatte, dass er sich zu den beiden begab, die sich am Rande des Hangars, an der Treppe zu den Vorbereitungsräumen der Piloten aufhielten.

\par

\WR{Captain Fiscale ist sich sicher, dass das in Ordnung geht}, versicherte Anna. \WR{Wir brauchen jetzt jeden guten Piloten. In Kriegszeiten ist es erlaubt, jede Person in den Dienst des Konglomerats zu bitten. Ungeachtet des… Hintergrunds.}

\par

Nico seufzte. \WR{Niemand wird mit einem ehemaligen Kämpfer der Capital Fellowship an seinem Flügel fliegen wollen. Ich weiß, Sie haben sich für mich eingesetzt. Aber ich befürchte nach allem, ich bin der falsche.}

\par

Morten stellte sich ungefragt zu den beiden. \WR{Ganz recht. Vielleicht überzeugen Sie die Shutek ja von ihrer Idiologie und sie kommen auf die Idee, uns in den Rücken zu fallen.}

\par

Nicos Blick schoss förmlich zu Morten hinüber. \WR{Die Gerüchteküche funktioniert also noch.}

\par

\WR{In diesem Fall braucht es keine Gerüchte}, entgegnete Morten mit vor der Brust verschränkten Armen. \WR{Sie sind doch ein verurteilter Verbrecher, oder?}

\par

Anna Farley positionierte sich zwischen den beiden und sah in Mortens Richtung: \WR{Herr Curiosa hat seine Strafe verbüßt. Wir sollten uns nun nicht durch…}

\par

Nico unterbrach die Geschwaderkommandantin. \WR{Nein, Lieutenant Witwer hat recht. Ja, ich bin ein Krimineller und ja, ich habe gegen die Union gekämpft. Aber diese Tage liegen hinter mir. Auch wenn einige das nicht akzeptieren können. Besonders bitter ist es, wenn diese Leute einen nicht einmal richtig kennen und trotzdem ein Urteil fällen.}

\par

Morten schnaubte. \WR{Sie glauben doch nicht im Ernst, dass Ihnen das irgendwer abnimmt. Nichts läutert einen Capital-Follower. Auch nicht diese Invasion.}

\par

\WR{Lieutenant Witwer}, begann Anna Farley und klang dabei mehr als streng, \WR{vielleicht haben Sie nicht richtig aufgepasst. Aber da draußen warten Unbekannte auf uns, die uns töten wollen, ohne dabei mit der Wimper zu zucken. Sie haben bereits zwei unserer Kolonien vernichtet und ich versichere Ihnen, sie hören nicht in Cygni auf. Ich brauche jetzt weder Paragraphenreiter noch Drückeberger. Ich brauche Soldaten.}

\par

\WR{Sollen Sie haben}, antwortete ihr Nico Curiosa. \WR{Wenn Sie jemanden finden, der mein Flügelmann sein kann, dann bin ich dabei.} Ton und Blick des Veteranen spiegelten Entschlossenheit wieder. Morten verstand. Er hatte ihn unbeabsichtigt dazu ermutigt, wieder in ein Cockpit zu steigen.

\par

Anna nickte zufreiden. \WR{Sehr gut. Ich bin bereit, Sie aufgrund ihrer Leistungen bei den privaten Sicherheitsdiensten, für die Sie geflogen sind, entsprechend zu befördern. Sie Sind ab sofort Major ehrenhalber. Herzlichen Glückwunsch.}

\par

Morten lachte kurz auf. \WR{Major? Das muss ein Witze sein. Piloten verdienen sich ihren Rang. Dieser Mann hat…}

\par

Nun war es an Anna Farley, ihr Gegenüber zu unterbrechen. \WR{Wenn Sie Problem mit dieser Beförderung haben, legen Sie offiziellen Protest ein, Lieutenant. Bis dahin lernen Sie die heutige Lektion. Regeln sind dazu da, uns zu dienen, nicht umgekehrt. War das deutlich?}

\par

\WR{Ja, Madam}, entgegnete Morten zackig und trat ab, nachdem ihn Farley wegtreten ließ.
