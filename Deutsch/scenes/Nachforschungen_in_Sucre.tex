Um nach Sucre zu kommen, hatten Klaus Rensing und Laura Gethas in einen kleineren Intrakontinentalzug umsteigen müssen. Laura hatte sich mit ihrem Partner darauf geeinigt, für den Rest der Fahrt, Musik zu hören. Sie fühlte sich seltsam dumpf, fast als stünde sie unter Schock. Aber ihr Verstand funktionierte noch genauso klar wie immer. Auf ihrem eigenen Handcomputer sah sie sich eine Grundrisszeichnung der vermeintlichen Wohnung des Hackers an. Es gab etliche Fenster und Fluchtmöglichkeiten. Sie beschloss, Klaus auf der Straße Wache halten zu lassen, während sie sich die Wohnung ansehen würde.

\par

Der Zug hielt an und Laura fürchtete sich schon fast vor dem Moment, an dem sie ihre Kopfhörer abnehmen und sich der Umwelt stellen musste. Als sie zusammen mit Klaus den Bahnhof von Sucre betrat, empfand sie etwas, dass sie nicht erwartet hätte~-- Erleichterung. Auf den ersten Blick wirkte Sucre alt und verfallen. Die Stadt bestand bereits seit lange vor der Seuche und war weitestgehend von ihr verschont geblieben. Laura hatte sie schon oft gemeinsam mit ihrer Exfreundin besucht und genoss vor allem ihre relative Abgeschiedenheit.

\par

Als sie ihren Blick über die Stadt streifen ließ, kam es ihr vor, als wäre die Zeit für Sucre einfach stehen geblieben. Kein Glas, keine Stahlgerüste. Weiß von den Mauern der Häuser und ein sattes Braun von den Ziegeln ihrer Dächer herrschten vor. Die Gebäude selbst waren größtenteils restauriert worden, doch der für Sucre typische koloniale Baustil war erhalten geblieben.

\par

Die Wärme ließ Laura Schweißperlen auf die Stirn treten, doch sie fühlte sich trotzdem kalt.

\par

\WR{In Ordnung}, begann Klaus, die Straßenkarte in seinem Buch lesend. \WR{Die Wohnung sollte nicht allzu weit weg sein. Ich glaube, wir sind in fünf Minuten da.}

\par

Laura nickte nur und folgte ihrem Partner. Die beiden durchquerten eine alte Straße, die gepflastert und nicht geteert war. Gerade trat eine ältere Frau in ihren ummauerten Garten hinaus, streifte über das bräunliche Gras und hängte eine Ladung Wäsche an eine Leine. Über dem Hauseingang hing eine ausgebleichte Flagge der Union.

\par

\WR{Bei den Temperaturen hier sind Schnellreiniger wohl überflüssig}, dachte Klaus laut und wischte sich mit dem Handrücken den Schweiß von der Stirn.

\par

Lauras Blick verlor sich in den grünen Hügel, die Sucre einschlossen. Es gab einen hohen Aussichtsturm, der noch aus dem letzten Jahrhundert stammte. Vereinzelt zogen sich ein paar kleine Wälder über die Kämme der Hügel und eine kleine Kapelle fügte sich so malerisch wie nahtlos ein.

\par

Die Religion hatte in den hundert Jahren nach der Seuche wieder einen Aufschwung erlebt, doch von den alten Riesen hatte keiner in seiner damaligen Form überlebt. Es gab zwar im wesentlichen noch die gleichen Glaubensrichtungen, doch sie hatten ihr Gesicht verändert und waren teilweise ineinander übergegangen.

\par

Laura hatte es selbst nach Jahren noch nicht mit sich ausmachen können, ob sie an eine Art Gott glauben wollte oder nicht.

\par

Aus einem nahen Straßencafé drang Gitarrenspiel. Ein Mann mit krausen Haaren und einem schwarzen Bart saß auf einer Mauer vor dem Restaurant und zupfte die Saiten. Seine Augen waren offen, doch er schien seine Umwelt kaum wahrzunehmen. Die wenigen Gäste des Cafés lauschten seiner Musik genauso gedankenverloren.

\par

Das Lied, das der Gitarrist spielte, klang leicht und fröhlich. Ungewöhnlich, wie Laura fand. Die Klänge dieser Region waren meistens melancholisch und regten sie zum Nachdenken an. Doch dieses beschwingte Stück regte sie bloß auf. Wahrscheinlich, so glaubte sie, wusste noch keiner der Gäste des Cafés etwas von der Ansprache des Präsidenten.

\par

Glaubt ihr, die Welt geht an euch vorbei, bloß weil ihr es langsam angehen lasst?, fragte sich Laura erbost. Doch dann wurde ihr schnell klar, dass sie nicht wirklich wütend auf diese Menschen war. Sie gestand sich ein, dass sie sogar selbst lieber an ihrer Stelle wäre. Auf einer Mauer sitzend, mit einem Glas Mate in der Hand und den Blick in den fast wolkenlosen Himmel gerichtet.

\par

Klaus und Laura passierten einen Schokoladenladen. Selbst nach Jahrhunderten war Sucre noch für seine süßen Spezialitäten bekannt. Die Tafeln und Formen im Schaufenster sahen verlockend aus. In allen Brauntönen und nur hier und da mit weißen Exemplaren durchsetzt präsentierte sich die Auswahl. Mit ihrer ehemaligen Freundin hatte Laura schon oft die Konditoreien der Stadt besucht. Geiz kannten die Verkäufer nicht und ließen gerne probieren~-- von jeder Sorte, wenn man es wollte.

\par

Ihr Partner stieß Laura an sagte: \WR{Gleich um die Ecke.}

\par

\WR{Folgendes}, erwiderte die Agentin. \WR{Warte bitte hier und behalt alle Ausgänge im Auge. Ich schau mir die Wohnung an und wenn er flieht, schnapp ihn dir.}

\par

Klaus wollte gerade widersprechen. Er hatte es nicht gerne, wenn er Schmiere stehen musste, während jemand anderes die Arbeit machte. Aber angesichts des Gesprächs im Zug, nickte er bloß.

\par

Laura erwiderte sein Nicken und huschte schnell um die Ecke. Fast hätte sie eine Frau umgerannt, die mit zwei vollen Taschen beladen die Straße entlang lief. \WR{Entschuldigung}, rief ihr Laura nach, doch die Angerempelte ging einfach weiter.

\par

Die Agentin erreichte schnell die Haupttür des kleinen, einfachen Hauses mit einem flachen, mit roten Ziegeln bedeckten Flachdach. Laura griff in die Innentasche ihres Gehrocks und zog ihre Dienstwaffe heraus. Das schwarze Metall glänzte im hellen Sonnenlicht.

\par

Die Eingangstür glitt zur Seite und sie trat in den Hausgang ein. Laut des Berichts des Vermieters lebte der Verdächtige gleich im Erdgeschoss. Es war dunkel. Dennoch fand Laura die Wohnungstür schnell. Natürlich öffnete sie sich nicht vor ihr, darum gab sie den Ausnahmecode ein, der ihr, zusammen mit ihrem Fingerabdruck, Einlass in die meisten Räume gewährte. Doch auch dieses mal tat sich nichts. Laura glaubte nun fest daran, vor der Wohnung des Hackers zu stehen. Es brauchte einiges Geschick, um einen Türöffner umzuprogrammieren oder gar einen eigenen einzubauen.

\par

Das war natürlich ein Problem. Wenn Sie es mit ihrem Geheimdienstcode nicht schaffte, die Tür aufzubekommen, dann würde es dem Vermieter auch nicht gelingen. Das hieß, sie brauchte eine Techniker.

\par

Klaus hat die Wohnung im Auge, dachte sie sich und entschloss sich, einfach zu klingeln. Kurz darauf schoss die Tür auf.

\par

Laura erinnerte sich an die Grundrisszeichnung des Hauses. Rechts und links neben dem Eingang konnte man sich leicht verstecken. Daher hielt sie ihre Waffe dicht am Körper, um sie nicht gleich aus der Hand geschlagen zu bekommen. Geschickt sicherte sie den Flur ab. Zu hören oder zu sehen gab es nichts. Die Wohnung wirkte so steril und unpersönlich wie ein Hotelzimmer. Und das obwohl der Mieter angeblich bereits seit ein paar Jahren in ihr leben sollte.

\par

Durch die Fenster drang ein mattes Licht in das Haus hinein. Laura sah in die Küche. Ein Herd, ein Frischhalter, ein Tisch. Auf der Spüle standen ein paar Teller und ein Vorhang verdeckte das Fenster. Alles wirkte seltsam leer und verlassen und Laura seufzte, als sie das Gefühl bekam, wieder zu spät zu sein.

\par

Langsam arbeitete sie sich in Richtung Wohnzimmer vor. An einer Garderobe im Flur hingen zwei Jacken. Viel zu dick für den gesamten Kontinent, mit Ausnahme vielleicht der Südspitzen von Argentinien oder Chile.

\par

Das Wohnzimmer wirkte genauso steril wie der Rest. Auf einem stilvollen aber einfachen Tisch stand ein Drei-D-Projektor und in alle vier Wände waren Lautsprecher eingelassen. Lauras Freunden hatte selbst ähnliche und sie waren nicht billig gewesen.

\par

Aber keine Bilder, nicht einmal Blumen ließen auf einen menschlichen Bewohner schließen. So sehr sie suchte, Laura konnte keinen einzigen persönlichen Gegenstand entdecken. Das ganze Wohnzimmer hätte auch genauso gut ein Ausstellungsstück in einem Möbelhaus sein können.

\par

Laura fuhr zusammen. Ein Gerät erwachte plötzlich zu Leben. Es war eine Kommunikationseinheit, deren Bildschirm nun blaues Licht in den Raum warf. Ein aufwendiges Modell, dass einen DDV-Projektor beinhaltete. Jemand anderes, der ein ähnliches Gerät hatte, konnte also sein Abbild bei einem Anruf zu seinem Gesprächspartner projizieren lassen. Das Piepen wirkte neutral, strapazierte aber dennoch Lauras Nerven.

\par

Sie trat an das Gerät heran und las auf dem Monitor, dass sich gerade ein Anrufer meldete. Die Kommunikationseinheit hatte einen Hörer, was ungewöhnlich war. Normalerweise wurden nur noch dann Hörer verwendet, wenn niemand mitbekommen sollte, was auf der anderen Seite gesagt wurde.

\par

Zögerlich und in der Hoffnung, etwas über die Identität des Hackers in Erfahrung bringen zu können, nahm Laura Gethas ab und hielt sich die Hörmuschel ans Ohr.

\par

\WR{Guten Tag, Frau Agentin}, grüßte eine Stimme mit mechanischem Ton. Laura erkannte sie wieder. Sie war Teil eines Computerprogramms, das verwendet wurde, um Blinden über eine Tastatur eine verbale Kommunikation zu ermöglichen. Hier diente sie wohl eher dazu, die Identität des Sprechers geheim zu halten.

\par

\WR{Wer ist da?}, fragte Laura, wobei sie Mühe hatte, ihren Ärger zurückzuhalten.

\par

Die Stimme antwortete sofort. \WR{Derjenige, den sie suchen. Ich weiß, dass sie jetzt gerade in meiner Wohnung sind. Und ich weiß, dass sie alleine sind.}

\par

Laura brachte sofort ihre Pistole in Schussposition und drehte sich zur Tür hin um. Doch niemand war zu sehen und sie konnte auch keine Schritte hören. Vorsichtshalber trat sie ein paar Schritte vom Fenster weg. Doch jemand von draußen sollte wegen der heruntergelassenen Rollläden nicht hineinsehen können.

\par

\WR{Keine Angst}, fuhr der Sprecher fort. \WR{Sie und ihre Partner, der wahrscheinlich gerade draußen auf mich wartet, sind nicht in Gefahr.}

\par

\WR{Woher wissen Sie, dass ich gerade in Ihrer Wohnung bin?}, fragte Laura ihren Gesprächspartner, um ihn zu beschäftigen. Während er antwortete, zückte sie ihr Buch und befahl ihm, sich mit der Kommunikationseinheit zu verbinden und den Anrufer zu ermitteln.

\par

\WR{Na ja, Sie haben mein Telefon abgenommen. Aber falls Ihre eigentliche Frage war, wie ich wissen konnte, dass Sie gerade \textit{jetzt} in meine Wohnung gekommen sind, dann lautet die Antwort: Ich habe ihren Türöffnungscode gelesen und Ihren Fingerabdruck erkannt.}

\par

\WR{Unsinn}, entgegnete Laura verärgert, als ihr Buch nur Fehlermeldungen ausspuckte. \WR{Sie können nicht wissen, welcher Fingerabdruck zu wem gehört. Diese Daten sind geheim.}

\par

Die Stimme klang amüsiert~-- Gefühlsregungen in die Worte einzubringen war eine Spezialität dieses Programms. Ein Smiley genügte. \WR{Nun, Vizeberaterin Gethas. Ich bin nicht umsonst Hacker. Ich habe mir ihre Dienstakte durchgelesen und ich denke, Sie sind sauber.}

\par

\WR{Sauber?}, fragte Laura erstaunt über die Fähigkeiten ihres Gegners.

\par

Die Stimme fuhr fort: \WR{Sie sollten übrigens nichts versuchen, mich mittels elektronischer Mittel zu verfolgen. Nicht dass sich Ihr Buch ein Virus einfängt.}

\par

Gerade in diesem Moment wurden die Seiten ihres Buches von Nanotinte geschwärzt und es reagierte auf nichts mehr. \WR{Verdammt}, murmelte sie.

\par

\WR{Keine Sorge}, antwortete ihr die Stimme. \WR{In ein paar Stunden können Sie Ihr Spielzeug wieder benutzen~-- bis dahin müsste es mir seine Speicherbank übermittelt haben.}

\par

Laura zögerte nicht. Sie zerriss das Buch, ließ es auf den Boden fallen und trat mehrfach fest mit den Absätzen auf den Rücken, um die Zentralprozessoren zu zerstören.

\par

\WR{Darauf war meine gesamte Musiksammlung gespeichert}, blaffte die Agentin. \WR{Ich hatte sie sonst nirgends hinterlegt. Wenn ich Sie erwische, hoffen Sie, dass ich nicht selbst das Verhör leiten darf.}

\par

\WR{Sind Sie heute ein wenig gereizt, Frau Gethas?}, fragte die Stimme. \WR{Ist es wegen Ihres Verwandten auf Pollux? Keine Angst. Ich habe mich bereits für Sie erkundigt. Ich weiß auch, dass Sie es bis jetzt nicht fertig gebracht haben. Marcello Caruso gehört zu den Überlebenden. Sie können einem Ex-Terroristen namens Nico Curiosa danken. In dessen Schiff ist er zusammen mit seiner Familie entkommen.}

\par

Laura fing an zu schluchzen. Tränen schossen ihr in die Augen und sie musste sich mit Gewalt an der Kommode festhalten, an die sie lehnte, um nicht zu Boden zu sinken. Auf einmal überfiel sie das irrationale Bedürfnis, sich bei dem Unbekannten zu bedanken.

\par

Stattdessen fragte sie nur: \WR{Was wollen Sie von mir?}

\par

Die Stimme blieb eine Weile lang still. \WR{Ich will mich Ihnen stellen. Und zwar nur Ihnen! Nicht Ihrem Partner oder sonst jemandem, von dem Sie glauben, man könnte ihm oder ihr vertrauen.}

\par

\WR{In Ordnung}, brachte Laura hervor. \WR{Wir treffen uns und ich nehme sie persönlich in Schutzhaft.}

\par

Die Stimme antwortete sofort: \WR{Nicht so schnell. Ich brauche Ihre Versicherung, dass niemand davon erfährt, bevor Sie mich festnehmen.}

\par

Laura lachte und war darüber selbst sofort überrascht. \WR{Hab ich Sie da richtig verstanden? Sie wollen, dass ich alleine an einen Ort komme, den wahrscheinlich Sie bestimmen und davor nicht einmal jemandem etwas davon erzähle? Woher soll ich wissen, dass Sie mich nicht einfach ermorden?}

\par

\WR{Das können Sie nicht wissen}, entgegnete die Stimme. \WR{Sie haben die Wahl. Entweder sie treffen sich mit mir und spielen nach meinen Regeln. Dann werde ich auspacken. Oder sie misstrauen mir und ich entkomme Ihnen für immer.}

\par

\WR{Darauf würde ich nicht wetten}, entgegnete Laura bitter.

\par

\WR{Das spielt keine Rolle.} Die Stimme nahm einen ungeduldigen Klang an. \WR{Fragen Sie sich einfach, was ich davon hätte, Sie zu ermorden. Nicht mehr als dass das der halbe Geheimdienst nach mir suchen und ich wahrscheinlich den falschen in die Hände fallen würde. Und denken Sie daran, was sie davon haben, mich ganz ohne Widerstand festnehmen zu können~-- im Gegensatz zu Ihrem Versagen in Freiburg.}

\par

Laura sah auf die zerbrochenen Rest, ihres Handcomputers. \WR{Sagen wir, ich gehe auf Ihr Angebot ein.}

\par

\WR{Dann treffen Sie mich heute Abend in Oslo. Svoldergata drei. Null Uhr drei dutzend, Ortszeit.}

\par

Er kennt wohl auch meine Arbeitszeiten, dachte sich Laura. Schließlich antwortete sie: \WR{In Ordnung. Ich bin einverstanden. Aber wenn Sie nicht da sind, dann setze ich Ihnen jedes einzelne Lied auf die Rechnung, dass auf meinem Handcomputer gespeichert war~-- und das wird teuer.}

\par

Natürlich würde sie das sowieso tun. Aber in diesem Moment war ihr keine bessere Drohung eingefallen.

\par

\WR{Ob ich da sein werde, hängt ganz von Ihnen ab}, antwortete die Stimme. \WR{Und glauben Sie nicht, Sie könnten mich reinlegen. Wenn Sie mit irgendjemandem über unser Treffen reden, werde ich das wissen. Und dann haben Sie verspielt.}

\par

Laura legte auf. Sie hörte schnelle Schritte auf der Straße. Ihren Partner erkannte sie sofort am Klang seiner Schuhe. Eilig hob sie die Trümmer ihres Handcomputers auf und stopfte sie in ihre Gehrocktasche.

\par

Gerade kam Klaus Rensing in die Wohnung gestürmt, mit der Waffe feuerbereit. Als er sie sah, wirkte er zunächst verwirrt. Dann fragte er: \WR{Ist alles in Ordnung? Du hast dich nicht gemeldet.} Sie nickte nur.

\par

\WR{Gab's irgendwas?}, fragte Klaus weiter.

\par

In Lauras Brust krampfte sich etwas zusammen. Sie vertraute ihrem Partner schon seit vielen Jahren uneingeschränkt und hatte keine Geheimnisse vor ihm. Besonders aber hatte sie ihn noch nie belogen.

\par

\WR{Nein}, schwindelte sie und hoffte, dass er es ihr abkaufen würde. \WR{Diese ganze Wohnung ist wie eine Hotelsuite. Ich glaube, unser Mann wohnt hier. Aber er war ziemlich vorsichtig.}

\par

Sie wünschte, Klaus Gesichtsausdruck lesen zu können. Doch bevor sie aus ihm schlau wurde, drehte sich ihr Partner um und zog eine lange Leuchtröhre aus seinem Sakko. Er begann, damit die Umgebung abzuleuchten. Hier und da tauchten einige Abdrücke auf, doch sie hatten alle eines gemeinsam. Anstatt das Reliefmuster einer Fingeroberfläche zu zeigen, waren es bloß Flecken.

\par

\WR{Hast du so etwas schon einmal gesehen?}, fragte Klaus nachdenklich.

\par

\WR{Ja}, antwortete ihm seine Partnerin. \WR{Dieser Kerl hat sich die Fingerkuppen geglättet. Man kann das mit speziellen Mitteln heutzutage relativ schmerzlos erreichen. Bei einigen Präparaten ist die Wirkung sogar umkehrbar.}

\par

Und sie wusste auch, wer solche Mittel anwendete.