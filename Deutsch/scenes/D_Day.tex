Schwarzer Bär trommelte genauso wie der Rest seiner Division mit den Schutlerstützen ihrer Gewehre auf den Boden des Transporters. Angeschnallt war niemand, obwohl es Vorschrift gewesen wäre. Die meisten pressten sich in ihre Sitze, oder hielten sich an den zahlreichen Metallstreben fest.

\par

Der Bauch des plumpen Landeschiffs beinhaltete zahlreiche Bänke für Soldaten aber auch ein großes Ladevolumen für Panzer und Läufer. Im Fall der Maximus elf war der Platz aber für weitere Truppenplätze eingesetzt worden. Kreuzpunkt war die Heimat zweier Bataillone der Phalanx. Kriegsgerät am Boden war also zur Genüge vorhanden.

\par

Der Geruch von Schweiß füllte nun die Innereien des Schiffes aus und dessen Antriebsdröhnen übertönte fast die Kampfschreie und das Trommeln seiner Insassen. Nur ein schmaler Glasstreifen bot einen Blick nach außen und seitdem die Maximus elf durch die Wolkendecke gesunken war, war ohnehin nicht mehr viel zu sehen. Doch der satt orangene Schein einer Explosion tauhte dennoch das Innere des Schiffes in einen matten Schein.

\par

Sofort verklangen die Rufe der Infanteristen, denn einige hatten erkannt, dass es sich bei dem getroffenen Objekt um einen anderen Landetransporter gehandelt hatte. Auch Schwarzer Bär schloss kurz die Augen und zwang sich, nicht an die Kameraden und vielleicht sogar Freunde zu denken, die er gerade verloren hatte.

\par

Stattdessen brüllte er: \WR{Dafür lassen wir diese Arschlöcher bluten!} Sogleich erklang lauter Beifall und wieder wurde mit den Gewehren auf den Boden getrommelt. Vielleicht kam es ihm nur so vor, doch irgendwie schienen Schwarzer Bärs Kameraden leiser geworden zu sein. Die Angst, mit jedem Augenblick zum nächsten Abschuss zu werden, spürte auch er.

\par

Sein Blick ging zum Einzigen, was ihm nun ein vages Gefühl der Sicherheit gab. Er trug ein schwere Maschinengewehr, das auf ihn immer wie ein langer Kasten mit zwei Rollen gewirkt hatte. Die erste enthielt frische Patronen, die zweite ein noch leeres Band für die verbrauchten.

\par

Fischauges Scharfschützengewehr dagegen war deutlich filigraner und strotzte nur so vor kleinen Schaltern, Rädchen und anderen Bedienelementen. Auch das Sturmgewehr \Wr{Eins Dutzenddrei}, das den Quasistandard unter den Phalanx-Waffen darstellte, wirkte mit seinem Zielfernrohr, dem Kleinstgranatenwerfer und der angebrachten Lampe deutlich filigraner.

\par

\WR{Bereitmachen, Landung in T minus zwei dutzend}, quäkte der Lautsprecher aus dem Steuerraum des Landefahrzeugs. Im Hintergrund war eine zweite Stimme zu hören gewesen, die von starkem Bodenfeuer gewarnt hatte.

\par

\WR{Und seid bloß vorsichtig}, hängte der Pilot an. \WR{Wir haben da unten feindliche Läufer gesichtet. Bezeichnung: \Wr{Megaschreiter}.}

\par

Das schien eine, der Soldaten zu viel zu werden. Er übergab sich mit schneeweißem Gesicht auf das Deck des Transporters. Seine Nebenmänner sahen verschämt zur Seite, doch schwarzer Bär suchte den Blickkontakt zu dem Mann. \WR{Keine Panik, kleiner}, rief er ihm zu, als sie sich endlich beide ansahen. \WR{Die können nichts. In vier fünf Stunden sitzen wir schon wieder auf der \EN{Heinlein} und besaufen uns.} Der Soldat rang sich ein gequältes Lächeln ab.

\par

Kurz darauf setzte das Schiff auf. Es klang wie Hagel, als die ersten feindlichen Salven gegen den Transporter prasselten. Ein Zug eisig kalter Luft drang ins Innere, als sich der Laderaum öffnete und den Blick auf eine verschneite Lichtung freigab.

\par

\WR{Schneller Ausstieg, Leute!}, rief Präfaktin Sali, die sich während der gesamten Landeoperation seltsam ruhig verhalten hatte. Sofort bildeten sich zwei Reihen hinter ihr, als sie aus dem Schiff spurtete, das Gewehr im Anschlag.

\par

Auch Schwarzer Bär stieß sich von der Wand ab und rannte los. Sein schweres Maschienngewehr hielt er aber noch gesenkt. Obwohl es bei seinem Körperbau kein Problem für ihn gewesen wäre, gleich zwei dieser Waffen zu halten, wünschte er sich nun trotzdem eine deutlich leichtere Strahlenwaffe, wie sie nur Spezialeinheiten und der Geheimdienst benutzen durften.

\par

Je näher er dem Ausgang kam, umso mehr erkannte er von der Landezone. Eigentlich hätten sie südlich einer kleineren Bergkette aufsetzen sollen, hinter der sich bereits das Meer an zahlreichen vereisten Küsten entlang erstreckte.

\par

Doch davon war nichts zu erkennen. Ein bis zwei Kilometer weiter erhob sich ein großer Berg mit mehreren kleineren Ausläufern. Vermutlich gehörte er zu dem Massiv, dass die Landetruppe überfliegen hätte sollen. Die Shutek waren also nicht nur im Nordern, sondern auch an der Südfront weiter vorgerückt, als erwartet.

\par

Am Rand eines nahen Waldstückes blitzte es unentwegt. Eine Gruppe von Shutek musste sich dort versteckt halten und eröffnete nun aus der Deckung heraus das Feuer. Ihre roten Strahlen schlug unerbittlich auf die Schutzfelder des Transporters. Normale Infanteriewaffen waren in der Regel kein Problem für die Blocker eines solchen Schiffes, die nicht nur das Gefährt selbst, sondern auch aussteigende Soldaten abschirmten. Aber Schwarzer Bär hörte nur das tiefe Grollen der feindlichen Strahlengewehr. Lautes Knallen oder Rattern, wie man es von chemisch getriebenen Waffen erwarten würde, fehlte komplett.

\par

Was zwei Dinge bedeutete. Zum einen besaßen die Shutek Infanteristen allesamt Strahlenwaffen und zum anderen gab es weit und breit keine eigene Verstärkung.

\par

Rana Sali gab einige Schüsse aus ihrem Sturmgewehr ab. Das Mündungsfeuer ließ den weißen Schnee in der Dämmerung hellgelb schimmern. Auch in Übungen war sie ihrer Division stets vorangeschritten um die eigene Opferbereitschaft zu demonstrieren.

\par

Die, nach wie vor kaum zu erkennenden, Shutek erkannten sie und konzentrierten ihr Feuer auf die Präfektin. Getroffen sackte sie zusammen. Ihre Hand krallte sich um den Abzug und sorgte dafür, dass ihre Waffe nach feuerte, als sie bereits bäuchlings im Schnee lag.

\par

Sofort hielt der Soldat zu Schwarzer Bärs rechten inne. Entsetzt sah er zur sterbenden Kommandantin und dem intensiven Abwehrfeuer der feindlichen Schützenreihe. Die Entladungen ihrer Gewehre verrieten zwar ihre Position, ließen die dunklen Stellen zwischen den Bäumen aber auch auf eine unheimliche Art und Weise immer wieder dunkelrot aufblitzen.

\par

Schwarzer Bär packte den Mann am Rucksack und zerrte ihn mit sich. Hinter den beiden bildete sich bereits ein kleiner Stau und so, wie er die Piloten der Starforce kannte, welche auch diesen Transporter steuerten, wollten sie vermutlich längst wieder abgehoben haben.

\par

Gemeinsam rannten die beiden Soldaten aus dem Gefährt. Nun waren sie~-- so wie die anderen, die mit ihnen ausgestiegen waren~-- erstes Ziel der Shutek. Einer der roten Strahlen schlug genau vor Schwarzer Bärs Füßen ein und schleuderte ihm eine Ladung von Schnee und Dreck ins Gesicht. Die plötzlich hereinbrechende Kälte spürte er gar nicht.

\par

Er ging noch einige Schritte und warf sich dann mit nach wie vor kaum geöffneten Augen auf den Boden. Das Gelände war ungünstig. Sie hatten zwar das höhere Terrain inne, doch lagen ohne wesentliche Deckung da und die Shutek mussten sich bereits eingeigelt haben, denn aus den Mündungsfeuern ihrer Strahlengewehre ging kaum eine Bewegung hervor.

\par

Schwarzer Bär brachte sein Gewehr in Anschlag und gab einige Schüsse ab. Der Knall seiner Waffe betäubte geradezu seine Ohre. Hätte er keine Hörgeräte, die nun anfingen, schädliche Geräusche herauszufiltern, würden seine Ohren nun so klingeln, wie der Nachklang deines Feuerstoßes.

\par

Der Qualm und auch der Dampf des erhitzen Schnees um den Lauf seiner Waffe herum, nahmen ihm die Sicht, gaben ihm aber auch die Zeit, kurz den Zustand der Division zu überprüfen, zu der er gehörte. Die Vitalwerte seiner Kameraden wurden ihm auf seinem Monokel eingeblendet. Mit Schrecken stellte er fest, dass bereits gut ein Drittel rot hinterlegt oder sogar ausgegraut waren. Nicht nur seine Kommandantin Rana Sali, sondern auch der Soldat, den er gerade noch neben sich her gezogen hatten, waren als tot markiert.

\par

Er brauchte sich nur kurz umzusehen, um dessen leblosen Körper neben sich im Schnee liegen zu sehen.

\par

Erschrocken fuhr er zusammen, als sich Fischauge zu seiner Linken auf den Boden fallen ließ. In der ganzen Zeit, in der er Kacper Piecek nun schon kannte, hatte er ihn noch nie so entschlossen gesehen. Dass es gerade dieser Augenblick sein würde, in der Fischauge seinen Mann stand, überraschte ihn.

\par

Routiniert zog er sein Zielfernrohr ans Auge und richtete die Waffe auf die feindliche Schützenreihe. Sofort zuckte er von seinem Visier zurück. \WR{Meine Güte, die sehen echt grusleig aus!}, brachte er nur hervor, ohne Schwarzer Bär anzusehen. Dann legte er die Waffe wieder an und Schoss. Die modernen Schalldämpfer seines Gewehrs schafften es eher schlecht als Recht, den lauten Knall abzumildern.

\par

Auch Schwarzer Bär eröffnete wieder das Feuer, genauso wie der verbleibende Rest der Division. Er zielte nicht, dafür streute sein Gewehr ohnehin zu sehr. Das Rattern der Waffen übertönte beinahe das Dröhnen des Transporters, als dieser abhob.

\par

Schwarzer Bär sah nur kurz hinter sich. Die Maximus elf erhob sich in die Luft und richtete ihre Scheinwerfer auf den nahen Waldrand. Einige kleine rote Leuchte in den Raketenrampen des Transporters ließen drauf schließen, dass diese gerade scharf gemacht wurden. Die Shutek gaben ein gutes Ziel ab. Ihr Mündungsfeuer war so leicht zu sehen, dass selbst ein halbwegs blinder Starforce-Pilot sie sehen musste.

\par

Doch dann traf die Maximus elf ihre eigene Medizin. Irgendwo vom Fuß des nahen Berges aus musste eine schwere Rakete abgefeuert worden sein. Die Sonne verschwand gerade hinter einer Felswand, so war es nicht einfach, den ursprung zu erkenne. Die Wirkung zeigte sich jedoch eindrücklich. Das Geschoss zerriss die Front des Landetransporters beinahe augenblicklich. Der Rest stürzte sofort brennend zu Boden, ließ den Untergrund erbeben und warf eine Welle aus Schnee über die Soldaten, die wenigen Meter weiter vorne am Boden lagen.

\par

Der Lärm des Aufpralls und das kreischen des sich verbiegenden Metalls war ohrenbetäubend, selbst mit modernen Kampf-Hörgeräten.

\par

\WR{Scheiße!}, rief Fischauge. \WR{Wo zur Hölle kam das her.}

\par

Schwarzer Bär kniff die Augen zusammen. Dann erkannte er den Übeltäter. Ein schwarz bemalter Kampfläufer ging gerade am Fuß des Berges entlang. Sein Mittelteil ähnelte dem Leib einer Spinne genauso, wie die acht Beinen auf denen er stand. Am Bauch des Vehikels hing etwas, das für alle sofort wie ein großes Geschütz wirkte und auf dem Rücken schien der Schreiter eine Raketenlafette befestigt zu haben.

\par

\WR{Das müssen diese Megaschreiter sein.} Fischauge musste schreien, um nach wie vor gehört zu werden. \WR{Für mich sehen die eher wie Spinnen aus.}

\par

Schwarzer Bär widersprach sofort. \WR{Vergiss es. Ich mag Spinnen.}

\par

Dann kam Bewegung auf das Schlachtfeld. Vielleicht angespornt durch den Verlust des Landetransporters stürmten die Infanteristen der Shutek den Abhang hinauf. Er war ohnehin recht flach und selbst die hohe Schneedecke schien sie nicht wirklich aufzuhalten.

\par

Niemand verbrachte viel Zeit damit, sich die Shutek genauer anzusehen. Noch halbwegs gezielte Salven auf die heranstürmenden Gegner abzugeben, war das einzige, woran nicht nur Schwarzer Bär nun dachte.

\par

Doch mit einem mal stand einer der Feinde direkt vor dem Überrest der Division. Und es war nicht die fast einen halben Meter lange Klinge an seinem Gewehr, die mehr nach einer Sense aussah, die Schwarzer Bärs Bick einfing. Auch nicht der filigrane Körper, der von einem schimmernden Panzer umschlossen war. Es war das Gesicht.

\par

Schwarzer Bär war sich nicht sicher, worin er da gerade sah. Ein Schädel. Das Gesicht eines Toten. Ein scheinbar nutzloser, weit offenstehender Mund. Keine Nase, Ohren oder Haare. Stattdessen hole, rot leuchtende Augen.

\par

Der Shutek schien auch kein Fleisch an sich zu tragen. Sein ganze Körper, einschließlich der dürren Beine, die ihn dennoch schnell tragen konnten, schien von einem Panzer umgeben und auch sein Gesicht, machte eher den Eindruck einer Maske.

\par

Dann richtete er sein riesenhaftes Gewehr auf Schwarzer Bär aus. Die beiden starrten sich an. Auf der einen Seite blankes Entsetzen und auf der anderen eine regungslose Fratze.

\par

Die mit einem Mal in tausend Teile zersprang. Schwarzer Bärs Blick fuhr nach links, wo nach wie vor Kacper Piecek im Schnee lag, sein Schwarfschützengewehr auf das den Körper des Shutek gerichtet, der langsam in sich zusammensackte. Grünlich blauer Rauch stieg auf, wo gerade noch der Kopf der Gestalt gewesen war.

\par

Schwarzer Bär fasste sich ein Herz. Er war kein Taktiker, doch wusste genau, dass die Feinde sie nun einfach überrennen würden, wenn sie nicht schnell etwas unternahmen. Schwerfällig ging er in die Hocke und gab einige Schüsse auf die heranrennenden Skeletthorden ab.

\par

\WR{Macht die Wichser platt!}, kreischte er und stürmte nun seinerseits auf die Feinde zu. Einige seiner Kameraden taten es ihm gleich und rückten laut schreiend vor. Schnell fielen die ersten Männer auf beiden Seiten.
