Eine halbe Stunde später fanden sich insgesamt einhundertzehn Personen im Einsatzbesprechungsraum der \EN{Regenvogel} zusammen. Der Raum selbst erinnerte an einen Hörsaal an einer Universität. Drei Sitzblöcke mit jeweils vier Stuhlreihen liefen nach unten hin auf den Präsentationsbereich zu, der mit einem großen Bildschirm und einem Rednerpult ausgerüstet war. Eine metallene Version des Emblems der Starforce zierte die Front des Pults und die Flagge der Union wurde auf dem Monitor angezeigt, so lange er für nichts anderes gebraucht wurde.

\par

Die Beleuchtung im Besprechungsraum war gedämpft, lies Kevin Wilson aber trotzdem erkennen, wie ihm Dexter Henningtons Freunde mit versteinerten Minen ansahen. Morten und sein Freund hatten sich, zusammen mit Jens Wörg und Kenji Batosay recht weit oben zu den anderen Piloten des Adleraugen Geschwaders gesetzt. Insgesamt waren alle zwanzig Piloten der Staffel anwesend, abgesehen von der Geschwaderkommandantin, die zusammen mit der Brückenbesatzung und den anderen Staffelführern den Einsatz plante. Noch war niemand vom Kommandostab zugegen. Ein leises Raunen ging durch die Reihen. Immer wieder wurde über den kommenden Einsatz spekuliert und darüber was das Schiff im Arktur System erwarten würde.

\par

Morten fühlte sich noch nicht ganz wohl in dieser neuen Umgebung. Für alle anderen, sogar für Kevin, schien das Prozedere völlig selbstverständlich zu sein. Aber obwohl Morten den Ablauf einer Einsatzbesprechung bestens kannte, machte ihn diese doch nervös. Besonders da die ganze Mission unter realen Bedingungen stattfand, was in der Starforce nur noch selten vorkam, nachdem die Capital Fellowship geschlagen worden waren.

\par

Gerade wollte er Kevin etwas fragen, um seine Nervosität durch Gespräche zu verdrängen, da hörte er einen Gesprächsfetzen aus der vorderen Reihe, der in stutzen ließ.

\par

\WR{Emilia, ich hab heute was interessantes mitgehört}, sprach einer der Piloten im Flüsterton, der sich Morten beim Hereinkommen als Pino Rodriguez vorgestellt hatte. \WR{Lieutenant Wallander von der Brücke hat mir in der Bar erzählt, er hätte auf Befehl des Kapitäns, alle Daten über das Omega dutzendeins Protokoll herunterladen sollen.}

\par

\WR{Echt}, gab die Gefragte überrascht von sich.

\par

Pino Rodriguez nickte bedeutungsvoll und fügte hinzu: \WR{Es sagte, es sein nur für alle Fälle. Aber wer weiß, was da alles auf uns zukommt.}

\par

Morten machte große Augen. Er glaubte, irgendwann einmal etwas über ein Omega dutzendeins Protokoll aufgeschnappt zu haben, konnte sich aber beim besten Willen nicht erinnern, worum es dabei gegangen war.

\par

Er wollte gerade Kevin oder einen der anderen fragen, was es bedeutete, da öffnete sich unten, neben dem Präsentationsbereich eine Tür und der Kommandostab kam nacheinander in den Besprechungsraum. Morten erkannte die meisten nicht aber wer unter den Anwesenden Captain Fiscale war, konnte er sich denken. Der erste Offizier überragte die Kapitänin um fast zwei Köpfe, doch ihre Kurzhaarfrisur und sein markanter Schnauzer strahlten beide auf ihre Weise eine gewisse Autorität aus.

\par

Vielleicht lag es nur daran, dass mit ihm und Kevin darüber gesprochen hatte. Doch irgendwie glaubte er, es den beiden ansehen zu können, dass sie bereits einige Runde gegen die Capital Fellowship im Ring hinter sich gebracht hatten.

\par

Viele schienen leicht überrascht, dass die Kommandantin die Einsatzbesprechung selbst leiten würde. Captain Fiscale trat bereits ans Rednerpult, unterhielt sich aber noch mit den Geschwaderkommandanten. Einen der drei kannte Morten bereits. Dexter Hennington stand breitschultrig, mit beiden Händen in den Taschen seiner Uniformhose, im Präsentationsbereich und grinste Kevin spöttisch entgegen. Seine Schildmütze trug er augenscheinlich nicht sehr gerne, so halbherzig wie sie auf seinem Kopf saß. Morten verwunderte das angesichts der Aufwendigkeit seiner Frisur, die gesehen werden sollte, nur wenig.

\par

Der zweite Geschwaderkommandant, der das Sternenflügel-Geschwader führte, wirkte ruhig und gelassen. Seine langen blonden Haare hingen ihm teilweise auf die Schulter. Das militärische Protokoll wurde auf der \EN{Regenvogel} offensichtlich sehr lax gehandhabt, was allerdings eher für einen Träger mit einer eingespielten Crew als für einen Sauhaufen sprach. Morten hatte von Kenji erfahren, dass der Mann Simon Maddeux hieß und außer der Leitung über die Sternenflügel auch noch die Position des Wing Commander innehatte. Er war also der Mann, der den Oberbefehl über alle Geschwader des Trägers besaß. In dieser Eigenschaft konnte er bei Einsätzen auch in jeder Staffel seiner Wahl mitfliegen. Morten träumte seit langem davon, später ebenfalls einmal eine solche Position einzunehmen.

\par

Mortens Blick blieb sofort an der dritten Geschwaderkommandantin hängen. Hätte man ein Poster von ihr mit Fliegerbrille und verschmitztem Grinsen als Rekrutierungswerbung verwendet, hätten sich die Neuzugänge an der Akademie sofort verdreifacht. Die Dame stand den Zuhörern breitbeinig mit den Händen hinter dem Rücken verschränkt gegenüber. Vermutlich nicht beabsichtigt, unterstrich ihre vorschriftsmäßige Haltung ihre attraktive Figur. Ihre dunklen Haare, welche dieselbe Farbe wie ihre Augen hatten, hatte sie zu einem einfachen Zopf zusammengebunden.

\par

Ihre Uniform, die taillierte Variante für Frauen, ließ ihre Kurven besser aussehen als sie es in manchem Ballkleid getan hätten. In diesem Moment begann Morten wirklich inständig zu hoffen zu den Adleraugen versetzt zu werden. Der Staffel, die Anna Farley kommandierte.

\par

Ehe Mortens Frühlingsgefühle mit ihm durchgingen, begann Captain Fiscale mit der Einsatzbesprechung: \WR{Guten Abend, meine Herren und Damen. Wir haben leider wenig Zeit, daher komme ich sofort zum Wesentlichen.}

\par

Hinter Fiscale Fiscale formierte sich eine taktische Darstellung das Remotus-Sektors, dem Raumgebiet indem sowohl Mortens alte Flugschule wie auch Pollux und Arktur lagen. Der Computer vergrößerte den Stern Arktur so lange, bis eine strategische Karte des Systems erschien. Der Stern gehörte zur Kategorie grün und hatte mehrere, unbewohnte Planeten auf eine Umlaufbahn um sich gebunden. Auch ein Asteroidengürtel umkreiste die Sonne in einigem Abstand.

\par

\WR{Bitte beachten Sie, dass diese Bilder nicht bestätigt sind}, bat Captain Fiscale, als sie erkannte, dass die meisten Piloten die Anzeige intensiv taxierten. \WR{Die Pinnacle Science Group hat im Auftrag der Regierung vor kurzem ein Schiff nach Arktur geschickt, um das System zu erkunden. Und genau um dieses Schiff geht es. Commander Samad, bitte geben Sie uns einen Einblick.}

\par

Der erste Offizier nickte hastig und trat ans Rednerpult. Er wirkte wie jemand, der dem Anblick einer großen Menge war er schon mehr als einmal gegenübergestanden hatte. Nach einem kurzen Räuspern begann er zu sprechen: \WR{Unser Zielobjekt heißt \EN{Virial}. Es ist ein wissenschaftliches Raumschiff der \EN{Ereignishorizont}-Klasse und gehört, wie Captain Fiscale bereits sagte, zur Pinnacle Science Group.}

\par

Auf dem Bildschirm wurden nun technische Daten über den Typus des Schiffes, und speziell die der \EN{Virial} angezeigt. Der Kreuzer war beinahe so lang wie ein mittlerer Träger, überragte also die \EN{Regenvogel} um ein paar hundert Meter. Das Schiff war allerdings sehr filigran und umschloss somit wenig Innenraum. Aber die gesamte Hülle war von allen denkbaren Messgeräten und Sensoren überzogen. Obwohl es wahrscheinlich weniger lange gebaut hatte werden müssen, war es sicherlich mindestens genauso teuer wie die \EN{Regenvogel}.

\par

\WR{Vor achtundvierzig Stunden hat die PSG zum letzten Mal Meldung von der \EN{Virial} erhalten}, fuhr Commander Samad fort. \WR{Das ist sehr ungewöhnlich, besonders, da sich in diesem System hier eine große Kommunikationsbasis auf der Oberfläche von Pollux Primus befindet. Die \EN{Virial} sollte also keine Probleme haben, Kontakt herzustellen, denn selbst wenn ihre Langstreckenkommunikation ausgefallen wäre, hat bestimmt jemand an Bord einen eigenen Nullzonentranciever.}

\par

Der Commander dachte kurz nach, wie er den nächsten Satz formulieren sollte. \WR{Wir versuchen schon seit einigen Stunden, die \EN{Virial} zu erreichen aber bislang ohne Erfolg. Das bedeutet, dass wir mit Plan B fortfahren. Wir haben die Hyperraumroute nach Arktur demnächst erreicht. Wir springen in das System und suchen dann selbst nach der \EN{Virial}. Da unser Schiff Tage bräuchte, um das ganze System nur grob mit den Abtastern zu durchsuchen, setzen wir mehrere Jägerstaffeln ein, um die Suche zu beschleunigen.}

\par

\WR{Das ist klar}, sagte Kevin leise zu Morten. \WR{Ich meine, sonst hätten zur Aufklärung die beiden Korvetten gereicht, die uns begleiten.}

\par

Commander Samad trat vom Rednerpult und wie auf Collonel Simon Maddeux. Der Wing Commander der \EN{Regenvogel} trat ans Mikrofon heran, während sich auf dem Bildschirm hinter ihm ein taktischer Plan des Arktur Systems formte. \WR{Guten Abend, Jungs und Mädels}, begann der Collonel. \WR{Wie der Commander bereits sagte, werden unsere drei Geschwader die Hauptlast bei dieser Mission tragen. Aber bevor ich mit der Erläuterung beginne, will ich noch etwas organisatorisches klären. Seit heute haben wir einige neue Piloten an Bord.}

\par

Der Collonel sah auf den, im Rednerpult integrierten, Monitor und rief dann auf: \WR{Second Lieutenant Kevin Wilson?}

\par

\WR{Hier}, gab Mortens Mitbewohner laut von sich und streckte die Hand in die Höhe.

\par

\WR{Second Lieutenant Morten Wittwer?}

\par

Der Aufgerufene erhob sich gemäß dem Protokoll, das man ihm in der Flugschule beigebracht hatte. Er spürte deutlich, wie die Blicke einiger im Raum auf ihn gerichtet waren, was nicht dazu führte, dass er sich wohler fühlte. Es beruhigte ihn allerdings ein wenig, dass es die anderen Neuzugänge ihm nachtaten, statt wie Kevin laut zu schreien.

\par

\WR{Sehr schön}, kommentierte Collonel Maddeux. \WR{Ich wünschte, ich hätte mehr Zeit gehabt, ihre Lebensläufe zu lesen aber hier geht im Moment alles ein bisschen drunter und drüber. Sie beide habe ich dem Adleraugen Geschwader zugeteilt~-- da können Sie am wenigsten falsch machen.}

\par

Einige der Anwesenden begannen bedeckt zu lachen. Simon Maddeux grinste Kevin und Morten entgegen um seiner Bemerkung die Schärfe zu nehmen.

\par

\WR{Ihre Geschwaderkommandantin ist, wie Sie eventuell schon wissen, Lieutenant Commander Anna Farley. Sie werden sie noch schätzen lernen.}

\par

In diesem Punkt stimmte Morten sofort still mit dem Wing Commander überein. Er war froh, in diesem Geschwader gelandet zu sein. Wenn Kringel und Kenji der Maßstab für die Piloten in der Einheit waren, dann wäre er sicherlich in guter Gesellschaft. Besonders aber wegen Anna Farley, von der Morten nur schwer die Augen lassen konnte, was ihn selbst überraschte. Er hätte sich mehr Disziplin zugetraut, insbesondere während eines echten Einsatzes.

\par

Collonel Maddeux fuhr mit seiner Erklärung fort: \WR{Kommen wir zum Wesentlichen. Für die, die es noch nicht erkannt haben, hinter mir seht Ihr die Übersichtskarte von Arktur.} Wieder ging ein verhaltenes Lachen durch die Reihe der Piloten. \WR{Die \EN{Regenvogel} wird sich nach dem Sprung hier befinden}, kündigte Collonel Maddeux an und ließ den Computer eine vereinfachte Darstellung des Trägers in der Nähe des Hyperraumknotens anzeigen. \WR{Da das Einsatzziel lautet, den verlorenen Raumer möglichst schnell zu finden, werden wir gleich nach unserer Ankunft mit Aufklärungsflügen beginnen. Wir werden als erstes die Positionen absuchen, von denen wir glauben, dass sie für die Eierköpfe auf diesem Forschungsschiff besonders interessant sein können. Dabei werden wir die Arbeit einigermaßen gleichmäßig unter den drei Geschwadern an Bord aufteilen.}

\par

Auf dem Bildschirm erschienen nun einige Symbole, die Jägerstaffeln darstellen sollten. Außerdem wurden Zielgebiete markiert, die als möglicher Aufenthaltsort galten.

\par

\WR{Mein Geschwader übernimmt die Gebiete sechs, sieben und acht, die größtenteils an Backbord der \EN{Regenvogel} liegen werden. Flügel Rot überprüft eine Position, von der man einen Mond genauer beobachten könnte. Flügel Blau checkt die Umlaufbahn dieses Planeten.}

\par

Collonel Maddeux wieß mit seiner Hand auf den großen Monitor hinter sich.

\par

\WR{Staffel Weiß wird sich den Raum dazwischen vorknöpfen. Wir fliegen exakt unsere Wegpunkte ab, es seihe denn, wir finden irgendetwas, das auf das vermistte Schiff hindeuten könnte. Jeder Flügel besteht aus vier Jägern. Die Pilotenzuteilung können Sie im Missionsdossier nachlesen. Und ich will kein Gemecker darüber, wer vielleicht nicht mit wem fliegen will. Wir bilden ein Team und genau das werden wir da draußen auch sein.}

\par

Der Collonel holte kurz Luft. \WR{Da die Ziele sehr nahe liegen werden, geben wir uns mit der Geschwindigkeit unserer Raumüberlegenheitsjäger der Falken-Klasse zufrieden. Irgendwelche Fragen?}

\par

Ein Großteil der Piloten aus Collonel Maddeux Geschwader schüttelte lässig den Kopf. Andere sahen angestrengt auf den Bildschirm und prägten sich die Flugroute ein. Morten würde dasselbe tun, falls er schon an seinem ersten Tag in einer Staffel mitfliegen sollte, denn obwohl der Missionsplan jederzeit abrufbar war, lohnte es sich immer, die Details fast auswendig zu kennen. Allerdings glaubte er kaum, dass er allzu schnell tatsächlich bei einem Einsatz dabei sein würde. Besonders nicht bei einem unter realen Bedingungen.

\par

Collonel Maddeux hängte noch etwas an, wobei er mit deutlich mehr Nachdruck sprach: \WR{Vergesst da draußen bitte eines niemals. Das Ganze mag zwar wie eine normale Such- und Bergungsoperation aussahen aber man weiß trotzdem nie genau, was einem erwartet. Was vor uns liegt ist ein realer Auftrag. Das bedeutet, wir schießen auch mit realen Waffen. Also ballert euch nicht gegenseitig den Arsch weg, bloß weil ihr denkt, ihr habt ein kleines, grünes Männchen gesehen, das euch den Mittelfinger zeigt.

\par

Die Union zählt auf uns. Und ganz besonders die Forscher auf diesem Schiff. Wir werden daher nicht weiter spekulieren, was mit dem Schiff sein könnte und was nicht. Wir finden es einfach, denn dafür haben wir uns verpflichtet.}

\par

Einige von Simon Maddeux Piloten begannen mit den Fingerknochen auf die Lehnen ihrer Sessel zu pochen, während dieser vom Rednerpult trat und Dexter Hennington ans Mikrofon ließ. Dieser blickte zunächst mit finsterer Mine in die Menge und begann dann mit ernstem, strengen Tonfall zu erklären: \WR{Die Gebiete eins zwei und drei vertraut die Kommandoebene uns an. Das wird eine sehr ernste Sache, Jungs und wir werden uns da draußen richtig anstrengen um unserem Ruf gerecht zu werden.}

\par

Dexters Geschwaderkameraden nickten einheitlich. Morten und Kevin zogen nur skeptisch die Augenbrauen hoch. Einige der anderen Anwesenden schienen ähnlich zu empfinden.

\par

\WR{Die Zielorte liegen zwischen der \EN{Regenvogel} und einem möglicherweise bewohnbaren Planeten. Gold übernimmt Gebiet eins, Grau Gebiet zwei und Schwarz Gebiet drei. Wie die Staffeln des Sternenflügel-Geschwaders fliegen auch wir die Falken und zwar in Vierergruppen mit folgender Zuteilung.}

\par

Major Hennington begann aufzulisten, wer wessen Flügelmann sein würde. Bei drei Staffeln mit je vier Jägern brauchte er eine Zeit lang. Morten vermutete, er zählte die Einteilung nur dazu auf, um länger zu reden als Collonel Maddeux.

\par

Dann sah Dexter eindringlich zu seinen Piloten und versuchte dabei streng und gebieterisch auszusehen. Aber auf Morten wirkte es eher, als wäre er ein Neandertaler, der angestrengt nach seiner Keule Ausschau hielt.

\par

\WR{Vergesst nicht, Jungs}, sprach Dexter überzeugt. \WR{Wir sind die Eisenhämmer. Wir haben einen Ruf zu verlieren, also strengt euch da draußen an. Ich erwarte nicht weniger als euer Bestes.}

\par

Mortens Nebensitzer, ein Pilot den er bisher nicht kannte, lachte leise auf, lehnte sich zu Morten und flüsterte ihm zu: \WR{Lass dich nicht beeindrucken. Das sind bloß hohle Phrasen. Die fliegen auch nicht besser als wir.}

\par

Während dessen stolzierte Dexter Hennington vom Rednerpult und überließ es Anna Farley. Innerlich stieß Morten einen langen Seufzer aus, als die Pilotin ins Rampenlicht des sonst eher spärlich beleuchteten Raumes trat. Er geriet sogar ein wenig ins Stocken, als Sie zu ihm und Kevin sah und sagte: \WR{Mister Wittwer, Mister Wilson, willkommen bei den Adleraugen. Ich bin sicher, Sie werden sich bei uns bald sehr wohl fühlen.}

\par

Morten musste sich beherrschen, um nicht wie ein kleines Kind eifrig zu nicken. Einige der Piloten des Geschwaders drehten sich kurz zu ihm und Kevin und nickten anerkennend zu als Zeichen des Willkommenheißens.

\par

Anna Farley fuhr fort: \WR{Kommen wir jetzt zu unserer Aufgabe, meine Herren und Damen. Das Kommando hat beschlossen, dass wir bei diesem Einsatz eine ergänzende Rolle spielen werden. Wir werden drei Gebiete inspizieren, die sich an Steuerbord der \EN{Regenvogel} befinden werden, sobald wir das Arktur System erreicht haben werden.}

\par

Der Wandbildschirm hinter der Geschwaderkommandantin vergrößerte einen Abschnitt, der weiter entfernt lag, als die beiden, zuvor gezeigten. \WR{Wir halten es eher für unwahrscheinlich, dass sich die \EN{Virial} in diese Gebiete begeben hat}, spekulierte Lieutenant Commander Farley vorsichtig. \WR{Dennoch sollten diese Gebiete aus Symmetriegründen ebenfalls untersucht werden. Es handelt sich um ein Asteroidenfeld, bei Zielort neun. Einer Stelle, die günstig zur Beobachtung der Sonne ist, bei Nummer vier. Und dem Raum dazwischen, das wäre dann Zielort fünf.}

\par

Morten sah angestrengt auf den Monitor. Er prägte sich die Route und die ungefähren Koordinaten so gut ein, wie es ihm möglich war.

\par

\WR{Da es, wie gesagt, unwahrscheinlich ist, dass sich die \EN{Virial} dort befindet, werden die Staffeln Grün, Orange und Gelb in Zweiergruppen unterwegs sein. Und zwar in folgender Besetzung:}

\par

Anna Farley sah auf die, ins Rednerpult integrierte Konsole. \WR{Staffel Gelb wird von Lieutenant Master und second Lieutenant Gérard geflogen. Sie beide übernehmen Zielgebiet vier.}

\par

\WR{Ja, Madam}, erklang die Stimme eines Piloten, der einige Reihen vor Morten saß.

\par

Die Geschwaderkommandantin wirkte so, als würde sie noch immer mit einer Entscheidung hadern, als sie weiter vorlas. \WR{Ich habe beschlossen, dass Staffel Orange aus Lieutenant Batosay und second Lieutenant Wilson bestehen und Zielgebiet fünf untersuchen wird.}

\par

Sofort ging ein Raunen durch die Reihen der Piloten. Einige schwer zu deutende Blicke gingen in Kevins Richtung. Dieser ballte die Hände zu Fäusten und strahlte wie ein kleiner Junge vor dem Weihnachtsbaum.

\par

Leise sagte er zu Morten: \WR{Ich fass es nicht. Ich hab’s geschafft!}

\par

Kenji berührte Kevin an der Schulter und bedeutet ihm so, wieder zuzuhören. Lieutenant Commander Farley fuhr mit der Besprechung fort: \WR{Es ist ungewöhnlich, Anfänger sofort bei realen Missionen einzusetzen. Ich denke aber, es tut Ihnen sehr gut, wenn Sie gleich erfahren, wie es da draußen zugeht. Also bleiben Sie am Boden~-- im übertragenden Sinne natürlich!}

\par

Die meisten Piloten der Adleraugen, Morten eingeschlossen, lachten kurz auf. Kevin ließ sich in seiner Freude aber nicht bremsen. Dass er wirklich aufgeregt und begeistert war, konnte man ihm leicht ansehen.

\par

Erst jetzt bemerkte Morten, wie seltsam einige Piloten aus Dexters Geschwader ihn und Kevin ansahen. Major Hennington selbst, blickte seine Kollegin schief aus den Augenwinkeln an.

\par

\WR{Selbiges gilt für Sie Mister Wittwer}, sprach Anna Farley in Mortens Richtung.

\par

Dieser fuhr zusammen. Sollte er etwa auch mitfliegen? Er fand es nachvollziehbar, dass Kevin einbezogen wurde. Seine Noten waren zwar weniger überzeugend als Mortens, doch er hatte sich bei einigen simulierten Einsätzen als sehr talentiert erwiesen und sich einen guten Ruf erarbeitet, wie er nicht müde wurde, zu berichten. Morten hingegen hatte unter seinen Lehrern als relativ phantasielos gegolten und ihm war fehlende Initiative vorgehalten worden.

\par

\WR{Sie und ich bilden Flügel grün}, erklärte Lieutenant Commander Farley, wobei Morten das Herz in die Hose rutschte. Nun sahen ihn auch die Piloten der Eagle zumindest fragend an. \WR{Wir werden uns das neunte und letzte Zielgebiet vorknöpfen. Ich wüsste zwar nicht, was die \EN{Virial} bei einem Asteroidenfeld verloren haben könnte aber man kann nie wissen.}

\par

Morten hatte Mühe, der Geschwaderkommandantin zuzuhören. Er war, gelinde gesagt, überwältigt. Er würde tatsächlich einen realen, spannenden Einsatz fliegen. Auch wenn dabei wahrscheinlich nicht das geringste passieren würde. Es war ein echte Herausforderung. Und dann zu allem würde er auch noch an der Seite von Major Anna Farley mitfliegen. Der vermutlich interessantesten Pilotin des ganzen Schiffs.

\par

Diese war bereits mit der Einsatzbesprechung fortgefahren. \WR{Da unsere Zielgebiete weiter entfernt liegen als die der anderen Staffeln, werden wir nicht mit Jägern der Falken-Klasse, sondern mit Aufklärern fliegen. Ihre Hyperraumschlitten werden uns bei unserer Aufgabe genauso helfen wie ihre hervorragenden Sensoren und Abtaster. Aber ich brauche bestimmt niemandem zu sagen, dass Aufklärers der Argus-Klasse extrem schwer zu kontrollierende Fluggeräte sind. Sie sind sehr schwach gepanzert und bewaffnet und haben kein eigenes Schwerkraftmodul. Also gehen Sie vorsichtig mit ihnen um und bauen Sie keine Bruchlandung. Das gilt besonders für Sie beide, Mister Wittwer und Mister Wilson.}

\par

Morten bemerkte, wie sein Nebenmann lachte und ihm freundschaftlich auf die Schulter klopfte. Kevin schien sich durch diese Bemerkung, wie so oft, in seiner Ehre gekränkt zu fühlen und schüttelte schmollend den Kopf.

\par

Major Farley verließ das Rednerpult und übergab das Wort wieder an Captain Fiscale. Die markige Frau trat ans Mikrofon und sprach in ernstem Ton: \WR{Die Kampfpatrouille wird wie immer von Staffel Purpur in üblicher Besetzung geflogen. Staffel Violett bleibt in Bereitschaft, ebenfalls in üblicher Zusammenstellung.} Die Kapitänin unterbrach sich. Ihr war anzusehen, dass sie nach den richtigen Worten rang. \WR{Lassen Sie mich noch eines sagen. Sie haben es zwar schon zur Genüge gehört aber wir befinden uns hier auf einem echten Einsatz. Unter Gefechtsbedingungen. Man erwartet viel von uns und es gibt eine ganze Menge Arbeit. Aber wir müssen das als Chance sehen. Wir haben glücklicherweise seit achtzig Jahren keinen Krieg mehr führen müssen und die Capital Fellowship ist schon ein ganzes Jahrdutzend lang zerschlagen. Gelegenheiten sich zu beweisen sind selten geworden. Unsere nächste Mission ist so eine Chance. Also lassen Sie uns unsere Arbeit gut machen, damit jeder weiß, auf die \EN{Regenvogel} ist Verlass. Wegtreten!}

\par

Sämtliche Anwesenden erhoben sich und begannen zu klatschen. Morten und Kevin folgten enthusiastisch diesem Beispiel. Zwar gehörte so ein Vorgehen nicht zum Protokoll aber es schien ein Ritual zu sein, dem sich die beiden natürlich nicht verwehren wollten.

\par

Während die anderen bereits den Raum verließen, blieben Morten und Kevin noch zurück. Das Gedränge auf die beiden einzigen Eingänge war sowieso zu groß um sofort hindurch zu kommen.

\par

\WR{Am besten ihr legt euch noch ein bisschen hin}, riet ihnen Kringel bevor er sich daran machte, sich durch die Massen zu kämpfen.

\par

Kenji folgte Jens und sagte hastig: \WR{In dem Packet, dass ihr vom Quartiermeister bekommen habt sind Schlafpillen, falls ihr nicht einschlafen könnt. Am besten ihr schlaft bis zum Sprung durch, denn danach geht’s sofort los.}

\par

Morten nickte geistesabwesend. In Gedanken saß er schon im Cockpit und flog mit Anna Farley durch die Gegend. Es half ihm nichts, er musste sich eingestehen, dass er sich wohl ein wenig verschossen hatte. So war es ihm schon lange nicht mehr gegangen. Wirklich Hals über Kopf verliebt hatte er sich zum letzten Mal lange vor der Flugschule auf Corna Primus. In der Mittelschule hatte er seine Augen zwei volle Jahre nicht von seiner Sitznachbarin lassen können. Seine Noten hatten darunter enorm gelitten. Eine derartige Ablenkung konnte er sich nun nicht mehr erlauben.

\par

\WR{Sag mal, was ist eigentlich mit dir los?}, fragte Kevin, verwundert über Mortens verträumtes Gesicht. \WR{Du siehst aus, als hättest du... ich meine, als würdest du... Ach, keine Ahnung. Irgendwie seltsam.}

\par

Morten reagierte gar nicht auf Kevins Frage sondern sah in die Menschenmenge, die sich auf die Tür zu bewegte. Seine Blicke suchten und fanden Anna Farley, die mit ihrer schlanken Erscheinung keine Probleme hatte, sich durchzuschlängeln.

\par

Kevin bemerkte langsam aber sicher, was mit seinem Stubengenossen los war. Sein Gesichtsausdruck rangierte irgendwo zwischen Spott und tiefen Verstädnis.

\par

\WR{Du treibst es manchmal echt auf die Palme}, kommentierte Kevin schließlich. \WR{Wir sind gerade mal ein paar Stunden hier, morgen in aller Frühe starten wir auf unsere erste richtige Mission und du wendest deinen Hals schon nach unserer Geschwaderkommandantin.}

\par

Morten seufzte bloß. Er hätte gerne etwas schlagfertiges geantwortet, um Kevin über den Mund zu fahren aber ihm wollte einfach nichts einfallen. Schließlich meinte dieser: \WR{Kaum zu fassen, dass man mich immer als Lebemensch bezeichnet.}
