\WR{Rumpf bricht! Starten Rettungskapseln}, sprach der Kommunikationsoffizier der \EN{Crossguard} durch die Lautsprecher Brücke der \EN{Regenvogel}. Das Schiff hatte gerade eben einen schweren Torpedotreffer abbekommen und es war bereits aus großer Entfernung zu erkennen, dass es mit dem Schlachtkreuzer zu Ende ging. Die Übertragung lief weiter, auch wenn der Mensch an der Gegenstelle längst seinen Posten geräumt hatte.

\par

\WR{Alle Mann von Bord!}, warf im Hintergrund zu hören. \WR{Raus hier. Alle!}

\par

Maas Petrarca sah auf seine Anzeigen uns sagte dann in sein Headset: \WR{Staffel grau: Neues Ziel. Geben Sie den Rettungskapseln der \EN{Crossguard} nach Möglichkeit Deckung.}

\par

Der Feind rückte unerbittlich näher. Noch sammelte sich ein Großteil der Shutek-Flotte beim näheren Mond des Planeten. Doch mittlerweile war mindestens ein Träger identifiziert worden, der unentwegt Jäger- und Bomberwellen in den Orbit schickte.

\par

\WR{Die machen wirklich Druck!}, sagte Lieutenant Wallander gedankenverloren.

\par

Captain Fiscale stimmte ihm wortlos zu. Nicht erst, als sie mit ansehen musste, wie die \EN{Crossguard} in einem hellen Blitz verschwand. Das Kollabieren ihres Perpetuum Mobiles schien für kurze Zeit den Raum um das Schiff herum zusammenzuziehen. Als der Spuk vorbei war, blieben nur noch wenige glühende Trümmer zurück.

\par

\WR{Madam, vielleicht können wir gemeinsam mit der \EN{Artiglio} einen Gegenangriff koordinieren. Die feindliche Streitmachen abfangen, bevor sie die Umlaufbahn erreicht}, rief Lieutenant Commander Petrarca von unten zu seiner Kommandantin hinauf.

\par

\WR{Negativ}, war ihre prompte Antwort. \WR{Wir ziehen unsere Flotte so eng wie möglich zusammen und lassen sie kommen. Wenn Sie eintreffen, dann verwickeln wir sie so lange wie möglich in einen engen Nahkampf. Ein paar von denen glauben dann auch dran.}

\par

Der Steuermann sah sich die taktische Situation auf seiner eigenen Konsole an und gab dann zu bedenken: \WR{Madam, wie Sie schon selbst gesagt haben. Die Shutek könnten sich aufteilen. Dann erreichen wir einen Teil von ihnen bestimmt nicht mehr, bevor sie in Feuerreichweite des Planeten kommen.

\par

Fiscale ging schnellen Schrittes zur Kommunikationsstation und bedeutete einem überraschten Wallander mit einer Kopfbewegung, dass sie sich selbst setzen wollte. Dann breitete sie ihre Hände über die Konsole aus und öffnete einen Kanal an die Bodenkontrolle. Kurze Zeit darauf erschien Lega Gajjars mittlerweile völlig verschwitztes Gesicht auf dem Schirm. Sie trug nunmehr auch keine Galauniform mehr, sondern eine vollständige Rüstung.

\par

\WR{Captain}, grüßte sie heißer. \WR{Was kann ich für Sie tun.}

\par

\WR{Legatin, ich muss wissen, wie Sie die Lage einschätzen.}

\par

\WR{Wir können die Shutek nicht aufhalten}, gab Gajjar ohne zu zögern zurück. \WR{Die Shutek kommen mit alleim, was sie haben. Die Stadt wird überrannt. Wir haben vielleicht noch drei Stunden. Eher eine oder zwei. Können Sie Jäger für einen Luftangriff erübrigen? Die bodengestützten Geschwader fallen wie die Steine vom Himmel.}

\par

Natalia Fiscale fasste sich schwer atmend an die Stirn. Sie brachte nun kaum noch über die Lippen, was sie nun sagen wollte. \WR{Negativ, Legatin. Es tut mir leid, aber eine Flotte der Shutek macht sich jeden Moment auf den Weg hierher. Wir brauchen alles, was fliegen und schießen kann, hier oben, um sie aufzuhalten. Wenn es nur ein einziger Zerstörer in die Umlaufbahn schafft, dann haben Sie da unten gleich ein dutzend Atombomben auf dem Hals.}

\par

\WR{Und wenn die Silos fallen, oder am Ende noch erobert werden…} Fiscale sah genau im Gesicht ihrer Gesprächspartnerin, dass sie verstand, worauf sie hinaus wollte.

\par

\WR{Wie viele Zivilisten befinden sich noch in Yêxīn?}

\par

\WR{Nur noch, wen wir nicht gefunden haben}, erwiderte die Legatin. \WR{Die Stadt ist menschenleer.}

\par

\WR{Hören Sie…} Fiscales Stimme versagte. \WR{Unsere Zerstörer sind in einer möglichen Abwurfposition.} Die Köpfe gleich mehrerer Anwesenden fuhren schlagartig zur Kommandantin herum. \WR{Wenn wir jetzt ein Bombardement durchführen, dann könnten wir die Bodentruppen der Shutek im Süden aufhalten. Aber, die meisten von ihnen sind schon innerhalb der Stadt…}

\par

Legat Gajjar löste für einen Moment den Blickkontakt vom der Videokonferenz. Als sie Fiscale wieder in die Augen sah, waren ihre eigenen feucht. \WR{Ich verstehe, Captain. Zielen Sie gut. Ich fände es sehr schade, wenn es nur einer dieser Bastarde überleben würde.}

\par

Captain Fiscales Blick wurde hart. \WR{Werden sie nicht. Versprochen. \EN{Regenvogel}, Ende.}

\par

Kaum, dass der Bildschirm wieder seinen Standard-Hintergrund zeigte, sagte Lieutenant Wallander laut: \WR{Madam, dass könne wir nicht tun! Wir können nicht unsere eigenen Soldaten beschießen! Geschweige denn die Zivilisten, die noch in Yêxīn sind! Das wäre Mord. Ich werde…}

\par

\WR{Ihr Protest ist notiert!}, donnerte Captain Fiscale und stand wieder auf. Wieder suchte ihr Blick Commander Samad und fand schnell seinen zugedeckten Körper. \WR{Aber es gibt keine Alternative. Wenn die Shutek die Silos erreichen, dann können wir ihre Flotte nicht mehr abfangen und dann werden es ihre Atom- statt unserer Nullzonenbomben sein, die Kreuzpunkt Primus aus der Eiszeit brennen.} Der Kommunikationsoffizier starrte sie mit offenem Mund an. \WR{Geben Sie den Befehl an jeden Zerstörer im Orbit durch. Ziel ist Yêxīn und das Umfeld. Feuermuster: Null Null Null, Zerstörung, Null.}

\par

Nils Wallander schüttelte den Kopf. Sein Kiefer zitterte. \WR{Tut mir leid, Madam. Aber diesen Befehl, kann ich nicht weiterleiten.}

\par

Fiscale nickte. \WR{Das kann ich verstehen.} Sie sah zu Wallander vertreter. \WR{Hollowitz, Sie haben ab jetzt die Leitung über die Kommunikation. Lieutenant Wallander, Sie sind freigestellt.}

\par

\WR{Madam…}

\par

\WR{Verlassen Sie bitte die Brücke}, bekräftigte Fiscale. \WR{Sofort.}

\par

Der suspendierte Chef der Kommunikationsabteilung warf seinen Kopf förmlich zur Seite und eilte zum Lift.

\par

Eine Weile herrschte betretenes Schweigen, in der jeder es vermied, die Kommandantin anzusehen. Doch es wurde schlagartig wieder lauter, als Lieutenant Commander Petrarca rief: \WR{Bomberstaffel im Anflug! Nuklearalarm. Atomraketen werden gleich abgeschossen!}

\par

\WR{Wo zur Hölle kamen die her?}, fragte Fiscale verärgert. \WR{Das sind ja keine fünf Trinmeter mehr!}

\par

Ihre Frage blieb unbeantwortet. Schon war die erste Salve Torpedos abgefeuert. Die Geschosse waren sogar mit bloßem Auge zu erkennen und rasten scheinbar unaufhaltsam auf die \EN{Regenvogel} zu.

\par

Das taktische Hologramm zeigte, wie einige Abfang- und Verteidigungsjäger den Flugkörpern entgegen traten. Und tatsächlich verschwanden einige der gelben Symbole von der Anzeige. Doch es waren derart viele, dass sie jedem klar war: einige würden es durch die Verteidigung schaffen.

\par

\WR{Batterien zum Bug!}, schrie Petrarca seine Kanoniere an. \WR{Alles auf Einschlag vorbereiten!}

\par

Nun eröffneten die kleineren Geschütze der \EN{Regenvogel} das Feuer und jagten den Torpedos ihre Entladungen entgegen. Die Strahlen trafen auf einige der Geschosse und ließen sie verglühen. Doch zwei der Raketen brachen durch und erbarmungslos auf den Schutzfeldern ein. Der Lärm war schier ohrenbetäubend und sofort fiel das Licht und jeder Computer aus.
