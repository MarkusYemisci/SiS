\WR{Captain, wenn Sie es tun wollen, dann jetzt!}, rief Legatin Gajjar. \WR{Die Shutek überrennen die Stadt. Wir sind geschlagen. Es sind kaum noch Zivilisten hier.}

\par

Natalia Fiscale stand wie eingefroren auf ihrer eigenen Brücke. Hilflos sah sie das Übersichtshologramm an. Wie sie teilweise schon durch die Fenster der Kommandozentrale erkennen konnte, hatten die Shutek ihre Flotte tatsächlich aufgeteilt. Eine Gruppe näherte sich Kreuzpunkt Primus, während die andere geradewegs auf die Blockade der Starforce zuhielt.

\par

\WR{Ein passender Name hat Ihr Schiff}, hängte Gajjar nun etwas leiser an. \WR{Lassen Sie es regnen.} Dann endete die Übertragung.

\par

Captain Fiscale riss sich aus ihrer Schockstarre und begab sich an die Reling. Zu Lieutenant Petrarca rief sie: \WR{Einsatzleitung, haben wir irgendeine Chance, noch einen Luftangriff gegen die Front der Shutek zu fliegen?}

\par

Der Mann schüttelte mit betretenem Blick den Kopf. \WR{Auf keinen Fall, Madam. Die Shutek sind längst in Yêxīn eingefallen. Unsere Piloten würden nichts treffen. Und von dort aus können sie mit ihrer Artillerie schon fast die Silos unter Beschuss nehmen.}

\par

Fiscale schwieg für einen langen Moment. In dieser Zeit erschien ihr die Schlacht völlig fern, die gerade vor ihren Augen zu toben begann. Die ersten Schiffe der Shutek lieferten sich bereits einen harten Schlagabtausch mit den vorgeschobenen Zerstörern und Fregatten. Strahlenentladungen flogen hin und her. Rissen Hüllen auseinander und bohrten brennende Löcher in die Rümpfe des jeweils anderen Schiffes.

\par

Währenddessen schlichen sich weitere Shutek in Feuerposition um die Oberfläche des Planeten in einer nuklearen Sintflut untergehen zu lassen.

\par

\WR{Kommunikation. Geben Sie Feuerbefehl.}

\par

Wallander Vertreter schien es möglichst schnell hinter sich bringen zu wollen. Ohne seine Kapitänin anzusehen, beugte er sich über das Mikrofon seiner Station und sagte: \WR{An alle Zerstörer. Bombardement nach Muster Null Null Null Zerstörung Null. Feuer frei.}

\par

Jedes der pfeilförmigen Schiffe bis auf eines drehten daraufhin dem Planeten ihre Unterseite zu und öffneten fast zeitgleich die Bombenschächte. Nullzonensprengköpfe wurden in Position gebracht. Und noch ehe die Shutek nahe genug herankommen konnten, um vielleicht die ein oder andere Bombe abzufangen, wurden sie abgeworfen.

\par

Völlig klanglos fielen mit einem mal ein paar Dutzend blau schimmernde Kugeln mit langem Schweif auf die Oberfläche zu. Aus der Ferne erinnerten sie an harmlose Sternschnuppen. Doch Fiscale wusste, dass diese langsam auf die Oberfläche zufliegenden Lichter die zerstörerischsten Waffen waren, welche die Menschheit bis zur Zeit der Union entwickelt hatte.

\par

Um ein Vielfaches vernichtender als eine Atombombe. Aber auch deutlich kontrollierbarer. Die instabile Nullzone im Kern der Waffe ließ sich sehr einfach auf ein bestimmtes Gebiet begrenzen. Auch entstand keine ionisierende Strahlung und Fallout wurde somit vermieden.

\par

Unaufgefordert hatte Maas Petrarca das Übersichtshologramm auf den Wirkungsbereich der Nullzonenbomben zentriert. Mittlerweile hatten die diskusförmigen Waffen die Stratosphäre erreicht. Ihre Triebwerke glichen Luftmassenbewegungen aus und steuerten sie so sicher auf ihre Zielgebiete zu.

\par

In diesen warteten zahllose Shutek aber auch viele hundert Menschen auf ihr Ende.
\ortswechsel
Legatin Gajjar griff nach ihrer Waffe. Es war schon eine Weile her, seit sie selbst ein Gewehr getragen hatte. Ihre Tage als Infanteristin lagen weit hinter ihr. Seit etlichen Jahren kommandierte sie nun schon die Streitkräfte der Phalanx in Yêxīn.

\par

Patronen, Rüstungen und Gewehre waren aus ihrem Alltag gewichen, seit sie in ihr beschauliches Büro mit schönem Blick auf den Platz des Commonwealth bezogen hatte. Nun konnte sie sich mehr Gedanken über Zimmerpflanzen und die Innenausstattung ihres Arbeitsplatzes machen. Das Waten durch Regen und Matsch war dem Unterschreiben von Versetzungsplänen und Beförderungen gewichen. Kameradschaft hatte sie gegen Kollegialität eingetauscht. Der Gang der Dinge in den höheren Rängen des Konglomerats.

\par

Doch nun holte sie wieder ein, wofür sie ausgebildet worden war. Gerade noch hatte sie sich noch mit einem halben Dutzend leitender Offiziere beraten. Ein Hologramm in der Mitte ihres Büros zeigte, wie die Stadt aus allen Richtungen von roten Punkte überflutet zu werden schien.

\par

Viele klein und unscheinbar. Andere größer und an einen ausgelaufenen Tintenkleks erinnernd. Aber alle schienen sie viel harmloser, als sie es in Wirklichkeit waren.

\par

Gajjar sah durch ihr Fenster. Gerade eben hatte der Platz des Commonwealth noch so friedlich wie immer gewirkt. Ein weitläufiger Halbkreis vor dem Gebäude, dass einmal der Regierungssitz des Planeten und seiner Kolonien gewesen war. Bis zum Ende des Routenkriegs und der Gründung der Union in dessen Folge.

\par

Der dunkle Marmor erschien so schlicht wie edel. Genauso, wie das breite Gebäude mit seinen beiden ausladenden Flügel und seinem hohen Turm auf dem heute die Flagge der Union wehte. Nur knapp über einem Wetterhahn. Den Sinn für das schrullige hatten die Kreuzpunkter zu keinem Zeitpunkt verloren.

\par

Nun war der Platz aber menschenleer bis auf ein paar einsame Wachmänner. Die meisten davon Soldaten der Union. Aber auch ein paar Polizisten hatten sich der hoffnungslosen letzten Garde angeschlossen.

\par

Gajjar sah auf ihre Uhr. Es war beinahe soweit. Weniger als zwei Minuten und dann wäre alles vorbei. Und schon stürmten die ersten Shutek auf den Platz. Die Legatin sah zu ihrer Sekretärin. Auf eine ausgebildete Soldatin, die jedoch ihren Dienst quittiert hatte. Trotzdem wirkte die Strahlenpistole in ihrer Hand nicht deplatziert.

\par

\WR{Wollen wir es einfach abwarten?}, fragte sie Legatin Gajjar. Die junge Frau schüttelte mit einem bitteren Lächeln den Kopf. \WR{Nein. Es macht zwar keinen Unterschied. Aber ich will versuchen, noch ein paar von denen mitzunehmen.}

\par

Die ersten Schüsse fielen. Gajjar sah ein letztes mal aus dem Fenster und beobachtete, wie die Strahlenentladungen auf ihre Kameraden niedergingen. Die Wachen hatten keine Chance, das war sofort klar. Dafür gab es bei weitem zu wenige Gegner.

\par

Gemeinsam rannten die beiden Frauen los und ließen die Treppen zum Haupteingang eilig hinter sich. Ihre Schritte hallten laut auf dem steinernen Boden, in den das Emblem der Phalanx eingelassen war. Es hatte der Legatin nie gefallen. Das stilisierte Schild war ihr immer zu plump und einfallslos erschienen.

\par

Die Türen Glastüren öffneten sich zischen und ließen sofort Klänge der Schlacht an ihr Ohr dringen. Das Donnern von Gewehren und die verzweifelten Schreie der Soldaten, die sich irgendwie zu koordinieren versuchten.

\par

Gajjar zielte nicht lange. Sie hatte nur noch wenige Augenblicke und brauchte sich um keine Munition oder Möglichkeiten zur Deckung zu sorgen. Die Granate aus dem zweiten, breiteren Lauf ihres Gewehrs schlug mitten in den Reihen der Shutek ein. Einige der gepanzerten Soldaten wurden von der Wucht der Explosion zu den Seiten hin davon geschleudert.

\par

\WR{Angriff!}, rief die Legatin und animierte so die sich verstandenen Truppen aus ihrer Deckung hervorzukommen und laut brüllend sich gegen die anrückenden Gegner zu werfen.

\par

Die Männer und Frauen des kleinen Trupps fielen einer nach dem anderen. Doch das schonungslose Feuer ihrer Schützenreihe warf auch einige der Angreifer von den Beinen. Eine Handgranate flog in hohem Bogen über das Schlachtfeld und landetet nur wenige dutzend Meter vor der Legatin auf dem Boden. Als sie detonierte, verwandelte sie gleich mehrere Shutek in Aschewolken.

\par

Durch den Rauch des Explosion zuckten weitere Strahlen und zeugten von der schieren Anzahl der anrückenden Gegner. Eine der Entladungen fand ihren Weg zu Gajjars Schulter und warf sie zu Boden.

\par

Als sie zum Liegen kam, und seltsam ruhig wurde, konnte sie das leise Pfeifen hören, das immer lauter wurde. Von ihrer eigenen Entspannung völlig überrascht sah sie gen Himmel. Zu den funkelnden Sternen gesellten sich zahlreiche blaue Lichter, die langsam aber sicher zu großen Kugeln anschwollen.

\par

Als sie den Kopf drehte, erkannte sie, wie sogar einige der Shutek einfach regungslos stehen blieben und mit ansahen, wie eine der Bomben auf einen Punkt, ganz in ihrer Nähe zufiel. Kurz nachdem die blaue Kugel fast den gesamten Himmel über dem Platz des Commonwealth einnahm, ging es sehr schnell.

\par

Jeder und alles spürte nur noch kurz den enormen Sog. Der Himmel verschwand, wurde schwarz und jeder Mensch, jeder Shutek, jedes lose Bruchstück des Marmorbodens wurde auf einen einzelnen Punkt ein paar Straßen weiter hin gerissen.

\par

Niemand bekam mehr mit, wie sich die Nullzone aufbaute und in nicht einmal einer einzigen Sekunde einen gesamten Stadtteil kreisförmig vor einer hellblau gleißenden Mauer vor sich her schob und dabei Megatonnen an Masse innerhalb kürzester Zeit verschlang.
