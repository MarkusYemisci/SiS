Klaus Rensing ging auf den regungslosen Körper seiner Partnerin und Freundin zu. Eine Ladung Schrot direkt auf ihren Oberkörper gerichtet hatte ihre Bestimmung zumindest schnell erfüllt. Er schluckte. Manche Dinge waren einfach notwendig, auch wenn er sie nicht gerne tat.

\par

Es hatte Tage gegeben, da hatte er Laura als weinerliche Nervensäge erlebt. Jemand, der einfach nicht aufhören konnte, zu denken. Der jedes Ereignis durchleuchten wollte, nur um am Ende an der eigenen Schuld zu zerbrechen.

\par

Er hatte nicht gelogen~-- zumindest nicht voll und ganz. Sie war tatsächlich als Kandidatin für den Term infrage gekommen. Er hatte ihr Potential erkannt. Ihre deduktiven Fähigkeiten aber hatte er geglaubt, dass sie in der Lage gewesen wäre, das große und ganze zu erkennen. Ein Irrtum, wie er beim Anblick ihres Körpers nun glaubte.

\par

Das fahle Licht, dass von der regnerischen Ebene in die Mensa des Krankenhauses herein drang, unterstrich sein Empfinden. Sie hatte nicht erkennen können, wie ihr Leben hätte werden können. Stattdessen hatte sie nur die Dunkelheit und den Regen gesehen.

\par

\WR{Tut mir leid, Laura}, sagte er. \WR{Ich hoffe, du findest Frieden, wo auch immer du jetzt bist. Vielleicht holt doch ja Johanna ab. Stell dir das mal vor.} Klaus lächelte zufrieden. \WR{Ihr beide seid im Himmel. Und du gräbst wie damals ihren Sarg aus. Aber sie ist noch am Leben. Und dann geht ihr beide gemeinsam nach Hause. Ein schönes Ende für dich.}

\par

Klaus beugte sich über Laura. Ihre Augen waren weit aufgerissen und starrten nun an die Decke. Der Schrecken, den sie gespürt haben musste, bevor sie getroffen worden war, hatte mit Sicherheit nicht lange gedauert. So legte er seine Schrotflinte beiseite und legte ihr die Hände auf die Stirn.

\par

Zu spät bemerkte er, was er längst hätte sehen müssen. Sie war unverletzt. Keine Wunde an der Brust und auch keine irgendwo sonst. Das einzige, was ungewöhnlich schien, war ein heftiges Blinken aus ihrem Revers.

\par

Ihre Faust traf sein Gesicht mit einer Geschwindigkeit, die er ihr niemals zugetraut hätte. Er taumelte beiseite und Laura schwang sich in einer schnellen, gleitenden Bewegung auf die Füße. Hastig sah sie sich nach ihrer Strahlenkanone um, fand sie aber nicht. Ohne fiel darüber nachzudenken, trat sie Klaus Schrotflinte beiseite. Es war ein völlig veraltetes Modell, das nur von Sportschützen verwendet wurde und direkt mit zwei Patronen in die beiden Läufe geladen wurde. Er hatte vermutlich noch Extramunition, aber selbst wenn er die Waffe erreichte, würde er erst einmal nichts damit anfangen können, denn die beiden Schüsse waren verbraucht.

\par

\WR{Du steckst voller Überraschungen}, sagte Klaus spöttisch und rieb sich die getroffene Schläfe. \WR{Wie hast du das gemacht? Mit einer Panzerweste sähest du viel fetter aus.}

\par

Laura blieb wie angewurzelt stehen. Sie nahm keine bestimmte Haltung ein, war aber bereit, sich zu verteidigen. \WR{Ein persönlicher Blocker. Strahlenpistolen sind nicht das einzige, was wir beim Geheimdienst als erstes bekommen.}

\par

Klaus lachte kurz. \WR{Okay, ich bin beeindruckt. Ich hab dich echt unterschätzt. Aber die Entscheidung war eindeutig. Du musst leider weg.}

\par

Sie versuchte es unauffällig zu tun, ließ ihren Blick aber immer energischer umhergehen, als sie ihre Pistole nirgendwo sah.

\par

\WR{Suchst du die hier?}, fragte Klaus breit grinsend und zog ihre Waffe aus seiner Jacke. \WR{Hast du fallen lassen. Du warst schon immer ein bisschen ungeschickt.} Laura glaubte, fast so etwas wie Genuss in seinem Blick zu erkennen, als er den Strahler auf sie ausrichtete. \WR{Ich war nie gut im Schach. Außer in so etwas hier. Und ich glaube, die \EN{Königin} ist gerade matt gesetzt worden.}

\par

Als er den Abzug betätigte, gab die Pistole einen Warnton von sich, tat sonst aber überhaupt nichts. Klaus sah überrascht auf die kleine Anzeige über dem Griff, die ihm sagte, dass er kein genehmigter Nutzer war.

\par

\WR{Beim Schach setzt man den König matt, du beschissener Vollidiot}, belehrte ihn Laura und stürmte auf ihn los.
