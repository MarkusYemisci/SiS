Das Klima in Douala, der Hauptstadt der Union, hatte sich seit Henry Otis letztem Aufenthalt nicht geändert.
In der Küstenstadt, die auch häufig als \Wr{The Bigger Apple} bezeichnet wurde, herrschten die meiste Zeit des Jahres eine trockene Hitze und ein wolkenloser Himmel vor.
Dennoch empfand der Präsident den Aufenthalt in dieser Stadt alles andere als unangenehm.
Douala war eine der hochentwickeltsten Metropolen der Erde, mit einer langen Geschichte, die bis in die Jahrhunderte vor die Seuche zurückreichte.
Allerdings hatte sich die Stadt erst nach dem Gabbot-Virus zu der entwickelt, die sie heute war.

\par

Nachdem die Seuche aus unbekannten Gründen plötzlich ausgestorben war, waren viele Menschen nach Afrika und die polarnahen Regionen geflohen.
Dort hatte sich der Virus, aufgrund des extremen Klimas, deutlich weniger stark ausgebreitet.
Da immer mehr und mehr Überlebende ein neues Leben in Afrika begonnen hatten, hatten sich die bisher benachteiligten Städte des Kontinents zu den neuen Metropolen der Welt entwickelt.

\par

Douala war das beste Beispiel dafür.
Dank der Nähe zum Meer genoss die Stadt noch die Vorzüge eines etwas milderen Klimas und über die drei Jahrhunderte hatte sie sich zum blühendsten Schmelztiegel der Herkünfte entwickelt, den Menschen bisher gesehen hatten.

\par

Die Stadt nahm mittlerweile so viel Fläche ein wie Berlin.
Angefangen von geschmackvollen, alten Häusern in den Museumsdistrikten in der Vorstadt, über die riesigen Finanz-, Transport- und Einkaufszentren in den Wolkenkratzerschluchten des ökonomischen Viertels bis hin zu den modernen Regierungsgebäuden, die im krassen Gegensatz zur Bauweise der ursprünglichen Häuser standen, zeigte die Stadt ein edles und zeitloses Gesicht.
Architektonische Stile aus den unterschiedlichsten Epochen seit dem Ende der Seuche vereinigten sich in dieser einst kleinen Metropole.

\par

Das die Natur um die Stadt herum weitestgehend vom Virus verschont geblieben war, trug auch einen großen Teil zur Schönheit des ganzen Areals bei.

\par

Henry Otis hatte gerade das Parlamentsgebäude erreicht und freute sich nun über die Klimatisierung.
Trotz der hohen Temperaturen trug der Präsident nach wie vor seinen Gehrock.
Etwas anderes, so glaubte er, würde unpassend und dahergekommen wirken.
Dennoch schwitzte er beinahe mehr, als sein Membrandeodorant bewältigen konnte.

\par

Schnell durchquerte er die Vorhalle des Gebäudes, das aus der Zeit kurz nach der Seuche stammte.
Er hatte keine Zeit, die kunstvollen Verzierungen zu bewundern, die sich auf den Marmorsäulen mit moderner Architektur fanden.
Das Forum würde demnächst mit seiner Tagung beginnen und es würde schlecht aussehen, wenn er zu spät käme.

\par

Bei dieser Gelegenheit ärgerte sich Otis über die Sitzungsregelungen.
Obwohl er die Stadt Douala sehr faszinierend fand und immer wieder gerne besuchte, wäre es für ihn praktischer gewesen, wenn die Parlamentstagungen im Hauptquartier nahe Wellington stattfinden würden.
Zumal die meisten anderen Treffen auch dort abgehalten wurden.

\par

Ein kurzer Blick auf die Uhr machte dem Präsidenten klar, dass die Sitzung bald beginnen würde.
Er hätte nicht einmal mehr Zeit, sich seine Ansprache noch einmal durchzulesen.
Normalerweise wäre das für ihn kein größeres Problem gewesen.
Sein ganzes Leben hielt er nun schon Reden vor einem großen Publikum.
Seine Notizen brauchte er kaum noch.
Aber dieses mal fühlte er seit langem wieder unsicher.
Was Kanzlerin Reveneur ihm eröffnet hatte, war mehr als geeignet, ihn aus der Ruhe zu bringen.
Selbst beim schreiben der Rede hatte er sich kaum konzentrieren können.

\par

Die Vorhalle zum großen, an das Barocke erinnernde, Sitzungssaal war wie leergefegt, was hieß, dass die Abgeordneten bereits ihre Plätze eingenommen hatten.
Henry Otis entschloss sich, den Raum von der anderen Seite aus zu betreten.

\par

Der Sitzungssaal hatte einen halbkreisförmigen Grundschnitt und besaß die Fläche eines Fußballfelds.
Der, an sich schmucklose, Raum erinnerte an ein Theater, was hin und wieder auch metaphorisch verstanden werden konnte.
Präsident Otis, sein Stab und der jeweilige Redner hatten ihren Sitz auf einer kleinen Empore die im Zentrum des Halbkreises den Zuschauerreihen entgegenstand.
Die Sitzgruppen, die komfortablen Platz für fast achthundert Gäste boten, verliefen zum Saalinnern hin abfallend, was viele an einen altertümlichen Hörsaal erinnerte.

\par

Die Wände des Raumes bestanden hauptsächlich aus Fenstern, die bis zu fünfzehn Meter hoch waren.
Über der Empore der Redner und des Präsidentenstabes war das Banner der Union in Marmor in die Wand eingelassen.
Der Stein passte gut zum hölzernen Mobiliar des Saals.

\par

Präsident Otis erreichte die Rednerempore, kurz bevor seine Vizepräsidentin die Sitzung eröffnete.
Lertha Akintola, hatte in ihrer Rolle als präsidiale Vertreterin hauptsächlich repräsentative Aufgaben.
Außerdem war sie für die Ordnung und die Leitung der Debatten im Forum verantwortlich.

\par

Die Strenge, die sie dazu oftmals brauchte, stand ihr bereits in ihr rüstiges Gesicht geschrieben.
Trotz ihrer fünfzig Jahre wirkte Akintola immer noch agil und energisch und noch keine einzige Falte hatte sich auf ihre Haut verirrt.
Henry Otis hatte einmal gescherzt, sie hätte es ihnen einfach verboten.
Auch ihre Haare hatten noch ihre vollkommen natürliche Farbe.

\par

Akintola hatte ihren Kollegen noch nicht bemerkt.
Ungerührt begab sie sich ans Rednerpult, schaltete das Mikrofon ein und begrüßte die Anwesenden:
\WR{Guten Abend.
Wir sind heute zusammengetroffen, um eine Sitzung des Forums abzuhalten.
Die Tagesordnung liegt Ihnen allen vor, ich werde sie daher nicht noch einmal verlesen.}

\par

Ein Teppich leiser Geräusche erklang, als die Abgeordneten begannen, die Liste in ihre Computerkonsolen zu laden.
Diese waren in die hölzernen Tische eingelassen und konnten für viele verschiedene Aufgaben genutzt werden.
Ein Abgeordneter konnte sich während einer Vorlesung problemlos über thematische Fragen schlau machen, und so ein besseres Urteil fällen.
Auch wenn diese Technologie veraltet war und die meisten ihre persönlichen Bücher zu den Sitzungen brachten, hatte man sich noch nicht darauf einigen können, sie zu ersetzen.

\par

\WR{Leider gelang es uns nicht, während der letzten Sitzung einen einheitlichen Kurs zu beschließen, was die Abgabepolitik betrifft.
Die Meinungen, die auf eine Veränderung der Regelungen bestehen, mögen eine Minderheit sein, sollten aber gehört werden. Insbesondere, weil Gnosis glaubt, dass diese Thema den Bürgern derzeit am meisten am Herzen liegt.}
Eine unterschwellige Anklage gegen jene, die versucht hatten, dieses Thema kleinzureden war deutlich zu hören.
Akintola hatte dabei aber die Fakten auf ihrer Seite.
Der Gnosis-Supercomputer analysierte die persönlichen Daten eines jeden Unionsbürgers ohne Unterlass.
Einschließlich den ausgetauschten Meinungen.
Sowohl solche, die öffentlich gemacht wurden, als auch solche, die nur privat zugänglich waren.

\par

Was eine Einzelperson sagte, erfuhr niemand.
Aber ein Gesamtbild wurde stets von Gnosis zusammengestellt und an alle Entscheidungsträger weitergeleitet.

\par

Vizepräsidentin Akintola sah unauffällig auf ihr Buch.
Er lag, für das Publikum nicht sichtbar, auf dem Rednerpult und zeigte persönliche Notizen an.
Diese ablesend, sprach sie weiter: \WR{Ich denke, Herr Fester Shile, von Kreuzpunkt Primus, wollte gerade eine Rede halten, als wir am Ende der Zeitbeschränkung angelangt waren.
Daher schlage ich vor, wir schließen direkt daran an.
Das Thema ist ja nach wie vor das gleiche.
Herr Shile, kommen Sie bitte an das Rednerpult.}

\par

Der kleine Mann von Kreuzpunkt Primus begab sich in Richtung der Empore.
Er ging recht gemütlich und ließ sich Zeit, was ihn als einen erfahrenen Redner auswies, den nur noch wenig nervös machen konnte.

\par

Während dessen hatte sich Präsident Otis still und heimlich auf einen Stuhl gesetzt und tat nun so, als sei er schon immer da gewesen.
Aufmerksam verfolgte er Fester Shiles Rede.
Der Wirtschaftsexperte forderte, wie schon seit Längerem, eine Umstellung der Abgaberegeln.
Eine heikles Thema, wenn man bedachte, wie viel in der Union von den Abgaben abhing. 
Am Anfang jedes Jahres und jedes Monats, schrieb sich die Union ein Plus in ihre eigenen Kassen.
Mit diesem Geld konnten die staatlichen Garantien, wie Bildung und Gesundheitsvorsorge, gedeckt werden.
Am Ende eines Jahres, beziehungsweise eines Monats wurde dieses Plus wieder ausgeglichen, in dem der Betrag von sämtlichen Geldquellen der Union, von den Ersparnissen des kleinen Mannes bis zu den größten Firmenvermögen, wieder gestrichen wurde.
Ganz egal wie groß oder klein die Konten auch waren, von jedem wurde der Anteil abgezogen, den die entsprechende Geldquelle am Gesamtvermögen hatte.

\par

Im Prinzip unterschied sich das System nicht viel von der Erhebung von Steuern.
Allerdings hatte es den Nachteil, dass sich jeder ausrechnen konnte, wie viel Geld er abgezogen bekäme.
Dies hatte besonders vor den Zahltagen eine lähmende Wirkung auf die Wirtschaft, da viele Versuchten, ihr Vermögen in Sachwerte umzuwandeln.
Allerdings wollte in dieser Zeit auch am liebsten niemand Geld annehmen, da sich dadurch das eigene Vermögen erhöhte und eine größeres absolutes Guthaben abgebucht wurde.

\par

Der größte Unterschied zu den unterschiedlichen Steuersystemen der Vergangenheit bestand jedoch in der Berechnung der Abgabehöhen. 
Offiziell war es die Aufgabe des Ökonomierats, diese zu bestimmen.
Doch in der Realität fütterte dieser lediglich den zweiten Supercomputer der Union, genannt \Wr{Permutare}, mit Daten.

\par

Noch in einem höheren Maße als Gnosis war Permutare mit einer Semiintelligenz programmiert und in der Lage, sämtliche Geldflüsse in der Unio Terrestris zu erfassen und zu optimieren.
Viele sahen es als eine Art Frevel an, einem Computer de facto die wirtschaftliche Führung des Staates anzuvertrauen, der zudem noch eine Unmenge an sensiblen Daten speicherte.
Genau genommen wusste er über praktisch jeden Besitz Bescheid.
Doch Henry Otis fand es nur fair.
Permuatre legte die finanzielle Belastung jedes einzelnen nach rein rationalen Kriterien fest.
Er war unbestechlich.
Die einzige Möglichkeit für irgendwen, sich einen Vorteil zu erschleichen, wäre gewesen, ihn umzuprogrammieren.
Doch dagegen hatte man vorsorglich Maßnahmen getroffen.
Der Gnosiskomplex war noch vor dem präsidialen Büro der am besten geschützte Ort der ganzen Union.
Die eigentliche Sicherung bestand jedoch darin, dass das semiintelligente Betriebssystem Moralsysteme verstehen konnte.
Darum war Permutare nicht gewillt, Programmierungen durch Hacker anzunehmen.

\par

Fester Shiles Vorschlag, einen neuen Bemessungsfaktor einzuführen, löste heftige Diskussionen aus, bei denen die Vizepräsidentin nicht nur einmal eingreifen musste.
Einige der besserverdienenden Abgeordneten verließen den Saal als Ausdruck stummen Protests.
Noch konnten sie es sich leisten.
Aber in ihrem eigenen Interesse würden sie beim nächsten Treffen nicht verfrüht gehen, wenn über ein Vorschlag von Fester Shile abgestimmt werden würde.

\par

Präsident Otis langweilte sich über eine Stunde lang.
Wirtschaftliche Fragen lagen nicht auf seiner starken Seite.
Er hatte gleich mehrere ökonomische Spezialisten einberufen, als er den Regierungssitz bezogen hatte.
Nun wünschte er sich ihre Hilfe herbei, denn einigen Ausführungen konnte er nur mit großer Mühe folgen.
Hin und wieder fragte er die Vizepräsidentin um Rat, die ihm diesen diskret und unauffällig zukommen ließ.
Einer der Gründe, warum Henry Otis mit Akintolas Diensten mehr als Zufrieden war.
Sie war die kompetenteste Stellvertreterin, die er jemals gekannt hatte.
Immer knallhart aber gerecht, kompetent aber auch mitfühlend.
Attribute, die einen guten Politiker zu Zeiten der Union so sehr ausmachten, wie sie einen Staatsmann vor der Seuche angeblich ausgemacht hatten.

\par

Nach guten fünf Stunden des Debattierens, bei denen um einiges mehr Meinungen ausgetauscht worden waren, als anfangs angenommen, war die Tagesordnung dahin.
Die Vizepräsidentin schüttelte beim Blick auf die Uhr häufig den Kopf.
Die gesamte Phase des Diskutierens, wurde nur ein mal durch eine halbstündige Pause unterbrochen.
Präsident Otis hatte keinen Hunger gehabt, ihm lag schon den ganzen Tag ein Thema auf dem Magen.
Und momentan sah es nicht danach aus, als könnte er seine politischen Sorgen öffentlich machen.
Die Diskussion über die Abgaben ging weiter und weiter und noch lange war keine Einigung in Sicht.

\par

Henry Otis entschloss sich, von seinem Recht als Präsident Gebrauch zu machen.
Unauffällig wandte er sich an seine Stellvertreterin und flüsterte:
\WR{Frau Akintola, ich hatte für heute noch ein anderes Thema geplant.
Denken Sie, es wäre vertretbar, dieses einzuschieben?}

\par

Akintola überdachte die Anfrage einen Moment lang.
Theoretisch konnte sie den Präsidenten nicht daran hindern, etwas zu sagen.
Er war der einzige, der das Recht hatte, ohne weiteres einen Einschub zu machen.
Dennoch hatte Henry Otis dies niemals getan.
Er achtete sehr darauf, mit seiner Stellvertreterin gut zurecht zu kommen.
Daher sprach er sich meistens mit ihr ab, wenn er etwas im Forum anbringen wollte.

\par

\WR{Ich denke schon}, antwortete sie schließlich tonlos.
\WR{Ich werde eine Vertagung des Themas anregen.
Wir sind sowieso schon fast wieder am Ende der Zeit.}

\par

Die Vizepräsidentin begab sich zum Rednerpult, an dem gerade ein Abgesandter von Pollux sprach.
So diskret wie möglich, bat sie ihn, bald zu einem Ende zu kommen.
Nicht gerade begeistert brachte der Mann seine Rede nach fünf Minuten zu einem Ende und kehrte ins Publikum zurück.
Wie nach den meisten Ansprachen herrschten rege Diskussionen, die sich zum Teil über große Distanzen durch den Saal erstreckten.

\par

Lertha Akintola regulierte die Lautstärke des Mikrofons am Rednerpult ein wenig nach oben und sprach:
\WR{Ich darf um Ruhe bitten.}
Und nach einem weiteren \WR{Ruhe bitte!} verstummten die Gespräche allmählich.

\par

Das war das Zeichen für Präsident Otis, an den Rednerstand zu treten.
Er blieb einen Moment lang still und wartete darauf, dass sich unter den Abgeordneten des Forums eine erwartungsvolle Stille breit machte.
Hier und da begannen einige der Senatoren leise mit einander zu flüstern.

\par

Schließlich begann Henry Otis seine Ansprache.
\WR{Meine Damen und Herren.
Ich weiß, ich unterbreche eine wichtige Diskussion und dafür möchte ich mich entschuldigen.
Es ist zwar mein Privileg als Präsident in eine laufende Gesprächsrunde einzugreifen aber ich habe nicht die Absicht, dieser Versammlung ins Handwerk zu pfuschen.}

\par

Otis fand es schwierig die Ausdrücke in den Gesichtern der beinahe achthundert Senatoren zu lesen.
Die meisten konnte er gar nicht erkennen und die, welche ihn ansahen, wirkten zu Teil verwundert oder waren gänzlich ausdruckslos.

\par

\WR{Ich denke, ich habe ein wichtiges Anliegen, dass uns alle betrifft}, fuhr der Präsident fort.
\WR{Wie Sie wissen, liegt mir der Frieden sehr am Herzen.
Seit meinem Einstig in die Politik arbeite ich daraufhin, unsere Gesellschaft sicherer zu machen und Verständnis zu fördern.}

\par

Einige der Abgeordneten begannen bereits interessiert zum Rednerpult hinauf zu sehen.
Andere konnten sich nicht so einfach vom letzten Thema verabschieden und beschäftigten sich mit ihren Computerkonsolen.
Der Präsident fuhr fort.
\WR{Um Ihnen zu verdeutlichen, wie wichtig dieses Thema für mich ist, werde ich Ihnen nun eine kleine Anekdote erzählen.}

\par

An dieser Stelle erhoben sich einige Köpfe, von ihren Konsolen und sahen Henry Otis entgegen.
Bisher hatte der Präsident äußerst selten auf bildhafte Sprache zurückgegriffen, geschweige denn darauf, Geschichten zum besten zu geben.

\par

Er räusperte sich und begann zu erzählen:
\WR{Vor über achtzig Jahren, hat die Menschheit eine ihrer größten Sünden verbrochen.
Obwohl wir wussten, wie wichtig ein Menschenleben ist, haben wir uns nur allzu bereit in einen Konflikt gestürzt.
Den Routenkrieg.}
Einige der Senatoren zeigten Reaktionen auf diese Aussage.
Andere blieben scheinbar ungerührt.
\WR{Die Erdallianz hat dem Commonwealth von Kreuzpunkt den Krieg erklärt.
Wie es dazu kam, brauche ich Ihnen nicht zu berichten.
Einige von Ihnen sind mit unserer Geschichte wahrscheinlich noch besser vertraut als ich.
Aber eines, denke ich, kann klar gesagt werden.
Keine der beiden Seiten hatte Skrupel, einen Krieg zu beginnen.
Das Commonwealth hofften durch militärische Überlegenheit ihren Vorsprung auf dem wirtschaftlichen Sektor zu bestätigen, während die Erdallianz glaubte, mit einem Militärschlag die Capezinressourcen und die Hyperraumrouten ihres Gegners an sich reißen zu können.
Das Ergebnis war der verheerendste Konflikt nach dem zweiten Weltkrieg.}

\par

Hier und da pflichteten Abgeordnete dem Präsidenten nickend zu.
Lertha Akintolas Gesichtszüge erstarrten zu Eis, als sie sich langsam denken konnte, worauf der Vorsitzende hinaus wollte.

\par

\WR{Und in den Wirren dieses Krieges}, sprach Henry Otis weiter, \WR{bewies eine kleine Gruppe von Soldaten, dass die Menschen die Lektion noch nicht verlernt hatten, die ihnen die Seuche beigebracht hatte.
Es war die Besatzung einer Haftanstalt des Commonwealth, die Kriegsgefangene innehatte.
Irgendwann entschied der Kommandant dieser Einrichtung für sich, dass der Krieg sinnlos war.
Er entließ alle Kriegsgefangenen aus der Haft legte seinen Dienst nieder.
Diese Tat war ein entscheidender Grundstein für den späteren Frieden und auch dem Hervorgehen der Union aus den Resten der beiden Seiten.}

\par

Die Vizepräsidentin trat an ihren Vorgesetzten heran.
Sie sagte nichts, versuchte aber Otis mit ihrer Anwesenheit zu warnen, denn sie wusste, was dieser vorhatte.

\par

\WR{Was denken Sie, ist mit diesem Mann und seiner Mannschaft passiert?}, fragte der Präsident.
Die Zuhörer reagierten nicht auf diese rhetorische Frage aber hier und da begannen leise Diskussionen aufzuflammen.
\WR{Sie wurden hingerichtet}, fuhr Henry Otis fort.
\WR{Das Militärgericht, vor das die Männer und Frauen gestellt wurden, fand, es sei untragbar für einen Offizier, Befehle nicht auszuführen.
Aber in Wirklichkeit ging es nicht darum.
Es ging darum, dass diese Männer und Frauen selbstständig gedacht hatten, ihre eigenen Entscheidungen getroffen hatten.
Und sie haben das Richtige getan.
Aber es ist nun mal ein militärischer Grundsatz, dass Anordnungen ausgeführt werden.
Und wer dies nicht tut, der muss eliminiert werden.
Es gibt keine Individualität in einer Armee.
Keine selbstständigen Gedanken oder eigene Entscheidungen.
Nur der blanke Gehorsam!}

\par

Der Präsident hielt mittlerweile das Rednerpult eng umklammert und seine Stimme nahm mehr und mehr einen erregten Ton an.
\WR{Sie werden vielleicht sagen: Heutzutage ist das anders.
Das Konglomerat legt Wert auf Bildung und Eigenverantwortung.
Und Todesstrafen gibt es keine mehr.
Aber Sie irren!}

\par

Henry Otis lies einen Moment verstreichen um seine Aussage durch die Reihen der Abgeordnete nachschwingen zu lassen.
Nun hatte er die ungeteilte Aufmerksamkeit.
Niemand hätte damit gerechnet, dass der Präsident derart emotional werden würde.
Und keiner hätte geglaubt, dass er vorhatte, wieder ernsthaft gegen die Armee ins Feld zu ziehen.

\par

\WR{Die Union hat eine Arme.
Und eine Armee hat die Aufgabe zu zerstören.
Unser Konglomerat mag anders sein.
Etwas milder und offener als andere Streitkräfte.
Mit angeblich anderen Zielsetzungen.
Aber es ist und bleibt eine Armee.
Wir brauchen jedoch keine.
Es herrscht Frieden, schon seit achtzig Jahren.
Und Verbrecher, wie die Capital Fellowship, die eine Armee scheinbar rechtfertigten, sind nichts, womit die Polizei nicht fertig werden würde.
Eine Armee ist eine \EN{Schande} und eine Beleidigung, an alles, woran wir glauben!
Wie können wir Frieden predigen, wenn wir selbst Waffen besitzen, die so zerstörerisch sind, wie nichts je zuvor?
So lange nur eine Patrone in nur einem Gewehr steckt, dass von einem verantwortungslosen Menschen getragen wird, der alles tut, was man ihm aufträgt, ist unser Frieden in Gefahr.}

\par

Ein leichtes Rot hatte Präsident Otis Gesicht gefärbt.
Diese Wut hatte sich schon zu lange in ihm aufgestaut und nun war er kaum noch Herr über das, was er sagte.
Die Vergangenheit schien wieder lebendig zu werden.
Er sah sich selbst, vor Jahren reden, und die Armee zu kritisieren.
Damals hatte er seine bitterste Niederlage eingesteckt.
Aber er hatte sich entwickelt.
Er hatte eine Schlacht verloren aber Otis war wild entschlossen den Krieg zu gewinnen.

\par

Den Grund, der ihn gerade zu diesem Zeitpunkt dazu gebracht hatte, seinen alten Feldzug gegen das Konglomerat wieder aufzunehmen würde er aber nicht vor der Masse ausbreiten.
Das Grandadmiral Burns eine Aktion über seinen Kopf hinweg befohlen hatte, würde er mit ihm selbst bereden.

\par

Jeder der Anwesenden konnte schon erahnen, wie die nächsten Worte des Präsidenten lauten würden.
Eine fast spürbare Spannung lag über der Versammlung.
Einige der Senatoren waren sogar unbewusst aufgestanden, angesichts der Debatte die nun wohl bald auf die Versammlung zu rollen würde.
Alle Augen ruhten auf dem Präsidenten der, flankiert von Lertha Akintola und Richard Bellegardè, beantragte:
\WR{Meine Gründe und meine Überzeugungen kennen Sie.
Ich reiche hiermit den Antrag ein, die militärischen Zweige des Konglomerats schnellstmöglich abzuschaffen.}

\par

Von einem Moment auf den anderen war die gespannte Stille im zentralen Senat in ein wildes, lautes Gerede umgeschlagen.
Es gab kaum noch einen Abgeordneten, der noch auf dem Stuhl saß. Spontan bildeten sich Trauben von Menschen, die wild miteinander diskutierten.
Einige riefen Rednerempore unverständliche Sätze entgegen.
Andere verließen offenbar wutentbrannt den Saal, wahrscheinlich auf dem direkten Weg zu nächsten Zweigstelle Spectare State Vision um schnellstens ihre Meinung kundzutun.

\par

Obwohl niemand Ausschreitungen vermutete, begannen die Sicherheitsoffiziere, die den Saal bewachten, unruhig zu werden.
Es war ihnen kaum zu verdenken, denn so ein gefühlsgeladenes Aufeinanderprallen verschiedenster Ansichten hatte es im Senat seit Gründung der Union nicht mehr gegeben.
Männer und Frauen rannten hin und her.
Zückten ihre Bücher um sich mit Außenstehenden in Verbindung zu setzen oder fingen an, sich gegenseitig anzuschreien.

\par

Otis glaubte einige Satzfetzen aus dem Stimmengewirr heraushören zu können, dass unterdessen die Lautstärke eines Classop-Konzerts angenommen hatte.

\par

\WR{Das wollte der alte Mann schon immer!}, brüllte einer der Abgeordneten aus der vordersten Reihe, worauf ihm ein anderer, roten Kopfes, entgegenschrie:
\WR{Sie haben doch keine Ahnung!
Er hat völlig Recht!
Wir müssen mit diesen barbarischen Überbleibseln endlich aufräumen, wenn wir als Kultur weiterkommen wollen.}

\par

Lertha Akintola warf dem Präsidenten einen Blick zu, der strenger war, als jeder Ausdruck, den er bislang von ihr kannte.
Dann begab sie sich ans Rednerpult und versuchte mit voller Lautstärke der Lautsprecher wieder Ruhe in die aufgewühlte Menge zu bringen.

\par

Unterdes wandte sich Richard Bellegardè, der Collonele Senator, an den Präsidenten.
Er musste laut reden, um gehört zu werden:
\WR{Diesmal ist Burns nicht auf einem Truppenmanöver und es geht auch nicht um eine simple Etatkürzung.
Machen Sie sich auf einen harten Kampf gefasst.}

\par

\WR{Keine Sorge}, antwortete Henry Otis in erbarmungslosem Ton.
\WR{Ich habe nicht vor zu verlieren~-- schon wieder.}

\par

\WR{Wieso tun Sie das?}, wollte der oberste Senator aufgebracht wissen.
\WR{So kurz vor den Wahlen.
Sie haben doch bereits so viel erreicht.
Warum können Sie Ihre Fehde mit dem Militär nicht endlich beenden?}

\par

Der Präsident sah Richard Bellegardè durchdringend an.
Langsam und ruhig antwortete er ihm:
\WR{Ich tue das nicht, weil ich mich für die Niederlage rächen will, die ich vor Jahren gegen diese dämlichen Keulenschwinger erlitten habe.
Ich tue das, weil ich es wirklich so empfinde, wie ich es gesagt habe.
Bedingungsloser Gehorsam und brutale Disziplinen gehören der Vergangenheit an.
Die Menschen verdienen Frieden.
Und sie werden ihn endlich bekommen.}

\par

Der oberste Senator wandte sich von Otis ab und ging.
Der Präsident sah ihm mit reumütigen Augen nach.
Er hatte heute viele Freunde verloren und sich neue Feinde geschaffen.
Aber es hätte nicht anders passieren können.
Er wollte endlich die Entwicklung der Union zu einem Ende bringen und die Elemente, die er als Saat der Gewalt annahm ein für alle mal tilgen.
Und wenn man ihn dafür hasste, dann musste es so sein.

\par

Dieses mal hatte er weit bessere Chancen auf Erfolg.
Er war nun Präsident und nicht einfacher Senator, der monatelang dafür kämpfen musste, endlich eine Abstimmung zu erwirken~-- nur um dann zu verlieren.

\par

Der Senat würde über die Abschaffung der Armee ein Votum abgeben.
Ob früher oder später, spielte keine Rolle.
Und falls die Entscheidung zugunsten einer Schließung des Militärs fallen würde, bräuchte er sie nur noch abzusegnen und zu hoffen, das die Militärs nicht rechtzeitig einen Bürgerentscheid erwirken könnten.
Den würde es sowieso geben, wenn der Senat mit einem Nein stimme würde.
Diese Entscheidung würde Otis kippen und so eine direkte Abstimmung durch die Bürger der Union auslösen.

\par

Egal wie es ausgehen würde, diesmal konnte er gewinnen und die Menschheit von dem befreien, was er für ihre älteste Geißel hielt.
Und weder Reveneur noch Burns würden ihm dann noch unverschämt ins Gesicht grinsen können.