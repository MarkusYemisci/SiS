\WR{Ich glaube, das ist aus \Wr{Neunzehnhundertvierundachtzig}}, gab Hanna Moyér schließlich zu bedenken, nachdem sie über eine halbe Stunde lang nicht mehr gesagt hatte. Sie und Marco Bellendi saßen alleine einem der Besprechungsräume auf \EN{Minerva}. Durch die Fenster neben dem länglichen Tisch war Kreuzpunkt Primus Tagseite zu erkennen. Die vielen schneebedeckten Gebirge ließen klar werden, wieso diese Welt als der \Wr{weiße Planet} galt.

\par

Marco horchte auf. \WR{Was meinst du damit?} Es erleichterte ihn ungemein, das die Offizierin ihm mittlerweile das Du angeboten hatte.

\par

\WR{Das ist ein politischer Roman}, erklärte sie. \WR{Geschrieben ungefähr vier Jahrdutzende vor der Seuche. Wir haben ihn in der Schule durchgenommen. Die Handlung ist recht düster und enthält anstößige politische Sichtweisen.}

\par

Marco schrieb etwas in sein Buch neben die Transkription der Cygni-Botschaft. Wenig später hatte Gnosis nicht nur die letzten Sätze, sondern auch den Rest des Textes als Zitate aus Werken der Vorseuchenära identifiziert. Seine Augenbrauen formten ein V. \WR{Du hast recht. Jeder dieser Sätze stammt aus einem Buch oder einem Film. Darum sind sie auf Englisch. Alles Material, dass vor der Stunde Null entstanden ist. Das erklärt, warum unsere Computer den Ursprung dieses Textes nicht sofort erkannt hat. Erstens sind es Zitate aus unterschiedlichen Werken. Und zweitens braucht man für die meisten davon eine gewisse Gnosis-Freigabe. Der Teil darüber, dass wir großen Spaß beim Tanzen und, naja…}, Marco suchte salonfähige Worte, \WR{der sexuellen Reproduktion haben stammt aus einem nicht jugendfreien Krimi. Dafür muss man volljährig sein.}

\par

Hanna lächelte. \WR{Zum Glück bin ich schon zwei dutzend zehn.}

\par

\WR{\Wr{Neunzehnhundertvierundachzig} enthält, wie du schon gesagt hast, brisante politische und gesellschaftliche Thesen. Um das lesen zu dürfen, muss man Klassenstufe acht in Basal und Gemeinschaftskunde erreicht haben.}

\par

Hanna ließ ihren Blick wieder über die Nachricht schweifen. \WR{Es erklärt leider nicht, wieso die Shutek mit Zitaten zu uns sprechen. Oder was sie uns eigentlich sagen wollen.}

\par

\WR{Kurze Antwort}, begann Marco. \WR{Sie finden uns scheiße. Lange Antwort: sie wollen irgendetwas verstecken. Bevor es annähernd moderne Computertechnik gab, haben Verbrecher wie Entführer oder Verbrecher ausgeschnittene Buchstaben aus Zeitungen auf ein Blatt Papier geklebt, um damit eine zusammenhängende Botschaft zu schaffen. Auch etwas, das man aus Vorseuchenliteratur lernen kann. Vielleicht ist das ganz hier ja ähnlich. Aber wenn ich ehrlich sein soll: ich habe keine Ahnung.}

\par

Moyér erhob sich und trat an die Fenster. Das All beruhigte sie zuverlässig jedes mal, wenn ihre Gedanken rasten. Ihr Blick suchte die Einzelheiten auf Kreuzpunkts Oberfläche. Für sie völlig geräuschlos, schoben sich massive Wolkenformationen über den Himmel des Planeten. In dessen Lüften tobten nun intensive Schneestürme, doch aus der Umlaufbahn betrachtet, wirkten diese nur wie das harmlose Treiben im Inneren einer Schneekugel.

\par

\WR{Ich erinnere mich noch gut, wie ich mich gefühlt habe, als ich zum ersten mal die Erde vom All aus betrachtet habe}, erinnerte sich Marco zurück. \WR{Es ist ein wundervoller Anblick. Ich bin kein Austronaut, sondern Biologe. Aber von oben betrachtet, erkennt man erst wirklich, dass die ganze Welt ein einziger großer Organismus ist.}

\par

Hanna schenkte ihm ein Lächeln. \WR{Ich habe oft genau dasselbe gedacht. Sie atmet, isst, trinkt. Wird krank und wieder gesund. Die Natur verschwendet nichts. Egal, was passiert. Am Ende reiht sich alles zu einer sinnvollen Geschichte zusammen. Aber nicht so, wie in einem Theaterstück oder etwas in der Richtung. Diese Dinger werden geschrieben. Das Leben läuft ganz von alleine.}

\par

\WR{Dann siehst du es also wie die Shutek}, schlussfolgerte Marco. \WR{Bloß nicht so pessimistisch.}

\par

Hanna nahm wieder Platz. \WR{Denkst du, die Shutek haben in ihrer Nachricht über das Leben philosophiert?}

\par

\WR{Ich weiß es nicht}, gestand Marco ein. \WR{Vielleicht auch nur über den Platz der Menschheit darin. Der Teil mit \Wr{We are not what was intended.} macht mir Sorgen. Das bedeutet so viel wie: \Wr{Wir entsprechen nicht dem Plan.} Wenn das ihre Ansicht ist, dann steckt eine Ideologie hinter ihren Angriffen. Das wiederum bedeutet, dass wir vielleicht niemals mit ihnen reden können werden, selbst, wenn wir unsere Kommunikationsformen gegenseitig verstehen. Genauso gut könnte man versuchen, vernünftig mit religiösen Fanatikern zu sprechen.}

\par

Hannas Gesicht wurde nachdenklich. \WR{Meine Oma war Christin, weißt du? Und zwar eine überzeugte. Ich bin sicher, ihr Glauben hätte auch dem der Menschen vor der Seuche zur Ehre gereicht. Aber auch sie hat sich nicht völlig aus der Realität verabschiedet. Sie sagte immer: \Wr{Liebe deinen nächsten, wie dich selbst. Dann weißt du auch, wo du anfangen musst.}}

\par

Marco lachte. Doch sein Lachen verklang schnell, als sein Blick auf die Notizen über die bisherigen Angriffe der Shutek fiel. Insbesondere dem Nuklearwaffeneinsatz gegen Pollux Primus. Mittlerweile hatten Aufklärungsflüge bestätigt, dass fast die ganze Oberfläche mit Atombomben beschossen worden war.

\par

\WR{Warum Nuklearwaffen?}, fragte er vor sich hin. \WR{Wir wissen, dass die Shutek über Nullzonen-Sprengköpfe verfügen. Genauso wie wir. Verglichen mit diesen Waffen sind Atombomben ein Witz.}

\par

Hanna, welche die gleiche militärische Ausbildung wie alle Angehörigen der Navy erhalten hatte, antwortete: \WR{Zum einen sind Atombomben billig. Schon vor dem Routenrkrieg konnte sie jede noch so drittklassige Macht herstellen. Zum anderen begrenzen Nullzonen-Sprengköpfe ihre Destruktivität. Kaum ionisierende Strahlung, wo man sie nicht haben will. Fast kein Fallout. Eine saubere Sache. Aber vielleicht wollen die Shutek genau das Gegenteil.}

\par

Diese Möglichkeit war Marco noch gar nicht in den Sinn gekommen. Die Shutek schienen sich grundlegend von der Menschheit zu unterscheiden. Vielleicht störte sie der nuklare Winter und die Verseuchung nicht, die nach einem Angriff wie auf Pollux folgte.

\par

Schließlich gab Marco auf, darüber nachzudenken und stellte stattdessen eine Frage, die Hanna Moyér ohne Zweifel beantworten konnte. \WR{Wieso ich? Wer hat mich vorgeschlagen? Ich meine, ich bin ein gescheiterter Forscher ohne richtige Arbeit.}

\par

\WR{Du hast eine richtige Arbeit}, antwortete seine Gesprächspartnerin. \WR{Du bist Lehrer. Das ist eine wirklich wichtige Aufgabe.}

\par

\WR{Du weißt, wie ich das meine.}

\par

Hanna blieb eine Weile lang ruhig. Schließlich entschloss sie sich, die Karten auf den Tisch zu legen. \WR{Ich es nicht sicher wissen~-- dafür fehlen mir die nötigen Abzeichen. Aber ich würde sagen, du bist ein Sündenbock.}

\par

Marco lächelte bitter. \WR{Klar. Über Außerirdische zu spekulieren ist eine sehr unsichere Sache. Selbst, wenn man sie, wie wir nun, vor sich hat. Ich bin nicht der einzige kluge Kopf, in diesem Projekt. Aber wenn irgendetwas schief geht… Wenn falsche Entscheidungen aufgrund unserer Empfehlungen getroffen werden…}

\par

\WR{Dann rollt \textit{dein} kluger Kopf}, schloss Hanna.

\par

Marco sah gedankenverloren in den Raum. \WR{Dann mache ich besser keine Fehler mehr.}

\par

Moyér hatte seine Antwort nicht wirklich mitbekommen. Auf ihrem Handcomputer wurde eine Nachricht von der Kommandozentrale der \EN{Minerva} angezeigt. \WR{In Ordnung}, brachte sie hervor und klang dabei wie jemand, der seine Anspannung mit aller Gewalt zu verstecken versuchte. \WR{Ich muss dich jetzt leider alleine lassen. Das gesamte Militärpersonal wird gerade auf Gefechtsstationen beordert.}

\par

\WR{Hatten wir das nicht schon?}, fragte Marco. \WR{Wir haben den Shutek den Krieg erklärt, oder?}

\par

Derweil er noch sprach, suchten Hannas Augen bereits fieberhaft das leere All ab, so, als erwarte sie jeden Augenblick einen Feuersturm aus dem Nichts heranrollen. \WR{Ich fürchte, es ist noch etwas akuter, als das. Die Shutek sind hier.}
