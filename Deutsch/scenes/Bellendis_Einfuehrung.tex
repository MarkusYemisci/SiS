Eine weitere Woge des Beifalls schwappte über das Auditorium Maximum der Sina-Treystan-Universität hinweg. Man hatte sich sehr viel Mühe dabei gegeben, dem sonst recht schmuklosen und schlichten Saal einen festlichen Anstrich zu geben. Vor den beiden großen Fensterfronten, welche die Räumlichkeit einfassten, waren etliche Buchsbäume in aufwendig hergestellten Töpfen aufgereiht worden. An diesem Abend schafften sie es sogar, von der tristen Landschaft um den Campus herum abzulenken. Dies allerdings nur, weil die sonst gleißend helle Sonne bereits untergegangen war.

\par

Als die Universität gegründet worden war, hatte man beschlossen, sie nicht, wie die größte Stadt des Planeten~-- Neu Kazan~-- mit einer transparenten Kuppel zu überziehen. Es hätte zu unnatürlich gewirkt, lautete die offizielle Erklärung zu dieser Entscheidung. Allerdings war Mutter Natur auf Wega Primus nicht zu Hause. Die Welt war im Wesentlichen ein einziges Ödland. Die Nord- und die Südhalbkugeln unterschieden sich lediglich durch die Farbe des Sandes, die langsam aber stetig von einem knochenfarbenen Grau in ein schmutziges Ockergelb überging.

\par

Lediglich die Gebirgsketten boten ein wenig Abwechslung. Die Lufttemparatur nahm auf Meeresniveau Werte zwischen zehn bis zwanzig Grad Celsius an. Auf den Bergen jedoch sank sie an einer Höhe von etwa tausend dreihundert Metern rapide unter Null. Der Schnee blieb fast das ganze Jahr über liegen und verdeckte das sonst so dreckig wirkende Gestein. Nur im sogenannten Sommer fegten heftige Gewitter über die nördlichen Regionen und wuschen den Schnee kurzfristig ab.

\par

Einheimische Vegetation war nur recht spärlich gesät. Die verbreitetste indigene Baumart wirkte, als hingen zerfetzte Plastiktüten in ihren Wipfeln. Allerdings handelte es sich dabei um Membranen, welche die Luftfeuchtigkeit aufnehmen und der Mutterpflanze zur Verfügung stellen konnten.

\par

Marco Bellendi war der erste gewesen, der entdeckt hatte, dass es sich dabei um einen Symbionten und kein Pendant zu den Blättern oder Nadeln irdischer Vegetation handelte. Er hatte leider auch zu den Unglücklichen gehört, die diese Membranen tatsächlich berührt hatten. Es hatte Tage gedauert, bis seine Hände endgültig von dem glibbrigen Schleim befreit gewesen waren, der von der Pflanze sekretiert wurde.

\par

Das Wega in der Union einen so hohen Stellenwert besaß, hatte zwei Gründe. Zunächst besaß das System nach Kreuzpunkt Primus die meisten Hyperraumknoten und war somit schnell zu einem bedeutenden Umschlagplatz für interstellaren Handel geworden. Der Raumhafen auf der Südhalbkugel zählte zu einem der größten überhaupt.

\par

Zum anderen war Wega Primus der erste, allein für die Industrie besiedelte Planet. Vier riesige Atomosphärenprozessoren der Marke \Wr{Weyland-Burke} filterten Giftstoffe aus der Luft und stellten permanent neuen Sauerstoff her. Bellendi lächelte unwillkürlich wie bitter bei dem Gedanken, dass diese gigantischen Kolosse mehr Mühe damit hatten, die anthropogenen Schadsubstanzen zu eliminieren, als jene, die es vor der Kolonisation in der Atmosphäre gegeben hatte.

\par

Wega Primus war deswegen als Industriestandort interessant geworden, weil der Planet wenig eigenes Leben, das man hätte schädigen könnte, beherbergte und außerdem eine viel geringere Schwerkraft als die Erde aufwies. Lediglich in den menschlichen Siedlungen sorgten Generatoren für künstliche Gravitation. In den Produktionsanlagen konnte ein kräftiger Arbeit notfalls selbst einen kleineren Stahlträger anheben.

\par

Das ein Biologe wie Marco Bellendi seinen Weg nach Wega Primus gefunden hatte, war eher ungewöhnlich. Zwar gab es eine mikrobiologische Fakultät an der Universität, doch diese besaß so wenig Rennomé, das man dort kaum freilig arbeitete. Lediglich das Ausbildungszentrum für Maschinenbauer und Hyperarumingenieure konnte im Gesamtvergleich mit dem Unionsstandard mithalten. Genauso wie auf Corna war auch auf Wega die Bildungsoffensive ein reines Prestigeprojekt junger Politiker gewesen, die soch noch nicht profiliert hatten.

\par

Und doch gab Marco Bellendi vor, er interessiere sich brennend für eine offene Stelle an der Fakultät für Exobotanik. Nicht, dass er nicht gerne dort gearbeitet hätte. Allerdings konnte er sich auch etliche Positionen vorstellen, die ihm lieber waren. Beispielsweise der angesehenste Exobiologe an der Shen-Liu-Universätit auf Kreuzpunkt Primus zu sein, wie er es einmal gewesen war. Und nun brachte er Sechstklässlern bei, wie sich Frösche vermehrten.

\par

\WR{Ich bedanke mich vor allen Dingen bei meinen Labormitarbeitern}, sagte die Rednerin, deren Name Marco bereits vergessen hatte, obwohl die ältere Dame nur wenige Minuten zuvor erst vorgestellt worden war. \WR{Ein Projekt wie dieses kann nur gelingen, wenn man ein gut funktionierendes Team hat, dass mit viel Eigeninitiative und guten Ideen mitwirkt.}

\par

Marco hatte nicht einmal eine Ahnung, von welchem Projekt die namenlose Professorin gerade sprach. Schon seit anderthalb Stunden beweihräucherten sich die akademischen Köpfe der Universität selbst. Als Ausklang des akademischen Jahres, das nicht einmal ansatzweise mit dem irdischen übereinstimmte, wurden Ehrungen ausgesprochen, Professoren verabschiedet und Preise verliehen. Eine der langweiligsten Veranstaltungen, die Marco Bellendi an diesem Abend hätte besuchen können. Selbst der Tanzabend in der Spelunke vor seiner viel zu kleinen Wohnung hätte ihn mehr gereizt.

\par

Ungeduldig ließ er seinen Blick durch die Menge schweifen, fand jedoch niemanden von Interesse. Die relevantesten Beziehungen knüpfte man nun mal nicht beim tanzen, versuchte er sich zu sagen. Dass die Rednerin nicht die erste in einer langen Reihe war, nicht die letzte sein würde und sich dessen nicht einmal bewusst schien, half ihm nicht, sich zu entspannen. Genauso wenig, wie seine volle Blase. Hastig schlug er die Beine auf eine neue Art übereinander und fragte sich, ob er seinen Sitznachbarn langsam unangenehm auffiel.

\par

Wieder erklang Applaus, als die Rednerin ihren letzten Satz sprach und vom Pult abtrat. Allerdings nur, um sofort wieder zurückzukehren und anzuhängen: \WR{Und \textit{natürlich} gebührt auch ein großes Lob den vielen Studenten!} Marco widerstand fast gewaltsam dem Drang, sich seine Hand gegen die Stirn zu schlagen. \WR{Durch ihren enthusiastischen Einsatz haben wir es geschafft, viel mehr zu leisten, als im Vorjahr und endlich den verdienten Lohn unserer Bemühungen einzustreichen.}

\par

Am liebsten hätte Marco der Proffesorin entgegen gerufen, dass vermutlich kaum ein Student anwesend war, der in ihrer Fakultät arbeitete. Viel wahrscheinlicher war es, dass diese gerade die Pflanzen rauchten, die sie sonst nur im Labor züchteten.

\par

Die anwesenden Studierenden gehörten vermutlich zu den Maschinenbauern und versuchten~-- wie Marco ebenfalls~-- Eindruck bei den richtigen Personen zu schinden. Allerdings hatten sie die besseren Karten im Vergleich zu ihm.

\par

Marco versuchte, sich nicht über die selbstherrlichen Professoren und anderen Universitätsmitarbeiter zu ärgern. Genau genommen hatten sie den Applaus redlich verdient, den sie gerade ernteten. Allerdings strapazierten sie seine Nerven, indem sie seinen Plan zur Wiederbelebung seiner Karriere so vehement zu einem Geduldsspiel werden ließen.

\par

Keiner der Redner überzog seine Zeit, um ihn zu ärgern. Sie alle freuten sich vermutlich schon seit einiger Zeit auf diesen Ballabend, was bereits an den aufwändigen Kleidern und Gehröcken zu erkennen war, gegen die Marcos fast abgewetzt wirkte.

\par

Zu Beginn der Veranstaltung hatte Marco noch versucht, zuzuhören und etwas von den Vorträgen mitzunehmen. Einige der Ansprachen behandelten auch tatsächlich interessante Themen und viele verwendeten ausgeklügelte Hologramme zur Darstellung der berichteten Sachverhalte. Doch mit der Zeit konnte seine Aufmerksamkeit nur sinken. Insbesondere, da er sich gedanklich bereits auf sein Gespräch mit dem Leiter der biologischen Fakultät vorbereitete, dass er hoffte, während des informellen Teils des Balls führen zu können.

\par

Es war seine beste Chance. Der Ball war öffentlich. Ansonsten hätte man ihm vermutlich die Teilnahme verwehrt. Aber so half er zumindest, den Saal voller aussehen zu lassen.

\par

Später hatte er dann beschlossen, sein Zeitempfinden zu trainieren, indem er die Dauer der Vorträge schätzte, ohne auf die Uhr zu sehen. Doch selbst das hatte er früher als später aufgegeben.

\par

Eine weitere Dreiviertelstunde verging, in der Marcos Hände vom Applaudieren zu schmerzen anfingen, bis der Dekan die erlösenden Worte sprach und das Buffet eröffnete. Marco sprang sofort auf und bewegte sich in Richtung der Toiletten. Den einen Kaffee, den er vor dem Ball zum munter machen getrunken hatte, bereute er seit Stunden zutiefst. Doch die Schlange vor den stillen Räumen war länger als er befürchtet hatte. Viele Frauen und Männer standen bereits an. Im selbe Maße wuchs die Traube an Gesprächspartnern um die wichtigen Persönlichkeiten dieses Abends. Inklusive Professor Malik Murough, dem Inhaber des Biologischen Lehrstuhls der Universität. Wenn einer Marco Bellendi einstellen würde, dann er. Murogh legte kaum Wert auf den Ruf eines Mitarbeiters, lediglich auf seine Leistung.

\par

\WR{Einen Sekt?} Die Frage des Kellners, der mit einem Silbertablett voller Gläser mit Perlwein bewundernswert sicheren Fußes durch die Menge glitt, erschien Marco in diesem Fall fast wie Hohn. Er lehnte dankend ab und ging auf Professor Murogh zu. Dabei fest hoffend, dass seine Blase aus dem selben Material wie ein moderner Fuball bestand.

\par

Ein knapper Blick in sein Buch, dass ihm wiederum eine Aufnahme seines Gesichtes zeichnete, ließ ihn zumindest etwas ruhiger werden. Er sah nicht so schlimm aus, wie er es zunächst befürchtet hatte. Was sein Äußeres betraf, war er von Anfang an nicht besonders unsicher gewesen. Schon in der Mittelstufe hatte er etliche Verehrer und Verehrerinnen gehabt. Und dies sicher nicht wegen seiner guten schulischen Leistungen.

\par

Nur, dass seine Locken schon wieder durchschienen, ärgerte ihn. Friseure waren für jemanden mit seinem Budget nicht die oberste Priorität. Er nahm sich vor, sich ein andermal über die Verteilungspolitik der Union zu ärgern. Nahrungsmittel, Kleidung, Unterrichtsmaterialien und sogar Personennahverkehr waren allesamt kostenlos. Aber für eine gute Frisur musste er selbst aufkommen.

\par

Bereits während er der Gruppe um Professor Murogh näherkam, waren die meisten Blicke auf ihn gerichtet. Jeder kannte den Ruf, der Marco Bellendi vorauseilte und obwohl sich keiner etwas anmerken lassen wollte, konnte dieser doch erkennen, dass er ihnen schon aus dieser Entfernung unangenehm und peinlich war. Lediglich Murogh selbst sah ihm fest und bestimmt in die Augen.

\par

Glücklicherweise hatte sein Eintreffen die laufenden Gespräche unterbrochen, so dass er sich nicht der schwierigen Herausforderung stellen musste, sich in eine Gruppe zu integrieren und brav zu warten, bis er die Chance hatte, etwas zu sagen und sich dabei nicht durch das Gefühl, ein Fremdkörper zu sein, wieder vertreiben zu lassen.

\par

Seine Verbeugung vor den Anwesenden wirkte geschickt. Umgangsformen waren niemals sein Problem gewesen. \WR{Guten Abend, Professor}, grüßte Bellendi. \WR{Eine eindrucksvolle Rede. Sie haben Ihre Fakultät wirklich in Schuss gebracht.} Einer der Beisteher wandte seinen Blick ab und schien mit einem Seufzen zu ringen.

\par

\WR{Danke}, erwiderte Murogh kurz angebunden.

\par

Marco wollte gerade weitersprechen, als eine neue Person zu der Gruppe hinzutrat. Der Neuankömmling hatte bereits eine Halbglatze und sein weißer Vollbart wirkte genauso durchschnittlich gepflegt, wie sein roter Pullover, der farblich weder zum Rest seiner Kleidung, noch zum Anlass oder Ambiente des Abends passte. Doch Herold Lesc-Bublé brauchte niemanden mehr zu beeindrucken. Zumindest nicht durch Äußerlichkeiten. Seinen überwältigend guten Ruf hatte er sich durch die Moderation mehrerer DDV-Sendungen mit mehr oder weniger wissenschaftlichem Hintergrund erarbeitet.

\par

Zwar arbeitete er Hauptberuflich als Lehrbeauftragter für Exobiologie, versuchte aber vielmehr den Begriff abzuschaffen, als tatsächlich etwas Neues auf diesem Gebiet zu entdecken. Marco Bellendi schien es oft so, als betreibe er seine Forschung nur, um das ganze Feld zu entzaubern und ihm seine medienwirksame Faszination zu nehmen.

\par

Bereits seit die Menschen interstellare Reisen unternahmen, hatte sich eine Frage dringender denn je gestellt. War die Spezies \Wr{Homo sapiens sapiens} die einzig intelligente Art im Universum oder gab es andere Wesen, die ebenfalls einen höheren Entwicklungsstand erreicht hatten. In der ganzen Union war die Erde bislang der einzige Planet mit einem nennenswerten Artenreichtum geblieben. Der Kreuzpunktadler war das größte Lebewesen, das jemals anderswo als auf Sol Primus beschrieben worden war.

\par

Vieles sprach gegen die Existenz von intelligentem außerirdischen Leben. Insbesondere das Ausbleiben jeglicher Signale extraterrestrischen Ursprungs. Dennoch hatte Marco Bellendi nie aufgehört zu hoffen. Lesc-Bublé hingegen versäumte keine Gelegenheit, öffentlichkeitswirksam alle Belege für die \Wr{Seltene-Erde-Hypothese} darzulegen.

\par

\WR{Professor Murogh wird sich sicher über \textit{Ihre} Meinung freuen, Herr Bellendi}, sagte er schließlich.

\par

Zum ersten mal wünschte sich Marco, er wäre in einem Dojo, statt einer Festhalle. Doch er konnte es sich nun nicht leisten, sich seinen Ärger anmerken zu lassen. Und dieser war groß. Lesc-Bublé war nur seinetwegen hier, das war ihm klar. Eine Veranstaltung wie der Abschlussball eines akademischen Jahres auf Wega konnte ihn einfach nicht interessieren. Aber er hatte wohl von der freien Stelle an der biologischen Fakultät erfahren und sich denken können, wer der erste Bewerber wäre.

\par

\WR{Wie machen sich Ihre Schüler?}, fragte Lesc-Bublé.

\par

\WR{Hervorragend.} Bellendi zwang sich, so seinen Gesichtsausdruck nicht widerspiegeln zu lassen, wie er sich fühlte. \WR{Ich muss sie allerdings daran erinnern, nicht zu viel DDV anzuschauen. Das ist schlecht für ihre Fantasie und ihre Entwicklung.}

\par

Der Mann in dem roten Pullover positionierte sich demonstrativ zwischen Bellendi und Murogh. \WR{Fantasie ist genau das, was Sie Ihren Schülern beibringen können. Passen Sie nur auf, dass Sie das Bildungsniveau nicht zu sehr senken.}

\par

Eine derart offene Anfeindung hatte Marco nicht erwartet~-- zumindest nicht ein einer direkten Konversation. Bei Ansprachen im DDV hatte Lesc-Bublé allerdings noch nie ein Blatt vor den Mund genommen. Nun galt es, sich nicht provozieren zu lassen. Darum ignorierte er den Einwurf schlichtweg und wandte sich erneut an Murogh: \WR{Professor, mir ist zu Ohren gekommen, dass es in der Fakultät für Mikrobiologie bald eine freie Stelle geben wird. Es ist zwar nicht ganz mein Bereich…}

\par

\WR{Richtig}, fuhr Lesc-Bublé dazwischen. \WR{Ihr Bereich liegt ist eher die Fotografie. Insbesondere dem Aufpeppen von ansonsten eher unspektakulären Bildern.}

\par

Bellendi sprach unbeirrt weiter. \WR{… aber ich denke, es ist ohnehin an der Zeit, mich beruflich zu verändern. Ich würde mich sehr freuen, wenn Sie meine Bewerbung entgegennehmen würden.}

\par

Dem Mann im roten Rollkragenpullover schien es gar nicht zu gefallen, einfach ignoriert zu werden, weswegen er sich entschloss, die Schärfe seiner Anfeindungen auf eine neue Ebene zu bringen. \WR{Vielleicht war ich nicht deutlich genug. Keine Universität mit einem Rest an Selbstachtung wird einen Betrüger wie Sie einstellen. Wissenschaft ist nicht das Ihre. Weder im Mikro- noch im Exobereich. Überlassen Sie das lieber Leuten mit mehr Sachverstand.}

\par

\WR{Ich denke nicht, dass Sie hier arbeiten oder sonst irgendwie mit den Personalentscheidungen dieser Einrichtung zu tun haben.} Marco Bellendi ärgerte sich sofort darüber, laut geworden zu sein. Lesc-Bublés Lächeln zeigte ihm, wie sehr er sich darüber freute, seinen Widersacher in Rage gebracht zu haben.

\par

\WR{Da haben Sie Recht}, antwortete dieser. \WR{Ich bin nur ein aufmerksamer Mahner.}

\par

Professor Murogh ahnte, dass ein Eklat im Anmarsch war, der die Stimmung des Abends in Gefahr bringen konnte. Darum sagte er: \WR{Vielen Dank. Aber ich denke, wir wissen alle über die relevanten Einzelheiten aus Herrn Bellendis beruflicher Laufbahn Bescheid und müssen nicht permanent daran erinnert werden.}

\par

Marco sah seine Chance gekommen und schloss an: \WR{Ich versichere Ihnen, ein Vorfall wie der von Persei zwei wird sich nicht wiederholen. Seit damals habe ich viel…}

\par

Nun war es an Murogh, zu unterbrechen. \WR{Herr Bellendi. Leider muss ich Ihnen sofort absagen. Wir glauben, dass es besser ist, die offene Stelle mit jemandem zu besetzen, der etwas jünger ist. Es ist vor allem eine Position, auf der man viel lernen kann. Sie hingegen wären hoffnungslos unterfordert. Ich wünsche Ihnen aber alle Gute auf Ihrer Suche.}

\par

Ein unwillkürliches Seufzen folgte. Marco hatte keinen Zweifel daran, dass die Person, welche die Stelle letzten Endes antreten würde, auch gute zwölf Jahre älter sein konnte, als er selbst. Auch wenn es Murogh niemals zugeben würde, war es sein Ruf, der ihm nach wie vor anhing. Eine so prompte Abfuhr ließ jede Hoffnung schwinden, irgendwann wieder in der wissenschaftlichen Welt Fuß fassen zu können. Und sei es nur als Laborant, der halbtags arbeitete.

\par

\WR{Vielen Dank, trotzdem}, zwang sich Marco Bellendi zu sagen. \WR{Es ist schon spät und ich habe noch einige Klassenarbeiten zu korrigieren. Darum sollte ich jetzt besser nach Hause gehen.}

\par

Murogh nickte bedächtig, während Lesc-Bublé zwar leise, aber dennoch gut hörbar murmelte: \WR{Das ist das erste Intelligente, was er heute von sich gegeben hat.}