Vivien drehte sich noch einmal um. Trotzig ließ sie ihre Tasche auf den Boden des Stardecks donnern. Ihr vorwurfsvoller Blick und die traurigen Gesichter seiner Kinder ließen Nicos Herz beinahe einfrieren.

\par

\WR{Ich kann nicht glauben, dass du das tust}, sagte seine Frau. Sie sah nicht zu ihrem Sohn oder ihrer Tochter, denn sie hatte sich vorgenommen, niemals vor ihnen mit ihrem Mann zu streiten.

\par

Nico verschränkte die Arme vor der Brust. \WR{Schau, ich schulde es diesen Leuten. Und wer weiß. Bei meiner Vergangenheit nehmen sie mich eventuell gar nicht.}

\par

Vivien verdrehte die Augen. \WR{Du hast es mir versprochen. Du wirst nie mehr ins Cockpit eines Jägers steigen.}

\par

Von denen gab es im Hangardeck der \EN{Regenvogel} mehr als genug. Einer der windschnittigen Falken wurde gerade repariert. Mitten auf dem Flugdeck flogen die Funken eines Schweißgeräts, was davon zeugte, wie überlastet die Wartungsbuchten waren.

\par

\WR{Es tut mir leid.} Nico ließ seine Arme sinken und hob sie dann in einer Geste der Entschuldigung. \WR{Ich weiß, du kannst dir das nicht vorstellen. Aber ich muss es einfach tun. Ich bin Pilot, kein Landwirt. Das habe ich gelernt und das kann ich am besten. Du hast es doch selbst gesehen. Was auf Pollux passiert ist, zeigt, was uns bevorsteht. Diese Angreifer. Es braucht jetzt Profis, die sie zurückschlagen.}

\par

\WR{Du hast eine Familie!}, rief Vivien wütend. \WR{Wer kümmert sich um uns, wenn du…} Besonders jetzt vermied sie es, ihre Kinder anzusehen.

\par

Nico kannte seine Frau gut. Sie besaß außergewöhnlich gute Qualifikationen. Falls er fallen sollte, könnte sie für sich und vier andere sorgen. Doch das jetzt zu erwähnen, wäre zwar eine rationale aber sicher keine sehr einfühlsame Antwort gewesen. Statdessen versucht er, Vivien in den Arm zu nehmen. Sie wich zurück, doch er ließ nicht locker. Schießlich fanden sich zuerst ihre Hände und dann ihre Lippen.

\par

\WR{Hier ist ein Versprechen, das ich nicht brechen werde}, begann er. \WR{Ich komme zurück. Egal, was passiert. Irgendwann wird das alles zu Ende sein. Und dann setzen wir uns zur Ruhe.}

\par

Vivien verkniff sich eine Träne. Ihre Hände schlossen sich enger um die ihres Mannes. \WR{Warum? Warum du?}

\par

\WR{So läuft das Geschäft}, erwiderte er lächelnd. \WR{Wenn ich wirklich bei der Starforce mitfliegen darf, dann fragt niemand mehr danach, was früher gewesen ist. Wenn man uns schon nicht mit Reichtümern überschüttet, dann wird man garantiert dafür sorgen, dass ich endgültig rehabilitiert werde.}

\par

\WR{Sorg' dafür, dass das nicht posthum passiert!}, forderte sie. \WR{Du schuldest diesen Leuten nichts. Deine Schuld ist schon seit Jahren beglichen. Wehe, wenn ich herausfinde, dass du das das alles nur machst, um ein Held zu werden.}

\par

Nico lächelte verschmitzt. \WR{Höchstens um deiner zu werden. Aber ich weiß doch, dass ich das schon bin.}

\par

In Viviens Gesichts arbeitete es sofort. Sie biss die Zähne zusammen und ihr Mann wusste, dass er kurz davor stand, entweder eine Ohrfeige oder einen Kuss zu bekommen. Zu seinem Schluss entschied sich Vivien für letzteres.

\par

\WR{Ich liebe dich}, sagte er, als sich ihre Lippen wieder trennten.

\par

Nun war es an Vivien, verschmitzt zu grinsen. \WR{Und ich dich. Aber glaub mir, wenn du dir auch nur ein Haar krümmen lässt, bring ich dich um.}

\par

\WR{Schon in Ordnung Schatz}, begann er. \WR{Du brauchst mir keinen Mut zu machen. Ich schaff das schon.}

\par

Sie deutete auf ihre Manteltasche. \WR{Ich werde dich anrufen. Und es ist mir völlig egal, welches dämliche Militärprotokoll vielleicht dagegen spricht. Wenn du nicht abnimmst, dann gibt's Ärger.}

\par

Mit diesen Worten, nahm Vivien wieder ihre Tasche und warf sie sich über die Schulter. Nachdem Nico auch seine beiden Kinder gedrückt hatte, nahm sie sie an die Hand und stieg in die wartende Fähre ein. Selbst die wenigen Flüchtlinge von Pollux hatten die Möglichkeiten der stark angeschlagene \EN{Regenvogel} belastet. Aber glücklicherweise war die Bereitschaft zur Hilfe in der Union überall sehr groß. Vivien hatte sich bereits eine Bleibe auf einem Habitatring, dass um Proxima III kreiste organisiert. Sie würde es ohne größere Probleme schaffen, darüber war sich Nico sehr sicher.

\par

Nach wenigen Minuten hob der Transporter ab und machte wieder Platz für die Kampfflieger, die auf eine Reparatur warteten. Nico betrachtete den Abflug wehmütig vom Aussichtsbereich des Bereitschaftsraums aus. Anna Farley stand wortlos neben ihm.

\par

Ein Teil von ihm hoffte, dass sein Antrag nicht angenommen werden würde. So wäre er schnell wieder bei seiner Familie und könnte sich mit ihnen eine neue Existenz aufbauen. Erst dann würde er überhaupt damit anfangen können, um die Freunde zu trauern, die er auf Pollux verloren hatte.

\par

\WR{Zunächst einmal möchte ich Ihnen für Ihre Bereitschaft danken}, sagte Major Farley. \WR{Und ich werde mein möglichstes tun, um sie in die Pilotenriege aufnehmen zu lassen. Aber sie müssen mir glauben, dass das nicht einfach wird. Ihre Vorstraft…}

\par

Nico hob abwehrend die Hand. \WR{Ich sollte \textit{Ihnen} danken. Ich verstehe, dass sie nicht mehr als ihr Bestes tun können.}

\par

Farley nickte. \WR{Glaube Sie denn, Sie kommen mit diesen Jägern klar? Es ist eine Weile her, seitdem sie zum letzten mal in einem Cockpit eines Kampffliegers gesessen sind. Und seither hat sich viel verändert.}

\par

\WR{Habe ich Zugriff auf den Simulator?}

\par

\WR{Nein}, war Farleys sofortige Antwort. \WR{Aber Sie können ihn benutzen. Ich öffne ihn für Sie. Brauchen Sie sonst noch etwas?}

\par

Nicos Blick fiel auf eine Gruppe von Piloten, die auf ein paar Kisten im Hangar zusammensaßen und miteinander zu scherzen schienen. \WR{Ja. Ich würde gerne Kontakt zu ein paar der Mädels und Jungs knüpfen. Sie werden es bald wissen. Und wenn es hart auf hart kommt, werde ich auf die Hilfe meiner Flügelmänner angewiesen sein. Bis dahin sollte ich einen guten Eindruck hinterlassen haben.}