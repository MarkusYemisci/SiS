Kreuzpunkt Primus war jener bekannte Planet, welcher der Erde noch am nächsten kam. Während die meisten Welten der Union entweder karg und öde oder durch Terraforming wenigstens annähernd erdähnlich gemacht worden waren, bot die ehemalige Hauptwelt des Commonwealth von sich aus eine beeindruckende Flora und Fauna. Kreuzpunkt Primus war auch der einzige Planet, der eine nennenswerte Tierwelt vorzuweisen hatte. Der Kreuzadler, ein vogelähnliches Wesen ohne Federn aber mit einer Flügelspannweite von guten zweieinhalb Metern, galt als die Berühmtheit der Fauna.

\par

Das Leben auf Kreuzpunkt Primus konzentrierte sich auf die äquatornahen Regionen, da sich die Welt in einer Eiszeit befand und sich die Pole weit ausgedehnt hatten. Im Wesentlichen gab es zwei dicht besiedelte Kontinente, die jedoch genügend Lebensraum und Nahrung zur Verfügung stellten, dass sich Kreuzpunkt Primus problemlos selbst versorgen konnte.

\par

Auch aus diesem Grund hatte die Liga der Wissenschaft gerade diesen Planeten gut zweihundert Jahre zuvor als Basis für ihre Tiefenraumforschung ausgewählt. Aber vor allem die vielen Hyperraumrouten, die dem System seinen Namen gaben, hatten die Forscher von damals angelockt. Das Kreuzpunkt System verfügte über durch seine einzigartige Planetenbewegung über neun vermessene, permanent nutzbare Knoten und einige weitere, die jedoch nur zeitweise offen waren.

\par

Die Wissenschaftsliga hatte aber nicht vorausgeahnt, dass Kreuzpunkt Primus schon bald seine eigene Regierung bilden und anfangen würde, die umliegenden Systeme zu besiedeln~-- die Geburt des Commonwealth.

\par

Kreuzpunkt lag im äußersten Ring, also zwischen zwanzig und dreißig Lichtjahren von der Sonne entfernt. Dazwischen gab es den zweiten Ring, den das Commonwealth und die Irdische Allianz unter sich aufgeteilt hatten. Dabei war ersteres deutlich erfolgreicher, da die vielen Hyperraumrouten des Kreuzpunkt-Systems eine wesentlich schnellere Ausbreitung ermöglichten. So gingen fast alle Welten, deren Ressourcen die Herstellung von Capezin ermöglichten an das Commonwealth und da es damals noch keine Plasmatriebwerke gegeben hatte, bekam die Irdische Allianz ihre liebe Not, Treibstoff für Raumschiffe zu produzieren.

\par

Kreuzpunkt Primus, zunächst belächelt als \Wr{Schneeball des Universums}, zeigte sich nun als die aufstrebende Welt des frühen vierundzwanzigsten Jahrhunderts schlechthin. Das Commonwealth strebte danach, seine Überlegenheit gegenüber der alten Erde unter Beweis zu stellen und das neue Zentrum der Menschheit zu werden. Obwohl lediglich die wohlhabenderen Bürger Kreuzpunkts diese Einstellung mittrugen und ihre Welt als reiner ansahen, da sie niemals von der Seuche heimgesucht worden war, geriet die Irdische Allianz schnell in Bedrängnis.

\par

Doch längst nicht jeder Bürger des Commonwealth stand hinter seiner Regierung. Besonders die industrielastigen Welten erlitten immer wieder Engpässe jeder Art. Arbeitslosigkeit und sogar Hunger waren die Folge. Als Aufstände ausbrachen reagierte Kreuzpunkt Primus mit Repression.

\par

Corna wurde bald das Zentrum des Widerstands und erhielt Unterstüzung durch die Irdische Allianz, die darauf hoffte, ihren Widersacher von innen heraus mürbe machen zu können. Gegen Ende des Jahres zweitausend sechshundert achtundzwanzig begann der Routenkrieg. Das Commonwealth hatte einen Konvoi, der versucht hatte, die Rebellen von Corna mit Waffen zu versorgen, abgefangen und vernichtet. Auf der Erde, war eine Kriegserklärung schnell unterzeichnet worden, da man hoffte, dem Commonwealth wenn schon nicht wirtschaftlich, dann wenigstens militärisch Herr zu werden. In diesem Jahr war der Winter über Belgrad~-- dem damaligen Regierungssitz~-- sehr früh hereingebrochen. Schon im September war der erste Schnee gefallen.

\par

Nach dem Krieg war Kreuzpunkt Primus ein wichtiges Standbein bei der Gründung der Unio Terrae geworden. Die Regierung jener Welt hatte schnell erkannt, dass ihr Planet seine Bedeutung auf Dauer nur halten konnte, wenn er sich für die Union durch mehr als seine zahlreichen Hyperraumrouten unabdingbar machen würde, denn die konnten notfalls auch umflogen werden. So hatte man sich darauf geeinigt, dass zwar die administrativen Institutionen auf der Erde verbleiben würden, aber das wirtschaftliche Zentrum seinen Sitz auf der ehemaligen Zentralwelt des Commonwealth beziehen sollte. Wenige Jahre nach der globalen Etablierung der Union hatten bereits die Staatsbank, die Börsenaufsicht und etwa zwei Drittel aller Unternehmen ihre Hauptsitze auf Kreuzpunkt Primus aufgebaut. Und als zweitausend sechshundert vierzig die Inbetriebnahme des Permutare beschlossen worden war, hatte es niemanden wirklich überrascht, dass er in der Nähe von Prosperity Crosspoint, Kreuzpunkts Hauptstadt, gebaut werden sollte.

\par

Nicht nur als wirtschaftlicher Standort, sondern auch als wichtigster Stützpunkt des äußeren Rings hatte sich der Planet behauptet. Die Raumstation \EN{Minerva} war noch vor dem Routenkrieg erbaut worden und bis zur Fertigstellung der \EN{\EN{Athena}} in der Erdumlaufbahn die größte Festung der Union im All gewesen.

\par

Marco Bellendi durchquerte unsicher ihre langen Korridore. Nachdem Hanna Moyér ihn vor einem Tag auf Wega rekrutiert hatte, war ihm kaum Zeit geblieben, sich zu rasieren und saubere Kleidung anzuziehen. Nun trug er die herkömmliche Freizeitgarderobe der Flotte. Ein dunkelblaues T-Shirt mit dem Emblem der Navy darauf und eine viel zu weite Hose. Frische Klamotten anzuziehen war für Marco auch bitter nötig geworden, nachdem er sich auf dem Flug nach Kreuzpunkt so oft übergeben hatte, dass er selbst aufgehört hatte, mitzuzählen. Dabei war er sich nicht einmal sicher gewesen, ob es ihm besser ergangen wäre, hätte er sich nicht tags zuvor dem guten Wein zugetan.

\par

Nun war er auf der Suche nach Besprechungsraum K dreidutzend sechs, von dem er nicht die leiseste Ahnung hatte, wo er sich wohl befinden konnte. Hanna Moyér hatte ihm nur gesagt, dass man ihn dort erwarte, um eine generelle Lagebesprechung abzuhalten und die Analyse der spärlichen Daten, die über die Angreifer zur Verfügung standen, zu koordinieren. Marco wusste nicht, wer sonst noch dort sein würde oder was man überhaupt von ihm erwartete.

\par

Ihm war immer noch recht übel, obwohl er Zeit gehabt hatte, sich eine Weile lang ins Bett zu legen. Schwer atmend sah er aus dem nächstbesten Fenster. \EN{Minerva} schwebte zurzeit über Kreuzpunkts Nachtseite. Die großen Metropolen der bewohnbaren Äquatorregion strahlten ihr helles Licht ins All hinaus. Beleuchtete Straßen und Parks erschienen aus der Umlaufbahn betrachtet wie Adern und Venen der Städte. In der Ferne blinkten die Positionsleuchten eines großen Raumschiffs. Marco Bellendi kannte sich nicht gut genug mit der Navy aus, um sagen zu können, zu welcher Gattung es gehörte. Er konzentrierte sich lediglich auf die periodisch aufleuchtenden Lampen, um sich von seinem Unwohlsein abzulenken. Eine bekannte Stimme ließ ihn herumfahren.

\par

\WR{Hier sind Sie}, rief Lieutenant Moyér. \WR{Alle warten schon auf Sie. Wo sind sie gewesen?}

\par

Der ehemalige Wissenschaftler begegnete ihrem eine Antwort fordernden Blick mit Schulterzucken. \WR{Es wäre hilfreich gewesen, wenn Sie mir gesagt hätten, wie ich zu diesem Besprechungsraum komme.}

\par

Die junge Frau seufzte und deutete hastig auf einen Computerzugang an der Wand des Korridors. Als hätte sie es schon hundert mal gesagt, erklärte sie: \WR{Hier hätten Sie sich einen Lageplan anzeigen lassen können.}

\par

\WR{Hätte ich}, erwiderte Bellendi. \WR{Wenn der Computerzugang für mich nicht gesperrt wäre.}

\par

Hanna Moyér blieb einen Moment ruhig, dann fuhr sie fort: \WR{Wie auch immer. Folgen Sie mir einfach. Ich bringe Sie hin.}

\par

Die beiden gingen eine Weile lang wortlos nebeneinander her. Marco war sich nicht sicher, ob das Schweigen seiner Begleiterin von ihrer militärischen Spartanität herrührte oder ob sie einfach nicht mit ihm sprechen wollte. Da verwunderte es ihn umso mehr, als sie plötzlich fragte: \WR{Warum wurden sie gefeuert, Herr Bellendi?}

\par

Marco blieb abrupt stehen und sah Moyér völlig überrascht in die Augen. Einen Moment lang. glaubte er, sie wolle ihn auf den Arm nehmen, doch dann bemerkte er, dass sie die Antwort auf ihre Frage tatsächlich noch nicht zu kennen schien.

\par

\WR{Ich dachte die ganze Welt wüsste davon}, sagte er mehr an sich selbst gerichtet. \WR{Haben Sie keine Hintergrundrecherche über mich angestellt, als sie mich rekrutiert haben?}

\par

Lieutenant Moyérs Augen verengten sich. \WR{Wie schon gesagt, ich habe Sie nicht rekrutiert. Jemand von oben wollte sie haben. Außerdem ist vor wenigen Tagen ein ganzer Planet ausgelöscht worden. Die Streitkräfte sind aufgescheucht und ich hatte besseres zu tun, als mir Kryptaartikel über Sie durchzulesen.}

\par

Es gefiel Marco keineswegs auf diesen Teil seiner Vergangenheit angesprochen zu werden. Allerdings sah er ein, dass Hanna Moyér sich ihre Frage leicht selbst beantworten konnte. Die Gründe für seinen Rauswurf waren monatelang in so gut wie jedem Medium breitgetreten worden. Wenn er es ihr erklärte, hatte er zumindest die Chance seine Sicht der Dinge zu schildern.

\par

\WR{Sie wissen, dass ich Biologe bin?}, fragte er vorsichtig, worauf seine Begleiterin nur kurz nickte. \WR{Nach meinem Studium auf Wega habe ich angefangen nichtirdische Tierarten zu katalogisieren. Diese Arbeit hat mir unglaublichen Spaß gemacht. Die Untersuchung des Kreuzadlers geht auf die Arbeit meiner Arbeitsgruppe zurück.}

\par

\WR{Kommen Sie bitte zum Punkt, Herr Bellendi}, forderte Moyér, die weiter und weiter auszuschreiten schien.

\par

Der angesprochene seufzte. \WR{Gut. Nachdem ich so viele außerirdische Tiere kennen gelernt hatte, habe ich mich gefragt, ob es vielleicht auch außerirdisches Leben gibt, dass so intelligent ist, wie wir~-- wenn nicht gar intelligenter. Ich bekam die Mittel für ein groß angelegtes Forschungsprojekt zur Suche nach...}, Bellendi unterbrach sich, \WR{na ja, nach Außerirdischen. Wir hatten keine Ahnung, wie wir sie uns vorzustellen hatten aber wir wussten, dass unsere Chancen auf erdähnlichen Planeten am besten standen. Die Entwicklung des Lebens auf der Erde scheint immerhin keine Bedingungen zu erfordern, die man nicht potentiell sonst wo finden kann. Wir kennen ja schon lange nichtirdische Tiere, Pflanzen, Pilze, Bakterien und so weiter.} Hanna Moyér nickte. \WR{Dann war da Iota Persei. Ein Stern, welcher der Sonne recht nahe kommt. Ultrafernerkennung hat schon vor einigen Jahrzehnten einen Planeten in der habitablen Zone entdeckt, der erdähnlich ist. Er war einer unserer besten Kanditaten. Und tatsächlich sahen die Bilder unserer Sonde sehr vielversprechend aus.}