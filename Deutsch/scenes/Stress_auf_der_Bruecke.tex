\WR{Ist die \EN{Artiglio de Leone} wieder flugfähig?}, wollte Captain Fiscale wissen, während sie sich über die Schulter ihres Kommunikationsoffiziers beugte.

\par

Wallander presste sich seinen Kopfhörer ans Ohr, um die Übertragung gegen das Dröhnen der Geschütztürme besser hören zu können. \WR{Ja, Madam}, fasste er den Bericht zusammenzufassen, \WR{Eine Notbesatzung hat eine Ersatzbrücke im Kontrollraum des Flugdecks errichtet. Ein Riekaan Marques ist jetzt der Captain. Er möchte sich mit ihnen besprechen, um sich für den Gegenangriff zu koordinieren.}

\par

Fiscale sah aus dem Fenster, wie sie es schon oft getan hatte. Kreuzpunkt Primus nahm nun einen gewaltigen Teils ihres Gesichtsfelds ein. Er sah so anders aus, als die Erde, an die sich sich lebhaft erinnerte, auch wenn sie schon seit einer langen Zeit nicht mehr dort gewesen war. Obwohl an vielen Stellen der tiefblaue, fast purpurne, Ozean zu erkennen war, waren weiß und grau die dominierenden Farben. Weiße Wolken und ein grauer Grund. Durchzogen von goldenen Flecken und Linien, die von den Lichtern der Städte stammten.

\par

In einiger Entfernung war nun mittlerweile auch die \EN{Artiglio de Leone} zu sehen, deren Brückenturm nach wie vor brannte. Die Form des mittleren Trägers war simpel. Das Flugdeck im Wesentlichen ein großer, hohler Kasten, der jedoch Platz für zahlreiche Jäger und auch Nullzonenbomer bot. Allein das bedeutete ein weitaus größeres Offensivpotential, als es die \EN{Regenvogel} aufbieten konnte. Zu diesem käme normalerweise noch ein leistungsstarkes Hautpgeschütz auf den Aufbauten, das der \EN{Artiglio de Leone} aber nun fehlte. Dort war nur noch ein brennendes Loch zu sehen.

\par

Fiscales nächster Blick ging zum Übersichtshologramm, dass dank dem Einsatz einiger gestresster Techniker wieder funktionierte. Commander Samads Leichnam lag jedoch nach wie vor, nur mit einer Uniformjacke zugedeckt, am Rand der Brücke.

\par

Für den Moment hatte die Union die Lufthoheit im Orbit um Kreuzpunkt Primus. Aber nach wie vor stießen Jägerschwadrone der Shutek in die dünnen Reihen der Starforce vor und hinterließen jedes mal abgeschossene Jäger, leckende Schiffe und zerstörte Geschütze. Die \EN{Crossguard} hatte bereits die Hälfte ihrer Kanonen verloren und diente der \EN{Artiglio de Leone} nun als wenig mehr als eine verstärkte Panzerung.

\par

\WR{Ich rede mit ihm, sobald ich kann}, versprach Fiscale an Wallander gewandt. \WR{Sie können ihm aber schon mal sagen, dass wir uns um diese anrückenden Zerstörer und Kreuzer kümmern müssen. Ich würde sogar wetten, dass die sich aufteilen wollen, um Bomben auf den Planeten zu werfen.}

\par

Mit diesen Worten spurtete sie die Treppe hinunter und sprach Elshe Schwarzschild an. \WR{Wie ist der Zustand auf der Oberfläche? Wenn die Verstärkung der Shutek eintrifft, dann müssen diese Silos feuern können.}

\par

Lieutenant Schwarzschild löste ihren Blick von den Binokularen. \WR{Noch sind alle planetaren Verteidigungsanlagen unter unserer Kontrolle, Madam. Und die Front im Norden scheint zu halten. Vermutlich können die Shutek dort wegen der höheren Berge schlechter ihre Truppen landen.

\par

Aber im Süden sieht es schlimm aus. Mehrere Landeschiffe sind dort in einer Ebene beim Spechtgipfel gelandet. Major Henningtons Staffel konnte das nicht verhindern. Akinatown wird evakuiert, was Kräfte bindet. Ich bin keine Expertin, was den Bodenkampf betrifft, aber ich denke, die Shutek erreichen die Silos innerhalb der nächsten drei vier Stunden. Und so wie ich höre, sieht Legat Gajjar das ähnlich.}

\par

Fiscale klopfte der jungen Frau auf die Schultern und ging dann wortlos davon. Wieder sah sie zu dem Übersichtshologramm. Unter ihrem Kommando standen nun fünf Zerstörer, drei Fregatten und zwei Kreuzer. Besonders erstere könnten genügend Feuerkraft entfesseln, um die gesamte Armee der Shutek auszulöschen, die auf Yêxīn zumarschierte.

\par

Das Problem bestand jedoch darin, dass sich selbst Nullzonenbomben nicht kontrolliert genug zünden lassen konnten, um nicht auch noch die Stadt und alle Soldaten und Bewohner mit ihn den Tod zu reißen, die noch dort befanden.

\par

Das Eintreffen von drei Neuankömmlingen auf der Brücke riss Captain Fiscale von diesen Gedanken los. Sie brauchte die drei Piloten nicht einmal böse anzusehen, damit diese unsicher auf den Boden blickten.

\par

\WR{Wen haben wir da. Die Lieutenants Witwer und Wilson.}

\par

\WR{Madam, ich entschuldige mich für das Verhalten, meiner Männer}, begann Anna Farley, sobald die Kapitänin ihren Satz beendet hatte. \WR{Ich erteile Lieutenant Wilson sofort Flugverbot. Dieser Vorfall wird nicht wieder vorkommen.}

\par

\WR{Allerdings wird er das nicht!}, donnerte Captain Fiscale nun direkt an den immer noch auf seine Füße starrende Kevin Wilson gewandt. \WR{Wissen Sie eigentlich, dass ich das Recht hätte, Sie zu erschießen? Wir sind im Krieg und in einer direkten Kampfhandlung verwickelt. Befehlsverweigerungen können unter Umständen Sieg oder Niederlage ausmachen. Was haben Sie sich dabei gedacht?} Kevin sagte nichts. Er rührte sich nicht einmal. \WR{Antworten Sie, verdammt noch mal!}

\par

Schließlich sah Mortens Freund auf. \WR{Meine Lebenspartnerin war in Gefahr. Ich konnte das nicht einfach so passieren lassen. Ich musste ihr helfen.} Er klang so ton- wie kraftlos.

\par

Captain Fiscale schnaubte. \WR{Sehen Sie in diese Ecke!} Sie deutete mit weit ausgestrecktem Zeigefinger auf den toten Körper ihres ersten Offiziers. \WR{Commander Samad liegt schon seit Stunden hier. Er war mein engster Kollege und bester Freund. Und jetzt ist er genauso tot wie ihre Freundin. Aber deswegen setze ich nicht die Sicherheit des Schiffes und meiner Kameraden aufs Spiel.}

\par

Kevin nickte nur. \WR{Ich verstehe, Madam. Ich akzeptiere jede Strafe, die sie für angebracht halten.}

\par

\WR{Oh, Sie akzeptieren sie sogar}, spottete Fiscale. \WR{Ich wusste gar nicht, dass der Angeklagte erst zustimmen muss, bevor man ihn disziplinieren darf. Da bin ich ja erleichtert.}

\par

Es überraschte wohl jeden, dass es ausgerechnet Morten war, der einwarf. \WR{Captain, bei allem Respekt. Er hat seine Partnerin verloren. Ich will seine Fehler nicht entschuldigen, aber das sollten wir im Kopf behalten.}

\par

Zunächst warf ihm Captain Fiscale einen vernichtenden Blick zu. Doch sie wandte sich Anna Farley zu, als Morten diesem standhielt. \WR{Was ist mit Mister Curiosa? In ihrem Bericht steht, er sei abgestürzt.}

\par

Die Flügelkommandantin nickte. \WR{Ich habe ihn nach Edhor Peak befohlen, um Wilson zurück zu holen. Zusammen haben die beiden dann den Rückzug angetreten. Aber Curiosa wurde von einer Boden-Luft-Rakete getroffen und ist abgestürzt. Leider war keiner unserer Jäger in Reichweite. Wir haben keinen Funkkontakt aber das Wrack seines Fliegers sendet ein automatisches Notsignal. Ich denke, er lebt vielleicht noch und würde gerne ein Aufklärungsflug starten um ihn und vielleicht auch andere…}

\par

\WR{Schlagen Sie sich das aus dem Kopf}, entgegnete Fiscale ruhig. \WR{Dafür fehlt uns wirklich jede Kapazität. Curiosa gilt bis auf weiteres als vermisst.}

\par

Es war nicht schwer zu erkennen, dass Anna Farley gerne noch mehr gesagt hätte. Doch etwas hielt sie zurück. Darum nickte sie der Kommandantin nur entgegen. \WR{Gut}, fuhr diese fort, \WR{ruhen Sie sich etwas aus. Aber rechnen Sie damit, dass sie in spätestens zwei Stunden wieder starten müssen. Außer Sie, Herr Wilson.} Morten schwante bereits bei der zivilen Anrede nichts gutes. \WR{Ich entlasse Sie hiermit aus dem Dienst. Ab sofort sind Sie nur noch ein aufsässiger Zivilist, der im Weg steht. Major Farley, eskortieren Sie ihn auf Ihrem Weg nach unten in die Brig. Wegtreten!}

\par

Zu Mortens Überraschung nickte Kevin nicht einmal. Er ging einfach neben Anna Farley her in den Lift. Schließlich entschied sich auch Morten, nicht auf den nächsten warten zu wollen und eilte den beiden nach.

\par

Captain Fiscale blieb ebenfalls für einige Augenblicke ruhig stehen. Immer wieder sah sie im Geiste die Situation auf der Oberfläche vor sich. Die Lage war unmissverständlich einfach. Wenn es die Shutek schaffen würden, die Verteidigungsanlagen im Süden zu zerstören~-- und derzeit trennte sie nur die schwer angeschlagene Hauptstadt von den Silos in der Tundra dahinter~-- dann war die Schlacht verloren. Die Verstärkung der Shutek würde heranrücken und selbst, wenn die Schiffe um die \EN{Regenvogel} die Lufthoheit im Orbit behaupten können würden, wäre der Weg für Bomben auf die Oberfläche trotzdem frei.

\par

Schlimmer noch wäre zweifellos, wenn die Shutek die Orbitalverteidigung einnehmen könnten. Zwar schien dies technisch nur schwer vorstellbar, doch das Computervirus und auch die neuerliche Blockade der Nullzonenkommunikation ließ Captain Fiscale diesbezüglich gedanklich offener werden.

\par

Die Alternative erschien ihr aber in Wirklichkeit keine zu sein. Bomben auf die eigenen Leute. Selbst, wenn die Bewohner es rechtzeitig aus Yêxīn heraus schaffen würden, was bereits unwahrscheinlich schien, dann würden dennoch tausende Soldaten ihr Leben verlieren.

\par

Natalia Fiscale sah zu dem zugedeckten Körper ihres ersten Offiziers.
