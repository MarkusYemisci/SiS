Der Bordtag der Raumstation \EN{Athena} orientierte sich an der Standardzeit der Erde. So wurde die Beleuchtung in den nicht lebenswichtigen Bereichen gedämmt, wenn es in Afrika Nacht war. Momentan aber herrschte volle Beleuchtung. Wie man aus den, der Erde zugewandten Fenstern erkennen konnte, war in Europa helllichter Tag. Dem ein oder anderen erschien es gelegentlich etwas eintönig, immer nur Europa, Afrika und Teile von Asien erkennen zu können. Da sich die Basis auf einem stationären Orbit befand, war der abgekehrte Teil der Erde niemals zu sehen.

\par

Die \EN{Athena} war die größte Raumstation, die jemals von Menschenhand erschaffen worden war. Ihr, im ganzen betrachtet, säulenförmiges Zentralsegment beinhaltete allein vierhundert Decks und maß einen Durchmesser von anderthalb Kilometern. Die peripheren Anlagen, die integrierten Schiffswerten und die Habitatscheibe erweiterten die Raumstation noch einmal um einiges.

\par

In der \EN{Athena} liefen alle militärischen Kommandostrukturen des Konglomerats zusammen. Nur die Polizei und der Geheimdienst hatten ihre höchsten Instanzen an anderen Stellen.

\par

Permanent dockten größere und kleinere Schiffe an die Station an oder legten von ihr ab. Die allermeisten davon waren militärische Raumer der Starforce oder der Phalanx. Sie reichten von dreißig Meter langen Transportern bis zu großen Schlachtschiffen der \EN{Huǒ yǔ}-Klasse. Aber keines dieser Schiffe war derart groß, wie der anderthalb Kilometer lange schwere Träger, der gerade in der einer der Mega-Werften der \EN{Athena} konstruiert wurde. Die Trockendocks der Raumstation waren die einzigen Fertigungsanlagen, die derart große Schiffe, wie die \EN{Königin} KsT produzieren konnten.

\par

Sie war auch der Grund, weshalb sich Grandadmiral Norton Burns seit neustem häufiger auf der Kommandostation aufhielt. Der Bau des Schiffes war seine eigene Idee gewesen und das ehrgeizige Projekt unterlag indirekt seiner Befehlsgewalt.

\par

Momentan hielt der Admiral gerade ein Treffen mit den Verantwortlichen des Konstruktionsteams und Commodore Ibrahimovic ab, der die Fertigstellung des Schiffes überwachte. Die Männer und Frauen standen um einen Tisch versammelt, dessen gesamte Oberfläche als Anzeigetafel diente. Auf dem Display waren die Daten der \EN{Königin} abgebildet. Die Zeichnung zeigte das Schiff, wie es später aussehen sollte. Obwohl es das größte und massigste Schiff der Starforce war, wirkte es trotzdem windschnittig und dynamisch, auch wenn die Konstruktion etwas zusammengewürfelt aussah.

\par

Der Hauptrumpf war im Wesentlichen ein langer Quader, durch den der Haupthangar verlief und unter dessen Ende zahlreiche Plasmaturbinen angebracht waren, die dem Gigant Schub verleihen sollten. An der Backbordseite, etwa ein Drittel der Länge vom Heck aus, war ein großer Turm angebracht, der die Mannschaftsquartiere und die Kommandosektion beinhaltete. An der Stelle, an der er mit dem Rumpf zusammenlief, fanden sich außerdem sechs Schnellstart-Röhren, in denen später zu jedem Zeitpunkt Abfang- und Verteidigungsjäger bereitstehen sollten. Die Steuerbordflanke wurde vom langen Sekundärhangar gesäumt, der hauptsächlich für Frachtoperationen gedacht war, in denen aber mit Hilfe eines ausgeklügelten Tunnelsystems auch nacheinander sämtliche Jäger des Haupthangars verlegt werden konnten.

\par

Die \EN{Königin} war groß genug und so umfassend ausgerüstet, dass sie eine gesamte kleine Trägergruppe ersetzen konnte. Und genau das war ihr größtes Problem. Schiffe ihrer Art hatten das Potential, früher oder später an die Stelle ganze Flotten treten zu können. Und obwohl die Konstruktion der \EN{Königin} Unsummen verschlang, war sie im fertigen Zustand dennoch kostengünstiger als ein Verband aus mehreren, kleineren Raumern. Das schmälerte die Militärausgaben und diese wiederum waren zu einem nicht mehr wegzudenkenden Wirtschaftsfaktor geworden.

\par

Admiral Burns fiel es schwer zu verstehen, wieso Präsident Otis dies bei all seinen Aversionen gegen die Armee nicht sehen konnte.

\par

Viele Blicke gingen nun durch die Fensterfront des kleinen Besprechungsraums auf den Rumpf des massigen Trägers. Der Admiral wandte allerdings sein Augenmerk wieder seinen Kollegen zu, als der Chef des Ingenieurstrupps mit seinem Bericht begann: \WR{Trotz der Budgetkürzungen durch das Parlament sind wir gut vorangekommen}, der Offizier deutete auf einige Bereiche des Schiffes, die am äußeren Rumpf lagen. \WR{Wir haben nun fast die komplette Hülle fertig gebaut und den Großteil der Oberfläche bereits gepanzert. In zehn bis eindutzend Wochen wird die gesamte Außenhaut fertig verbaut sein. Dann können wir uns endlich der Innenausrüstung und den Geschütztürmen widmen.}

\par

Commodore Lara Ibrahimovic, eine rüstige Frau in den Fünfzigern, quittierte den Bericht des Ingenieurs mit einem zufriedenen Nicken und merkte an: \WR{Das sollte nicht allzu lange dauern. Die Einfassungen der Innenräume sind ja jetzt schon fertig.}

\par

\WR{Das stimmt zwar, Madam. Aber man sollte nicht vergessen, wie viel Arbeit die Montage der Geschütztürme macht. Das Schiff ist schwerer bewaffnet, als alles andere, dass die Union jemals gebaut hat. Die \EN{Königin} wird über vier Hauptgeschütze verfügen, die auch alle mit Energie versorgt werden müssen. Die meisten Schiffen haben höchstens zwei.}

\par

Grandadmiral Burns musterte angestrengt die Pläne. Der schwere Träger war tatsächlich von Kanonentürmen nur so übersät. Dazu waren es Sondermodelle mit eigener Feuermannschaft, Zielsuchsystem und Energiequelle. Jedes dieser Geschütze musste speziell für die \EN{Königin} hergestellt werden.

\par

\WR{Wann schätzen Sie, können wir den Jungfernflug ansetzen?}, fragte der Admiral den Ingenieur hoffnungsvoll.

\par

Dieser war zwar unwillig seinem Vorgesetzten eine Antwort zu geben, die ihm nicht gefallen könnte, sah sich aber trotzdem gezwungen zu entgegnen: \WR{Nicht vor zwei Monaten. Vorher haben wir die nötigen Teile einfach nicht zusammen. Und die Mannschaften der Konstruktionsabteilung schieben jetzt schon Doppelschichten.}

\par

\WR{Keine Sorge. Ungeduld ist nun sowieso unangebracht}, beschwichtigte Commodore Ibrahimovic. \WR{Wir bauen nun schon über fünf Jahre an diesem Schiff. Wir haben das Ziel fast erreicht.}

\par

Grandadmiral Burns nickte zur stummen Zustimmung. Sein Blick ging zu einer anderen Werft, die hinter der gitterförmigen Konstruktionsanlage der \EN{Königin} lag. Dort wurde ihr Schwesterschiff, die \EN{Kǎi zé} KsT konstruiert. Allerdings bestand diese aus nicht viel mehr als einem filigranen Gestell aus Stahlträgern, dass die spätere Form bestenfalls erahnen lies.

\par

Einer der Architekten, der das Schiff mit entworfen hatte meldete sich erstmalig zu Wort: \WR{Ich denke auch, dass wir uns keine Gedanken über den Zeitplan zu machen brauchen. Wir sind wirklich dicht am Ziel. Außerdem ist der Bau relativ problemlos abgelaufen. Und aus den Schwierigkeiten, die wir trotzdem hatten, konnten wir für den Bau der \EN{Kǎi zé} eine Menge lernen. Ich glaube nicht, dass es in ihrem Fall wieder fünf Jahre dauern wird, bis die Konstruktion abgeschlossen ist.}

\par

\WR{Wie sollen wir weiter verfahren?}, fragte Grandadmiral Burns in die Runde. \WR{Wie weit ist die Auswahl der Besatzung vorangeschritten?}

\par

Commodore Steinert, der für die Innenausstattung und die Besatzung verantwortlich war, antwortete prompt: \WR{Wir haben die blauen Löwen, der \EN{Guòrén} als fünftes der sechs Geschwader ausgewählt, die auf die \EN{Königin} verlegt werden sollen. Die Piloten sind größtenteils sehr erfahren und haben einen sehr guten Ruf in der Starforce.}

\par

Norton Burns nickte zufrieden. Er wollte gerade etwas sagen, als sich das Schott zum Besprechungsraum öffnete und einer der insgesamt sechsundzwanzig Meldeoffiziere der Raumstation eintrat. Der Grandadmiral erkannte den Mann sofort. Andernfalls hätte er notfalls die Anzeige der Baupläne des schweren Trägers ausschalten müssen, um sie weiterhin geheim zu halten.

\par

Der Kommunikationsoffizier rückte seine Mütze zurecht und straffte seine Uniformjacke, bevor er beiläufig aber immerhin mit abgespreitztem Daumen salutierte und dann Meldung machte: \WR{Admiral, wir haben gerade einen Priorität eins Ruf der Pinnacle Science Group erhalten. Der Firmenleiter möchte sie persönlich sprechen. Er sagt, es sei dringend.}

\par

Burns schüttelte andeutungsweise mit dem Kopf. Er fragte sich, was ein Unternehmensboss nun von ihm wollte. Und vor allem, wieso die Anfrage nicht über die üblichen Kanäle lief. Die PSG hatte ihren Sitz auf Kreuzpunkt Primus. Und dort hielt man vom Dienstweg nicht viel. Gab es ein Problem, dann war eine Investition stets die Antwort.

\par

\WR{Es tut mir Leid, Lieutenant, aber ich bin in einer wichtigen Besprechung}, gab der Grandadmiral schnell zurück. \WR{Sagen Sie ihm, ich werde mich mit ihm in Verbindung setzen sobald ich kann.}

\par

Der Meldeoffizier kratzte sich nervös am Kopf und verzog das Gesicht. \WR{Nun, Sir, ich habe dem Mann bereits gesagt, dass sie wahrscheinlich keine Zeit haben. Aber er sagte möglicherweise seien Mitarbeiter seiner Firma in Gefahr, die schnellstmöglich Hilfe brauchen.}

\par

\WR{Und wieso hat er dann nicht die zentrale Notfallstelle des Konglomerats gerufen?}, hakte Admiral Burns nach.

\par

Der Offizier antwortete hastig: \WR{Er meinte, die Zeit würde drängen und er wolle sich deshalb lieber gleich mit den Verantwortlichen in Verbindung setzen. Er sagte auch, dass die üblichen Mittel der Notfallstelle vielleicht nicht ausreichen würden.}

\par

Daraufhin machte sich Admiral Burns widerwillig auf den Weg. Zu den anderen Offizieren sagte er schlicht: \WR{Führen sie die Besprechung bitte ohne mich zu Ende. Sie haben alle Freiheiten, die sie brauchen.} Es störte ihn sehr, dass er sich dem Willen eines Zivilisten~-- zudem einem Unternehmer~-- beugte. Aber zum einen konnte er die Pflichten des Konglomerats seinen Schützlingen gegenüber nicht ignorieren und zu anderen würde sich ein einflussreicher Mann, wie der Leiter der PSG einer war, auch im politischen Sinne eventuell als wichtiger Verbündeter herausstellen.

\par

Die Anwesenden nickten teils und bestätigten die Anordnungen, worauf sich der Grandadmiral dem Kommunikationsoffizier in die Transmitterzentrale folgte. Beide durchquerten den großen Kommandoraum mit schnellen Schritten. Etwas Zeit brauchten sie aber dennoch, denn der komplett verkuppelte Raum war so große wie ein halbes Fußballfeld.

\par

Kurze Zeit später betrat der Admiral den Transmitterraum. Ein kleines Separee, dass durch milchiges Glas vom Rest der Kommandozentrale abgetrennt war. Das Innenleben des Raums war denkbar einfach gehalten. Es gab drei Stühle, eine Konsole zur Kontrolle der Kommunikation und zwei mannshohe Glasscheiben, die gerade aus dem Boden herausgefahren wurden, als Norton Burns den Raum betrat. Die Glasscheiben waren mit einem Metallrahmen und einigen, schwach leuchtenden blauen Lampen in diesem versehen. Sie dienten zur bildlichen Darstellung der Gesprächspartner. Eine komplexe optische Täuschung ließ das Bild fast dreidimensional erscheinen. Vor der Erfindung des drei-D-View war diese Technik der Ersatz für dreidimensionales Fernsehen gewesen.

\par

Als Admiral Burns seinen Geheimcode in die Konsole eintippte bauten sich die Darstellungen von zwei Männern auf. Der eine war schon recht betagt, mindestens so alt wie Burns selbst und trug einen kostspieligen, seidenen Gehrock. Seine grau melierten Haare fielen über einen sauberen Seitenscheitel. Zweifellos war der andere Mann im Justizsystem der Union tätig, denn er trug eine Robe. Vermutlich handelte es sich um einen Rechtsanwalt, denn es erschien wahrscheinlich, dass der Chef der Pinnacle Science Group keinen Volksanwalt zu Hilfe holen würde, der nur für die Verfolgung von Straftaten verantwortlich war.

\par

Als die Verbindung aufgebaut war, begann der Ältere zu sprechen: \WR{Ich grüße Sie, Grandadmiral Burns. Ich bin Horatio Balsato, ich leite die PSG. Das ist Herr Pedro Mendez. Er wurde uns von der Union als Rechtsbeistand zugeteilt.}

\par

\WR{Guten Tag}, schob der Rechtsanwalt ein.

\par

Admiral Burns hätte an dieser Stelle normalerweise salutiert um seine Gesprächspartner zu begrüßen. Hier beließ er es bei einem Nicken, da ein Salut von einigen Zivilisten als unpassend empfunden wurde. Nur bei Präsident Otis tat er es trotzdem.

\par

\WR{Vielen Dank, dass Sie sich für uns Zeit genommen haben. Ich würde gerne sofort zur Sache kommen}, bat Balsato. \WR{Möglicherweise sind Menschen in Gefahr.}

\par

Der Grandadmiral nickte abermals. Sein Gesprächspartner schien sich sammeln zu müssen, bevor er fortfuhr, was Burns verwunderte. Jemand, der eine so hohe Position bekleidete wie Balsato, war in der Regel rhetorisch hochbegabt und darauf getrimmt, in einer Konversation stets einen gefassten Eindruck zu waren. \WR{Im Moment ist eines unserer Schiffe, die \EN{Virial}, im Arktur System unterwegs um dieses zu untersuchen. Unser Firmenprotokoll schreibt vor, dass sich jedes Forschungsschiff, dass in den Randgebieten operiert, in bestimmten Zeitabständen Situationsberichte senden muss. Die Meldungen der \EN{Virial} sind nun aber schon seit über eindutzend vier Stunden überfällig.}

\par

Admiral Burns nahm nun auf einem der Stühle Platz und stellte sich auf eine längere Unterredung ein. \WR{Könnte es sein, dass das Schiff nur ein mechanisches Problem hat und deswegen keine Berichte mehr sendet?}, fragte er, bemüht nicht zu unbedarft bezüglich technischer Fragen zu wirken.

\par

Balsato setzte einen schwer einzuordnenden Gesichtsausdruck auf, der irgendwo zwischen Besorgnis und Nachdenklichkeit schwankte. \WR{Ich gebe zu, dass ist nicht ausgeschlossen}, gestand der Firmenchef ein. \WR{Aber wir halten es für unwahrscheinlich. Auf der \EN{Virial} arbeiten mindestens vier Personen, die Bücher mit eigenen Nullzonentrancievern besitzen.}

\par

Dass ein Großkonzern derartige persönliche Informationen über seine Angestellten sammelte, überraschte Admiral Burns nicht. Dass Balsato aber bereit war, dies einzugestehen, schon.

\par

\WR{Wir wollen wirklich keinen falschen Alarm auslösen}, schob der Anwalt ein. \WR{Aber solche Vorfälle sind selten. Und da es sich eventuell um einen ernstzunehmenden Notfall handelt, hielten wir es für klug, uns gleich an die Sie zu wenden.}

\par

Der Firmenleiter ergriff sofort das Wort, kaum das sein Anwalt zu Ende gesprochen hatte. \WR{Die Starforce ist die einzige Institution, die Arktur in einer akzeptablen Zeit erreichen kann. Die Bergungsschiffe der Handelsmarine sind weit über den Raum der Union verstreut. Sie würden Tage brauchen. Wenn wirklich ein schwereres Problem vorliegt, ist das viel zu lange.}

\par

Die Argumente leuchteten dem Leiter den Konglomerats langsam ein. Die Notfall-Eingreifschiffe waren zwar hervorragend ausgerüstet aber eher darauf konzipiert, in den inneren Systemen der Union zu arbeiten.

\par

Die Starforce war, in den Augen des Admirals, tatsächlich die richtige Wahl für die Aktion. Besonders, da Admiral Burns durch eine gelungene Rettung deutlich machen konnte, dass die Armee ihrem Schutzauftrag auch in Friedenszeiten nachkommen konnte. Entgegen der Meinung des Präsidenten.

\par

Aber bei diesem würde auch die erste Hürde auftreten. Henry Otis würde vielleicht sogar darauf bestehen, dass die Bergungsschiffe der Handelsmarine den Auftrag ausführen sollten. Besonders da es offiziell auch ihre Aufgabe war.

\par

\WR{Nun}, begann Burns, \WR{von mir aus könnte sofort eine Rettungsaktion starten. Allerdings befürchte ich, das Präsident Otis dem eventuell nicht zustimmen wird. Wie sie wissen steht er der Armee sehr… kritisch gegenüber.}

\par

Anscheinend hatte Anwalt Mendez genau diesen Einwand erwartet, was ihn auf den Plan rief. \WR{Herr Balsato hatte eine ähnliche Befürchtung. Daher hat er mich gebeten \textit{ihren} Entscheidungsspielraum in dieser Sache auszuleuchten. Und ich habe eine gesetzliche Regelung gefunden, die es hochrangigen Offizieren des Konglomerats gestattet, kleinere Aktion, die deutlich keine militärischen Absichten beinhalten, auch ohne die Zustimmung des Präsidenten zu befehlen. Sofern die Aktion nicht mehr als eine Woche in Anspruch nimmt.}

\par

Ohne es zu wollen zog Admiral Burns beide Augenbrauen hinauf. Ihm war diese Klausel gar nicht bekannt gewesen. Juristische Nischen waren selten in den Gesetzen der Union, besonders nach der Reform durch Henry Otis. Aber wenn der Anwalt es sagte, dann würde es stimmen. Zu lügen oder zu übertreiben war für ihn in diesem Fall sinnlos, denn Burns konnte die entsprechende Passage einfach nachschlagen.

\par

\WR{Ich bitte Sie inständig, ein Schiff loszuschicken}, brachte der Firmenleiter der PSG hervor. \WR{In den Arbeitsverträgen meiner Firme steht eindeutig, dass die Sicherheit der Mitarbeiter garantiert wird und ich habe nicht vor diese Klausel zu brechen.}

\par

Norton Burns nickte. \WR{Einverstanden. Wir werden alles versuchen, ihr verlorenes Schiff zu finden.}

\par

\WR{Vielen Dank}, kam es von Horatio Balsato zurück. Der Stein, der ihm vom Herzen fiel war praktisch zu hören. \WR{Wir halten Sie auf dem laufenden, falls wir doch noch Nachricht von der \EN{Virial} bekommen.}

\par

Der Anwalt schloss die Besprechung: \WR{Es wurde veranlasst, dass sie Zugriff auf alle Daten über das Schiff und die Crew bekommen. Vielen Dank auch von meiner Seite.}

\par

\WR{Keine Ursache. Auf Wiedersehen}, verabschiedete sich der Grandadmiral und schloss die Kommunikationsverbindung.

\par

Sein nächster Weg führte ihn schnellen Schrittes zum strategischen Planungsbereich der Kommandobrücke. Der wohnzimmergroße Bereich stand voller transparenter Glaswände, die Flottenbewegungen, Schiffsdaten und vieles mehr anzeigten. Insgesamt vierzehn Offiziere bemannten die Station.

\par

Burns begab sich zum strategischen Oberbefehlshaber, Admiral Alas Sinke.

\par

\WR{Admiral, es gibt Arbeit}, brachte Burns schnell hervor und salutierte während dessen zackig. Sinke tat es ihm gleich und wandte sich ihm zu. \WR{Wunderbar. Ich glaube langsam fangen unsere Jungs an zu faulenzen.}

\par

\WR{Die Pinnacle Science Group hat Kontakt zu einem Schiff in Arktur verloren}, berichtete Burns und wandte sich einer Glaswand zu, die Positionen aller Träger der Starforce in der ganzen Union anzeigte. \WR{Haben wir irgendwelche Schiffe in der Nähe, Admiral?}

\par

Der strategische Befehlshaber bediente eine Konsole mit flinken Fingern und las die Informationen auf einem Flachbildschirm ab. \WR{Jawohl, Sir}, antwortete er nach kurzer Zeit und deutete auf der großen, gläsernen Anzeigetafel auf das Pollux System. \WR{Der leichte Träger \EN{Regenvogel} befindet sich zur Zeit direkt im Nachbarsystem. Sie erhält gerade Versorgungsgüter und Piloten von Konvois aus anderen Systemen.}

\par

Grandadmiral Burns sah sich das Bild genauer an. \WR{Wie ist der Zustand der \EN{Regenvogel}?}

\par

\WR{Voll einsatzbereit}, entgegnete Sinke. \WR{Wenn wir ihr jetzt den Befehl geben, die Route nach Arktur anzusteuern und die Konvois umleiten, kann sie den Sprung morgen früh durchführen.}

\par

Norton Burns nickte zufrieden. \WR{Sehr gut. Ich brauche sofort eine sichere Verbindung zum Kapitän der \EN{Regenvogel}.}