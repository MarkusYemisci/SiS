Bis vor einigen Augenblicken war das Endspiel der Fußballweltmeisterschaft noch sehr spannend gewesen. Weder Corna noch der Mars hatten ein Tor geschossen. Am laufenden Bande hatte es Chancen gegeben und zwar auf beiden Seiten. Eine echte Herausforderung, selbst für starke Nerven. Einmal hatte es der Cornaer Stürme Arthur Römig fast geschafft ein Tor zu erzielen, indem er aus gut zwanzig Meter Entfernung einfach abgezogen hatte. Aber der Ball war knapp über die Querlatte gesaust. Nur ein paar Zentimeter tiefer und der Torwart der Auswahl vom Mars hätte nicht die geringste Chance gehabt.

\par

Knappe zehn Minuten später hatte Römigs Sturmpartnerin Amanda Riya nur fünf Meter vom Tor entfernt geschossen. Aber Torwart Almerto Laguni hatte den Ball gerade noch so mit offener Hand abgelenkt.

\par

Dann, in der neunundsiebzigsten Minute, war das Spiel gekippt. Eine Ecke für den Mars, ein Kopfball und der Ball war ins Cornaer Netz gegangen.

\par

Jenkins und Leeroy, die sich das Spiel als drei-D-Schau ansahen, hatten den Spielern des Mars die übelsten Flüche nachgebrüllt, die sie kannten, als das Tor gefallen war. Aber es half nichts. Zumal sie gerade auf Pollux Primus stationiert waren, dort einige Radaranlagen überwachten und daher garantiert nicht gehört wurden.

\par

Das bitterste in dieser Situation war der irdische Spielmodus dieses Turniers. Würde dieses Spiel im sonst unionsweit gültigen Reese-Modus ausgetragen, hätte der Mars mit einem Sieg, der nur auf einem Standardtor beruhte lediglich zwei der für den Rundensieg nötigen drei Punkte erzielt. Es hätte mindestens ein weiteres Spiel gegeben.

\par

Zusammen mit ihrem Vorgesetzten hatten sie einen ganzen Kontinent für sich alleine. Außer dem kleinen Außenposten des Konglomerats stand kein einziges anderes Gebäude mehr auf der übergroßen Insel.

\par

Da sowohl Rat als auch Leeroy von der Erde stammten, und diese gegen den Mars ausgeschieden war, lagen ihre Sympathien klar bei Corna. Aber wie es im Moment aussah, würde ihre Mannschaft wieder einmal gegen den Mars scheitern, denn dessen Spieler und Spielerinnen stellten einfach ihren Strafraum zu und kickten den Ball weg, wenn er ihrem Tor zu nahe kam.

\par

Tatsächlich hatte es Corna noch niemals geschafft, den Mars zu schlagen. Weder in einem Freundschaftsspiel noch in einer Begegnung, in der es um einen Titel ging. Diesmal hatten sie bis vor kurzem eine wirklich gute Chance gehabt. Überhaupt waren sie als Außenseiter fulminant ins Turnier gestartet und hatten eine sehr überzeugende Leistung gezeigt. Aber nun lagen sie mit einem Treffer zurück und den aufzuholen, würden sie kaum schaffen.

\par

\WR{So ein Dreck!}, donnerte Jenkins und schmiss seine unzerbrechliche Glasflasche in eine Ecke, wo sie mit lautem Getöse aufschlug. \WR{Sie waren so gut! Und jetzt holt sich der Mars zum fünften Mal den Titel. Diese dämlichen Sparfußballer. Ein Törchen genügt denen.}

\par

Leeroy gab sich ruhig, war aber innerlich genauso aufgewühlt. \WR{Man sollte eine Regel einführen. Jeder der gegen den Mars spielt bekommt zwei Tore Vorsprung. Mal sehen ob die Ärsche dann immer noch so hinten rein stehen.}

\par

Mutlos beobachteten die beiden die holographische Projektion. Corna spielte immer noch gut. Sie schoben sich den Ball zu als hätten sie in ihrem Leben noch nie etwas anderes getan. Aber ihre Attacken prallten immer wieder an der dichten Abwehr der Marsianer ab.

\par

Leeroy wandte sich vom Spielfeld ab und verabschiedete sich sogleich auch von der Vorstellung, dass ein sympathischer Außenseiter wie Corna jemals die Weltmeisterschaft gewinnen könne. Sein Blick ging zu den Radarbildschirmen, die er eigentlich ständig im Auge behalten sollte. Aber wie er schnell erkannte war es ziemlich sinnlos, ständig auf den monotonen Monitor zu stieren. Die einzigen beiden Kontakte waren die Raumstation der Starforce und eine kleine Patrouille, die gerade ihre Runde machte. Andere Schiffe gab es nicht. Das letzte mal, als annähernd so etwas wie Handlung auf die Bildschirme getreten war, war als ein Forschungsschiff Pollux in Richtung Arktur passiert hatte. Da hatte das Radar nach langer Zeit wieder gepiept und Leeroy aus seinen Tagträumen gezerrt.

\par

Das nächste mal, dass ein Zivilschiff ankommen würde, wäre wohl in zwei Wochen, wenn ein Frachter die abseits gelegene Kolonie auf dem Hauptkontinent von Pollux beliefern würde. Bis dahin wäre Leeroy und Jenkins Arbeit genauso sinnlos wie langweilig.

\par

Ebenfalls langweilig würde das Endspiel wohl nun verlaufen. Nur noch Jenkins sah mit einem Auge auf die Projektion des Spielfelds. \WR{Mars im Ballbesitz}, erklang die Stimme des Kommentators aus den Boxen des Holograhen. \WR{Ambanelli stürmt aufs Tor der Cornaer zu. Wird er die Sache für seine Mannschaft jetzt klar machen?}

\par

Leeroy schielte zum Hologramm hinüber und sah, wie der einzige Stürmer des Mars schoss. Aber der Torwart der Cornaer hatte keine Probleme damit, den Ball abzufangen. Ein weiter Abschlag folgte. Leeroy tätschelte Jenkins gegen die Schulter, als er erkannte, wie der Ball im Mittelfeld landete. Genau dort, wo Arthur Römig völlig ungedeckt stand. Außerdem war die Abwehr des Mars ziemlich in der Mitte zusammengerückt und Amanda Riya stand~-- nicht im Abseits~-- in einer sehr verlockenden Position. Jenkins faltete die Hände und betete, dass Römig die Chance genauso deutlich vor sich sah, wie er gerade.

\par

Und tatsächlich tat er es. Der Ball schlitterte weit aber dennoch präzise über den Platz. Riya rannte im selben Moment los, indem Römig ihr zupasste. Mühelos und bereits mit einer beachtlichen Geschwindigkeit erreichte sich den Ball und setzte sich hinter die gesamte Fünferkette des Mars.

\par

\WR{Du liebe Zeit!}, rief der Kommentator. \WR{Sie wird doch nicht…}

\par

Die Stürmerin kam in den Strafraum. Jenkins und Leeroy sahen sie rasch über ihre Schulter blicken. Arthur Römig, ihr Sturmpartner rannte frontal aufs Tor zu und hatte den Strafraum ebenfalls erreicht. Der Torwart des Mars rannte aus seinem Allerheiligsten auf die Stürmerin zu, die nun schon beachtlich nahe ans Tor gekommen war, bereit ihr den Ball direkt vom Fuß zu reißen.

\par

Doch Amanda Riya blieb ruhig und passte den Ball mit der Hacke nach hinten, kurz bevor der Torwart ihn zu fassen bekam. Arthur Römig, der sich bereits geschickt in Position gebracht hatte, nahm den Ball locker an, der durch eine Lücke in der Abwehr gerollt kam. Ohne lange zu überlegen zog er ab~-- und traf.

\par

Jenkins und Leeroy sprangen grölend auf und fielen sich in die Arme. Genauso wie Amanda Riya und Arthur Römig. Selbst durch die drei-D-Übertragung war leicht zu erkennen, dass es im Stadion einen bemerkenswerten Wechsel gegeben hatte. Die Fans des Mars, die gerade noch am jubeln über ihren vermeintlichen fünften Titel waren, verstummten plötzlich und fingen dann mit lauten Buhrufen an. Die Cornaer Fans, die gerade noch am Pfeifen gewesen waren, übernahmen nun den Part der Jubelnden.

\par

Jenkins öffnete eine Flasche Bier und goss sie sich in den Hals. Leeroy sprang in dem kleinen Radarüberwachungsraum umher und grölte und johlte lauthals herum.

\par

Wenn ihr Vorgesetzter das mitbekommen würde, dann würde sicherlich eine Strafversetzung folgen, die sie an einen noch langweiligeren Ort als Pollux bringen würde. Aber er war außer Sicht und die beiden feierten munter weiter.
\ortswechsel
Währenddessen jagte ein Pilot seinen Raumüberlegenheitsjäger der \EN{Falken}-Klasse mit panischer Angst durch das All. Sein Steuerbordtriebwerk brannte lichterloh aber er wollte es dennoch nicht abschalten. Denn wenn er das tat, würden sie ihn kriegen.

\par

Ein Teil seiner Backbordtragfläche war einfach abgerissen und über die Scheibe seines Cockpits zogen sich schon bedrohliche Risse um die sich Eis zu sammeln begann. Aber es war ihm immer noch besser ergangen als seinen Geschwaderkameraden. Keiner der drei lebte noch. Der Pilot erinnerte sich nur noch daran, wie er ihre Jäger verbrennen sah und daran, wie ihre angsterfüllten Schreie in seinem Kopfhörer gegellt hatten.

\par

Ein hastiger Blick auf das Radar verriet ihm, dass sie ihn noch nicht erreicht hatten. Die zehn roten Punkte kamen aber bedrohlich nahe. Und das schlimmste war, dass er momentan der einzige war, der sie zu sehen schien. Die Sensoren seiner Heimatbasis, der Raumstation Kaukasia, würden nicht weit genug reichen. Aber die des Satellitensystems von Pollux Primus müssten dazu in der Lage sein. Irgendwie schien niemand die Unbekannten zu bemerken, die nun hinter ihm her waren.

\par

Er versuchte schon die ganze Zeit, jemanden zu warnen aber sein NZT war beschädigt und momentan gab es nichts das innerhalb seiner Kurzwellenreichweite lag. Er musste es einfach bis zur Kaukasia schaffen. Er musste!

\par

Wenn er nur noch eine kleine Weile durchhielt. Erschrocken, stellte er fest, dass sein Triebwerk Schub verlor. Die Kontakte kamen näher und näher. Er musste sein Signal jetzt absetzen und hoffen, dass ihn jemand hört: \WR{Notruf, Notruf! Hier ist rot vier von der Bereitschaftspatrouille. Ich werde angegriffen, ich wiederhole, ich werde angegriffen!}

\par

Einen schrecklichen Moment lang, in dem die gelben Blickpunkte immer näher rückten, geschah nichts. Der Pilot kniff die Augen zu. Doch dann quäkte eine anonyme Männerstimme aus seinem Kopfhörer: \WR{Hier ist der Leitstand der Kaukasia. Rot vier, wiederholen sie bitte, der Empfang ist sehr schlecht.}

\par

Dem Pilot war klar, dass es wohl nicht am Empfang gelegen hatte. Der Mann am anderen Ende glaubte wahrscheinlich, nur nicht richtig zu hören.

\par

\WR{Ich werde angegriffen! Sie sind hinter mir her!}, brüllte der Pilot in sein Mikrofon.

\par

\WR{Was? Wer ist hinter Ihnen her? Wie ist Ihr Status?}

\par

\WR{Ziemlich mies}, antwortete der Pilot wahrheitsgemäß. \WR{Starten Sie sofort alle Jäger, die sie haben, da draußen sind verdammt viele von denen. Ich konnte die Kontakte nicht mal zählen!}

\par

Wieder dauerte es eine Weile, bis eine Antwort kam. Der Mann auf der Kaukasia schien sich zu fragen, ob das ganze eine Art Jux sein sollte. \WR{Wer hat Sie angegriffen, Rot vier?}

\par

\WR{Keine Ahnung}, plapperte der Pilot hastig weiter. \WR{Es waren jedenfalls keine Piraten. Sie sind mit sicher auf der Fährte. Starten Sie bloß alles was sie haben, sonst sind wir alle dran!}

\par

Erneut ließ die Entgegnung der Raumstation auf sich warten. \WR{Bleiben Sie ruhig, Rot vier. Wir schicken Abfangjäger um sie zurück zu eskortieren. Wir haben im Moment niemanden außer Ihnen auf dem Radar.}

\par

\WR{Die sind zu weit weg!}, schrie der Pilot förmlich. \WR{Ich kann Sie sehen! Zehn Stück. Dicht hinter mir!}

\par

Der Assistent des Kommunikationsoffiziers hatte mittlerweile den Kommandant der Kaukasia gerufen. Der alte, bauschbärtige Mann stand nun bei der halbmondförmigen Konsole der Kontaktabteilung.

\par

\WR{Was genau ist passiert, Lieutenant?}, wollte er vom Kommunikationsleiter wissen.

\par

Dieser musste sich beherrschen um nicht mit Schulterzucken zu reagieren. \WR{Das wissen wir leider nicht. Aber die roten Jungs scheinen angegriffen worden zu sein. Bislang haben wir nur Kontakt zu Rot vier. Er sagte, der Rest des Geschwaders sei verloren.}

\par

\WR{Alarmstufe zwei}, ordnete der Kommandant an.

\par

Ein Warnton erklang durch die ganze Basis und riss mindestens zwei Hundertschaften von Offizieren aus dem Schlaf oder von den Drei-D-Holographen. Der Abend war, wie jeder andere auch, dermaßen ruhig verlaufen, dass niemand mehr mit etwas unvorhergesehenem gerechnet hätte. Die Hälfte der Mannschaft sah sich in den Quartieren das Endspiel an, das in Echtzeit direkt von der Erde aus übertragen wurde. Die andere Hälfte schlief oder schob laxen Spätdienst.

\par

\WR{Sein Jäger ist definitiv beschädigt. Abfangjäger wurden gestartet, Sir}, berichtete der Chef der Radarabteilung, seinen Blick auf einen großen Statusmonitor haltend. \WR{Der Angriff muss passiert sein, als die Staffel für fünf Minuten außerhalb unserer Radarreichweite war.}

\par

Der Kommunikationsoffizier hängte mit tiefer Besorgnis an: \WR{Wenn das stimmt, hatten sie nicht mal Zeit einen Notruf abzusetzen.}

\par

\WR{Etwaige Feinde müssten sich bereits innerhalb der Radarreichweite der planetaren Suchsysteme befinden}, gab der Chef der Überwachung zu bedenken. \WR{Aber bisher haben wir noch keine Bestätigung von der Einrichtung.}

\par

Der Kommandant überlegte still. Eines stand fest. Die Gefahr schien real zu sein. An den Kommunikationsoffizier gewandt, befahl er: \WR{Holen Sie die Besatzung der Radarstation an die Leitung, Lieutenant!}
\ortswechsel
Als Amanda Riya gerade noch ihren Fuß in die Bahn des herannahenden Fußballs brachte, und ihn so ins Tor beförderte, bevor sie umfiel, verfielen Jenkins und Leeroy in ihr bisher grölendestes Gebrüll. Das schrille Zirpen, dass vom Anstehen eines Prioritätssignals kündete oder der Warnton des Radars bekamen sie nicht mehr mit. Selbst wenn sie weniger Bier getrunken hätten, wäre der Geräuschpegel noch bei weitem zu hoch gewesen um etwas zu verstehen.

\par

Es waren nur noch zwei Minuten übrig. Die Nachspielzeit war bereits angebrochen. Das Team des Mars war in wildes Gebolze verfallen, die Spieler waren wütend und alle Nase lange wurde die Begegnung wegen eines Foulspiels unterbrochen.

\par

Wenn jetzt nicht etwas ganz blödes passierte, dann war Corna Weltmeister. Rat und Leeroy tanzten und sangen und würdigten ihre Konsolen keines Blickes mehr.

\par

Der Kommandant der Kaukasia sah besorgt aus den Fenstern des Kommandozentrums. Da draußen kam gerade ein Jäger auf die Station zu gerast, der wie verrückt brannte und von irgendetwas verfolgt wurde. Aber niemand wusste, was es war.

\par

\WR{Rot vier, ihr Steuerbordtriebwerk hat Feuer gefangen. Schalten Sie es ab, sonst fliegen Sie noch in die Luft}, rief der Kommunikationsoffizier in sein Mikrofon.

\par

Die Stimme des Piloten klang mittlerweile panisch. \WR{Wenn ich es abschalte jagen die mich in die Luft!}

\par

\WR{Gibt es bereits eine Rückmeldung von der Bodenstation?}, fragte der Kommandant. Mit jedem seiner Worte schien seine Ungeduld zuzunehmen.

\par

\WR{Nein, Sir}, antwortete ihm der Kommunikationsoffizier atemlos. \WR{Ich habe drei Priorität-eins-Signale gesendet. Die scheinen keines davon bekommen zu haben.}

\par

Gerade wollte der Kommandant seinen Einsatzleiter anweisen, eine Fähre auf den Planeten zu schicken, als der Chef der Radarabteilung meldete: \WR{Sir, ich habe eine unbekannte Sprungsignatur auf dem Schirm. Sie ist so stark, dass wir sie orten können obwohl sie weit hinter unserer übliche Reichweite liegt. Aber dort gibt es überhaupt keine Hyperraumroute.}

\par

Die Stimme des Mannes klang, als würde er seinen eigenen Worten nicht glauben. Der Kommandant kam zu ihm herauf gerannt um sich selbst zu überzeugen. Doch ehe er angelangt war, meldete der Radarchef aufgeregt: \WR{Sir, wir haben fünfzehn neue Kontakte. Sie folgen Rot vier!}

\par

\WR{Oh mein Gott!}, brüllte der Pilot so laut, dass er sogar noch mehrere Meter weiter aus dem Lautsprecher zu hören war. \WR{Jetzt sind wir alle dran!}

\par

Dann folgte statisches Rauschen. Einige Blicke gingen zu den Fenstern. Draußen war gerade ein heller Lichtblitz aufgeflammt. Keiner sprach es aus aber alle wussten, dass es die Explosion von rot vier gewesen war.

\par

\WR{Schicken Sie Nachrichten an Admiral Piet und Legath Habbot}, polterte der Kommandant. \WR{Sagen Sie ihnen, Sie sollen ihre Basen auf Alarmstufe zwei bringen. Admiral Piet soll Jäger starten lassen.}

\par

Der Kommunikationsoffizier nickte und begann die Anweisungen an die planetaren Starforce der Starforce und der Phalanx zu senden.

\par

\WR{Sollen wir ebenfalls Jäger starten, Sir?}, fragte der Einsatzleiter, dem kalter Schweiß übers Gesicht rann.

\par

Der Kommandant nickte, woraufhin der Lieutenant vom Einsatz die Flügelkommandanten aus den Betten und der Bar holte.

\par

Die ganze Raumstation explodierte nun in Aktion. Wo vor wenigen Minuten noch Müßiggang und Gemütlichkeit geherrscht hatte, begann alles wild durcheinander zu rufen und zu laufen. Unterdessen sahen der Einsatzleiter und der Kommandant gespannt auf eine Anzeige. Darauf war zu erkennen, wie die Abfangjägerstaffel den fünfzehn anfliegenden Kontakten entgegenkam. Weiter hinten war ein größerer Blickpunkt zu sehen, der gerade dabei war, in mehrere kleine zu zerfallen. Langsam begannen die Kontakte wieder zu verschwinden.

\par

\WR{Was ist da los?}, fragte der Kommandant wütend den perplex aussehenden Radaroffizier.

\par

\WR{Keine Ahnung}, antwortete dieser wahrheitsgemäß. \WR{Sie sind außerhalb unserer Radarreichweite. Ich denke wir haben Sie nur deshalb aufgespürt, weil sie beim Austritt aus dem Hyperraum viel Energie abgegeben haben.}

\par

Plötzlich erschienen weitere Kontakte auf den Schirmen. Ebenfalls alle außerhalb der Radarreichweite. Es schien als würde eine ganze Flotte gerade den Hyperraum verlassen.

\par

\WR{Mister Garrison}, fragte der Kommandant aufgeregt den Navigationsbeauftragten, \WR{gibt es an der Stelle, an der diese Blickpunkte auftauchen, potentielle Hyperraumknoten?}

\par

Der Angesprochene schüttelte sofort den Kopf. \WR{Auf keinen Fall. Die Planetenkonstellationen schließen das eindeutig aus.}

\par

Dann traf die Staffel Abfangjäger auf die herannahenden Jäger, die kurze Zeit zuvor Rot vier abgeschossen hatten. Wie auf dem Radarbildschirm leicht zu erkennen war, eröffneten sie augenblicklich das Feuer. Zwei der Abfangjäger verschwanden sofort darauf von der Bildfläche. Im selben Moment klangen die Todesschreie der Piloten durch von der Kommunikationsstation herüber. Der Kommandant taumelte, wie von einer unsichtbaren Faust getroffen, benommen zurück und musste sich an einem Handlauf festhalten.

\par

Draußen im Weltall spielte sich im selben Moment eine tödliche Schlacht ab. Immer wieder blitzten helle Lichterscheinungen auf, die wahrscheinlich die Explosionen der Abfangjäger waren. Mittlerweile richteten sich etliche Blicke auf den völlig mitgenommen aussehenden Befehlshaber, der immer noch ungläubig durch die Fensterfront starrte.

\par

Endlich fing er sich und brüllte los: \WR{Kampfstationen! Alarmstufe eins! Sofort alle Geschütztürme bemannen und alle Jäger starten!}

\par

Hektisch wurden Befehle weitergegeben und einige Offiziere erhoben sich hastig von ihren Plätzen um die Anweisungen auszuführen.

\par

\WR{Sir}, meldete der Kommunikationsoffizier, \WR{Admiral Piet von der Starforce-Basis fordert einen Lagebericht.}

\par

Der Kommandant kam zu ihm heruntergerannt. \WR{Dann geben sie ihm einen. Und sagen Sie ihm, dass ein Angriff unmittelbar bevorsteht! Er soll die planetaren Schilde hochfahren und alle seine Jäger starten.}

\par

Ohne Verzögerung begann der Offizier an der Funkanlage eine Übertragung an den Admiral zu senden. Der alte Mann hatte wahrscheinlich senkrecht im Bett gestanden, als er vom Befehlshaber einer Raumstation den Befehl erhalten hatte, auf Alarmstufe zwei zu gehen.

\par

\WR{Weitere Bogeys im Anflug!}, berichtete der Chef des Radars. Die Angst schien ihm ins Gesicht geschrieben. \WR{Fünfundzwanzig weitere Jäger… Nein, nicht alles sind Jäger. Zehn davon sind groß genug um Bomber zu sein.}

\par

Der Kapitän stürmte zu seiner Station um sich ein Bild von der Lage zu machen. Tatsächlich näherte sich eine gewaltige Streitmacht der Raumstation. Eine kleine Gruppe spaltete sich jedoch ab und schien auf den Planeten zuzufliegen. Der Kommandant hoffte inständig, dass Admiral Piet ihn nicht für verrückt halten und tatsächlich seine Jäger starten würde. Von der Staffel Abfangjäger war mittlerweile nichts mehr zu sehen. Das konnte nur bedeuten, dass sie bereits alle tot waren.

\par

Der Kommandant fühlte sich, als sei seine Kehle zusammengeschnürt worden. Er wusste nicht genau, wer die Menschen gewesen waren, die gerade dort draußen ihr Leben verloren hatten. Aber es war ihm auch egal. Jeder einzelne seiner Untergebenen lag ihm am Herzen und es schmerzte ihn, wenn auch nur einer von ihnen sein Leben lassen musste.

\par

\WR{Wie geht der Starvorgang voran?}, fragte er seinen Einsatzleiter.

\par

Der antwortete ihm mit bestürzter Miene. \WR{Im Moment gar nicht. Laut Vorschrift sollte schon mindestens eine Staffel draußen sein aber da tut sich einfach nichts.}

\par

Dem Kapitän war auch klar wieso. Keiner seiner Leute hatte jemals wirklich gekämpft. Er selbst ebenfalls nicht. Und die spärlichen Trainingssimulationen hatten sie nicht annähernd auf das vorbereiten können, was gerade auf sie zukam.

\par

Und schon kamen die ersten der Unbekannten in Feuerreichweite. Keiner hatte eine Ahnung, wer sie waren. Die Jäger sahen so fremdartig wie angsteinflößend aus. Schüsse prallten gegen die Blocker der Raumstation und verursachten mit jedem Einschlag den Eindruck, als würde das Bild, das man sah, schwingen, als wären die Schilde eine Wasseroberfläche, die von Wellen durchzogen wurde.

\par

Obwohl die Raumstation Kaukasia genauso viele Geschütze wie ein leichter Träger besaß, feuerten nur fünf Türme auf die Angreifer, die sich nicht einmal große Mühe gaben, dem Beschuss auszuweichen.

\par

\WR{Egal, mit was sie auf uns schießen}, rief der Einsatzleiter freudig. \WR{Es durchdringt unsere Schutzfelder nicht.}

\par

Tosender Beifall erklang im Leitstand der Kaukasia. Aber irgendwie befürchtete der Kommandant, dass das noch lange nicht alles war, was die Angreifer zu Stande brachen. Und er hatte, wie sich bald herausstellte, Recht.

\par

\WR{Torpedoalarm!}, rief der Chef der Radarabteilung verzweifelt. \WR{Wahrscheinlich Nullzonensprengköpfe. Acht Stück, wurde eben von zwei der Bogeys abgeschossen.}

\par

Durch die Fenster des Leitstandes war leicht zu erkennen, wie die Geschosse herannahten. Die Sprengköpfe selbst waren noch nicht zu sehen aber ihre giftgrünen Kondensspuren zogen auf die Raumstation zu.

\par

\WR{Alle Geschütztürme auf die Torpedos ausrichten!}, befahl der Kommandant hastig. \WR{Wir müssen Sie unbedingt abschießen, bevor sie aufschlagen.}

\par

Einige der Strahlentürme drehten sich kurz darauf und eröffneten das Feuer auf die Torpedos. Aber wie zu erwarten gewesen war, waren die Geschosse sehr schwer zu treffen. Zu allem Übel rührten sich einige Türme überhaupt nicht.

\par

\WR{Was ist da los?}, fragte der Kommandant verärgert und deutete auf die Geschütze, die keine Anstalten machten, sich zu bewegen.

\par

Der Einsatzleiter musste nun seine Stimme schon deutlich erheben um noch gehört zu werden, so laut war es mittlerweile geworden: \WR{Die Hälfte der Türme ist nicht besetzt. Die Mannschaften müssten sie schon längst erreicht haben.}

\par

\WR{Jeder nicht bemannte Laserturm wird sofort auf Computersteuerung umgeschaltet}, bellte der Kommandant. \WR{Alle sollen versuchen, die Torpedos abzuschießen.}

\par

Kurz darauf ließ die Kaukasia einen Feuersturm los. Jedes Geschütz, dass im richtigen Schusswinkel stand, spie nun Strahlenentladungen in Richtung des Torpedopulks. Einer der Flugkörper wurde getroffen und explodierte sofort in einer weißen Glutblase, die einen weiteren Torpedo ebenfalls zur Explosion brachte.

\par

Doch der Schwarm begann sich zu zerstreuen. Jedes Torpedo nahm eine andere Stelle der Station ins Visier. Der Einsatzleiter hämmerte in einer schier unglaublichen Geschwindigkeit auf seine Instrumente ein um die Türme neu auszurichten.

\par

Kurz darauf wurde ein weiterer Torpedo getroffen, doch die anderen kamen allmählich näher. Es war ruhig geworden im Leitstand der Raumstation. Fast alle Augen blieben auf die herannahenden Flugkörper gerichtet, die wie ein Hagel aus Pfeilen auf die Station heran gerast kamen.

\par

Jetzt endlich erhob sich die erste Staffel an Verteidigungsjägern vom Startdeck und brauste los. Mit Entsetzen beobachtete die halbe Brückenmannschaft, wie gleich zwei der Flieger kurz nach dem Start abgeschossen wurden. Egal wer die Angreifer waren, ihre Waffen waren denen der Starforce mindestens ebenbürtig, so schnell wie sie die Blocker der Jäger durchschlagen hatten.

\par

Der Kommandant wusste bereits in dem Moment, in dem er das Gleichgewicht verlor und hart auf den Boden prallte, was die Erschütterung ausgelöst hatte. Es waren die Torpedos, die gerade auf die Schutzfelder geprallt und explodiert waren.

\par

Ein Blick durch die Fensterfront zeigte ein völlig verzerrtes und waberndes Weltall, so stark waren die Entladungen in den Schilden gewesen. Doch dann war die Sicht mit einem mal wieder klar. Der Kommandant, der bereits übel ahnte, was das bedeutete, erhob sich und fragte seinen Einsatzleiter: \WR{Wie ist der Zustand der Schutzfelder?}

\par

\WR{Ausgefallen}, antwortete der Mann schlicht und hängte an. \WR{Weitere Torpedos im Anflug.}

\par

Alle Blicke huschten zu den Fenstern. Tatsächlich waren nun wieder etliche grüne Schweife zu sehen, die auf die Station zuhielten.

\par

\WR{Die Jäger sollen Sie abschießen. Mit den Kanonen erwischen wir niemals alle!}, donnerte der Kommandant, heftig mit seiner eigenen Stimme ringend.

\par

Der Einsatzleiter sah ihn verzweifelt an und antwortete nicht weniger mutlos: \WR{Welche Jäger, Sir?}

\par

Einen Wimpernschlag später explodierte das Hangardeck, das von drei Torpedos getroffen worden war. Zuerst hüllte ein Feuerball die quaderförmige Erweiterung der Station ein, bevor das ganze Gebilde von innen heraus zu zerplatzen schien.

\par

Bevor irgendjemand etwas sagen konnte, traf ein Torpedo den Leitstand. Erneut wurde der Kommandant, genauso wie jeder andere, von den Beinen geholt. Verletzt wurde allerdings niemand. Jede Kommandozentrale eines Schiffes oder einer Station der Starforce verfügte über Brückendeflektorschilde, die einen eigenen Generator besaßen. So waren die Brücken beinahe unangreifbar.

\par

Allerdings würde das niemandem auf der Kaukasia helfen, wenn der Rest der Station in die Luft fliegen würde. Der Kommandant richtete sich ein weiteres Mal auf und ließ seinen Blick über den Leitstand gleiten. Einige Konsolen waren durch den Einschlag zerstört worden und überall schwelten Dampfwolken von brennenden Trümmern umher.

\par

Die Gesichter der Brückenbesatzung spiegelten alle gleichermaßen Todesangst und pure Verzweiflung wieder. Einige Offiziere schienen Panik zu bekommen und rannten schreiend davon. Andere standen nur wie gelähmt da und starrten mit ungläubigem Blick aus dem Fenster.
