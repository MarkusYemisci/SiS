Morten sah die Transporter einen nach den anderen in einem hellen Lichtstrahl im Hyperraum verschwinden und er konnte zu ersten Mal an diesem Tag zumindest ein wenig aufatmen. Wenigstens ein paar tausend Menschen hatten ihre Flucht von Pollux Primus überlebt. Doch Mortens Erleichterung verflog schnell, als ihm klar wurde, dass sein eigenes Überleben und das seiner Freunde keinesfalls feststand.

\par

Sein Blick ging zur \EN{Regenvogel}. Der Träger hatte gerade eine Salve von vier mit einem Hochleistungssprengstoff geladenen Raketen auf die anrückenden Großkampfschiffe abgefeuert und wendete nun hart. Die Angreifer schienen davon unbeeindruckt und setzten ihre Verfolgung fort. Als leichter Träger verfügte die \EN{Regenvogel} einfach nicht über die Offensivkraft, um anderen Schiffen wirklich gefährlich zu werden. Das mussten ihre Jäger besorgen.

\par

Zumindest hatte es der Träger geschafft, die flüchtenden Transporter vor schlimmerem zu schützen. Nun prasselte jedoch ein Hagel von Strahlen auf das Schiff ein, dass bereits einige Schäden hatte hinnehmen müssen.

\par

Eines der Triebwerke brannte und verlor ständig Treibstoff, was selbst für Morten aus einiger Entfernung zu sehen war. Lieutenant Wallander Stimme erklang auf allen Frequenzen. Er gab zügig aber deutlich die Reihenfolge der Landungen durch. Und allein die Tatsache, dass er bereits nach kurzer Zeit damit fertig war, zeigte Morten, dass von den Jägern der \EN{Regenvogel} längst nicht alle zurückkommen würden.

\par

Er selbst war einer der letzten, die landen würden. Sein leichter Jäger eignete sich gut zur Verteidigung von größeren Schiffen und wurde somit bis zuletzt gebraucht. Da sich die \EN{Regenvogel} jedoch auf direktem Kurs zur Hyperraumroute befand, hieß das aber auch, dass er zum Landen nur einen Versuch hatte.

\par

Unerwartet dröhnte ein lautes \WR{Nein! Ihr verdammten Schweine!} durch den Äther. Morten erkannte diese Stimme schnell. Sie gehörte Dexter Hennington. Nur wenig später fuhr er fort: \WR{\EN{Regenvogel}, hier ist Hennington. Wir haben Maddeux verloren. Ich nähere mich so schnell ich kann, aber springen Sie notfalls ohne mich!}

\par

Morten schluckte. Er hatte Simon Maddeux zwar nicht gekannt aber dennoch ließ ihn diese Neuigkeit nicht kalt.

\par

Die \EN{Regenvogel} beschleunigte noch einmal. Die ersten Jäger begannen bereits mit der Landung und die beiden Korvetten fuhren ihre Hyperantriebe hoch. Sie hatten einige Jäger abgewehrt, die anscheinend die Wega-Route hatten belagern wollen. Morten hatte sich auch an diesem Kampf beteiligt, doch seine Schüsse hatten die Angreifer kaum gerührt. Die Jäger waren größer gewesen, als jene, die er und Kevin abgeschossen hatten. Für sie mussten die Treffer aus den Bordkanonen der Haie nicht mehr als Mückenstiche gewesen sein. Doch die Geschütze der beiden Korvetten hatten sie schließlich vertrieben.

\par

\WR{Torpedoalarm!}, gellte es aus Mortens Kopfhörern. Die Stimme gehörte Lieutenant Wallander. \WR{Grau drei. Sie sind am nächsten. Versuchen Sie ein Abfangmanöver.}

\par

\WR{Oh Gott}, flüsterte der Gerufene eher zu sich selbst. Dann antwortete er schlicht \WR{Verstanden.} und gab Vollgas. Sein Nachbrenner-Capezin war fast komplett verbraucht, doch es reichte noch, um ihn an der \EN{Regenvogel} vorbeischießen zu lassen. Die Gefechtskontrolle übermittelte ihm einige Zieldaten, die auf seinem Radar als gelbe Markierungen erschienen. Es waren vier Geschossen, die mit hoher Geschwindigkeit auf das Trägerschiff zu rasten. Die Verteidigungsjäger mussten bereits gelandet sein, sonst hätten sie mit ihren Antitorpedo-Geschossen sicher bessere Chancen gehabt.

\par

Morten wünschte diese Vögel wären noch nicht unten, doch ihm war klar, wieso sie bereits auf dem Flugdeck standen. Die \EN{Regenvogel} hatte mittlerweile eine derart hohe Geschwindigkeit erreicht, dass sie viel zu langsam waren, um als letztes zu landen.

\par

Wieder ertönte Wallander Stimme. Diesmal auf dem allgemeinen Gefechtskanal. \WR{Grau zwei. Landen Sie endlich! Wir haben keine Zeit für diesen Blödsinn!}

\par

Morten seufzte, als er Kevins Stimme hörte. \WR{Vergessen Sie es! Mortens braucht Hilfe.}

\par

\WR{Bringen Sie ihren Vogel sofort runter! Das ist ein Befehl!}, schrie Wallander wütend. Doch Morten erkannte auf dem Radar, dass Kevin nicht daran dachte zu gehorchen und sich ebenfalls den Geschossen nährte.

\par

Ohne die Zeit zu haben, mit Kevin diskutieren zu können, peilte Morten den ersten Torpedo an. Er erkannte ihn bereits mit bloßem Auge und zog daher den Beschleunigungshebel nach hinten, um nicht selbst mit dem Geschoss zu kollidieren. Dann ließ er seine Kanonen sprechen. Kurz darauf verschwand der Torpedo in einem hellen Gleißen.

\par

Die anderen drei Geschosse zogen in einem Sekundenbruchteil an ihm vorbei, so dass er lediglich noch ihre grünlichen Kondensstreifen erkennen konnte. Hastig wendete er seinen Jäger und folgte dem Pulk an Torpedos. Kurz darauf explodierte ein weiterer und Kevins Siegesschrei ließ Morten wissen, wer den Treffer gelandet hatte.

\par

Mit dem letzten Treibstoff seiner Nachbrenner folgte Morten letzten beiden Geschossen. Sie hatten die \EN{Regenvogel} schon fast erreicht und bei den deutlich sichtbaren Schäden würde ein weiterer Einschlag das Ende des Trägers bedeuten.

\par

Morten fluchte, als ihm klar wurde, dass sich die Torpedos genau zwischen ihm und der \EN{Regenvogel} befanden. Wenn er vorbeischoss, würde er sein eigenes Schiff treffen. Aus demselben Grund feuerte die \EN{Regenvogel} auch nicht selbst auf die Geschosse.

\par

Morten begann seine letzte verbliebene Rakete eine Erfassung vornehmen zu lassen.

\par

\WR{Ich hab den rechten fast!}, rief Kevin und ließ seine Waffen sprechen. Doch die Strahlen verfehlten ihr Ziel und schossen auch nur knapp an der \EN{Regenvogel} vorbei. Während Lieutenant Wallander vor Wut in sein Mikrofon brüllte, schlug der Torpedo ein. Das Leuchten der Explosion war so hell, dass sich Morten die Hand vor die Augen halten musste. Erst nach einigen Augenblicken verschwand das grelle Licht und erlaubte einen Blick auf die getroffene \EN{Regenvogel}. Die Steuerborddüsen, die am Rumpf und nicht an den Flügeln angebracht waren, brannten lichterloh und der austretende Rauch vernebelte Morten schnell die Sicht. Der Träger schien Schlagseite zu bekommen und abzudriften.

\par

Als ein entsprechender Ton Morten wissen ließ, dass seine Rakete ihr Ziel gefunden hatte, wartete er keine Sekunde und ließ sie starten. Wenige Meter, bevor der letzte Torpedo die \EN{Regenvogel} traf, erreiche die Rakete ihr Ziel und ließ es verglühen.

\par

\WR{Morten, wir müssen nebeneinander landen}, ließ ihn Kevin über Funk wissen, der auf dem Radar erkannt hatte, dass alle anderen Jäger bereits gelandet waren und das Schiff die Hyperraumroute bereits fast erreicht hatte.

\par

\WR{Verstanden}, antwortete Morten. \WR{\EN{Regenvogel}, wir kommen viel zu schnell rein. Holen Sie schon mal den Feuertrupp.}

\par

Inständig hoffend, dass das Flugdeck des Trägers vom Torpedoeinschlag verschont geblieben war, begannen die beiden Haie mit ihrem Landeanflug. Morten behielt den Landestreifen immer im Auge. Nun hatte er nicht mehr das ganze Deck für sich alleine und musste sich möglichst links halten. Zumindest war die \EN{Regenvogel} nun selbst sehr schnell, was ihm die Landung zumindest so lange erleichtern würde, bis er sich im Schwerkraftfeld des Trägers befinden würde. Bereit, den Beschleunigungshebel sofort zurückzuziehen, flog Morten in das Flugdeck ein.

\par

Er erschrak, als er sah, dass die Ränder des Decks voller Jäger standen. Tukarevs Mannschaft hatte es wohl nicht rechtzeitig geschafft, die Vögel ganz von der Landebahn zu bekommen. Sein Jäger setzte hart auf und Morten spürte sofort, dass er viel zu schnell war. Doch dieses mal würde er keine zweite Chance bekommen. Er schaltete seine Antriebe aus und zündete die Bremsraketen.

\par

Das Radgestell unter seinem Flieger brach auseinander und der Hai prallte mit dem Bauch auf den blanken Boden. Massenweise Funken hinter sich her ziehend, schlitterte Mortens Schiff über das Deck und rutschte unkontrolliert davon, bis es gegen Dexter Henningtons schweren Jäger prallte, den dieser kurz zuvor gelandet haben musste.

\par

Die Haltegurte schienen sich in Mortens Brust zu schneiden, als sein Hai schlagartig zu stehen kam. Das Letzte, was Mortens sah, bevor die \EN{Regenvogel} in den Hyperraum sprang, war eine brennende Heckdüse von Dexters Jäger.
