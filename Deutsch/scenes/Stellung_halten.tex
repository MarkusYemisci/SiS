\WR{Alle Backbordseite auf die \EN{Astaroth} abfeuern!}, befahl Captain Fiscale. Sie musste schreien, um noch gehört zu werden, so laut ächzten die Schutzfelder mittlerweile unter jedem Treffer auf. \WR{Er kommt uns viel zu nah und wir wollen nicht seine Feuerkraft zu spüren bekommen.}

\par

Der Einsatzleiter gab die Anordnungen an die Kanoniere weiter, die augenblicklich neue Zieldaten eingaben. Die großen Glasmonitore zur linken zeigten, wie sich die Vektoren nun alle auf die Flanke der \EN{Astaroth} konzentrierten. Kaum spieen die Kanonen donnernd Feuer.

\par

\WR{Jägerstaffel im Anflug}, meldete Petrarca an niemanden speziell gerichtet. \WR{Staffel Charlie soll sofort beidrehen und sie abdrängen.}

\par

Captain Fiscale sah angestrengt aus dem Fenster. Schnell erkannte sie einen der Verteidigungsjäger aus der Charlie-Staffel. Hoffnungslos veraltete Maschinen mit vier verschiedenen Tragflächen. Zwei große am Heck und zwei kleinere auf der Höhe des Cockpits. Alle hingen voller Raketen.

\par

Verteidigungsjäger der \EN{Wachhund}-Klasse waren langsam aber dafür wendig, schlecht gepanzert aber dafürhervorragend bewaffnet. Ihre Aufgabe war es, nah bei einem Schiff zu bleiben und alles zu zerstören, was den anderen durch die Lappen ging. Sie waren der Libero im Jägerarsenal der Starforce.

\par

Und außerdem einer der wenigen Jäger, die Platz für zwei Piloten boten. Notfalls konnten sie von einer Person geflogen werden, doch hatte es Vorteile, auch den zweiten Sitz zu besetzen. Der Copilot konnte die Heckkanone bedienen und außerdem einen Überblick über mögliche Ziele behalten.

\par

Umso schlimmer war es, wenn einer von feindlichen Salven getroffen wurden und brennend in die Leere trudelte. Natalia Fiscale zwang sich an etwas anderes zu denken, als die beiden Piloten, die gerade ihr Leben verloren hatte, als das Schiff explodierte.

\par

Mit der Meldung von Elshe Schwarzschild folgte schnell eine ungewollte Ablenkung. \WR{Torpedoalarm!}

\par

Nun wurde Fiscale klar, weshalb der zweite Shutek-Zerstörer permanent die Steuerbordseite der \EN{Regenvogel} beharkte. So wurden die Schutzfelder aufgeweicht und feindliche Torpedos würden deutlich mehr Schaden anrichten.

\par

\WR{Steuerbordkanonen und Raketenwerfer auf diesen Torpedoschwarm konzentrieren!}, rief Fiscale, doch es war längst zu spät.

\par

Die \EN{Regenvogel} hatte sich weit in die feindlichen Reihen hinein gewagt. Sie und ihre Eskorte lieferten sich nun einen harten Austausch von Strahlenkanonen aus ihren Hauptgeschützen. Die Geschosse, die nun auf den Träger zurasten waren zu klein, um von einem anderen Schiff zu stammen. Einer oder mehrere Bomber mussten sich unbemerkt in Feuerposition gebracht haben. Doch noch bevor Fiscale die Übeltäter auf dem Übersichtshologramm ausmachen konnte, schlugen zwei der Torpedos auf die Schutzfelder ein, während weitere im Abwehrfeuer untergingen.

\par

Der Blitz der Explosion war so heftig, dass sich fast jeder auf der Brücke schützend die Hände vor die Augen hielt. Einigen wurde dies zum Verhängnis, als sie durch die folgende Erschütterung von den Füßen geholt wurden und zu Boden stürzten.

\par

Über das Grollen der Detonation hinweg, war das Brechen des Glases zu hören. Einer der Torpedos musste auf die Brückenschilde gestoßen sein. Diese griffen auf einen eigenen Generator zurück und waren somit unabhängig von den restlichen Schutzfeldern. In der Theorie sollte somit die Kommandozentrale eines Schiffes sogar den Treffer eines Marschflugkörpers mit Nullzonensprengkopf aushalten können. Doch in der Realtität hatte es der Torpedo geschafft, die Fenster der Brücke zu beschädigen.

\par

Der sternförmige Riss breitete sich unter lautem Knarzen immer weiter und weiter aus, bis sich eine Schutzwand automatisch ausfuhr und den beschädigten Bereich überdeckte.

\par

Captain Fiscale war eine der ersten, die gestürzt waren und wieder auf die Beine kamen. Ihre Ohren klingelten noch und sie musste mehrfach die Augen zusammenkneifen, um wieder ansatzweise klar sehen zu können.

\par

\WR{Commander Samad, sprechen Sie mit dem Maschinenraum!}, befahl sie. \WR{Ich brauche einen Schadensbericht.}

\par

Hastig sah sie sich nach ihrem ersten Offizier um, als keine Antwort erklang. \WR{Abdel!} Sie schrie unvermittelt, als sie ihren Stellvertreter regungslos am Boden liegen sah. Er musste nur wenige Meter neben dem gebrochenen Fenster gestanden haben. Um ihn herum verteilt lagen etliche Glassplitter.

\par

Fiscale schreckte zurück, als sie Samads Körper zu sich drehte. Sie wurde bleich und ihr Gesicht verzog sich zu einer Fratze der Panik, als sie die fingerlange Scherbe sah, die sich dem ersten Offizier in den Schädel gebohrt hatte. Sein ganzer Leib war von Schnitten überzogen und seine leeren Augen sahen ins Nichts.

\par

Ihr erster Reflex war es, nach einem Sanitäter zur rufen. Doch ihr Verstand zwang sie, zunächst nach Samads Puls zu fühlen. Schnell gestand sie sich ein, dass er weder armete, noch einen Herzschlag hatte.

\par

Ihr Blick ging durch eines der noch unbeschädigten Fenster. In die Steuerbordflanke war ein ein enormer Bruche gerissen worden. Trümmerteile und große Mengen der Atmosphäre wurden ins All hinaus geblasen. Nach einem solchen Treffer gab es vermutlich dutzende Verletzte. Jeder Sanitäter an Bord würde nun dringend gebraucht werden.

\par

Samad war tot. Der Mann, der Natalia Fiscale praktisch während ihrer gesamten Karriere begleitet hatte und ihr beigestanden hatte, wie es vermutlich nicht einmal ein Ehemann gekonnt hatte, existierte nicht mehr. In einem Moment war noch da gewesen. Und nun war er weg. Das war die schwer zu fassende aber nicht zu ändernde Realität.

\par

Die Kommandantin zog ihre Uniformjacke aus und warf sie über den leblosen Körper ihres ersten Offiziers.

\par

\WR{Alle Mann die Brücke verlassen!}, rief Einsatzleiter Petrarca. \WR{Raus hier und runter zum Hangar. Wir nutzen die Flugzentrale als provisorische Brücke.}

\par

Fiscale schoss förmlich an das Geländer und brüllte zu den Offizieren. \WR{Befehl aufgehoben! Wir halten hier oben die Stellung!}

\par

\WR{Madam, wir können nicht bleiben!}, rief Petrarca. \WR{Die Brückeschilde sind schwer getroffen. Eine weitere Salve vertragen wir nicht.}

\par

\WR{Wir bleiben und kämpfen, verdammt noch mal!}, raunzte die Kommandantin ihn an. \WR{Schalten Sie den Kondensator für die Brückeschilde ab, laden Sie ihn auf und fahren Sie ihn wieder hoch!}

\par

\WR{Dann sind wir schutzlos, Captain!}, mahnte der Einsatzleiter.

\par

Fiscale wollte zunächst einen Blick auf das Übersichtshologramm nehmen, gab dies aber wieder auf, als dieses immer wieder flackerte. Stattdessen konsultierte sie ihren Handcomputer.

\par

\WR{Funk. Teilen Sie der \EN{Impervious} mit, dass sie sich zwischen diesen Zerstörer und uns manöverieren soll, bis unsere Brückenschilde wieder bei din prodin sind.}

\par

\WR{Aye, Madam}, antwortete ihr Lieutenant Wallander und gab widerwillig die Anweisungen durch. Auch ihm war klar, dass dies mit einem enormen Risiko für die \EN{Impervious} einherging, die nun die Ersatzzielscheibe stellen musste.
