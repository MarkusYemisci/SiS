Laura ließ eine Gruppe feiernder Menschen hinter sich. Einer von ihnen hatte sie wild gestikulierend zum Tanzen aufgefordert. Sie hatte die Person ignoriert und war weitergegangen. Die Hände in den Taschen und den Blick auf den Boden gerichtet.

\par

Es kam ihr vor, als hätte sie ihre Wohnung schon seit Ewigkeiten nicht mehr gesehen, als sie endliche ihre Haustüre erreichte. So vieles hatte sich verändert, seitdem sie es verlassen hatte, um zur Arbeit zu fahren und auf einen nach wie vor unbekannten Datendieb angesetzt zu werden.

\par

Ihr Fingerabdruck und die Gesichtserkennung der Türkamera öffneten ihr den Zugang. Als sie eintrat, nahm sie den gewohnten Geruch war, der sie seit jeher im Treppenhaus erwartete. Wie so vieles zuvor an diesem Tag, ließ auch er ihr Herz schwer werden. Für einen Moment musste sie sich sogar zwingen, tief einzuatmen, so sehr schnürte sich ihre Kehle zu.

\par

Sie betrat ihre Wohnung, legte ihre Sachen ab und stieg unter die Dusche. Wieso sie das tat, wusste sie selbst nicht. Genauso wenig, wieso sie sich noch die Mühe machte, die Wohnung zumindest oberflächlich zu säubern und ein paar Dinge wegzuräumen. Sie tat das nicht für sich. So viel wusste sie. Ihr einziger, zweifelhafter Vorteil daraus war, dass sie noch etwas Zeit gewann.

\par

Der Untergang der Sonne erinnerte sie daran, dass diese abgelaufen war. Sie hatte früher oft am Fesnter gesessen~-- egal wo auf der Welt~-- und sich das Schauspiel angesehen. Das Schwinden der Sonne war ein besonderer Abschied, denn es war schon mit dem Ausklingen des letzten Lichtstrahls sicher, dass sie wiederkommen würde. Ein Trost, der früher einmal wahr Wunder gewirkt hatte. Jede Nacht musste enden und auch nach dem ausdauerndsten, schlechten Wetter folgte früher oder später der blaue Himmel.

\par

Laura zwang sich vom Fenster weg und in ihr Schlafzimmer. Sie setzte sich auf ihr Bett und dachte noch einmal gründlich nach. Sollte sie noch jemandem schreiben? Ihre Familie würde es nicht verstehen, das wusste sie. Abschied zu nehmen und zu versuchen, es zu erklären, konnte nichts ändern. Han kam ihr in den Sinn. Er hatte Angehörige zurückgelassen. Sie könnte ihnen mitteilen, was sie wusste. Aber dann würde sie ihnen auch erklären müssen, dass sein Tod ihre Schuld war. Und wenn sie nur spezifisch genug würde, dann brachte sie Hans Familie ebenfalls in Gefahr.

\par

Er hatte gewusst, worauf er sich einließ. Sie hatte es ihm erklärt und sogar angemerkt, dass sie Klaus verdächtigte. Vielleicht hatte er selbst Recherchen weitere angestellt und war über etwas gestolpert, dass ihm noch mehr ungewollte Aufmerksamkeit eingebracht hatte.

\par

Schnell gestand sie sich ein, dass es nichts zu sagen gab. Der Term hatte vermutlich bereits alle Beweise vernichtet. Dass sie die Leiche nicht aus dem Krankenhaus geschafft hatte, bevor sie es sich angesehen hatte, gab ihr bereits Rätsel auf. Vielleicht, so mutmaßte sie, war es auch ein Test für Klaus Loyalität gewesen. Wenn, dann hatte er ihn bestanden. Der positiv denkende Erfolgsmensch.

\par

Laura seufzte nicht, als sie daran dachte. Auch sonst reagierte ihr Unterbewusstsein kaum. Sie war müde. Also legte sie sich in ihr Bett und zog die Decke über sich. Die Wärme, die sich schnell einstellte, gab ihr ein Gefühl von Geborgenheit, dass sie nun gut gebrauchen konnte.

\par

Sie hatte gehört, dass Menschen, die bereits ein ähnliches Vorhaben realisiert hatte, wie sie, Panik bekommen hatten. Dass sie es sich im letzten Moment noch einmal anders überlegt hatte, nur um dann in Qualen statt in Frieden zu gehen.

\par

Auf ihrem Nachttisch lag bereits ihr Buch bereit. Sie schlug es auf und ließ es eine Melodie spielen, von der sie nicht wusste, woher sie stammte. Viele Informationen waren während und nach der Seuche verloren gegangen. Es konnte überall her stammen. Vielleicht die Begleitmusik eines Films oder auch nur etwas, das in einem Aufzug gespielt wurde. Am Ende spielte es keine Rolle. Die Klänge beruhigten sie und nur darum ging es.

\par

Ihr nächster Griff ging in ihre Nachttischschublade. Darin fand sie ihren Plan B, von dem sie sich jede Nacht vergewissert hatte, dass er noch da war. In dem kleinen Döschen fand sie zwei Pillen. Eine davon war ein starker Betablocker, die andere enthielt ein Gift. Nicht so potent wie manch anderes. Selbst Zyankali, dass schon vor Jahrhunderten zum selben Zweck benutzt worden war, wirkte schneller. Aber dafür wirkte die Pille so sanft wie sicher.

\par

Es gab nichts mehr zu tun. Nichts mehr zu sagen. Laura war müde und wollte schlafen. Mit einem kleinen Schluck Wasser~-- mehr hatte sie gar nicht in das Glas gefüllt~-- nahm sie die Pillen und knipste ihre Nachttischlampe aus.
