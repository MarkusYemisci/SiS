Das aufgeraute Metall der illegalen Waffe funkelte an zahllosen Stellen im fahlen Licht einer alten Schreibtischlampe. Diese bildete einen kuriosen Kontrast zu der todbringenden Gerätschaft in Laura Gethas Händen. Sie besaß die Lampe, seit sie vier Jahre alt war und dementsprechend kindlich waren auch die Motive auf ihrem Schirm. Früher hatte ihr der lachende, sichelförmige Mond am besten gefallen. Die Sonne, die bis zu ihren nicht vorhandenen Ohren grinste, war ihr schon damals ein wenig zu aufdringlich gewesen. Heute kam es Laura vor, als würde sie sie verhöhnen.

\par

Es war ihr egal. Vor einigen Wochen war sie noch so zornig gewesen. Nun fand sie es ironisch. Sie hatte damals geglaubt, es ginge ihr schlecht. Nun fühlte sie fast nichts mehr. Weder Zorn noch Trauer. Und doch ging es ihr jetzt so viel schlechter.

\par

Laura Gethas, Agentin des Konglomerats, war stets um ihr Äußeres bemüht gewesen. Sie hatte sich immer wieder gesagt, dass sie nicht eitel war. Mit jeder vergangenen Stunde, die sie aufgewandt hatte, um in einem der durchaus exklusiveren Bekleidungsgeschäfte ein Hemd oder eine Hose auszuwählen, hatte sie jedoch an dieser Aussage mehr gezweifelt.

\par

Nun konnte sie mit Recht behaupten, nichts auf ihr Auftreten zu geben. Sie wusch sich noch. Kämmte sich die Haare. Aber dabei beließ sie es. Warum sie sich überhaupt noch die Mühe gab, wusste sie nicht. Ihre Wohnung, die sie bislang stets akribisch rein gehalten hatte, erinnerte nun eher an ein Klassenzimmer vor den Sommerferien. Hin und wieder aufgeräumt, aber alles andere als sauber. Sie konnte sich nicht erinnern, wann sie zuletzt den Boden gefegt oder die Fenster geputzt hatte.

\par

Das letzte mal, dass sich ihre Wohnung in einem wirklich guten Zustand befunden hatte, war gewesen, als Sarah sie verlassen hatte. Schon damals hatte sie keinerlei Drang verspürt, ihr Zuhause sauber zu halten. Aber es war ihr als eine Art Verpflichtung ihrer Partnerin gegenüber erschienen, den Ort so wohnlich wie möglich zu machen. Sie hatte erwartet, dass Sarahs Weggang ihr viel größere Schmerzen bereiten würde. Vielleicht hatte sie es kommen sehen. Vielleicht war sie gar nicht mehr in der Lage so viel zu empfinden.

\par

Dinge konnten passieren, auch wenn sie nicht sein durften. Die Welt würde sich weiterdrehen. Sie konnten jeden treffen und sie trafen früher oder später auch jeden. Akzeptieren. Das schien das Patentrezept gegen Tragödien zu sein. Laura hatte diesen Ratschlag von jedem Mitglied ihres immer kleiner werdenden Freundeskreises bekommen. Wie die Erde sein. Einfach weiterdrehen, immer weitermachen. So als sei sie angetrieben von einer unsichtbaren Kraft, die einem auf die unmöglichsten Umwege schicken konnte. Aber am Ende kannte sie nur ein einziges Ziel.

\par

Der Tod hatte Laura immer bemerkenswert wenig Angst gemacht. Sie war weder eine Anhängerin einer alten noch neuer Religion und glaubte demnach auch nicht an ewiges Leben oder Wiedergeburt. Mit dem Tod war alles vorbei. Bewusstsein, Denken, Empfinden, Wollen. Alles, was einen Charakter ausmachte hörte auf zu existieren. Das alles erschien zunächst furchteinflößend. Doch für sie hatte der Tod immer zum Leben dazu gehört. Und sie starb auch nicht zum ersten mal.

\par

Laura hatte niemals erwachsen werden wollen. Sie hatte sich davor gefürchtet, nicht mehr sie selbst zu sein. Das Kind Laura Gethas, die einzige Person, die sie sich damals hatte vorstellen können, zu sein. Ihr waren Erwachsene immer seltsam vorgekommen, denn sie hatten Wert auf Dinge gelegt, mit denen sie nichts hatte anfangen können. Und das Schöne im Leben, wie zu Spielen, oder ein und denselben Film einhundert mal anzusehen und jedes mal aufs neue zu bangen und zu hoffen, schien den Alten völlig abzugehen. Sie hatten immer Termine, Arbeit zu erledigen. Jede Minute zählte so wie jede Münze. Alles musste rationiert werden. Zeit und Geld. Bis zum Schluss im Schwimmbad zu bleiben oder das dritte Eis zu kaufen und zu essen war nicht drin. Nur zu Arbeiten, um dann Abends auf dem Sofa zusammenzusacken, auszuruhen und wieder zu arbeiten, schien wichtig.

\par

So hatte Laura nicht werden wollen. Und sie hatte geweint, als ihr klar geworden war, dass sie in diesem Fall gar keine Entscheidung treffen konnte. Sie war erwachsen und selbst zu einer Person geworden, die es verstand, mit Geld und Zeit zu haushalten. Eine Frau, die Verantwortung für sich und andere übernahm. Ein reifer Mensch.

\par

Sie erinnerte sich zwar noch an ihre Kindheit, an ihre Angst und den Unwillen vor dem Erwachsenwerden. Aber sie konnte mit diesen Erinnerungen nichts mehr anfangen. So, als seien es Informationen, die genausogut aus dem Gehirn eines Anderen stammen konnten. Das Kind Laura hatte aufgehört zu existieren. War sie nicht damals schon gestorben? Starben Menschen nicht jeden Tag aufs Neue? So sehr sie auch wollte, sie fand keine Antwort auf diese Fragen. Vielleicht gab es so etwas wie einen ganz bestimmten Menschen, einen eindeutigen Charakter, nicht. Möglicherweise war er eher wie Treibholz auf einem Fluss, das irgendwann hineingefallen war, sich andauernd veränderte und dann morsch wurde und verfaulte.

\par

Die Szenerie schien Laura seltsam passend zu sein. Von den Straßen drang schwach organgefarbenes Licht herein und gab ihrer Wohnung etwas Unwirkliches. Es war, als würden sich nicht einmal mehr die Laternen Mühe geben, ihr zu gefallen. Die Welt schien sie nicht mehr haben zu wollen. Zuerst waren ihre Freunde, von denen sie eine nicht unwesentliche Anzahl gehabt hatte, immer ruhiger geworden. Jeder einzelne von ihnen hatte versucht, sie aufzubauen. Doch mehr und mehr war sie sich wie eine Maschine vorgekommen, die nicht mehr richtig funktionierte und die man reparieren musste. Aber so destruktiv und falsch ihr Freundeskreis ihre Gefühle auch fand, so berechtigt kamen sie ihr selbst vor. Ihre Depression~-- als das hatte man ihren Zustand irgendwann diagnostiziert~-- schien für alle eine Krankheit zu sein. Ihr kam sie eher wie die logische Konsequenz vor. So wie Blut, dass zwingend aus einer offenen Wunde dringen musste.

\par

Laura hatte dieselben Parolen immer wieder gehört. Von ihren Freunden, ihrer Familie, Arbeitskollegen, ihrem Therapeuten. \WR{Du musst einfach weitermachen, loslassen, dich ablenken, mal was anderes sehen!} Egal welcher Tipp es auch war, fast immer beinhaltete er das Wort \Wr{einfach}. Aber genau das war es nicht. Laura wusste, dass sich niemand aus ihrem Umfeld vorstellen konnte, was sie durchmachte. Erwachsene Menschen trugen Verantwortung. Und für die meisten war Verantwortung so etwas wie eine Garantie auf Erfolg. Eine Versicherung, dass man genau das tat, was man sollte oder was nötig war und dabei keinesfalls scheiterte. Und doch passierte es. Manchmal verlor dann ein Unternehmen eine Menge Geld. Hin und wieder stürzte sich jemand auch selbst in den Ruin und war dann für den Rest seines Lebens auf Essensrationen angewiesen. Vielleicht endete Versagen auch einmal darin, dass eine Traumaufzeichnung oder ein Theaterstück beim Publikum durchfiel und Tausendschaften enttäuschter Zuschauer würden sich nie wieder ein Werk des gescheiterten Künstlers ansehen. Sein Ruf war verloren.

\par

In einigen Fällen führte Versagen jedoch auch dazu, dass Menschen starben.

\par

Als Jugendliche hatte Laura oft an Selbstmord gedacht. Sie war immer ein aufgewühlter Mensch gewesen und als emotionale Person oft an den Rand eines Nervenzusammenbruchs geraten. Der Gedanke daran, einfach ein paar Pillen zu schlucken und dem Druck ein Ende zu bereiten, war ihr damals sehr einladend vorgekommen. Doch eine Hoffnung auf sonnigere Tage, und wenn das nicht, dann ihr schierer Selbsterhaltungstrieb hatten stets dazu geführt, dass sie derartige Ideen nicht einmal ansatzweise in die Tat umgesetzt hatte.

\par

Als eine der wenigen tatsächlichen Empfindungen der letzten Wochen war sie überrascht gewesen, wie leicht es ihr gefallen war, ihren Plan ins Rollen zu bringen. Sie hatte keinen Moment gezögert, als sie die Verbindungen ausgenutzt hatte, die ihre Arbeit mit sich brachte, und sich eine Pistole gekauft hatte. Sie war nicht billig gewesen und hatte einiges an Finanzierung bedurft, denn es handelte sich um eine Strahlenwaffe. Im Vergleich zu Gift oder chemisch getriebenen Geschossen bot sie etliche Vorteile. Keine Reue nach dem Einnehmen und auch keine Kugel, die im Gehirn stecken blieb und einem für den Rest des Lebens sabbernd an einen Rollstuhl fesseln würde. Eine Strahlenwaffe war eine saubere Sache. Sie konnte sich praktisch überall treffen, wenn sie aus nächster Nähe schoss und wäre dann sofort Tod.

\par

Laura dachte zum zigsten mal darüber nach, einen Abschiedsbrief zu schreiben. Nicht für sich selbst, sondern für die Menschen die ihr einmal nahe gestanden hatten. Doch sie gestand sich ein, dass es sie nicht mehr interessierte, wie sehr oder wie wenig man um sie trauern würde. Es spielte einfach keine Rolle mehr.

\par

Nach einem kurzen Aufatmen hob sie die Waffe wie selbstverständlich. Es ging alles so einfach. Sie erwartete mehr Widerstand, so etwas wie ein letztes Aufbäumen ihres Lebenswillens. Doch er kam nicht. Alles was sie fühlte war Erleichterung, als sie die kalte Mündung an ihrer Schläfe spürte. Einmal noch den Zeigefinger anspannen und es wäre vorbei. Kein Nachdenken mehr. Keine Laura, die jeden Tag, jede Stunde und jede Sekunde dasselbe Gesicht vor Augen sah. Sie war sich nicht sicher, welche der alten Religionen den Satz \WR{Ruhe in Frieden}, geprägt hatte. Doch nun kam er ihr mehr als passend vor.

\par

Aus einem Grund, den sie selbst kaum verstand, zögerte sie noch. Sie verstand nicht, was sie noch zurückhielt. Ihre Zeit war gekommen. Laura musste einfach sterben. Daran führte kein Weg vorbei. Es musste passieren.

\par

Ein lauter Knall beendete die Situation. Er kam von der Tür. Laura seufzte. Einen Moment lang dachte sie darüber nach, einfach trotzdem abzudrücken. Doch sie brachte es nicht übers Herz, denn ihr war klar, wer vor ihrer Tür stand. Sie konnte es Klaus einfach nicht antun, ihre Leiche als erster zu finden. So, als wäre sie dann noch anwesend, um ihm in die Augen sehen zu müssen. Ein absurder Gedanke, wie sie sich sagte.

\par

Also versteckte sie die Pistole in einer Schreibtischschublade und ging zur Tür. Sie ließ sich dabei Zeit, denn eigentlich wollte sie ihren Spiegel bei der Polizei gar nicht sehen. Klaus Rensing war Kommisar und arbeitete in dieser Eigenschaft oft mit dem Geheimdienst zusammen. Immer dann, wenn es um Fälle ging, in denen man die Agenten mit ihren weitreichenden Befugnissen nicht zu viel unbeobachteten Freiraum lassen wollte. Eine der vielen Lösungen, um die einzelnen Organe des Konglomerats miteinander zu verzahnen.

\par

Klaus verstand. Er war dabei gewesen. Gerade deshalb fragte sich Laura oft, wie er seine unbekümmerte Art weiter beibehalten konnte. Vielleicht war es seine Maske oder seine Art, mit dem umzugehen, was passiert war. Jendenfalls schien er damit erfolgreicher zu sein, als Laura. Selbst wenn der Bell-Fall auch an ihm nicht vorbeigegangen war, ohne Spuren zu hinterlassen.

\par

Die Augen des Polizisten suchten so schnell wie gründlich Lauras Wohnung ab. Sie wusste, wonach er Ausschauh hielt. Waffen, Alkohol und Drogen. Genau das, was er bei einer Person in Lauras Lage erwarten würde. Doch die Pistole sah er nicht und Laura war einfach nicht die Art von Mensch, die ihre Sorgen mit Rauschmitteln betäubte. Irgendwann begegnete sie seinem misstrauischen Blick mit der Ruhe der Verzweiflung.

\par

\WR{Aha, da hat jemand aufgeräumt.} Klaus setzte ein Lächeln auf, das zumindest äußerlich zusammen mit seinem breiten Kiefer echt und selbstsicher wirkte. Laura gab ihm keine Antwort. Wenig überraschend war ihr nicht nach plaudern zumute. Zumindest verzichtete er auf Fragen nach ihrem Befinden. Auf eine gewisse Art, verstand Klaus sie besser als alle anderen. Er wusste, dass es ihr dreckig ging und dass sie weder reden wollte noch konnte. Ihm war klar, was sie durchmachte und das sie alleine wieder auf die Beine kommen musste. Der einzige Fehler, den sie ihm unterstellte, war, dass er glaubte, sie würde das schaffen.

\par

\WR{Hast du Lust, ein bisschen joggen zu gehen?}, wollte er wissen und versuchte, beiläufig zu klingen. So, als wäre er zufällig auf das Thema gekommen. Doch obwohl ihr Partner schwer zu lesen war, wusste Laura, dass er nicht ohne Grund bei ihr vorbeigekommen war. Er wollte sich um sie kümmern. Eine frühere Variante ihrer selbst hätte sich darüber gefreut. Auch wenn sie schon damals ihre liebe Not damit gehabt hatte, mit dem stämmigen Mann, den Klaus Rensing darstellte, mitzuhalten.

\par

\WR{Nein, ich habe keine Lust und ich bin auch nicht dafür angezogen}, antwortete sie kurz angebunden. Klaus grinste nun. \WR{Das wäre auch ein Riesenzufall, oder?}

\par

Sie seufzte. \WR{Du wirst nicht aufgeben. Du wirst mich so lange bequatschen, bis ich mitkomme.} Er nickte nur. \WR{Na gut. Sollst du haben. Du wirst schon merken, was du davon hast.}

\par

Warum schrie sie ihn eigentlich nicht an? Sie hatte so ziemlich jeden, der ihr nahe stand verletzt. Mit Worten, die so gezielt auf die Schwächen und Ängste ihrer Nächsten abgezielt hatten, dass sie es selbst kaum glauben hatte können, so gemein sein zu können. Es hatte eine unglaubliche Gefühlskälte gebraucht, um ihr Umfeld endgültig zu vergraulen. Und wenn sie ehrlich war, brachte sie es einfach kein weiteres mal fertig, jemanden aus ihrem Leben zu vertreiben. Für einen Moment wünschte sie, es würde Klaus nicht geben.

\par

\WR{Ich geh danach noch ins O'Bannons}, fuhr er fort. \WR{Corna steht im Halbfinale, falls du das noch nicht mitgekriegt haben solltest~-- Und irgendwas sagt mir, dass du das nicht hast.}

\par

\WR{Joggen und \textit{nur} Joggen. Oder du gehst alleine}, entgegnete Laura in einem Ton, der keinen Widerspruch zuließ. \WR{Die verlieren sowieso}, hängte sie frustriert an. Sofort war sie wütend auf sich. Keine drei Minuten vorher hatte sie sich eine Pistole an den Kopf gehalten und war Augenblicke davor gestanden, den Abzug zu ziehen. Und jetzt spürte sie endlich etwas und das war Frustration über ein ungleiches Fußballspiel.

\par

\WR{Wie du meinst. Zieh dich um, ich bin sowieso spät dran}, forderte ihr Partner.

\par

\WR{Du hast ja keine Ahnung}, dachte sie sich. Dann ging sie mit einem Seufzen in ihr Badezimmer, in dem ihre Sportkleidung nach wie vor auf dem Waschapparat lag.

\par

Widerwillig streifte sie sich ihr Trikot über. Doch am Ende hatte sie noch ihr ganzes Leben, um zu sterben.
