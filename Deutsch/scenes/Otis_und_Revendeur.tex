\WR{Nocht etwas mehr Wein?}, fragte der Ober und mied dabei den Blick des Präsidenten.
Henry Otis schätzte die Professionalität des Personals.
Sein präferiertes Restaurant, das \Wr{Haus am Wald}, ließ ihm selten Grund auch nur zur leisesten Kritik.
Er liebte alles Klassische und der kleine Nebenspeisesaal, für den die meisten Menschen selbst mit dem nötigen Kleingeld monatelang warten mussten, wenn sie überhaupt auf die Bewerberliste kamen, war genau in seinem Sinne eingerichtet.
Überaus stilvoll.
Die massigen, stählernen Säulen waren dezent vergoldet, die Wände mit hölzernen Verschlägen versehen.
Jede der Lampen, deren gläsernen Schirme sich gerade so sehr voneinander unterschieden, dass sie als handgemacht erkannt wurden, spendeten ein Licht, dass im Konstrast der verschneiten Gegend besonders warm wirkte.
Der weinrote Teppichboden war ebenfalls kein maschinelles Fabrikat.
Und er war alt.
Er stammt aus der Zeit, kurz nachdem sich die Menschheit von der Seuche erholt hatte und hatte einst das Präsidentenamt in Tunis geziert.
Dort hatte er gelegen, bis er aufwändig restauriert worden war.

\par

Otis musste nur die Hand heben, um dem Ober mitzuteilen, dass er keinen weiteren Wein wünschte~-- auch, wenn dieser wirklich vorzüglich schmeckte.
Seine Stille war keine Unhöflichkeit.
Die Utnerhaltungen an den anderen Tischen waren allesamt ausgesprochen leise und er wollte nicht unangenehm auffallen.
Nachdem auch die Dame ihm gegenüber abgewunken hatte, verbeugte sich der Kellner und ging auf dem weichen Teppich davon, ohne das geringste Geräusch zu verursachen.

\par

\WR{Schwer zu glauben, dass wir immer noch in der Union sind.}
Thérèse Revendeur klang nachdenklicher als sonst.
\WR{Kein Mensch darf in seinem Wert über einem anderen stehen.}
Sie kannte wie Otis seblst, große Teile der Verfassung auswendig.
\WR{Aber sieh dir diesen Ort an.
Wir werden bedient wie die Könige, während andere Leute auf Essensmarken angewiesen sind.
Es muss niemand hungern aber eine Garantie auf gutes Essen gibt es auch nicht.}

\par

Otis schnaubte.
\WR{Das gerade von \textit{dir} zu hören ist erstaunlich.
Glaubt man nicht auf Kreuzpunkt Primus daran, dass jeder im Leben genau das bekommen soll, was er sich auch erarbeitet?
Dass es nicht die Aufgabe des Systems ist, dass jeder dasselbe Resultat, sondern dieselbe Chance bekommt?}

\par

Revendeur trank den letzten Schluck ihres Weines.
Sie war bei weitem nicht so edel angezogen, wie ihr Gegenüber.
Weder ihr unauffälliger Pullover, noch ihre dunkle Hose, noch die Reste des Schneematsches an ihren schon oft getragenen Füßen, ließ erahnen, wer sie war.
Selbst ihr Gesicht war diesbezüglich die reinste Irreführung.
Ihre Frisur, ein mädchenhafter Bob, ließ sie harmlos und bieder erscheinen, was ihre zierliche Figur noch unterstrich.
Und ihre großen Augen gaben ihr einen gütigen Ausdruck.
Henry Otis wusste es besser.

\par

\WR{Ich sage nicht, dass es einen fundamentalen Fehler in der Philosophie der Union gibt}, führte Revendeur aus.
\WR{Aber es ist mit Sicherheit eine Fassade.
\Wr{Alle Menschen sind gleich viel wert},
\Wr{Niemand darf sich über einen anderen erheben},
\Wr{Unsere Würde ist nicht an Äußerlichkeiten gekettet},
\Wr{Materieller Besitz ist nicht entscheidend}.
All das sind sehr prätentiöse Aussagen.
Schau dir die Union an, Henry.
Wir sind nach wie vor die erbarmungslose Affenherde, in der die Alphatiere sich auf den höchsten Baum kämpfen.
Die Union kann diesen Umstand nur besser verschleiern.
Ich habe dein Interview gesehen.
Davon hast du nichts gesagt.}

\par

Otis lehnte sich zurück und verschränkte die Arme vor der Brust.
Sofort bereute er diesen Zug, da er seiner Gesprächspartnerin damit gerade bestätigt hatte, in die Defensive gegangen zu sein.
Bestimmt hatte sie eine erstklassige Ausbildung darin erhalten, Körpersprache lesen zu können.
Aber er bezeweifelte, dass sie eine solche überhaupt gebraucht hätte, denn sie war ein Naturtalent darin, andere Menschen einzuschätzen.
\WR{Du vergisst so ziemlich alles, was die Union bisher erreicht hat.
Der Aufbau von Corna und Wega zum Beispiel.
Als die Wirtschaft dort noch von der unsichtbaren Hand gelenkt wurde, haben die Menschen gehungert.
Jetzt hat nicht nur jeder etwas zu essen, sondern auch Arbeit.
Ich würde sagen, die Hierarchie in der Affenherde ist etwas flacher geworden.}

\par

\WR{Alles Nebensächlichkeiten}, wischte Revendeur das Argument beiseite.
\WR{Zumindest bezogen auf das, worüber ich rede.
Wusstest Du, dass mein Patenkind einmal hier essen gehen wollte?
Zusammen mit ihrer damaligen Verlobten.
Sozusagen als ruhige Variante eines völlig überlaufenen Hochzeitsmals.
Sie verdient nicht schlecht, das kann ich dir sagen.
Sie kam sogar persönlich her, um einen Platz zu reservieren.
In ihrem schönsten Kleid.
Abgelehnt.}
Otis rollte mit den Augen.
Sie sprach einfach weiter, denn seine Reaktion zu diesem Zeitpunkt war nicht von Interesse.
\WR{Wenn ich hierher komme~-- angezogen, so wie jetzt~-- dann bekomme ich sofort einen Platz.
Und ich habe weniger Geld, als Sabrinè.
Aber ich bin die Kanzlerin von Kreuzpunkt Primus, darum sitze ich überall in der ersten Reihe.}

\par

\WR{Hat dieser Witz auch eine Pointe?}, fragte Otis gereizt.

\par

\WR{Sie soll dir sagen, dass Macht und Erfolg Hand in Hand gehen und nicht davon abhängen, in welchem politischen System man gerade lebt.
Es ist egal, ob man der Neanderthaler in der Höhle oder die einflussreichste Person in der Union ist.
Am Ende zählt, wie gut man seinen Willen durchsetzen kann.}

\par

Otis war der Appetit vergangen. Trotzdem zwang er sich dazu, einen Bissen seines Steaks zu nehmen.
In einer Sache hatte Thérèse recht.
Sein Essen stammte zwar genauso wenig von einem echten Tier, wie die Masse, die man bei der öffentlichen Essensausgabe bekam.
Aber man hatte sich deutlich mehr Mühe gegeben, es so wirken zu lassen und er hatte nicht das geringste Bedürfnis, jemals wieder Nahrung von geringerer Qualität zu sich zu nehmen.
Und dabei wollte er auch nicht an viele Millionen Menschen denken, die genau das mussten.

\par

\WR{Ja, wir Menschen werden immer Machtkämpfe austragen}, gab ihr Otis recht.
\WR{Aber wir sind nicht die einzigen Kontrahenten.
Es gibt da einen Konflikt, den du gerne ignorierst, weil er die nicht in den Kram passt.
Seit der Seuche hat sich einiges getan.
Der Pragmatismus ist nicht mehr…}

\par

\WR{Oh bitte nicht dieser Unsinn!}
Es war sonst nicht an Revendeur ihre Gesprächspartner zu unterbrechen.
\WR{Ich weiß, worauf du hinaus willst.
Pragmatismus gegen Idologie.
Ich sage dir, was ich von der Lesart unserer Historiker halte.
Die Menschen haben damals nach der Seuche nicht ihre Ressentiments gegeneinander fallen gelassen, weil sie als Gesellschaft so edel geworden sinst, dass sich einander zu helfen plötzlich idiologisch zwingend geworden ist.
Sie haben das aus reinem Eigennutz getan.
Man hilft als Europäer nicht, ein Afrikanisches Land aufzubauen, weil man es für das Richtige hält.
Die Leute damals haben ihre Feindbilder vergessen, weil es damals der effizienteste Weg des Überlebens war.}

\par

Otis lächelte matt.
\WR{Und genau da liegst du falsch.
Notwendigkeit und Überzeugung damals Frieden geschlossen.
Darauf wollte ich hinaus.
Idiologie war beim Wiederaufbau genauso wichtig, wie pragmatische Überlegungen.}
Er bemerkte sofort, wie wenig Revendeur ihm zustimmte.
\WR{Glaub es, oder lass es sein.
Aber, wenn es anders gewesen wäre, dann hätten wir heute nicht die Union.}

\par

\WR{Und unsere Großeltern hätten sich nicht bekriegt?}, entgegnete Revendeur sofort.
\WR{Kaum, dass die Menschheit die Seuche hinter sich gelassen hat, ging alles wieder seinen gewohnten Gang.
Zwei große Machtblöcke.
Die Mächtigen und Einflussreichen entwickeln gegenläufige Interessen.
Krieg.}

\par

Das Steak schmeckte saftig.
Von alleine wäre Otis niemals darauf gekommen, dass es anders als die Kartoffeln, ein synthetisches Produkt war.
\WR{Und was wurde aus der Macht?}
Otis Frage war natürlich rein rhetorischer Natur.
\WR{Die Erdallianz wurde genauso von seinen Bürgern umgeworfen, wie das Commonwealth.}

\par

Revendeur begann nun selbst zu lächeln und wirkte mit einem mal so gefährlich, wie sie auch war.
\WR{Das \textit{Militär} hat den Umsturz bewältigt.
Was mich zu dem Grund bringt, weswegen ich heute mit dir sprechen wollte.}

\par

Ein erneutes Augenrollen des Präsidenten folgte.
\WR{Und ich habe mir schon gedacht, du möchtest wahrmachen, was du nach unserer Trennung gesagt hast.
Dass wir in gutem Kontakt bleiben.}

\par

Revendeur schwieg.
Nicht, weil ihr keine Antwort eingefallen wäre, sondern, weil keine davon ihren Vorstellungen von Höflichkeit entsprach.
Ihr Gesicht blieb ausdruckslos.

\par

\WR{Also, bitte}, forderte Otis.

\par

\WR{Vernetz sein ist alles}, begann sie.
\WR{Ich habe vor kurzem von Horatio Balsato gehört.}

\par

Otis verachtete ihn.
Genauso, wie jeden anderen Firmenboss, dem er jemals begegnet war.
Nicht, dass sie so unzugänglich für Argumente waren, wie etwa die Offiziere des Konglomerats.
Sie waren gedanklich meistens deutlich offener.
Und bereit, sehr viel Entgegenkommen zu zeigen.
Allerdings nur, sofern ihnen die Kompensation angemessen erschien.
Viele von ihnen waren der lebende Beweis dafür, dass Revendeurs Ansichten etwas für sich hatten.
Diese Menschen taten, was richtig war.
Aber nur, wenn es einen Nutzen für sie hatte.

\par

Die Kanzlerin fuhrt fort.
\WR{Er hatte eine Bitte.
Eines seiner Schiffe ist ihm wohl abhanden gekommen.
Ausgesprochen peinlich für ein Unternehmen mit einer Reputation wie sie die PSG hat.
Und er glaubte, nur die Navy könnte es wiederfinden.}

\par

\WR{Hat er es denn wiedergefunden?}, wollte Otis wissen.

\par

\WR{Noch nicht}, war Revendeurs prompte Antwort.
\WR{Aber die Starforce hat ihre Suche noch nicht beendet.}

\par

Otis warf seine Serviette verärgert auf den Tisch.
\WR{Natürlich nicht. Ich habe auch keine Aktion wie diese genehmigt und wurde auch nicht gefragt.}

\par

Revendeur lächelte nun nur noch schärfer.
\WR{Und das wirst du auch nicht mehr müssen.
Horation kam zu mir, weil er sich erhofft hat, dass ich dich dazu umstimmen könnte, die Navy den Job machen zu lassen.
Nicht die zuständigen Stellen unserer Rettungskräfte.
Aber dann muss ihm jemand gesagt haben, dass er dein Einverständnis nicht braucht.}

\par

\WR{Natürlich braucht es die}, gab Otis verwirrt zurück.
\WR{Militäraktionen dürfen nicht durchgeführt werden, wenn nicht der Präsident oder der Rat zustimmt.}

\par

\WR{Er hat wohl eine Gesetzeslücke entdeckt}, erklärte Revendeur.
\WR{Du musst bei deiner nächsten Justizreform gründlicher sein.}
Sie beobachtete den Präsidenten zunächst vorsichtig.
Auf den ersten Blick erschien er fast paralysiert aber sie wusste, wie es nun in ihm arbeitete.
Sollte er sofort zu seinem Buch greifen, um zu bestätigen, was sie sagte?
Oder sollte er vielleicht einfach so tun, als würde ihn das alles wenig beeindrucken, nur um dann möglichst schnell die Rechnung zu bezahlen und wütend nach Hause zu stürmen.
So entschied sie sich, noch einen drauf zu setzen.
\WR{Darum habe ich dir erzählt, was der Begriff \Wr{Macht} für mich bedeutet.
Man übergeht dich einfach.
Du lässt dir auf der Nase herumtanzen.
Von Militärs, Geschäftsleuten…
Du kannst noch so sehr so tun, als würdest du an die Gleichstellungsnormativen der Union glauben.
Aber, wenn Burns dich einfach so umgeht, dann \textit{muss} dich das einfach sauer machen.}

\par

Revendeur lehnte sich zurück und begann im Geiste Wetten darauf abzuschließen, wie die unmittelbare Reaktion ihres Amtskollegen aussehen würde.
Was er jedoch danach tun würde, darüber war sie sich sicher.