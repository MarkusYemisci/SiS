Die Aussicht war wunderschön. Und auch Morten hätten sie so gesehen, wenn er sich nicht gerade in einem Gefecht befunden hätte. Derzeit verhielt sich das Radar zwar ruhig, doch nahm er sich trotzdem keine Zeit, um die Wolkendecke zu betrachten, über die er gerade hinweg flog.

\par

Am Horizont ging gerade der Stern Kreuzpunkt unter. Er sah der Sonne recht ähnlich, außer, dass er noch ein wenig weißer wirkte, als sein Gegenstück über der Erde.

\par

In dieser Höhe wirkte alles so friedlich. Der schier endlose Himmel. Die flauschigen Wolken. Der Stern, der zumindest warm wirkte. Aber die Sensoren in Mortens Jäger eine Außentemperatur von minus zweidutzend Grad Celsius.

\par

Im Cockpit selbst war es warm. Morten schwitzte schon seit einer Weile und wünschte sich, er könne einfach die Haube öffnen.

\par

\WR{Hier Claudius fünf.} Die Stimme des Marineinfanteristen war über Funk nun deutlich schlechter zu verstehen. Zwar waren die Cockpits der Falken gut schallisoliert, aber ganz und gar ließ sich das Dröhnen der Luftreibung nicht aussperren. \WR{Wir beginnen mit dem Landeanflug. Die Maximus-Truppe nimmt sich die Südfront vor. Wir haben berichte über Bodenfeuer. Ich hab zwar keine Ahnung, wie die Shutek jetzt schon Boden-Luft-Geschütze einrichten konnte, aber das spielt jetzt auch keine Rolle. Halten Sie die Augen offen.}

\par

\WR{Verstanden, Claudius}, bestätigte Anna Farley. \WR{Dex, dreh ab und gib der Maximus-Welle Deckung. Gute Jagd.}

\par

\WR{Halt dich ran, Baby. Ich liege in den Abschüssen schon weit vorne}, prahlte Dexter Hennington.

\par

Wenig später meldete sich Kevin zu Wort: \WR{Oh Mann, wo hat der denn Zählen gelernt.}

\par

Morten sah zu seiner Linken. Eine große Anzahl Jäger drehte ab. Im Orbit hatten sich Staffeln der \EN{\EN{Artiglio} de Leone} dem roten und blauen Geschwader angeschlossen. Ihre grün-weiß markierten Maschinen wirkten teilweise schon sehr mitgenommen. Einige der Piloten mussten bereits seit Stunden in der Luft sein.

\par

\WR{Statusaktualisierung über die Minvera}, meldete Elshe Schwarzschild auf dem allgemeinen Gefechtskanal. \WR{Die Station trifft auf die Atmosphäre. Achten Sie auf Trümmerteile!}

\par

Kevin klang ehrlich besorgt, als er sagte: \WR{Ich hoffe, es haben zumindest ein paar Leute da raus geschafft.}

\par

Morten nickte still. Dann fragte er: \WR{Deine Freundin. Wo lebt sie?}

\par

\WR{In Edhor Peak. Ziemlich mittig auf dem Hauptkontinent. So wie's aussieht, ist sie zumindest vorerst sicher.}

\par

\WR{Gut.}

\par

Mortens Blick ging wieder zum Radar. Derzeit waren zwar etliche eigene Jagdmaschinen und Landetransporter zu sehen, aber Bogeys waren nicht zu erkennen. \WR{Ich wünschte, wir hätten ein paar Aufklärer hier}, sinnierte er. \WR{Kein Feindkontakt, aber ich bin sicher, die sind da.}

\par

\WR{Rote Staffel}, begann Anna Farley, \WR{lösen und angreifen! Aber bleibt bei den Transportern, Jungs! Diese Kästen müssen sicher auf den Boden und auch wieder weg.}

\par

Der Befehl richtete sich auch an einige Jäger der ehemaligen Alpha-Staffel der \EN{\EN{Artiglio} de Leone}, die kurzerhand vom roten Geschwader einverleibt worden war. Genauso wie zwei Maschinen von einem planetaren Stützpunkt. Ihre weiß-grünen Markierungen wiesen sie als planetare Jäger aus.

\par

Morten ließ seinen Fieger Anna folgen. Sie kippte ihre Maschinen geschmeidig zur Seite weg und schoss kurz darauf durch die Wolken. Als er ebenfalls in die dichte Decke aus Eiskristallen eindrang, sah er sofort nur noch weiß. Verärgert darüber, daran nicht gedacht zu haben, drehte er hastig den Kontrast seines Zielmonokels auf Maximum, während Eiskristalle auf seiner Frontscheibe schmolzen.

\par

Nur in der schematischen Anzeige sah er den Transporter Claudius fünf neben sich her fliegen. Es war ein Wunder, dass sich das massige Schiff überhaupt in der Luft halten konnte. Neben den Plasmaturbinen und Bremsraketen waren nun auch Gegenkraft-Antriebe am Werk, die verhinderten, dass der Transporter wie ein Fels vom Himmel fiel. Sehr manövrierfähig war er dabei allerdings nicht.

\par

Unvermittelt meldete sich eine Stimme, offensichtlich von der Front aus, per Funk. \WR{An alle Landeeinheiten!} Im Hintergrund waren Schreie und Maschinengewehrfeuer zu hören. \WR{Feindliche Truppen nicht zu stoppen. Fallen nach Kreuzstadt zurück. Landezone direkt vor die Stadtgrenze verlegen!}

\par

Anna Farley überlegte zunächst kurz. Vermutlich überprüfte sie auch die Identität des Senders. Dann befahl sie: \WR{Rote Jungs, ihr habt's gehört. Neuer Kurs direkt auf Kreuzstadt zu. Ich wollte dass Pendel sowieso schon immer mal sehen!}

\par

Einige der Piloten quittierten den Befehl, während Morten seinen Jäger wortlos auf den neuen Kurs ausrichtete. Aus der Höhe der Wolkendecke war es kein großer Unterschied, nur wenige Grade.

\par

Der Höhenmesser in Mortens Cockpit fiel rasant. Er zeigte nur noch wenige Trinmeter. Bald schon würde er die Nase hochziehen müssen, um nicht mit vollem Schub in den Boden zu schießen. Fast genauso schnell fiel die Treibstoffanzeige. Ein Jäger verbrauchte innerhalb einer Atmosphäre noch deutlich mehr Energie. Er musste nicht nur gegen den Luftwiderstand angehen, sondern vergab auch viel Leistung an die Blocker, die nun ständig Partikel vom Rumpf fernhalten mussten.

\par

Unvermittelt kreischte Mortens Raketenalarm auf. \WR{Beschuss vom Boden!}, schrie er in den Funk, sobald er erkannt hatte, von wo aus das Geschoss stammte. Dann ließ er seinen Jäger hart über den Backbordflügel abkippen.

\par

Die Sicht war nicht viel besser. Über Kreuzstadt tobte ein Schneesturm und der Himmel verbarg sich hinter einer dicken, grauen Wolkendecke. Darum erkannte er nur schemenhaft, wie die Rakete seinen Jäger nur knapp verfehlte. Dem Warnton nach zu urteilen, hatte es sich bei dem Detektor um einen Wärmesucher gehandelt. Verglichen mit der Umgebung, würden seine Plasmaturbinen strahlen wie die Sonne. 

\par

Mit ein paar schnellen Tastendrücken verlegte Morten den Fokus der Blocker auf das Heck seines Schiffes und stellte kurzfristig seinen Antrieb aus. Sofort begann sein Jäger einen plumpen Gleitflug. Die leichten Schutzfelder hatten nicht nur den Zweck, feindliche Geschosse abzuhalten, sondern auch, Emissionen zu zerstreuen. Würden sie die Infrarotstrahlen einer Plasmaturbinen nicht in tausend verschiedene Richtungen umleiten, wären diese ein leichtes Ziel.

\par

Das Piepen erstarb. Die Rakete hatte aufgegeben. Sie war frontal auf ihn abgefeuert worden, was eine Zielerfassung erschwerte. Der Antrieb war mit Abstand der beste Punkt zum Aufschalten für eine wärmesuchende Rakete.

\par

Dann schien der Boden auf Morten zu zu rasen. Mit der wenigen Vorsicht, die er aufzubringen im Stande war, drückte er den Triebwerkshebel durch und sah den Schnee vor sich aufbrausen, während er so dicht über einen Hang preschte, dass sein Kollisionsalarm wie wild aufheulte.

\par

Als er sein Schiff wieder gefangen hatte, sah er Kreuzstadt vor sich. Die Metropole war genauso wunderschön, wie ihm viele schon oft gesagt hatten. Ihre Hochhäuser waren zumeist uncharakteristisch aufwändig verziert für eine Stadt, die so viel Wert auf Effizienz legte. Genauso überflüssig wie die schnörkelreichen Fassaden, die Türme und die unverhältnismäßig großen Fachwerkhäuser erschien die Stadtmauer.

\par

Der plusförmigen Grundriss der Stadt war aus Mortens Perspektive nicht zu erkennen. Momentan nahmen allerdings sowieso eher die riesenhaften Spinnen seine Aufmerksamkeit ein.