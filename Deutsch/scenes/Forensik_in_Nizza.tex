Han Liao war ein Computerfachmann, so klischeehaft, wie man ihn sich nur vorstellen konnte. Seine langen Haare wirkten ungepflegt und waren anscheinend schon seit einigen Tagen nicht mehr gekämmt worden. Der breite Oberlippenbart passte nicht zu den sonst feinen Zügen des jungen Mannes. Sein Laborkittel war hier und da befleckt und darunter trug er ein verwaschenes Holzfällerhemd. Seine Jeanshose musste eine Form von modischem Protest darstellen, da es unter den fünfunddreißig Milliarden Bürgern der Unio Terrestris vielleicht eine Handvoll gab, die diese Überbleibsel von dem, was einst als Neuzeit bekannt gewesen war, noch trugen.

\par

Vor einigen Jahren hatte Liao sein Studium in Ingenieurswissenschaften mit Bestnoten bestanden. Daher hatte es niemanden verwundert, dass er mit seiner Bewerbung bei Riethink, einer der besten, nichtstaatlichen Computerentwicklungsfirmen, Erfolg gehabt hatte. Aus unerfindlichen Gründen hatte ihn die Arbeit dort doch schnell gelangweilt und er hatte gekündigt. Wenig später war er von der Polizei angeworben worden, um die Inhaber ehemaliger Capital Fellowship Konten zu identifizieren. Die Arbeit hatte ihm derart gefallen, dass er sich für eine unbefristete Anstellung beworben und sich schnell ins Forensikzentrum in Nizza hochgearbeitet hatte.

\par

Laura Gethas hatte Liaos Labor schon einige male betreten. Es wirkte genauso unaufgeräumt und chaotisch, wie der Mann war, der darin arbeitete. Han mochte keine Gesellschaft, darum hatte man ihm einen kleinen Arbeitsbereich zugestanden, in dem er alleine werken konnte. Das Labor war ein Anbau im Innenhof des Forensikzentrums, dessen Wände komplett aus Glas bestanden. Hin und wieder sah man jemanden von der Straße kommen. Die Hände voll mit Proben, die untersucht werden sollten. Sonst war die Gegend jedoch sehr ruhig.

\par

Das Forensikzentrum stand~-- wie sehr viele andere staatliche Gebäude auch~-- auf einem Berg über der Stadt, in der Nähe eines Museums, das vor der Seuche einmal ein Observatorium gewesen war. Im Gegensatz jedoch zu den Häusern der Stadt war es nicht im Glas-und-Stahl-Stil, sondern in Anlehnung an die barocke Bauweise errichtet worden. Es stammte noch aus dem Routenkrieg und war zu jener Zeit Sitz des Geheimdienstes der Erdallianz gewesen.

\par

Obwohl es Nacht war, schien ganz Nizza zu feiern. Überall stiegen Feuerwerke in den Himmel und selbst auf dem Berg war noch die laute Musik zu hören, die unten in den Straßen gespielt wurde.

\par

\WR{Was feiern die denn?}, fragte Klaus Rensing und gähnte sogleich. Er hatte den ganzen Tag darum gekämpft, das Besuch des Hackers in Nizza untersuchen lassen zu dürfen. Am Ende hatte er einige Gefallen eingefordert und sein Ziel erreicht.

\par

\WR{Keine Ahnung}, entgegnete ihm Laura Gethas ebenso müde. \WR{Die Fußball WM ist vorbei und irgendwer hat gewonnen.}

\par

Ihr Partner seufzte. \WR{Was für eine Überraschung. Wahrscheinlich der Mars. Ich will nicht wieder das grinsende Gesicht von deren Trainer im DDV sehen.}

\par

\WR{Dann schau es dir eben nicht an}, entgegnete Laura trocken.

\par

\WR{Fußball. Ein Idiotensport}, murmelte Han Liao mit seiner hohen Stimme. \WR{Man sollte meinen, dass es nach sechsdin Jahren langsam langweilig wird, zweidutend zehn Leuten dabei zuzuschauen, wie sie einer Kugel hinterherrennen.}

\par

Klaus Rensing seufzte ein zweites mal. \WR{Keine Chance. Manche Dinge bleiben immer interessant. Sich mit Frauen zu treffen zum Beispiel.}

\par

Liao ignorierte diese Provokation. Er hätte dem Polizisten erklären können, warum er es für klüger hielt, alleine zu leben, glaubte aber nicht daran, dass Klaus Rensing seine Argumente einsehen würde. Stattdessen reichte er ihm das Buch des Hackers.

\par

\WR{Es hat eine Weile gedauert. Aber ich habe einige Daten extrahieren könne, bevor sich der Speicher selbst gelöscht hat.}

\par

\WR{Warum ist das denn passiert?}, wollte Laura Gethas wissen.

\par

Liao zuckte mit den Schultern. \WR{Ich bin nicht ganz sicher. Aber ich glaube der Countdown auf dem Bildschirm des Geräts war Teil eines Sicherungssystems. Wahrscheinlich hat euer Verdächtiger das Gerät so eingestellt, dass es seinen Speicher löscht, wenn er nicht alle vierundzwanzig Stunden einen Code eintippt.}

\par

Die Agentin nickte. \WR{Das hatten wir schon befürchtet. Was haben wir denn rausbekommen?}

\par

Han Liao lächelte stolz und wies auf einen Computer, der inmitten einer Schraubenzieherarmee, optoelektronischen Platinen, offenen Handcomputern und allerlei nicht identifizierbaren Maschinen auf einem Schreibtisch stand. Der Bildschirm zeigte den näheren Umkreis von Sucre, der Hauptstadt des ehemaligen Boliviens. Laura erkannte die Gegend schnell wieder, denn sie hatte sie gemeinsam mit ihrer ehemaligen Freundin schon öfter bereist. Das bergige Umland und die wenige Vegetation, die auf dieser Satellitenaufnahme zu erkennen war, kamen ihr sehr bekannt vor.

\par

\WR{Anscheinend konnte sich euer Mann ins die Krypta Scientia einklinken, ohne das der Zugriff registriert wurde. Was gelinde gesagt, ziemlich beeindruckend ist}, erklärte Han Liao. \WR{Aber auf seinem Buch waren noch einige Standorte gespeichert, bei denen er sich eingeklinkt hat. Das war leider auch so ziemlich das einzige, was ich bisher dekodieren konnte.}

\par

\WR{Es gibt also noch mehr}, dachte Klaus Rensing laut.

\par

Han bejahte. \WR{Aber ich konnte leider nur einen Bruchteil der Speicherbank auslesen, bevor sie sich gelöscht hat. Die Methode, einen Speicher von außen abzutasten, anstatt ihn irgendwo anzuschließen, ist noch nicht die stabilste Möglichkeit.}

\par

Laura hatte den beiden nur mit einem Ohr zugehört. Sie interessierte sich eher für die markierten Standorte um Sucre herum, die auf der Satellitenaufnahme zu sehen waren. Alle befanden sich mehr oder weniger auf einem Kreis, mit der Stadt im Zentrum. Sie lagen alle in kleinen Vororten, die es zweihundert Jahre davor noch gar nicht gegeben hatte. Die also alle in einer Ära errichtet worden waren, in der man versucht hatte, eine Kommunikationsleitung in jeden auch noch so kleinen Ort zu verlegen, anstatt ihn über drahtlose Wege zu verbinden.

\par

\WR{Ich glaube, unser Hacker wohnt in Sucre}, verkündete sie und unterbrach damit eine Grundsatzdiskussion zwischen ihrem Partner und Han Liao über die Vor- und Nachteile seiner Untersuchungsmethoden. \WR{Um zu hacken geht er in eines der nahen Dörfer und nutzt die direkten Leitungen, die es da noch gibt.}

\par

\WR{Warum sollte er das tun?}, fragte Klaus. \WR{In Großstädten wie Sucre gibt es genauso Leitungen, die auch noch viel schneller sind. So wie in Freiburg.}

\par

Laura kannte die Antwort. \WR{Die Verteilerknoten in Großstädten anzuzapfen ist viel riskanter. Sie regeln den Datenverkehr zwischen den Speicherzentren und den Drahtlos-Antennen und sind entsprechend gut gesichert. Er greift sie wahrscheinlich nur an, wenn er muss. Die Direktleitungen, die es in vielen Dörfern noch gibt, sind zwar viel langsamer aber dafür ist es auch einfacher, einzubrechen.}

\par

Klaus hatte sich unterdessen zu Hans Computer begeben und besah sich die markierten Punkte. \WR{Du könntest recht haben}, sagte er. \WR{Laut den Daten ist es schon eine Weile her, seitdem er sich von diesem Orten aus eingeschlichen hat. Wahrscheinlich war das vor seinem aktuellen Auftrag, für den die höhere Geschwindigkeit der Verteilerknoten nötig war.}

\par

Laura nickte. \WR{Wir konzentrieren uns mit der Suche am besten auf Sucre und Umgebung. Lassen dort das Phantombild so oft zeigen wie möglich. Vielleicht hat ihn irgendwer gesehen und kann uns weitere Hinweise geben.}

\par

\WR{Sonst tappen wir leider ziemlich im Dunkeln}, warf Klaus Rensing ein und nahm sich einen Stuhl. Han Liao warf ihm einen schwer zu deutenden Blick zu. \WR{Ich hab das Phantombild ins Gesichtserkennungsprogramm der öffentlichen Überwachungssysteme eingegeben. Bisher hat es bloß Nieten ausgespuckt. Es ist fast so, als hätte keine Kamera der ganzen Union diesem Mann jemals gefilmt.}

\par

\WR{Glauben Sie, der Hacker hat es irgendwie geschafft, alle Aufnahmen von ihm aus der Vergleichsdatenbank zu löschen?}, fragte Laura Gethas an Han Liao gewandt.

\par

Dieser winkte sofort ab. \WR{Das ist Unsinn. Allein die schiere Anzahl der ganzen Bilder, die von jedem von uns im Lauf unseres Lebens schon gemacht wurden, würde es unmöglich machen, sie alle zu löschen. Von den Sicherungen ganz zu schweigen.}

\par

\WR{Aber einen Grund muss es geben}, entgegnete Klaus Rensing, \WR{dafür, dass die Vergleichsroutine keine Übereinstimmung findet.}

\par

Hierzu war selbst Liao überfragt. \WR{Hm}, begann er. \WR{Die einzige Möglichkeit, die mir einfällt, nicht in den Datenbanken der öffentlichen Überwachung aufzutauchen wäre, von vorne herein, nicht darin abgespeichert zu werden.}

\par

Klaus Rensing gähnte. Er erhob sich und schwang sich in seinen Mantel. \WR{Es war ein langer Tag}, sagte er müde. \WR{Laura, ich geh nach Hause. Wir sehen uns morgen im Büro.}

\par

Die Angesprochene nickte nur abwesend. Ihr schwante bereits ein Verdacht, der ihr nicht gefiel.