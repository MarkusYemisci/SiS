Morten saß einfach da. An einer Stelle des Korridors, an der selten jemand vorbei kam. Er lauschte dem Ächzen des Schiffes und versuchte, die Situation endlich real werden zu lassen. Seit die \EN{Regenvogel} nach Kreuzpunkt gesprungen war, hatte er Angst um sein Leben und das der Menschen um ihn herum. Sogar Kevins.

\par

Er funktionierte. So viel stand fest. Auch für seinen nächsten Einsatz war er sich seiner Fähigkeiten sicher. Aber das war es auch. Darüber hinaus fühlte er sich leer, so als gäbe es ihn gar nicht wirklich.

\par

Und fast hätte seine Existenz auch ein jähes Ende genommen. Immer wieder sah er die beiden Shutek ihm nachjagen. Wäre Curiosa nicht gewesen, dann hätten sie ihn erwischt, dessen war er sich sicher.

\par

Erst, als Lieutenant Wallander an ihm vorbei kam, schreckte er aus seinen Gedanken auf. Der Kommunikationsoffizier sah ihn misstrauisch an. \WR{Alles in Ordnung, Witwer?}

\par

Morten reagierte zunächst nicht. Dann ließ er sich aufhelfen. \WR{Ja, danke Ihnen, Lieutenant. Was machen Sie hier? Sollten Sie nicht auf der Brücke sein?}

\par

Der Kommunikationschef schluckte. \WR{Das sollte ich, ja. Captain Fiscale hat mich von Dienst freigestellt. Ich habe einen Befehl missachtet.}

\par

Morten lächelte bitter. \WR{Die Arrestzelle wird heute wirklich voll werden, wenn das so weiter geht.}

\par

\WR{Ich muss nicht in die Brig}, antwortete Wallander, klang dabei aber keineswegs erleichtert. \WR{Captain Fiscale hat wohl verstanden, warum ich ihr nicht gehorchen konnte. Was sie vorhat…}

\par

Nun läuteten Mortens Alarmsirenen. Was konnte die Kommandantin vorhaben, was ihr Insubordination eines langjährigen Untergebenen einbrachte. \WR{Das wäre?}, fragte er, unsicher die Antwort hören zu wollen.

\par

Wallander schluckte. \WR{Bomben auf Yêxīn. Die ganze Südfront. Sie will die Shutek aus der Umlaufbahn aus mit Nullzonenköpfen eliminieren.}

\par

\WR{Oh mein Gott}, hauchte Morten und ließ sich wieder gegen die Wand sinken. Er hatte Kreuzpunkt Primus stets gehasst. Es steckte zu viel Corna in ihm, als dass er diesem Planeten, dem ehemaligen Herz der Unterdrückung, etwas gutes hätte abgewinnen können. Aber die Menschen, die dort unten lebten, hatten diesen Untergang nicht verdient. Genauso wenig wie Nico Curiosa.

\par

\WR{Vielleicht ist es die einzige Möglichkeit}, gab Wallander zu bedenken und verschwand dann schnellen Schrittes. Erst jetzt bemerkte Morten Anna Farley etwas weiter hinten im Korridor warten und richtete sich erneut auf. Doch die Art, wie sie ging~-- sie schlenderte fast~-- und das knappe Lächeln, dass sie ihm zuwarf, ließen ihn zögern, die nun vorgeschriebene Habachtstellung einzunehmen.

\par

\WR{Morten, ich mochte dich von Anfang an}, sagte sie. \WR{Du bist ein guter Pilot und ich weiß, in dir steckt auch ein guter Freund. Für Kevin und vielleicht auch für Nico.} Der Angesprochene sah sie bloß fragend an, nicht wissend, wie er sich verhalten sollte. \WR{Regeln bedeuten dir alles. Keine Ahnung warum}, gab sie zu. \WR{Ich könnte mich jetzt als Küchenpsychologin versuchen und sagen, es ist wegen deiner Prägung. Waisenhaus. Zieheltern. Schwierigkeiten bei der Berufsfindung. Der Wunsch nach einem geregelten Leben halt. Aber im Endeffekt spielt es keine Rolle. Du musst die Regeln befolgen. Die geschriebenen und die ungeschriebenen.}

\par

Dann klopfte sie ihm freundschaftlich gegen den Oberarm und verschwand, noch mit diesen Worten auf den Lippen. Morten sah ihr verdutzt nach, doch seine Verwunderung ließ schnell nach, als ihm klar wurde, was er zu tun hatte.
\ortswechsel
Zu Mortens milder Überraschung war der Vorraum der Brig nicht leer. Kevin Wilson hatte Besuch. Neben dem Polizisten, der sichtlich unter Stress stand, stand auch Major Dexter Hennington, nach wie vor in volle Fliegermontur gekleidet, vor der Zelle des jungen Mannes und starrte ihn an, mit den Armen vor der Brust verschränkt.

\par

\WR{So sehen also Helden aus}, spottete der Major.
\WR{Du konntest dein Mädchen wohl nicht retten, was? Ich hab's gleich gewusst.
Seit dem ersten Moment, seit dem ich dich gesehen hab.
Du bist zu nichts zu gebrauchen.
Ein Volltrottel, der ins Cockpit gestolpert ist.
Ihr verdammten Jungspunde widert mich an.
Wisst einfach nicht, wo euer Platz ist, hä.}

\par

Der Polizist erhob beide Augenbrauen. Er unterstand zwar der regulären Befehlskette der \EN{Regenvogel}, war aber auch seinen eigenen Vorgesetzten einen Bericht schuldig, in welchem Hennington nicht gut wegkommen würde.

\par

Kevin hingegen schien seinen Gast gar nicht wahrzunehmen. Er saß einfach nur da und wippte auf der Pritsche auf und ab. Das Gesicht auf die gefalteten Hände gestützt. Er weinte nicht. Aber man musste ihn nicht so gut kennen gelernt haben, wie Morten, um zu sehen, was in ihm vorging.

\par

\WR{He, ich rede mit dir!}, donnerte Dexter Hennington.

\par

\WR{Aber er nicht mit dir}, war Mortens rasche Antwort, die sofort die Köpfer aller Anwesenden in seine Richtung schnellen ließ. \WR{Ich glaube, er braucht jetzt etwas Ruhe}, gab Morten weiter zu bedenken. \WR{Seine Freundin ist gestorben. Ich denke, da hat er sich etwas Abstand verdient.}

\par

Hennington schien sich dazu entschlossen zu haben, Morten zu ignorieren.
Er drehte sich wieder der Plexiglasscheibe zu und sagte: \WR{Ich hab's gesehen.
Edhor Peak.
Brennend und verloren.
Kann mir eine bessere Art zu sterben vorstellen.}

\par

Morten verzog angeekelt das Gesicht. \WR{Warum tun Sie das, Major?}, fragte er völlig ernst. \WR{Das sind Menschen, die Sie hätten beschützen sollen. Und jetzt spotten sie über deren Schicksal?} Auch der Wachtmeister trat nun unauffällig auf den stellvertretenden Flügelkommandanten zu.

\par

Widerwillig wie schnaubend drehte sich Hennington zu Morten um und durchbohrte ihn mit seinen Blicken. \WR{Also, zwei Möglichkeiten, Jungchen. Entweder du verpisst dich jetzt und ich vergesse, was du gesagt hast. Oder du bleibst und ich polier dir die Fresse. Und danach gehe ich zum Captain und sage ihr, du hättest versucht, den Gefangenen zu befreien. Ja, du hast richtig gehört. Ich lüge ihr mitten ins Gesicht. Und sie wird es mir glauben, ganz egal, was du oder diese beiden Vögel hier behaupten werdet. Selbst wenn ich Ärger bekomme, dann haut mich meine Anna wieder raus, so wie immer. Und ich werde es ihr dann gebührend zurückzahlen.}

\par

Nun lächelte Morten ganz unwillkürlich. Er zuckte mit den Schultern und sagte: \WR{Das ist schon möglich. Ich schätze, dann muss ich dir jetzt eben eine reinhauen.}

\par

Wenig überraschenderweise war die erste Reaktion des Wachmannes, den losstürmenden Dexter Hennington zu ergreifen. Doch dieser war bei weitem zu wütend und zu schnell, als das dies funktioniert hätte. Sein erster Fausthieb hätte Morten beinahe erwischt, doch er schaffte es, sich unter dem weiten Schwinger wegzuducken. Der Gegenangriff traf Hennington mitten ins Gesicht und er taumelte gegen den Schreibtisch des Wachhabenden. Ein sofort darauf folgender Schlag brachte dem Major die Ohnmacht.

\par

Erst, als der Wachtmeister seine Pistole auf Mortens Torso richtete, und ihn genauso entgeistert ansah, wie Kevin Wilson, realisiert dieser, was er getan hatte. Sein Mund stand offen und nur langsam spürte er den unglaublichen Schmerz in seiner rechten Hand.

\par

\WR{Lieutenant Witwer, ich verhafte Sie hiermit wegen Angriffs auf ein Konglomeratsmitglied während einer Gefechtssituation.} Als würde das Schiff die Dramatik der Situation unterstreichen wollen, ächzte es laut unter einem neuerlichen Treffer. \WR{Treten Sie bitte in die Zelle. Ich informiere den Captain.}

\par

Morten wurde schlecht. Er konnte sich kaum bewegen, als der Wachtmeister das Plexiglas der Brig öffnete und ihn mit  seiner Pistole hinein wies. Beinahe hätte er sich übergeben, als ihm wirklich klar wurde, dass er vermutlich gerade seine Karriere weggeworfen hatte.

\par

\WR{Du kannst echt hart zuschlagen}, komplimentierte ihn Kevin. \WR{Aber du weißt schon, dass das wirklich saudumm war?}

\par

Morten sagte nichts. Nach wie vor mit weit aufgerissenen Augen sah er auf den Boden vor sich und konzentrierte sich darauf, kontrolliert ein- und wieder auszuatmen, um sich nicht sofort zu übergeben.

\par

Kevin, der dankbar war für alles, was ihn von den Gedanken an seine Freundin ablenkte, legte Morten seinen Arm auf die Schulter. \WR{Wieso um alles in der Welt hast du das getan?}

\par

Der Gefragte sah auf und seine Stirn legte sich in Falten. \WR{Tja, ich denke, ich wollte einfach dein Freund sein.}