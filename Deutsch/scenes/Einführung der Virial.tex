Angst und Verwirrung.
Mehr war Joe Tenner zuerst nicht in der Lage zu empfinden.
Sein Geist versuchte vergeblich zu verarbeiten, was geschah, aussichtslos bemüht, eine Logik hinter dem Chaos zu erkennen.
Es war offensichtlich, dass er keinem seiner fünf Sinne mehr trauen konnte.
Bilder erschienen vor Joes geistigem Auge. Doch er wusste nicht, ob sie echt waren und ihn durch seinen Seenerv erreichten oder einfach nur seiner Phantasie entsprangen.
Alles erschien ihm wie ein böser Traum, der sich schon seit einer Ewigkeit hinzuziehen schien.
Ständig wechselte die Szenerie.
Pausenlos schossen neue, unfassbare Gedanken durch Joes Bewusstsein, das krampfhaft versuchte, den heranstürmenden Eindrücken irgendwie Ordnung zu verleihen.
Dann war es vorbei.
So schnell und plötzlich, dass Joe erschrak und seinen Rücken gegen die Lehne des Stuhls presste, auf dem er saß.
Zuerst rang er nach Luft, bis er schließlich bemerkte, dass er keine Schwierigkeiten hatte, zu atmen und sich allmählich beruhigte.
\par
Das erste Bild, dass sich vor seinen Augen auftat, war die unendliche Dunkelheit, durchsäht mit kleinen Punkten.
Unzählige, hell leuchtende Sterne schienen von einem Moment auf den nächsten aufzutauchen.
Und jedes Mal, wenn Joe glaubte alle Sterne sehen zu können, entdeckte er wieder einen neuen Schwung, noch schwächer leuchtender Punkte. Einige waren wahrscheinlich tausende von Lichtjahren entfernt.
Andere dieser Punkte stellten in Wirklichkeit gar keine Sterne sondern ganze Galaxien dar.
Joe Tenner hatte keine Ahnung gehabt, wie unendlich und allumfassend das Weltall wirken konnte, bevor er zum aller ersten Mal aus dem Fenster eines Raumschiffs gesehen hatte.
Und obwohl er diesen Anblick nun schon etliche Male bewundert hatte, kam es ihm auch dieses mal so vor als würde er frei zwischen den Gestirnen schweben, anstatt im Kommandozentrum der \Eigenname{Virial} zu sitzen.
Langsam kam er, wie auch die meisten anderen Menschen in seiner Nähe, wieder zu einander.
Er ballte seine Hände kräftig zu Fäusten und erfreute sich über das bekannte Gefühl der Anspannung, als seine Knöchel weiß wurden.
Auch seine restlichen Sinne wurden langsam aber sicher wieder aktiv.
\par
Bewusst atmend unterdrückte er die Rebellion seines Magens, der unbedingt seinen gesamten Inhalt auswerfen wollte.
Nach seinen ersten Sprüngen hatte Joe Tenner um einiges mehr Schwierigkeiten damit gehabt.
Mittlerweile gehörte das flaue Gefühl in seiner Brust schon fast zur Routine.
Ein Ärgernis.
Aber kein unerwartetes.
\par
\WoertlicheRede{Warum tue ich mir das eigentlich immer wieder an?}, murmelte er leise vor sich hin.
\par
Sein Nebensitzer, Eddie Carlyle, begann sofort zu grinsen, als er Joes Selbstgespräch mitbekommen hatte. \WoertlicheRede{Weil du die Knete brauchst, natürlich}, begann er zu sticheln.
\par
Joe grunzte und entgegnete sofort: \WoertlicheRede{Ach halt's Maul. Du stehst auch nicht besser da. Besonders nicht nach deinen ganzen Sportwetten.}
\par
Eddie Carlyle antwortete irgendetwas, fand aber kein Gehört.
Joe sah aus dem fünf mal zehn Meter großen Frontfenster der Brücke.
Der riesige, orangeroten Stern des Arktur-Systems bot einen atemberaubenden Anblick.
Das Gestirn war gut fünfundzwanzig mal so groß wie Sol und sein Licht schien so intensiv, dass die Konturen des Kommandoraums hellgelb schimmerten.
Hätten die Fenster der Brücke keine abschirmende Wirkung gehabt, wären die Menschen auf ihr längst rot wie Hummer.
Doch der Stern war weit mehr als eine einfache leuchtende Kugel.
Durch die filternden Scheiben war ein direkter Blick auf die Korona möglich.
Selbst aus der riesigen Entfernung, welche die \Eigenname{Virial} von Arktur trennte, waren die unzähligen Sonnenflecken, Konturen und Eruptionen zu sehen, die teilweise größer waren als zehn Erden zusammen.
\par
Joe Tenner erstarrte kurz bei diesem Anblick.
Regungslos saß er da und blickte den orangeroten Riesen an.
Er hätte sich wahrscheinlich noch eine ganze Weile nicht aus dem Bann des Sterns befreien könne, wenn ihn nicht ein anderes Besatzungsmitglied unsanft angerempelt hatte.
Als Joe dem Übeltäter einen bösen Blick zuwerfen wollte, war dieser schon weiter gerannt.
Die Hand fest vor den Mund gepresst, lief der Mann in Richtung Aufzug.
Joe Tenner konnte sich ein kurzes Grinsen nicht verkneifen.
Unbewusst griff er sich in die Tasche seines Gehrocks und ertastete die kleine Papiertüte, die er immer bei sich trug.
Für den Fall, dass er seinen Magen einmal nicht rechtzeitig beruhigen können würde.
Er verfuhr schon seit Jahren so. Schon seit seinem ersten Sprung.
Damals hatte er eben keine Tüte dabei gehabt und seinen Mageninhalt über die Konsolen seiner Sitznachbarn verteilt.
Die Beziehung zu seinen damaligen Kollegen hatte sich durch diesen Vorfall nicht gerade gebessert.
\par
\WoertlicheRede{Verdammte Anfänger}, fing Eddie Carlyle an zu schimpfen. \WoertlicheRede{Was essen diese Jungspunde bloß, dass ihnen derart schnell die Galle hochkommt?}
Joe belächelte die aufbrausende Art seines Sitznachbarn.
Wenn irgendwo irgendetwas nicht richtig lief, dann konnte man sich darauf verlassen, dass Eddie sich als erster darüber beschwerte. Dennoch hatte er ihn gerne als Kollegen. Seine Sprüche und Meckereien brachten etwas Abwechslung in das sonst recht eintönige Leben an Bord der \Eigenname{Virial}. 
Kurze Zeit später hörte Joe den Mann am Steuer melden: »Hyperraumssprung abgeschlossen, mein Herr. Wir haben das Arktur System erreicht.«
Johann Paulsen, ein hagerer Mann in seinen Fünfzigern, löste die Sicherheitsgurte seines Sessels und erhob sich. Wie die meisten höheren Firmenangestellten trug auch er einen tadellos geschnittenen, dunklen Gehrock aus kostspieligem Stoff. Seinen Kragen hatte er hochgestellt und das Stoffband, dass an eine Krawatte aus dem zwanzigsten Jahrhundert erinnerte, einwandfrei zurechtgerückt.
Joe sah an sich selbst hinab. Seine Kleidung wirkte nicht annähernd so formell und korrekt wie die des Kommandanten. Er selbst hatte einen Pullover einem Hemd vorgezogen und auf eine Krawattenadaption gänzlich verzichtet. Von dem hohen Qualitätsunterschied zwischen den Stoffen der Gehröcke ganz zu schweigen.
Einen noch krasseren Gegensatz bildete jedoch sein Kollege Eddie Carlyle. Dieser trug nicht einmal einen abgewetzten Anzug sondern lediglich ein Flanellhemd und eine durchgesessene Cordhose. Und obwohl die Gleichstellungsnormative Benachteiligung aufgrund von Kleidung verbot, wunderte es Joe dennoch, wie Eddies Bewerbung bei einem renommierten Unternehmen wie der Pinnacle Science Group durchgekommen sein konnte.
Johann Paulsen zückte einen kleinen Handcomputer, der flacher war als ein kleiner Finger und machte einige Notizen. Das Gerät war etwas größer und breiter als die üblichen Versionen, damit man problemlos im Stehen mit einem Druckstift darauf schreiben konnte anstatt sich für jede Notiz eine holographische Tastatur anzeigen lassen zu müssen.
Joe vermutete, dass es sich bei dem Aufschrieb um Belanglosigkeiten handeln würde, die später in irgendeinem Firmenprotokoll wieder auftauchen würden.
Eddie starrte Gedankenverloren durch die Fenster hinaus ins Weltall. Leise murmelte er: »Ich glaube, so weit draußen im Nichts war ich noch nie. Wieso können wir nicht eigentlich mal ein paar interessante Kolonien besuchen?«
Sein Nebensitzer Joe lies seinen Blick ebenfalls durch die Unendlichkeit des Raums schweifen. Da das gesamte Kommandozentrum, bis auf die rückwärtige Wand, von Fenstern begrenzt wurde, war es nicht schwierig sich in den Bann des Alls ziehen zu lassen. Als Joe seinen Kopf in den Nacken legte und direkt nach oben sah, gab es nichts mehr, dass seinen Blick behinderte und er fühlte sich, als würde er frei zwischen den Sternen schweben. Nur gelegentliche Lichtreflexionen an den Scheiben erinnerten ihn daran, dass der Kommandoraum durch Spezialglas von einem halben Meter Dicke vom Rest des Kosmos getrennt war. Den eigentlichen Schutz boten jedoch Schildpanzer. Die \Eigenname{Virial} besaß, wie die meisten Schiffe, zwei Generatoren dafür. Einer, der den ganzen Rumpf in ein Feld einhüllte und ein weiterer, der dies nur mit der Brücke tat.
Eddie hatte sich bereits wieder seiner Arbeitsstation zugewandt, als Joe ihm antwortete: »Sieh‘s positiv. Während des Sprungs haben wir uns innerhalb von wenigen Sekunden weiter fortbewegt, als viele Menschen in ihrem ganzen Leben.«
Der Angesprochene sah nur gedankenverloren auf seine Bildschirme. Joe entschied, es ihm gleich zu tun. Herr Paulsen würde sowieso schon bald einen Lagebericht haben wollen und dann müsste er ihm Rede und Antwort stehen können. Das war immerhin seine Aufgabe im Kommandoraum.
Er überwachte die passiven Suchsysteme der \Eigenname{Virial}. Das bedeutete einerseits das Echtzeitradar und andererseits die Emissionssensoren, die überall auf der Hülle des Forschungsschiffes verteilt waren. Die Technik des Radars beeindruckte Joe immer wieder. Es war in der Lage, eine etwa zweieinhalb astronomische Einheiten große Sphäre beinahe augenblicklich abzusuchen. Die dem Radar eigene Abstrahlung war dabei von anderen kaum noch zu orten, was allerdings ein taktisches Detail war, das seit dem Routenkrieg keine Bedeutung mehr besaß.
Um einiges empfindlicher noch als das Radar war jedoch das Netz der Emissionssensoren. Diese konnten selbst geringste Abweichungen zur normalen Hintergrundstrahlung aufspüren. Da sich Strahlung aber nur höchstenfalls mit Lichtgeschwindigkeit fortbewegte, arbeiteten die Sensoren mit einer Verzögerung. Diese konnte länger oder kürzer sein, je nachdem wie weit die Strahlungsquelle entfernt lag.
Als Joe damals die Stellenanzeige für den Radarposten auf der \Eigenname{Virial} in die Hände gefallen war, hatten ihn gerade diese technischen Raffinessen gereizt. Er hatte sich häufig gefragt, ob er die Stelle auch angenommen hätte, wenn ihm schon damals klar gewesen wäre, dass seine Arbeit hauptsächlich aus stumpfsinnigem Starren auf einen Bildschirm bestehen würde. Aber jedes Mal kam ihm in den Sinn, was er alles verpasst hätte, wenn er sich nicht beworben hätte.
Er hatte Teile des Universums gesehen, die anderen Menschen niemals zu Gesicht bekommen würden. Fremde Planeten, die unglaublich bizarr aber gleichzeitig auch wunderschön waren. Eine Nebelbank die sich über Dunen von astronomischen Einheiten erstreckte. Und nicht zuletzt die Sonne. Was von der Erde aus nur eine leuchtende Scheibe war, wirkte aus der Nähe wie ein niemals enden wollendes Feld aus weißem Feuer. Joe erinnerte sich daran, wie er durch eine Absorberscheibe auf den Stern gesehen hatte und sich vorgekommen war, als würde er in reines, lebendiges Licht sehen.
Solche Erlebnisse waren wahrscheinlich der Hauptgrund dafür, warum er sich nicht mittlerweile einen anderen Job gesucht hatte. Und er vermutete, dass es sich bei den meisten der Anderen auf der \Eigenname{Virial} genauso verhielt. Immer wieder wurde die Bezahlung vorgehalten aber in Wirklichkeit waren sie alle nur neugierige Romantiker, der sich die Welt ansehen wollten.
Was Paulsen betraf, war sich Joe allerdings nicht so sicher. Wie alle leitenden Angestellten der Pinnacle Science Group hatte auch er wenig mit einem Naturwissenschaftler gemein. Die PSG betraute meistens Firmenfunktionäre mit dem Kommando über ein Forschungsschiff. Wahrscheinlich weil man in der Führungsriege nicht annahm, dass ein experimentierfreudiger Forscher, der bei jeder Anomalie anhielt und nachsah, ein Raumschiff genauso ökonomisch befehligen konnte, wie ein Betriebswirt. Obwohl die PSG an sich ein sehr magnates Unternehmen war, kannten ihre Geldmittel dennoch Grenzen. Und selbst im fünfundzwanzigsten Jahrhundert war fast nichts teuerer als Raumfahrt.
Joe dachte dabei an sein letztes Gehalt. Neuntrin Naira Quartalsverdienst waren ein Wort. Die Bezahlung war sicherlich das Letzte, worüber er sich ärgerte. Eher über das, was er für sein Geld tat. Er hatte acht Jahre lang studiert und besaß ein Klasse-A-Diplom in Physik und zwei C-Diploma in Mathematik und Ingenieurswissenschaft. Zu der Zeit, als dieser Titel nicht nur für Mediziner verwendet worden war, hätte er sich als Doktor Tenner bezeichnen können.
Natürlich brauchte Joe sein Fachwissen gelegentlich um die Funktionsweise des Radars nachvollziehen zu können aber die meiste Zeit saß er nur vor einem Monitor und sah einer Linie beim rotieren zu.
Herr Paulsen ging im Kommandoraum auf und ab. Hin und wieder sah er einigen der Forscher über die Schulter, kaum verstehend, was auf den Bildschirmen vor sich ging. Wie Joe und Eddie beobachtet hatten, schien die Teleskopüberwachung die Lieblingsstation des Kommandanten zu sein. Entweder weil die Bilder der Fernrohre oft imposant aussahen und auch für einen Laien leicht zu verstehen waren, oder aufgrund der jungen Spezialistin für Optik.
Schließlich forderte Herr Paulsen einen Statusbericht ein und notierte die, immer gleich lautende, Antwort gewissenhaft auf seinem Handcomputer. Als es an Joe Tenner war, seine Meldung zu machen, sagte auch er: »Alle Systeme arbeiten normal.«
»Wie sieht es mit den aktiven Suchsystemen aus?«, fuhr Paulsen fort und richtete seinen Blick auf Eddie.
Dieser entgegnete leichthin: »Alles Paletti«, ohne sich umzudrehen.
»Wie bitte?«, fragte Herr Paulsen unmittelbar. Seine Stimme klang zwar freundlich aber alle im Kommandoraum spürten deutlich, dass dem Kapitän die Antwort missfiel.
Eddie seufzte über die Vehemenz, mit der Paulsen an seinem Schema F festhielt. Nachdem alle Blicke erwartungsvoll auf ihn gerichtet waren, antwortete der Sensorspezialist erneut: »Alle Systeme arbeiten funktionsgemäß.«
Herr Paulsen nickte nur kurz und fuhr dann mit seiner Befragung fort. Eddie Carlyle lehnte sich zu seinem Nachbarn und flüsterte leise aber hörbar verärgert: »Oh Mann. Ich glaub, er schreibt sogar ein Protokoll, wenn er auf die Toilette geht.«
»Denk dir nichts«, antwortete Joe sofort. »Es ist normal, dass ein Konzern ganz genau wissen will, was los ist. Das Problem ist, dass die Jungs und Mädels von den oberen Rängen keine Ahnung haben, was wirklich wichtig ist und was nicht. Und so wird eben jeder Mist aufgeschrieben und dokumentiert.«
Eddie rollte mit den Augen. »Stimmt. Trotzdem ist es nicht zu viel verlangt ›Alles Paletti‹ interpretieren zu können, oder?«
»Paulsen ist eben so«, entgegnete ihm Joe sofort. »Ich glaube er will sich einfach absichern. Wenn er aus deinem Mund hört, dass alle Systeme funktionierten ist er aus dem Schneider. Aber wenn du sagst ›Alles Paletti‹, dann kann das viel heißen. Das es deiner Seele gut geht, zum Beispiel. Oder das dir die Entwicklung deiner Geldanlagen gefällt. Verstehst du, was ich meine?«
Joes Nebensitzer wandte seinen Blick ab und grinste verbittert. »Ich denke es wäre besser, wenn einer von uns Wissenschaftlern das Kommando hätte. Nicht so ein Paragraphenreiter, der zwar alle Regeln kennt, dafür aber keine Ahnung von Forschung hat.«
Joe brummte leise. Eigentlich konnte er seinem Sitznachbarn nur zustimmen. Aber irgendwie hatte er sich schon damit abgefunden, wie die Hierarchie in der PSG funktionierte.
»Weißt du«, begann er mit einer Antwort, »ich glaube, wir verschwenden nur unsere Energie, wenn wir uns jetzt über alles aufregen. Im Prinzip haben wir doch was wir wollen. Wir reisen durchs Universum! Wir können uns Sachen anschauen, den sich der Rest der Menschheit nicht einmal vorstellen kann.«
Eddie sah auf sein Computerterminal. Auf dem Monitor, der nur aus einer dünnen Glasschicht und einem metallenen Rahmen bestand, war ein Halbkreis zu sehen, der sich immer wieder um eine Achse in der Mitte des Bildschirms drehte.
»Du denkst also, der Rest der Menschheit kann sich keine einfallslose drei-D-Animation vorstellen?«
Kurz darauf war Herr Paulsen mit der Einholung der Statusberichte fertig. Er atmete angestrengt aus und begann dann, laut genug um von jedem gehört zu werden, zu reden: »Meine Damen und Herren. Wir haben Arktur erreicht und werden jetzt mit unserem Auftrag beginnen. Bitte behalten Sie im Hinterkopf, dass die PSG von jedem erwartet, dass er seine Aufgabe gewissenhaft und pflichtbewusst durchführt.« So manch einer musste sich einen mürrischen Seufzer verkneifen. »Wir werden nun als aller erstes mit einer groben Erfassung sämtlicher Himmelskörper in diesem System beginnen. Wie wir die folgende Vermessung vornehmen besprechen wir heute Abend. Wir haben eine Menge Arbeit vor uns, also lassen Sie uns anfangen.«
Eddie grummelte frustriert. Joe warf ihm einen beschwichtigenden Blick zu. Im Gegensatz zu seinem Kollegen hatte er sich schon seit geraumer Zeit mit Paulsens Art arrangiert. Er hatte sich vorgenommen, nicht den Kapitän für die Dinge verantwortlich zu machen, die ihn an seiner Arbeit störten. Denn, selbst wenn Herr Paulsen ein absoluter Rebell und Weltveränderer anstatt eines regelversessenen Arbeitstiers gewesen wäre, hätte sich an der Art, mit der die PSG ihre Angestellten führte, nicht das Geringste geändert.
So machte sich Joe Tenner einfach an die Arbeit. Er würde die nächsten Stunden genießen, denn während dieser Zeit hing vieles an ihm. Danach, wenn alle Körper des Systems zumindest oberflächlich erfasst wären, würde er nur noch starr auf seinen Monitor sehen und hoffen, dass sie ihm doch noch irgend etwas neues zeigen würden. Ein Wunsch, der bisher niemals in Erfüllung gegangen war.
