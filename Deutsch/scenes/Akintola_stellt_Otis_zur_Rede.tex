Präsident Otis saß hinter seinem Schreibtisch, umhüllt von Bergen und noch mehr Bergen an Papier. Er wälzte bereits die ganze Nacht lang Bücher und suchte verwundbare Stellen in der Legitimierung der Phalanx und der Starforce. Dies tat er, obwohl ein Bataillon an Beratern in genau diesem Moment derselben Aufgabe nachgingen.

\par

Nur seine Schreibtischbeleuchtung brannte und sein Büro, dass die Ausmaße eines geräumigen Wohnzimmers hatte, wurde in ein seichtes Dämmerlicht getaucht. Ausschließlich die vier leeren Kaffeetassen, die neben dem Präsidenten auf dem Tisch standen hielten Henry Otis noch wach.

\par

Am gestrigen Nachmittag hatte es Stunden gedauert, bis sich die Lage im zentralen Senat wieder beruhigt hatte. Einige, sehr hartnäckige Abgeordnete waren noch lange Zeit diskutierend sitzen geblieben, als das Treffen bereits offiziell beendet gewesen war. Mit einem neuen Feldzug seitens des Präsidenten, gegen die Armee hatte wirklich niemand gerechnet. Jede Nachrichtenagentur, die etwas auf sich hielt, hatte Otis eine Nachricht geschrieben, in der sie um ein Interview bat. Bis jetzt hatte er noch keinen Brief beantwortet aber er würde auf die Angebote zurückkommen. Medienpräsenz war nun wichtig. Er musste die Massen auf seine Seite ziehen. Im Reden war er Grandadmiral Burns überlegen. Hier konnte er seine Karten ausspielen.

\par

Henry Otis war gerade dabei wieder einzudösen, als ihn ein Piepen darauf aufmerksam machte, dass jemand an seiner Tür stand. Der Computer identifizierte die Person als Vizepräsidentin Akintola. Normalerweise hätte sie sich bei der Sekretärin angemeldet aber um diese Urzeit war der Vorraum zum präsidialen Büro nicht besetzt. Sofort drückte Otis auf eine Taste an der Unterseite seiner Tischplatte um die Tür zu öffnen.

\par

Lertha Akintola betrat den Raum. Ihr Gesicht war wie immer ausdruckslos und gelassen aber der Präsident kannte sie nun schon lange genug um zu wissen, dass sie ziemlich ungehalten war. Er konnte sich auch schon vorstellen weswegen. Nachdem er die Senatoren in wilde Diskussionen gestürzt hatte, war es die Aufgabe der Vizepräsidentin gewesen, den Tumult wieder aufzulösen. Und bei dieser Gelegenheit war es für sie wirklich eine Sisyphusarbeit gewesen.

\par

Aber nicht nur das dürfte Lertha Akintola verärgert haben. Die politische Unruhe, die Otis mit seinem Antrag die Armee abzuschaffen, würde ihr noch eine Menge Wirbel machen. Obwohl sie selbst die Ansichten des Präsidenten auf keinen Fall teilte, hatte sie bisher immer ein hervorragendes Arbeitsverhältnis zu ihm gehabt. Und das würde nun leiden, denn jetzt würden Streitgespräche die Diskussionen zwischen dem Präsident und seiner Vize dominieren.

\par

\WR{Ich hätte nicht geglaubt, dass Sie so weit gehen würden}, leitete Lertha Akintola die erste verbale Auseinandersetzung ein.

\par

Präsident Otis lehnte sich in seinen Stuhl zurück und atmete tief durch. Dann entgegnete er: \WR{Gegen die Armee anzugehen war schon immer eines meiner Ziele. Irgendwann musste ich es doch einmal in Angriff nehmen, oder?}

\par

\WR{Sie hätte sich wenigstens vorher mit mir absprechen können}, fuhr die Vizepräsidentin fort. Ihr Stimme blieb dabei ungewöhnlich ruhig. \WR{Ihnen muss doch bewusst gewesen sein, dass die nächsten Wochen im Senat nur noch aus endlosen Grundsatzdiskussionen und Selbstdarstellungen bestehen wird. Kein einziger Abgeordneter wird jetzt noch alltägliche Themen besprechen wollen. Jetzt wittert jeder seine Chance, eine Position in diesem Grabenkampf zu beziehen. Jeder politische Grundsatz der Union lehnt das Parteienwesen ab. Aber durch ihre Tat von gestern Mittag werden Sie den Forum in zwei Lager zerschlagen. Wenn nicht sogar mehr.}

\par

Henry Otis musste sich bemühen, nicht zu schlucken. Er hätte nicht gedacht, dass Akintola sofort in die Vollen gehen würde. Beschwichtigend begann er: \WR{Bitte, Lertha. Ich hatte niemals die Absicht Schaden anzurichten. Ich weiß, dass mein neustes Anliegen sehr plötzlich kam und einige Unannehmlichkeiten bereitet hat. Dafür will ich mich entschuldigen.}

\par

Akintola, stemmte die Hände in die Hüften und gab gestikulierend zurück: \WR{Versuchen Sie nicht, mich milde zu stimmen. Sie brauchen sich auch nicht zu entschuldigen. Sie wussten, was im Senat los sein würde und es tut Ihnen auch nicht leid.} Die Vizepräsidentin machte eine kurze Pause. Es geschah selten, dass sie nach Worten suchen musste und jedes Mal wenn es dazu kam, wurde es ernst. \WR{Mein Herr, erlauben Sie mir zu sagen: Sie lassen sich durch Ihre persönliche Abneigung der Armee gegenüber leiten. Ihnen geht es nicht um moralische Probleme, Sie sind sauer, dass Sie vor Jahren im Senat eine Niederlage im Kampf gegen das Konglomerat erlitten haben. Und jetzt wollen Sie es den Lamettaträgern heimzahlen. Und ihr Neffe…}

\par

Henry Otis schoss aus seinem Sessel herauf. Seine Augen blitzten. \WR{Hören Sie mal}, begann er erregt. \WR{Ich denke, Sie fallen aus der Rolle. Wie wollen Sie wissen, worum es mir bei dieser Diskussion geht? Ich glaube nicht, dass wir eine Armee brauchen. Und ich glaube, die Phalanx und die Starforce sind eine Schande für unsere Union, genauso wie für jede andere Staatsform, die Gewalt verabscheut. Und das mein Neffe sich der Armee angeschlossen hat, hat nicht das geringste damit zu tun.}

\par

Lertha Akintola war einen Schritt zurück getreten, als Präsident Otis aufgesprungen war, und sich mit geballten Fäusten auf die Tischplatte gestützt hatte. Doch dann verfinsterte sich ihr Gesicht. \WR{Ich glaube, Sie sind es, der aus der Rolle fällt}, brachte sie zwischen zusammengekniffene Zähnen hervor. \WR{Unsere Armee ist keine Schande. Und sie steht auch nicht mit unseren Grundfesten im Konflikt. Wir haben nicht einmal eine Wehrpflicht. Keiner muss zur Starforce oder den Truppen. Und keiner von denen, die gehen ist ein hirnloser Zinnsoldat, wie Sie es meistens darstellen. Das sind Menschen, wie Sie und ich, die nur das aus ihrem Leben machen wollen, was sie für richtig halten. Außerdem hat uns die Armee wirklich dabei geholfen Arbeitsplätze zu schaffen ganz zu schweigen von ihren Diensten, als die Capital Fellowship Bomben in Menschenmengen geworfen hat.}

\par

Otis winkte wütend ab. \WR{Ach hören Sie doch auf. Die Capital Fellowship war kurzzeitig eine große Bedrohung, das gebe ich zu. Aber sie waren nichts, womit eine nichtmilitärische Einheit nicht ebenfalls fertig geworden wäre.}

\par

\WR{Sagen Sie das der Crew der Sentinel}, gab Lertha Akintola gefährlich ruhig zu bedenken.

\par

Das Argument war gut. Die Sentinel war ein mittlerer Träger gewesen, der durch einen Bomberangriff der Capital Fellowship zerstört worden war. Und das trotz modernster Abwehrtechnologie und zahllosen Kampffliegern zu seinem Schutz.

\par

Henry Otis lachte nur kurz auf, und antwortete dann: \WR{Dass es die Starforce nicht einmal schafft, ihre Raumschiffe vor diesen prähistorischen Irren in Sicherheit zu bringen, spricht für sich.}

\par

Die Vizepräsidentin schnaubte wütend. Sie hatte etwas in ihren Augen, dass dem Präsidenten zuvor noch nicht aufgefallen war. Er spürte es deutlich. Etwas war anders. Otis und Akintola hatten sich zwar schon oft in die Haare gekriegt aber jetzt war es wirklich ernst.

\par

\WR{Sie haben in Ihrer Ansprache von gestern den Routenkrieg erwähnt}, sprach Lertha Akintola leise. \WR{Und mit einem hatten Sie Recht. Der Krieg hätte verhindert werden können. Aber es war nicht das Militär, dass unbedingt Blut vergießen wollte. Es waren die sturen und kurzsichtigen Politiker beider Seiten, die Schuld waren.}

\par

Der Präsident durchbohrte seine Gesprächspartnerin mit seinen Blicken. \WR{Sie sollten sich vielleicht einmal überlegen, ob nicht Sie die sture kurzsichtige Politikerin sind. Das Commonwealth und die Erdallianz haben sich damals so lange gegenseitig umgebracht, bis sie nicht mehr konnten. So lange bis Admiral Vergara einen erfolgreichen Putsch durchgeführt hatte. Es hätte gereicht, wenn nur einer eingelenkt hätte. Genauso ist es heute. Die Fronten sind verhärtet und nun geht es darum, wer länger durchhält. Und diesmal gebe ich nicht klein bei. Ich stehe ein, für das was ich glaube.}

\par

\WR{Mein Gott. Sie haben wirklich keine Ahnung, oder?}, begann die Vizepräsidentin aufs neue. \WR{Jeder Mensch soll frei sein, von Geburt an. Und sein Leben gestalten und glauben, wie er es für richtig hält}, zitierte sie die erste Grundfeste. \WR{Was Sie glauben, mein Herr, ist nicht mehr wert, als das was andere Menschen denken.}

\par

Präsident Otis ließ sich in seinen Sessel fallen. Er schloss die Augen und sagte: \WR{Ich denke, es ist besser, wenn Sie jetzt gehen.}

\par

Lertha Akintola machte auf dem Absatz kehrt und ging auf die Türe zu. \WR{Erwarten Sie keine weitere Unterstützung}, sagte sie tonlos, bevor sie gänzlich verschwand.
