Das leerstehende Krankenhaus wirkte so trist wie unauffällig und war damit ein Spiegelbild der ganzen Gegend. Die Po-Ebene erstreckte sich über mehrere dutzend Kilometer südlich der Alpen und war früher ein wichtiger Industriestandort des ehemaligen Italien gewesen, das heute, genauso wie alle anderen Staaten der Vorseuchenerde nur noch als Verwaltungspolygon existierte.

\par

Nachdem billiger und vor allem nicht unter Naturschutz stehender Lebensraum auf weniger lebendigen Welten wie der Erde verfügbar geworden war, hatte es auch die hiesige Industrie auf andere Planeten gezogen. Zurückgelassen hatte sie dabei eine weite Fläche, die zwar mit den Jahren immer grüner wurde, deren schönster Anblick aber trotzdem nach wie vor die nahen Alpen bildeten.

\par

An diesem Morgen tröpfelte es ein wenig. Der Himmel war wolkenverhangen und der Horizont verschwand wie in einer grauen Suppe. Laura störte sich nur wenig am Wetter aber Klaus Rensing hatte sich mit seinem Lieblings-Kapuzenmantel ausgerüstet.

\par

Momentan galt seinen Aufmerksamkeit aber weder dem Regen noch dem Vorhaben, dass die beiden gerade verfolgten. Er sah angestrengt in sein Buch, dessen Seiten praktischerweise wasserabweisend waren.

\par

\WR{Scheiße, es stimmt also tatsächlich. Kreuzpunkt wird angegriffen.}
So besorgt hatte ihr Partner nur selten geklungen.

\par

\WR{Die Shutek?}

\par

\WR{Natürlich}, antwortete er, in fast schon genervtem Ton. \WR{Niemand weiß, was dort gerade vor sich geht. Das Konglomerat hat zwar die Starforce und die Phalanx geschickt aber die Kommunikation zum ganzen System ist abgebrochen.}

\par

Lauras Blick wurde leer. \WR{Wie vor ein paar Wochen in Pollux.} Dann schüttelte sie entschieden den Kopf und traf fest den Blick ihres Partners. \WR{Darum machen wir uns später Gedanken. Wenn du schon unbedingt bei diesem beruflichen Selbstmordversuch mitmachen willst, dann brauche ich dich voll konzentriert.}

\par

Klaus lächelte knapp und klopfte auf Lauras Schulter. \WR{Das klingt schon eher nach der Agentin, der ich vor ein paar Jahren zugeteilt wurde. Ich bin froh, das mich Han ins Vertrauen gezogen hat. Verrückt, dass du das allein durchziehen wolltest. Du hättest wissen müssen, dass ich dir dabei helfen will. Scheißegal, welche beruflichen Folgen das haben könnte.}

\par

Die beiden besahen sich wieder das verlassene Krankenhaus. Es war ein Bau aus der Blütezeit der Erdallianz und wirkte daher wie alle Häuser aus jener Ära nach einer billigen Kopie von Vorseuchenarchitektur. Dieses bestand aus einem Hauptteil und zwei in stumpfem Winkel abgeknickten Seitenflügeln. Die Fassade erinnerte ein wenig an die großen Bauten, wie sie im frühen zwanzigsten Jahrhundert im nunmehr völlig verfallenen New York gestanden hatten. Nur die graue Farbgebung passte nicht dazu. Dafür aber zur Phantasielosigkeit, für welche die Erdallianz zu ihrer Hochzeit bekannt gewesen war.

\par

\WR{Er ist also da drin?}, fragte Klaus ungläubig.

\par

\WR{Han hat einen offenen Datenknoten im Zentralcomputer dieses Klotzes geschaffen}, erklärte Laura. \WR{Ich hoffe, seine Arbeit wirkt überzeugend genug. Aber wenn es funktioniert, glaubt unser Hacker, da drin eine lokale Kopie diverser Dienstakten zu bekommen. Inklusive meiner und der von O'Shea. Sie sind gefälscht, um noch verlockender auszusehen und implizieren einen Zusammenhang zur Sektion.}

\par

Auch Klaus war der Verschwörungsmythos um die Sektion nicht unbekannt gewesen. Und tatsächlich stand er ihr weniger skeptisch gegenüber, als Laura oder Han. \WR{Ich hoffe nur, diese Typen gibt es nicht wirklich. Denn falls doch, dann könnte es gut sein, dass sie Wind von unserer Idee bekommen haben und heute auch hier auftauchen.}

\par

Wie aufs Stichwort zog Laura ihre Strahlenpistole aus dem gut unter ihrem schwarzen Mantel versteckten Halfter. \WR{Glaube ich nicht. Aber wenn, dann hätte ich einige Fragen an diese Leute.} Der Lauf der Waffe sprang auf und gab eine drehbare Spule frei. Laura ließ sie drehen so schnell sie konnte und damit ein unerwartet beruhigendes Klicken erklingen. Zusammen mit dem hochfrequenten Zirpen, dass mit der steigenden Ladung des Kerns einher ging.

\par

\WR{Ziemlich unfair, dass ihr Leute vom Geheimdienst dieses Spielzeug zuerst ausprobieren dürft.} Nun klang Klaus so ehrlich neidisch, wie er zuvor besorgt geklungen hatte.

\par

\WR{Tröste dich}, antwortete Laura. \WR{Wir kriegen dieses Zeug noch vor der Phalanx.} Was zwar stimmte, jedoch darüber hinwegtäuschen sollte, dass sie die Pistole für sündhaft viel Schwarzgeld illegal erworben hatte. Ebenso wie darüber, wofür sie sich die Waffe zugelegt hatte.

\par

Ihr Partner zog eine doppelläufige Schrotflinte aus seinem Ledermantel, was ihm einen verwunderten Blick einbrachte. \WR{Was?}, fragte er verwirrt. \WR{Ich bin hier in meiner Freizeit. Mehr als dieses Ding konnte ich eben nicht auftreiben.}

\par

\WR{Wo hast du das her?}, wollte Laura sofort wissen. \WR{Aus dem Museum? Da bist du ja mit einem Messer besser dran.}

\par

\WR{Steckt in meinem Stiefel.}

\par

Die beiden traten auf das große Eingangsportal des Krankenhauses zu. Die beiden Türen waren einmal aus Glas gewesen. Nun bestanden sie nur noch aus einigen Bruchstücken und einem Scherbenhaufen. Auf der Erde waren Gebäude, für die sich niemand interessierte, sehr selten geworden. Dennoch gab es sie.

\par

Laura hielt ihre Waffe mit beiden Händen fest und duckte sich unter dem zersplitterten Glas weg. Klaus folgte ihr auf dem Fuß.

\par

Das Innere der Empfangshalle war dunkel und wirkte, als sei seit Jahrzehnten niemand mehr darin gewesen. Hinter der Rezeption stand ein alter Drehstuhl, dessen Futter zerrissen heraus hing. Der Tresen war teilweise auseinandergenommen worden. Allerdings musste den Arbeitern irgendwann die Lust vergangen sein und sie hatten den Rest einfach stehen gelassen. Das monotone Tropfen von Wasser durch ein kleines Loch in der Decke rundete den verfallenen Eindruck des Gebäudes ab.

\par

\WR{Der Computerraum ist im ersten Obergeschoss}, flüsterte Laura und begab sich zur Tür des Treppenhauses.

\par

Ihr Partner griff in seinen Jackentasche und zog einen kleinen Zylinder heraus. Nach einen leisen Klicken gab dieser einen unauffälligen Lichtschein ab. Klaus hatte absichtlich keine allzu helle Taschenlampe mitgenommen, um den Gesuchten nicht zu alarmieren.

\par

\WR{Was machen wir, wenn er versucht, abzuhauen?}, wollte er schließlich wissen.

\par

\WR{Wir rufen Verstärkung}, war Lauras Antwort, die so leise wie monoton klang. \WR{Aber erst, wenn wir bestätigt haben, dass er wirklich hier ist. Sonst sehen wir wieder dumm aus.}

\par

Die Agentin sah aus einem der Fenster. Das Tageslicht, das von außen in den grauen Bau hineindrang, reichte kaum aus, um das Innere zu erhellen. Zumindest gab es nirgendwo Anzeichen eines Fluchtwagens oder einer anderen Möglichkeit, sich schnell abzusetzen. Das hatten die beiden bereits überprüft, bevor sie das Gebäude betreten hatten.

\par

Klaus schien beim Blick aus dem Fenster an das Gleiche gedacht zu haben. \WR{Seltsam}, sinnierte er. \WR{Dieser Kerl ist bisher so vorsichtig gewesen. Aber jetzt rennt er ohne Rückendeckung oder Fluchtstrategie einfach so in die Falle.}

\par

Laura ließ sich Zeit, um zu antworten. Zunächst konzentrierte sie sich auf die Ecke, die sie gerade umrundete. Als ihr der Bereich dahinter sicher erschien, sagte sie: \WR{Da gibt es eine Reihe von Möglichkeiten. Vielleicht weiß er, dass wir alleine und inoffiziell nach ihm suchen. Wenn das so ist, dann versucht er möglicherweise, uns einfach zu erledigen. Zwei Probleme weniger für ihn.}

\par

Die beiden durchquerten einen Korridor mit scheinbar unzähligen Krankenzimmern, von denen zumindest die Türen alle identisch aussahen. Lediglich die Nummern~-- noch im klassischen Zehnersystem aufgedruckt~-- unterschieden sie voneinander. Die Zimmer dahinter zeigten allerdings differenzierte Stufen des Verfalls. Bei einem fehlte so gut wie die ganze Inneneinrichtung, einschließlich des Fensters. Bei einem anderen hatte das Bett sogar noch eine Decke, die allerdings völlig vergilbt und schmutzig war.

\par

\WR{Wieso lässt man so eine Bruchbude stehen?}, fragte Klaus ehrlich verwundert.

\par

Laura hörte ihm nicht zu. Stattdessen versuchte sie, sich auf ein Geräusch am Ende des Ganges zu konzentrieren. Es war so leise gewesen, dass sie nicht einmal sicher war, ob sie es wirklich gehört hatte.

\par

Schließlich erklärte sie: \WR{Es gab schon mehrere Volksentscheide in Mailand, um dieses Ding abzureißen. Aber alle wurden aus finanziellen Gründen zurückgewiesen. Scheinbar gab es wichtigeres.}

\par

Von Klaus folgte nur verständnisloses Brummen. Dann schreckten beide auf. Laura merkte erst, dass es eine Maus war, die ihr zwischen den Beiden hindurch rannte, als sie bereits durch Kimme und Korn auf den kleinen Nager zielte.

\par

\WR{Das war ja klar}, schimpfte Klaus etwas lauter, als es sinnvoll gewesen wäre.

\par

\WR{Hier lang}, erwiderte Laura nur und zeigte auf einen Raum, dessen Türe sich doch von den anderen Unterschied. Sie hatte eine Schleuse wie diese bereits einmal gesehen. Das erweiterte Immunsystem, dass fast jeder Mensch der Union in sich trug, machte Infektionen, die tatsächlich zum Ausbruch kamen, sehr selten. So hatte nicht jedes Krankenhaus mehr eine Isolationsstation. Dieses hier allerdings schon.

\par

Das schwere Zugangsschott war entfernt worden, aber der Durchgang, der wie eine Ziehharmonika von innen aussah, war erhalten geblieben. Die flexiblen Wände des Korridors bestanden aus nahtfreiem Kunststoff und waren damit völlig luftundurchlässig.

\par

Der Lauftsteg quitschte, als Laura und Klaus darüber liefen. Instinktiv versuchten beide, leiser zu sein. Doch das alte Metall bog sich trotzdem gleichermaßen lautstark unter ihren Füßen. Keiner sagte etwas, doch Laura atmete auf, als sie den Korridor endlich hinter sich gebracht hatte und wieder auf normalem Laminat stand.

\par

Doch sie fuhrt augenblicklich zusammen, als zwei kleine Notfallschotten aus Plexiglas an den Enden der Schleuse unverhofft aus dem Boden schossen und Klaus im Inneren einschlossen.

\par

\WR{Scheiße!}, schrie dieser lauthals und fuhr mit seiner Flinte im Anschlag herum. Doch auf der anderen Seite der Schleuse hatte sich mit Ausnahme der ausgefahrenen Glaswand nichts verändert.

\par

Laura schoss auf die Türkontrolle zu. Doch die Konsole schien überhaupt keine Energie zu bekommen. Keiner der Knöpfe oder Schalter reagierte und auch die Anzeigefläche blieb dunkel.

\par

\WR{Hol mich hier raus!}, rief ihr Klaus entgegen und wirkte dabei fast panisch.

\par

\WR{Alles klar}, schrie sie zurück. \WR{Piss dich nicht ein. Jetzt wissen wir zumindest, dass unser Mann hier ist. Er hat wohl eine eigene Türverriegelung eingebaut.}

\par

Laura sah sich hastig nach dem Auslöser der Falle um. Schnell fielen ihr zwei Kabel auf, die von dem Plexiglasschott weg führte. Sie folgte den dünnen Leitungen bis zu einer Klappe in der Wand. Hastig riss sie die Lüftungsabdeckung ab. Jedoch nicht, ohne dabei zu bemerken, diese nicht so voller Staub war, wie der Rest des Raumes. Dahinter kam ein Handcomputer und eine Batterie zum Vorschein.

\par

\WR{Alles klar}, murmelte Laura, als sie das Gerät in Augenschein nahm. \WR{Dieses Ding hält die Schotten geschlossen. Sobald die Batterie leer läuft, sollten sie wieder aufgehen.}

\par

Klaus, der von seiner Position aus nicht sehen konnte, was vor sich ging, drückte förmlich seine Nase gegen das Glas. \WR{Kannst du die Batterie nicht einfach abklemmen?}, rief er ihr entgegen. Seine Stimme klang durch das Glas hindurch gedämpft. Doch selbst so war nicht zu überhören, wie unheimlich ihm zumute war. \WR{Ich bin hier in einem luftdichten Raum eingesperrt. Wehe als nächstes strömt hier Säure oder Giftgas rein.}

\par

Laura wollte gerade die Kontakte lösen, als sie zurückschreckte. \WR{Scheiße!}, rief sie erschrocken. Dann sah sie zu Klaus. \WR{Da ist noch eine Sprengkapsel! Wenn ich die Batterie entferne, dann fliegt sie mir um die Ohren.}

\par

Klaus fuhr sich mit den Händen nervös durch die Haare. \WR{In Ordnung… Kannst du sie vielleicht einfach mit deiner Strahlenkanone zusammenschmelzen?}

\par

\WR{Zu gefährlich}, war Lauras sofortige Antwort. \WR{Die Batterie ist sowieso in ein zwei Minuten leer. Ich denke, diese Falle sollte uns bloß aufhalten.}

\par

Als wäre das ein Stichwort gewesen, erklangen Schritte im Gang hinter Laura. Hastig fuhr sie herum, sah aber nur noch einen Schatten hinter einer Abzweigung verschwinden.

\par

\WR{Das ist er!}, rief sie. \WR{Er versucht wieder, sich abzusetzen. Halt durch, ich komme wieder!} Mit diesen Worten nahm sie die Verfolgung auf. Klaus hämmerte zwar wie wild gegen die Scheibe, doch Laura nahm die dumpfen Schläge und rufe fast kaum noch wahr.

\par

Stattdessen spurtete sie um die Ecke, hinter der sich der Unbekannte gerade geflüchtet hatte. Ihr Waffe hielt sie schussbereit, aber auf die Boden gerichtet, so dass man sie ihr nicht einfach aus der Hand schlagen konnte, wenn sie um die Biegung kam.

\par

Vor ihr erstreckte sich ein weiteres Treppenhaus. Von unten her drang das Geräusch trampelnder Füße. Mehr als einmal stolperte sie fast, als sie so schnell sie konnte die Stufen hinunter rannte.

\par

Eine große Flügeltüre schwang immer noch nach außen und wieder nach Innen, als sie unten ankam. So vorsichtig und langsam sie konnte, schob sie sich hindurch und fand sich in der Messe des Krankenhauses wieder.

\par

Auf der Terrasse standen keine Möbel mehr aber im Inneren waren nach wie vor etliche Tische mehr oder weniger ordentlich aufgereiht. Durch die offensichtlich undichte Fensterreihe drang bereits Wasser nach innen. Der Küchenbereich lag ziemlich im Dunkel, doch es war deutlich zu erkenne, dass dort jedes halbwegs wertvolle Utensil entfernt worden war. Wo vielleicht mal eine Spülmaschine gestanden hatte, klaffte nun nur noch ein großes Loch.

\par

Selbst in diesem nahezu leeren Raum war der Hacker kaum zu erkennen. Er war neben der zweiten Zugangstür der Cafeteria in die Hocke gegangen, eine Pistole im Anschlag.

\par

Laura zielte sofort auf ihn, drückte aber nicht ab. Ihre Ausbildung an der Waffe hatte ihren Weg in ihre Instinkte und ihr muskuläres Gedächtnis gefunden. Hätte er sie töten wollen, hätte der Hacker längst geschossen.

\par

\WR{Lassen Sie die Waffe fallen!}, schrie Laura so laut sie konnte. Zunächst machte ihr Gegenüber keine Anstalten, dieser Anweisung nachzukommen. \WR{Ich sag es nicht noch einmal! Lassen Sie Ihre Pistole fallen oder ich drücke ab.} Ihr Finger spannte sie bereits um den Abzug und zog ihn schon einmal ein paar Millimeter zurück.

\par

Nach einem weiteren Augenblick, der Laura wie eine Ewigkeit erschien, legte der Mann tatsächlich seine Pistole auf den Boden. Langsam und auf jede Bewegung bedacht erhob er sich und trat ins Licht. Schnell blieb für Laura kein Zweifel, dass es sich tatsächlich um den Mann handelte, den sie in Freiburg bereits fast gestellt hätte. Auch wenn er sich seinen Schnurrbart abrasiert hatte~-- falls dieser überhaupt jemals echt gewesen war. Auch seine Augenhöhlen schienen tiefer geworden zu sein und er wirkte so bleich wie hohlwangig.

\par

\WR{Ihr Partner?}, fragte der Mann tonlos.

\par

\WR{Wird definitiv bald hier sein}, mahnte Laura.

\par

\WR{Dann habe ich nicht mehr viel Zeit, Ihre Fragen zu beantworten.}