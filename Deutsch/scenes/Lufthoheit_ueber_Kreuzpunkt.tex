Morten riss das Steuer erneut herum. Gerade hatte er den ersten Shutek abgeschüttelt, da hatte sich schon der nächste an sein Heck geheftet. Er kämpfte gegen seinen Fluchtreflex an und richtete seinen Jäger auf eines der Landeschiffe der Phalanx aus. Nach dem Herausfallen aus dem Hyperraum hatten sich die Shutek schon über die Verstärkung hergemacht. Mindestens zwei Geschwader an feindlichen Maschinen befand sich im tiefen Orbit um Kreuzpunkt Primus.

\par

Die \EN{Heinlein} und ihre beiden Begleiter hatten sofort den Kurs geändert, in der Hoffnung, die Gegner auf eine falsche Fährte zu locken. Allerdings hatten sie nur knapp ein dutzend Jäger an sich binden können. Die restlichen stürzten sich bereits auf die anfliegenden Landetransporter.

\par

Diese massigen Schiffe, von denen jedes bis zu vierzig Soldaten oder aber größeres Angriffsgerät aufnehmen konnte, hatten Not, den Angreifern auszuweichen. Ihre Waffen waren dafür geeignet, eine Landezone leer zu räumen. Gegen feindliche Jäger waren sie fast nutzlos. Selbst die beiden Raketenbatterien an den Rümpfen der Klötze waren dazu nicht zu gebrauchen. Sie waren nur für Bodenziele gedacht.

\par

Morten eröffnete das Feuer auf einen anfliegenden Jäger. Das Schiff mit breiten Flügeln steckte nur zwei bis drei Treffer ein, die zwar auf seine Blocker einprasselten, das Schiff an sich aber nicht beschädigten. Trotzdem drehte der Flieger ab. Morten verfolgte ihn nicht. Es ging nun nicht um Abschüsse, sondern nur darum die Transporter zu schützen. Dabei hieß es: Schießen, Abdrehen, schnell Zurückkommen und das ganze wieder von vorne.

\par

\WR{Mann, sind die scheißeschnell}, fluchte Kevin über Funk.

\par

Morten stimmte ihm im Geiste zu, antwortete aber nicht. Er war zu beschäftigt damit, das Radar im Auge zu behalten. Dann gellte ein Hilferuf durch den Äther. \WR{Claudius elf hier! Wir nehmen viele Treffer hin!}

\par

Das Radar zeigte den entsprechenden Transporter. Dann war der Blickpunkt auch schon wieder verschwunden. In einiger Entfernung flammte das getroffene Schiff auf und ging dann in einem Feuerball auf. Der angsterfüllte Schrei des Piloten hallte noch einen Augenblick nach, bevor die Übertragung abbrach.

\par

Doch der Übeltäter war diesmal kein Jäger. Eine feindliche Corvette~-- ein kleines Großkampfschiff, dass einen schweren Bomber in seiner Größe kaum übertraf~-- hatte sich in die Flugbahn der herannahenden Transporter begeben.

\par

\WR{Rot drei!}, hörte Morten Farley sagen. \WR{Fetter Bogey auf Komma null zehn! Zielbezeichnung \Wr{Bazuzu}, sehen Sie ihn?}

\par

\WR{Positiv}, gab der gefragte schnell zurück.

\par

\WR{Waffenstatus?}

\par

\WR{Habe noch drei Verfolgungsraketen und zwei ungelenkte.}

\par

\WR{Der ist ihrer}, sagte Farley. \WR{Rot vier, beidrehen. Wir geben Feuerschutz!}

\par

Ohne weitere Worte nahm Morten die Corvette ins Fadenkreuz. Im Augenwinkel sah er, wie Anna Farley und Kevin Wilson neben ihm in Stellung gingen.

\par

Die Corvette roch den Braten und richtete sofort zwei ihrer Geschütze auf die anfliegenden Jäger. Morten riss seinen Steuerknüppel hin und her, als die Strahlen an seinem Jäger vorbeizogen.

\par

\WR{Rot vier, du bist zu dicht!}, warnte er Kevin, bei einem besonders gewagten Ausweichmanöver. Dieser wich wortlos aus und gab einige Schüsse aus seinen Bordkanonen ab. Die Strahlen prallten auf die Schutzfelder der Corvette, ohne dass diese darauf reagierte.

\par

Morten erkannte langsam aber sicher mehr von seinem Ziel. Besonders die grell leuchtenden Triebwerke fielen ihm ins Auge. Wenn die Shutek ihre Schiffe auch nur halbwegs ähnlich gestalteten, wie das Konglomerat, dann zählten diese zu den verwundbarsten Stellen. Doch obwohl die Düsen sehr auffällig waren, nahmen sie vergleichsweise wenig Platz ein und würden schwer zu treffen sein.

\par

Morten hielt seinen Jäger jetzt ruhiger, was ihn gefährlich nah an die Strahlen des Abwehrfeuers brachte. Eine der giftgelben Entladungen kam ihm sogar so nah, dass sein ganzes Cockpit für Kurze Zeit in ihrer Farbe leuchtete.

\par

Sein Fadenkreuz fand die Plasmaturbinen der Corvette, der er nun rasend schnell näher kam. Mit pochendem Herzen klappte er den Sicherheitsheben an seinem Steuerknüppel hoch und machte die ungelenkten Raketen scharf. Mit größter Kraft kämpfte er den Drang zurück, die Geschosse sofort abzuschießen. Noch war die Corvette zu weit weg. Sie würde ausweichen können oder die Flugkörper vom Himmel holen. Die Entfernungsangabe zeigte noch knapp ein trin Meter an.

\par

\WR{Gib Zunder, Mann!} Kevin schrie, obwohl Morten über Funk keine Probleme hatte, ihn zu verstehen.

\par

Nur noch ein knapp einen halben Trinmeter. Ein Strahl streifte die Blocker um Mortens Jäger und brachte jeden Alarm in seinem Cockpit zum aufheulen. Sein Fadenkreuz lag nun genau über der großen Turbine an Steuerbord.

\par

\WR{Abdrehen!}, gab Morten über Funk durch und zerdrückte förmlich den Auslöser für die Raketen. Die beiden Geschosse lösten sich von unter den Tragflächen und sausten auf ihr Ziel zu. Als sie auftrafen, war Morten bereits abgedreht. Nur aus dem Augenwinkel nahm er noch war, wie die mit Sprengstoff vollgestopften Raketen in die Luft gingen und die beiden Turbinen der Corvette in tausend Stücke sprengte.

\par

Es folgten schnell Folgeexplosionen, die das ganze Schiff zerriss. Schnell blieb nur noch ein Feld von Trümmern übrig, durch welche schnell die ersten Landetransporter der Phalanx durchstießen.

\par

Der Verlust der Corvette schien die feindlichen Reihen durcheinander gebracht zu haben. Es dauerte tatsächlich eine Weile, bis die Jägerstaffeln der Shutek wieder zurück zu seiner koordinierten Formation fanden.

\par

Durch einige Beifallrufe konnte Morten Kevin sagen hören: \WR{Dieser Kübel zählt trotzdem nur als ein Abschuss!}
