Die Szenerie erschien viel ruhiger und friedlicher, als sie es wohl tatsächlich war. Schwarzer Bär pustete sich in seine Hände. Sofort kondensierte Wasser in der Luft und stieg in die Schwärze der Nacht auf. Seine Handschuhe hatte er an abgelegt, weil er mit ihnen das Gefühl, den Griff seiner Waffe zu spüren, vermisste.

\par

Mittlerweile war es längst dunkel geworden. Irgendwo in der Richtung, in die Schwarzer Bär gerade sah, musste die Haupstadt liegen. Sie selbst war durch den dichten Schneefall und den Nebel der Schlacht nicht zu erkennen, doch Yêxīns Lichter strahlten in ihrem unverkennbaren Orange durch die Nacht.

\par

Hin und wieder blitzte jedoch ein heller Lichtschein auf, der von der entfesselten Schlacht zeugte. Genauso wie das Stampfen und Donnern und das grelle Klirren, dass Schwarzer Bär inzwischen als das Hauptgeschütz der Riesenspinnen kannte.

\par

Auch im Himmel blitzte es hin und wieder. Die Wolkendecke um den Spechtgipfel~-- Yêxīns Hausberg~-- war zwar dicht, doch die Spuren des Luftkampfes blieben dennoch deutlich. Mittlerweile konnte Schwarzer Bär die Antriebe der Shutek-Jäger von denen der Starforcemaschinen relativ gut unterscheiden. Bei den ersteren mischten sich auf seltsame Art ein giftiges Grün in den sonst blutroten Antriebsstrahl.

\par

Hier und da zuckten auch die Entladungen von Strahlenkanonen durch die dicke Wolckendecke. Und schließlich stürzten Feuerbälle auf den Boden.

\par

\WR{Der Himmel brennt}, begann Fischauge, \WR{der Boden brennt. Herzlich willkommen beim Ende der Welt.}

\par

Schwarzer Bärs Blick löste sich nicht von den Wolken. Es tauchten zwar immer wieder die satt orangenen Antriebsflammen mit blauem Kern auf, die unübersehbar zu den Starforce-Jägern gehörten, doch die Shutek schienen die Übermacht zu besitzen.

\par

Das Geräusch war leise und verschwand schnell im Rauschen des Windes und dem fernen Dröhnen der nicht mehr allzu fernen Schlacht. Aber dennoch hörten es viele der Frauen und Männer von dem, was einmal Rana Salis Division gewesen war. Das metallischen Rattern von Ketten, die angezogen und dann wieder locker gelassen wurden. Schwarzer Bär hatte es nicht genau gesehen, aber er glaubte zu wissen, dass dies tatsächliche Ketten waren, welche die Riesenspinnen zur Fortbewegung brauchten.

\par

\WR{Gehen wir weiter!}, befahl er laut an alle gerichtet. Nach Salis Tod und dem, aller anderen Offiziere, war er nun der Kommandant. Und das nur, weil seine Ausbildung bereits am weitesten vorangeschritten war. Er hatte sie nicht einmal abgeschlossen und nun trotzdem Centurio auf Probe.

\par

Fischauge winkte den anderen Überlebenden zu. Das traurige Dutzend verstand seine Handzeichen und begab sich in eine kollektiv geduckte Laufhaltung.

\par

Eine junge Frau~-- Schwarzer Bär kannte sie blos als \Wr{Minx}~-- schloss zu ihm auf. Ihre weiße Haut passte erschreckend genau zur Farbe des Schnees. Und ihre schwarzen Haare zur Finsternis. \WR{Centurio, ich habe Präfekt Saint John an der Strippe.} Sie deutete auf ihr Funkgerät. Und obwohl dieses bereits antiquiert war, gab es keinen Fingerzeig darauf, woher ihre Redewendung ursprünglich stammte. \WR{Die Shutek überrennen die Stadtgrenzen. Edhor Peak ist völlig ausgebrannt. Luftangriff. Keine Nullzonenbomben aber es ist wohl trotzdem ziemlich schlimm.}

\par

\WR{Dann sollten wir uns beeilen}, sagte Schwarzer Bär. \WR{Ich will die Feier keinesfalls verpassen. Nur ein kleiner Sturmangriff in drei Stunden ist mir zu wenig.}

\par

Minx sah auf den Boden. In Wirklichkeit hörte sie aber konzentriert auf das, was man ihr durch ihr Headset mitteilte. Mit ihrem Zielmonokel und Kontakten in den Spitzen ihrer Handschuhe konnte sie die relativ komplizierte Funkanlage kontrollieren, die sie nun mit der Kommandoebene verband.

\par

\WR{Centurio, Legat Gajjar denkt über einen Rückzug nach. Wir sollen \textit{nicht} zur Hauptstadt ziehen, sondern uns zu einem der Extraktionspunkte begeben.}

\par

\WR{Das kann doch nicht ihr Ernst sein!}, spie Schwarzer Bär aus. \WR{Yêxīn braucht jede Frau und jeden Mann. Alle, die eine Waffe tragen können.}

\par

Minx schüttelte den Kopf. \WR{Boss, ich höre mir gerade alle gängigen Frequenzen an. Die Schlacht ist verloren. Wir müssen hier weg. Nur wenn wir heute leben, dann können wir morgen noch weiterkämpfen.}

\par

Wieder ließ Schwarzer Bär seinen Blick über das breite Tal schweifen, an dessen Ende Yêxīn brannte. Selbst wenn es nicht eine so finstere Nacht gewesen wäre, hätte sich die Division kein gutes Bild über die Situation machen können. Dazu war ihre Position zu ungünstig.

\par

Der gerade erst inoffiziell beförderte Centurio musste ein wenig lauter sprechen, um das Knallen einiger Mörser zu übertönen, die selbst aus einigen Kilometern Entfernung noch lauthals Feuer spuckten. Auch die Einschläge ihrer Waffen dröhnten wie ein Donner durch das Tal.

\par

\WR{Was sind unsere Alternativen, Immunes?}, fragte er. \WR{Wo ist der nächste Extraktionspunkt?}

\par

Minx deutete in Richtung des Hügels, auf dem sie vor nicht ganz drei Stunden gelandet waren. \WR{Fünfzehn Kilometer in dieser Richtung. Aber es ist unklar, wann die Starforce uns rausholen kann. Derzeit sammeln die Truppentransporter wohl Divisionen aus dem Landesinneren ein und und fliegen sie zur Hauptstadt.}

\par

\WR{Vergessen sie's}, urteilte Schwarzer Bär spöttisch. \WR{Selbst, wenn wir es rechtzeitig da hin schaffen, ist es nicht sicher, ob wir da auch jemandem begegnen.}

\par

Sein Blick ging mehr oder weniger gleichzeitig an zwei Orte. Der eine war der Spechtgipfel. Ein Berg von etwas weniger als anderthalb Kilometern Höhe. Außer einem großen Felsen, der scheinbar völlig deplaziert im Tannenwald seines Nordhanges stand, schien er eher unscheinbar. Nur das Licht der beiden Monde ließ ihn scheinbar in der Dunkelheit dezent leuchten.

\par

Zum anderen sah er sich die taktische Übersichtskarte an, die sein Zielmonokel ihm anzeigte.

\par

\WR{Da oben gibt es einen kleinen Stützpunkt der Starforce}, sagte er. \WR{Was ist denn damit?}

\par

Minx schüttelte sofort den Kopf. \WR{Hat eine Bombe abbekommen. Gleich beim ersten Luftangriff. Kein Funkkontakt. Da oben sind alle tot.}

\par

\WR{Scheiße}, fluchte Schwarzer Bär leise. Der Aufstieg wäre ohnehin nicht einfach gewesen. Aber ohne, dass es dort jemanden gab, der sie in Empfang nahm, war es sinnlos.

\par

\WR{Also doch kämpfend untergehen?}, fragte Fischauge und versuchte dabei, seiner Stimme einen epischen Ton zu verleihen. Allerdings schwang darin eine deutliche Angst mit.

\par

Schwarzer Bär hatte absolut kein Bedürfnis danach, in einem sinnlosen Kampf zu sterben. Er war der erste, der zuschlug aber hatte sich nicht alleine damit durch sein Leben gekämpft. Auch der Rückzug gehörte dazu.

\par

\WR{Was ist mit dem Notsignal in drei Trinmeter Entfernung. Süd-Süd-West.}

\par

Minx sah sich ihre eigene Übersichtskarte an. \WR{Könnte ein abgestürzter Jäger der Starforce sein. Keine Ahnung.}

\par

\WR{Da müssen wir hin!}, entschied Bär kurzerhand.
\WR{Wenn der Pilot noch lebt, dann kann er uns hier rausfliegen.
Auf dem Berg brennt kein Feuer.
Vielleicht gibt's da oben noch ein paar einsatzbereite Maschinen.}

\par

\WR{\textit{Wenn} er noch lebt}, gab Kacper Piecek zu bedenken.
\WR{Und selbst dann wird es kein angenehmer Flug.}
Er brauchte nicht in den Himmel zu deuten, um auf die Lufthoheit der Shutek  hinzuweisen.
\WR{Ist das unsere beste Option?}

\par

Schwarzer Bär verdrehte die Augen. \WR{Mein Freund, die beste Option wäre jetzt, in einem schicken Hotel einzuchecken und sich in einen Whirlpool zu legen. Aber ich sehe hier nirgends einen.}

\par

Minx atmete schwer. \WR{Ich glaube, ich hab die Kiste vor etwa zwei Stunden abstürzen gesehen. Kam aus Richtung Edhor Peak. Es waren zwei. Einer hat's geschafft, dem anderen ist eine Rakete in den Arsch geflogen. Ich bekomme keine Funksignal. Irgendwie glaube ich nicht, dass die arme Sau noch lebt.}

\par

\WR{Er muss!}, donnerte Schwarzer Bär. \WR{Abmarsch!}