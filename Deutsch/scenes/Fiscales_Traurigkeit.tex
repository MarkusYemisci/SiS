Abdel Samad traute seinen Augen kaum. Die Kabinentür der Kommandantin war offen gewesen und er hatte sich nur kurz über die Sprechanlage angekündigt, bevor er eingetreten war. Natalia Fiscale saß an ihrem Schreibtisch, das Gesicht in ihren Armen vergraben und schluchzte. Als sie ihren Gast bemerkte drehte sie ihren Sessel gegen die Wand und wischte sich mit einem Ärmel ihrer Uniform über das Gesicht.

\par

Commander Samad schluckte und überlegte, ob er besser wieder gehen sollte. Doch die Kommandantin erhob sich und trat ihm entgegen. Ihre Augen waren gerötet und unter ihnen hatten sich blaue Ringe formiert, die ihr von den letzten beiden schlaflosen Nächten geblieben waren. Sie räusperte sich, bevor sie fragte: \WR{Was kann ich für Sie tun, Commander?}

\par

Abdel Samad entgegnete möglichst einfühlsam: \WR{Ist alles in Ordnung, Natalia?}

\par

\WR{Nein!}, gab diese gereizt zurück. \WR{Absolut nicht. Ein Haufen Mistkerle versucht uns umzubringen. Und bei der Pollux Kolonie haben sie es geschafft. Fast vier Millionen Tote. Und wir haben die Hälfte unserer Piloten verloren. Ich habe sie alle gekannt. Ich glaube, nichts ist in Ordnung.}

\par

Abdel Samad konnte den Blick der Kommandantin nicht halten und sah auf den Boden. Schließlich erwiderte er entschlossen: \WR{Wir haben getan, was wir konnten, Captain.}

\par

Natalia Fiscale lachte bitter. \WR{Wenn Sie damit Recht haben, kapitulieren wir besser gleich. Wenn das Beste, das wir tun können, darin besteht, zu retten was zu retten ist, dann ist das nicht viel.}

\par

Ihr erster Offizier versuchte, Worte zu finden. Ihm selbst ging es wenig anders. Doch das Adrenalin, dass immer noch durch seine Venen jagte, hielt ihn noch davon ab, viel nachzudenken. Er fragte sich, ob er genauso zusammenbrechen würde wie seine Kommandantin, wenn er sich einmal in seine Kabine zurückziehen und versuchen würde, sich zu entspannen. \WR{Ich wollte damit nur sagen, dass Sie sich nichts vorzuwerfen haben. Ich habe schon einige Kämpfe hinter mir, die weniger heiß hergingen als die Schlacht bei Pollux. Und nicht nur einmal habe ich Kapitäne dabei durchdrehen sehen. Aber Sie sind ruhig geblieben und haben Ihre Pflicht so gut erfüllt, wie man es nur erwarten kann. Ich weiß nicht, was noch kommt. Aber ich weiß, dass ich froh bin, auf diesem Schiff unter ihrem Kommando zu sein.}

\par

Nun war es an Captain Fiscale zu schweigen. Tatsächlich musste sie erneut Tränen zurück kämpfen, als sie sich daran erinnerte, was sie bereits mit ihrem ersten Offizier durchgemacht hatte. \WR{Nun, Commander}, sagte sie nach einer Weile, \WR{was wollten Sie mir mitteilen?}

\par

Samad nahm einen kurzen Blick auf seinen Handcomputer. \WR{Chefingenieur Bashir glaubt, dass die internen Schäden in spätestens drei Tagen beseitigt sein werden. Aber der Torpedotreffer und der zerstörte Geschützturm müssen in einem Trockendock repariert werden.}

\par

Fiscale hatte bereits mit diesen Meldungen gerechnet. \WR{Wie lange noch, bis wir Wega Primus erreichen?}

\par

\WR{Eine Stunde}, ließ sie ihr erster Offizier wissen. \WR{Die Raumstation im Orbit ist in voller Alarmbereitschaft. Ich habe bereits mit ihrem Kommandanten gesprochen. Er hat einen Platz für uns in einer der geostationären Werften reserviert. Durch die ganzen Etatkürzungen ist dort sowieso nicht viel los.}

\par

Captain Fiscale nickte ruhig. Die Gegenwart ihres ersten Offiziers beruhigte sie ein wenig. Zu wissen, dass sie nicht alleine stand, gab ihr wieder Kraft. So verließ sie ihr spartanisches Quartier und sagte: \WR{Wenn Sie möchten, können Sie mich aufs Lazarettdeck begleiten, Commander.} Und zu Samads Überraschung hängte sie an: \WR{Ich möchte mit unseren Neuzugängen reden.}

\par

Auf dem Weg dorthin, wurden die beiden Führungsoffiziere vom Kommunikationschef abgepasst. Nils Wallander wirkte seltsam nervös und sparte sich jeden bösen Blick, den er Samad normalerweise zuwarf, wann immer sich eine Gelegenheit dazu bat. Die Informationen, die er nun teilen wollte, hatte er sich nicht auf seinem Handcomputer gespeichert, sondern sie direkt in sein privates Buch geschrieben. Diese Unregelmäßigkeit verwunderte Captain Fiscale an ihrem Kommunikationsoffizier.

\par

\WR{Madam}, begann dieser schließlich betreten und vergewisserte sich, dass außer Samad und der Kommandantin niemand anwesend war. \WR{ich denke, ich weiß nun, wie es zu den vielen Störungen in unseren Funksystemen kommen konnte.}

\par

Fiscale begann das Blatt zu lesen, dass Wallander aus seinem Buch riss und ihr in die Hand drückte. \WR{Ein Virus?}, fragte sie ungläubig.

\par

Der Nachrichtenchef hielt seinen Ausdruck so gut im Griff, dass er damit jedes Pokerspiel für sich entschieden hätte. \WR{Ja, Madam. Es hat die Nullzonentranciever angegriffen und lahmgelegt. Eine physikalische Störung gab es nicht~-- zumindest, was unsere Langstreckenkommunikation betrifft. Die Kurzstreckensysteme wurden durch herkömmliche Störsender lahmgelegt. Aber damit kommen wir mittlerweile klar.}

\par

\WR{Wir wissen, dass die \EN{Virial} ähnliche Schwierigkeiten mit ihrem NZT hatte}, gab  Samad zu bedenken. Dass Wallander ihn nicht sofort anfuhr, zeugte vom Ernst der Lage. Stattdessen erklärte dieser: \WR{Genauso wie so ziemlich jede Einrichtung auf Pollux Primus, die einen NZT besaß. Es gibt noch keine Bestätigung, aber ich denke, sich das Virus von dort aus ausgebreitet hat. Sowohl wir als auch die \EN{Virial} hatten automatisch Kontakt mit mehreren Bodenstationen, als wir Pollux durchflogen haben.}

\par

Fiscales Gesicht wirkte so düster wie ein altes Gemälde. \WR{Das bedeutet, dieser Angriff war von langer Hand geplant.}

\par

\WR{Aber wer auch immer es war}, sinnierte Samad fort, \WR{kannte die Funktionsweise unserer Computer, sonst hätte er kein wirksames Virus entwickeln können. Wenn es wirklich Außerirdische waren, dann hatten sie Hilfe von Menschen.}
