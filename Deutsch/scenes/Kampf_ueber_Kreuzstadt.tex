\WR{Vorsicht, Rot drei}, warnte Anna Farley, \WR{du hast einen an der Ferse!}

\par

Morten sah nur auf sein Radar, jedoch nicht über seine Schulter oder auf den Monitor, der eine Ansicht der rückwärtigen Kamera zeigt. Tatsächlich folgten ihm zwei feindliche Jäger, während er selbst auf den Boden zuraste.

\par

Der Kommandant der Phalanx-Basis in Kreuzstadt war definitiv kompetent. Andauernd sprach er sowohl mit seinen eigenen Leuten als auch mit den Piloten über Funk und koordinierte die Verteidigung. Das Ausmaß des Überblicks, den er über das Schlachtfeld haben musste, beeindruckte Morten durchaus, denn er hatte ihm vor wenigen Augenblick die Position eines Gefechtsfelds durchgegeben, bei dem die feindlichen Reihen sehr anfällig für Jägerangriffe waren.

\par

Nun befand er sich im Tiefflug und schoss auf die genannte Stelle zu. Er konnte die kleine Schlucht bereits erkennen, durch die sich Truppen der Shutek näherten. Die Sonne ging gerade unter und schien ihm entgegen, doch seine Cockpitscheibe verhinderte, dass ihn ihre Strahlen blendeten.

\par

Währenddessen kamen aber auch die beiden Jäger mit Zielbezeichnung Iblis eins und zwei seinem Heck beunruhigend nahe. Demnächst wären sie in Feuerreichweite.

\par

Morten sah über die Nase seines Schiffes hinweg. Im Schein der schwindenden Sonne konnte er erkennen, wie eine große in die Richtung rannte, aus der er kam. Die Truppen der Phalanx waren geschlagen und verließen ihre Stellung. Wenn er nicht eingriff, würden die Shutek durchbrechen.

\par

Ohne weiter zu überlegen, reduzierte er seine Höhe noch mehr und brauste nur wenige dutzend Meter über die Köpfe der flüchtenden Soldaten. Dabei konnte er nicht sehen, wie diese ihre Gewehre in die Höhe rissen, und seinem Schiff zujubelten.

\par

Mit einem schnellen Daumendruck machte er seine leichten Strahlenkanonen scharf und eröffnete das Feuer. Im schwächer werdenden Licht der Dämmerung erhellten die Geschosse den Wald, der die Klamm bewuchs und tauchten die beschossene Zone zunächst in satt orangenes Licht und dann in ein Flammenmeer.

\par

Nur Sekundenbruchteile ließen auch die beiden Jäger der Shutek ihre Waffen sprechen. Morten musste hart hochziehen, um einen Treffer zu entgehen. Die Weinroten Entladungen bohrten sich in eine Felswand, an der er senkrecht in die Höhe stieg. Die Strahlen sprengten mühelos Brocken von der Größe eines Hauses aus dem Massiv. Donnernd löste sich eine Schuttlawine und überrollte die nunmehr brennende Schlucht. Falls es dort noch Shutek gegeben hatte, waren diese gerade von tonnenschwerem Gestein überrollt worden.

\par

Mortens Falke, ein windschnittiger mittlerer Jäger, war dafür gebaut worden, auch Flüge innerhalb einer Atmosphäre durchführen zu können. Die breiten Tragflächen sorgten für den nötigen Auftrieb, so, dass sich das Schiff auch ohne Antigravationstriebwerke in der Luft halten konnte. Doch nun steuerte er geradewegs in den Himmel. Der senkrechte Flug machte diesen Vorteil zunichte und er war ein leichtes Ziel.

\par

Darum ließ er den Falken über den rechten Flügel abkippen und raste nun ungebremst auf die Oberfläche zu. Im selben Moment zog er jedoch den Steuerknüppel so fest er konnte ab sich heran und stellte alle Klappen auf einen Steigflug ein. Der Schweiß trat ihm auf die Stirn, als er den Boden immer näher kommen sah. Die Baumwipfel wirkten bereits greifbar und kurz bevor er die Zapfen an den Tannen erkennen konnte, erreichte der Falken wieder eine fast horizontale Flugbahn.

\par

Ein Blick auf das Radar verriet ihm, dass er seine Verfolger zwar nicht abgeschüttelt aber zumindest ein wenig Abstand gewonnen hatte.

\par

\WR{Hier rot drei}, schrie er fast in sein Headset. \WR{Hab immer noch zwei Bogeys an mir dran. Ich werde sie nicht los!}

\par

\WR{Immer mit der Ruhe, Morty.} Nico Curiosas Stimme zu hören, ließ seine Angst nicht im geringsten schwinden. \WR{Habe die beiden Drecksäcke im Ziel. Halt nur noch etwas durch.}

\par

Mortens Jäger schoss über die Baumkronen hinweg und blies dabei den Schnee von ihren Ästen. Immer wieder musste er die Flugbahn den sich gabelnden Schluchten und Hügeln anpassen. Dabei gingen ihm schnell die Optionen aus. Hochsteigen konnte er nicht, da er sich damit sofort wieder ins Visier der Angreifer begab. Aber er wusste auch nicht, wie lange er noch in der Schlucht manöverieren konnte, in die er seinen Jäger gerade hineingleiten hatte lassen.

\par

Dann schossen zwei grüne Strahlenpaare an seinem Schiff vorbei und sprengten eine Felswand  zu seiner rechten in tausend Stücke.

\par

Kevin sah immer wieder auf das Radar. Sein Freund war in Schwierigkeiten, doch er konnte sich derzeit nicht von seinem eigenen Ziel lösen. Mit Geschwindigkeiten, die innerhalb einer Atmosphäre mehr als halsbrecherisch waren, jagte er zwei Bogeys nach.

\par

Einer der beiden drehte ab, kurz bevor die Raketenerfassung abgeschlossen war. Doch der andere hielt nach wie vor auf die Mauern von Kreuzstadt zu. Maschinengewehrfeuer sauste den beiden Fliegern entgegen. Doch selbst wenn die Geschosse den Shutek getroffen hätte, wäre dessen Schiff mit Sicherheit unversehrt geblieben.

\par

Dann fielen die ersten Schüsse von Seiten des Gegners. Lange giftig grün leuchtende Strahlen schlugen erbarmungslos gleich in mehrere Häuser ein und verwandelten sie nahezu augenblicklich in brennende Bretterhaufen.

\par

\WR{Scheiße!}, fluchte Kevin. Er scheute sich, seine Bordkanonen einzusetzen. Das Risiko, die eigene Stadt zu treffen, schien ihm bei weitem zu hoch. Aber die Standard-Verfolgungsrakete unter seiner Tragfläche würde noch mindestens fünf Sekunden brauchen, um ihr Ziel zu fixieren. Die einzige Chance bestand darin, tiefer als der Shutek zu fliegen und ihn dann von unten zu beschießen. So würden verfehlte Schüsse in den Wolken verschwinden.

\par

Nach einem tiefen Atemzug drückte Kevin seinen Steuerknüppel vorsichtig von sich weg. Sein Falke begann einen flachen Sinkflug und reduzierte die ohnehin bereits gefährlich geringe Höhe noch weiter.

\par

Nur Sekundenbruchteile später raste über zwei Dächer hinweg, deren Ziegeldecke sein Fahrtwind mit Leichtigkeit abdeckte. Weitere Häuser überflog er in einer Höhe von wenigen Metern.

\par

\WR{Hör auf mit dem Scheiß, rot vier!}, forderte Anna Farley über Funk. \WR{Du krachst gleich auf die Straße!}

\par

Kevin ignorierte die Übertragung. Er war zu beschäftigt damit, einem Plattenbau auszuweichen, der deutlich höher war, als die meisten anderen Gebäude. Seine Augen klebten an dem Shutek-Jäger.

\par

Dieser schien zu erahnen, was sein Verfolger vorhatte und erhöhte die Geschwindigkeit. Gleichzeitig wich er steuerbordseitig aus.

\par

\WR{Oh nein, du kommst nicht davon!}, sagte Kevin zu sicht selbst~-- aber mit eingeschaltetem Funkgerät. Dann zog er den Abzug voll durch und seinen Jäger nach oben. Bereits die zweite Salve fand ihr Ziel. Getroffen und brennend geriet der Shutek ins Trudeln. Kevin wollte ihm noch eine weitere Salve mitgeben um ihn in der Luft zu sprengen. Doch diese Schüsse gingen ins Leere und so begann der Shutek endgültig zu schlingern.

\par

\WR{Oh oh…}, brachte Kevin nur noch hervor, kurz bevor der Jäger in das Wahrzeichen von Kreuzstadt preschte. Was vor einer Sekunde noch ein dreihundert Meter hoher Turm aus Granitsteinen gewesen war, in dessen Mitte ein fast ebenso langes Pendel hin und her geschwungen war, verwandelte sich schnell in einen Haufen loser Steine, die in sich zusammen stürzten. Das Pende selbst war durch den Aufprall in der Mitte entzwei gebrochen und fiel nun so laut scheppernd zu Boden, das der hohle, metallene Ton noch durch die geschlossenen Cockpitscheiben der Unionsjäger zu hören war.

\par

Kevins Herz hätte ihm noch mehr bis zur Brust geschlagen, wenn der Turm nicht unzugänglich und der Platz darum menschenleer gewesen wäre.

\par

Morten begann Blut und Wasser zu schwitzen. Die beiden feindlichen Jäger waren mittlerweile erneut auf Kernschussdistanz herangekommen. Und schon zischten die Strahlen nur knapp an seinem Schiff vorbei. Die kurzen, hell und rot schimmernden Entladungen bohrten sich nur wenige dutzend Meter vor ihm in den Boden und entwurzelten ganze Bäume. Eine Fontäne aus Erdreich und Felsbrocken schoss in die Höhe und so gut wie die ganze Vegetation um die Einschlagsstelle herum fing Feuer.

\par

\WR{Verdammt!}, fluchte Morten. \WR{Rot zwei, wo bist du?}

\par

\WR{Ganz ruhig, Morty. Ich bin gleich bei dir.} Nicos Antwort klang sogar über Funk noch bedrohlich. \WR{Nur noch ein paar Sekunden.}

\par

Morten schoss über einen kleinen Hügel hinweg. Er überquerte die Spitze so niedrig, wie er nur konnte. In seinem Dasein als Pilot hatte er schon Landungen durchgeführt, in denen er weniger flach über den Untergrund hinweg geschossen war. Eine kleinere Bergspitze umflog er in einem möglichst engen Bogen und tauchte danach in einen sich anschließendes Tal ab. Doch die beiden Verfolger schienen deutlich wendiger als er. Mühelos glichen sie sich seiner Flugbahn an.

\par

Gerade, als er die nächste Salve erwartete, ertönte eine stattdessen eine neue Stimme aus seinen Kopfhörern. Sie gehörte eindeutig keinem der Piloten oder irgendeinem anderen Menschen. Stattdessen klang sie absichtlich verzerrten, wie durch einen Computer generiert.

\par

\WR{Es ist ein fröhlich Ding um aller Menschen Sterben: Die Engel freuen sich, die Seele heimzuführen; Der Teufel freut sich, im Fall sie ihm gebühren.}

\par

Dann trafen ihn die ersten Entladungen direkt achtern. Die Blocker versagten fast sofort und abgeschwächte Teile der Strahlen versengten sein Heck.

\par

\WR{Ich bin getroffen!}, brüllte er in sein Funkgerät. \WR{Muss aussteigen!}

\par

Bevor er jedoch den Auslöser betätigen konnte, der sein Cockpit vom Rest des Jägers absprengen würde, raste ein anderer Falke direkt unter seinem Bauch hindurch und zerlegte einen der beiden Shutek schneller, als Morten auf seine Heckansicht schauen konnte. Alles, was er noch sah, war, wie das Wrack des feindlichen Jägers in ein Waldstück hinein stürzte und dieses im Licht der Dämmerung für einen kurzen Augenblick taghell erleuchtete.

\par

So geistesgegenwärtig wie er sein konnte, ließ er seinen eigenen Flieger herumschwingen und hielt auf den verbleibenden Verfolger zu. Dieser machte sich gerade daran, sich in den Himmel hinauf zu schrauben, um Nicos Bordkanonen zu entgehen.

\par

Doch so sehr er sich auch hin und her wand, schaffte er es nicht, sich Mortens Raketenerfassung zu entziehen. Das kleine oval, dass auf seinem Zielmonokel über seinem Ziel angezeigt wurde, umschloss schließlich den feindlichen Jäger und er drückte ab.

\par

\WR{Das ist dafür, dass du Schiller falsch zitiert hast!}, rief Morten in sein Funkgerät, der in der feindlichen Übertragung ein deutsches Gedicht aus der Vorseuchenzeit erkannt haben wollte. Er beherrschte zwar weder diese Sprache, noch waren ihm irgendwelche Dichtungen aus jener Ära wirklich bekannt, doch er hatte zumindest die Namen einiger Dichter schon einmal in der Schule gehört.

\par

Die Rakete schlug ein und holte ihr Ziel zugleich vom Himmel. Im steilen Steigflug hatte der Shutek keine Gegenmaßnahmen zum Einsatz bringen können. Und verlor der feindliche Flieger sofort sämtliche Energie und stürzte wie ein Stein vom Himmel.

\par

\WR{Ich glaube eher, das war Göthe}, antwortete Nico Curiosa, der Morten über Funk gehört hatte. \WR{Guter Schuss, Junge.}
