Der alkoholfreie Sekt, den man auf der \EN{Regenvogel} bekam, schmeckte Morten und Kevin gleichermaßen gut. Außer ihnen saß niemand mehr in der Bar des Schiffes. Die meisten anderen Piloten hielten sich in ihren Quartieren auf und ruhten sich vor dem Einsatz aus. Ein paar halfen auch der Deckmannschaft, die Jäger flott zu machen. Morten hatte anfangs auch Hand anlegen wollen aber Kenji hatte ihm davon abgeraten. Der Chefmechaniker, ein gewisser Allan Tukarev, den aber alle nur als den Büffel bezeichneten, sollte neue Piloten anscheinend nicht sehr gerne haben, besonders wenn sie gerade von der Akademie kamen. Morten wusste nicht, ob er die Behauptung glauben sollte aber Kenji hatte nicht wie jemand gewirkt, der absichtlich übertrieb. Jedenfalls fand er den Spitznamen des Deckchefs ziemlich passend. Der muskulöse Oberkörper des von Kreuzpunkt Primus stammenden Mannes erinnerte viel mehr an einen prall gefüllten Koffer als an einen menschlichen Torso. Außerdem hatte er kurz geschorene Haare, deren Fläche durch tiefe Geheimratsecken stark eingeschränkt wurde.

\par

Morten hatte ihn nur einmal kurz gesehen, als er murmelnd durch die Gänge gelaufen war. Er hätte Morten fast über den Haufen gerannt, schien es aber nicht einmal bemerkt zu haben.

\par

Da er sich nicht weiter nützlich hatte machen können, hatte er sich wie Kevin auch noch ein wenig schlafen gelegt. Die Medizin, die ihnen Kenji empfohlen hatte, hatte wirklich Wunder gewirkt. Ohne die Pille, die, je nach Dosis für eine bestimmte Zeit an Tiefschlaf sorgte, hätte Morten niemals einschlafen könne. Der Saustall in seiner und Kevins Kabine war natürlich immer noch nicht beseitigt und sorgte dafür, dass sich keiner der Beiden besonders wohl fühlte. Dennoch war Morten klar, wenn er warten würde, bis ihm Kevin beim aufräumen half, würde es Wochen dauern, bis das Quartier wieder sauber wurde. Daher entschloss er sich, es nach dem Einsatz alleine zu tun.

\par

Die Pille hatte ihm auf jeden Fall ein paar Stunden Tiefschlaf beschert. Laut dem PhPhalanxkonzern, dass die Medizin entwickelt hatte, ersetzten drei Stunden Schlaf mithilfe der Pille eine ganze Nacht. Obwohl Morten sich doch recht ausgeruht fühlte, konnte er dem nicht ganz zustimmen. Aber er war froh ein bisschen geschlafen zu haben. Das half ihm, ein wenig zur Ruhe zu kommen und anzufangen, sich auf der \EN{Regenvogel} heimisch zu fühlen.

\par

\WR{Hast du den Wein bei der Selbstbedienungsbar bezahlt}, fragte Kevin nach einem langem Moment der Ruhe um das Schweige zu brechen.

\par

Morten nickte nur.

\par

Vor eine

\par

Vor einem wichtigen Einsatz gemeinsam etwas zusammen zu trinken war ein altes Ritual, das die beiden seit dem Beginn ihrer Akademiezeit kannten. Normalerweise wurde es von gut eingespielten Flügelmännern praktiziert, doch sowohl Kevin als auch Morten fehlten die entsprechenden Kontakte und das Dienstalter.

\par

Ein interessantes Paradoxon, wie Morten fand. Keiner der älteren Piloten war anwesend. Also wurde das gemeinsame Pseudobesäufnis vor einem Kampfeinsatz~-- das von den meisten Lehrern der Flugschule als wichtigste ungeschriebene Regel bezeichnet wurde~-- nach etlichen Jahren in im Alltag gar nicht mehr gewürdigt.

\par

Am Ende hatte doch jeder seine eigenen Bräuche. Gerüchten zufolge hörte sich der erste Offizier der \EN{Regenvogel}, Commander Samad die komplette Oper \Wr{Carmen} an. Ob dies stimmte, blieb unklar. Und Kevin und Mortens Kabine lag zwei Stockwerke unterhalb der Quartiere der Brückenoffiziere.

\par

Obwohl die Mannschaften und die Piloten gleichermaßen zur Starforce gehörten, waren die beiden Gruppen mehr als unterschiedlich. Während Piloten häufig bemüht waren, ihren Ruf als draufgängerische Abenteurer zu bestätigen, galt die Mannschaftsriege eher als regelbewusst, kontrolliert und diszipliniert. Somit wunderte es Morten, dass er am Ende die Karriere eines Piloten eingeschlagen hatte.

\par

Commander Samad hingegen war eine gute Mischung. Er war selbst geflogen, wie man Morten gesagt hatte. Aufgrund seiner langjährigen Erfahrung war es ihm leicht möglich gewesen, zur Mannschaftsriege zu wechseln.

\par

Morten hatte herausgefunden, wie man in der Bar an etwas zu trinken kam. Ein Terminal übertrug eine Getränkekarte an den eigenen Handcomputer und nach elektronischer Bezahlung öffnete sich ein geräumiger Wandschrank voll mit den verschiedensten Flaschen in allen Farben und Formen.

\par

Morten nippte an seinem alkoholfreien Sekt. Kevins Augen richteten sich nur auf seinen Handcomputer, ein schmuckloses, militärisches Modell, dass er auf der Theke aufgestellt hatte. Auf dem kleinen Display des Geräts liefen gerade die Nachrichten des Spectare State Vision mit minimaler Zeitverzögerung.

\par

Die Sprecherin ergriff ein neues Blatt von einem ordentlich aufgetürmten Stapel. Dass kein Computer zum Ablesen der Neuigkeiten verwendet wurde, sondern echtes Papier gehörte gewissermaßen zum guten Ton der Nachrichtenagenturen und war ein Anachronismus, den mittlerweile niemand mehr verstand.

\par

Mit neutraler, unbeteiligter Mine begann die Sprecherin zu berichten: \WR{Noch immer hält die Ankündigung des Präsidenten, erneut eine Abschaffung der Armee anzustreben, den Senat und die Foren der Planetensysteme in Atem. Das brisante Thema ist nahezu der einzige Gesprächsstoff in sämtlichen politischen Gesprächsrunden geworden.}

\par

Kevin schüttelte heftig den Kopf. \WR{Oh Mann}, begann er, \WR{wenn der alte Wirrkopf damit durchkommt werden wir vielleicht arbeitslos. Und dann geben sie uns irgendeine Arbeit.}

\par

Morten leerte sein Glas mit einem Schluck. Es kam ihm so vor, als ließe das Prickeln der Kohlensäure viel schneller nach als bei echtem Sekt. \WR{Du kannst immer auswandern}, gab er nur zurück.

\par

\WR{Wie immer eine große Hilfe}, entgegnete Kevin sofort. \WR{Weißt du, irgendwie bin ich froh, nicht an deinem Flügel fliegen zu müssen. Kameradschaft scheint nicht dein Ding zu sein.}

\par

Der Angesprochene vermied Kevins Blick. Stattdessen nahm er einen weiteren Schluck aus seinem Glas.

\par

\WR{Vergiss es einfach}, sagte Kevin und wollte aufstehen. Doch Morten antwortete schnell: \WR{Es tut mir leid, dass ich dir nicht gegen Major Hennington helfen konnte.} Er schluckte, ohne mehr von dem Sekt getrunken zu haben. \WR{Aber es gelten Regeln auf diesem Schiff. Geschriebene und ungeschriebene. Ob und das gefällt, spielt keine Rolle. Wir müssen uns einfügen. Nur so können wir funktionieren.}

\par

Unterdessen war die Nachrichtensprecherin fast am Ende ihrer Sendung angelangt. Sie hob ein letztes Blatt und verlas: \WR{Weitere Nachrichten. Gestern Mittag kam es auf der Erde, genauer in der Stadt Freiburg, zu einer Verfolgungsjagd zwischen Vertretern des Geheimdienstes und der Polizei und einem Mann, dem verschiedene Computerverbrechen zur Last gelegt werden. Augenzeugen zu Folge floh der Mann vom Tatort eines solchen Delikts und entkam in die Kanalisation der Stadt. Trotz stundenlanger Suche der Polizei blieb er verschwunden. Eine Frau wurde leicht verletzt, als sie vom Flüchtenden umgestoßen wurde.Weder die Polizei noch der Geheimdienst wollten sich zu dem Vorfall äußern aber es wurde angekündigt, dass demnächst ein detailliertes Phantombild veröffentlicht würde. Damit verabschieden wir uns und ich wünsche Ihnen im Namen des Spectare State Vision einen angenehmen Tag.}

\par

\WR{Wir funktionieren nicht. Wir leben hier!} In jedem Gespräch, das Morten mit Kevin bislang geführt hatte, war sein Stubengenosse emotional geworden. Es schien zu seiner Art zu gehören, ständig zumindest in milde Rage über irgendetwas zu geraten. Doch der Ton dieses Satzes verkündete mehr als Temperament. Kevin war tatsächlich wütend auf ihn.

\par

\WR{Es gehört nun mal zur Art vieler Vorgesetzten, die Neuen runterzuputzen}, verteidigte sich Morten. \WR{Du hättest Hennington nicht provizieren müssen. Was hast du dir dabei gedacht.}

\par

Die erwatete Entgegnung Kevins blieb zunächst aus. In dem Gesicht des jungen Mannes arbeitete es. Schließlich lehnte er sich zurück, um etwas Abstand zu Morten zu gewinnen. \WR{Vielleicht habe ich es nicht nötig, mich von einem Idioten wie diesem Dexter Hennington runtergeputzt zu werden.} Kevin sprach merkwürdig ruhig. \WR{Weißt du, ich komme nicht aus einer privilegierten Familie. Und erspare mir jetzt bitte diesen Mist mit: \Wr{In der Union hat jeder dieselbe Chance}.} Tatsächlich hatte Morten gerade etwas ähnliches anmerken wollen, obwohl er noch gar nicht wusste, worauf sein Stubengenosse hinaus wollte. \WR{Ich wurde nicht für die Oberstufe zugelassen. Meine Noten waren einfach zu schlecht. Dieser verdammte Deutschlehrer. Jedenfalls haben meine Eltern immer an mich geglaubt! Darum haben sie mich auf ein Internat in den autonomen Welten geschickt. Sie haben ihr Haus verkauft, damit ich dort meinen Abschluss machen konnte, denn diese Schulen sind nicht billig. Nur so konnte ich überhaupt zur Flugschule. Und dann soll ich vor einem Penner wie Hennington kuschen?}

\par

Kevin trank schnell sein Glas aus. Um eine Antwort Mortens zu vermeiden, stand er sofort auf und sah auf eine große, analoge Uhr, die über dem Eingangsschott zur Bar hing. Null achthundert, noch zwanzig Minuten bis zum Sprung.

\par

\WR{Jetzt wird’s ernst}, sprach Kevin in übertrieben epischen Ton, um das Thema zu wechseln und seine Bemerkung unkommentiert stehen zu lassen. \WR{Weißt du den Weg zum Hangar oder den Umkleidekabinen?}

\par

Morten drehte sich zu seinem Kameraden um. Sein Gesichtsausdruck ging ins Überraschte. \WR{Das ist ein Witz oder? Du kennst den Weg zur Bar schon am ersten Tag auswendig aber hast keine Ahnung wie man zum Hangar kommt?}

\par

\WR{Jep}, gab Kevin ungerührt zurück. \WR{Also wo geht es denn nun lang?}

\par

Morten wollte seinem Kumpel gerade etwas empörtes entgegen werfen, da kam er ins Stocken. \WR{Hm… ehrlich gesagt weiß ich das selbst nicht. Ich fürchte, wir sind in Schwierigkeiten. Wenn man den Sprung verschieben muss, weil wir nicht an Ort und Stelle sind…}

\par

\WR{Ach Quatsch, wir werden den Hangar schon finden, dass ist immerhin der größte Hohlraum von diesem Eimer hier. Der wird ja wohl kaum zu übersehen sein}, sagte Kevin mehr zu sich.

\par

Er und Morten schritten betreten in den Gang hinaus. Beide sahen sich um und blieben erst einmal stehen. Die vielen Korridore der \EN{Regenvogel} sahen fast alle gleich aus und nirgendwo gab es ein Hinweisschild oder etwas ähnliches. Lediglich einige Zahlen in Maschinensprache waren hier und da an den Wänden zu finden.

\par

\WR{Oh Scheiße}, fluchte Morten leise.

\par

Gerade dann entfaltete sich der wahrscheinlich größten Zufall des Tages. Kenji und Kringel kamen, sich laut miteinander unterhaltend, um die Ecke gebogen. Die beiden wussten mit Sicherheit, wo es zum Flugdeck ging.

\par

\WR{Hi}, grüßte Kringel beiläufig und beachtete Morten und Kevin nicht weiter.

\par

Kenji hingegen schien zu bemerken, dass die beiden orientierungslos herumstanden und fragte mit leicht hämischem Unterton: \WR{Was ist los? Keine Ahnung, wo’s lang geht, oder?}

\par

Morten wollte die Wahrheit bereits eingestehen, da schnitt ihm Kevin rasch das Wort ab: \WR{Nein, nein. Wir wissen natürlich, wo wir hin müssen. Wir haben uns nur gerade überlegt ob wir…}, er unterbrach sich \WR{ob wir noch unsere Kabine etwas aufräumen sollten.}

\par

Jens Wörg schüttelte sofort den Kopf. \WR{Das solltet ihr lieber lassen. Dafür ist keine Zeit mehr. Wir springen in einer starken Drittelstunde. Bis dahin müssen wir in unserer Fliegerkluft sein und angeschnallt in unseren Sicherheitsalkoven stehen.}

\par

\WR{Natürlich. Das habe ich ihm auch schon mehrmals gesagt.}

\par

Kevin warf ihm einen ungehaltenen Blick zu und beide folgten den zweien die schnellen Schrittes den Gang entlang liefen. Kurz darauf erreichte die Gruppe einen Lift. Als dieser sich in Bewegung setzte und eine angespannte Stille entstand entschloss sich Kringel, die Ruhe zu beenden. \WR{Aufgeregt?}, fragte er tonlos.

\par

Morten nickte sofort und antwortete wahrheitsgemäß: \WR{Ziemlich. Ich bin bis jetzt nur ein einziges mal mit einem Aufklärer der Argus-Klasse geflogen. Es ist eine ziemliche Herausforderung.}

\par

\WR{Ach was}, platzte es förmlich aus Kevin heraus. \WR{Man gibt uns als Neulinge sicherlich keine lebensgefährlichen Aufgaben. Außerdem haben wir mehrere hundert Trainingsstunden im Simulator hinter uns. Das war im zweiten Jahr. Da nimmt man Flüge im Aufklärer durchgenommen, weißt du nicht mehr?}

\par

Kenji lächelte ansatzweise. \WR{Ich kenne aber keinen Simulator, die Schwerelosigkeit und die starken Fliehkräfte richtig nachstellen kann, die in einem Aufklärer auf einem warten.}

\par

Kevin gab nur ein verächtliches Grummeln von sich aber Morten stimmte Kenji schweigend zu. Ein Aufklärer war nach der Sharp der zweitmodernste Jäger der Starforce. In keinem anderen Flieger war so viel moderne Technik auf so engem Raum zusammengepfercht wie bei einem Aufklärer der Argus-Klasse. Ein solcher Jäger verfügte über weitreichende Abtaster und Teleskope, war extrem schnell und außerdem in der Lage Hyperraumsprünge in andere Systeme durchzuführen. Dafür war bei der Konstruktion aber auch auf eine Menge verzichtet worden. Die Schilde eines Aufklärers waren nicht viel dicker als ein Blatt Papier, er hatte kein eigenes Schwerkraftfeld und außer zwei verkürzten Bordkanonen auch keine weiteren Waffen mehr.

\par

Aber es gab noch etwas anderes, dass Morten noch um einiges nervöser machte. Er würde mit Anna Farley fliegen. Immer wieder schossen ihm die selben Fragen durch den Kopf. War Sie streng oder eher lässig? Verlangte sie einem viel ab oder hatte sie mit Anfängern Nachsicht? Morten war sich sicher, wenn er nicht ein Auge auf seine Geschwaderkommandantin geworfen hätte, würden ihn diese Fragen nicht so sehr beunruhigen. Eigentlich zählte für ihn nur, was sie von ihm halten würde. Dies wiederum ließ ihn an seiner eigenen Professionalität zweifeln.

\par

Kurz darauf erreichte der Lift ein tieferes Deck. Morten und Kevin folgten den beiden, schon etwas erfahreneren Piloten durch die engen Korridore, die zumindest scheinbar völlig systemlos verliefen. Beide erinnerten sich an ihre Zeiten auf der Akademie. Dort hatten sie Wochen gebraucht um sich allmählich zurecht zu finden. Doch nun grinste Kevin triumphierend. Dank ihm würden sie den Hangar finden, ohne jemanden fragen haben zu müssen.

\par

Kenji und Jens gingen auf eine recht kleine Türe zu. Morten warf seinem Kameraden Kevin einen fragenden Blick zu, denn für ihn sah sie nicht nach der Tür zum Flugdeck aus. Außerdem spielte sich in dem Korridor, durch den sie gerade gingen, verdächtig wenig Aktivität ab.

\par

Kevins zufriedener Ausdruck verwandelte sich in einen der Überraschung als hinter der Tür einige Pissoirs und Waschbecken zum Vorschein kamen. Kringel und Kenji betraten den Toilettenraum, während Morten und sein Mitbewohner wie angewurzelt stehen blieben.

\par

\WR{Oh Misst}, hörte man Kevin leise sagen.

\par

Kenji sah die beiden ratlos an und sagte: \WR{Also entweder ihr kommt rein oder ihr bleibt draußen. Wieso schaut ihr so überrascht?}

\par

\WR{Aha}, begann Jens sofort in verhöhnendem Ton. \WR{Ihr wisst also doch nicht, wo’s langgeht oder? Sonst wärt ihr uns kaum zu den Toiletten gefolgt.}

\par

Kevin wollte gerade eine Ausrede vorbringen, als Morten ihm zuvorkam, um weitere Peinlichkeiten und Ausflüchte zu vermeiden. \WR{Stimmt. Wir wollten einfach nicht, dass wir wie Anfänger wirken, denn eigentlich sollten wir den Kübel doch kennen wie unsere Westentasche.}

\par

Kringel winkte sofort ab, während Kenji bereits zu den Toiletten schritt. \WR{Unsinn}, meinte er. \WR{Glaubt mir, dass dauert noch eine Weile. Aber ich hab eine gute Nachricht für euch. Ihr seid fast da. Geht nur noch in die Richtung und dann die schmale Treppe runter. Dann sehr ihr schon den Umkleideraum.}

\par

Jens winkte beständig in den Korridor abwärts, während er sprach. Morten bedankte sich und machte sich auf den Weg. Kevin sah ihn mürrisch an. \WR{Gut gemacht, wie du uns verraten hast}, brachte er hervor.

\par

Morten erhob beide Augenbrauen und beobachtete, wie Kevins Gang unbewusst stolzer, aufrechter und trotziger wurde. Kurz darauf erreichten die beiden tatsächlich eine Treppe, die sie ein Deck tiefer führte. Unten tummelten sich bereits einige Piloten in Einsatzkleidung. Auf den Oberarmpartieen ihrer Overalls waren die Embleme ihrer jeweiligen Geschwader angebracht. Einige trugen Fliegerhelme unter den Armen, die sie mit den unterschiedlichsten Farben und Bildern verziert hatten. So wie Morten Kevin einschätzte, würde sein Helm auch bald von schnittigen Bildern strotzen.

\par

Beide betraten die Umkleide, die durch eine Milchglastür vom Gang abgetrennt war. Morten erinnerte der Saal an den Umziehraum der Sporthalle seiner alten Schule, denn an den Wänden aufgereiht standen Bänke und Spinde und zwei weitere, undurchsichtige Glastüren führten zu zwei Gemeinschaftsduschen. Die Schotten standen offen und scheinbar war gerade niemand in den Waschräumen.

\par

Morten war überrascht als er sah, dass sich Männer und Frauen unbeeindruckt voreinander umzogen, obwohl abgetrennte Kabinen bereitstanden. Andererseits war ein solches Bild typisch für den Pragmatismus im Konglomerat und teilweise auch die Emanzipation in der gesamten Union. Noch vor zweihundert Jahren hatte sich wahrscheinlich kein Mensch eine ernst zu nehmende, gemischte Fußballmannschaft vorstellen können. Und nun standen zwei im Finale der Weltmeisterschaft.

\par

\WR{Hey, mach den Scheiß aus!}, beschwerte sich eine Pilotin lautstark, die Morten als Emilia Caruso kannte. \WR{Hier wird nicht geraucht.}

\par

Ein anderer Pilot, der sich gerade mit genussvollem Gesichtsausdruck eine Wasserzigarette angezündet hatte, warf ihr einen strafenden Blick zu und verstaute seine Tagesuniform in einem Spind.

\par

Kevin und Morten hielten nach ihren Schränken Ausschau. Die meisten Spinde waren mit Namensschildern bestückt und auf denen die keines hatten, waren Namen kunstvoll als Graffiti zu lesen.

\par

\WR{Hier drüben, ihr Frischfische}, rief ihnen ein breit grinsender Pilot zu und zeigte auf zwei ramponiert aussehende Schränke. Kevin zog eine Fratze und öffnete unter lautem Quietschen die Tür. Beide fanden in ihren Spinden Fliegeroveralls, wie die, die sie bereits auf die \EN{Regenvogel} mitgebracht hatten. Nur das diese mit den Emblemen ihrer Staffel bestickt waren.

\par

Morten fühlte sich anfangs etwas unwohl, als er dem Beispiel der anderen folgte und sich vor allen anderen umzog. Aber ihm wurde schnell klar, dass ihn niemand wirklich beachtete und außerdem musste er sich auch nicht vollends nackt präsentieren.

\par

Kevin kam kurz ins Stocken, als sich Emilia Caruso neben ihm umzuziehen begann. Obwohl sie Kevin unmöglich gesehen haben konnte, sagte sie in ernstem Ton: \WR{Kleiner, wenn du deine Stilaugen nicht wo anders hinlenkst gibt’s auf dem Schwarzmarkt bald ein paar sehr begehrte Organe mehr.}

\par

Normalerweise hätte Mortens Stubenkamerad sofort zurück geblafft aber auch er schien langsam etwas nervös zu werden. Morten warf ihm ein hämisches Grinsen zu und zog den Reißverschluss seines Overalls zu. Dann besah er sich den Fliegerhelm, den er ebenfalls in seinem Spind fand. Zu seiner positiven Überraschung war er entweder völlig neu oder erst wenige male benutzt worden. Der Helm, den Morten auf der Flugschule gestellt bekommen hatte, war mit den Schuppen und Haaren von mindestens zwanzig Piloten gefüllt gewesen.

\par

Auch das Zielmonokel, dass in einem schmalen Nebenfach lag, wirkte neu und unbenutzt. Morten warf Kevin einen kurzen Blick zu. Seine Ausrüstung wirkte längst nicht so neu. Der Helm hatte einige Kratzer und das Glas des Zielmonokels war beschmiert. Kevin begann es mit seinem Ärmel abzureiben.

\par

Kurz darauf betraten Kringel und Kenji die Umkleide und begannen ihre Pilotenkluft anzulegen. Jens kam zu Morten herüber geschlendert und betrachtete sein Arbeitsmaterial. Anerkennend erhob er beide Augenbrauen und winkte Kenji zu sich.

\par

\WR{Schau mal}, sprach er und zeigte dabei auf Mortens Helm. \WR{Der Kerl bekommt das neueste Zeug.}

\par

Kenji grinste, machte ein Gesicht als würde er irgendetwas suchen und musterte Morten theatralisch. Schließlich sagte er lächelnd: \WR{Du siehst eigentlich sauber aus. Hab schon gedacht du wärst einem der Oberen in den Allerwertesten gekrochen.}

\par

Gerade als er das gesagt hatte, kam ein stämmiger, blonder Pilot auf Morten zu und warf ebenfalls einen Blick auf seinen Helm. Das Gesicht des Mannes verriet, dass ihm irgendetwas durch den Kopf ging, das ihm zu schaffen machte.

\par

\WR{Tut mir leid für dich}, meinte der Pilot schließlich zu Morten. \WR{Du hast die Zweitgarnitur von Jamal Gryphus bekommen. Na ja, irgendjemand musste sein Zeug schließlich kriegen. Wird schon kein Fluch darauf liegen.}

\par

Kenji und Jens wirkten plötzlich recht betreten. Als der Mann den Umkleideraum verlassen hatte fragte Morten vorsichtig und mit einer unguten Vorahnung: \WR{War ist Jamal Gryphus?}

\par

Kringel räusperte sich und antwortete dann: \WR{Er war ein Anfänger, etwa in deinem Alter. Ist vor ein paar Monaten bei einem Routineflug ums Leben gekommen. Der erste Verlust hier an Bord. Vor ihm hat es noch jeder sicher auf den Boden zurück geschafft.}

\par

\WR{Meine Güte}, hauchte Morten. \WR{Was ist denn passiert?}

\par

Kenji erinnerte sich zurück. \WR{So viel ich weiß ist seine Steuerbordtreibstoffleitung explodiert. Sie haben Büffel mindestens drei mal verhört und wollten wissen ob vielleicht irgend ein Mechaniker Scheiße gebaut hatte. Aber sie haben nichts gefunden. Es war anscheinend einfach ein verdammt blöder Zufall.}

\par

\WR{Mach dir keine Gedanken}, gab Jens Wörg schließlich gebieterisch von sich. \WR{Sie dich einfach als Erbe.}

\par

Morten nickte schnell aber er fühlte sich kaum besser. Die Sachen eines Toten zu tragen war absolut das Letzte, dass er wollte. Aber wenn er sich bei einem der Vorgesetzten beschwerte würde er sich bloß unbeliebt machen und die Ausrüstung mit jemandem zu tauschen kam ihm wie ein makaberer Rückzieher vor.

\par

Kurze Zeit später erklang die Stimme des Koordinationsoffiziers aus den Lautsprechern: \WR{Achtung, an alle. Wir machen in zehn Minuten den Sprung durch die Lichtmauer. Bitte suchen sie die Ihnen zugewiesenen Schutzalkoven auf.}

\par

Kevin und Morten sahen sich nach dem nächsten Schutzraum um. Gegenüber der zwei Eingänge zu den Duschen gab es eine weitere Tür mit der Aufschrift \Wr{Sprungzone}. Diesmal wussten sie genau, was zu tun war. Wenn ein Schiff durch den tiefen Hyperraum flog musste sich die ganze Mannschaft sicherheitshalber entweder auf Stühlen festschnallen oder sich in kleinen Einlassungen in der Wand sichern. Für Piloten traf letzteres zu, denn sie mussten, genauso wie die technische Mannschaft, sofort nach dem Sprung los eilen um ihre Posten zu besetzen.

\par

Gerade als sie sich auf den Weg machen wollten, betraten Dexter Hennington, Anna Farley und Simon Maddeux die Umkleide. Die Fliegerklüfte der drei wirkten makellos und waren absolut korrekt angelegt.

\par

\WR{Hergehört, meine Lieben}, begann Simon Maddeux in ernstem Ton. \WR{Ich weiß, ihr habt das heute schon zig mal gehört aber ich will, dass ihr den Einsatz ernst nehmt. Nicht allein wegen dem Forschungsschiff, dem wir helfen sollen, sondern auch weil wir als Angehörige der Starforce eine gute Figur machen wollen. Die meisten von euch dürften es bereits mitgekriegt haben. Otis will unsere Jobs streichen. Zeigen wir ihm, was wir davon halten und tun da draußen unser Bestes.}

\par

Sofort applaudierte der ganze Raum. Auch Morten und Kevin stimmten mit ein.

\par

\WR{Auch für die, die heute nicht fliegen}, hängte Simon an. \WR{Wir sind ein Team und wir verhalten uns auch so. Jeder passt auf jeden auf. Los geht’s!}

\par

Schnurstracks schritten die drei durch den Raum und betraten die Schutzzone. Als Morten Anna Farley einsatzbereit in ihrem Overall sah, spürte er Erleichterung, darüber, sich nicht neben ihr umgezogen haben zu müssen. Ihre gute Figur lenkte seinen Blick gleich für mehrere Sekunden ab. Erst als sie ihm einen Blick zuwarf, sah er schnell in eine andere Richtung und versucht, nicht zu ertappt zu wirken.

\par

Kevin bemerkte, wem Morten nachsah. Grinsend schnippte er die Finger vor dem Gesicht seines Kameraden, der daraufhin aufschreckte. Beide reihten sich in die Warteschlange ein und betraten den Schutzraum.

\par

Für die vierzig Piloten, die in ihm Platz hatten, war er recht klein. Eigentlich bestand er nur aus einer Fensterfront, durch die man auf den Hangar hinabsehen konnte, einer Tür, die zu einem Treppengerüst hinab aufs Flugdeck führte und den Alkoven in der Wand.

\par

Die Piloten begannen sich in die Nischen zu stellen und sich fest anzugurten. Sensoren im Verschlussmechanismus und ein Temperaturfühler bestätigten dem Bordcomputer wenn sich jemand im Alkoven angegurtet hatte. Die anderen schienen genau zu wissen, welcher Alkoven ihrer war. Morten und Kevin standen da und sahen fragend in die Runde.

\par

\WR{Die beiden dort drüben}, erklärte ihnen Anna Farley knapp und deutete auf zwei unbesetzte Nischen.

\par

Morten setzte sich in Bewegung und stellte sich in den linken Alkoven, nachdem er sich die Nummer des Geräts eingeprägt hatte. Aus seiner Ausbildung wusste er genau, wie er sich anzugurten hatte. Trotzdem dauerte es eine Weile, bis er sich den Helm aufgezogen und die Riemen umständlich um die Brust und die Beine geschlungen hatte. Dann zog ein Mechanismus die Bänder automatisch fest.

\par

\WR{Ist das dein erster Sprung mit einem so großen Schiff?}, fragte der blonde Pilot, der Morten schon vorher angesprochen hatte.

\par

Aus Reflex wollte sich Morten vorbeugen um seinen Gesprächspartner zu sehen aber die Gurte waren so festgezurrt, das er sich keinen Millimeter bewegte. Schließlich antwortete er einfach gerade aus: \WR{Ja. Bisher war ich jedes mal auf Passagierschiffen mit überdimensionalen Schwingungsausgleich. Ich kann mir vorstellen, dass Sprünge hier an Bord sicher härter werden.}

\par

Kringel antwortete schnell, als wollte er den blonden Piloten daran hindern, eine allzu beängstigende Antwort zu geben: \WR{Eigentlich geht immer alles glatt. Rein theoretisch könnten wir uns diese Sicherungen sparen, denn bei einem genauen Sprung würde es uns so vorkommen, als sei gar nichts nichts gewesen. Das Problem ist nur, dass man den Weg oft nicht vollkommen korrekt berechnen kann. Besonders bei unerforschten Systemen wie dem, in das wir jetzt gleich fliegen. Dann kann es sein, dass es etwas holpriger wird.}

\par

\WR{Vielleicht hättest du einen Kotzeimer vor dich stellen sollen}, warf Kevin spöttisch ein.

\par

Morten seufzte.
