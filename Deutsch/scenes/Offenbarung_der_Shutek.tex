\WR{Mein Neffe ist auf Cygni stationiert}, begann Henry Otis zögerlichl. Er spürte überdeutlich, dass Grandadmiral Burns ihn am liebsten aus den Fenstern seines Präsidialbüros geworfen hätte. Dass er nun etwas von ihm wissen musste, gefiel ihm nicht. Doch es war ihm zu wichtig, also schluckte er seinen Stolz hinunter. \WR{Sind Sie sicher, dass die Evakuierung der Basis bereits abgeschlossen ist?}

\par

Der Admiral taxierte den Präsidenten interessiert. Normalerweise hielt sich dieser stets unter seinesgleichen auf. Aber Lertha Akintola und Richard Bellegarde sowie ein paar andere Politiker und Sekretäre, die Burns nicht kannte, hallten sich um einen Holoprojektor in der Mitte des weiträumigen Raums versammelt.

\par

Otis war zu Burns geschlichen, kurz nachdem die Echtzeitübertragung zur \EN{Rosalind Franklin} etabliert worden war. Seine Haltung, die sonst immer aus einem vorbildlichen durchgestreckten Rücken und unnahbarem Gesicht bestimmt wurde, war nun eher dem Anblick eines Bittstellers gewichen. Ein Anzeichen echter Sorge.

\par

\WR{Lieutenant Otis fliegt Begleitschutz für den Konvoi. Ich habe die Bestätigung, dass das Konglomerat die Oberfläche vollständig geräumt hat. Sie haben also freie Hand für ihren… Plan.} Mehr hatte Burns in diesem Moment nicht für seinen Gesprächspartner übrig. Statt sich weiter mit dem Präsidenten zu unterhalten, begab er sich zu den anderen.

\par

Nicht wenige Augenpaare waren in diesem Moment auf Marco Bellendi gerichtet. Dieser trug zwar mittlerweile einen Gehrock, doch seine schickere Kleidung wurde nun von einem verschwitzten Gesicht aufgewogen. Im Gegensatz zum Rest blickte er angestrengt in sein Buch, das nun alle Daten erhielt, welche von der \EN{Rosalind Franklin} gesendet wurden.

\par

Der holographische Projektor zeigte gleich mehrere Dinge. Neben der schematischen Darstellung des Cygni-Systems waren jedoch die grobkörnigen Aufnahmen der Außenbordkameras des Forschungsschiffs der hauptsächliche Blickfang. Die Flugobjekte, die sie zeigten, waren eindeutig shutek. Doch die Auflösung war zu gering, um viel erkennen zu können. Lediglich, dass ihre Hüllen grünlich wie metallisch schimmerten und an einigen Stellen von Rillen und Riffeln überzogen waren, so, dass sie fast wie ein Brustkorb wirkten.

\par

Tatsächlich wartete die \EN{Rosalind Franklin} in gebührendem Abstand und konnte somit nur Bilder liefern, die aus den Daten ihrer Echtzeitsensoren gerendert waren.

\par

\WR{Die Übertragung läuft}, bestätigte Bellendi, beziehungsweise sein holographisches Ebenbild. \WR{Und so wie es aussieht, sind sie auch auf Empfang. Sie bekommen die Daten, die wir ihnen schicken, aber sie reagieren überhaupt nicht.}

\par

Nach einem kurzen Blick auf seinen militärischen Handcomputer flüsterte einer seiner Berater Grandadmiral Thrawn zu: \WR{Die Flotille hält nach wie vor auf Cygni Primus zu. Sie erreichen eine optimale Entfernung zum Abwurf von Bomben in vier Minuten und drei Dutzend Sekunden.}

\par

Burns nahm selbst einen Blick auf die Anzeige und vergewisserte sich, dass sowohl die \EN{Rosalind Franklin} als auch der kleine Verband der Starforce in gebührender Entfernung Stellung hielten.

\par

\WR{Was genau sagen Sie ihnen, Herr Bellendi?}, erkundigte sich Otis.

\par

Marco antwortete, ohne den Blick von seinem Buch zu nehmen. \WR{Wenn wir vernünftig mit ihnen reden wollen, und ihnen sagen wollen, dass wir in Frieden kommen, dann müssen wir ihnen vielleicht erst mal die entsprechenden Begriffe erklären.} Nach einigem Blättern fand er eine Seite, auf der mehrere Spezies zu erkennen waren, die in symbiotischer Beziehung zueinander standen. \WR{Frieden ist ein philosophisches Konzept, dass in der sonstigen Natur nicht in der Form vorkommt, wie wir Menschen ihn uns vorstellen. Darum wollte ich mit Omega dutzendeins zunächst die pragmatischen Grundlagen für so etwas wie Frieden festlegen.}

\par

Grandadmiral Burns Aufmerksamkeit schien sich jedoch auf etwas anderes zu richten. Immer wieder huschte sein Blick von der schematischen Darstellung des Cygni-Systems zu den langsam klarer werdenden Bildern der Shutek hin und her. \WR{Ihre Schiffe haben keine Fenster. Das deckt sich mit den Abtastungen aus Pollux. Wenn wir bloß wüssten, was das für Tanks in ihrem Schiff sind.}

\par

Gleich von Anfang an war eine aktive Abtastung durch die \EN{Rosalind Franklin} von allen Anwesenden abgelehnt worden. Das Forschungsschiff sollte sich möglichst unauffällig verhalten und nichts tun, was die Shutek provozieren könnte. Das es dabei unentwegt Botschaften an die feindlichen Schiffe sendete, schien dabei nur Marco Bellendi zu stören.

\par

\WR{Du liebe Zeit, es funktioniert!}, brachte dieser mit einem mal hervor, als in seinem Buch tatsächlich eine eingehende Nachricht auf einer Kurzstreckenfrequenz zu verzeichnen war. \WR{Wir haben hier eine Antwort von einem Schiff der Shutek. Eine Audiodatei. Und zwar in der Standardkodierung der Krypta Scienctia.}

\par

\WR{Das bedeutet, sie verstehen unsere Systeme}, schlussfolgerte Burns. \WR{Das bestätigt unsere Befürchtungen bezüglich des Computervirus das die \EN{Regenvogel} und die Bodenstation auf Pollux befallen hat.} Nun fühlte sich der Großadmiral plötzlich recht unwohl. Er befand sich im selben Raum mit fast zwei Dutzend Zivilisten, die gerade alle wichtige militärische Details über die Shutek mitbekamen. Zwar hatte man ihnen im Vorfeld die nötige Sicherheitseinstufung zumindest zeitweise eingeräumt. Doch die Eile, die nötig gewesen war, hatte umfassende Hintergrundüberprüfungen fast unmöglich gemacht.

\par

\WR{Was sagen Sie, Herr Bellendi?}, fragte der Präsident Otis tonlos. \WR{Haben Sie verstanden, was wir ihnen sagen wollen?}

\par

Marco nahm seinen Blick nicht von dem Transkript der Nachricht, dass nun in seinem Buch geschrieben stand. Noch bevor er die gesamte Nachricht gelesen hatte, wurde ihm klar, dass es Mut brauchen würde, sie vor allen Anwesenden wiederzugeben, obwohl er selbst gar nicht zugegen war, sondern sich nach wie vor auf der \EN{Minerva} befand. \WR{Das hören Sie sich besser selbst an}, sagte er schließlich, verband sein Buch mit dem Tonsystem des Präsidialbüros und ließ die Botschaft wiedergeben. Noch vor dem ersten Wort herrschte Totenstille.

\par

\WR{To which the universe would hollowly respond, \Wr{My ways cannot be known, oh man.} Which is to say, \Wr{My ways do not make sense, nor do the ways of those who dwell in me.}}

\par

Sofort brach eine wilde Diskussion aus, weswegen Bellendi die Wiedergabe unterbach. \WR{Ist das Englisch?}, fragte einer der Berater des Präsidenten. Dieser nickte nur. Die Stimme des Sprechers hatte tonlos geklungen. Wie die eines veralteten Computerprogramms, dass zwar die Aussprache der Wörter korrekt umsetzen konnte, dem jedoch jedes Maß für Betonung fehlte.

\par

\WR{Ruhe!}, forderte Grandadmiral Burns. \WR{Herr Bellendi, das war nicht alles, oder?}

\par

Der Exobiologe ließ die Wiedergabe weiterlaufen, nachdem die Gespräche etwas abgeebbt waren. \WR{\Wr{Don't think of it as dying}, said Death. \Wr{Just think of it as leaving early to avoid the rush.}} Mit dem Ende des Satzes änderte sich auch die Stimme des Vorlesers. Die Tonqualiotät nahm ab und klang mehr nach einer alten, verrauschten Aufnahme. \WR{What sick ridiculous puppets we are. And what gross little stage we dance on. What fun we have dancing and fucking. Not a care in the world. Not knowing that we are nothing. We are not what was intended.}

\par

\WR{Was soll dieser Mist bedeuten?}, hörte Marco jemanden in selbst durch die Nullzonenverbindung murmeln.

\par

Der Abschluss der Nachricht wurde wieder von der computergenerierten Stimme gesprochen. \WR{If you want a picture of the future, imagine a boot stamping on a human face~-- forever.} Danach folgte nur noch ein lautes Piepen.

\par

Erneut kamen lautstarke Diskussionen auf. Einige der Anwesenden waren des Englischen mächtig und die Worte verfehlten ihre Wirkung nicht. Angst machte sich breit und selbst Grandadmiral Burns traten Schweißperlen auf die Stirn. \WR{Lieutenant, geben Sie den Befehl an die \EN{Rosalind Franklin} sich sofort zurückzuziehen. Die verbliebene Staffel soll ihren Abflug decken. Noch sollten sie weit genug weg sein, um nicht mehr von den Shutek eingeholt zu werden.}

\par

\WR{Moment}, warf Henry Otis ein. \WR{Wir haben gerade den ersten Kontakt mit diesen Fremden gehabt. Wir sollten versuchen, einen Dialog mit ihnen aufzubauen. Herr Bellendi, sind die Sutek noch auf Empfang?}

\par

Der gefragte sah angestrengt in sein Buch, während Norton Burns an den Präsidenten herantrat und ihm leise genug, um von niemand sonst gehört zu werden, sagte: \WR{Henry, glauben Sie mir bitte. Sie werden angreifen. Ich habe keine Ahnung, was sie mit ihrer Nachricht sagen wollten. Für mich ergibt es keinen Sinn, außer, dass es ziemlich bedrohlich klingt. Aber ich weiß, dass sie sich in einer Gefechtsformation befinden, und auf unsere Kolonie zusteuern.}

\par

\WR{Die Messgeräte der \EN{Rosalind Franklin} bestätigen, dass sie uns noch zuhören}, berichtete Bellendi durch sein Hologramm. \WR{Aber ich glaube nicht, dass sie wirklich mit uns reden wollen. In ihrer Nachricht ging es um den Tod und Gewalt. Ich fürchte, sie haben uns gerade gesagt, dass sie uns vernichten wollen.}

\par

An den Chef der Kommunikationsabteilung gewandt, befahl der Präsident: \WR{Bitte stellen Sie mich durch. Die Shutek verstehen eindeutig unsere Sprache, denn sie verwenden sie ja selbst. Ich möchte ihnen etwas mitteilen.}

\par

\WR{Kommunikationswege sind offen}, meldete der Verantwortliche kurz darauf, was sowohl ihm als auch dem Präsidenten entsetzte Blicke, nicht nur von Grandadmiral Burns einbrachte.

\par

Henry Otis schloss kurz die Augen und sagte dann: \WR{An die Unbekannten im Cygni-System. Ich spreche für die Administration der Unio Terrae. Wir verstehen Ihre Nachricht nicht aber wir müssen ihr verhalten als feindselig einstufen, wenn Sie Ihren Anflug auf den Planeten nicht sofort abbrechen.}

\par

Wie viele bereits vermuteten, kam von den Shutek keinerlei Antwort. Mehr und mehr Augenpaare richteten sich auf den Präsidenten. Richard Bellegarde schaffte es derweil aber kaum, seinen Freund und Kollegen anzusehen. Er kannte den Präsidenten schon seit langem und wusste, dass er sonst ein Fels in der Brandung war. Nun wirkte Otis geradezu hilflos. Bei Diskussionen im Senat konnte er so gut wie jeden Gegner in Grund und Boden reden. Nun hatte er es mit Feinden zu tun, die sich gar nicht erst auf ein Gespräch einließen.

\par

\WR{Ich befehle Ihnen, sich zurückzuziehen!}, schrie er, ohne zu wissen, ob seine Botschaft überhaupt noch gesendet wurde. Grandadmiral Burns berührte ihn an der Schulter und flüsterte. \WR{Herr Präsident, Sie können nichts tun. Sie haben Ihre Linie lange genug verteidigen. Jetzt überlassen Sie das mir. Ich bitte Sie!}

\par

\WR{Nein!}, antwortete der Präsident laut. \WR{Das kann ich nicht zulassen. Das Militär ist \textit{nicht} die Lösung. Es war nie die Lösung!}

\par

Einer der anwesenden Offiziere winkte Burns zu sich, ehe dieser etwas erwidern konnte. Er deutete auf die schematische Ansicht des Cygni-Systems. \WR{Die Flottille der Shutek trennt sich gerade. Zwei Schiffe und ein Dutzend Jäger halten jetzt auf die \EN{Rosalind Franklin} zu.}

\par

Die Präsenz der Navy und der Starforce konnte es schon allein zahlenmäßig nicht mit den Angreifern aufnehmen, so viel war eindeutig. Grandadmiral Burns sah sich das Hologramm nur eine kurze Weile lang an. Dann befahl er: \WR{Alle Einheiten sollen den Rückzug antreten. Die Jäger sollen sich ebenfalls aufteilen. Eine Staffel gibt der \EN{Rosalind Franklin} Feuerschutz. Die andere hilft den Großkampfschiffen der Navy beim Rückzug.}

\par

Sofort gab der Offizier die Anordnungen weiter. Der Raum explodierte erneut in ein Durcheinander aus affektierten bis panischen Kommentaren der anwesenden Senatoren und Regierungsbeamten.

\par

\WR{Das wird wieder genau dasselbe Desaster wie in Pollux}, gab ein besonders blasierter Abgeordneter von sich, von Anfang an wenig für Otis vollpazifistische Politik übrig gehabt hatte. \WR{Mit einem Sauhaufen wie dem Konglomerat ist nicht einmal gegen die autonomen Welten etwas zu holen.}

\par

Grandadmiral Burns, der gerade mit festem Blick auf das schematische Hologramm gerichtet, auf und ab ging, nahm sich kurz die Zeit um dem Mann ein verärgertes \WR{Halten Sie die Klappe, Sie aufgelasener Trottel!}, entgegen zu werfen.

\par

Die Antwort ignorierte er, denn etwas an den Bewegungen der eigenen Jäger irritierte ihn. Auch einer der Zerstörer der Navy brach seinen Flug zur Hyperraumroute ab und steuerte stattdessen in die Richtung der Angreifer.

\par

\WR{Was passiert da, Lieutenant?}, wollte Burns verärgert wissen. Noch ehe der Angesprochene antwortete, hing er an: \WR{Stellen Sie mich bitte zu diesen Einheiten durch.} Ein charakteristisches Piepen deutete an, dass die Verbindung hergestellt war. \WR{Captain, hier spricht Grandadmiral Burns. Oberkommandant der Offensivkräfte des Konglomerats. Ihr Befehl lautet, zur Basis zurück zu kehren!}

\par

Nun erschien das Bild einer Schalte auf die Brücke des abtrünnigen Zerstörers. Burns kannte den Kommandanten. Er war mit ihm auf die Akademie gegangen und war auch danach noch eine Zeit lang mit ihm in Kontakt geblieben. Nun saß der ältere Mann mit fast schon klischeehaftem weißen Vollbart persönlich an der Kommunikationszentrale seines Schiffes.

\par

\WR{Grandadmiral, wir haben uns schon in Pollux zurückgezogen. Mit wem auch immer wir es hier zu tun haben~-- es wird Zeit, zu zeigen, dass sich die Union wehren kann.}

\par

\WR{Ich werde es nicht noch einmal sagen}, drohte Burns. \WR{Ole, wenn Sie nicht sofort Ihre Befehle  befolgen, dann bringe ich sie persönlich auf die Anklagebank.}

\par

\WR{Das hier ist für Pollux Primus. Lassen Sie es sich eine Lehre sein, Sir. Wenn man sich von jedem Raudi herumschubsen lässt, landet man irgendwann für immer am Boden.} Seiner eigenen Brückenbesatzung rief der Kapitän zu: \WR{Hauptgeschütz laden und Marschflugkörper scharf machen!}, bevor die Verbindung abbrach.

\par

 Burns schlug mit seiner Faust auf den Tisch den Präsidenten, ohne darauf zu achten, dass es nicht sein eigener war. \WR{Dieser verdammte Sturkopf.} Dann wandte er sich an seinen eigenen Kommunikationsoffizier, der nach wie vor mit der hastig aufgebauten Ausrüstung kämpfte. \WR{Holen Sie mir irgendjemand auf diesem Zerstörer an die Leitung, der etwas Verstand hat. Er soll Captain Ole Heimdall notfalls mit Gewalt von der Brücke holen. Aber dieses Schiff \textit{muss abdrehen}!}

\par

Unterdessen war Henry Otis an den Oberkommandanten herangetreten. \WR{Grandadmiral}, begann er mit brüchiger Stimme. \WR{Mein Neffe. Ist er auf dem Weg zur Hyperraumroute?}

\par

So sehr er den Präsidenten mittlerweile verachtete. Dass dieser sich um ein Familienmitglied sorgte, konnte er verstehen. An eine Wachtoffizierin gewant, fragte er: \WR{Welcher dieser Jäger wird von Thomas Otis geflogen?}

\par

Die Offizierin betätigte einige Kontrollen und versah so die Darstellung einer der Flieger mit Klammern. \WR{Lieutenant Otis steuer Rot drei. Er hat sich Captain Heimdall angeschlossen, Sir.} Dabei sah sie den Präsidenten fast um Entschuldigung bittend an.

\par

Dieser unterband bereits einen Einwurf seitens Otis mit einer raschen Handbewegung und sagte dem Kommunikationsoffizier: \WR{Rufen Sie die ganze Staffel sofort zurück! Sagen Sie den Piloten, dass Sie nicht auf Heimdall hören dürfen.} Der Mann gab die Befehle wiederholt weiter. Doch keiner der Jäger schien Anstalten zu machen, umzukehren.

\par

Weder Otis, noch irgendjemand anderen, der das Geschehen verfolgt hatte, wunderte es, dass die einzige Entgegnung über Funk von Rot drei stammte. \WR{Grandadmiral, es tut mir sehr leid. Aber wir können diesen Befehl nicht ausführen. Wir müssen anfangen, die Union zu verteidigen. Otis, Ende und aus.}

\par

\WR{Verdammt, das sollen Sie nicht alleine!}, donnerte Burns, der mittlerweile dem Kommunikationsoffizier über der Schulter hing und ihn durch den wütenden Ausruf einem Tinitus näher gebracht hatte.

\par

Ein geflüstertes \WR{Nein}, blieb der einzige Kommentar des Präsidenten, als die Jäger sich in die Reihen der Feinde stürzten.

\par

Wie der Kampf verlief, war größtenteils nur über das taktische Hologramm zu erahnen. Einer der provisorisch aufgebauten Bildschirme zeigte jedoch Bilder einer Kamera, die direkt über den Kanonen einer der Jäger montiert war. Immer wieder zuckten gelbe Strahlen durch den sonst stockdunklen Raum. Kurz war einer der Gegner zu sehen. Breite Flügel zierten seinen Jäger und auch für Laien waren seine Waffen zu erahnen.

\par

Dann rauschten unverhofft rötliche Strahlen aus einer anderen Richtung heran und die Übertragung endete in statischem Rauschen und Schneegrieseln auf dem Bildschirm.

\par

\WR{Sir, die Staffel hat Rot eins und drei verloren}, meldete die Wachtoffizierin fast kleinlaut.

\par

Grandadmiral Burns mied den Blick zum Präsidenten und hielt sich stattdessen gequält die Stirn. \WR{Wiederholen Sie den Befehl zur Rückkehr!}, bellte er schließlich dem Kommunikationsoffizier entgegen.

\par

\WR{Diese Jäger haben absprengbare Pilotenkanzeln}, stammelte Otis. \WR{Er muss es rausgeschafft haben. Es kann nicht anders sein!}

\par

Burns sah hoffnungsvoll zur Wachtoffizierin. Doch diese schüttelte nur den Kopf. Die meisten Abgeordneten und Regierungsbeamten waren still geworden. Richard Bellegarde trat an den Präsidenten heran, der fast katatonisch schien und nur noch hin und her wippte.

\par

Henry Otis weinte nicht. Stattdessen wirkte sein Gesicht wie versteinert. Als hätte ihn etwas voll und ganz von der Wirklichkeit abgekoppelt. Schließlich nahm er hinter seinem Schreibtisch Platz und starrte in den Raum. Bellegarde redete weiter auf ihn ein, doch er hörte ihn nicht. Die einzigen Worte, die ihn erreichten, waren die seines Neffen.

\par

Die letzte Bedeutungsvolle Unterhaltung mit Thomas. Damals war es um seine Entscheidung gegangen, der Starforce beizutreten. Warum hatte er das getan? Henry Otis war sich nicht sicher. Er erinnerte sich nur an seine eigenen Argumente. Das jede Armee ein Haufen von Barbaren sein musste. Dass Gewalt ausschließlich Gegengewalt säte. Dass Waffen niemals etwas gutes bewirkten und Relikt der unzivilisierten Vergangenheit darstellten.

\par

Aber warum war Thomas der Starforce denn beigetreten. Otis wusste es nicht, denn es hatte ihn nicht interessiert. Der Verrat war alles, was ihn damals gekümmert hatte.

\par

Nun sah er in den Raum, in dem sein schlimmster Feind gerade von einem anderen abgelöst wurde. In dem Grandadmiral Burns und seine Untergebenen genauso verzweifelt und panisch umherliefen, wie er sich fühlte und genauso wütend schrien, wie er es gerne getan hatte.

\par

Und nicht erst, als die taktische Anzeige Ole Heimdalls Zerstörer verschwinden zeigte, wurde ihm klar, dass es nun die Shutek waren, die kein Interesse zeigten. Nicht für Thomas Otis Leben oder das irgendeines anderen.