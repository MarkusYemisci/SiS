\WR{Wie gerade bekannt wurde, sind erste Versuche, mit den außerirdischen Angreifern im Cygni-System, Kontakt aufzunehmen, gescheitert.} Laura war zu müde, um ihre übliche Morgenzeitung zu lesen. Ihr Buch zeigte nun stattdessen eine Aufnahme des Senders Spectare. Warum sie sich diesen staatlichen Sender so gerne ansah, wusste sie selbst nicht. War sie jedoch ehrlich zu sich selbst, gefiel ihr die Nachrichtensprecherin sehr. Dass diese unter ihren Kollegen und Kolleginnen die meiste Sendezeit erhielt, zeigte ihr, dass sie damit nicht alleine war.

\par

\WR{Präsident Otis, dessen Neffe bei dem darauf folgenden Angriff ums Leben kam, war noch nicht zu einer Stellungnahme bereit. Aus dem Präsidialsitz in Wellington gab es ebenfalls noch keine Verlautbarungen.}

\par

Laura sah von ihrem Buch auf. Doch sie nahm die verschneite Landschaft um Oslo kaum wahr. Der Baum, unter dem sie sich niedergelassen hatte, trug so viel Schnee, dass sich seine Äste bereits zu biegen begannen. Doch das kalte Weiß hatte sie nur interessiert, als sie es von der Bank gewischt hatte.

\par

Das der Präsident nicht vorschnell Kommentare abgab, war allgemein bekannt. Doch in einem solchen Fall gab es stattdessen mindestens ein paar beschwichtigende Sätze aus seinem Büro.

\par

\WR{Wie von neutralen Beobachtern an Bord der Rosland Franklin~-- dem Forschungschiff, das die Operation leitete~-- mitgeteilt wurde, hat eines der Navy-Schiffe versucht, die Shutek auf eigene Faust anzugreifen und wurde dabei laut Augenzeugenberichten vernichtet.}

\par

Bislang war Laura zu sehr mit ihrem Fall befasst gewesen. Er hatte sie nicht nur von Johanna Bell abelenkt. Doch nun fing auch sie an, sich Sorgen zu machen. Die Disziplin im Konglomerat war immer schon ein Kritikpunkt gewesen. Durch die ständigen Ermutigungen, auch an den niedrigrangigsten Soldaten, stets mitzudenken, kam es immer wieder zu mehr oder weniger spektakulären Befehlsverweigerungen. Doch niemals in einem solchen Ausmaß.

\par

\WR{Gradnadmiral Norton Burns~-- Chef der offensiven Bestandteile des Konglomerats~-- hat sich nach dem Vorfall mit einer kurzen Ansprache vorwiegend an Militärangehörige gewandt.} Das Bild wechselte zu dem weißhaarigen Mann mit der Igelfrisur, der noch missmutiger wirkte, als er es auf irgendeiner anderen Aufnahme tat. \WR{Lassen Sie sich eines gesagt sein}, begann er und zu seinem scharfen Ton fehlte lediglich der erhobene Zeigefinger, \WR{Sie sind Soldaten der Union. Was sie tun oder nicht tun entscheidet eventuell über den Fortbestand unserer Art. Was auch immer Sie persönlich glauben, haben Sie Vertrauen in Ihre Vorgesetzten und befolgen Sie Ihre Befehle!}

\par

Laura klappte ihr Buch zu. Sie war zu ihrer einstigen Lieblingsstelle in ganze Oslo gekommen, um dort nachzudenken. Aber sofort als sie den Ort erreicht hatte, von dem aus sie die Buch überblicken konnte, hatte sie angefangen, sich mit Nachrichten abzulenken.

\par

Ablenkung schien das Allheilmittel zu sein. Ihr Therapeut hatte es ihr empfohlen. Genauso wie Klaus und sämtliche ihrer Freunde. Schach spielen, Wandern, Zeitung lesen, Traumfilme ansehen. Egal was. Hauptsache, es schlug die Zeit bis zum Schlafengehen tot.

\par

Zuletzt hatte der Hacker ihr erlaubt, ihre Gedanken in produktive Bahnen zu lenken. Aber nun war ihr der Fall entzogen worden und sie hatte nicht mehr, worin sie sich flüchten konnte. Schlimmer noch war, dass sie genau wusste, was sie tun musste. Aber fühlte sich nicht dazu imstande. Nicht jetzt. Also gab es nur eine andere Option.

\par

Sie saß schon seit einer halben Stunde auf derselben Bank. Tatsächlich hatte der Schnee bereits ihre Füße bedeckt, so still war sie gesessen. Nun klopfte sie sich ihren Mantel ab und schlug erneut ihr Buch auf. Es trug noch immer den Geruch neuer Elektronik.

\par

Nachdem sie Han Liaos Namen hineingeschrieben hatte, schrieb die Tinte von selbst einige Wörter dahinter. Ungeduldig malte sie einen Kringel um \Wr{Anrufen}. Wenig später erklang Hans Stimme. \WR{Was gibt's? Ich bin gerade mitten in einem Kartenspiel.}

\par

Laura lächelte unwillkürlich. \WR{Hör mal, ich habe über den Fall nachgedacht. Können wir uns vielleicht mal darüber unterhalten?}

\par

Eine Weile lang blieb Han still. Laura konnte sich auch denken, weshalb. Mit Sicherheit hatte ihn O'Shea bereits darüber informiert, dass sie mit der Ergreifung des Hackers in Zukunft nichts mehr zu tun haben sollte. Und damit war alles, was Han herausfand oder heruasgefunden hatte, für sie Verschlusssache. Dennoch wunderte es sie nicht, als er kurz angebunden antwortete: \WR{Komm einfach mal runter nach Nizza. Ich habe hier vielleicht tatsächlich etwas, dass du dir ansehen solltest.}

\par

\WR{Bis bald}, verabschiedete sich Laura und wischte den Schnee zur Seite, der sich allein während des kurzen Gesprächs auf ihrem Buch angesammelt hatte. \WR{Ich hasse die Kälte.}
\ortswechsel
Etwa eine Dreiviertelstunde später stand Laura Gethas erneut vor Han Liaos Arbeitsbereich. Dieser war im Prinzip ein Glaskasten mitten in einem Innenhof. Das Haus, das ihn nach Norden begrenzte, wirkte alt und schmuddelig. So, als würde der Putz schon bald von seinen grauen Mauern fallen. Der Übergang zum schicken Hochhaus im Norden hingegen wirkte schon deutlich einladender.

\par

\WR{Ich habe mich immer gewundert, wie jemand mit einer so ausgeprägten Sozialphobie wie du, in diesem Aquarium arbeiten kann}, sinnierte Laura.

\par

Han drehte sich nicht einmal zu ihr um, sondern blieb über etwas gebeugt sitzen, dass sie hinter seinem massigen Körper nicht erkennen konnte. \WR{Wo ist Klaus?}

\par

\WR{Der hat Feierabend}, antwortete Laura wahrheitsgemäß aber nicht vollständig.

\par

Nun wandte Han sich zu ihr und zeigte ihr ein Blatt Papier, auf dem ein unscharfes Foto abgedruckt war.

\par

Laura nahm es an sich und musste nicht lange suchen. \WR{Der Hacker. Hast du das aus seinem Buch gerettet?}

\par

\WR{Er ist gut bei allem, was er tut}, antwortete Han. \WR{Aber selbst die besten machen Fehler. Sein Selbstzertörungsprogramm hat alle Daten aus den Speichern gelöscht und die intelligente Tinte vollständig auslaufen lassen. Aber er hat dieses Bild auf einer Seite anzeigen lassen, die er selbst wohl selten verwendet hat. Dadurch wurde es durch das Auslaufen der Tinte nicht genug verwischt. Hätte er es sich auf derselben Seite anzeigen lassen, auf denen er sich auch Filme oder zumindest andere Fotos angesehen hat, dann wäre das Papier viel zu abgenutzt gewesen, als dass man etwas hätte rekonstruieren können.}

\par

Laura hielt die Abbildung ins Licht. Sie schwitzte. Noch immer trug sie ihre Dienstkleidung, die für ein verschneites Norwegen und nicht ein Nizza ausgelegt war, dessen Klima ozeanisch geprägt war.

\par

\WR{Eine bessere Auflösung ist wohl nicht drin, oder?}

\par

\WR{Glaubst du, ich würde dir sonst diesen Tintenkleks geben?}, erwiderte Han tonlos.

\par

\WR{Gesichtserkennung?}

\par

\WR{Keine vernünftigen Ergebnisse. Das Foto ist zu unscharf. Es gibt für jede dieser Personen mindestens ein dion Treffer.}

\par

Im linken Bereich des Bildes war der Hacker zu sehen. Er stand der Kamera mit am nächsten. Im Hintergrund waren weitere Personen zu erkennen, deren Gesichtszüge aber teilweise kaum auszumachen waren. Ein Mann zur rechten des Hackers machte jedoch eine Ausnahme. Es war aber nicht nur seine Hakennase und die tief sitzenden Augen, die Laura einen Namen ins Gedächtnis riefen. Seine ganze Haltung, die leichte Neigung zur Seite. Alles passte wunderbar zu Sergei DeGaulle.

\par

\WR{Erkennst du jemanden?}, fragte Han, in dessen Stimme so etwas wie echtes Interesse mitschwang.

\par

\WR{Was hat unser Mann mit dem ehemaligen Chef der Argusabteilung zu tun?}, fragte Laura eher an sich selbst gewandt. An Han hängte sie an: \WR{Gibt es das Bild in der Krypta?}

\par

Nun wissend, dass nicht nur der Hacker, sondern auch Sergei DeGaulle auf dem Foto zu erkennen waren, führte Han eine neuerliche Suche durch. Allerdings mit dem gleichen Resultat wie sein erster Vergleich zwischen dem Foto und einer Unzahl an Abbildungen in der Krypta. \WR{Nein. Zumindest nicht für meine Sicherheitseinstufung. Das bedeutet, es ist entweder niemals hochgeladen worden, oder zu geheim für einen Stufe Phi Zugang. Und der ist ziemlich hoch.}

\par

Tatsächlich beneidete Laura Han um seine hohe Freigabe. \WR{Was wissen wir über DeGaulle?}

\par

\WR{Das fragst du mich?}, entgegnete Han verwirrt. \WR{Du bist länger bei der Abteilung. Solltest du deinen obersten Chef nicht kennen?}

\par

Laura wollte es lieber vermeiden, mit ihrem eigenen Buch auf die Krypta Scientia zuzugreifen. Darum setzte sie sich ungefragt neben Han, was ihr einen stoisch genervten Blick einbrachte, und trug die entsprechenden Suchbefehle ein. Hans Sicherheitseinstufung erlaubte die Ansicht fast seiner ganzen Dienstakte.

\par

\WR{Geboren zweitausend sechshundert achtzehn~-- Erdstandard. Mit dreidutzend zwei zum Geheimdienst gegangen. Noch mal zwei Jahrdutzende später zum Leiter der Argusabteilung gewählt worden. Viele Auszeichnungen…}, Laura las so schnell sie konnte. \WR{Hat einen gehörigen Beitrag zur Zerschlagung der Orestes-Mafia geleistet. Später ist er dann aus gesundheitlichen Gründen aus dem aktiven Dienst ausgeschieden.}

\par

Hätte Han eine regere Mimik gehabt, hätte er nun genauso wie Laura seine Augenbrauen weit hochgezogen. \WR{Wegen Krebs? Wer bekommt denn heute noch Krebs?}

\par

\WR{Nicht jede Krebsart wird durch unser ergänzendes Immunsystem erkannt und eliminiert}, antwortete Han. \WR{Hin und wieder schaffen es die kleinen Biester doch, durchzubrechen. Nichts, womit die Ärzte nicht fertig würden. Aber die Voruntersuchungen sind ziemlich aufwendig.}

\par

Laura stockte beim studieren von DeGaulles medizinischer Akte. \WR{Hier steht, dass er einen Hirntumor hatte, der nur schwer zu entfernen war. Der Mann, mit dem ich mich in Oslo getroffen habe, hat mich gefragt, ob ich kürzlich wegen einer Erkrankung des Gehirns behandelt worden wäre.}

\par

\WR{Und du glaubst, es gibt einen Zusammenhang zu Sergei DeGaulle?}, fragte Han und klang dabei alles andere als überzeugt.

\par

Laura strich sich über ihr Kinn. \WR{Vermutlich nicht. Dazu müsste ich wie ein Verschwörungstheoretiker denken. Aber für unseren Hacker und seinen Freund scheinen Gehirnoperationen dieser Art eine Rolle zu spielen.}

\par

\WR{Hat dieser Mann gesagt, wieso?}, wollte Han wissen, doch erhielt nur ein Kopfschütteln als Antwort.

\par

Schließlich bat Laura: \WR{Kannst du bitte alle Dienstakten heraussuchen, denen sich der Hacker bemächtigt hat?}

\par

Nur wenige Augenblicke später erschien eine Übersicht auf einem von Hans zahlreichen Monitoren. \WR{Das sind nur die, von denen wir wissen. Ich nehme an, du möchtest jetzt wissen, welche dieser Personen ebenfalls eine Gehirnoperation hinter sich gebracht haben?}

\par

\WR{Nicht nur die Operationen selbst}, erläuterte Laura. \WR{Auch alles, was dazu sonst noch dazu gehört. Und wenn sie nur mit einem Arzt gesprochen haben oder einen Tomographie haben lassen.}

\par

\WR{Oder wenn sie sich den Schädel haben vermessen lassen, damit ein Künstler eine schicke Büste modellieren kann}, hängte Han an, während er die Suchabfrage formulierte.

\par

\WR{Du hast ja Sinn für Humor}, bemerkte Laura tonlos. \WR{Wo hast du dir denn den runtergeladen?}

\par

Die Algorithmen des Gnosis arbeiteten gewohnt schnell. Auf dem Monitor wurden die bisherigen Ergebnisse dargestellt und tatsächlich waren einige Akten hervorgehoben. Viele der Namen kamen Laura sogar bekannt vor und es dauerte nicht lange, bis sich ihr ein Muster auftat. Eines, dass sie bereits einmal gesehen hatte.

\par

\WR{Die Sektion}, sagte sie bloß.

\par

Han schien diesmal nicht überrascht. Es wäre auch verwunderlich gewesen, wenn er von dieser speziellen Verschwörungstheorie, die hauptsächlich innerhalb des Geheimdienstes kursierte, noch nie etwas gehört hätte.

\par

\WR{Ich weiß nicht}, begann er. \WR{Dass diese Namen teilweise mit denen übereinstimmen, die man einem Geheimbund zuschreibt, klingt nach reinem Zufall. Und zwar nicht mal einem besonders eindrucksvollen.}

\par

Laura seufzte. \WR{Du hast wahrscheinlich recht. Ich meine, wir alle kennen diese Geschichte. Eine Untergrundorganisation, die wichtige Stellen unterwandert, um ohne rechtliche Legitimation den Willen des Staates durchsetzten soll. Es gab vermutlich nicht eine einzige Nation auf der Welt, die nicht angeblich eine ähnliche Vereinigung gehabt haben sollte.} Nach einer Weile der Ruhe hängte sie an: \WR{Allerdings spielt es auch keine Rolle, was ich oder du glauben, oder? Wenn unser Hacker immer noch auf Dienstakten aus ist, oder irgend etwas anderes, dass seine Hypothese belegen könnte, dann können wir ihm vielleicht eine Falle stellen.}

\par

Han vermied den Augenkontakt, als er fragte: \WR{So wie in Freiburg?}

\par

Laura ignorierte die Stichelei, auch wenn sie sich fragte, wieso Han sie gerade nun provozieren wollte. Er sprach sonst nicht viel und ließ sich schon gar nicht auf verbale Machtspiele ein.

\par

\WR{Ich weiß, dass er mich im Auge hat. Das bedeutet, er hat vermutlich auch mein berufliches Umfeld im Blick}, mutmaßte sie. \WR{Han, kannst du die Dienstakten von ein paar Vorgesetzten manipulieren? Am besten O'Shea.}

\par

\WR{Ich soll was?}, platzte es aus dem Computerfachmann heraus.

\par

\WR{Der Hacker wird vermuten, dass es eine undichte Stelle in unseren Reihen gibt. Jemanden, der den Attentäter von Oslo mit Informationen über das Treffen versorgt hat. Wenn der Hacker tatsächlich an seine Verschwörungstheorie von der Sektion glaubt, wird er diese Ratte entweder in mir oder meinen Vorgesetzten vermuten.}

\par

Han hielt beide Hände in Abwehrhaltung erhoben. Eine solch lebhafte Gestik kannte Laura kaum von ihm. \WR{Weißt du, was du da von mir verlangst? Eigentlich dürften wir nicht mal über dieses Thema sprechen, ist dir das klar?}

\par

Laura drehte sich mitsamt ihres Sessels in Hans Richtung. Was er sagte, stimmt vollkommen. Er riskierte in diesem Moment bereits seine Anstellung, auch wenn sie sich sicher gewesen war, dass ihn das nicht störte.

\par

\WR{Ich verstehe, was du meinst}, sagte sie schließlich. \WR{Und du hast Recht. Falls etwas schief gehen sollte, wird man auf dich kommen und dann wirst du entlassen. Falls irgendwer herausfindet, woher ich von Sergei DeGaulle weiß, wirst du ebenfalls entlassen.} Han wollte etwas sagen, doch Laura fuhr fort. \WR{Ich weiß, was du von Bauchgefühlen hältst. Und ich bin in diesen Belangen normalerweise ganz bei dir. Aber im Moment sagt mir meine Intuition, dass mehr hinter den Handlungen des Hackers steckt. Ich sage nicht, dass ich an Verschwörungstheorien über Geheimbünde glaube. Aber dieser Mann macht nicht zum Spaß, was er tut. Und sein Partner ist schon zu Tode gekommen. Etwas passiert hier und ich muss einfach wissen, was es ist. Eine bessere Begründung für meine Bitte habe ich nicht.}

\par

Han legte seine Hände auf die altmodisch wirkende Tastatur und begann zu tippen. \WR{Es wird nicht leicht werden, die Akten von O'Shea zu fingieren. Nicht nur unser Hacker darf davon nichts mitbekommen. Wenn jemand aus der Abteilung mitkriegt, was wir hier treiben, sind wir erledigt. Ich melde mich bei dir, wenn ich fertig bin.}

\par

Überraschenderweise musste sich Laura genau jetzt eine Träne verkneifen. Vielleicht, weil sie nicht damit gerechnet hatte, dass gerade jemand wie Han an sie glauben würde. \WR{Danke, ich …}

\par

\WR{Nicht jetzt}, würgte er sie sofort ab. \WR{Mich arbeiten lassen, später danke sagen.}