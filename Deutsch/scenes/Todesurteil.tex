Dilara Bashir sah ungläubig auf ihren Dosimeter. Die \EN{Regenvogel} war getroffen worden, so viel stand fest. Hätten die Schutzfelder nicht gehalten, dann befände sich bereits ein breites Loch in der Hülle, gerade an der Stelle, an der Funken sich nun gerade aufhielt.

\par

Dann wäre sie jetzt bereits tot. So blieb ihr vielleicht noch eine halbe Stunde. Es musste ein leichter Torpedo mit nuklearem Gefechtskopf gewesen sein, der gerade ganz in der Nähe der Außenhülle explodiert war. Zu nah und zu heftig war die Detonation gewesen, als das Schutzfelder oder die Hülle die enorme Strahlung hätte aufhalten können, die dabei frei geworden war.

\par

Es hatte auch nicht gerade geholfen, dass sich Chefingenieurin Bashir gerade im Steuerbordtriebwerk der \EN{Regenvogel} aufgehalten hatte. Nur ein halber Meter Panzerplatten trennten sie vom Weltall.

\par

Funkens Handcomputer erwachte wieder zum Leben. Die Brücke rief sie. \WR{Bashir hier}, sagte sie und war überrascht davon, wie normal ihre Stimme klang. So sprach also jemand, der gerade mal eine halbe Stunde zu leben hatte. Ihr erweitertes Immunsystem würde versuchen, die Auswirkungen der tödlichen Dosis einzudämmen. Aber bei der Menge an Gammastrahlung, die sie abbekommen hatte, würde es damit nicht mehr lange erfolgreich sein.

\par

Captain Fiscales Stimme erklang von der anderen Seite. \WR{Lieutenant, ist alles in Ordnung bei Ihnen, wir haben gerade einen schweren Treffer hingenommen.}

\par

\WR{Ist mir gar nicht aufgefallen.} Bashir klang sarkastisch. \WR{Ich bin gerade nicht im Maschinenraum. Die Steuerbordtreibwerke haben meine heilenden Hände gebraucht. Ich gehe jetzt wieder zurück und gebe dann einen Bericht. Aber ich denke, hier unten ist fast alles noch unversehrt.}

\par

\WR{Das will ich hoffen}, mahnte die Kapitänin. \WR{Wir sind alles, was noch zwischen den Shutek und Kreuzpunkt Primus steht.}

\par

\WR{Sagen Sie mir was neues!}, forderte Bashir rhetorisch und schaltete sofort den Kommunikator aus. Sie hustete und hielt sich reflexartig die Hand vor den Mund. Sie musste nicht erst das Blut auf ihr sehen, um zu wissen, wie es um sie stand. Der Geschmack nach Eisen reichte ihr völlig.

\par

\WR{Alte Schlampe}, hängte sie für sich selbst an und fragte sich, wieso sie ihre Abneigung für die Kommandantin eigentlich noch verstecken wollte.

\par

Auf eine verquere Weise, war sie dankbar dafür, dass es so schnell gehen würde. Sie würde kaum Zeit haben, zu realisieren, was mit ihr geschah, bevor es zu Ende wäre. Und ihre letzten Augenblicke konnte sie noch damit verbringen, ihren Kameraden und Freunden zu helfen. Darum machte sie sich auf in Richtung ihres Maschinenraumes. Zumindest würde sie dort unter Freunden sterben.
