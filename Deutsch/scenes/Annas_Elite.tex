\WR{Als ich von den \Wr{ungeschriebenen Gesetzen} geredet habe}, begann Anna Farley, während sie vor Morten und Kevin hin und her tigerte, \WR{da meinte ich nicht: \Wr{Schlagt einen Vorgesetzten.} Was ist in Sie gefahren, Lieutenant Witwer.}

\par

Morten war aufgestanden, als Farley vor das Plexiglas ihrer Zelle getreten war. Kevin hingegen war sitzen geblieben. Ohne zu einem allzu positiven Ergebnis zu gelangen, bewertete er jede mögliche Antwort, die er seiner Vorgesetzten nun geben konnte.

\par

\WR{Es tut mir leid}, sagte er schließlich.

\par

Annas Blick durchbohrte ihn förmlich. In ihrem Gesicht arbeitete es und es war leicht zu erkennen, dass sie darüber grübelte, ob sie ihr Gegenüber nun anschreien wollte, oder nicht. Letzten Endes drehte sie sich einfach um und ging weiterhin auf und ab.

\par

\WR{Sie beide sind schon was}, sagte sie und klang dabei sogar verärgerter als zuvor. \WR{Morten, Sie meine ich ganz besonders. Zuerst kriechen Sie mir in den Arsch und buckeln vor jeder Regel und jetzt werfen Sie Ihre Karriere scheinbar völlig gedankenlos aus dem Fenster. Ich versteh's einfach nicht.}

\par

Morten wurde rot und schluckte. Dabei war er über sich selbst überrascht. Vor allem darüber, dass es ihn mehr störte, dass Anna seine Versuche des Kennenlernens offensichtlich als Anbiedern missverstanden hatte, als, dass sie recht hatte, was seine berufliche Zukunft betraf.

\par

\WR{Madam}, begann er schließlich zögerlich. \WR{Ich konnte die Art, mit der Major Hennington über das gesprochen hat, was uns auf unserem letzten Einsatz passiert ist, einfach nicht ertragen. Gefühlskälte gehört zu unserer Aufgabe dazu, ja. Aber nicht so! Nicht, wenn es um Menschen geht.}

\par

Zu seiner Überraschung blieb Anna abrupt stehen und nickte. \WR{Sie stehen unter massivem Stress, Lieutenant. Und Hennington ist selbst nicht gerade dafür bekannt, zimperlich zu sein. Aber Sie haben zwei Fehler gemacht. Sie haben sich erstens von einem Polizisten zuschauen lassen und zweitens haben Sie ihn \textit{während eines Gefechts} angegriffen. Wissen Sie, was das bedeutet?} Morten verstand, dass dies eine rhetorische Frage war und blieb daher still. \WR{Die Shutek formieren sich für einen geballten Angriff gegen den Planeten und unsere Einheiten in der Umlaufbahn. Es wird schwer sein, sie abzuwehren. Die Kapitänin glaubt, dass dies der beste Zeitpunkt ist, die experimentellen Rapiere einzusetzen. Aber, tja, Major Hennington hat eine leichte Gehirnerschütterung und kann nicht fliegen. Genauso, wie Lieutenant Batosai. Beide wollte ich als Flügelmänner.}

\par

Morten wurde erneut schlecht und er befürchtete, erneut die Toilette in der kleinen Zelle benutzen zu müssen. Und zwar wieder nicht in ihrer Hauptfunktion. Kevin störte etwas anderes: \WR{Kenji? Was ist mit ihm?}

\par

Anna Farleys Blick traf ihn voller Überraschung. \WR{Sie wissen es nicht?} Kevins fragender Gesichtsausdruck beantwortete ihre Frage, noch ehe sie sie ganz gestellt hatte. \WR{Kringel ist tot.} Sie sprach, über sich selbst verwundert, leiser, als sie das sagte. \WR{Er muss die nerven verloren haben. Die Shutek haben ihn erwischt.}

\par

\WR{Verdammt!}, fluchte Kevin laut und schlug mit beiden geballten Fäusten gegen das Plexiglas, dass daraufhin jedoch nur dumpf nach hallte. Er war so schnell aufgesprungen, dass Morten ihn gar nicht neben sich bemerkt hatte und zusammenfuhr.

\par

Nach einer schier endlosen Bedenkzeit fragte Farley: \WR{Fühlen Sie sich fit genug, zu fliegen?}

\par

Morten zuckte. \WR{Mit Ihnen?}

\par

\WR{Nein, mit Amelia Earhart}, donnerte Farley sofort zurück. \WR{Ich brauche zwei gute Piloten. Und, obwohl sie offensichtlich keinen Sinn für Disziplin haben, sind Sie das.}

\par

Kevin schnaubte und ließ sich wieder auf seine Pritsche fallen. \WR{Sie müssen ziemlich verzweifelt sein, dass sie uns beide fragen.}

\par

\WR{Bin ich auch}, entgegnete Farley so ernst wie ehrlich. \WR{Wenn ich meine Befugnisse soweit ausdehne, wie ich es in einem dutzend und drei Dienstjahren nicht getan habe, kann ich dann auf Sie beide zählen?}

\par

Morten nickte sofort eifrig. \WR{Ich werde Sie nicht enttäuschen, Madam.}

\par

\WR{Ich vielleicht schon!}, donnerte Kevin Wilson sofort hinterher. \WR{Collonel, ich bin kein guter Pilot. Wenn ich einer wäre, dann hätte ich es entweder geschafft, sie zu retten oder wenigstens ihre Befehle zu befolgen.}

\par

Nun war es an Anna Farley wütend gegen die Plexiglasscheibe zu schlagen~-- wenn auch nur mit einer einzelnen Faust. \WR{Mister Wilson, ich habe keine Zeit für eine Aufmunterungsrede. Ich brauche jetzt Soldaten! Helfen Sie der \EN{Regenvogel} aus dieser Lage, dann können Sie wenigstens etwas für \textit{eine} Dame tun. Und nebenbei spricht dies für Sie, wenn es zur Verhandlung kommt. \textit{Falls} es zu einer Verhandlung kommt, denn die müssen wir erst einmal erleben.}

\par

Kevin Wilson schien mit sich selbst kämpfen zu müssen, als er sich ein zweites mal von der Pritsche erhob. Um Fassung ringend nahm er neben Morten Haltung an.

\par

Erneut schien Anna Farley ihre Worte genau abzuwägen. Schließlich begegnete sie jedoch Kevins wütendem aber dennoch unsicheren Blick. \WR{Vielleicht sollte ich Ihnen das nicht sagen. Es ist noch nichts bestätigt und ich will Ihnen keine ungerechtfertigten Hoffnungen machen. Aber kurz vor dem finalen Luftangriff ist eine fliehende Fähre in Edhor Peak gelandet. Sie hat einige Zivilisten an Bord genommen. Und der Name ihrer Partnerin stand auf der Passagierliste.}

\par

Sofort schossen Tränen in Kevins Augen und er saugte förmlich die Luft des halben Raumes in seine Lungen.

\par

\WR{Ich habe mit niemandem gesprochen!}, mahnte Farley. \WR{Und ich werde es in diesem Chaos auch nicht tun.}

\par

Kevin nickte nur, wirkte aber mit einem Jahr angespannt und bereit und um Jahre jünger. Er machte keine Anstalten, sich die Tränen aus dem Gesicht zu wischen. Anna Farley drückte ihren Daumen auf die Kontrollen der Arrestzelle und die Tür öffnete sich.

\par

\WR{Sie sind nicht frei}, mahnte sie erneut. \WR{Betrachten Sie Ihre Cockpits als neue Zelle.}

\par

\WR{Danke, Madam}, erwiderte Morten sofort. Kevin schloss sich später aber mit Bedacht deutlich an. Beide folgten ihrer Staffelkommandantin in die Korridor hinaus. Der Wachtmeister, der während der Unterhaltung außen gewartet hatte, beäugte die Gruppe nun sehr kritisch.

\par

\WR{Sagen Sie, Madam}, begann Morten, \WR{dürfen Sie uns eigentlich aus der Haft entlassen? Captain Fiscale hat angeordnet, dass wir beide am Boden bleiben.}

\par

Farley warf ihm einen vernichtenden Blick zu, der ihm bedeutete, still wie das All zu werden. Als die drei den nächsten Lift erreichten, das Flugdeck als Ziel gewählt war und sich die Türen geschlossen hatte, antwortete Farley: \WR{Das hier ist ein Träger. Das bedeutet, es gibt zwei Befehlsketten. Die der Navy und die der Starforce.}

\par

\WR{Aber das bedeutet nicht, dass Sie Befehle der Kommandatin außer Kraft setzen können, oder?}, entgegnete Morten schnell.

\par

Anna Farley seufzte mit Absicht laut. \WR{Nein. Es bedeutet aber, dass sie es nicht sofort mitbekommt, wenn ich es tue. Und jetzt halten Sie Ihre verdammte Klappe. In Ihrem Lebenslauf haben Sie so gute Noten vorzuweisen. Aber offensichtlich sind Sie dämlich wie ein Stück Brot.}

\par

Morten entschied sich augenblicklich, seine kaum offizielle Kommandantin nicht weiter zu reizen, auch wenn ihm nicht wohl mit dem Gedanken war, einen klaren Befehl zu missachten. Dabei machte er sich weniger Sorgen um sich selbst. Auch für Anna konnte dies ernste Konsequenzen nach sich ziehen.

\par

Die drei waren schnell an ihren Spinden und noch schneller hatten sie ihre Ausrüstung angelegt. Morten bemerkte erst jetzt den kleinen Spiegel, der auf der Innenseite der Metalltür hing. Er sah in sein Gesicht und erkannte den Mann nicht wieder, der seinen Blick nun erwiderte. Vor wenigen Wochen hatte er zum ersten mal seine Fliegerkluft an Bord der \EN{Regenvogel} angelegt. Aber das war ein anderer Mensch gewesen. Niemand, der den Tod gesehen oder selbst tödliche Gewalt angewandt hatte. Niemand, der sich verliebt hatte und auch niemand, der nun zu seiner rechten einen Freund erkannte.

\par

Kevin Wilson traf seinen durchdringenden Blick und wirkte zunächst verwirrt. Doch er verstand schnell und ohne Worte, was Morten dachte und nickte ihm zu.

\par

\WR{Also los meine Herren!}, spornte Anna Farley an. \WR{Heben wir ab, bevor es zu spät ist?}

\par

\WR{Ich hörte, die Flotille der Shutek trifft erst in gut einer Dreiviertelstunde ein}, gab Morten zu bedenken.

\par

Die Flügelkommandantin wurde nicht langsamer in ihrem Gang zum Hangar hin, als sie antwortete: \WR{Korrekt. Aber wenn wir jetzt starten, dann haben wir noch Zeit für eine kleine Spritztour. Besonders Sie beide.}

\par

Der Hangar wirkte seltsam leer. Fast alle Jäger waren im Einsatz und die beschädigten Maschinen wurden unter Decks zusammengeflickt. Nur die drei Rapiere, noch ohne jegliche Markierung oder gar Kratzer, standen nah an der Abschirmung zum All bereit. Vor der Treppe hinunter hielt Anna noch einmal kurz inne.

\par

\WR{Ich bin mir absolut sicher, dass Nico Curiosa noch da unten ist}, sagte sie entschlossen. \WR{Ich kann das nicht von Ihnen verlangen. Andererseits stecken Sie schon in genügend Ärger. Es wird Sie nichts kosten, wenn Sie einen kurzen Flug in der Nähe seiner Absturzstelle unternehmen. Vielleicht finden Sie ihn. Falls ja: geben Sie seine Position durch. Bevor das Bombardement beginnt, werden die Soldaten der Phalanx evakuieren. Eventuell nimmt ihn jemand mit.}

\par

Zu Kevins großer Überraschung war es Kevin, der elanvoll bejahte. Anna nickte nur zufrieden und ging dann als erste die Treppenstufen zur Startbahn hinunter. Kevin hielt Morten noch für einen Moment zurück. \WR{Du? Vor einem Tag wolltest du ihn noch eigenhändig von Bord werfen.}

\par

\WR{Wie du sagtest}, antwortete der Gefragte. \WR{Das war gestern. Bevor er mir den Arsch gerettet hat. Ich hatte einen Fluglehrer an der Akademie. Sein Name war Martin Siegel. Er hat mir eines eingebläut. Wenn dein Flügelmann für dich da ist, dann sei auch für ihn da. Egal wann und egal wer es ist.}

\par

\WR{Guter Mann}, sinnierte Kevin, doch sein Blick war längst von den Jägern eingefangen worden. Sie waren noch so neu, dass jedes einzelne Teil zu blitzen schien. Im Vergleich mit den anderen Fliegern der \EN{Regenvogel} wirkten sie auf den ersten Blick sogar recht plump. Es fehlte jede Art der Verzierung. Auch gab es keine offenliegenden Kanonen oder Raketenaufhängungen. Doch die Rapiere waren vollgestopft mit modernster Technik. Jedes Exemplar kostete ein Vermögen und selbst damit rechneten sich die Aufwendungen für ihre Entwicklung nicht.

\par

\WR{Also los}, sagte Kevin zu sich selbst und begann an der Leiter hinauf zu klettern, dich nicht an den Jäger heran geschoben werden musste. Er fuhr sie selbst aus.

\par

Nun war es an Morten, ihn aufzuhalten. \WR{He, Murphy}, rief er ihm zu. \WR{Deine Mutter ist so fett, sie könnte als Ankerpunkt für einen Hyperraumsprung dienen.}

\par

Kevin lachte herzhaft. \WR{Mach dich auf was gefasst!}, drohte er und schwang sich in die Pilotenkanzel seines Jägers.

\par

Kurz darauf bestieg auch Morten das Cockpit seines schlanken wie schlichten Jägers. Sofort stieg ihm der Geruch der neuen Materialien in die Nase. Fast wie bei einem gerade erst gekauften Auto. Zügig startete er die Diagnoseprogramme. Es blieb nicht viel Zeit, denn die mit dem Mutterschiff gekoppelten Sensoren zeigten bereits die ersten feindlichen Kräfte auf Angriffskurs.

\par

Während die Checkliste vom erstaunlich schnellen Computer abgearbeitet wurde, öffnete Morten eine Funkverbindung zu Kevin.

\par

\WR{Bin fast mit den Checks durch}, berichtete dieser sofort. \WR{Noch ein paar Sekunden und wir sind in der Luft!}

\par

Morten sah auf seinen eigenen Bildschirm. \WR{Weißt du, was mir gerade aufgefallen ist?} Kevin blieb still. \WR{Unsere kleine Staffel hat noch gar keinen Namen. Wir sollten ihr einen geben.}

\par

\WR{Okay…}, antwortete sein Flügelmann erwartungsvoll.

\par

Morten zögerte einen Augenblick lang. \WR{Deine Freundin. Wie ist ihr Name?}

\par

\WR{Julia}, erwiderte Kevin prompt. \WR{Julia Marchat.}

\par

\WR{In Ordnung, Julia eins: bereit zum Start.}

\par

Kevin lachte verhalten. \WR{Julia zwei: bereit.}

\par

Nun schaltete sich auch Anna Farley in das Gespräch ein. \WR{Leute. Ihr habt wenig Zeit. Ich spreche mich unterwegs mit dem Rest unserer Staffeln ab und überlege mir einen Plan zu Verteidigung. In zwei dutzend Minuten müsst ihr Nico gefunden haben, oder ihr fliegt zurück, verstanden?}

\par

\WR{Bestätigt}, gaben die beiden fast zeitglich zurück.

\par

Morten ließ seine Landeklappen einfahren. Der Rapier erstaunte ihn jetzt bereits. Anders als die anderen, älteren Jäger der Starforce, schwebte er einfach geduldig über dem Boden. Natürlich fraß der Antigravitationsantrieb viel Energie. Aber die Leistung des Reaktors machte dies mühelos wett.

\par

\WR{Na dann los!}, rief Morten über Funk und drückte den Geschwindigkeitsregler ganz an den Anschlag. Nur unmaßgeblich langsamer als der Falke, den er vor einigen Stunden noch geflogen war, schoss der Rapier aus dem Hangar heraus. Kevin war ihm bereits dicht auf den Füßen. Die \EN{Regenvogel} wurde bereits im rückwärtigen Monitor kleiner und kleiner und Morten zweifelte daran, dass Fiscale ihnen Jäger auf den Hals schicken würde.

\par

Falls sie es doch tat, würde er sich jedoch noch um einiges schuldiger fühlen. Die zwei Rapiere fehlten dem Inventar der \EN{Regenvogel} bereits jetzt.

\par

Morten hatte mit der Lenkung seines neuen Fliegers zu kämpfen. Statt einem Steuerknüppel flog sich dieser mit zwei parallelen Griffen, die gleichzeitig das Bedienen von regulären Triebwerken und Antigravitations-Scheiben erlaubten.

\par

Schließlich hatte er den Dreh heraus und die beiden Jäger rasten der Atmosphäre des Planeten entgegen.