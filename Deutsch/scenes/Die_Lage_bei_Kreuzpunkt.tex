Und schon prasselten die ersten Strahlensalven auf die Schutzfelder der \EN{Regenvogel} ein. Keine Sekunde zuvor war Captain Fiscale zu sich gekommen und hatte noch mitbekommen, wie die grellen Lichter um ihr Schiff sich verflüchtigt hatten wie Wasserdampf.

\par

Drei Jäger stoben kurz vor der Brücke des Trägers auseinander und brausten an den Seiten des Kommandozentrums vorbei. Maas Petrarca versuchte noch, die Situation zu überblicken, meldete aber bereits: \WR{Ich zähle mehr als zwei dutzend Bogeys in umserem inneren Verteidigungsring. Alles Jäger, bisher keine Bomber.}

\par

\WR{Feuer erwidern und sofort mit den Jägerstarts beginnen!}, rief Fiscale an ihren Kommunikationschef gewandt. \WR{Haben Sie verstanden?}, schrie sie, als dieser nicht sofort antwortete.

\par

Der blauhaarige Lieutenant Wallander bejahte kurz darauf und hängte an: \WR{Captain, wir haben unsere NZT-Verbindung mit dem Hauptquartier verloren!}

\par

\WR{Das Computervirus?}, fragte Fiscale verärgert.

\par

\WR{Nein, Madam. Alle Systeme reagieren. Es muss ein physikalisches Phänomen sein}, war die prompte Antwort. \WR{Ich sage Miss Schwarzschild, sie soll sich darum kümmern.}

\par

Fiscale, die nunmehr zur Kommunikationsstation gekommen war, schüttelte heftig den Kopf. \WR{Befehl zurück! Lieutenant Schwarzschild?}

\par

Die Radarchefin rief ihr vom unteren Brückebereich etwas unverständliches zurück. Das Dröhnen der Kanonen, die den Jägern Strahlen entgegenspien, übertönte jedes andere Geräusch für einen kurzen Augenblick.

\par

\WR{Langstreckenabtastung des Raums um Kreuzpunkt Primus. Ich muss wissen, was dort passiert und zwar genau!} Die Chefin der Radarabteilung nickte den Befehl der Kommandantin ab und rannte zu ihrer Station zurück. Während dessen hämmerten mehrere Strahlensalven gegen die Schilde der Brücke. Die Kanonniere hämmerten sekündlich aktualierte Zieldaten und -prioritäten in ihre Konsole. Auf ihren mannshohen Glasmonitoren waren Schussrichtungen der Kanonen und die Trajektorien der feindlichen Jäger eingezeichnet.

\par

\WR{Flugdeck meldet: \Wr{Startvorgang eingeleitet.}}, rief Lieutenant Wallander Fiscale entgegen. Diese richtete ihren Blick für einen Augenblick auf die Startfläche ihres Schiffes. Und schon hoben die ersten Abfang- und Verteidigungsjäger der Bravo und Charlie Staffeln ab. Ihre voll gezündeten Nachbrenner leuchteten blutrot vor dem unendlichen Schwarz.

\par

\WR{Was macht die Kurzstreckenkommunikation?}, die Kapitänin hatte sich mittlerweile über die Schulter des Kommunikationsoffiziers gebeugt. Wallander presste sich den Kopfhörer mit beiden Händen auf die Ohren und ließ ihn nur los, um hin und wieder Eingaben in seine Konsole zu machen. Schließlich bekam er große Augen.

\par

\WR{Ein Ruf von der Oberfläche}, meldete er hoffnungsvoll. \WR{Es ist Legat Gajjar von der Hauptstadt-Garnison.}

\par

\WR{Durchstellen, Lieutenant.}

\par

Auf Fiscales Anweisung hin, baute Wallander eine Videoverbindung zum Hauptquartier der Garnison auf. Auch die Kurstreckenkommunikation schien beeinträchtigt, denn immer wieder flackerten Bildstörungen über den Monitor. Kurz darauf war eine ältere Dame mit langem schwarzen Zopf zu erkennen, die eine Paradeuniform der Phalanx trug, darüberhinaus allerdings keinen sehr feierlichen Eindruck machte. Sie schwitzte und die Anspannung stand ihr ins Gesicht geschrieben.

\par

\WR{\EN{Regenvogel}, Brücke, bitte kommen.}

\par

\WR{Captain Fiscale von der \EN{Regenvogel} hier}, antwortete ihr die Kommandantin sofort. \WR{Wir sind soeben angekommen und brauchen einen Statusbericht.}

\par

Es dauerte ein wenig, bis die Generalin antwortete. Kreuzpunkt Primus war zwar bereits vage zu erkennen~-- der Planet wirkte durch die Panoramascheiben der Brücke etwa so groß, wie der Mond von der Erde aus betrachtet~-- doch immer noch weit genug entfernt, um für konventionelle Kommunikationswege eine merkliche Verzögerung herbeizuführen.

\par

\WR{Captain, wir haben die Minvera verloren.} Fiscale rang nach Atem. Die Installation, von der Gajjar sprach, war die zweitgrößte Festung im Repertoire der Navy. Ihr Verlust, so kurz nach Beginn der Schlacht, ließ die feindliche Stärke noch überwältigender erscheinen. \WR{Sie wird in wenigen Stunden auf die Atmosphäre treffen. Evakuierung ist im Gange. Die \EN{Artiglio de Leone} antwortet nicht. Das letzte, was wir wissen, ist dass sie einen Treffer auf der Brücke hinnehmen musste. Sie driftet, aber im Moment ist ihr Orbit stabil. Und sie konnte alle ihre Jäger starten. Außerdem sind Vögel von der Minvera und einigen Bodenstationen in der Luft. Die Shutek sind noch nicht im Orbit. Aber sie haben bereits Bodentruppen gelandet! Unsere Streitkräfte werden überrant!}

\par

Commander Samad, der auch zur Kommunikationssation gekommen war, runzelte die Stirn. \WR{Bodentruppen}, sagte er leiste mehr zu sich selbst. \WR{Anders als in Pollux.}

\par

\WR{Ich nehme an, ihr Ziel sind die orbitalen Verteidigungsanlagen bei Juligrad}, spekulierte Legat Gajjar nach einiger Zeit. \WR{Wir haben immer noch etliche Marschflugkörper, die wir in die Umlaufbahn schießen können. Ein Bombardement bekommen Sie vermutlich erst mal nicht hin. Aber sie werden es versuchen. Mehrere ihrer Schiffe sind auf dem Weg. Kreuzer, Fregatten, Zerstörer und etwas, das die Jungs vom der Starforce-Basis für einen Träger halten.}

\par

Fiscale streckte ihre Hand in Richtung des Ruders aus, als wolle sie es selbst ergreifen. \WR{Steuermann, direkten Kurs auf Kreupunkt Primus, heizen Sie die Triebwerke richtig auf!} Dann wandte sie sich wieder der Videokonferenz zu. \WR{General, bitte verbinden Sie Ihre Computer mit unseren. Wenn die \EN{Artiglio de Leone} tatsächlich keine Brücke mehr hat, dann führt jetzt die \EN{Regenvogel} die Flotten im Raum.}

\par

\WR{Sollen Sie haben}, sagte Gajjar zu. \WR{Und Sie sollten sich auch mit der Crossguard koordinieren. Das Schlachtschiff hält wacker seine Position bei \EN{Minerva}. Aber sie liegen unter schwerem Beschuss und halten sicher nicht mehr lange durch.}

\par

Fiscale schluckte. Sie kannte die Verteidigungsstruktur des Kreuzpunkt-Systems. Die \EN{Artiglio de Leone} war ein mittlerer Träger mit einem vollen Jägerkontingent, und gleichzeitig das Flaggschiff der kreuzseitigen Heimatflotte. Die Crossguard war zwar kurz nach dem Routenkrieg ein stolzes Schlachtschiff gewesen, doch ihre Waffen und Verteidigungssysteme waren mittlerweile längst überholt. Gegen einen gut ausgerüsteten Feind wie die Shutek würde sie sicher selbst mit Jägereskorte nicht lange überleben.

\par

\WR{Geben Sie uns bitte auch günstige Landezonen durch}, fuhr Fiscale schließlich fort. \WR{Wir haben drei Basisschiffe dabei. Jedes voller Infanteristen, die den kleinen grünen Männchen endlich in den Arsch treten wollen.}

\par

\WR{\EN{Klein} sind sie nicht. Machen Sie sich keine Hoffnungen. Ich gebe Sie jetzt an meine Kommunikationsabteilung durch. Schön, dass Sie da sind und gute Jagd!}

\par

Fiscale wollte sich gerade wieder auf ihren Stuhl setzen, da deutete ihr erster Offizier gerade aus auf die Fenster. \WR{ich fürchte, die Shutek lassen uns nicht so einfach zum Rest der Feiergemeinde durch.} Tatsächlich waren deutlich drei Schiffe der Shutek zu erkennen. Zwei davon schienen auch auf die Entfernung zu den \Wr{Pferderippen-Schiffen} zu gehören, denen die \EN{Regenvogel} bereits in Pollux begegnet war. Das andere war größer und hatte einen sehr auffälligen Bug, der fast an ein Zwiebeldach aus dem Moskau vor der Seuche erinnerte.

\par

Elshe Schwarzschilds Bericht folgte kurze Zeit später. \WR{Laut Kreuzpunkt treffen diese Schiffe die Spezifikationen von Fregatten beziehungsweise einem Kreuzer. Zielbezeichnungen: \EN{Astaroth}, \EN{Amon} und \EN{Aeshma}.}

\par

\WR{Mister Wallander, unsere Zerstörer, sollen Flankenpositionen einnehmen}, befahl Fiscale. \WR{Das wird der erste Fang des Tages.}

\par

Noch ehe dieser ein \WR{Jawohl, Madam}, von sich geben konnte, schrie Schwarzschild: \WR{Alarm: Marschflugkörper!}
