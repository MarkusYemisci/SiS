Die Sonne ging gerade unter, als Laura eine kleine Nebenstraße von Neuheim entlang ging. Das Frühjahr musste über den Schwarzwald hereingebrochen sein, denn nicht nur Laura war deutlich zu warm angezogen. Die meisten Passanten, begegneten ihr in dicken Wintermänteln. Und sie schwitzte besonders in ihrer schwarzen Dienstkleidung. Ihr Körpergeruch war ein weiterer, doch längst nicht der größte Quell ihrer Verunsicherung.

\par

Da half auch die idyllische und sonnige Szenerie nicht viel. Zwar sprießten am ein oder anderen Baum schon vereinzelt die ersten Blüten, doch keiner der warmen Sonnenstrahlen erreichte Lauras Inneres.

\par

Es wunderte sie nicht. Ein linker Haken des Schicksals hatte das Haus der Bells am Ende statt am Anfang der Straße stehen lassen, so dass Laura sich viele Minuten lang dazu zwingen musste, den Weg weiter zu gehen.

\par

Während dessen war sie öfter als je zuvor in ihrem Leben versucht, einfach umzukehren und nach Hause zu gehen. Vor ihrer Reise in den südlichen Schwarzwald, dessen Name von den vielen dunklen Nadelbäumen herrührte, hatte ihr Vorgesetzter sie über drei Stunden lang ins Gebet genommen und ihr noch einmal klar gemacht, was er von ihrem Verhalten in Oslo hielt. Doch sie hätte dieses Gespräch, das vielmehr zu einem Monolog O'Sheas verkommen war, lieber zehnmal geführt, statt sich der Unterhaltung zu stellen, die sie nun schon seit Monaten vor sich herschob.

\par

Dankbarkeit war, was sie als erstes empfand, als sie Thalia Bell vor die Tür ihres Hauses treten sah. Die Mutter von einem~-- bis vor kurzem noch zwei~-- Kindern sah nicht gut aus. Ihre Haare hatte sie zu einem unordentlichen Knoten hochgesteckt und ihrem Gesicht fehlte jede Farbe. So, als sei sie nur eine Puppe, die sich wie an Fäden bewegte. Genauso lose wie eine Marionette hielt sie auch die beiden Gartengeräte, mit denen sie wohl vorhatte, ihren sonst gut gepflegtes Blumenbeet vor ihrem Haus in Schuss zu halten.

\par

Thalia hatte Laura sofort entdeckt und nun konnte sie nicht mehr umdrehen. Auf den letzten zehn Metern ihres Weges musste sie nur noch den Blick der trauernden Mutter ertragen, doch ihre Füße funktionierten nun wie von selbst.

\par

\WR{Es tut mir so leid!}, sagte Laura die vier Worte, die ihr schon seit dem letzten Herbst nicht über die Lippen kommen wollten und bemerkte nicht, wie ihr dabei die Tränen in die Augen stiegen.

\par

Bell ließ daraufhin ihre Harke und die Schaufel fallen. Auf sie kämpfte mit den Tränen, schien dabei aber auf deutlich mehr Übung zurückgreifen zu können, wie Laura. Einen Moment lang wusste diese nicht, ob sie sich nicht demnächst eine Ohrfeige einfangen würde. Doch dann hob die Mutter ihre Utensilien auf und deutete mit dem Kopf in Richtung ihrer Haustür.

\par

Laura hatte nicht das geringste Bedürfnis, das Heim der Bells zu betreten. Aber nun konnte sie keinen Rückzieher machen. Dafür war sie schon weit gegangen. Wortlos folgte sie Thalia ins Innere.

\par

Die Wohnung wirkte mit ihren klassischen Möbeln gemütlich. Selbst nach allem, was sich dort abgespielt hatte. Die Entführung eines Kindes, der Zusammenbruch einer Ehe, das Verhör durch eine Agentin der Argus-Abteilung. Laura hängte dankbar ihren langen Mantel in die Garderobe, als Thalia Bell ihr dies anbot und schämte sich sofort für ihr Schwitzen. Für jemanden, der Oslo gewöhnt war, bedeutete ein sonst kühler Schwarzwald im Frühjahr trotzdem Wärme.

\par

\WR{Vor der Seuche gab es hier eine alte Tradition}, begann Bell, während sie in die Küche ging. \WR{Die Menschen haben hier \Wr{Fastnacht} gefeiert. Irgendwann zwischen Ende Januar und Anfang März. Hing wohl irgendwie mit Ostern zusammen~-- sie wissen, diesem Christenquatsch.}

\par

Laura nickte, obwohl Thalia Bell sie nicht sehen konnte, blieb aber sonst wie angewurzelt stehen.

\par

\WR{Einige wollen dieses Fest wieder aufleben lassen. Wenn Sie mich fragen, gehört so ein Blödsinn nach Klae. Aber jeder kann ja machen, was er will. Und so torkeln mir schon seit ein paar Tagen betrunkene, verkleidete Deppen in mein Blumenbeet. Ihre Festhalle ist wohl in der Nähe.}

\par

\WR{Das ist…}, Laura stockte, \WR{blöd.} Genauso hätte sie auch ihre eigene Aussage im Angesicht der Situation beschrieben. Bell reagierte nicht darauf. Sie kam mit zwei Tassen Tee aus ihrer Küche und deutete auf das Wohnzimmer. Laura folgte widerstandslos. Es war klein, wirkte aber genauso stimmig, wie der Rest des Hauses. Im Sommer war es sicher schön, sich auf die Terasse zu setzen und in die tieferen Regionen des Tals zu sehen.

\par

Thalia Bell hatte wohl Lauras Gedanken erraten. \WR{Hier haben wir oft zusammen zu Abend gegessen. Auch wenn es noch so kühl war, wie jetzt. Oder im Herbst. Aber jetzt sind es nicht mehr die Temperaturen, an denen es scheitert, hab ich recht?}

\par

Laura öffnete den Mund, fand aber keine Worte.

\par

\WR{Schon in Ordnung}, erwiderte Thalia Bell. \WR{Sie können dazu nichts passendes sagen. Das verstehe ich. Setzen Sie sich.}

\par

Nach wie vor wortlos nahm Laura auf dem Ecksofa platz und die Tasse Tee von ihrer Gastgeberin entgegen. Nachdem diese anfing zu trinken, nahm sie auch einen Schluck. Er schmeckte gut. Aus den Händen einer Frau, die Routine darin hatte, unter Stress Essen und Trinken zuzubereiten. Aus den Händen einer Mutter.

\par

\WR{Also}, fing diese an. \WR{Sind Sie hergekommen, um sich bei mir zu entschuldigen?}

\par

Laura nickte.

\par

\WR{Wissen Sie, Sie sind früh dran}, antwortete Thalia sofort und zeichnete damit ein Bild, dass dem genauen Gegenteil von Lauras eigenem Empfinden entsprach. \WR{Noch vor sechs Monaten haben Sie mich verdächtigt, das Verschwinden meiner Tochter vorgetäuscht zu haben, während ein Verrückter sie vermutlich gerade verbuddelt hat.} Dabei traten ihr erneut Tränen in die Augen, die sie nicht so leicht zurückhalten konnte. \WR{Nach so einer Sache, hätte ich sie nicht einmal in den nächsten drei Jahren hier erwartet.}

\par

\WR{Es tut mir so unendlich leid}, hauchte Laura, wurde aber sogleich lauter. \WR{Ich wünschte, das alles wäre nie passiert. Ich wünsche mir das jeden Tag, jede Sekunde. Ich wünschte, er hätte es nicht geschafft, mich an der Nase herumzuführen und dass ich Johanna schneller gefunden hätte. Es war meine Aufgabe. Aber ich habe sie nicht erfüllt. Ich habe zugelassen, dass er sie tötet.}

\par

Thalia, die gerade begonnen hatte, zu weinen, wurde mit einem mal ruhig.

\par

\WR{Ich schwöre Ihnen, wenn es irgendetwas gäbe, das ich tun könnte, ich würde es machen. Wenn ich mein Leben gegen das von Johanna eintauschen könnte. Ich würde keine Sekunde zögern. Seit dem Tod ihrer Tochter hatte ich keinen freudigen Moment mehr. Ich trauere und es fühlt sich an, als hätte ich niemals etwas anderes getan.}

\par

Unvermittelt schnaubte Thalia Bell. \WR{Wissen Sie, wie egoistisch Sie sind? \EN{Sie} trauern? Es ist nicht an Ihnen, traurig zu sein. Dass tuen ich als Johannas Mutter und ihr Bruder. Sie sind nur eine Beamtin, die ihre Arbeit so gut gemacht hat, wie sie es eben konnte. Glauben Sie etwa, Johanna wäre ihretwegen gestorben?}

\par

Laura blinzelte verwunderte und rang nach Worten, während Thalia bereits fortfuhr. \WR{Johanna ist tot, weil ein sadistisches Arschloch sie ermordet hat. Das ist der Grund. Darum ist sie gestorben. Nicht wegen Ihnen, Gott oder dem Schicksal oder was weiß ich. Also hören Sie auf, sich zu bemitleiden. Das macht die Dinge nicht ungeschehen.}

\par

Bell nippte an ihrem eigenen Tee. Sie schien ihn entweder bei weitem zu heiß oder zu ungenießbar zu finden, denn sie stellte die Tasse sofort wieder ab. Unterdessen versuchte Laure weiter vergeblich, in Worte zu fassen, was sie empfand. Vor ihrem Eintreffen hatte sie sich genau zurecht gelegt, was sie sagen wollte. Nun fand sie nichts davon mehr angemessen. Dankbar hörte sie einfach zu, als Thalia Bell fortfuhr. \WR{Wissen Sie, ich kann nachvollziehen, wie es Ihnen zumute sein muss. Mir ging es anfangs auch nicht anders. In jeder ruhigen Minute habe ich an Anbrimad Hannibal gedacht. An sein krankes Lächeln während der ganzen Gerichtsverhandlung. Wie er das Urteil abgenickt hat, als wär's etwas, dass er von Anfang an eingeplant hatte.} Sie nahm einen weiteren Schluck. Doch Laura konnte erkennen, dass sie dies nicht tat, weil ihr Bedürfnis nach Tee so groß geworden war, sondern, weil ihre Stimme versagte.

\par

\WR{Es hat mich absolut krank gemacht. Ich war so wütend, ich habe mein halbes Zimmer zertrümmert. Sogar mit meinem Sohn habe ich nur noch gestritten. Von mir war nicht mehr viel übrig, wissen Sie? Genauso wie von Ihnen nur noch wenig da ist, das sehe ich Ihnen sofort an.

\par

Aber irgendwann habe ich weitergemacht. Angefangen, wieder zu leben. Nicht, weil ich damit Hannibal ein Schnippchen schlagen wollte~-- denn unser Leben kaputt machen, war genau, was er wollte. Nicht nur die Leben meiner Familie. Sondern auch Ihres, das wissen Sie hoffentlich. Dagegen können wir nichts tun, so sehr uns das auch verzweifeln lässt.

\par

Ich habe es getan, weil es wichtigere Dinge gab, als Anbrimand Hannibal. Meine Tochter ist gestorben. \EN{Das} ist wichtig. Sie fehlt mir so sehr…} Thalia schluchzte und Laura saß wie eingefroren da. \WR{Das ist, worum es geht. Ich will nicht an dieses kranke Arschloch denken. Ich will mich an Johanna erinnern. Sie hat so viel Liebe in dieses Haus gebracht, ohne es zu wissen. Würde sie uns jetzt hören können, sie könnte sich gar nicht vorstellen, wie groß der Riss ist, den sie hinterlassen hat.}

\par

Laura schluckte und strich sich über ihre glühende Stirn, um den Augenkontakt mit Thalias Bell zumindest kurz abbrechen zu können.

\par

\WR{Ich danke Ihnen, dass Sie hergekommen sind, Frau Gethas}, sagte diese schließlich etwas gefasster. \WR{Sie haben sicher sehr strikte Vorschriften, was Gespräche mit Angehörigen betrifft. Vermutlich durften Sie sich gar nicht bei mir entschuldigen.}

\par

\WR{Das ist richtig}, stimmte Laura zu. \WR{Aber die Regeln waren nicht schuld, dass ich nicht früher hier war.}

\par

Bell nickte verstehend. Dann sagte sie: \WR{Bitte verstehen Sie mich nicht falsch, aber Sie sollten jetzt gehen. Sie hätten gar nicht kommen brauchen. Zumindest nicht meinetwegen. Ich weiß, wie Sie sich fühlen und was Ihnen durch den Kopf geht. Wenn es Ihnen hilft: Sie sind nicht schuld an ihrem Tod und ich bin nicht wütend auf Sie. Nicht mehr. Ich bin hier nur noch nicht… fertig.}

\par

Laura sprang förmlich auf. \WR{Natürlich. Bitte verzeihen Sie.}

\par

Thalia Bell begleitete Sie an die Türe. Bevor Laura jedoch ging, sah sie sie noch einmal durchdringend an. \WR{Sie haben gesagt, Sie würden Ihr Leben gegen Johannas eintauschen.} Laura nickt tief. \WR{Wissen Sie, meine Tochter hat das Leben und die Menschen um sich geliebt. Sie hatte noch so viele Träume~-- wollte Gärtnerin werden. Und sie hatte der Welt noch so viel zu geben. Sie können Ihr Leben nicht tauschen. Aber wenn Sie diesen Wunsch wirklich haben, dann sollten Sie zumindest so leben, dass es ein gerechter Tausch wäre.}