Familie war Laura früher sehr wichtig gewesen. Ihre Eltern waren einfache Menschen, die ihr manches nicht hatten beibringen können, was sie in ihrem späteren Leben gebraucht hätte. Das wichtigste jedoch schon. Erst spät während ihrer Zeit auf der Akademie, war ihr ein Sprichtwort aus der Zeit vor der Seuche begegnet. \WR{Blut ist dicker als Wasser.} Und nicht nur ihr Vater oder ihre Mutter, sondern so ziemlich jedes Mitglied ihrer großen Verwandschaft hatte ihr die Bedeutung dieses Spruches näher gebracht.

\par

Diese Zeiten, in denen jeder Streit ausgeräumt werden konnte und die Liebe zu Bruder, Mutter oder Vater ungerührt weiter bestand, war der Realisierung gewichen, dass auch das stärkste Band reißen konnte. Laura sah ein, dass sie ihrer Familie viel zugemutet hatte, doch sie wusste nicht, was sie hätte anders machen können. Am Ende waren die Dinge gekommen, wie sie eben gekommen waren und niemand hatte verstehen können, wie es in ihr aussah.

\par

Daran, Vorwürfe zu erheben dachte Laura nicht. Nur jemand, der in ihrem Beruf arbeitete, konnte nachvollziehen, was sie durchgemacht hatte. Ihre Mutter~-- so empathisch und liebevoll sie auch war~-- konnte kaum wissen, wovon sie sprach, wenn sie versicherte, dass mit der Zeit alles besser werden konnte und das sie vergessen würde, was sich zugetragen hatte.

\par

Ihr Bruder war so idiotisch wie pragmatisch. Für ihn war es also besonders leicht zu sagen, dass er keine Schuld bei ihr sah. Vorsichtiger aber nicht weniger überzeugt, hatte sich ihr Vater über die Katastrophe geäußert, die Laura durchlebt hatte. Bei so viel Unverständnis war es nicht verwunderlich gewesen, dass ein großer Streit die Fronten verhärten und voneinander entfernen würde. Es hatte auch nicht viel dafür gebraucht. Eine schmatzende Tante, ein paar falsche Worte am Mittagstisch und schon waren die lauten Worte und die Tränen da.

\par

Nachdem sich dieses Spiel einige male wiederholt hatte, wollte niemand mehr, dass Laura über das Wochenede zu Besuch kam. Niemand hatte das laut ausgesprochen. Aber es war klar, dass es niemandem recht war. Darum wollte sie auf einen Abschied verzichten. Ihr Bruder war bereits zur Arbeit gegangen und ihre Eltern, beide Krankenpfleger von Beruf, schliefen aus. So war es für sie kein Problem, sich ins Badezimmer zu schleichen, ohne jemandem zu begegnen.

\par

Nun rief auch für sie nach dem Wochenende wieder die Arbeit und obwohl sich Laura allerorts gleichermaßen schlecht fühlte, fiel es ihr trotzdem schwer, sich aufzuraffen. Ihr Spiegelbild sah ihr traurig entgegen, während sie sich lustlos frisch für den Dienst machte.

\par

Kurze Zeit später erreichte Laura die Haustür. Die zwei gläsernen Türhälften glitten zur Seite und sie trat ins Freie. Das Haus ihrer Familie stand in einem Vorort von San Salvador, daher begann sie ein wenig zu schwitzen, als sie in ihrem Dienstmantel auf die Straße trat. Noch einmal drehte sie sich kurz um und warf einen Blick auf das Heim, aus dem sie gekommen war. Die Bauweise des Hauses war überall in der ganzen Union zu finden. Viele gläserne Wände, tragender Stahl und ein flaches Dach auf dem die Fahne der Unio Terrae wehte. Tatsächlich sah das Haus sogar ein wenig instabil und zerbrechlich aus, da es größtenteils aus Metallpfeilern und Fenstern zu bestehen schien. Aber die Architekten versicherten immer wieder, dass die Konstruktion zu den sichersten überhaupt gehörte.

\par

Laura machte sich keine Gedanken darüber sondern beschleunigte ihre Schritte um den Interkontinentalzug noch rechtzeitig zu erwischen.

\par

Sie bewegte sich schnell durch die Straßen auf dem Weg zum Bahnhof. Die meisten anderen Häuser der südländischen Stadt fanden sich ebenfalls im Glasstil der Union. San Salvador war während der Seuche fast vollständig zerstört und erst Jahrhunderte später wieder neu aufgebaut worden. Laura fragte sich häufig wie die Stadt wohl in ihrem ursprünglichen Zustand ausgesehen hatte. Im Museum gab es viele Fotos und Modelle des San Salvadors aus dem zwanzigsten Jahrhundert aber trotzdem konnte sie sich kein lebendiges Bild der Vergangenheit ausmalen.

\par

Nach kurzer Zeit hatte sie den Bahnhof erreicht, der ebenfalls in der gläsernen Bauweise errichtet worden war. Sie war froh über die Kühle im klimatisierten Innenbereich, als sie ihn schließlich betrat. In der großen Haupthalle des Bahnhofs herrschte, wie jeden morgen, hektische Betriebsamkeit. Die Massen an Menschen erinnerte Laura an die Wasser verschiedener Ströme, die ineinander flossen. Einer der Hauptströme bewegte sich auf die Fahrstühle und Rolltreppen zu, die zu den unterirdischen Gleisen der Internkontinentalzüge führten.

\par

Laura hatte andere Ziele. Sie ging vorbei an einer Bäckerei, die ihre Waren mit großem Umsatz an die Reisenden verkaufte, und verschaffte sich so freie Sicht auf den zehn mal sechzehn Meter großen Hauptmonitor der an der gläsernen Bahnhofsdecke hing.

\par

Sie erkannte schnell, dass sie ihren Zug nach Oslo in Norwegen noch nicht verpasst hatte. Allerdings sagte ihr eine Anzeige, dass sie, nicht wie üblich zu Gleis drei dutzend, sondern diesmal zu Bahnsteig zweidutzend eins musste.

\par

Laura war froh, den Zug wohl noch zu bekommen. Ansonsten hätte sie fünfzig Minuten auf den nächsten warten müssen. Schnell durchquerte sie die Haupthalle des Bahnhofs und stieg eine Treppe hinauf, die zu einer von Glas umhüllten Brücke führte. Diese erstreckte sich über alle fünfzig Bahnsteige des Bahnhofs und bot die Möglichkeit die Gleise zu erreichen. Insgesamt hatte der Bahnhof acht solcher Brücken, die an der gesamten Länge der Haupthalle entlang verliefen. Und eine war voller als die nächste.

\par

Der Betrieb lichtete sich allerdings, als Laura zu den Gleisen kam, die für die Interkontinentalzüge gedacht waren. Die meisten Menschen fuhren zur Arbeit irgendwo in Süd- oder Nordamerika. Nur ein kleinerer Teil wollte nach Übersee.

\par

Schließlich erreichte Laura Gleis zweidutzend eins, nachdem sie eine gut zehn Meter hohe Treppe hinunter gegangen war. Tiefer durfte die Brücke zu den Gleisen nicht verlaufen, sonst würden einige mehrstöckige Transportzüge nur noch schwer hindurch passen. Selbst die meisten Personenzug-Waggons hatten die Ausmaße des Rumpfes eines großen Flugzeugs aus dem zwanzigsten Jahrhundert. Zu Zeiten der Union hatten Züge zumindest auf der Erde fast alle anderen Transportmittel abgelöst.

\par

Der Zug nach Oslo war gerade im Begriff einzufahren. Das Bild, dass sich ihr bot, überraschte Laura immer wieder. Dem Zug war sein hohes Gewicht bereits anzusehen. Trotzdem war er so leise, dass die Gespräche der vielen Menschen seine Laute übertönten. Da der Zug auf Magnetschienen stand, gab es kaum etwas, dass laute Geräusche hätte erzeugen können. Nur das Perpetuum Mobile der Lokomotive brummte gemächlich vor sich hin.

\par

Laura Gethas verstand nicht übermäßig viel von Physik, wusste jedoch, dass Perpetua Mobilia nicht wirklich existierten und es sich dabei lediglich um einen flapsigen Trivialnamen für einen Treystangenerator handelte. Dieser bestand aus einer Spule die sich ständig drehte, was der Vorstellung eines sich selbst mit Energie versorgenden Systems sehr nahe kam~-- sie aber keinesfalls erfüllte.

\par

Die Türen des Zuges öffneten sich und kleine Treppen wurden als Zustiegshilfe ausgefahren, gerade als Laura den Zug erreichte. Zusammen mit etlichen anderen betrat sie den hellweißen Wagon, dessen Eingang ihr am nächsten war. Eine spezielle Auswahl war nicht notwendig, da man mit Sicherheit einen Sitzplatz bekommen würde. In einem Zug zu stehen, der während seiner Fahrt auf mehrfache Schallgeschwindigkeit beschleunigen würde, war Stehenbleiben schlichtweg unmöglich~-- selbst mit den modernsten Trägheitsabsorbern.

\par

Aus Gewohnheit setzte sich Laura in den zweiten Stock des Wagons und schnallte sich an. Vielleicht würde sie während der Fahrt noch ein wenig schlafen können.

\par

Kontrolleure oder Schaffner, die sie hätten wecken können, gab es keine. Freier Personentransport war eine der vier Grundgüter, die von der Union laut Verfassung kostenlos zur Verfügung gestellt wurden. Aber großen Komfort durfte man nicht erwarten.

\par

Kurze Zeit später schlossen sich die Einstiegstüren des Zuges und er begann sich mit einem winzigen Bruchteil seiner Maximalgeschwindigkeit vorwärts zu bewegen. Eine sanfte Computerstimme erklang aus den Lautsprechern, als der Zug den Bahnhof verlassen hatte.

\par

\WR{Guten Morgen, liebe Fahrgäste. Wir hoffen, sie werden eine angenehme Fahrt in diesem Zug haben. Wir möchten Sie daran erinnern, dass es unerlässlich ist, dass sie sich anschnallen, bevor das Fahrzeug in die zweite Beschleunigungsphase eintritt. Beachten Sie bitte, dass das Vehikel sich nicht in Bewegung setzen kann, bevor elektronische Sicherungen bestätigt haben, dass sich jeder Fahrgast angeschnallt auf seinem Platz befindet. Vielen Dank und einen schönen Tag.}

\par

Laura seufzte. Sie verstand, dass es notwendig war, diese Meldung durchzugeben, für den Fall, dass jemand zum ersten mal mit einem Interkontinentalzug fuhr. Aber sie zum scheinbar tausendsten Mal zu hören, begann langsam monoton zu werden. Zumal der Computer des Zuges sich wahrscheinlich in diesem Moment mit ihrem Buch in Verbindung setzte und ihr ein Informationsblatt zur Benutzung von Interkontinentalzügen übermittelte.

\par

Ein Angestellter betrat gerade Lauras Abteil, als der Zug die Stadtgrenze überquert hatte. Er vergewisserte sich, dass jeder dabei war, sich anzuschnallen. Falls sich jemand nicht anschnallen würde, wäre es seine Aufgabe dafür zu sorgen, dass der Entsprechende entweder den Sicherheitsbestimmungen nachkam oder aussteigen musste.

\par

Seine langen Dreadlocks und die verträumten Augen ließen Laura glauben, dass der Mann ein Künstler war, der mit der Anstellung als Zugbegleiter seiner Rechtspflicht auf Arbeit nachkam. Er arbeitete wahrscheinlich nicht in Vollzeit und hatte genügend Zeit, seiner Muse nachzugehen.

\par

Die Geschwindigkeit des Zuges erhöhte sich merklich, als das Vehikel durch die Landschaft des früheren El Salvadors brauste. Die Wagons mussten mittlerweile eine Geschwindigkeit von mehreren hundert Stundenkilometern haben. In etwa sechzig Minuten würde der Zug die Ostküste Mittelamerikas erreicht haben. Der Boden schien sich vom Zug zu entfernen, da die Überlandgleise meistens zehn bis zwanzig Meter über Grund verliefen.

\par

Früher hatte Laura Zugfahren stets genossen. Felder, Städte und Wälder schossen an ihrem Fenster vorbei, während der Zug mit gut dreihundert Kilometern in der Stunde über das weite Land des ehemaligen Honduras raste. Hin und wieder schossen die Wagons durch einen Tunnel, den sie aber nach wenigen Sekunden wieder hinter sich ließen, als sei er gar nicht da gewesen. Die Tunnels wurden zahlreicher, als der Zug in die bergige Landesmitte kam. Jetzt würde es nicht mehr lange dauern, bis Boca Vieja an der Küste erreichen wäre.

\par

Die Fahrt näherte sich damit auch schon ihrem Ende, denn die Überlandfahrt brauchte immer am meisten Zeit. Trotzdem entschloss sich Laura ein wenig Musik zu hören. Sie zückte ihr Buch, ignorierte die neue Nachricht, die der Computer des Zugs ihr gesendet hatte und griff auf ihre Musikdatenbank zu, indem sie nur das Wort Musik auf jene Seite kritzelte, die sie als ihre leere interaktive Seite vorgesehen hatte. Die Sammlung umfasste circa fünfhundert tausend Stücke. Die meisten davon waren Oldies aus dem zwanzigsten und einundzwanzigsten Jahrhundert, für die es kein Urheberrecht mehr gab. Laura hatte vor einigen Wochen ihr Buch instruiert, frei erhältliche Lieder, die ihr zusagen könnten, in ihre Kammer zu laden. Der Krypta Scientia war voll von alter Musik, daher war es kein Problem gewesen.

\par

Einen Augenblick später griff Laura in ihre Manteltasche und zog ihre Kopfhörer heraus. Ohrimplantate boten zwar eine bessere Klangqualität aber zu einem solchen Eingriff hatte sie sich nicht durchringen können. Daher trug sie auch noch eine Brille anstatt sich einer Augenoperation zu unterziehen. Dabei fragte sie sich, ob ihr Unmut gegenüber allem, was mit Kybernetik zu tun hatte, von ihrer klassisch irdischen Erziehung stammte. Wäre sie beispielsweise auf einer jüngeren Welt wie Kreuzpunkt Primus geboren worden, hätten Implantate sicher zum allgegenwärtigen Alltag gehört.

\par

Das erste Lied erklang auch sogleich. Der Text war Englisch. Laura hatte diese alte Sprache~-- die, wie Mandarin, in den autonomen Welten noch teilweise gesprochen wurde~-- zwar in der Schule gelernt, konnte aber trotzdem nicht so ganz verstehen, wovon das Lied handelte. Irgendetwas von Unschuld und ihren Vorzügen.

\par

Wenig später war auch schon Boca Vieja am Horizont zu erkennen. Der Zug würde die Stadt in weniger als einer Minute erreicht haben. Als er das Gebirgbe verließ führte der Schienenverlauf in eine küstennahe Ebene.

\par

Kurz vor der Stadtgrenze von tauchte der Zug unter die Erde ab. Keine Überraschung für Laura, die die Strecke bestens kannte. Sie hatte gewusst, dass der Zug keinen Zwischenstop in der Hafenstadt machen, sondern sie unterfahren würde. Boca Vieja hatte seine eigene Interkontinentalanbindung nach Europa.

\par

Der Licht in der Kabine wurde heller, als die Waggons durch den unterirdischen Tunnel schossen. Laura fühlte sich dabei immer etwas unwohl. Nicht unverständlich, besonders da der Zug nun zum zweiten Mal beschleunigte. Niemand konnte es hören, als der Waggons auf Schallgeschwindigkeit gingen. Dabei half die Tatsache, dass in den Interkontinentaltunneln ein viel geringerer Luftdruck herrschte. Schutzfelder, wie sie auf Raumschiffen zum Einsatz kamen, hielten konstant einen Bruchteil des überirdischen Wertes aufrecht. Der Zug hatte das Festland schnell hinter sich gelassen.

\par

Nun war auch die Aussicht um einiges schöner, wie Laura fand, denn die Wagons schossen nun unter Wasser in einer Röhre aus verstärktem Plexiglas in Richtung Europa. Jenseits dieses Rohrs gab es nichts als den Ozean. Hin und wieder waren Fischschwärme oder Korallenriffe zu erkennen. Jedoch nur für wenige Sekundenbruchteile, denn der Zug hatte mittlerweile auf gute zehntausend Kilometer in der Stunde beschleunigt. Würden nicht starke Trägheitsabsorber, die sonst nur im interstellaren Flugverkehr zum Einsatz kamen, ihre Arbeit tun, wäre die Fahrt mittlerweile sehr unangenehm geworden.

\par

Die Plexiglasröhre, durch die der Zug schoss, war klanglich abgeschottet. Eine absolute Notwendigkeit, denn der Lärm des Zuges hätte sonst das Ökosystem des Ozeans enorm schädigen können. Und wie das gesamte Ökosystem der Erde hatte auch das der Meere schon genug unter der Seuche gelitten.

\par

Die unglaubliche Geschwindigkeit des Zugs und die gemächliche Ruhe einer Unterwasserlandschaft passten zwar nicht zusammen aber trotzdem kannte Laura wenig was sie so faszinierte, wie die Sonne, die in das tiefblaue Wasser hinab scheinen zu sehen. Aber noch um einiges beeindruckender empfand sie die scheinbare Bewegung der Sonne. Da der Zug mit mehrfacher Schallgeschwindigkeit in Richtung Osten raste, schien die Sonne in unglaublichem Tempo aufzusteigen. Innerhalb kürzester Zeit verwandelte sich das morgenrote Wasser in eine hell erleuchtete Landschaft, die von der Sonne durchstrahlt wurde. Es kam Laura vor, als würde der Tag im Zeitraffer ablaufen oder als würde sie der Sonne entgegenfahren.

\par

Ein Blick auf die Uhr zeigte ihr die lokale Zeit, die rasant voran schritt. Bei der Abfahrt hatte das Display noch sieben Uhr morgens angezeigt, nun war es schon bedeutend später.

\par

Plötzlich schien die Sonne wieder unterzugehen. In der nördlichen Hemisphäre war Winter und daher waren die Tage kürzer. An ihrem Zielort Oslo, wäre bei ihrer Ankunft vier Uhr Nachmittag und die Sonne wäre wieder im Begriff, sich zu verabschieden.

\par

Das Gefühl einen Tag komplett ausgelassen zu haben, belastete Laura hin und wieder merklich. Aber sie tröstete sich, dass sie die verlorenen Stunden am Ende der Woche praktisch wieder zurückbekommen würde~-- auch wenn das mit einem nicht unerheblichen Jet-Lag einherging und sie dafür ihre Familie besuchen musste.

\par

Wenig später hatte der Zug die Küste Norwegens erreicht. Trotz des starken Trägheitsausgleichs war das Abbremsen auf Unterschallgeschwindigkeit deutlich zu spüren. Laura wurde einmal mehr klar, weswegen so sehr auf das Anschnallen bestanden wurde. Ohne den Gurt ihres Sitzes hätte sie sich nun wahrscheinlich an der gegenüberliegenden Wand wiedergefunden.

\par

Immer noch mehrere hundert Kilometer in der Stunde schnell, fuhr der Zug am Abhang eines Fjords vorbei, bevor er festes Land unter sich hatte. Wenig später überquerte er eine Brücke, die über ein weiteres Fjord führte. Der plötzliche Ortswechsel lies Laura lächeln. Die Erde hatte so viel Verschiedenes zu bieten. Und alles war nun so einfach zu erreichen.

\par

Die vielen Schneeflocken, die vom Himmel fielen wirkten durch die schnelle Bewegung des Zugs wie weiße Streifen, die kurz am Fenster vorbei rauschten. Noch verstärkt wurde das Gefühl der Geschwindigkeit von den Wolken, die in Gegenrichtung zogen und sich so noch schneller zu bewegen schienen. Hin und wieder huschte ein verschneiter Ort oder eine Stadt am Fenster vorbei. Die Landschaft bot noch vergleichsweise wenig zu sehen. Das gebirgige Land voller Schnee wurde nur manchmal von einem Wald oder einem zugefrorenen See unterbrochen.

\par

Kurz darauf war Oslo am Horizont zu erkennen. Die hellen Lichter der Stadt gingen fast unter im Weiß des Schnees, der überall lag. Hohe Wolkenkratzer, im Stil des späten zweiundzwanzigsten Jahrhunderts machten den Großteil der Innenstadt aus. Sie waren errichtet worden, als Oslo nach der Seuche wieder von Menschen bezogen worden war. Da die Hauptstadt des früheren Norwegens wesentlich früher wieder bewohnt worden war, als San Salvador, fanden sich viel weniger Gebäude, die im gläsernen Stil der Union gebaut worden waren. Diese waren eher in den Vororten zu sehen, die sich wie die Wurzeln einer Pflanze immer weiter von der Hauptstadt auszustrecken schienen.

\par

Oslo selbst war eine der größten Metropolen der Welt. Seit dem einundzwanzigsten Jahrhundert hatte sich die Fläche, welche die Stadt einnahm, fast vervierfacht. Die neuen Stadtgebiete hatten sich hauptsächlich nach Norden und nach Osten erstreckt. Südlich der Stadt nahm ein Raumhafen fast den Platz einer gesamten Insel ein. Fähren, die durch das rotgelbe Leuchten ihrer Tiebwerke zu erkennen waren, starteten und landeten, aus der Ferne betrachtet, scheinbar im Meer. Die Insel auf welcher der Raumhafen stand, war von Lauras Standpunkt aus nur schwer zu erkennen. Einige, kleinere Wolken schienen rot zu leuchte, wann immer eine Fähre sie durchflog.

\par

Ein paar Minuten später hatte der Zug die Stadtgrenze erreicht. Ab dort schirmte eine Röhre den Zug von der Stadt ab. Als Lärmschutz und um Unfälle zu verhindern. Der Zug wurde nun noch einmal merklich langsamer. Der Zielbahnhof war nah und die Wagons bewegten sich nun nicht schneller als solche aus dem späten zwanzigsten Jahrhundert.

\par

Das gab Laura Gelegenheit, sich die Stadt anzusehen. Sie ließ ihren Blick nicht zum ersten mal über die verschneiten Häuser schweifen, die in den unterschiedlichsten Baustilen errichtet worden waren. Aber selbst an diesem Tag erkannte sie immer wieder Neues.

\par

Als der Zug tiefer in die Stadt eingefahren war, schienen sich die Häuserfronten bis hinter den Horizont zu erstrecken. Dieser Anblick wurde nur einige Male durch Parkanlagen und Stadien unterbrochen.

\par

Dann hatte der Zug den Bahnhof erreicht. Die Anlage war noch um einiges größer als ihr Gegenstück in San Salvador. Drei große Haupthallen für den Interkontinentalverkehr und mehrere kleinere Nebengebäude für den Nahverkehr nahmen etwa die Fläche einer Kleinstadt ein. Der Bahnhof war sogar noch umfangreicher als der Oslo-Raumhafen.

\par

\WR{Die Fahrt ist in wenigen Augenblicken beendet}, verkündete die immersanfte Stimme des Computers. \WR{Sie können nun die Anschnallgurte lösen. Bitte beachten Sie, dass Sie den Zug durch die Türen am Ende und am Anfang jedes Abteils verlassen können.}

\par

Laura seufzte und stand auf. Bald würde die Ruhe der Zugfahrt durch die Hektik der Großstadt abgelöst werden. Das es wirklich so war, wurde ihr bewusst, als sie den Zug verließ und sich in den anonymen Menschenstrom einreihte, der sich stetig auf die Rolltreppe zur Unterführung zu bewegte. Lärm und starre Gesichter hüllten Laura von überall her ein.

\par

Erst als sie in der Unterführung angekommen war, löste sich der Strom einigermaßen auf. Doch hier begann das nächste Problem. Die Unterführungen waren, aufgrund der enormen Größe des Bahnhofs, mehr ein unterirdisches Labyrinth. Und obwohl Laura den Weg schon viele Male gegangen war, konnte sie sich nicht immer sofort erinnern, wo sie nun eigentlich lang musste. Zumal der Zug häufig an unterschiedlichen Gleisen hielt.

\par

Gemächlich zog sie ihr Buch aus der Innentasche ihres Mantels und ließ es sich mit dem Rechner des Bahnhofs verbinden. Nachdem sie die Anfrage nach dem Weg eingegeben hatte, zeichnete die intelligente Tinte scheinbar wie von Geisterhand einen Lageplan und ihre eigene Position ein. Ein langer grüner Strich wies ihr den Weg.

\par

Nach zehn Minuten, in denen sie sich mittels Laufbänder durch die Unterführungen bewegt hatte, erreichte sie die Fahrstühle zur Haupthalle eins. Sie war froh, aus den Katakomben heraus zu kommen. Trotz der hellen Beleuchtung und der freundlichen Gestaltung empfand sie die Umgebung immer als bedrückend. Außerdem dröhnten es in den Gängen jedes Mal unangenehm, wann immer ein Zug über die Unterführung fuhr. Das Geräusch war zwar nicht wirklich laut aber trotzdem sehr intensiv.

\par

In der Haupthalle angekommen kämpfte sich Laura durch die Menschenmassen zu einer der sechzehn Türen und verließ das Gebäude. Auf der Straße stehend kam sie nicht umhin sich umzudrehen und die Größe der riesigen Halle zu bestaunen. Das Gebäude war so groß, dass gut zehn Stockwerke darin Platz hatten. Aus einer Informationszeitschrift wusste Laura, dass der gesamte Bahnhof über fünfhundert kleinere Geschäfte beherbergte, die fast alle einen reißenden Umsatz machten.

\par

Schließlich wand sie sich jedoch ab und ging durch die Straßen der Stadt. Zur ihren Seiten erhoben sich große Wolkenkratzer in die Höhe, deren oberste Stockwerke durch den stärker werdenden Schneefall kaum noch zu erkennen waren. Tatsächlich war die Schneedecke abseits der Straßen und auf den Dächern der Häuser schon einen halben Meter hoch. Die Straße selbst war, wie viele andere begehbare Gebiete, mit einer Substanz behandelt, sodass sich kein Schnee absetzen konnte. Daher waren die Wege ständig feucht. Das Schmelzwasser lief in den Untergrund ab, wurde dort gespeichert und trinkbar gemacht.

\par

Laura konnte mitten auf der Straßen gehen. Autos gab es zur Zeiten der Union keine mehr. In den ländlicheren Gebieten waren oft ihre Nachfolger, die Hovercrafts, zu finden. Aber in den Innengebieten von größeren Städten wurden diese nur von der Polizei und den Rettungskräften verwendet. Für andere Anwendungen waren sie nicht gestattet.

\par

Die Untergrundbahnen boten eine gute Alternative aber Laura wollte unbedingt noch ein wenig frische Luft schnappen, bevor sie ins Büro ging.

\par

Nach gut dreißig Minuten Fußmarsch hatte sie ihre Arbeitsstelle erreicht. Das Geheimdienstzentrum der Erde war zwar um einiges kleiner als der Bahnhof aber immer noch ein recht imposantes Gebäude. Auch es war im Glasstil errichtet worden, hatte aber in den niedrigeren Stockwerken aus Sicherheitsgründen Mauern anstatt Fensterfronten. Auf den höheren Ebenen waren einige der Fenster abgedunkelt. Ermöglicht durch eine Technik, die durchsichtiges Glas innerhalb kurzer Zeit in etwas verwandelte, dass wie purpur schimmerndes Metall wirkte. Auch das Tor des Haupteingangs war nicht durchsichtig sondern bestand aus milchigem Panzerglas.

\par

Laura schritt über den Platz vor dem Geheimdienstzentrum. Der Boden vor dem Eingang war mit Granit bepflastert. Das Wappen der Union~-- ein Kreis mit einem zwei mal gegabelten Zweig darin~-- war in die Steinfläche eingelassen.

\par

Als Laura vor dem Eingangstor Halt machte, öffnete es sich nicht automatisch. Um in das Gebäude eintreten zu können, musste sie sich zunächst identifizieren. Seufzend legte sie ihren Daumen auf eine gekennzeichnete Fläche des Eingangsterminals. Im Bruchteil einer Sekunde schoss eine winzige Nadel aus dem Terminal heraus und entnahm Lauras Finger eine Blutprobe. Anschließend besprühte eine Düse den kaum verletzten Daumen mit einer Substanz, die über der Wunde eine Membran bildete und sie so verschloss.

\par

Zu spüren war davon wenig. Der ganze Vorgang war so schnell abgelaufen, dass Lauras Nervenzellen keine Chance zu einer Reaktion gehabt hatte.

\par

Kurz darauf gab das Terminal den Durchgang frei und Laura betrat das Gebäude. Der Empfangssaal war überschaubar klein. Am anderen Ende der Halle war eine kleine Sitzgruppe aufgebaut, hinter der ein Brunnen mit abstrakten Plastiken stand. Ein großer, altertümlich wirkender Schlüssel~-- das Emblem des Geheimdienstes~-- zierte den Marmorboden. Zu Lauras Linken befand sich ein Schalter. Der Offizier dahinter warf ihr einen kurzen Blick zu, grüßte sie mit einem Winken der Hand und kümmerte sich dann wieder um seinen Computer. Rechts ging es zu den Treppen und den Aufzügen.

\par

\WR{Ihr Name, ihr Geburtstag und ihren Dienstausweis bitte}, erklang eine Laura sehr bekannte Stimme. Sie gehörte einem Soldaten der Phalanx, der zusammen mit einem Partner, den Eingangsbereich bewachte. Der Geheimdienst hatte keine eigenen Wachen. Der Sinn und Zweck dahinter war, die einzelnen Bestandteile des Verteidigungskonglomerats voneinander abhängig zu machen und so dafür zu sorgen, dass sich die Institutionen gegenseitig kontrollieren konnten.

\par

Laura griff in ihre Tasche und übereichte dem Soldaten ihr Buch, aufgeschlagen auf einer Seite, die ihre Anstellung in der Argus-Abteilung nachwies.

\par

\WR{Vizeberaterin Laura Ghetas. Geboren vierdin siebendutzend sieben. Tag eindin siebendutzend acht}, sagte die Agentin wie jeden Tag auf. Beim ersten Mal hatte sie ihr Buch zurate ziehen müssen, um den zwanzigsten Februar zweitausend sechshundert sechsundsechzig ins Standarddatum der Union umzuwandeln. Mittlerweile kannte sie die Zahlen auswendig.

\par

Natürlich kannte der Wachmann sie bereits persönlich. Aber die Bestimmungen waren äußerst strikt. Jeder musste kontrolliert werden. Und die Agenten in höheren Positionen wurden sogar noch gründlicher überprüft.

\par

Mit dem Rang eines Vizeberaters stand Laura auf den mittleren Sprossen der Karriereleiter. Ihre Stellung entsprach der eines ersten Offiziers bei der Starforce. Allerdings waren Beförderungen beim Geheimdienst etwas leichter zu erreichen, als bei den anderen Institutionen des Konglomerats, da Verhaftungen von Kriminellen deutlich häufiger vorkamen als tatsächliche Gefechte irgendeiner Art. So mussten Polizisten und Agenten oft nicht die offiziellen Zeitintervalle für einen Karriereaufstieg abwarten.

\par

Laura begab sich die Treppe hinauf. Ihr Büro lag gleich im ersten Stock. Der Ausblick war zwar etwas mager aber dafür musste sie nicht weit laufen. Abgesehen von dem ganzen Weg, den sie jeden Morgen zur Arbeit ging.

\par

An der Tür zu ihrem Büro gab Laura ihren persönlichen Sicherheitscode zur Überprüfung ein und wurde dann vom Computer, per blinkender Textnachricht auf einem Display, aufgefordert, ihr Buch gegen den Scanner zu halten. Als zweite Identitätsüberprüfung suchte sich ein semiintelligentes Programm per Zufallsprinzip zusätzlich noch eine weitere Methode aus. An diesem Tag entschied es sich für eine Abtastung ihrer Retina.

\par

Nachdem diese abgeschlossen war, öffneten sich die Türen ihres Büros. Es war nicht wirklich groß aber sie hatte sich bislang darin immer wohl gefühlt. Etliche Zimmerpflanzen und viele persönliche Bilder sorgten für sie für eine angenehme, weniger sterile Atmosphäre.

\par

Auf ihrem Holzschreibtisch, den sie bewusst ausgewählt hatte, türmte sich der übliche Papierkram. Es erschien Laura manchmal ein wenig überflüssig, dass sie sich ihre Akten ausdruckte, obwohl eigentlich alles mit Hilfe ihres Buches erledigt werden konnte. Andererseits fand sie um einiges komfortabler, Informationen auf einem Blatt Papier wirklich physikalisch vor sich zu haben, anstatt sie sich immer wieder durch intelligente Tinte aufschreiben zu lassen.

\par

Außerdem war Papier ein leicht herzustellendes Gut. Besonders da es synthetisch produziert werden konnte ohne dafür, wie früher, Wälder abholzen zu müssen. Und sie hatte damit eine Ausrede parat, mit dem Füller schreiben zu können, den sie zu ihrem zwölften Geburtstag geschenkt bekommen hatte.

\par

Um den Wust an Formularen und Bescheinigungen würde sie sich später kümmern. Zuerst setzte sie sich an ihren Schreibtisch und startete ihr Computerterminal. Einen Augenblick später begrüßte sie die Benutzeroberfläche des semiintelligenten Programms Isechion drei, das die Grundlage aller Computer des Konglomerats darstellte. Wesentliche Unterschiede zu Computerprogrammen der Zeit vor der Seuche gab es nur ein paar. Isechion drei war lernfähig und konnte sich an gewisse Situationen anpassen. Außerdem konnte es für nahezu alle Aufgaben benutzt werden, die einem Nutzer einfielen, ohne dazu von außen mit neuen Programmen bestückt zu werden. Es generierte eigenständig völlig neue Anwendungen, die auf die Bedürfnisse des Benutzers zugeschnitten waren, konnte also seinen eigenen Code umschreiben und ergänzen.

\par

Von richtiger künstlicher Intelligenz waren die Entwickler der Union allerdings noch weit entfernt. Obwohl die Computer dieser Zeit schon nicht mehr im Binärsystem operierten, hatte es noch niemand geschafft, das Gehirn eines Menschen digital umzusetzen.

\par

Allerdings erweckte der wirtschaftliche Hauptcomputer des Ökonomierats, genannt Gnosis, hin und wieder den Eindruck er wäre ein denkendes Wesen. Er war in der Lage, verbal mit seinen Benutzern zu kommunizieren. Aber in Wirklichkeit wurde auch er nur von einem hoch entwickelten, lernfähigen Programm gesteuert, dass allerdings in der Lage war, nicht nur vordefinierte Abläufe auszuführen, sondern philosophische Grundlagen zur Entscheidungsfindung heranzuziehen.

\par

Laura war ganz froh, dass es noch keine echte künstliche Intelligenz gab. Sie empfand das menschliche Denken als etwas erhabenes, das nicht einfach nachgebaut und dupliziert werden konnte. Würde es irgendwann künstlich nachgestellt werden können, dann fiele ein weiterer Aspekt weg, der Menschen zu etwas Besonderem machte. 

\par

Gerade als sie anfangen wollte, sich um ihren Rechner zu kümmern, hörte sie Schritte auf dem Gang, die sie nur allzu gut kannte. Sie gehörten ihrem Vorgesetzten.

\par

Berater Terence O’Shea kam kurz darauf durch die Tür zu Lauras Büro spaziert. Wie sie sofort erkannte, hatte sich der mittlerweile fünfundsechzig Jahre alte Leiter ihrer Abteilung am Wochenende die Haare schneiden lassen. Auch wenn die graue Farbe auf das wirkliche Alter seiner Frisur hindeutete, hatte O’Shea noch einen relativ ausgeprägten Haarwuchs und wirkte auch sonst jugendlich.

\par

Auch dem Gesicht des Mannes sah man sein Alter nicht wirklich an. Er machte einen agilen und aufgeweckten Eindruck, der auch nicht täuschte.

\par

\WR{Hallo, Frau Ghetas. Wie war ihr Wochenende?}, fragte er, nachdem er in Lauras Büro angekommen war. Er bemühte sich, wie immer, unverbindlich und locker zu klingen und sprach wie immer zunächst über Belanglosigkeiten, bevor er zum Wesentlichen kam.

\par

Laura lehnte sich in ihren Sessel zurück. \WR{Schön}, log sie. \WR{Und bei Ihnen? Ich sehe, Sie haben sich eine neue Frisur zugelegt.} Sie hoffte mit dieser Nachfrage von ihrer eigenen Situation ablenken zu können.

\par

Der Abteilungsleiter lächelte ehrlich und fuhr sich durch die Haare. \WR{Ja, stimmt. Ich dachte, es wäre mal Zeit für etwas frischen Wind.}

\par

\WR{Passt gut zu Ihrem Gehrock}, kommentierte Laura.

\par

O’Shea nickte zufrieden und nahm sich einen Stuhl, um seiner Untergebenen gegenüber zu sitzen. Laura hatte es ihm sich schon vor Jahren frei gestellt, sich in ihrem Büro wie zuhause zu fühlen.

\par

Etwas ernster begann er zu sprechen: \WR{Ich habe eine gute und eine schlechte Nachricht für sie, Frau Ghetas. Welche möchten sie zuerst hören?}

\par

\WR{Die gute natürlich}, antwortete Laura sofort, wie sie es schon oft getan hatte.

\par

\WR{Sie werden sich eine Weile lang nicht mit Papierstößen befassen müssen}, begann O’Shea zu erklären. \WR{Dafür gibt’s eine Menge anderer Arbeit. Ich hab einen neuen Fall für Sie. Könnte interessant werden.}

\par

Laura hob beide Augenbrauen. \WR{Sie haben meine volle Aufmerksamkeit.}

\par

O'Shea überreichte seiner Mitarbeiterin einen Datenträger der einem magnetischen Speichermedium aus dem zwanzigsten Jahrhundert ziemlich ähnlich sah. Solche Disketten waren aber nur deshalb so groß, weil sie eine Menge mechanischer Sicherungen gegen ungewollten Zugriff beinhalteten.

\par

\WR{Es geht um einen Hacker oder eine Hackerin}, führte der Lauras Vorgesetzter aus. \WR{Er klinkt sich in fremde Kommunikations-Benutzerprofile ein, um sich illegale Programme zu besorgen und geheime Daten zu stehlen.}

\par

Verwunderung machte sich in Lauras Gesichtsausdruck breit. \WR{Wäre das nicht eher ein Fall für die E-Division der Polizei?}

\par

O'Shea nickte. \WR{Eigentlich schon. Ich denke die hätten den Fall auch gerne übernommen. Aber der Verdächtige hat versucht, ins Zentralsystem des Konglomerats einzudringen. Das macht es zu einer Sache das Geheimdienstes. Unsere Korrespondenzeinrichtung auf Kreuzpunkt Primus hat das ganze Geschehen vor ein paar Stunden gemeldet. Dienstakten und Versetzungspläne haben ihn anscheinend besonders interessiert. Auf der Diskette finden Sie alles was wir bisher wissen. Sie ist nur auf Ihre Biometrik eingestellt, also können nur Sie das Material einsehen.}

\par

\WR{Klingt interessant}, gestand Laura ein. \WR{Werde ich wieder mit Inspektor Rensing zusammenarbeiten?}

\par

Berater O’Shea nickte erneut. \WR{Ja. Er wird auch diesmal wieder ihr Partner sein. Er ist schon auf dem Weg von Paris hier her. Müsste eigentlich spätestens in einer halben Stunde ankommen.}

\par

\WR{In Ordnung. Wir kümmern uns um den Kerl}, sagte Laura, froh darüber nicht schon wieder ermüdende Papierarbeit leisten zu müssen. Diese war geradezu prädestiniert dafür, dass sie beim lesen abschweifte und an Dinge dachte, die sie zumindest unter der Woche gerne vergaß.

\par

O’Shea verabschiedete sich und ging. Laura war froh darüber, dass sie wieder mit Klaus Rensing zusammenarbeiten konnte. Der Inspektor arbeitete schon einige Zeit beim Außendienst der Polizei. Laura hatte schon ein paar Fälle gemeinsam mit ihm gelöst.

\par

Ein Gesetz der Union bezüglich des Konglomerats setzte voraus, dass bei Ermittlungen des Geheimdienstes, die Sicherheitskräfte immer mit involviert sein mussten. Damit sollte verhindert werden, dass die Aktionen des Geheimdienstes zur undurchschaubaren Verschlusssache wurde. Meistens arbeiteten daher immer ein Agent und ein Ermittler von der Polizei gemeinsam an einem Fall. Anfangs war Laura nicht begeistert darüber gewesen. Sie hatte befürchtet, ihr Partner würde ihr nur über die Schulter sehen, ihr Ermittlungen behindern und am Ende den ganzen Lohn alleine einstreichen. Aber Inspektor Klaus Rensing hatte sie positiv überrascht.

\par

Demonstrativ schob Laura den Papierstapel auf die Seite.