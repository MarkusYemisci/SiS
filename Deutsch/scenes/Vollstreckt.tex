Dilara Bashir hatte immer eine sehr gesunde Hautfarbe gehabt, was Morten ganz besonders an ihr gefallen hatte. Das, und ihre braunen Augen, die von Lebensfreude nur so strahlten.

\par

Nun war beides vergangen. Sie wirkte totenbleich und ihre Pupillen bewegten sich zwar noch, schienen aber längst nichts mehr zu sehen.

\par

Morten hätte sich gerne über ihr Krankenbett gebeugt, um zumindest zu versuchen, ihren Blick einzufangen. Doch er hatte Angst, das Gleichgewicht zu verlieren. Die Handschellen taten das ihre, doch eigentlich war es Funkens Anblick, der ihn schwanken ließ.

\par

Die Ärzte der \EN{Regenvogel} hatten sie alleine auf ihre Behandlungsbahre gelassen. Nach der Schlacht hatte es zahllose Verletzte gegeben, um die sie sich nun nach allen Kräften kümmern mussten. Menschen, die noch um ihr Leben kämpften.

\par

Dilara stand kurz vor ihrer letzte unvermeidlichen Niederlage.

\par

Morten schluckte seine Übelkeit hinunter. Das Lazarett der \EN{Regenvogel} stank gleichermaßen nach Desinfektionsmitteln und Verwesung. So, als hauche des Schiff selbst sein letztes Leben aus. Der Monitor über zeigte, wie ihr Herzschlag immer schwächer und unregelmäßiger wurde.

\par

\WR{Morty, bist du das?}, fragte sie, kaum hörbar.

\par

Er stürzte zu ihrem Bett. Der Wachmann trat heran, um ihn zu stützen. Sein Blick sprach Bände über das, was er gerne mit Mortens Fesseln gemacht hätte.

\par

\WR{Ich bin da, Funken}, sagte er nur. \WR{Es tut mir so leid. Ich…}, er stockte, \WR{ich weiß nicht, was ich tun kann.}

\par

Ein Zucken ihres Mundwinkels zeugte davon, dass sie versuchte, zu lächeln. Doch selbst ihre vollen Lippen hatte das Leben verlassen. \WR{Schon okay. Ich war immer gerne bei euch Piloten. Ich dachte, ich hätte nur noch ein paar Minuten. Aber ich darf bei euch gehen.}

\par

\WR{Du kannst nicht gehen!}, forderte Morten wütend. \WR{Du hast mir einen Drink versprochen!}

\par

Nun lächelte Funken tatsächlich ein ganz klein wenig. \WR{Ich hab's gehört. Ihr habt die Shutek besiegt. Trink einen für mich mit, okay?}

\par

Die Tränen rannen nur so über sein Gesicht. Hätte er nun in Kevins Richtung gesehen, hätte er auch in dessen Gesicht dieselbe Hilflosigkeit erkannt, die nun in sein eigenes geschrieben stand.

\par

\WR{Versprich mir}, bat sie, \WR{komm nach Hause. Du… Ihr müsst…} Dilara klang nun so leise, dass sie kaum noch zu verstehen war. Ihr Mund bewegte sich noch, aber die Anzeigen über ihrem Kopf zeigte, wie ihr Blutdruck absank und ihr Herzschlag durch eine nahezu flache Linie dargestellt wurde. Die mordernen Medikamente und Geräte eines Feldlazaretts der Union erlaubten einen sehr friedvollen Übergang. Keine Krämpfe und kaum Schermzen.

\par

\WR{Scheiß drauf}, fluchte der Wachman, der nach wie vor hinter den beiden Piloten stand. Mit seinem Fingerabdruck und einem Schlüsselchip öffnete er Mortens Handschellen.

\par

Dieser schüttelte sie ungelenk ab und Griff nach Dilaras Hand. Ob sie ihn noch spüren konnte, wusste er nicht. Ihre Wimpern zuckten und noch ging ihr Atem flach, als er sich über sie beugten.

\par

\WR{Bitte!}, wimmerte er. \WR{Bitte nicht!}

\par

Auch Kevins Handschellen fielen krachend auf die Fliesen und er berührte kräftig Mortens Schultern. Ein leises Piepen öffnete die Schleusen für weitere Tränen. Der Monitor zeigte Dilara Bashirs offizielles Ende an. Reduzierte sie auf ein paar medizinische Parameter, die alle auf Null gesunken waren und einen leblosen Leib.

\par

Morten registrierte kaum, wie er, genauso wie Kevin mit neuerlich angelegten Handschellen aus der Krankenstation geführt wurden. Nur nur dumpf nahm er die Schreie war. Ausgestoßen von Menschen, die schwer verletzt waren oder selbst Freunde verloren hatten. Er stolperte einfach vorwärts, ohne zu wissen, wohin er eigentlich ging.