\WR{Reg dich ab}, forderte Kevin Wilson und schenkte Morten einen weiteren Schnaps ein. Morten bewunderte dabei, wie schnell sich sein Stubenkamerad an Bord der \EN{Regenvogel} eingelebt hatte. Er bediente sich an der Bar, als stünde sie in seinem eigenen Wohnzimmer. Auch die Gläser fand er sofort. Doch die eigentlichen Anzeichen dafür, dass sich Kevin bereits wie zu Hause fühlte, waren subtiler. Die Art, wie er sich durch das Schiff bewegte und mit den Leuten an Bord sprach. So, als hätte er noch niemals irgendwo anders gelebt.

\par

\WR{Soll das ein Witz sein?}, fragte Morten an diesem Tag bereits zum zweiten mal. \WR{Dieser Kerl hat auf Schiffe der Union gefeuert.}

\par

\WR{Vor Ewigkeiten}, entgegnete ihm Kevin deutlich leiser und sah dabei verstohlen zu den Nachbartischen. Für einen kurzen Moment hatten viele der anderen Piloten ihre Blicke vom beeindruckenden, weil riesenhaften, Raumdock abgewandt und sie dem lauten Morten Witwer zugeworfen.

\par

Dieser wollte nicht weiter auffallen und antwortete daher nichts mehr. Doch Kevin führte das Thema fort. \WR{Weißt du, ich kann schon verstehen, warum damals viele nichts mehr von der Union wissen wollten. Nach dem Routenkrieg ging es erst mal darum, die Kolonien aus der Armut zu holen. Manche sind dabei besser weggekommen. Andere schlechter.}

\par

Morten sah in sein Glas. \WR{Du weißt, dass ich von Corna stamme? Mach dir also erst mal klar, mit wem du hier sprichst.}

\par

\WR{Dann müsstest du doch am besten verstehen, warum die Union nicht immer beliebt war}, entgegnete Kevin nach einem Seufzen. \WR{Der \EN{Permutare} wurde eingerichtet, um die beste Verteilung für alle Bürger zu berechnen. Aber nicht jede Kolonialregierung war einverstanden. Bis der Wohlstand erst mal bei allen ankam, hat es gedauert. Und selbst, wenn man als erstes drankam, fühlten sich die Leute oft bevormundet.}

\par

Morten musste mit sich ringen, um nicht zu laut zu werden. \WR{Das ist kein Grund, zum Terroristen zu werden!}

\par

\WR{Wer weiß, welche Gründe er hatte. Ich hörte, die Captial Fellowship hat viele ihrer Leute dazu gezwungen, für sie zu kämpfen. Andere hat sie einfach verarscht.} Kevin klang uncharakteristischer weise so, als wüsste er tatsächlich, wovon er sprach. \WR{Ich finde nur, wir sollten, Nico nicht gleich verdammen. Wir sind keine Richter.}

\par

\WR{Wir?}, fragte Morten.

\par

\WR{Ja. \EN{Wir}. Wir sind doch Zimmergenossen und Flügelmänner. Wann fangen wir endlich an, uns so zu verhalten?}

\par

Morten fiel es leichter, als er erwartet hätte, sich ein Lächeln abzuringen. \WR{Wir verhalten uns wie Flügelmänner, seid du mir über Pollux Primus die Haut gerettet hast. Das Manöver war ziemlich beeindruckend.}

\par

\WR{Herzlichen Dank}, entgegnete Kevin. Sein Stolz war kaum zu überhören. \WR{Wer weiß, vielleicht lässt man \EN{mich} auch mal wieder zurück ins Cockpit, wenn schon alte Freiheitskämpfer wieder fliegen dürfen. Dann kannst du dich vielleicht revanchieren.}

\par

Morten hatte nicht richtig zugehört. Seine Gedanken ging immer wieder zu Nico Cusiosa zurück. \WR{Ich weiß nicht}, sagte er zerstreut. \WR{Ich verstehe einfach nicht, wie irgendwer, diesem Kerl vertrauen soll.}

\par

\WR{So wie ich hörte, hat er eine Frau und zwei Kinder}, gab Kevin zu bedenken. Doch auch diese Antwort bekam Morten nicht richtig mit. Es half seiner Aufmerksamkeit auch nicht, dass einer der Offiziere am Nachbartisch die Lautstärke seiner DDV-Übertragung immer höher drehte. Der holographische Projektor im Buch des Piloten zeigte eine Nachrichtenübertragung.

\par

\WR{Nach dem Rücktritt von Präsident Henry Otis füllt nun seine Stellvertreterin die Position aus}, verlas die Sprecherin. \WR{Über die Gründe des plötzlichen Rücktritts wurden seitens des Präsidialbüros vorerst keine Angaben gemacht. Experten glauben jedoch, dass die Amtsniederlegung mit dem Tod des Neffen des Präsidenten zusammenhängt.} Schnell war der Tisch mit dem holographisch projizierenden Buch zum Zentrum der Pilotenlounge geworden. \WR{Die erste Amtshandlung der neuen Präsidentin war es, das gesamte Konglomerat in maximale Alarmbereitschaft zu versetzen. Sämtliches Personal, dass sich derzeit im Urlaub befindet, wurde mit sofortiger Wirkung in den Dienst zurückberufen. Auch etliche Offiziere in Pension wurden bereits eingezogen.}

\par

Die Nachrichtensprecherin wirkte nur kurz verwirrt, als sie offenbar Anweisungen aus der Regie bekam und fortfuhr: \WR{Wir haben soeben erfahren, dass Präsidentin Aktintola zu einer Ansprache aus dem Präsidialbüro geladen hat. Wir schalten nun in Echtzeit zur Übertragung dorthin.}

\par

Morten hatte Henry Otis Büro schon einige male gesehen. Der Raum, in dem die neue Präsidentin nun stand, war definitiv ein anderer. Vermutlich hatte man es nicht geschafft, das bisherige Amtszimmer rechtzeitig freizuräumen. Nun stand Lertha Akintola vor einem weißen Vorhang, auf den das Emblem der Union aufgestickt war.

\par

\WR{Meine Herren und Damen. Ich werde mich kurz fassen}, begann die Präsidentin, die ohnehin nicht für Umschweife bekannt war. \WR{Es tut mir sehr leid, ihnen berichten zu müssen, dass alle Bemühungen, eine friedliche Kommunikation mit dem außerirdischen Volk, dass wir mittlerweile als \Wr{Shutek} bezeichnen, nicht möglich war. Es ist daher meine traurige Pflicht, sie darüber zu informieren, dass wir uns seit heute Morgen um sechs Uhr~-- Unionsstandard~--offiziell im Kriegszustand befinden.}

\par

Selbst durch die DDV-Übertragung hinweg war das Raunen, dass durch die Reihen der Reporter ging, nicht zu überhören.

\par

\WR{Jetzt ist nicht die Zeit, um über Fehler zu sprechen, die vielleicht gemacht wurden}, fuhr die Präsidentin fort. \WR{Mehr als je zuvor sind wir nun in unserer Einigkeit gefragt. Seit heute Morgen gibt es keine Erden- oder Kreuzbürger mehr. Auch keine Unionsbürger oder Angehörige der autonomen Welten. Ab heute gibt es nur noch \EN{eine} Menschheit.}

\par

Akintola nahm einen Blick auf ihren Aufschrieb und fuhr dann fort: \WR{Ich habe niemals einen Hehl daraus gemacht, keine große Rednerin zu sein. Viele Männer und Frauen die vor mir kamen und die nach mir kommen werden, haben bessere Worte gefunden. Ich möchte daher jemanden zitieren, der vor der Seuche gelebt haben muss. Historiker sind sich nicht sicher, in welchem Kontext dieser Mann, der wohl ein einflussreicher Politiker war, diese Sätze sprach. Doch sie erscheinen mir heute genauso passend, wie sie es wohl damals waren.}

\par

Jeder Blick ging nun zu dem Hologramm der Präsidentin. \WR{\Wr{Menschheit}. Dieses Wort sollte heute eine neue Bedeutung für uns alle haben. Wir können uns nicht länger von unseren kleinlichen Differenzen verzehren lassen. Unser gemeinsames Ziel wird uns verbinden. Wir werden nicht still in der Nacht verschwinden. Wir werden nicht untergehen, ohne zu kämpfen. Wir werden weiterleben. Wir werden \EN{überleben}. Vielen Dank.}

\par

Mit diesen Worten verließ Lertha Akintola das Rednerpult, ohne sich um die zahllosen Fragen zu kümmern, die ihr die Anwesenden Reporter nun nachriefen. An Bord der \EN{Regenvogel} hatten ihre Worte die gewünschte Wirkung jedoch nicht verfehlt. Die Pilotenlounge ging in lautem Beifall unter und Kevin quittierte die Übertragung mit einem zackigen Salut.