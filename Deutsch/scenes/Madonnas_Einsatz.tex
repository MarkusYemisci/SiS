\WR{You look a little sad, boy, I wonder why}, dröhnte es mit einem mal durch den Staffelkanal. Morten sah auf seinen Kommunikationsmonitor, brauchte aber gar nicht richtig hinzusehen, um zu wissen, dass die Übertragung von Kevin kam.

\par

\WR{Mach das aus, du Geisteskranker!}, rief er ihm zu. Instinktiv suchten seine Augen nach dem Jäger seines Freundes. Aber alles, was er erkannte, war Nicos Falken, der mit voll zugeschalteten Nachbrennern auf den Triumphbogen zu raste. Dicht gefolgt von zwei Jägern der Shutek, die vermutlich nur deshalb nicht feuerten, weil sie befürchten mussten, ihr Ziel zu verfehlen, und den riesigen Metallring zu treffen.

\par

\WR{I follow you around but you can't see}, war die nächste Zeile, des altertümlichen Liedes, die nun auf dem Gefechtskanal zu hören war. \WR{Kaum zu glauben, dass die Bordcomputer dieser Dinger \textit{Musik} gespeichert haben!} Kevin klang euphorisch.

\par

Morten wollte sich weiter beschweren, musste sich aber auf das Steuern seines Jägers konzentrieren. Im Simulator war er schon zig male einen Rapier geflogen. Der experimentelle Jäger galt als der heilige Gral unter den Fliegern der Union. Mit solch komplexer Technologie er vollgepfropft war, so komplex steuerte er sich jedoch auch.

\par

Eigentlich boten die klobigen Flügel, aus denen bereits die Köpfe der zahlreichen Raketen des Arsenals herausragten, viel zu viel Luftwiderstand, als dass der Rapier innerhalb einer Atmosphäre geschmeidig manövrieren konnte. Dies ließ sich über den Antigraviationsantrieb regeln. Allerdings steuerte sich dieser nicht automatisch.

\par

Während er noch mit der richtigen Flughöhe kämpfte, bemühte er sich, den linken der beiden Jäger in der Nähe des Fadenkreuzes zu behalten, der Nico verfolgte. Eine seiner Raketen führte bereits eine Zielpeilung aus.

\par

\WR{Too wrapped up in yourself to notice}

\par

Auch wenn ihn die dröhnende Musik eher störte, fand er den Text seltsam passend.

\par

\WR{So you choose to look the other way

\par

Well, I've got something to say

\par

Don't try to run I can keep up with you}

\par

Die Rakete hatte ihr Ziel erfasst und Mortens Fadenkreuz leuchtete in einem satten Rot.

\par

\WR{Hab mein. Du?}, fragte er Kevin über Funk. Dieser schaltete endlich seine Musikwiedergabe aus und antwortete: \WR{Geladen und feuerbereit. Ich bin echt froh, dass wir feuern können, während unsere Tarnkappen aktiv sind.}

\par

\WR{Warum sollten wir das nicht können?}, fragte Morten verwundert. \WR{Feuer frei!}

\par

Und schon lösten sich zwei Raketen voller Capezin von den Tragflächen der beiden Jäger und schossen auf ihre Ziele zu. Morten kniff vor Ärger die Augen zusammen, als beide Geschosse denselben Shutek-Flieger trafen und ihn in einen brennenden Feuerball verwandelten. \WR{Verdammt, ich sagte, ich nehme mir den an Backbord vor!}, rief er verärgert in sein Headset.

\par

\WR{Tut mir leid!}, antwortete Kevin sofort. \WR{Aber da bin ich in Rückenlage geflogen.}

\par

Morten stöhnte. \WR{Vollidiot. Lösen und angreifen!}

\par

Der Kondensator für die Tarnkappe seines Jägers hatte beinahe die Nullmarke erreicht. Morten schaltete sie gänzlich aus, um mehr Energie für andere Systeme übrig zu haben. Kevin schien dieselbe Idee gehabt zu haben. Er enttarnte sein Schiff ebenfalls und gab Vollgas. Der Shutek, der sich von Achtern Steuerbord Nicos Jäger näherte, drehte ab und begann mit einer weiten Kurve.
\ortswechsel
\WR{Kacke!}, rief Nico. \WR{Dieses Ding schießt zurück!} Er ließ seinen Jäger ein wenig im Zickzack fliegen, um den Strahlenkanonen, die an der Unterseite des Triumphbogens montiert waren, auszuweichen. Aber seine Geschwindigkeit war bereits so hoch, dass er kaum noch effektiv manövrieren konnte.

\par

Die stärksten seiner Geschosse waren ungelenkte Raketen, die nichts weiter taten, als geradeaus zu fliegen und irgendwann zu explodieren. Allerdings bedeutete dies auch, dass Nico sein Ziel dauerhaft fixieren musste, um einen Treffer landen zu können. Zudem würden die Geschosse in der Atmorphäre von Luftmassenbewegungen und der Schwerkraft beeinflusst.

\par

Sein Zielcomputer war mit den Abtastern der \EN{Jungen Maid} gekoppelt. So konnte er eine Schusslösung errechnen, die Nico nur noch umsetzen musste. Die gedachte Linie der voraussichtlichen Raketenflugbahn wurde ihm auf sein Zielmonokel eingeblendet.

\par

Das Abwehrfeuer wurde immer heftiger und Nico wich ein wenig nach Backbord aus. Wieder heulte der Alarm für das bevorstehende strukturelle Versagen seines Schiffes auf. \WR{Krieg dich doch einfach mal wieder ein!}, schrie er seinen Vogel an, während er durch einen kleineren Wolkenausläufer stieß.

\par

Offenbar verwendeten die Abwehranlagen der Shutek eine optische Zielerfassung, denn für kurze Zeit hatten sie ihr Feuer eingestellt. Genug für Nico, um seine Zielerfassung auszurichten und abzudrücken.

\par

Unter lautem Rauschen lösten sich die beiden \Wr{Dummsprengköpfe}, wie sie im Pilotenjargon hin und wieder genannt wurden, von den Tragflächen seines Schiffes und schossen auf ihren Bestimmungsort zu.

\par

Doch noch lange bevor sie auf den Ring treffen konnten, prallten sie auf Schutzfelder, die bereits augenscheinlich sehr kraftvoll zu sein schienen.

\par

\WR{Kein Treffer}, meldete Nico und drehte ab. \WR{Schutzfelder oder sogar Schilde. Ich versuche gleich einen neuen Anflug, aber mir sitzt schon wieder eine Nervzecke im Nacken.}

\par

Überraschenderweise war es Marco Bellendi, der sich über den Staffelkanal meldete: \WR{Nein, Mister Curiosa. Wir haben die Wirkung des Einschlags gemessen. Die Schildblase hat eine Leistung von gut zehn Oktiwatt. Da bräuchten wir Nullzonensprengköpfe. Hat einer ihrer Jäger solche?}
