\WR{\EN{Regenvogel} KlT an Staffel Violett.
Sie haben Landeerlaubnis in der Reihenfolge eins, zwei, drei}, sprach der Kommutionsoffizier des Trägers in sein Mikrofon.
Genau wie die militärischen Handcomputer ein völlig veraltetes Modell.
Aber auch so einfach, dass ein Schulkind es reparieren konnte.

\par

Die Antwort verstand Abdel Samad nicht.
Er hatte wirklich gute Ohren, doch stand viel zu weit vom Kommunikationsoffizier entfernt, um irgendwelche Wörter aus dessen Kopfhörer dringen hören zu können.

\par

So tigerte der erste Offizier der \EN{Regenvogel} die obere Sektion der Brücke auf und ab.
Bei der Konstruktion der Kommandozentrale hatten sich die Ingenieure stark an den Decks früherer Seegelschiffe orientiert.
Der Sessel des Kapitäns und das Ruder befanden sich auf einem erhöhten Teil.
Ein Steuerrad fehlte natürlich, doch dort, wo es gestanden hätte, hatte der Navigator seinen Arbeitsplatz.
Eine üppige Anordnung mehrerer Konsolen direkt vor dem Handlauf, von dem aus man nicht nur den unteren Teil der Bücke überblicken konnte, sondern auch eine hervorragende Sicht auf die Sterne hatte.
Der Kommandoraum war überdeckt mit einer riesigen Kuppel, deren einzelne Panzerglasscheiben durch symmetrisch verlaufende Metallstreben zusammengehalten wurden.
Aber tatsächlichen Schutz bot der Extraschild der Brücke.
Eine kleines, blasenförmiges Schutzfeld um die Kommandozentrale herum, das in der Lage war, selbst sehr konzentriertes Feuer zu absorbieren.

\par

Der Steuermann war unterdessen beschäftigt, sich einige Navigationshologramme genauer anzusehen.
Die \EN{Regenvogel} wäre erst das zweite Schiff, dass nach Arktur springen würde und somit galt es, jede Berechnung am besten zehnmal zu prüfen.

\par

\WR{Sind das die Daten, die uns die PSG geschickt hat?}, fragte Commander Samad mit Blick auf die dreidimensionale Darstellung von Pollux und Arktur.

\par

\WR{Aye, Sir}, erwiderte der Navigator, ohne seine Augen von den Resultaten einer Berechnung zu wenden, die der Hauptcomputer gerade vornahm.
\WR{Da die \EN{Virial} den Sprung geschafft hat, sollten wir in der Lage sein, ihrem Kurs ohne größere Probleme zu folgen.}

\par

\WR{Zeit bis zum Sprung?}, hakte Samad weiter nach.

\par

Der Steuermann hielt kurz inne.
\WR{Nun, wir erreichen die Route in elf Stunden und zweidutzend vier Minuten.
Wir befinden uns bereits nicht mehr im Raum der Union.
Aber ich würde gerne trotzdem noch etwas Zeit einplanen, um die Sprungberechnungen zu prüfen.
Zwölf Stunden, Sir.}

\par

Samad nickte gedankenverloren.
\WR{Ich rede mit dem Captain.}

\par

Das Licht, in das die Brücke getaucht war, erinnerte den ersten Offizier an eine Straße bei nacht.
Die hohe Alarmbereitschaft wurde durch gelbe Singalleuchten angezeigt, die an den matten, surrealen Schein von Laternen erinnierten.

\par

Im unteren Bereich der Brücke herrschte derzeit deutlich mehr Betrieb.
An den Rändern der Plattform, die durch die durchsichtige Hülle des Kommandozentrums wie frei schwebend wirkte, arbeiteten die Kanoniere an ihren Stationen.
Maas Petrarca lief von einer der Glaswände zur nächsten.
Sah man durch diese hindurch, erkannte man die Schusslinien der Bordkanonen neben zahlreichen anderen eingeblendeten Informationen.

\par

In der Mitte der unteren Ebene, fast schon auf Höhe der Bugplattform, die nur der Aussicht diente, war der Jägerleitstand aufgebaut worden.
Commander Samad empfand die Positionierung als recht unglücklich.
Zwar befand sich der Leitstand in Hörentfernung der Radarzentrale, die unterhalb der oberen Ebene untergebracht war, doch die Kommunikationsstation war hinter dem Sessel des Kapitäns angesiedelt und somit völlig außer Gesprächsreichweite.
Der Nachrichtenoffizier musste also entweder nach unten brüllen oder per Sprechanlage mit den Lotsen im Leitstand sprechen.

\par

\WR{Kommando \EN{Regenvogel} an Kommando \EN{Albatross}}, sprach Nils Wallander in sein Headset.
Der Kommunikationsoffizier hatte von Natur aus blaue Haare.
Eine Mutation, die noch auf die Wirkung der Seuche zurückzuführen war.
Dass er sie sich nicht färbte, zeigte, wie viel er privat auf die Meinung anderer gab, wofür ihn Abdel Siegel insgeheim bewunderte. \WR{Sie driften schon wieder.
Bitte bleiben Sie auf Parallelkurs.}

\par

Die Antwort bekam Samad nicht mit.
Doch, dass Wallander nicht ganz darauf vertraute, dass die Mannschaft der \EN{Alabatross} einfach tat, wie ihr geheißen, zeigte sich, als er den internen Sprechfunk benutzte.
\WR{Schwarzschild, sagen Sie mir bitte sofort Bescheid, wenn sich etwas am Kurs unserer Backbordeskorte etwas ändert.}

\par

Abdel Samad hörte die junge Frau aus der Radarzentrale etwas sagen, doch konnte nicht genau verstehen, welche entnervte Antwort sie gab.
Sein Augenmerk fiel nun ohnehin auf Captain Natalia Fiscale.
Sie stand am Geländer des oberen Brückenbereichs und sah geistesabwesend durch die Glasscheiben.
Ihr Blick ging zwar in Richtung der \EN{Alabtross}, einer Corvette, deren Bauart man am besten als Zigarre mit zwei dreieckigen Flügeln beschreiben konnte.
Die Form war so einfallslos wie effizient.
Kleine Schiffe dieser Art waren schnell produziert und bewegten sich mit hohem Tempo.
Ideal zum schnellen Aufbau einer Verteidigung.

\par

Aber der erste Offizier glaubte nicht, dass die Kapitänin sich wirklich das eine der beiden Begleitschiffe ansah, dass der \EN{Regenvogel} für ihre Suchaktion in Arktur zugteilt worden war.
Als die kleine, stabile Frau mit viel breiteren Schultern als Hüften ihren Stellvertreter bemerkte, ließ sie das Geländer los, dass sie zuvor noch fest umschlossen gehalten hatte.
Kleine Schweisflecken blieben zurück und begannen sich sofort aufzulösen.

\par

\WR{Nervös?}, fragte Major Samad leise.

\par

\WR{Angespannt}, gab die Kommandantin zurück.
Sie lächelte, doch war noch nie gut darin gewesen, zu verbergen, wie es in ihr aussah.
Als sie bemerkte, dass sie ihrem langjährigen Kameraden nichts vormachen konnte, sagte sie: \WR{Ich habe keine Ahnung, was auf der anderen Seite dieser Route wartet.
Es ist wie damals in Ross zwei vier acht.}

\par

Commander Samad erinnerte sich an den Vorfall, der vor fast einem dutzend Jahren stattgefunden hatte.
\WR{Das sollte dich beruhigen.
Du bist damals mit allem fertig geworden, du wirst es heute auch.}

\par

\WR{Das ist nicht das Problem}, fuhr Captain Fiscale fort.
\WR{Die Capital Fellowship hat uns einen harten Kampf geliefert.
Und ich vertraue meiner Besatzung nicht weniger als damals.
Es ist nur: was ist, wenn wir wieder gegen Piraten oder Terroristen kämpfen müssen.
Die Kampagne damals hat mir alles abverlangt.
Und nicht wegen der schlaflosen Nächte oder dem Druck.
Sondern, weil wir auf Menschen feuern sollten.
Und weil wir es getan haben.}

\par

Commander Samad lehnte sich rücklings an den Handlauf.
Sein breiter Schnauzer und seine langsam grau werdenden Haare ließen ihn dabei wie einen sprichtwörtlichen Fels in der Brandung wirken.
Der deutliche Ansatz eines Bierbauchs tat das seine dazu.
\WR{Die Fellowship ist zerschlagen. Wenn da draußen wirklich noch welche von denen wegelagern, dann nehmen sie die Beine in die Hand, wenn sie uns bloß sehen.
Selbiges gilt für Piraten.
Diese Typen sind echt der Witz.}

\par

\WR{Du warst damals mein Gewissen}, erinnerte Fiscale und machte ihrem ersten Offizier damit klar, dass sie die Erinneriung an vergangene Tage in diesem Moment nicht so einfach abstreifen konnte oder wollte.
\WR{Ohne dich wäre ich nicht mit dieser Last fertig gewoden.}

\par

Commander Samad sah auf den Boden.
Er wurde fast rot.
Lob von der Kapitänin war selten und  kam mitten auf der Brücke schon gar nicht vor.
\WR{Du stellst dein Licht unter den Scheffel.
Ich denke nicht, dass du es nicht auch alleine gepackt hättest.
Aber das spielt keine Rolle.
Ich bin immer noch da, was auch immer passiert.
Und ich vermute mal, dass wir es hier mit nichts weiter als einer abgebrochenen Antenne zu tun haben.
Zeit ein paar Eierköpfe zu retten.}

\par

Natalia Fiscale blieb einen Moment ruhig.
Nach einer Weile fragte sie: \WR{Wie laufen die Vorbereitungen, Commander?}

\par

\WR{Alles im Zeitplan}, antwortete Samad.
\WR{Nur Mister Senkethi bittet um mehr Zeit für die Berechnungen.}

\par

\WR{Wie viel?}, fragte die Kommandantin sofort.

\par

\WR{Eine Stunde.}

\par

\WR{Er soll sich beeilen.}
Mit diesen Worten nahm sie wieder auf ihrem Sessel platz.
Der Stuhl rechts neben ihr blieb jedoch leer.
Abdel Samad zog es vor, ein wenig auf und ab zu gehen.
Nur zu sitzen und zu warten machte ihn mürbe.
Als er an der Kommunikationsstation vorbeikam sah Lieutenant Wallander zu ihm auf.

\par

\WR{Sir, bitte sagen Sie Major Farley, sie soll mit Hennington reden.
Ich bekomme ständig Anfragen von den neuen, weil die nicht blicken, wo sie hinsollen.
Vermutlich hat er sie genauso gut eingewiesen wie die letzten.}