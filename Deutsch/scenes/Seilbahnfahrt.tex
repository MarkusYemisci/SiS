\WR{Die sehen uns doch garantiert!}, sagte Minx, als sie gemeinsam mit den anderen Soldaten und Nico in Deckung ging. Einer der Marineinfanteristen hatte seine Fähigkeiten als Mechaniker spielen lassen und die Beleuchtung der Gondel ausgeschaltet. Trotzdem war sie im Licht der Monde sicher gut zu erkennen und vor allem auch zu hören.

\par

\WR{Vielleicht wissen diese Ärsche nicht, was eine Seilbahn ist}, sinnierte Kacper Piecek. \WR{Dann werden sie auch nicht versuchen, die Bergstation zu sichern.}

\par

Schwarzer Bär sah unterdessen angestrengt durch seinen Feldstecher. Die digitale Vergrößerung seines Zielmonokels war ihm zu unscharf. Doch auch durch das Fernglas konnte er nicht viel erkennen. \WR{Verlass dich auf das Gegenteil. Die sind da oben. Der Ausstieg wird die Hölle.}

\par

\WR{Und das habe ich immer beim Liften im Skiurlaub gesagt}, murmelte Nico.

\par

Seine Aussage würde durch Klicken und Rattern übertönt, dass Schwarzer Bärs Gewehr machte, als er einen neuen Munitionskanister einspannte. Nico trainierte selbst regelmäßig mit Hanteln. Trotzdem war er überrascht darüber, wie augenscheinlich entspannt der Marineinfanterist sein schweres Maschinengewehr mit einer Hand festhielt, während er es mit der anderen lud. Die Waffe, die er dem Shutek abgenommen hatte, und die selbst nicht gerade leichtgewichtig aussah, hatte er sich auf den Rücken gespannt.

\par

\WR{Die Klinge ist der Wahnsinn}, sagte Minx zu Nico, als sie sah, wie er das Shutek-Gewehr beäugte. \WR{Wird auf Knopfdruck scharf. Als hätte sie einen mikroskopischen 3D-Drucker an der Schneide, der sie bis auf ein einzelnes Molekül anschärft. Ich wünschte, wir hätte mehr Zeit, das Ding zu untersuchen.}

\par

\WR{Wenn wir es lebend auf die \EN{Heinlein} zurück schaffen, kannst du es dir so lange anschauen, wie du willst}, gab Schwarzer Bär beiläufig zurück, während er wieder durch seinen Feldstecher sah.

\par

Nico sah die Angst in Kacper Pieceks Gesicht stärker und stärker durchscheinen. Auch als Scharfschütze hatte er sicher den Nahkampf trainiert. Doch er fühlte sich auf einige Distanz mit dem Gegner garantiert um einiges wohler.

\par

\WR{Durchladen, Leute!}, rief Schwarzer Bär. \WR{Ausstieg in fünf Minuten!}

\par

Natalia Fiscale kniff ihre Augen zusammen. Das feindliche Schiff war schonungslos in die Reihen der Navyblockade geprescht. Noch war es relativ weit entfernt, doch die Kapitänin glaubte bereits, in ihm einen Zerstörer zu erkennen. Die Begleitschiffe der \EN{Artiglio} del Leone und der Regevogel hielten wie befohlen die Stellung.

\par

\WR{Die Baal ist in Schussweite!}, vermeldete Lieutenant Commander Petrarca, während das feindliche Schiff bereits begann, Breitseiten mit einer Fregatte auszutauschen.

\par

Captain Fiscale zögerte nicht lange. \WR{Feuer frei!}

\par

Kaum hatte Fähnrich Hollowitz den Befehl auf dem allgemeinen Flottenkanal durchgegeben, Richteten sämtliche Zerstörer fast zeitgleich ihre sekundären Strahlenkanonen auf das Ziel aus. Die Flottille der Navy hatte nunmehr eine deutlich dankbarerer Position, als zu Beginn der Schlacht und konnte so ihre Feuerkraft besser bündeln.

\par

Wie kleine rote Sterne begannen die Kanonen der Zerstörer zu leuchten, als sie ihre Läufe mit Unmengen an Energie aufluden. Die \EN{Impervious}, welche die Spitze der Flottille bildete, schaltete tatsächlich kurzfristig ihren Antrieb ab, um mehr Energie aus dem Perpetuum Mobile beziehen zu können.

\par

Dann brachen die Entladungen aus den sekundären Geschützen heraus. Fünf durchgehende Strahlen fanden fast im selben Augenblick ihr Ziel. Wie Säulen aus Feuer brandeten sie gegen die Schutzfelder der Baal. Schnell fielen diese aus und die Strahlen fraßen sich aus mehreren Richtungen in die Hülle des feindlichen Schiffs.

\par

Der Zerstörer feuerte noch wütend aus allen Rohren. Aber nur kurze Zeit darauf~-- spätestens, als ihn Schüsse aus der Hauptkanone der \EN{Impervious} trafen~-- ging er in lodernde Flammen auf. Kleinere und große Explosionen überzogen seine Außenhaut und schließlich platzte er auseinander.

\par

\WR{Ja!}, riefen gleich mehrere Offiziere der Brückenbesatzung wie aus einem Munde. Auch war vereinzelter Beifall zu hören. Doch Fiscale wusste bereits, dass dieser Abschuss nur einen kaum nennenswerten Erfolg darstellte. Tatsächlich hatten die Shutek mit diesem Vorstoß nur versucht, die Verteidigung der Blockade zu testen. Es warteten noch zwei dutzend weitere Raumer und zahllose Jäger auf ihren Einsatz.

\par

Die einzige Frage, die sich Fiscale nun stellte, war, ob sie versuchen würden, mit geballter Kraft die Schiffe in der Umlaufbahn anzugreifen oder sich tatsächlich weiter aufteilen würden. Inständig bat sie um ersteres. Denn so würde sie zumindest versuchen können, den Feind lange genug hinzuhalten, bis eine eventuelle Verstärkung eintraf. Ob diese kommen würde, war jedoch fraglich. Zwar hatten sich neben der \EN{Regenvogel} noch zwei weitere Träger auf den Weg gemacht, als der Angriff auf Kreuzpunkt gemeldet worden war, doch je nachdem, welche Routen sie wählten und wie viel die Kapitäne ihren Maschinen zuzumuten bereit waren, könnte es noch viele Stunden dauern, bis sie eintrafen.

\par

\WR{Mister Hollowitz}, begann Fiscale und versuchte, dabei hoffnungsvoll zu klingen, \WR{wie ist der Zustand unserer Nullzonen-Kommunikation? Ich würde mich jetzt \textit{wirklich} über eine Meldung unserer Flotte freuen.}

\par

Der Mann~-- seine blonden Haare und die vielen Sommersprossen verstärkten sein jugendliches Äußeres noch~-- warf der Kommandantin einen hilflosen Blick zu. \WR{Es tut mir leid. Aber momentan funktioniert gar nichts.}
\ortswechsel
Die Gondel hatte die Bergstation nicht einmal erreicht, da hagelten bereits die ersten Salven auf sie ein. Schwarzer Bär lugte vorsichtig über die untere Seite des Fensters hinweg. Dann ging er in die Hocke und gab ein Sperrfeuer auf die kaum erkennbaren Gegner ab. Nach wie vor waren die deutlich sichtbaren Entladungen ihrer Strahlenwaffen der beste Fingerzeig auf ihre Position.

\par

\WR{Sie zielen auf die Seilhalterung!}, rief Fischauge entsetzt.

\par

Nun fasste sich Nico ein Herz und ging ebenfalls in die Hocke. Der Knall seiner Pistole klingelte laut in seinen Ohren nach. Er zielte mit einem Auge und kniff das andere fest zu.

\par

Das Sperrfeuer schien die Shutek zumindest ein Stück weit in ihre eigene Deckung zurück zu drängen, denn ihre Salven wurden seltener. Keiner ihrer Schüsse traf das Kabel, bis die Gondel ihr Ziel erreichte.

\par

Mit lautem Gebrüll rannte Schwarzer Bär als erstes aus der Seilbahn. Nico sah einen Moment wie versteinert in das Gesicht des Mannes, der mit einer wuterfüllten Fratze auf einen Shutek zu stürmte, der gerade in die Bergstation eintrat. Die ersten Schüsse aus Bärs Maschienengewehr hämmerte auf die Brust der hühnenhaften Gegners. Doch entgegen jeder berechtigten Erwartung brachten die Treffer den Shutek nicht zum Fallen. Vor seiner Brust schien die Luft zu flimmern.

\par

Nico hatte so etwas bereits oft gesehen. Bei den Schutzfeldern von Raumschiffen. Dass diese als persönliche Blocker benutzt werden konnte, hätte er niemals gedacht. Wenn die Union über eine solche Technologie verfügen sollte, dann besaß sie derzeit sicher nur der Geheimdienst.

\par

Die Salve erreichte jedoch zumindest eine Wirkung. Das Gewehr des Shutek zerschellte, woraufhin dieser einen Schritt zurück wich. Jedoch nur, um Bärs eigene Waffe mit einem weit ausgeholten Schlag aus der Hand zu reißen. Das Maschinengewehr fiel scheppernd zu Boden, während sich sein Beiszter unter einem weiteren Hieb des Shutek weg duckte.

\par

Nico sah genau hin. Dieser Soldat war anders. Er trug eine Maske und wirkte noch einmal wuchtiger als die restlichen Shutek, die er bislang beobachtet hatte. Seine dicke Rüstung und die Hände, die man eher als Pranken bezeichnen musste, ließen keinen anderen Schluss zu, als dass sich dieser Kämpfer auf den Nahkampf konzentrieren sollte.

\par

Während der Rest der Gruppe ausstieg und auf die restlichen Feinde schoss, die gerade in den Raum gestürmt kamen, schwang Schwarzer Bär das erbeutete Gewehr vom Rücken und schlug mit der Klinge auf den Shutek ein. Ein breiter Schlitz bildete sich augenblicklich auf dem Panzer des Kämpfers, doch dieser schlug zurück. Mühelos bekam er Bärs Waffe zu greifen.

\par

Nico sah wie das Gesicht seines Kameraden vom Schmerz verzerrt wurde, als er mit dem Shutek um das Gewehr rang. Schnell aber geduckt rannte er aus der Gondel und brachte seine Pistole in Anschlag. Die Schüsse schienen völlig wirkungslos von der Rüstung des Shutek abzuprallen, nachdem er den Abzug mehrfach durchgezogen hatte. Doch damit erlangte er die ungewollte Aufmerksamkeit des bulligen Gegners. Dieser drückte Schwarzer Bär zu Boden und ging schnellen Schrittes auf Nico zu. Aus seiner Handgelenk fuhr sich eine Klinge aus, die jener der Gewehre in nichts nachstand.

\par

Nico blieb einen Moment lang wie angewurzelt stehen. Dann entschied er sich in die Gondel zurück zu rennen. Doch er brauchte nicht über die Schulter zu sehen, um zu wissen, dass sich diese gerade in die Tiefen am Abhang des Spechtgipfels verabschiedet hatte. Dafür reichte der laute Lärm und das Scheppern völlig aus.

\par

Ein Soldat wollte für den mittlerweile am Boden liegenden Nico in die Presche springen. Er trug ein schweres Schockgewehr, dass er hastig auf den bulligen Shutek ausrichtete. Der erste Schuss löste sich und ein Mündungsfeuer, länger und breiter als das massive Gewehr selbst, brach aus dem Lauf heraus. Doch die Druckwelle verfehlte ihr Ziel und schlug nur in die Wand des Raumes ein. Da diese einige Meter weit entfernt und die das Schockgewehr für den Kampf auf sehr kurze Distanzen ausgelegt war, richtete sie nicht einmal dort einen großen Schaden an.

\par

Der Shutek ließ seine Klinge schwingen und Nico musste mit weit aufgerissenen Augen mit ansehen, wie der Kopf des Soldaten, der gerade noch versucht hatte, ihn zu retten, von dessen Schultern fiel.

\par

Die Bilder der sterbenden Familie schossen ihm blitzartig ins Gedächtnis und er wurde so wütend, dass er die Schockstarre überwinden und auf das Gewehr des gefallenen Soldaten zu hechten konnte. Ohne genau zu zielen, hob er die massige Waffe an und drückte ab.

\par

Als er seinen Kopf anhob, sah er als erstes, wie der Körper des Shutek entlang eines glühenden Risses entzwei brach und krachend aber leb- und regungslos auf den Boden fiel.

\par

Nico erhob sich, ließ das Schockgewehr aber liegen. Stattdessen griff er wieder zu seiner Pistole und sah sich hastig um. Mehrere Leichname~-- die meisten davon Shutek~-- lange am Boden verteilt. Der Kampf war vorerst vorbei.

\par

Aus den jenen Köpfen der Außerirdischen, die einen Treffer abbekommen hatten, schien blaugrünes Blut zu tropfen. Aber bei genauerem Hinsehen schien es viel zu ölig und glitschig, um tatsächlich eine Körperflüssigkeit zu sein.

\par

\WR{Was sind das nur für Typen?}, fragte Nico ehrlich schockiert. \WR{Sehen aus wie Roboter. Aber… Nein. Das können keine sein. Das kann einfach nicht sein.}

\par

\WR{Hey!}, rief ihm Schwarzer Bär entgegen. \WR{Konzentrier dich, Mann. Scheiß drauf, was die sind. Darüber sollen sich später die Eierköpfe Gedanken machen. Wir müssen hier weg. Die gute Nachricht ist, die Starforce-Basis ist nur noch wenige din Meter weit entfernt. Aber wenn ihr Spinner von da oben tatsächlich ernst macht, dann haben wir nur noch höchstens zehn Minuten.}