Morten trat nervös von einem Bein aufs andere. Er und ein Dutzend anderer Piloten warteten im Bereitschaftsraum neben dem Flugdeck. Die restlichen Piloten waren entweder bereits gestartet oder bemannten gerade ihre Jäger. Im Fall eines  Gefechtsalarms gab es feste Pläne, wer mit welchem Flieger starten sollte. Als Neulinge gehörten Morten, Kevin und Jens zu den Ersatzpiloten, die sich in dem schallisolierten Raum neben dem Flugdeck aufhielten und auf Anweisungen warteten.

\par

Der Bereitschaftsraum verfügte über eine Kommunikationsstation und war nur durch eine Fensterfront vom Start- und Landedeck abgetrennt. Gerade hoben nacheinander die Verteidigungsjäger der \EN{Regenvogel} ab. Die massigen Schiffe mit vier Tragflächen und zahllosen Raketen und Antitorpedo-Geschossen dröhnten auf der Startrampe, als sie ihre unterproportioniert wirkenden Plasmaturbinen hochfuhren und davon rauschten. Schnell waren diese Jäger der \EN{Wachhund}-Klasse nicht, doch das mussten sie auch nicht sein. Im Grunde waren es schwer bewaffnete fliegende Festungen zur Verteidigung von Großkampfschiffen, die immer nahe bei ihrem Schützling blieben. In der Theorie sollten ihre gewaltigen Kanonen und Raketenarsenale feindlichen Bombern und deren Geschossen schnell den Garaus machen. Allerdings zweifelte Morten daran, dass dies in der Realität genauso funktionieren würde. Besonders wenn er die geringe Kampferfahrung der meisten Konglomeratssoldaten bedachte.

\par

Allerdings ging der Start nun viel schneller von statten als in Arktur. Und das, obwohl nicht nur ein paar, sondern schlicht alle Staffeln der \EN{Regenvogel} gestartet werden mussten. Deckchef Alan Tukarev, der Büffel, rannte mit immer finsterer werdender Mine umher und spornte seine Mannschaft schimpfend an, sich mehr zu beeilen. Gerade wurden die zwei riesenhaft wirkenden schweren Jäger der \EN{Regenvogel} von den Lastenaufzügen aufs Deck befördert, als die Kommunikationsstation des Aufenthaltsraums zum Leben erwachte.

\par

Lieutenant Wallander sichtlich gestresstes Gesicht erschien auf dem Monitor des Terminals. \WR{Lieutenant Wörg, bitte melden.}

\par

Kringel wirkte im ersten Moment überrascht, dass er und nicht einer der viel erfahreneren Piloten gerufen wurde, spurtete dann aber schnell los. \WR{Ich höre, Brücke}, sagte er, nachdem er sich die Kopfhörer aufgesetzt hatte.

\par

Wallander wartete nicht lange, sondern kam sofort zur Sache: \WR{Schnappen Sie sich Wilson und Wittwer und steigen sie in die Haie der grauen Staffel. Sie führen, da Sie der dienstälteste von ihnen drei sind. Weitere Befehle bekommen Sie, sobald sie in der Luft sind. Haben Sie das verstanden?}

\par

\WR{Jawohl, Sir}, bestätigte Kringel. Sein Gesicht wirkte mit einem mal wie versteinert.

\par

Dann drehte er sich zu Morten und Kevin um und erklärte ihnen: \WR{Ich fürchte unser Anfängerdasein kommt uns nicht zugute. Aus irgendeinem Grund will das Kommando Haie einsetzen und ich glaube, darum haben sie sofort an uns gedacht. Vermutlich, weil wir noch am ehesten mit diesen Dingern geflogen sind. Wir haben den Befehl, sofort mit der grauen Staffel zu fliegen. Ich hab das Kommando. Kevin, du bist zwei, Morten drei. Verstanden?}

\par

Beide nickten. In Kevins Gesichts war deutlich eine große Abenteuerlust zu sehen, während Morten sich erst einmal den Schweiß von der Stirn wischte. Ohne ein weiteres Wort zu wechseln spurteten die drei Anfänger auf das Startdeck der \EN{Regenvogel}. Der Büffel musste die Deckmannschaft wirklich angespornt haben, denn die drei Haie wurden bereits in Position gefahren. Gezogen wurden sie Hangar-Maulwürfen. Fahrzeugen die neben der Bereitstellung von Jägern noch zum Transport von schweren Waren eingesetzt wurden. Ein Mechaniker vermaß mit einem kleinen Handgerät, ob die Jäger auch parallel zur Startbahn ausgerichtet waren und zeigte anschließend Jens Wörg einen erhobenen Daumen. Dieser wandte sich noch einmal an Morten und Kevin. \WR{Hört zu, Männer. Ich weiß selbst nicht eher als ihr, was hier vorgeht. Aber ich kann mir nicht vorstellen, dass das ein Spaziergang wird. Also haltet beide Augen weit offen. Los geht}s}

\par

Mit diese Worten schwang sich Kringel die Leiter hinauf ins Cockpit seines Hais. Morten und Kevin schlugen ihre Fäuste gegeneinander und bemannten daraufhin ebenfalls ihre Schiffe, wobei ersterer sehr erleichtert darüber war, dass sein Kamerad nicht wieder auf einem \Wr{Deine-Mutter-Witz} bestand.

\par

Das Cockpit des Hais wirkte genauso alt und abgenutzt, wie der Jäger an sich vermutlich auch war. Haie waren das erste Gemeinschaftsprojekt der Erdallianz und des Commonwealth nach dem Routenkrieg geworden. Nach einer fast zehn Jahre langen Erprobungsphase waren die ersten Modelle fertig gestellt worden. Die damaligen Varianten hatten noch ein wenig anders ausgesehen. Doch die Gestaltunf des Rumpfes wie aus einem Guss, windschnittig und glatt, war über die Jahrzehnte erhalten geblieben. Flügelspitzen und Bug waren abgerundet und die vier Strahlenkanonen, die eng am platt wirkenden Rumpf verliefen, wirkten überdimensioniert lang.

\par

Bei der Cockpitgestaltung hatten die Ingenieure bewusst einen altertümlichen Ansatz gewählt. Die Instrumente waren zwar allesamt die Anzeige von Computerprogrammen, erschienen aber als Skeuomorphismus analoger Nadeln und Zeiger.

\par

Der Sitzbezug in Mortens Jäger war durchgesessen und es dezent aber unbeirrbar nach Erbrochenem. Vermutlich hatten sich gleich mehrere Piloten in der Dienstgeschichte des Fliegers in seinem Cockpit übergeben.

\par

Ein kurzer Blick auf die Anzeigen der elektronischen Checkliste verrieten Morten, dass seine Maschine einwandfrei funktionierte und bereit war zum Start.

\par

\WR{Leitstand an Staffel grau}, quäkte es aus Mortens Kopfhörern, gerade als dieser sein Zielmonokel herunterklappte und die Düsen warmlaufen ließ. \WR{Sie haben Starterlaubnis. Zwei und eins, Sie starten gleichzeitig, also Vorsicht! Dann sind Sie dran, Grau drei. Und viel Glück da draußen!}

\par

Kurz erklang etwas, das fast wie ein Freudenschrei klang und eindeutig von Kevin Wilson stammte aus Mortens Kopfhörern und die Jäger seiner beiden Staffelkameraden schossen fast gleichzeitig los. Die beiden Haie legten die Startbahn annähernd synchron zurück und bald waren von ihnen nur noch ihre Nachbrenner zu sehen.

\par

Gerade als Morten die Heckkamera zuschaltete, um sicherzugehen, dass sich niemand hinter seinem Flieger befand, tönte aus seinem Kopfhörer die Stimme des Deckoffiziers: \WR{Verdammt, Grau drei. Geben Sie Gas! Wir haben keine Zeit.}

\par

Um keine weitere Sekunde mehr zu verschenken, sparte sich Morten eine Antwort und drückte einfach nur den Schubregler ganz durch. Sein Hai begann die Startbahn entlang zu rasen. Die Wände schossen scheinbar an ihm vorbei und er musste sich beherrschen um nicht den Steuerknüppel erratisch nach rechts oder links zu reißen, während er mehr noch als in seinem Aufklärer durchgerüttelt wurde.

\par

Nach einen kurzen Augenblick hatte er die Energieabschirmung hinter sich gelassen und flog nun im absoluten Vakuum. Das Fehlen von Luftwiderstand ließ seinen Flieger rasant schneller werden. Gerade als er das Ende der Bahn vor sich sah, zog er den Steuerknüppel zu sich hin und hob ab.

\par

Kaum dass die violette Staffel ein paar hundert Kilometer zwischen sich und die \EN{Regenvogel} gebracht hatte, meldete sich bereits wieder Lieutenant Wallander. Sein Funkspruch war gleichzeitig an Jens, Morten und Kevin gerichtet. Statisches Rauschen und andere Störungen begleiteten die Übertragung, doch die Besorgnis in der Stimme des Kommunikationsoffiziers war dennoch unüberhörbar.

\par

\WR{Staffel Violett. Der Frachter \EN{Pales fünf} führt einen Konvoi von Flüchtlingen an, die es noch rechtzeitig von der Oberfläche geschafft hat. Wir haben etwa zwei Dutzend Bogeys aufgespürt, die sich den Transportern nähern. Unsere Abfangjäger sind auf dem Weg, brauchen aber Hilfe. Sie fliegen sonst die schnellsten Schiffe. Wir schicken Ihnen die Koordinaten.}

\par

\WR{Koordinaten erhalten}, erwiderte Jens Wörg nur ein paar Sekunden später. \WR{Wir sind auf dem Weg, \EN{Regenvogel}.} Morten und Kevin hatten Mühe, ihre Jäger in Formation mit dem wild herum schlenkernden Jens zu halten. \WR{Nachbrenner zuschalten}, befahl dieser und schoss auf die Koordinaten zu, die ihm der Träger kurz zuvor übermittelt hatten. Seine beiden Flügelmänner taten es ihm gleich. Morten behielt die Trägheitsabsorber seines Jägers im Auge. Vielleicht war die Technik einfach schon zu alt aber er wurde das Gefühl nicht los, dass sie auf eine Überlastung zusteuerten~-- und zwar um einiges schneller, als er es gewohnt war.

\par

\WR{Noch etwas}, hängte Lieutenant Wallander per Funk an. \WR{Senden Sie keine Warnungen. Diese Drecksäcke schießen auf Zivilisten. Blasen Sie sie vom Himmel.}

\par

\WR{Keine Sorge!}, bestätigte Kevin Wilson, bevor Jens Wörg die Chance dazu hatte. \WR{Heute ist Zahltag!}

\par

Morten hatte Schwierigkeiten, seinen Jäger geradeaus fliegen zu lassen. Obwohl der Schadensmonitor nichts anzeigte, glaubte er, dass etwas mit der Quersteuerung nicht stimmte. Das die Diagnoseprogramme kein Problem anzeigten, sprach gegen einen Fehler des Büffels und seiner Deckmannschaft. Wahrscheinlich hatte die jahrelange Lagerung zu Verfallserscheinungen in der Mechanik geführt.

\par

Unerwarteter Weise stammte der nächste Funkspruch an Staffel Grau von Anna Farley. \WR{Jens, sei vorsichtig}, mahnte sie. \WR{Ich starte gerade mit dem Rest der Staffel rot. Wir sind nur ein paar Minuten hinter euch. Wenn ihr mit den Feinden nicht klar kommt, haltet sie hin, bis wir da sind.}

\par

Morten sah auf seine Radaranzeige. In der Mitte der drei konzentrischen Kreisen gab ein Blickpunkt an, wo die letzten bekannten Koordinaten des Frachters lagen. Sonst waren nur noch Grau zwei und drei sowie die \EN{Regenvogel} samt Geleitschutz zu erkennen. Von Feinden fehlte bisher jede Spur.

\par

Durch die Cockpitscheibe war Pollux Primus zu sehen. Der Planet füllte mittlerweile fast das ganze vordere Blickfeld aus. Unter anderen Umständen hätte Morten die Aussicht sicher genossen. Pollux Oberfläche war hauptsächlich mit Wasser bedeckt. Nur wenige Wolken schwebten am Himmel und die seltenen, besiedelten Flecken waren an den grünen Wäldern zu erkennen, die sie umringten. Als er seinen Blick vom Nord- zum Südpol des Planeten schwenken ließ, wurde Morten erst klar, wie riesig der Himmelskörper war. Pollux Primus lag einfach still im All, scheinbar völlig unbeeindruckt von allem, was in seiner Umgebung geschah.

\par

Dann erschien auf der Oberfläche ein heller Lichtblitz, der so hell war, dass Morten ihn noch aus vielen hundert Kilometern Entfernung erkennen konnte. Kurz darauf folgten zwei weitere. Dem Piloten war klar, dass diese Lichter nur von heftigem Waffenbeschuss stammen konnten und zum ersten mal dachte er an die vielen Tausend Menschen, die wahrscheinlich dort unten gefangen waren und um ihr Überleben kämpften. Und im Moment sah es nicht gut für sie aus.

\par

Ein weiterer, stark gestörter Funkspruch schreckte Morten aus seinen Gedanken auf. Der Mann am anderen Ende klang kontrolliert und ruhig, doch die Angst schwang in seiner Stimme unüberhörbar mit. \WR{Notruf, hier ist Nico Curiosa, vom Flug dutzend eins. Unsere Koordinaten sind zwedutzend zehn, zu sechs dutzend Komma fünf zu dutzend sieben Komma acht eins sieben bezüglich Pollux. Feindliche Jagdschiffe steuern auf uns zu und werden uns demnächst erreichen.}

\par

Morten verglich die Koordinaten mit denen des Konvois. Anscheinend hatte sich Flug dutzend eins aus irgendeinem Grund um etwa dreitausend Kilometer vom Rest seiner Gruppe entfernt. Endlich zeigte auch Mortens Radar etwas an. Zwei blaue Blickpunkte repräsentierten zivile Schiffe. Der Computer identifizierte sie anhand ihrer Transpondersignale als Flug dutzend eins und dutzend zwei.

\par

\WR{Hier spricht Lieutenant Wörg. Staffel Grau. \EN{Regenvogel} KlT. Wir haben bereits unsere Nachbrenner laufen, halten Sie nur ein bisschen durch.}

\par

Die Abgeklärtheit und die Ruhe mit der Jens Wörg vorging, überraschte Morten nicht wenig. Er flog gerade mal ein Jahr länger als er und Kevin und hatte selbst noch keine Gefechtsmission hinter sich gebracht. Trotzdem wies er seine beiden Flügelmänner fast schon gelassen an: \WR{Jungs, ändert den Kurs. Wir helfen diesen Transportern aus der Patsche.}

\par

\WR{Jawohl, Boss}, bestätigte Kevin überschwänglich und verrückte die Nase seine Jägers nur um ein kleines Stück auf die beiden Transportern zu. Morten tat selbiges und erkannte sogleich vier weitere Blickpunkte auf dem Radar. Sie waren Rot eingefärbt, was nur bedeuten konnte dass der Computer zwischen diesen Raumern und denen aus Arktur eine Ähnlichkeit erkannte.

\par

\WR{Wir haben Gesellschaft}, meldete Morten per Funk und betrachtete die Entfernungsanzeige. Bis zu den Transportern war es nicht mehr weit.

\par

Und schon zuckten die ersten Energieentladungen über Mortens Cockpit. Es schienen einige, schlecht gezielte Schüsse von einem der vier Feinde gewesen zu sein, denn die Gegner hielten sich noch außerhalb seiner eigenen Feuerreichweite auf. Er hielt es für unwahrscheinlich, dass ihre Waffen ihn sehr viel früher erfassen konnte, als seine sie.

\par

\WR{Lösen und angreifen!}, rief Jens in sein Funkgerät und ließ seinen Jäger nach Steuerbord wegkippen.

\par

Kevin brüllte laut in sein Funkgerät und zog seinen Abzug. Morten hätte ihn am liebsten aufgefordert, die Klappe zu halten, doch er war zu beschäftigt, den Salven auszuweichen, die einer der Feinde andauernd auf ihn abfeuerte.

\par

Indem er die Sicherheitsklappe über dem zweiten Abzug seines Steuerknüppels öffnete, machte er seine Raketen scharf. Mit der linken Hand, wählte er willkürlich eine seiner vier Verfolgungsrakten aus und befahl dem Computer eine Erfassung vorzunehmen. Doch die Ellipse, auf Mortens Zielmonokel, die für die Raketenpeilung stand, huschte nur wild hin und her.

\par

Während dessen zündete Jens Wörg noch einmal seine Nachbrenner und schoss auf einen der Gegner zu, so, wie er es auf der Flugschule gelernte hatte. Möglichst auf einer gerade Linie, so dass sein Feind nur wenig Angriffsfläche seines Jägers ins Visier nehmen konnte. Doch wie er schnell feststellte, bewegte sich der rote Blickpunkt ein wenig von ihm weg.

\par

Mit einem Druck auf eine der zahllosen Tasten neben dem Radarbildschirm ließ er den Computer eine Zielerfassung vornehmen. Mit einem mal war sein Zielschirm mit Fragezeichen nur so übersät. Abgesehen von der schematischen Darstellung dessen, was Jens für einen Jäger hielt, bot die Zusammenfassung kaum Informationen über das Ziel. Allerdings schien klar zu sein, wo dieser Bogey hin wollte. Seine Peilung entsprach einem der beiden Transporter.

\par

\WR{Verdammt}, murmelte Jens und wünschte sich, in einem Abfangjäger zu sitzen, der noch weiter beschleunigen hätte können.

\par

Als der Abstand zu Flug dutzend zwei nur noch wenige hundert Kilometer betrug, schaltete er die Nachbrenner aus und verlangsamte auf Gefechtsgeschwindigkeit, um nicht am Ziel vorbei zu schießen.

\par

Jens Zielmonokel erhöhte den Kontrast um den Transporter und den Jäger, damit diese in der Dunkelheit des Alls besser erkennbar wurde. Hastig ließ er den Computer mit einer Raketenerfassung beginnen.
\ortswechsel
\WR{Grau zwei, ich stecke in Schwierigkeiten!}, sprach Morten in sein Mikrophon und ließ seinen Jäger wild herum schlingern. Seine versuchte Raketenerfassung war fehlgeschlagen und nun war sein Gegenüber endgültig in Feuerreichweite. Die Strahlenbündel zogen nur knapp an Mortens Jäger vorbei.

\par

Hastig drehte er seinen Flieger auf den Rücken und schoss nur wenige Meter an seinem Gegner vorbei. Als er nach oben sah, erhaschte er einen kurzen Blick auf den Feind. Der Jägertyp sah dem sehr ähnlich, der ihm bereits im Asteroidenfeld von Arktur die Hölle heiß gemacht hatte. Nur die Flügel schienen nicht ganz so weit vom Rumpf abzustehen. Aber auch dieser Flieger schien kein Cockpit zu haben. An der stelle an der Morten eines vermutet hätte, fanden sich lediglich ein paar rote Flächen, die auf ihn fast wie Augen wirkten.

\par

\WR{Bin zur Stelle}, quäkte Kevin Wilsons Antwort aus Mortens Kopfhörern. \WR{Im Gegensatz zu dir, als ich Probleme hatte.}

\par

Morten überlegte nur eine Sekunde, bis ihm ein Manöver eingefallen war, mit dem er seinen Verfolger überlisten konnte. \WR{Wir sollten Hase und Igel mit ihm spielen}, schlug er vor und bezog sich damit auf ein Manöver, dass nach einem sehr alten Märchen benannt war.

\par

\WR{Verstanden. Du fliegst voraus}, antwortete Kevin.

\par

Morten verschwendete keine Zeit mit einer Antwort, sondern flog eine weite Schleife mit hoher Geschwindigkeit, bei der er seinen Feind auf dem Radar ständig im Auge behielt. Kevin heftete sich eng an seine Versen und folgte seinem Jäger mit wenigen hundert Metern abstand.

\par

Beide steuerten auf den Gegner zu, der seinerseits ebenfalls ein Wendemanöver durchführte und so frontal auf Morten zusteuerte. Wieder zuckten giftgrüne Geschosse an dessen Jäger vorbei. Morten erwiderte sporadisch das Feuer, ohne wirkliche Hoffnung zu haben, seinen Gegner zu treffen.

\par

Die Zahl auf dem Entfernungsmesser nahm rasant ab. Als sie die drei Trinmeter-Marke passierte rief Morten \WR{Grau zwei: Jetzt!} in sein Mikrofon und zog seinen Jäger steil nach oben. Sein Gegner tat es ihm gleich, denn so hatte er den Bauch von Mortens Flieger voll im Visier, der eine viel größere Trefferfläche und somit ein verlockendes Ziel bot.

\par

Doch hinter Morten flog direkt Kevin, der nun seinerseits den Bauch des feindlichen Jägers ins Visier bekam und nur noch das Feuer zu eröffnen brauchte. Die Einschläge seiner Strahlenwaffen ließen zuerst die Schilde des Angreifers hell aufflackern und bohrten sich schließlich in seine Hülle. Kurz darauf zerfiel der Jäger in einen brennenden Trümmerhaufen, begleitet von einem wilden Siegesschrei Kevin Wilsons.
\ortswechsel
\WR{Ja! Ich hab einen erwischt}, dröhnte es aus Jens Wörgs Kopfhörer. Doch dieser war damit beschäftigt, den Jäger ins Visier zu bekommen, der gerade das Feuer auf Flug dutzend zwei eröffnete. Eine seine Verfolgungsraketen wurde in diesem Moment vom Bordcomputer mit den Zieldaten des Jägers gefüttert.

\par

\WR{Komm schon!}, fauchte Jens den Computer an, der quälend lange dafür brauchte, der Rakete die Zieldaten einzuprägen.

\par

Durch die Cockpitscheibe erkannte er, wie der Angreifer Salve um Salve auf den Personentransporter abfeuerte. Die dünnen Schilde des Passagierschiffs waren schnell durchschlagen und so fraßen sich die Geschosse bald durch die Hülle des Transporters. Eine helles Feuer hüllte die Stelle ein, an der die Außenhaut des Schiffes aufriss und Jens konnte erkennen, wie Luft aus dem Inneren nach außen strömte. Der Sauerstoff gefror sofort und verlor sich in glitzernde Partikel.

\par

Nach einer endlos erscheinenden Sekunde ertönte endlich das laute Pfeifen aus Jens Kopfhörern, das die abgeschlossene Raketenerfassung signalisierte. Ohne weiter zu warten zog er den Abzug und ließ das Projektil von der Unterseite seiner Tragfläche starten.

\par

Doch lange bevor das Geschoss sein Ziel erreichte, ging der Transporter in Flammen auf. Völlig geräuschlos schien sich das Schiff in einen großen Feuerball zu verwandeln. Trümmerteile schossen in alle Richtungen davon und zeugten als einzige vom Tod der dreihundert Menschen an Bord des Transporters.

\par

Jens wollte schreien. Doch was er sah lähmte ihn. Er bekam kaum noch mit, wie seine Rakete den feindlichen Jäger traf und ihn innerhalb von Sekundenbruchteilen in Stücke sprengte.
