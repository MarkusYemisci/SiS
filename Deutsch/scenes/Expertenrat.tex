\WR{Sie haben also frei erfundene Aufnahmen vorgelegt, damit ihre Reise nach Iota Persei zwei finanziert wird}, schlussfolgerte Moyér, noch ehe Bellendi seine Geschichte zu Ende erzählt hatte. Dieser atmete tief und blieb sogar kurz stehen. In diesem Moment schien er am liebsten nirgendwo hin sehen zu wollen. Schon gar nicht in die Augen des jungen Lieutenants.

\par

Nach einem weiteren Schnaufen, antwortete er: \WR{Richtig. Ich bin nicht stolz darauf. Aber in diesem Moment war ich mir so unglaublich sicher, dass es mir der Betrug wie ein notwendiges Übel erschien. Nichts verwerfliches. In der Aussagenlogik kann man etwas richtiges aus einer falschen Aussage folgern und die Gesamtaussage bleibt dennoch wahr.}

\par

Sehr zu Bellendis Überraschung lächelte Moyér breit. \WR{Sie haben also einen der einflussreichsten Geschäftsmänner der Union um eine octinae Naira geprellt.}

\par

\WR{Ein octin vier septin elf pin neun qin neun trin vier dutzend Naira, um genau zu sein}, verbesserte Marco betreten. \WR{So stand es zumindest in der Ankalgeschrift.}

\par

Moyér hob überrascht eine Augenbraue. \WR{Sie sind vorbestraft?} Doch Bellendi schüttelte sofort den Kopf. \WR{Es blieb bei einer Entlassung durch meine Universität. Als die Mannschaft des PSG-Schiffs Rosalind Franklin die Umlaufbahn von Iota Persei zwei erreicht hatte, war schon klar, dass auf der Oberfläche höchstens ein paar Farne zu sehen waren Aber kaum tierisches Leben. Schon gar keine intelligenten Lebensformen. Die Linien, die wir gesehen haben wollten, waren das Resultat eines einfachen Programmierfehlers.

\par

Was es aber gab, waren reichlich seltene Erden, die sehr leicht abzubauen sein dürften. Und so haben dann doch ein paar Firmenarbeiter diese Welt, auf der noch nie ein Mensch zuvor gewesen war, betreten und sich die Schürfrechte gesichert.}

\par

\WR{Darum hat man die Anklage fallen gelassen?}, schloss Moyér.

\par

\WR{Genau}, Bellendi wirkte nach wie vor beschämt. Doch die Tatsache, dass es der PSG dann weniger um Gerechtigkeit, als um neue Profitmöglichkeiten gegangen war, linderten diesen Zustand ein wenig. \WR{Was man aus mir an Schadenersatz herauspressen hätte können, ist ein Witz, verglichen mit dem, was die PSG an Iota Persei zwei verdient hat. Bei weiten nicht die ertragreichste Welt, aber es hat gereicht, um die Kosten bei weitem zu decken.}

\par

\WR{Wir sind da}, informierte ihn Moyér, als die beiden vor einer Milchglastür zum stehen kamen. \WR{Ich darf leider nicht mit hinein. Meine Sicherheitseinstufung genügt nicht. Viel Glück, Mister Bellendi. Und nicht vergessen: Fotografieren ist bei solchen Unterredungen grundsätzlich verboten.}

\par

Der Lieutenant verbeugte sich vor Marco und verschwand dann, nach wie vor mit einem verschmitzten Grinsen auf den Lippen. Bellendi hingegen wandte sich schwitzend der Tür zu, die sofort aufging, nachdem der Computer seine Anwesenheit registriert hatte.

\par

Chaotisch wäre eine Untertreibung gewesen, hätte man den Zustand des Besprechungsraumes beschreiben wollen. Der längliche Tisch aus gutem Holz war umringt von zwei Dutzend wild diskutierenden Personen, von denen keiner dem anderen richtig zuzuhören schien. Ein Mann schien sogar so sehr in seinem Gespräch aufzugehen, dass er gar nicht bemerkte, wie er wild gestikulierend sein Wasserglas umgeworfen hatte. Sein Gegenüber, eine ältere Dame, die Bellendi als renommierte Botanikerin wiedererkannte, hingegen, war selbst so mitgerissen von der Unterhaltung, dass sie es nicht für nötig hielt, ihr Gegenüber vom Verlust seines Getränks zu informieren.

\par

Das Kopfende des Tisches hatten jedoch zwei Hologramme inne, deren Gegenstück Bellendi sofort erkannte. Präsident Otis und Grandadmiral Burns. Beide schienen von dem Treiben reichlich verärgert und verglichen mit ihrer gerade mal ein paar Tagen alten, früheren Erscheinung um Jahre gealtert zu sein.

\par

Eine weitere Person fiel Bellendi sofort auf. Lesc-Bublé saß ihm praktisch genau gegenüber, würdigte ihm aber keines Blickes. Irgendwie überraschte es Marco nicht, seinen Widersacher hier zu treffen. Immerhin waren sie praktisch auf demselben Feld tätig, beziehungsweise tätig gewesen und damit von allen Anwesenden am meisten im Thema.

\par

\WR{Ruhe bitte}, donnerte Grandadmiral Burns schließlich verärgert. \WR{Mein Stab hat mir versichert, dass es sich bei ihnen um die klügsten Köpfe der Union handeln soll. Trotzdem plappern Sie durcheinander wie eine Horde Kinder in der Grundschule.}

\par

Die Gespräche erstarben allmählich. Marco Bellendi nahm sich einen Stuhl und sah sich erneut um. Seine Kleidung war zwar sauber, doch schien sie ihm nun kaum weniger unangebracht, als zuvor. Die anderen Wissenschaftler trugen allesamt Gehröcke oder gute Kleider. Von der schnittigen Uniform des Admirals ganz zu schweigen.

\par

Dieser strahlte selbst als Hologramm repräsentiert eine gewaltige Präsenz aus und hatte somit keine Mühe damit, die Anwesenden erfolgreich zur Räson zu rufen.

\par

\WR{Danke, Grandadmiral}, ergriff Otis hastig das Wort. \WR{Und Ihnen allen herzlichen Dank für Ihr Erscheinen. Falls jemand von Ihnen Freunde oder Verwandte unter den Opfern von Pollux hatte, möchte ich Ihnen noch einmal persönlich mein Beleid aussprechen. Ich versichere Ihnen, dass dieser Angriff nicht ungesühnt bleiben wird.} Bellendi bemerkte, wie der ein oder andere schwer atmete oder bedrückt schlucken musste. \WR{Doch nun sollte es uns darum gehen, Klarheit über die Situation zu schaffen. Sie alle haben seit Ihrer Einberufung vollen Gnosis-Zugang zu allen Daten, die uns über die Angreifer zur Verfügung stehen.}

\par

Tatsächlich hatte sich Marco einiges vom angesprochenen Infomaterial angesehen, sofern es zwischen dem häufigen Erbrechen möglich gewesen war.

\par

\WR{Wir müssen wissen, womit wir es zu tun haben, wenn wir angemessen auf die Bedrohung reagieren wollen}, gab Burns zu bedenken. \WR{Ich weiß, die wenigsten von Ihnen sind auf militärischem Gebiet sachkundig. Aber ich bin sicher, Sie können dennoch etwas beitragen.}

\par

Wieder versank der gesamte Raum in wirrem Gerede. Jeder der Anwesenden begann seine Hypothesen darzulegen. Lediglich Lesc-Bublé und Bellendi blieben zunächst ruhig. Letzterer fasste sich aber schließlich ein Herz und warf ein: \WR{Es könnte sich um eine Schwarmintelligenz handeln.}

\par

\WR{Ruhe!}, blaffte Burns erneut. \WR{Herr Bellendi, wiederholen Sie das bitte.}

\par

Die plötzlich eingetretene Stille machte den Angesprochenen noch nervöser als die Anwesenheit Lesc-Bublés oder die wilde Diskussionen. Dennoch sagte er: \WR{Eine Schwarmintelligenz. Ich habe mir die Abtastungen ihrer Schiffe sowie ihr Angriffsmuster angesehen. Flugformation, Taktik. Alles deutet darauf hin, dass diese Kerle ihre Handlungen wie eine Art Bienenstock oder ein Ameisenbau koordinieren. Die Tanks, die wir bisher in allen ihren Schiffen festgestellt hatten…}

\par

\WR{Blödsinn!}, fuhr Lesc-Bublé dazwischen. \WR{Herr Präsident, Grandadmiral, unser Marco Bellendi fährt wieder seine übliche Strategie. Wilde Spekulationen mit schönen Worten zu verpacken, um jeden in die Irre zu führen. Wir haben alle das Material gesehen. Wir wissen nicht einmal, ob wir es überhaupt mit Außerirdischen zu tun haben, wenn Sie mich fragen.}

\par

Burns ließ den Einwurf links liegen und sein Hologramm wandte sich nun direkt an Bellendi. \WR{Sie sagten, Ihre Taktik lässt auf einen Schwarm schließen. Wie kommen Sie darauf?}

\par

Marco schluckte. Er wollte den Admiral nicht mit warten lassen, während er sein Buch zückte und sich die Daten noch einmal ansah oder eine Skizze zeichnete. Darum rang er nun nach einer plakativen Erklärung.

\par

\WR{Stellen Sie sich einen Schwarm Fische vor}, forderte er auf. \WR{Jedes Individuum daraus bewegt sich scheinbar mit eigener Motivation und eigenem Ziel. Aber wenn es die Situation erfordert, bilden die Fische sofort eine perfekte Formation. Genauso schien es gewesen zu sein, als Kitty-Hawk seine Jäger gestartet hat. Davor flogen die Angreifer ohne einen geschlossenen Verband. Aber als unsere Flieger in der Luft waren, haben sie sich praktisch ohne Zeitverzögerung zusammengefunden um den Geschwadern zu begegnen. Diese spontane Simultanität ist typisch für einen Schwarm.}

\par

Für einen Augenblick glaubte Marco Bellendi so etwas wie beeindruckt zu sein aus dem Ausdruck des Grandamiral lesen zu können. \WR{Wenn es sich um einen Schwarm handelt}, schlussfolgerte dieser, \WR{dann gibt es eine \EN{Königin}, oder?}

\par

\WR{Nicht zwingend}, gab Bellendi sofort zurück. \WR{Ameisen und andere Insektenarten haben in der Regel \EN{Königin}nen. Aber bei Fischen sieht es schon anders aus. Im Augenblick wissen wir schlichtweg zu wenig um sicher sein zu können. Wir brauchen mehr Daten.}

\par

\WR{Danke, Herr Bellendi}, entgegnete Burns knapp.

\par

Präsident Otis schien bereits seit einiger Zeit dringend das Wort ergreifen zu wollen. Unruhig drehte er seinen Sessel hin und her. Nachdem der Admiral seinen Satz beendet hatte, fuhr er sofort dazwischen. \WR{Der Grund, wieso ich sie alle hier zusammengerufen habe, ist der Folgende: Wir brauchen eine Möglichkeit, mit den Shutek Kontakt herstellen zu können.}

\par

Großadmiral Burns Ausdruck war nun schwer zu lesen. Doch eine gewisse Erbostheit konnte er nicht verbergen. Bellendi hingegen schien eher überrascht darüber, dass die Angreifer nun scheinbar einen Namen bekommen hatten.

\par

Einem Physiker mit schlohweißen Haaren schien dies ebenfalls aufgefallen zu sein. \WR{Shutek? Woher kommt dieser Ausdruck?}

\par

\WR{Ein Tippfehler unserer Presseabteilung}, erklärte Otis. \WR{Er leitet sich vom Basal-Wort für \Wr{Angreifer} ab und ist inzwischen zu einer Art inoffiziellen Bezeichnung geworden. Wir haben uns entschlossen sie zu verwenden, bis wir eine bessere Alternative finden.}

\par

\WR{Das ist absurd}, rief Lesc-Bublé und stand ruckartig auf. \WR{Wenn wir nur noch alberne Wörter für diese Angreifer verwenden und ohne triftigen Grund annehmen, sie seien eine Art fremde Intelligenz, dann habe ich hier~-- genauso wenig wie irgend ein anderer Wissenschaftler, der sich selbst ernst nimmt~-- noch etwas zu tun.} Mit diesen Worten umrundete Lesc-Bublé hastig den Tisch.

\par

\WR{Gehen Sie ruhig, ich bin sicher, wir kommen ohne Sie klar}, kommenterte Bellendi, um seinen Widersacher am Gehen zu hindern. Tatsächlich blieb der Angesprochene abrupt stehen und warf Marco einen wütenden Blick zu. Doch noch ehe sich die beiden weiter in die Haare bekommen konnten, fuhr Präsident Otis fort. \WR{Derzeit müssen wir leider tatsächlich davon ausgehen, dass die Shutek uns gegenüber feindlich eingestellt sind. Aber es ist nun von allergrößter Bedeutung, dass wir nicht mit blinder Gewalt reagieren. Wie Herr Bellendi bereits richtig gesagt hat, wissen wir zu wenig, um auch nur etwas über die Beweggründe der Feinde sagen zu können. Wir sollten daher so lange davon ausgehen, dass eine friedliche Verständigung möglich ist, bis wir das absolute Gegenteil wissen.}

\par

\WR{Offensichtlich reichen ein paar dutzend Atombomben für diesen Grad an Bestimmtheit nicht aus}, hängte Grandadmiral Burns frustriert an.

\par

Präsident Otis wollte gerade fortfahren, als ihn der weißhaarige Physiker unterbrach. \WR{Wieso verwenden die Shutek überhaupt Atombomben? Wir wissen, dass sie über Nullzonenwaffen verfügen.}

\par

Bellendi war dies auch schon aufgefallen. Und obwohl ihm seine Hypothese dazu reichlich unausgegoren vorkam, äußerte er sie dennoch. \WR{Möglicherweise ist Gammastrahlung für sie längst nicht so gefährlich wie für uns. Es wäre eine Möglichkeit für sie, Pollux nur für sich selbst nutzbar zu machen. Die letzten Abtastungen lassen darauf schließen, dass jeder Mensch einen dicken Bleianzug oder zumindest ein persönliches Schutzfeld brauchen würde, um nun noch auf der Oberfläche überleben zu können.}

\par

Otis räusperte sich deutlich hörbar. \WR{Wie dem auch sei. Herr Bellendi, das Omega-dreizehn-Protokoll war ihre Idee. Wir wissen, dass die \EN{Virial} versucht hat, es anzuwenden, als sie den Shutek in Arktur begegnet ist. Aber es gab keine Antwort. Bitte schildern Sie uns, wie Ihr Protokoll aufgebaut ist und wieso es darauf keine Erwiderung gab.}

\par

Bellendi kratzte sich an der Stirn und nahm zuerst einen Schluck Wasser, während er sich überlegte, wie er seine Idee zur Kommunikation mit Außerirdischen am einfachsten darlegen konnte. \WR{Meine Grundannahme dazu war die folgende: Fremde verstehen nicht nur unsere Sprache nicht, sondern eventuell nicht einmal, was \Wr{Sprache} an sich sein soll. Darum war die Herausforderung für mein Team und mich, eine Möglichkeit zu finden, nicht Wörter, sondern direkt ihre \textit{Bedeutung} übermitteln zu können. Wir haben weiter angenommen, dass jede außerirdische Spezies, die zu Reisen im Weltall fähig ist, gleichzeitig auch dasselbe Verständnis von Physik und Mathematik haben sollte, wie wir. Wir haben daraufhin einen Weg gesucht, grundlegende sprachliche Konstrukte durch mathematische Begriffe darzustellen. Die Aussagenlogik hat uns dabei mit ihren Quantoren außerordentlich geholfen. Wir haben dabei…}

\par

\WR{Warum hat Omega-dreizehn nicht funktioniert}, fiel Grandadmiral Burns dem Biologen ins Wort.

\par

Unter anderen Umständen hätte es ihn geärgert, so krude unterbrochen zu werden. So war es ihm eine willkommene Unterbrechung. \WR{Nun, dieses Problem kennen wir doch alle}, begann er eine Antwort. \WR{Man schlägt sein Buch auf und schreibt sich stundenlang Nachrichten mit seiner Herzensdame. Meistens bekommt man praktisch sofort eine Antwort. Aber ab dem Moment, in dem man sie auf einen Kaffee einlädt, herrscht plötzlich Totenstille.}

\par

Genauso ruhig war es nun um Bellendi herum. Alle Augen waren auf ihn gerichtet.

\par

\WR{Was ich damit sagen will}, fuhr Marco fort, \WR{ist, dass es nicht unbedingt am Verständnis liegt. Vielleicht \textit{wollen} uns die Shutek einfach nicht antworten.}

\par

Norton Burns nickte anerkennend. Nach einem unangenehm stillen Moment, fuhr Präsident Otis fort: \WR{Vor wenigen Stunden wurde eine kleine Flotte der Shutek in fünf dutzend eins Cygni entdeckt. Die dortige Kolonie wurde bereits evakuiert. Aber eine kleine Eingriffstruppe der Navy ist auf dem Weg. Gemeinsam mit einem Forschungsschiff. Das Aufeinandertreffen wird in etwa sechs Stunden stattfinden. Bis dahin \textit{müssen} wir ein Konzept zur Kommunikation ausgearbeitet haben.}

\par

Zum ersten mal teilte Bellendi Lesc-Bublés Empfinden, als dieser~-- nach wie vor stehend~-- hervorbrachte: \WR{Cygni? Das System ist gerade mal elf Lichtjahr von der Sonne entfernt. So weit sind die Shutekt bereits gekommen?}

\par

\WR{Es gibt keine direkte Hyperraumroute}, entgegnete Burns sofort. \WR{Aber die Nähe beunruhigt nicht nur sie. Insbesondere, da unklar ist, wie die Shutek es geschafft haben, dorthin zu kommen. Wir wissen nicht genau, woher sie stammen. Aber alle Routen nach Cygni werden überwacht. Wir hätten sie also kommen sehen müssen.}

\par

Der Präsident erhob sich unvermittelt. \WR{Herr Otis. Das ist nun ihr großer Auftritt. Überarbeiten Sie Omega-dreizen. Entwerfen Sie eine Nachricht, die den Shutek klar macht, dass wir keine kriegerischen Absichten verfolgen und eine friedliche Koexistenz anstreben möchten. Auf Wiedersehen und gutes Gelingen.}

\par

Mit diesen Worten verschwand die Projektion des Präsidenten unvermittelt. Marco Bellendi hatte gerade einwerfen wollen, dass er sechs Stunden für eine sehr kurze Zeitspanne hielt. Doch er hatte längst realisiert, dass Henry Otis sich etwas in den Kopf gesetzt hatte und nun keine Zweifel am Kurs mehr duldete.

\par

\WR{Von mir aus auch viel Erfolg, Herr Bellendi}, hängte Grandadmiral Burns verständnisvoll an. \WR{Denken Sie daran, dass Sie das Unmögliche nicht in sechs Stunden und nicht in sechs Jahren schaffen können. Und mir würde es schon reichen, wenn Sie den Shutek irgendwie sagen könnten, dass wir ihnen demnächst mit Anlauf in Ihre außerirdischen Ärsche treten werden.}
