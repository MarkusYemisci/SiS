Laura rang um Atem. Sie war schon so oft mit Klaus joggen gewesen, dass sie genau wusste, wie schnell er rennen konnte. Seine Schritte hallten laut durch den Gang, als sie auf das kleine Behandlungszimmer zu rannte. Die Wände waren schmutzig und die Tapeten schälten sich bereits wie abgestorbene Hautschuppen herunter. Und so passten sie perfekt zu Lauras Empfinden.

\par

Sie wollte nicht durch den Raum flüchten, der am Ende des Korridors auf sie wartete wie ein offener Schlund. Doch sie wusste auch, dass ihr keine Wahl blieb. Ein kurzer Blick über die Schulter verriet ihr, dass Klaus kaum noch mehr als zehn Meter von ihr entfernt war. Sein Messer blitzte in der geschlossenen Faust. Und obwohl sich auf seinem Gesicht eine schier mörderische Verbissenheit zu sehen war, erkannte sie trotz vieler Schweisperlen darauf keine Erschöpfung.

\par

Ihr Blick ging wieder nach vorne und sie rammte die Türe zum Behandlungszimmer mit ihrer ohnehin schon schmerzenden Schulter auf.

\par

Hätte sie den Raum dahinter nicht bereits am Vortag gesehen, hätte sie nun gegen einen unwillkürlich Reflex, vor Überraschung stehen zu bleiben, ankämpfen zu müssen. Aber so nahm sie das gut ausgestattete und eindeutig noch in Verwendung befindliche Untersuchungszimmer einfach so hin.

\par

An einer der Wände waren moderne Konsolen aufgebaut, auf denen kein Körnchen Staub zu erkenne war. Genauso wenig, wie auf der Liege in der Mitte, dem MRT oder der hell leuchtenden Operationslampe. Überhaupt schien der ganze Saal völlig steril und sauber, im krassen Gegensatz zum Rest des Krankenhauses.

\par

Das einzige, was nicht ins Bild passte, war die Leiche, die mit eng angewinkelten Gliedmaßen und völlig ausgemergeltem Gesicht auf dem Untersuchgsbett lag. Auf den ersten Blick war die Frau nicht einmal als solche zu erkennen. Es hätte auch genauso gut ein Mann mit langen Haaren sein können, so sehr war das Gesicht eingefallen. Das rote Blut, dass ihr aus der Nase lief, bildete einen makaberen Kontrast zur ergrauten Haut der Toten.

\par

Laura war es leid, davon zu rennen. Sie wusste, dass sie in einem Handgemenge gegen ihren Partner keine Chance hatte. Ihr persönlicher Blocker war überladen und ihre Waffe verloren. Und das Klaus sein eigenes Gewehr nicht gesucht, sondern sie sofort verfolgt hatte, sprach Bände. Er konnte es sich nicht leisten, sie entkommen zu lassen.

\par

Hektisch suchte sie den Raum ab und schnell wurde sie fündig. Eine Eisenstange, die eigentlich eines der Beine eines nicht aufgebauten Rolltisches war, lag wie für sie bereit. Sie hob sie auf und schwang sie geschickt in eine Angriffsposition.

\par

Und schon kam Klaus durch die Tür. Anders als sie, blieb er wie angewurzelt stehen. Laura begriff schnell, dass er dieses Zimmer zum ersten mal betrat. \WR{Deine Leute haben eine seltsame Vorstellung von der Benutzung eines MRTs}, sagte sie. \WR{Irgendwas sagt mir, dass das hier nicht das Werk des Hackers oder einer seiner Freunde war.}

\par

Klaus sah Laura immer wieder in die Augen. Aber ging unterdessen vorsichtig auf den Leichnam zu. Das Messer ständig fest in der Hand haltend.

\par

\WR{War also doch etwas dran, an dem, was der Hacker behauptet hat?}, fragte Laura weiter. \WR{Han und ich haben uns dieses Krankenhaus nicht aus Spaß ausgesucht. Er hat nach Akten über Persönlichkeiten des Öffentlichen Interesses gesucht. Solche, die Kopfuntersuchungen wie sie hier über sich ergehen haben lassen. An Orten wie diesem hier.}

\par

Nun stand Klaus Mund offen. Derselbe Schauer, der auch tags zuvor über Laura gekommen war, stand nun auch ihm ins Gesicht geschrieben. Und verzog es in Fratze, ganz ähnlich der Frau, die nun unter ihm verweste.

\par

\WR{Ich bin nicht sicher, wer sie war}, sinnierte Laura. \WR{Aber ich glaube, ihr gehört ein Medienunternehmen im Orbit um Proxima Zentauri Primus. \Wr{Satellit sechs} heißt es, glaube ich.}

\par

Klaus drückte sich mit beiden Armen vom Untersuchungstisch weg. Er atmete tief ein, was ihm bei dem dezenten Geruch nach körperlichem Verfall, der den Raum erfüllte sicher nicht leicht fiel. Dann umschloss er sein Messer noch fester und ging schnellen Schrittes auf Laura zu.

\par

\WR{Es spielt keine Rolle, wer sie war}, sagte und blanke Wut schwang dabei mit. \WR{Es gibt einen Grund, warum sie dort liegt. Es gibt einen Sinn, warum mit ihr gemacht wurde, was auch immer mit ihr passiert ist. Der braucht mich aber nicht zu kümmern. Ich gehöre zum Term. Ich vertraue dem Term. Ich glaube an den Term. Ich bin der Term.}

\par

Laura erhob ihre Stange wie einen Baseballschläger. \WR{Das ist dann wohl euer Credo, nehme ich an.} Klaus stürmte schreiend auf sie zu. \WR{Hilft dir aber nichts, Arschloch. Jetzt gibt's Prügel!}