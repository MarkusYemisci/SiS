Hätte jemand das Wetter auf den Seychellen an diesem Abend als herrlich bezeichnen wollen, wäre keine Übertreibung dazu nötig gewesen. Selbst jetzt, als die Sonne untergegangen war und sich die Dämmerung ihrem Ende zuneigte, war es warm und der Himmel sternenklar. Der Wind trug den Geruch des Meeres von der Saint Anne Bay ins Stadion hinüber.

\par

Nur der Jubel der sechzigtausend Menschen in der Matthius-Reese-Arena war inzwischen ein wenig abgeklungen, übertönte aber dennoch die Wellen, die den fünfhundert Meter weit entfernten Stand erreichten.

\par

Praslin war kein Ort, auf dem man das Endspiel einer Fußballweltmeisterschaft erwarten würde. Die Insel stand unter Naturschutz und war gerade einmal achtunddreißig Quadratkilometer groß. Vor langer Zeit hatte sie einmal ein Entdecker \textit{Insel der Palmen} genannt, was auch nach den Verheerungen der Seuche noch zutraf. Der Größte Teil der Landschaft war von dichten Wäldern bedeckt. Nur an den Stränden gab es einige kleinere Städte. Insgesamt lebten nicht mehr als viertausend Menschen auf der Insel.

\par

Ganz anders als in Neu Venedig. Nachdem die Seuche ausgestorben war, hatte sich Afrika als die neue, wie die alte Wiege der Menschheit erwiesen. So war es nicht verwunderlich gewesen, warum gerade dort langsam wieder neue Zivilisationen entstanden waren. Zwar hatte der Gabbot-Virus auch in Afrika gewütet, doch das heiße, trockene Klima und die mancherorts geringe Bevölkerungsdichte hatten es der Krankheit schwerer gemacht, sich auszubreiten.

\par

Als der Virus schließlich ausgestorben war, so glaubten Historiker der Union, hatte weder jemand geahnt, dass dies auch für andere Teile der Erde der Fall gewesen war, noch dass die Krankheit niemals zurückkehren würde. So hatte man sich entschlossen, eine isolierte Stadt, im Meer, östlich von Afrika zu errichten. Die Saya de Malha Bank hatte sich dafür besonders angeboten. Ein Flachwassergebiet im indischen Ozean, dessen Wassertiefe an vielen Stellen nur wenige zehn Meter betrug. Vor der Seuche war die Bank aufgrund ihres Artenreichtums, ihrer Flora und ihrer Fauna ein bemerkenswertes Wunder der Natur gewesen. Doch was nicht von der Krankheit dahingerafft worden war, hatte die Besiedlung durch den Menschen nicht überlebt. Zur Zeit der Union lebten zwei Million in der Stadt auf Säulen.

\par

Und als beschlossen worden war, die Weltmeisterschaft zweitausend vierhundert auf der Erde zu veranstalten, hatte man sich bemüht, einen besonderen Platz zur Austragung des Finales zu finden. Neu Venedig war einer der zentralen Transportknoten der südlichen Hemisphäre und somit durch viele unterirdische Züge zu erreichen, war aber auch aufgrund ihrer platzsparenden Bauweise nicht geeignet, ein Fußballstadion zu beherbergen. So hatte man ein Stadion auf den nahen Seychellen zwischen Strand und Palmenwald errichtet. Eine Maßnahme, die wieder Entlastung in die langsam erlahmende Wirtschaft der Erde gebracht hatte.

\par

Nun hatten sich die vierundzwanzig Mannschaften des Turniers, von den unterschiedlichsten Planeten der ganzen Union noch einmal im Carl Maze Stadion eingefunden und genossen den Applaus der Masse. Hintereinander wurden die Hymnen der teilnehmenden Welten gespielt. Fahnenträger liefen Ehrenrunden auf der Aschenbahn der Arena und präsentierten die Wappen der Teilnehmer, immer darauf bedacht, nicht über die Verkabelung der zahlreichen Geräte zu stolpern. Manche waren für die Aufzeichnung der holographischen Übertragungen gedacht. Andere dienten dazu, mit Hilfe von modernen Sensoren, Abseitsstellungen zu erkennen und die Seiten- und Torlinien zu überwachen. Fehlentscheidungen waren selten geworden, da ein Leistungsstarker Computer, gekoppelt an Sensoren überall auf dem Spielfeld sowie in der Ausrüstung der Spieler jede Situation sofort bewerten konnte.

\par

Nun war es an Henry Otis, die Sieger zu küren. Die Mannschaft des Industrieplaneten Corna stand auf einem Podest und der Stolz in ihren Gesichtern war kaum zu übersehen. Der Präsident war froh, die endlosen Debatten im Forum einmal vergessen zu können. Erst als er der zweiten Sondersitzung in Folge beigewohnt hatte, die sich nur um die Abschaffung der Armee gedreht hatte, war ihm klar geworden, was er mit seinem Antrag ins Rollen gebracht hatte. So vieles hing mit den Streitkräften zusammen. Arbeitsbeschaffung, Perspektiven und nicht zuletzt das Sicherheitsbedürfnis der Bevölkerung. Er hatte sich einige Freund unter den Intellektuellen und Akademikern gemacht, die das Forum zu großen Teilen bildeten. Doch mit den Pragmatikern hatte er sich mächtige Feinde geschaffen.

\par

Nun wollte er sich zumindest für einen Moment lang von der Rolle als pazifistischer Moralapostel lösen, und wieder die Nähe zu seinem Volk suchen. Ursprünglich hatte er vorgehabt, die Preisverleihung an Richard Bellegardè zu delegieren. Doch nun freute er sich darüber, vom Publikum einmal nicht ausgebuht zu werden.

\par

Im Hintergrund wurde die Stimme des Kommentatorenduos zugeschaltet, als Präsident Otis auf das Podium zuging, den Arm mit zweiundzwanzig Medaillen behängt. \WR{Und damit neigt sich auch diese Weltmeisterschaft ihrem Ende zu}, begann Marcel Rice, ein populärer Kommentator, der mit dem Ausgang des Finales nicht zufrieden war. \WR{Ich denke heute haben sich die Außenseiter mit ganz viel Glück durchsetzen können. Aber ein verdienter Sieg sieht für meine Begriffe anders aus. Was meinst du, Claudio?}

\par

Sein Partner fuhr wesentlich enthusiastischer mit seinem Abschlussbericht fort: \WR{Ich finde, diese Weltmeister haben ihre Sache gut gemacht und gezeigt, das dieses Turnier selbst nach Jahrhunderten seinen Reiz nicht verloren hat, weil eben jeder hier seine Chance bekommt.} Henry Otis und Lertha Akintola, die ihn an diesem Tag noch keines einzigen Blickes gewürdigt hatte, schüttelten gerade den Erst- und Zweitplatzierten die Hände. \WR{Ich denke, man kann wirklich sagen, dass mit dieser Siegerehrung eine runde und vor allem spannende WM zu Ende geht. Der Präsident und seine Stellvertreterin übereichen jetzt die Gold- und Silbermedaillen an die Mannschaften aus Corna und vom Mars und ich denke auch die zweiten können mit ihrer Leistung…}

\par

Der Kommentator unterbrach sich, als er erkannte, wie ein Mann im feinen Zwirn auf das Spielfeld rannte. Keiner der Leibwächter des Präsidenten schien ihn aufhalten zu wollen. Der Mann lief, als würde er von Löwen gejagt werden, direkt auf das Staatsoberhaupt zu.

\par

Otis wandte sich um, als er seinen Nachrichtenbeauftragten auf sich zukommen sah. Dass der junge Mann einfach so in die Szenerie hinein rannte, und dabei mehr als gehetzt wirkte, ließ darauf schließen, dass es schlechte Nachrichten gab. Gute hatte zumeist Zeit.

\par

Der Präsident überreichte den beiden verbliebenen Spielern ihre Trophäen mit einem mal und wandte sich zu seinem Untergebenen um. Im ganze Stadion hatte sich in kurzer Zeit eine fast schon gespenstische Ruhe breit gemacht. Wo vor wenigen Augenblicken noch ausgelassen gejohlt und gefeiert worden war, richtete jeder seinen verwunderten Blick auf Otis und sein Gefolge.

\par

Der Nachrichtenbeauftragte trat so nah wie möglich an den Präsident und die Vizepräsidentin heran und sagte gerade laut genug, um noch gehört werden zu können. \WR{Bitte folgen Sie mir. Es gibt einen Notfall und Sie müssen sich so schnell wie möglich ins Hauptquartier nach Neuseeland begeben.}

\par

Lertha Akintola taxierte den jungen Mann mit teils fragenden, teils verärgerten Blicken. Henry Otis entgegnete: \WR{Was ist das für ein Notfall. Wir sind hier mitten in einer Preisverleihung.}

\par

\WR{Mein Herr}, begann der Nachrichtenbeauftragte diesmal bestimmter und mit Nachdruck. \WR{wir haben Schnee im September.}

\par

Dem Präsidenten war klar, dass damit keine Wettermeldung gemeint war. Eine derartige Nachricht hatte es seit Gründung der Unio Terrestris nicht gegeben. Und auch davor war sie nur ein einziges Mal ausgesprochen worden~-- als der Routenkrieg ausgebrochen war. Damit brach für ihn eine Welt zusammen.
