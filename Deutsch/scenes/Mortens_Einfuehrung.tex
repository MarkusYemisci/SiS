Lautes Gejohle ertönte im hinteren Raum des Langstreckenshuttles \EN{Galere vier sieben}, als der Torwart der Cornaer Mannschaft einen scharf geschossenen Elfmeter gerade noch so über den Querbalken lenken konnte. Das Hologramm war trübe und die Tonqualität ließ ebenfalls sehr zu wünschen übrig. Doch die Übertragung verfehlte ihre Wirkung nicht.

\par

Einer der Piloten, die mit der Fähre ihrem neuen Einsatzort entgegen rasten, hatte sein persönliches Buch der Gruppe zur Verfügung gestellt. Auf dem Boden zwischen den Sitzreihen lag es nun aufgeschlagen da, während der veraltete Holoprojektor ein dreidimensionales Bild des Spielfelds für alle in den Raum warf. Der Bildregisseur hatte zunächst den Ausschnitt um das Tor vergrößert. Nun zeigte er eine Aufnahme des Mittelkreises, in dem die eine Mannschaft starr mit frustrierten Gesichtern das Geschehen verfolgte, während die andere in Jubel ausbrach.

\par

Corna war der krasse Außenseiter dieser Weltmeisterschaft gewesen. Niemand hatte mit den Jungs und Mädels in orange-weiß gerechnet. Doch die Mannschaft war das perfekte Ebenbild des Planeten, von der sie stammte. Zäh und verbissen bis zur letzten Minute.

\par

Bereits lange vor dem Routenkrieg hatte sich Corna auf dem absteigenden Ast der Fernbesiedlung wiedergefunden, die von Kreuzpunkt Primus ihren Ausgang genommen hatte. Schnell war die Terraformierung auf ein Minimum zurückgefahren worden und Corna war zu einem vollkommenen Industrieplaneten verkommen. Kein eigenes Bildungssystem, keine signifikante Wirtschaft, lediglich Export von schweren Maschinen. Wen es auf diese traurige Welt verschlagen hatte, der wusste, dass die fetten Jahre für ihn vorbei waren. Insbesondere, als das Commonwealth von Kreuzpunkt damit angefangen hatte, seine Kolonien repressiver zu kontrollieren. Viele Aufstände und einen Weltkrieg später war Corna nun zum Vorzeigeprojekt der Union geworden. Armut und Chancenungleichheit hatte man sofort nach Gründung der Unio Terrestris mit aller Macht bekämpft. Ein effizienter Interstellartransport war eingerichtet worden, um den Bürgern bessere Bildungschancen auf anderen Planeten zu ermöglichen. Das Terraforming war wieder aufgenommen worden und mit den berühmten Salzhöhlen hatte Corna nun sogar eine Touristenattraktion erschlossen.

\par

Doch der Planet war nach wie vor eines. Hässlich. Eine Welt, die derart schlecht gelegen und unfruchtbar war, wie Corna, konnte kaum für etwas anderes als riesige Industriestädte verwendet werden. Daran konnte auch der gute Wille der Union nichts ändern.

\par

Zumindest hatte die Geschichte des Planeten zu viel Sympathie, auch außerhalb der Welt des Fußballs, geführt. Nun, während der Weltmeisterschaft jedoch, flammte die Begeisterung für die tapfere Mannschaft vom Gesäß des bekannten Univerums über alle Maßen und von überall her auf. Insbesondere die Piloten von der Erde, deren eigene Mannschaft zwei Wochen zuvor gegen das jetzige Gegnerteam ausgeschieden war, trugen unter ihren Uniformen das Trikot des Industrieplaneten.

\par

Es wurde schnell ruhiger, als der Torwart Cornas nun selbst zum Elfmeterpunkt lief. Die Spannung war praktisch greifbar, auch wenn die Fußballer eher in der Holographie eher wie kleine Spielzeugfiguren wirkten. Corna hatte in der ersten Hälfte sogar das Spiel an sich gerissen, doch dann eine Führung mit zwei Toren aus der Hand gegeben. Da die Weltmeisterschaft auf der Erde ausgetragen wurde, galt nicht der sonst übliche \Wr{Kreuzpunkter Modus}, nach dem im Sudden-Death-Verfahren so lange Spieler vom Feld gestellt und dieses verkleinert wurde, bis eine Mannschaft eine Führung über sieben Minuten halten konnte.

\par

Es war zu einem klassischen Elfmeterschießen gekommen, in dem weder Corna noch ihr Gegner von Tau Ceti in der ersten Runde einen Sieg hatten erringen können. Würde der Torwart nun einen Treffer erzielen, wäre das Spiel gewonnen und die Industriewelt mit ihren gerade einmal zwanzig Millionen Einwohnern würde ins Finale einziehen.

\par

Einige der Piloten hielten sogar den Atem an und niemand schenkte mehr dem prachtvollen Nebel Beachtung, der zur Backbordseite hin durch die Fenster zu bestaunen gewesen wäre. Die Übertragung vergrößerte den Bereich um das Tor herum, so dass der Torwart aus Tau Ceti etwa auf Ellenlänge anwuchs. Die Anspannung war sowohl dem Schützen als auch dem Torwart überdeutlich ins Gesicht geschrieben. Beiden rann der Schweiß in Strömen.

\par

\WR{Maxi Boyega wird sich jetzt der schweren Herausforderung stellen}, sagte der Sprecher, der, obwohl er eigentlich unparteiisch sein sollte, merklich auf der Seite Cornas zu stehen schien. Wahrscheinlich war auch er, wie die meisten anderen, von der Leistung des Außenseiters überrascht und beeindruckt. Als, zu Beginn des Turniers, die Mannschaften auf den Seychellen angereist waren, hatte sich niemand vorgestellt dass die Auswahl des Industrieplaneten Corna dermaßen erfolgreich spielen würde. Bis jetzt hatten sie jedes Spiel gewonnen.

\par

\WR{Im Stadion ist es buchstäblich totenstill}, fuhr der Kommentator fort. \WR{Das Carl Maze Stadion hier bei Praslin Island ist bis zum letzten Platz ausverkauft. Trotzdem sind die fünfdutzend neuntrin Zuschauer wie versteinert. Der nächste Schuss kann schon über den Ausgang dieser Partie entscheiden. Und jetzt geht es los.}

\par

Die nächste Sekunde erlebten die meisten wie in Zeitlupe. Der Ball hob vom Boden ab und schoss auf die rechte Ecke des Tors zu. Der Torwart aus Tau Ceti rannte los, kaum dass sein Gegenspieler den Ball getreten hatte. Als wüsste er, wo er hinfliegen würde, warf sich der Schlussmann in die Höhe und streckte beide Arme so weit aus wie er konnte. Morten erkannte genau, wie dicht der Torwart bereits mit seinen Händen am Ball war. Doch es reichte nicht. Selbst in voller Größe konnten es nur wenige Zentimeter gewesen sein, die der Ball an den Händen des Schlussmanns vorbei rauschte. Kurz darauf fragte sich Morten jedoch, ob die Kugel das Tor überhaupt treffen würde. Der Torwart war so weit in die Ecke gesprungen, dass es aussah, als müsste der Ball das Tor verfehlen, als der Torwart nicht mehr heranreichte.

\par

Doch der Ball traf. Er schlug mitten in den rechten Winkel zwischen dem rechten Pfosten und der Querlatte ein und sprang von dort aus im Rücken des Torwarts ins Netz.

\par

Erneut ging der Passagierraum der Fähre in lautem Gebrüll schier unter. Einige der Piloten verließen sogar regelwidrig ihre Plätze und vollführten spontane Siegestänze. Auf den beiden hintersten Bänken spielten sich sogar gleich zwei Vergehen ab. Eine sehr glücklich wirkende Soldatin nahm von ihrem Nebenmann einige Münzen an Schwarzgeld an, während ihre Nachbarn auf der anderen Seite der Fähre aus einem Flachmann zu trinken begannen. Die vorderen Reihen begnügten sich damit, immer lauter \WR{Corna! Corna! Corna!} zu skandieren.

\par

Nur ein einziger Pilot blieb ruhig auf seinem Sitz hocken und dachte nicht einmal daran, seinen Gurt zu öffnen. Sein Blick ging aus dem Fenster und traf erneut auf den Nebel, dessen Farben er nun zu bewundern begann. Er saß allein. Aus gutem Grund, denn er mochte keine leichte Konversation und schon gar keine Gespräche über Fußball. Das, obwohl sein Herz ebenfalls für die Cornaer geschlagen hatte. Doch er war im Dienst und allein das ansehen der Weltmeisterschaftsübertragung während eines offiziellen Fluges war bereits bestenfalls geduldet. Der Genuss von Alkohol oder Wetten, die sogar mit illegalen Zahlungsmitteln dotiert waren dagegen,3 waren ein klarer Verstoß und hätten auch mit einer unehrenhaften Entlassung enden können.

\par

Ein wenig munterte ihn der Ausgang des Spiels doch auf. Im Sport ging es um Leistung. Im Allgemeinen wurde nicht gefragt, welche Mannschaft es, abgesehen von ihren athletischen Merkmalen, \textit{verdient} hatte, zu gewinnen. Wer nicht gut genug spielte, der verlor. Unabhängig davon, wie sympathisch oder charakterlich vorbildlich er war. Nun hatten die Cornaer Sportler den Kritikern darlegen können, dass sie den Zuschauern mehr bieten konnten, als ein freundliches Gesicht. Sie hatten in einem fairen Wettkampf gesiegt~-- wenn auch nur äußerst knapp. Selbst der hohe Frauenanteil der Mannschaft, der von vielen, trotz quasi vollendeter Geschlechtergleichstellung, als unprofessionell hingestellt worden war, hatte sich nicht gerächt.

\par

Sicher würden die Kritiker einen Weg finden, ihren Unmut an der vermeintlichen Hinterwäldlermannschaft weiter Ausdruck zu geben. Denn es war unbestreitbar, dass bei diesem Sieg auch viel Glück eine Rolle gespielt hatte. Und das Finale würde gegen die Auswahl vom Mars stattfinden. Einer Mannschaft, gegen den die Cornaer bislang nicht einmal in Freundschaftsspielen einen Sieg hatten davontragen können.

\par

Der ruhige Pilot sah in sein eigenes Buch. Ein kleines Modell ohne aufwändigen Einband. Die erste Seite zeigte ihm Uhrzeit und Datum an. Es konnte nicht mehr lange dauern, bis die Fähre ihr Ziel erreicht hätte. Darum tippte er mit seinem Finger zweimal auf ein leeres Blatt. Eine Geste, die er seinem Buch schon vor einiger Zeit beigebracht hatte und die es zu einem Spiegel werden ließ.

\par

Er besah sich selbst. Der schönste war er nicht, das war ihm seit langem klar. Eine viel zu lange Nase, ein unsymmetrisches Gesicht, sein rechtes Auge stand höher als sein linkes, beide standen deutlich zu weit auseinander und sein Mund war viel zu breit. Aber immerhin sah er ordentlich aus. Auch seine Uniform saß wie angegossen und betonte seine hochgewachsene Statur.

\par

Die Abzeichen eines Second Lieutenants zierten erst seit einer Woche seinen Kragen. In diesen Rang hatte man ihn erhoben, nachdem er seine Abschlussprüfung an der Flugakademie mit Bravur bestanden hatte. Er hatte sich seinen Dienstort selbst aussuchen dürfen. Normalerweise legte er viel Wert auf ein verträgliches Umfeld, was eine planetare Basis auf irgendeiner kleinen Kolonie nahegelegt hätte. Doch er wollte etwas höher hinaus. Nicht weil er nach Ruhm, Anerkennung oder einem Leistungsbeweis gestrebt hätte, hatte er sich auf einen Träger versetzen lassen, sondern weil er genau auf einen solchen gewollt hatte.

\par

Ein Schraubenhaufen, der von einem Planetensystem zum nächsten raste und auf dem Weg möglichst viel Abwechslung bot. Auch wenn die sprichwörtliche Atmosphäre eines echten Planeten, auf dem man Boden unter den Füßen hatte, sicher fehlen würde, fand das Leben eines Starforce-Piloten im Weltraum statt. Und dort hatte er hingewollt, seit er seine Anstellung als Stahlarbeiter aufgegeben hatte, um auf die Akademie gehen zu können.

\par

Der Trubel um das Fußballspiel ebbte nach wie vor nicht ab und bereits die ersten fragenden Blicke trafen Morten Wittwer, den einzigen der Piloten, der nunmehr noch auf seinem Platz saß und auch der einzige, der von Corna stammte.